\section{Translating Meaning to Form}
%
%
%
In this section we shall be concerned with the various ways in
which languages express a given meaning. This will be useful in several
ways. First, it illustrates how diverse human languages are and what
kind of typological differentiation there is. Second, it shows us how
complex the algorithms for mediating form and meaning really are in
living languages. The reader will therefore get an idea of why
linguistic theories are so incredibly complicated and diverse. Third,
we shall see how the present analysis of languages as semiotic systems
fits vis a vis other theoretical frameworks.

We shall be concerned with the translation between form and
meaning. However, if by meaning we understand simply
model--theoretic meaning this task is at present impossible. Say
we have a language using certain basic properties of individuals
$P_i$, $i < m$, which are interpreted as subsets of the universe
$M$. Boolean combinations are likewise interpreted as subsets of
$M$. Now given an arbitary subset of $M$, which expression will be
used to refer to this set? If the set is finite there seem to be
tools available that make this possible, but if the set is
infinite then this in turn depends on the mode in which that set
is presented. (For example, suppose that the set is described by a
recursive procedure, naming the first two elements, say 1 and 2,
and then saying that the next element is the sum of all the
previous elements. Would you have guessed that this is the set
$\{1,2\} \cup \{3\cdot 2^n : n \in \omega\}$?) There is all reason
to believe that this problem is not the one we should attempt to
solve. Rather, given that we have a language (say, some language
of simple type theory) in which we express meanings, then we
should describe a procedure that takes elements of that language
and translates them into expressions of the language in question.
Thus, it may turn out that $t$ and $u$ are synomous terms but
their translations are nevertheless different (but again
synonymous if translation is biunique). This is the problem we
shall attack here, progressing from simple to complex.

First step. $F = \{\mbox{\tt a}, \mbox{\tt b}, \mbox{\tt c},
\mbox{\tt d}\}$, {\tt a} and {\tt b} have the same type, $e$, {\tt
c} has the type $e \pf t$ (intransitive verb) and {\tt d} the type
$e \pf (e \pf t)$ (transitive verb). The wellformed expressions of
this language are {\tt ca}, {\tt cb}, {\tt daa}, {\tt dab}, {\tt
dba} and {\tt dbb}. Call the language $L$. We shall now study the
ways in which these expressions are rendered in natural languages.
The first observation we make is that languages differ in the
relative order of the elements. English, for example, puts the
subject before the verb, the object after. Thus, if {\tt a} is
translated by {\tt Marcus}, {\tt b} by {\tt Paul}, {\tt c} by {\tt
sleeps} and {\tt d} by {\tt chases}, we get the following
correspondence.
%%%
\begin{center}
\begin{tabular}{l|l}
$L$ & English \\\hline
{\tt ca} & {\tt Marcus sleeps.} \\
{\tt cb} & {\tt Paul sleeps.} \\
{\tt daa} & {\tt Marcus chases Marcus.} \\
{\tt dab} & {\tt Marcus chases Paul.} \\
{\tt dba} & {\tt Paul chases Marcus.} \\
{\tt dbb} & {\tt Paul chases Paul.}
\end{tabular}
\end{center}
%%
In this way, we can highlight the different word order patterns.
For English, we may postulate the following basic signs in an {\bf
AB}--grammar.
%%
$$\auf \mbox{\tt Marcus}, e, \mbox{\tt a}\zu,
\auf \mbox{\tt Paul}, e, \mbox{\tt b}\zu, \auf \mbox{\tt sleeps},
e/t, \mbox{\tt c}\zu, \auf \mbox{\tt chases}, e/(e \backslash t),
\mbox{\tt d}\zu$$
%%
Other languages have different word order patterns, as we have
described earlier. However, more is in stock. Many languages use
{\it cases\/} to distinguish the arguments. Hungarian, for
example, marks the object of a transitive verb by a suffix {\tt
t}.
%%
\begin{center}
\begin{tabular}{l|l}
$L$ & Hungarian \\\hline
{\tt ca} & {\tt Marcusz alszik.} \\
         & {\tt Alszik Marcusz.} \\
{\tt ca} & {\tt P\'al alszik.} \\
         & {\tt Alszik P\'al.} \\
{\tt daa} & {\tt Marcusz l\'atja Marcuszt.} \\
          & {\tt Marcusz Marcuszt l\'atja.} $\ldots$ \\
{\tt dab} & {\tt Marcusz l\'atja P\'alt.} \\
          & {\tt Marcusz P\'alt l\'atja.} $\ldots$
\end{tabular}
\end{center}
%%
Word order is irrelevant (in Hungarian it is constrained by other
factors, so is not completely free). For the purpose of
translating back from the Hungarian sentence into the term,
however, word order is actually irrelevant, unlike in English.
Suppose for the sake of the argument that any permutation of the
sentences on the right hand side is a grammatical sentence of
Hungarian whose meaning is the one given on the left. Then
translation is still biunique. {\tt L\'atja P\'alt Marcusz.} for
example has meaning {\tt dba}. All we have to do is to look for
the word carrying the suffix {\tt t}. This word is the object.

Additionally, Hungarian shows what is known as {\bf head marking}.
Depending on whether the subject is singular or plural, and
depending on whether it is 1st, 2nd or 3rd person the verb form is
different (see also Section~\ref{kap5-1}).
%%
\\[2mm]
\begin{tabular}{l|l|l}
(Subject) & English & Hungarian \\\hline%%
1 Sg & {\tt I give.} & {\tt Adom.} \\
2 Sg & {\tt You give.} & {\tt Adsz.} \\
3 Sg & {\tt He/She/It gives.} & {\tt Ad.} \\
1 Pl & {\tt We give.} & {\tt Adunk.} \\
2 Pl & {\tt You give.} & {\tt Adtok.} \\
3 Pl & {\tt They give.} & {\tt Adnak.}
\end{tabular}
\\[2mm]
%%
We say that in Hungarian the verb agrees in number and person with
its subject. %%
%%%
\index{agreement}%%
%%%
English has traces of that phenomenon: if the
subject is in the first person singular the verb carries an
additional suffix {\tt s}. The reason why this is called head
marking is that this time it is the head (the verb) that is marked
for properties of its argument. Some languages not only have
agreement with the subject, they also have agreement with the
object, some even with the indirect object. We remark here that
this phenomenon is not restricted to verbs. A particularly
widespread phenomenon is the so--called {\bf possessor agreement}.
%%%
\index{agreement!possessor}%%
%%%
Once again we can use Hungarian to exemplify it.
%%%
\begin{center}
\begin{tabular}{l|l|l}
(Possessor) & English & Hungarian \\\hline%%
1 Sg & {\tt my car} & {\tt az \'en kocsim} \\
2 Sg & {\tt your car} & {\tt a te kocsid} \\
3 Sg & {\tt his/her/its car} & {\tt az \"o kocsija} \\
1 Pl & {\tt our car} & {\tt a mi kocsink} \\
2 Pl & {\tt your car} & {\tt a ti kocsitek} \\
3 Pl & {\tt their car} & {\tt az \"o kocsijuk}
\end{tabular}
\end{center}
%%
In German, the possessive pronouns distinguish the gender of the
possessor (in 3rd person) as well as number, gender and case of
the possessed item. For example, it is {\tt sein Auto} for {\tt
his car} but {\tt ihr Auto} for {\tt her car}. Put into the dative
this would be {\tt seinem Auto} and {\tt ihrem Auto}, and {\tt
seinen Autos}/{\tt ihren Autos} in the plural (and dative). Also,
adjectives in many languages agree with the noun they modify in
case, number, person and other categories. English possesses
number agreement with respect to the demonstratives {\tt this} and
{\tt that}. We have {\tt this city} but {\tt these cities}, and we
have {\tt that cities} but {\tt those cities}.

For the rest of this section we shall sketch the advantages and
disadvantages of the various types of languages, and then turn to
a formal analysis of case marking. Our discussion of the various
representations of the terms concerns the question whether the
representations are uniquely readable. Obviously, if they are not
this is a source of potential confusion in communication. In
logic, Polish Notation has been popular for a while since it is
very economical: once one is used to it, it has many advantages.
One is of course, that it does away with brackets. This means less
work in writing or typing formulae. On the face of it it seems
that any consistent way of putting the function symbol with
respect to its arguments is the same as any other. This, however,
is false. Suppose we write $\nicht$ before its argument and $\qu$
after. Then $\nicht \mbox{\tt p0}\qu$ is ambiguous between
$\mbox{\tt (}\qu\mbox{\tt (}\nicht \mbox{\tt p0))}$ and $\mbox{\tt
(}\nicht\mbox{\tt (}\qu\mbox{\tt p0))}$. Thus, we must be
consistent in either putting every functor before its arguments or
every functor following them. On the other hand, if this fixed, we
can specify for each functor independently in which way the
arguments have to be written down. For example, we may require
that $\pf$ takes its first argument second, and the second
argument first. As long as do not additionally employ the standard
notation this remains uniquely readable. Evidently, if
$\pf\varphi\chi$ is actually taken to mean what $\pf\chi\varphi$
means we had better use a different symbol, say $\leftarrow$. And
if so, it is clear that we might just as well declare the argument
that appears first to be the first argument {\it tout court}.
Trivial as this point may seem, it allows to see that nothing much
distinguishes a VSO language from a VOS language. However, one of
the most famous linguistic universals in language typology is the
claim that the subject precedes the object (Greenberg's Universal
No.~1), thus favouring VSO, SVO and SOV. From the figures given in
Section\ref{kap1-4} we conclude that this is the case in 19 out of
20 languages. We shall not elaborate on why this is so. We mention
only that theoretically SVO languages are at a disadvantage since
they are not uniquely readable. However, other factors come into
play. First, the immediate constituents of the verb typically are
noun phrases, and their structures can be much different so that
it is not immediate to what degree the ambiguity is real.
