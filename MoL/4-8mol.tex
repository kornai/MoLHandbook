\section{Formal Pragmatics: Context Dependency}
%
%
%
There are a number of elements in language whose meaning cannot be 
fixed purely in terms of model--theoretic conditions. Consider 
(\thesection.1) -- (\thesection.4) being uttered by Paul on 
October 27, 2002. 
%%
\\[2mm]
\begin{tabular}{ll}
(\thesection.1) & {\tt I am typing a letter.} \\
(\thesection.2) & {\tt It is cold here.} \\
(\thesection.3) & {\tt Today we are discussing abelian groups.} \\
(\thesection.4) & {\tt This is not correct.}
\end{tabular}
\\[2mm]
%%
In (\thesection.1), the word {\tt I} denotes Paul, so it is subtitutable 
with {\tt Paul} (modulo changes in the inflection). However, this is 
only so because Paul said this. If someone else, say Henry, would utter 
(\thesection.1), then {\tt I} would denote Henry, not Paul. Similarly, 
{\tt here} denotes a region close to Paul, the utterer of that sentence 
at the time he uttered it. {\tt Today} means October 27, 2002, and 
{\tt this} is whatever speaker (Paul) points at when uttering that 
sentence. 

David Kaplan has suggested that meanings are not functions from worlds 
to truth values but rather functions from contextes to functions from 
worlds to truth values. He calls such functions {\bf characters}. 
%%%
\index{character}%%
%%%
The context includes information such as {\it speaker}, {\it addressee}, 
{\it time\/} and so on. The character of {\tt I} is that function that 
takes a context $\sigma$ and returns for every world the speaker of 
$\sigma$. The word {\tt here} takes a context $\sigma$ and returns
a function that assigns to every world the region close to the speaker 
of $\sigma$. It is not difficult to upgrade the apparatus to accommodate 
for characters in place of meanings. In this way we can create a 
compositional account of languages with such context dependent elements 
as {\tt I}, {\tt here} and {\tt today}. There is an i,plicit convention 
in language that when a sentence is asserted, the character of the 
sentence is applied to the context given by the context of assertion. 
So, if Paul utters (\thesection.1) on October 27, 2002, in the mayor's 
villa in Ravenna, then the sentence means that Paul is typing a letter 
on October 27, 2002, and (\thesection.2) that it is cold in the mayor's 
villa in Ravenna on that day. 

The theory of characters can account for some phenomena but unfortunately 
not all and will be replaced by a different one. Notice that any theory 
of context dependence will have to differentiate between a sentence and 
an {\it utterance}.
%%%
\index{utterance}%%
%%%
An {\bf utterance of} $S$ is a physical event that realizes $S$, 
a so--called {\bf author} and an {\bf addressee}. To see that the 
physical event is not enough to characterize the utterance, consider 
a situation where a spy is sitting at a coffee table. At a neighbouring 
table sit some of his fellow agents, but opposite to him a spy of 
the opposite side. By saying something to the spy opposite to him 
he may actually convey a secret message to his fellow spies at the 
other table. The physical event is the same, but there are two utterances 
here. And they may convey totally different meanings because the 
code they are supposed to be interpreted may be different, as the 
case of the spies clearly suggets. 

For example, pronouncing the words of (\thesection.1) in the 
appropriate way counts as {\bf uttering} $S$. However, for the 
purposes of the theory writing that very sentence onto a blackboard 
or onto paper, reading that sentence in a book, even thinking it to 
oneself will be counted as utterances of it. The latter case might 
be somewhat stretched, but there is a clear difference between 
grasping an idea without words and pronouncing the thought in ones 
head. It is the latter that counts here. Once it is explicated what 
it means for a physical event $u$ to realize $S$ we are in a position 
to define the meanings of the elements {\tt I}, {\tt here} and 
{\tt today}. In the easiest case, {\tt I} denotes the person or 
things that actually produces $u$. In the case of Paul we would of 
course not say that it is Paul's vocal tract that utters $u$, 
but rather Paul himself. But there are more difficult cases. If 
the judge reads a witness' testimony aloud it is he who produces 
the utterance but the word {\tt I} will not denote him, it will 
denote the witness. In other words, what the word {\tt I} actually 
means is a nontrivial issue (see \cite{mccawley:person}). We 
shall ignore that issue and concentrate on the theoretical points 
raised by the analysis. 

Given the distinction between utterance and sentence we have to ask 
ourselves whether meanings are porperties of sentences or actually 
of utterances. Kaplan's idea is that it is a property of sentences, 
though is yields a function that depends on factors that only the 
utterance can reveal. Thus given an utterance $u$ of a sentence $S$, 
there is a way of determining a context $\sigma(u)$ that will give 
the meaning of $S$ if applied to $\sigma(u)$. Characters eliminate 
the need to talk about utterances. In fact, this account works as 
long as the context provides enough information so that the utterance 
is uniquely specified.

The execution of this idea requires care. Consider (\thesection.5).
%%%
\\[2mm]
\begin{tabular}{ll}
(\thesection.5) & {\tt Recall the definition on Page 142.} \\
\end{tabular}
\\[2mm]
%%%
What definition is to be recalled here? Given that you read this 
book now, it is that definition that appears on Page 142 in this 
very book. In other words, you would not know what definition is 
referred to if it was not clear which book you were reading. The 
question is: should the context provide this information or not?
If not, the sentence above is elliptical for 
%%%
\\[2mm]
\begin{tabular}{ll}
(\thesection.5) & {\tt Recall the definition on Page 142 in the book} \\
	& \quad {\tt you are reading now.} \\
\end{tabular}
\\[2mm]
%%%
This has eliminated the uncertainty at the price of treating the 
reference to a certain page as an elliptical statement. If this 
account is to work, however, the phrase that has been elided should 
be determinable from the meaning alone. This does not seem to be 
workable, though. Some books come in several volumes but are 
crossrefenced without telling you in which volume the definition is 
to be found. Say, you are told to look up Definition 3.2.4. Without 
knowing which volume contains Chapter 3 you cannot be certain that 
it means Definition 3.2.4 in the book you are reading. That book
might not contain that definition, in which case the reference 
is elliptical for something else. In sum, without reference to the 
world we are in it is impossible to say what phrase has been elided. 
So, one should refrain from treating such phrases as eliptical. 
Instead, has has to rely on the reader to do some reasoning in 
order to establish what is being referred to. 

The theory of characters is however not at ease with indexicals. 
Consider (\thesection.4). It means something like: the statement 
that I am indicating by some appropriate means is incorrect. But 
what is that statement? Suppose that statement is {\tt I want to 
be a mathematician.} Then, obviously, we should determine what 
{\tt I} refers to before we can say what that statement is. Suppose 
I point to a book on whose cover you find that sentence in addition 
to the letters {\it Paul Halmos}. You conclude that he is the 
author, you further conclude by inspecting it that it is his 
autobiography published in 1980, so that, finally, the statement 
we are after is {\tt Paul Halmos wanted to be a mathematician in 
1980.} or something like that. Nothing of that sort is revealed 
by (\thesection.4). 
