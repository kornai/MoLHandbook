\section{Grammar and Structure}
\label{einsvier}
%
%
%
Processes that replace strings by strings can often be considered 
as processes that successively replace parts of structures by 
structures. In this section we shall study processes of structure 
replacement. They can in principle operate on any kind of structure.
But we will restrict our attention to algorithms that generate ordered 
trees. There are basically two kinds of algorithms: the first is 
like the grammars of the previous section, generating intermediate 
structures that are not proper structures of the language; and the 
second, which generates in each step a structure of the language.

%%%
\index{multigraph}%%
%%%
Instead of graphs we shall deal with so--called {\it multigraphs}. 
A \textbf{directed multigraph} is
a structure $\auf V, \auf K_i : i < n\zu\zu$ where is $V$
%%%
\index{vertex}%%
%%%
a set, the set of \textbf{vertices}, and $K_i \subseteq V \times V$
%%%
\index{edge}%%
%%%
a disjoint set, the set of \textbf{edges} of type $i$. In our case
edges are always directed. We shall not mention this fact explicitly
later on. Ordered trees are one example
among many of (directed) multigraphs. For technical reasons we shall 
not exclude the case $V = \varnothing$, so that $\auf \varnothing, 
\auf \varnothing : i < n\zu\zu$ also is a multigraph. Next we shall 
introduce a
%%%
\index{vertex colouring}%%
%%%
colouring on the vertices. A \textbf{vertex--colouring} is a
function $\mu_V \colon V \pf F_V$ where $F_V$ is a nonempty set, the
set of \textbf{vertex colours}.
%%%
\index{vertex colour}%%
%%%
Think of the labelling as being a vertex colouring on the graph.
The principal structures are therefore vertex coloured multigraphs.
However, from a technical point of view the different edge relations
can also be viewed as colourings on the edges. Namely, if
$v$ and $w$ are vertices, we colour the edge $\auf v,w\zu$ by
the set $\{i : \auf v,w\zu \in K_i\}$. This set may be empty.
%%
\begin{defn}
%%%
\index{multigraph}%%%
\index{$\gamma$--graph}%%%
%%%%
An $\auf F_V, F_E\zu$--\textbf{coloured multigraph}
or simply a $\gamma$--\textbf{graph} (\textbf{over}
$F_V$ \textbf{and} $F_E$) is a triple $\auf V, \mu_V, \mu_E\zu$,
where $V$ is a (possibly empty) set and $\mu_V \colon V \pf F_V$ as
well as $\mu_E \colon V \times V \pf \wp(F_E)$ are functions.
\end{defn}
%%
\begin{figure}
\begin{center}
\begin{picture}(10,10)
    \put(1.5,2){\makebox(0,0)[r]{$w$}}
    \put(1.5,8){\makebox(0,0)[r]{$x$}}
    \put(8.5,8){\makebox(0,0)[l]{$y$}}
    \put(8.5,2){\makebox(0,0)[l]{$z$}}
        \put(4,5){\makebox(0,0)[r]{$p$}}
\put(2,2){\makebox(0,0){$\bullet$}}
\put(8,2){\makebox(0,0){$\bullet$}}
    \put(8,2){\vector(-1,0){5.8}}
        \put(5,1.8){\makebox(0,0)[t]{\tiny 1}}
    \put(8,2){\vector(0,1){5.8}}
        \put(8.2,5){\makebox(0,0)[l]{\tiny 2}}
\put(2,8){\makebox(0,0){$\bullet$}}
    \put(2,8){\vector(0,-1){5.8}}
        \put(1.8,5){\makebox(0,0)[r]{\tiny 2}}
    \put(2,8){\vector(1,-1){2.8}}
        \put(3.5,6.8){\makebox(0,0)[l]{\tiny 1}}
    \put(2,8){\vector(1,0){5.8}}
        \put(5,8.2){\makebox(0,0)[b]{\tiny 1}}
\put(8,8){\makebox(0,0){$\bullet$}}
\put(5,5){\makebox(0,0){$\bullet$}}
    \put(5,5){\circle{1}}
    \put(5,5){\vector(-1,-1){2.8}}
        \put(3.5,3.8){\makebox(0,0)[b]{\tiny 2}}
    \put(5,5){\vector(1,1){2.8}}
        \put(6.5,6.7){\makebox(0,0)[b]{\tiny 2}}
\put(5,0.5){\makebox(0,0){$\GG_1$}}
\end{picture}
\begin{picture}(10,10)
\put(3,3){\makebox(0,0){$\bullet$}}
    \put(2.5,3){\makebox(0,0)[r]{$a$}}
    \put(3,3){\vector(0,1){3.8}}
        \put(2.8,5){\makebox(0,0)[r]{\tiny 1}}
\put(3,7){\makebox(0,0){$\bullet$}}
    \put(2.5,7){\makebox(0,0)[r]{$b$}}
    \put(3,7){\vector(1,0){3.8}}
        \put(5,7.2){\makebox(0,0)[b]{\tiny 2}}
\put(7,7){\makebox(0,0){$\bullet$}}
    \put(7.5,7){\makebox(0,0)[l]{$c$}}
\put(5,0.5){\makebox(0,0){$\GG_2$}}
\end{picture}
%%
\begin{picture}(13,13)
    \put(1.5,2){\makebox(0,0)[r]{$w$}}
    \put(1.5,11){\makebox(0,0)[r]{$x$}}
    \put(11.5,11){\makebox(0,0)[l]{$y$}}
    \put(11.5,2){\makebox(0,0)[l]{$z$}}
\put(2,2){\makebox(0,0){$\bullet$}}
    \put(2,2){\vector(1,1){2.8}}
        \put(3.5,3.8){\makebox(0,0)[b]{\tiny 1}}
\put(11,2){\makebox(0,0){$\bullet$}}
    \put(11,2){\vector(-1,0){8.8}}
        \put(6.5,1.8){\makebox(0,0)[t]{\tiny 1}}
    \put(11,2){\vector(0,1){8.8}}
        \put(11.2,6.5){\makebox(0,0)[l]{\tiny 2}}
\put(2,11){\makebox(0,0){$\bullet$}}
    \put(2,11){\vector(0,-1){8.8}}
        \put(1.8,6.5){\makebox(0,0)[r]{\tiny 2}}
    \put(2,11){\vector(1,-1){2.8}}
        \put(3.5,9.8){\makebox(0,0)[l]{\tiny 1}}
    \put(2,11){\vector(1,0){8.8}}
        \put(6.5,11.2){\makebox(0,0)[b]{\tiny 1}}
\put(11,11){\makebox(0,0){$\bullet$}}
\put(5,5){\makebox(0,0){$\bullet$}}
    \put(4.5,5){\makebox(0,0)[r]{$a$}}
    \put(5,5){\vector(0,1){2.8}}
        \put(4.8,6.5){\makebox(0,0)[r]{\tiny 1}}
\put(5,8){\makebox(0,0){$\bullet$}}
    \put(4.5,8){\makebox(0,0)[r]{$b$}}
    \put(5,8){\vector(1,0){2.8}}
        \put(6.5,8.2){\makebox(0,0)[b]{\tiny 2}}
\put(8,8){\makebox(0,0){$\bullet$}}
    \put(8.5,8){\makebox(0,0)[l]{$c$}}
    \put(8,8){\vector(1,1){2.8}}
        \put(9.5,9.8){\makebox(0,0)[b]{\tiny 2}}
\put(6.5,0.5){\makebox(0,0){$\GG_3$}}
\end{picture}
\end{center}
\caption{Graph Replacement}
\label{fig:graphersetzung}
\end{figure}
%%%%
Now, in full analogy to the string case we shall distinguish
terminal and nonterminal colours. For simplicity, we shall study 
only replacements of a single vertex by a graph. Replacing a vertex 
by another structure means embedding a structure into some other 
structure. We need to be told how to do so. Before 
we begin we shall say something about the graph replacement in 
general. The reader is asked to look at
Figure~\ref{fig:graphersetzung}. The graph $\GG_3$ is the result of 
replacing in $\GG_1$ the encircled dot by $\GG_2$. The edge colours 
are $1$ and $2$ (the vertex colours pose no problems, so they are
omitted here for clarity).

Let $\GG = \auf E, \mu_E, \mu_K\zu$ be a $\gamma$--graph and
$M_1$ and $M_2$ be disjoint subsets of $E$ with $M_1 \cup M_2 = E$. 
Put $\GM_i = \auf M_i, \mu^i_V, \mu^i_E\zu$, where 
$\mu^i_V := \mu_V \restriction M_i$ and
$\mu^i_E := \mu_E \restriction M_i \times M_i$.
These graphs do not completely determine $\GG$
since there is no information on the edges between them.
We therefore define functions $\ein, \aus\colon
M_2 \times F_E \pf \wp(M_1)$, which for every vertex of $M_2$
and every edge colour name the set of all vertices of $M_1$
which lie on an edge with a vertex that either is directed
into $M_1$ or goes outwards from $M_1$.
%%
%%%
\index{$\ein(x,f)$, $\aus(x,f)$}%%%
%%%
\begin{subequations}
\begin{align}
\ein(x,f) & := \{y \in M_1 : f \in \mu_E(\auf y,x\zu)\} \\
\aus(x,f) & := \{y \in M_1 : f \in \mu_E(\auf x,y\zu)\}
\end{align}
\end{subequations}
%%
It is clear that $\GM_1$, $\GM_2$ and the functions
$\ein$ and $\aus$ determine $\GG$ completely.
In our example we have
%%
\begin{align}
\ein(p,1) & = \{x\} & \ein(p,2) & = \varnothing \\\notag
\aus(p,1) & = \varnothing & \aus(p,2) & = \{w,y\}
\end{align}
%%
Now assume that we want to replace $\GM_2$ by a different graph
$\GH$. Then not only do we have to know $\GH$  but also the 
functions $\ein, \aus \colon H \times F_E \pf
\wp(M_1)$. This, however, is not the way we wish to proceed
here. We want to formulate rules of replacement that are
general in that they do not presuppose exact knowledge about the
embedding context. We shall only assume that the functions
$\ein(x,f)$ and $\aus(x,f)$, $x \in H$, are systematically defined 
from the sets $\ein(y,g)$, $\aus(y,g)$, $y \in M_2$. We shall 
therefore only allow to specify how the sets of the first kind are 
formed from the sets of the second kind. This we do by means of
four so--called {\it colour functionals}. A \textbf{colour functional 
from} $\GH$ \textbf{to} $\GM_2$ is a map
%%%
\index{colour functional}%%
%%%
%%
\begin{equation}
\GF \colon H \times F_E \pf \wp(M_2 \times F_E)
\end{equation}
%%
In our case a functional is a function
from $\{a,b,c\} \times \{1,2\}$ to $\wp(\{p\} \times \{1,2\})$.
We can simplify this to a function from $\{a,b,c\} \times \{1,2\}$
to $\wp(\{1,2\})$. The colour functionals are called $\goth{II}$,
$\goth{IO}$, $\goth{OI}$ and $\goth{OO}$. 
%%%
\index{$\goth{II}$, $\goth{IO}$, $\goth{OI}$, $\goth{OO}$}%%%
%%%
For the example of Figure~\ref{fig:graphersetzung} we get the 
following colour functionals (we only give values when the 
functions do not yield $\varnothing$).
%%
\begin{align}
\goth{II} & \colon \auf b, 1\zu \mapsto \{1\} &
\goth{OI} & \colon \auf a, 2\zu \mapsto \{1\} \\\notag
\goth{IO} & \colon \varnothing &
\goth{OO} & \colon \auf c, 2\zu \mapsto \{2\}
\end{align}
%%
The result of substituting $\GM_2$ by $\GH$ by means of the
colour functionals from $\GF$ is denoted by $\GG[\GH/\GM_2 : \GF]$.
%%%
\index{$\GG[\GH/\GM : \GF]$}%%
%%%
This graph is the union of $\GM_1$ and $\GH$ together with the
functions $\ein^+$ and $\aus^+$, which are defined as follows.
%%
\begin{equation}
\begin{split}
\ein^+(x,f) := & \phantom{\cup}\, \bigcup \auf \ein(x,g) :
    g \in \goth{II}(x,f)\zu \\
    & \cup \bigcup \auf \aus(x,g) :
    g \in \goth{OI}(x,f) \zu \\
\aus^+(x,f) := & \phantom{\cup}\, \bigcup \auf \aus(x,g) :
    g \in \goth{OO}(x,f)\zu \\
    & \cup \bigcup \auf \ein(x,g) :
    g \in \goth{IO}(x,f) \zu
\end{split}
\end{equation}
%%
If $g \in \goth{II}(x,f)$ we say that an edge with
colour $g$ into $x$ is {\it transmitted as an ingoing edge of
colour $f$ to $y$}. If $g \in \goth{OI}(x,f)$ we say
that an edge with colour $g$ going out from $x$ is {\it transmitted
as an ingoing edge with colour $f$ to $y$}. Analogously for
$\goth{IO}$ and $\goth{OO}$. So, we do allow for an edge to
change colour and direction when being transmitted.
If edges do not change direction, we only need the functionals
$\goth{II}$ and $\goth{OO}$, which are then denoted simply by
$\goth{I}$ and $\goth{O}$. Now we look at the special case where
$M_2$ consists of a single element, say $x$. In this case a
colour functional simply is a function $\GF \colon H \times F_E \pf %
\wp(F_E)$.
%%
\begin{defn}
%%%
\index{graph grammar}%%
\index{graph grammar!context free}%%
\index{start graph}%%
%%%
A \textbf{context free graph grammar with edge replacement} ---
a \textbf{context free $\gamma$--grammar} for short --- is a
quintuple of the form
%%
\begin{equation}
\Gamma = \auf \GS, F_V, F^T_V, F_E, R\zu
\end{equation}
%%
in which $F_V$ is a finite set of vertex colours, $F_E$
a finite set of edge colours, $F_V^T \subseteq F_V$
a set of so--called \textbf{terminal vertex colours}, $\GS$ a
$\gamma$--graph over $F_V$ and $F_E$, the so--called
\textbf{start graph}, and finally $R$ a finite set of triples
$\auf X, \GH, \BF\zu$ such that $X \in F_V - F_V^T$ is a
nonterminal vertex colour, $\GH$ a $\gamma$--graph over $F_V$
and $F_E$ and $\BF$ is a matrix of colour functionals.
\end{defn}
%%
\index{derivation}%%
%%%%
A \textbf{derivation} in a $\gamma$--grammar $\Gamma$ is defined as 
follows.  For $\gamma$--graphs $\GG$ and $\GH$ with the colours
$F_V$ and $F_E$, $\GG \Pf^1_R \GH$ means that there is
$\auf X, \GM, \BF\zu \in R$ such that $\GH = \GG [\GM/\GX : \BF]$,
where $\GX$ is a subgraph consisting of a single vertex $x$
having the colour $X$. Further we define $\Pf^{\ast}_R$ to be
the reflexive and transitive closure of $\Pf^1_R$ and finally we put
$\Gamma \vdash \GG$ if $\GS \Pf^{\ast}_R \GG$. A derivation
\textbf{terminates} if there is no vertex with a nonterminal colour.
%%%
\index{$L_{\gamma}(\Gamma)$}%%%%
%%%
We write $L_{\gamma}(\Gamma)$ for the class of $\gamma$--graphs
that can be generated from $\Gamma$. Notice that the edge colours
only the vertex colours are used to steer the derivation.

We also define the productivity of a rule as the difference
between the cardinality of the replacing graph and the cardinality
of the graph being replaced. The latter is 1 in context free
$\gamma$--grammars, which is the only type we shall study here. So, 
the productivity is always $\geq - 1$. It equals $-1$ if the replacing 
graph is the empty graph. A rule has productivity $0$ if the replacing 
graph consists of a single vertex. In the exercises the reader will 
be asked to verify that we can dispense with rules of this kind.

Now we shall define two types of context free $\gamma$--grammars.
Both are context free as $\gamma$--grammars but the second type
can generate non--CFLs. This shows that the concept of 
$\gamma$--grammar is more general. We shall begin with ordinary 
CFGs. We can view them alternatively as
grammars for string replacement or as grammars that replace trees
by trees. For that we shall now assume that there are no rules of
the form $X \pf \varepsilon$. (For such rules generate trees
whose leaves are not necessarily marked by letters from $A$. This
case can be treated if we allow labels to be in $A_{\varepsilon}
= A \cup \{\varepsilon\}$, which we shall not do here.) Let 
$G = \auf \mbox{\tt S}, A, N, R\zu$ be such a grammar.
We put $F_V := A \cup (N \times 2)$.  We write $X^0$ for
$\auf X,0\zu$ and $X^1$ for $\auf X,1\zu$. $F_V^T := A \cup
N \times \{0\}$. $F_E := \{<, \sqsubset\}$. Furthermore, the
start graph consists of a single vertex labelled $\mbox{\tt S}^1$ 
and no edge. The rules of replacement are as follows.
Let $\rho = X \pf \alpha_0 \alpha_1 \dotsb \alpha_{n-1}$
be a rule from $G$, where none of the $\alpha_i$ is $\varepsilon$.
Then we define a $\gamma$--graph $\GH_{\rho}$
as follows. $H_{\rho} := \{y_i : i < n\} \cup \{x\}$.
$\mu_V(x) = X^0$, $\mu_V(y_i) = \alpha_i$ if
$\alpha_i \in A$ and $\mu_V(y_i) = \alpha_i^1$ if
$\alpha_i \in N$. 
%%
\begin{equation}
\begin{split}
\mu_E^{-1}(\{<\}) & := \{\auf y_i, x\zu : i < n\},  \\
\mu_E^{-1}(\{\sqsubset\}) & := \{\auf y_i, y_j \zu : i < j < n\}.
\end{split}
\end{equation}
%%
This defines $\GH_{\rho}$. Now we define the colour functionals.
For $u \in n$ we put
%%
\begin{equation}
\begin{split}
\GI_{\rho}(u, \sqsubset) & := \{\sqsubset\} & 
	 \GO_{\rho}(u, \sqsubset) & := \{\sqsubset\} \\
\GI_{\rho}(u, <)         & := \{<\} &
	\GO_{\rho}(u, <) & := \{<\}
\end{split}
\end{equation}
%%
Finally we put $\rho^{\gamma} := \auf X, \GH_{\rho}, \{\GI_{\rho}, %
\GO_{\rho}\}\zu$. $R^{\gamma} := \{\rho^{\gamma} : \rho \in R\}$.
%%
\begin{equation}
\gamma G := \auf \GS, F_E, F_E^T, F_T, R^{\gamma}\zu 
\end{equation}
%%
We shall show that this grammar yields exactly those trees that
we associate with the grammar $G$. Before we do so, a few remarks
are in order. The nonterminals of $G$ are now from a technical
viewpoint terminals since they are also part of the structure
that we are generating. In order to have any derivation at all
we define two equinumerous sets of nonterminals. Each nonterminal
$N$ is split into the nonterminal $N^1$ (which is nonterminal
in the new grammar) and $N^0$ (which is now a terminal vertex
colour). We call the first kind \textbf{active}, \textbf{nonactive}
%%%
\index{node!active}%%
\index{node!nonactive}%%
%%%
the second. Notice that the rules are formulated in such a way
that only the leaves of the generated trees carry active
nonterminals. A single derivation step is displayed in
Figure~\ref{fig:cfggraph}. In it, the rule $\mbox{\tt X} \pf %
\mbox{\tt AcA}$ has been applied to the tree to the left. The
result is shown on the right hand side.
%%
\begin{figure}
\begin{center}
\begin{picture}(13,13)
\put(2,5){\line(1,1){6}}
\put(2,5){\line(1,0){12}}
\put(14,5){\line(-1,1){6}}
\put(8,5){\makebox(0,0){$\bullet$}}
\put(8,5.5){\makebox(0,0)[b]{$\mbox{\tt X}^1$}}
\end{picture}
\qquad
\begin{picture}(13,13)
\put(2,5){\line(1,1){6}}
\put(2,5){\line(1,0){12}}
\put(14,5){\line(-1,1){6}}
\put(8,5){\makebox(0,0){$\bullet$}}
\put(8,5.5){\makebox(0,0)[b]{$\mbox{\tt X}^0$}}
\put(5,2){\line(1,0){6}}
\put(8,5){\line(-1,-1){3}}
% \put(8,5){\line(0,-1){3}}
\put(8,5){\line(1,-1){3}}
\put(5,1.5){\makebox(0,0){$\mbox{\tt A}^1$}}
\put(8,1.5){\makebox(0,0){\tt c}}
\put(11,1.5){\makebox(0,0){$\mbox{\tt A}^1$}}
\end{picture}
\end{center}
\caption{Replacement in a Context Free Grammar}
\label{fig:cfggraph}
\end{figure}
%%
It is easy to show that in each derivation only leaves
carry active nonterminals. This in turn shows that the
derivations of the $\gamma$--grammar are in one to one
correspondence with the derivations of the CFGs.
We put 
%%%
\begin{equation}
L_B(G) := h[L_{\gamma}(\gamma G)]
\end{equation}
\index{$L_B(G)$}%%%%
%%%
This is the class of trees generated by $\gamma G$, with $X^0$ and 
$X^1$ mapped to $X$ for each $X \in N$. 
The rules of $G$ can therefore be interpreted as conditions
on labelled ordered trees in the following way.
$\GC$ is called a \textbf{local subtree} of $\GB$
%%%
\index{subtree!local}%%
%%%
if (i) it has height 2 (so it does not possess inner nodes)
and (ii) it is maximal with respect to inclusion.
For a rule $\rho = X \pf Y_0 Y_1 \dotsb Y_{n-1}$ we define
$L_{\rho} := \{y_i : i < n\} \cup \{x\}$, $<_{\rho}\; :=
\{ \auf y_i, x\zu : i < n\}$, $\sqsubset_{\rho} := \{\auf y_i, y_j\zu
: i < j < n\}$, and finally $\ell_{\rho}(x) := X$, $\ell(y_i) := Y_i$.
$\GL_{\rho} := \auf L_{\rho}, <_{\rho}, \sqsubset_{\rho},
\ell_{\rho}\zu$. Now, an isomorphism between labelled ordered
trees $\GB = \auf B, <_{\GB}, \sqsubset_{\GB}, \ell_{\GB}\zu$
and $\GC = \auf C, <_{\GC}, \sqsubset_{\GC}, \ell_{\GC}\zu$
is a bijective map $h \colon B \pf C$ such that $h[<_{\GB}] 
= \; <_{\GC}$, $h[\sqsubset_{\GB}] = \sqsubset_{\GC}$
and $\ell_{\GC}(h(x)) = \ell_{\GB}(x)$ for all $x \in B$.
%%
\begin{prop}
Let $G = \auf \mbox{\tt S}, N, A, R\zu$. $\GB \in L_B(G)$ iff 
every local tree of $\GB$ is isomorphic to an $\GL_{\rho}$ such 
that $\rho \in R$.
\end{prop}
%%
\begin{thm}
Let \textbf{B} be a set of trees over an alphabet $A \cup N$
with terminals from $A$. Then $\textbf{B} = L_B(G)$ for a
CFG $G$ iff there is a finite set $\{\GL_i : i < n\}$ of trees 
of height 2 and an $S$ such that $\GB \in \textbf{B}$ exactly if
%%%
\begin{dingautolist}{192}
\item the root carries label $S$,
\item a label is terminal iff the node is a leaf, and
\item every local tree is isomorphic to some $\GL_i$.
\end{dingautolist}
\end{thm}
%%
We shall derive a few useful consequences from these considerations.
It is clear that $\gamma G$ generates trees that do not necessarily
have leaves with terminal symbols. However, we do know that the
leaves carry labels either from $A$ or from $N^1 := N \times \{1\}$
while all other nodes carry labels from $N^0 := N \times \{0\}$.
For a labelled tree we define the associated string sequence $k(\GB)$
in the usual way. This is an element of $(A \cup N^1)^{\ast}$.
Let $v \colon A \cup (N \times 2) \pf A \cup N$ be defined by
$v(a) := a$, $a \in A$ and $v(X^0) := v(X^1) := X$ for $X \in N$.
%%
\begin{lem}
Let $\gamma G \vdash \GB$ and $\vec{\alpha} = k(\GB)$.
Then $\vec{\alpha} \in (A \cup N^1)^{\ast}$
and $G \vdash \oli{v}(\vec{\alpha})$.
\end{lem}
%%
\proofbeg
Induction over the length of the derivation. If the length is
0 then $\vec{\alpha} = \mbox{\tt S}^1$ and $\oli{v}(\mbox{\tt S}^1) 
= \mbox{\tt S}$. Since $G \vdash \mbox{\tt S}$ this case is settled. 
Now let $\GB$ be the result of an application of some rule $\rho^{\gamma}$
on $\GC$ where $\rho = X \pf \vec{\gamma}$. We then have
$k(\GC) \in (A \cup N^1)^{\ast}$. The rule $\rho^{\gamma}$
has been applied to a leaf; this leaf corresponds to an 
occurrence of $X^1$ in $k(\GC)$. Therefore we have
$k(\GC) = \vec{\eta}_1 \conc X^1 \conc \vec{\eta}_2$. Then
$k(\GB) = \vec{\eta}_1 \conc \vec{\gamma} \conc \vec{\eta}_2$.
$k(\GB)$ is the result of a single application of the rule
$\rho$ from $k(\GC)$.
\proofend
%%
\begin{defn}
%%%
\index{cut}%%
%%%
Let $\GB$ be a labelled ordered tree. A \textbf{cut through} $\GB$ is
a maximal set that contains no two elements comparable by $<$. If 
$\GB$ is exhaustively ordered, a cut is linearly ordered and labelled, 
and then we also call the string associated to this set a 
\textbf{cut}.
\end{defn}
%%
\begin{prop}
Let $\gamma G \vdash \GB$ and let $\vec{\alpha}$ be a cut through
$\GB$. Then $G \vdash \oli{v}(\vec{\alpha})$.
\end{prop}
%%
This theorem shows that the tree provides all necessary information.
If you have the tree, all essential details of the derivation can
be reconstructed (up to commuting applications of rules). Now let
us be given a tree $\GB$ and let $\vec{\alpha}$ be a cut.
We say that an occurrence $C$ of $\vec{\gamma}$ in $\vec{\alpha}$
is a \textbf{constituent of category} $X$ \textbf{in} $\GB$ if 
this occurrence of $\vec{\gamma}$ in $\vec{\alpha}$ is that cut 
defined by $\vec{\alpha}$ on $\low{x}$ where $x$ carries the label 
$X$.
%%%
\index{constituent}%%
%%%
This means that $\vec{\alpha} = \vec{\eta}_1 \conc \vec{\gamma}
\conc \vec{\eta}_2$, $C = \auf \vec{\eta}_1, \vec{\eta}_2\zu$,
and $\low{x}$ contains exactly those nodes that do not belong
to $\vec{\eta}_1$ or $\vec{\eta}_2$. Further, let $G$ be a 
CFG. A substring occurrence of $\vec{\gamma}$ is 
a $G$--constituent of category $X$ in $\vec{\alpha}$ if there is a 
$\gamma G$--tree for which there exists a cut $\vec{\alpha}$ such that 
the occurrence $\vec{\gamma}$ is a constituent of category $X$. If 
$G$ is clear from the context, we shall omit it. 
%%
\begin{lem}
\label{wegmachen}
Let $\GB$ be a $\gamma G$--tree and $\vec{\alpha}$ a cut through
$\GB$. Then there exists a tree $\GC$ with associated string
$\vec{\gamma}$ and $\oli{v}(\vec{\gamma}) = \oli{v}(\vec{\alpha})$.
\end{lem}
%%
\begin{lem}
Let $G \vdash \vec{\alpha}_1 \conc \vec{\gamma} \conc \vec{\alpha}_2$, 
$C = \auf \vec{\alpha}_1, \vec{\alpha}_2\zu$ an occurrence 
of $\vec{\gamma}$ as a $G$--constituent of category $X$. Then $C$ is a
$G$--constituent occurrence of $X$ in $C(X) = \vec{\alpha}_1 \conc X \conc %
\vec{\alpha}_2$.
\end{lem}
%%
For a proof notice that if $\vec{\alpha}_1 \conc
\vec{\gamma} \conc \vec{\alpha}_2$ is a cut and
$\vec{\gamma}$ is a constituent of category $X$ therein then
$\vec{\alpha}_1 \conc X \conc \vec{\alpha}_2$ also is a cut.
%%
\begin{thm}[Constituent Substitution]
%%%
\index{Constituent Substitution Theorem}%%
%%%
Suppose that $C$ is an occurrence of $\vec{\beta}$ as
a $G$--constituent of category $X$. Furthermore, let
$X \vdash_G \vec{\gamma}$. Then $G \vdash C(\vec{\gamma}) =
\vec{\alpha}_1 \conc \vec{\gamma} \conc \vec{\alpha}_2$
and $C$ is a $G$--constituent occurrence of $\vec{\gamma}$ 
of category $X$.
\end{thm}
%%
\proofbeg
By assumption there is a tree in which $\vec{\beta}$
is a constituent of category $X$ in $\vec{\alpha}_1 \conc \vec{\beta} %
\conc \vec{\alpha}_2$.  Then there exists a cut $\vec{\alpha}_1 \conc X
\conc \vec{\alpha}_2$ through this tree, and by
Lemma~\ref{wegmachen} there exists a tree with associated
string $\vec{\alpha}_1 \conc X \conc \vec{\alpha}_2$.
Certainly we have that $X$ is a constituent in this tree.
However,  a derivation $X \vdash_G \vec{\gamma}$ can in
this case be extended to a $\gamma G$--derivation of
$\vec{\alpha}_1 \conc \vec{\gamma}\conc \vec{\alpha}_2$
in which $\vec{\gamma}$ is a constituent.
\proofend
%%
\begin{lem}
Let $G$ be a CFG. Then there exists a number $k_G$ such that 
for each derivation tree of a string of length $\geq k_G$ there 
are two constituents $\low{y}$ and $\low{z}$ of identical 
category such that $y \leq z$ or $z \leq y$,
and the associated strings are different.
\end{lem}
%%
\proofbeg
To begin, notice that nothing changes in our claim if we
eliminate the unproductive rules. This does not change the
constituent structure. Now let $\pi$ be the maximum of all
productivities of rules in $G$, and $\nu := |N|$. Then
let $k_G := (1 + \pi)^{\nu} + 1$. We claim that this is the
desired number.  (We can assume that $\pi > 0$. Otherwise
$G$ only generates strings of length 1, and then $k_G := 2$
satisfies our claim.) For let $\vec{x}$ be given such that
$|\vec{x}| \geq k_G$.  Then there exists in every derivation
tree a branch of length $> \nu$. (If not, there can be no more
than $\pi^{\nu}$ leaves.) On this branch we have two nonterminals
with identical label. The strings associated to these
nodes are different since we have no unproductive rules.
\proofend

%%%
\index{constituent part!left, right}%%
%%%
We say, an occurrence $C$ is a \textbf{left constituent part} 
(\textbf{right constituent part}) if $C$ is an occurrence of 
a prefix (suffix) of a constituent.  An occurrence of $\vec{x}$ 
contains a left constituent part $\vec{z}$ if some suffix of 
$\vec{x}$ is a left constituent part.  We also remark that 
if $\vec{u}$ is a left constituent part and a proper substring 
of $\vec{x}$ then $\vec{x} = \vec{v}\,\vec{v}_1\vec{u}$ with 
$\vec{v}_1$ a possibly empty sequence of constituents and 
$\vec{v}$ a right constituent part. This will 
be of importance in the sequel.
%%
\begin{lem}
\label{lem:halb}
Let $G$ be a  CFG. Then there exists a number $k'_G$ such that 
for every derivation tree of a string $\vec{x}$ and every occurrence 
in $\vec{x}$ of a string $\vec{z}$ of length $\geq k'_G$ 
$\vec{z}$ contains two different left or two different right 
constituent parts $\vec{y}$ and $\vec{y}_1$ of constituents that 
have the same category. Moreover, $\vec{y}$ is a prefix 
of $\vec{y}_1$ or $\vec{y}_1$ a prefix of $\vec{y}$ in case that 
both are left constituent parts, and $\vec{y}$ is a suffix of 
$\vec{y}_1$ or $\vec{y}_1$ a suffix of $\vec{y}$ in case that both 
are right constituent parts.
\end{lem}
%%
\proofbeg
Let $\nu := |N|$ and let $\pi$ be the maximal productivity of a rule from
$G$.  We can assume that $\pi \geq 2$. Put $k'_G := (2 + 2\pi)^{\nu}$.
We show by induction on the number $m$ that a string of length
$\geq (2+ 2\pi)^m$ has at least $m$ left or at least $m$ right
constituent parts that are contained in each other. If $m = 1$
the claim is trivial. Assume that it holds for $m \geq 1$. We shall 
show that it also holds for $m+1$.  Let $\vec{z}$ be of length 
$\geq (2+2\pi)^{m+1}$. Let $\vec{x} = \prod_{i < 2\pi+2} \vec{x}_i$ 
for certain $\vec{x}_i$ with length at least $(2 + 2\pi)^m$. By 
induction hypothesis each $\vec{x}_i$ contains at least $m$ constituent
parts. Now we do not necessarily have $(2\pi +2)m$ constituent
parts in $\vec{x}$. For if $\vec{x}_i$ contains a left part then
$\vec{x}_j$ with $j > i$ may contain the corresponding right part.
(There is only one. The sections in between contain subwords of
that constituent occurrence.) For each left constituent part
we count at most one (corresponding) right constituent part.
In total we have at least $(1 + \pi)m \geq m+1$ constituent parts.
However,  we have to verify  that at least $m+1$ of these are
contained inside each other. Assume  this is not the case, for all
$i$. Then $\vec{x}_i$, $i < 2\pi + 2$, contains exactly $m$ left or 
exactly $m$ right constituent parts.
Case 1. $\vec{x}_0$ contains $m$ left constituent parts inside
each other. If $\vec{x}_1$ also contains $m$ left constituent
parts inside each other, we are done. Now suppose that this is
not the case. Then $\vec{x}_1$ contains $m$ right constituent
parts inside each other.  Then we obviously get $m$ entire
constituents stacked inside each other. Again, we would be done
if $\vec{x}_2$ contained $m$ right constituent parts inside
each other. If not, then $\vec{x}_2$ contains exactly $m$ left
constituent parts. And again we would be done if these would
not correspond to exactly $m$ right part that $\vec{x}_3$
contains. And so on. Hence we get a sequence of length $\pi$
of constituents which each contain $m$ constituents stacked
inside each other. Now three cases arise: (a) one of the 
constituents is a left part of some constituent, (b) one of 
the constituent is a right part of some constituent. (For if 
neither is the case, we have a rule of arity $> \pi$, a 
contradiction.) In Case (a) we evidently have $m+1$ left 
constituent parts stacked inside each other, and in Case (b) 
$m+1$ right constituent parts. Case 2. $\vec{x}_0$ contains $m$
right hand constituents stacked inside each other. Similarly.
This shows our auxiliary claim. Putting $m := \nu + 1$ the main
claim now follows.
\proofend
%%
\begin{thm}[Pumping Lemma]
%%%
\index{Pumping Lemma}%%%
\label{thm:pumplemma}
%%%
Given a CFL $L$ there exists a $p_L$ such that
for every string $\vec{z} \in L$ of length at least $p_L$ and an
occurrence of a string $\vec{r}$  of length at least $p_L$ in
$\vec{z}$, $\vec{z}$ possesses a decomposition
%%
\begin{equation}
\vec{z} = \vec{u} \conc \vec{x} \conc \vec{v}
\conc \vec{y} \conc \vec{w}
\end{equation}
%%
such that the following holds.
%%
\begin{dingautolist}{192}
\item
$\vec{x} \conc \vec{y} \neq \varepsilon$. 
%$\vec{u} \conc \vec{w} \neq \varepsilon$.
\item
Either the occurrence of $\vec{x}$ or the occurrence of
$\vec{y}$ is contained in the specified occurrence of $\vec{r}$.
\item
$\{\vec{u} \conc {\vec{x}\,}^i \conc \vec{v} \conc {\vec{y}\,}^i
\conc \vec{w} : i \in \omega\} \subseteq L$.
\end{dingautolist}
%%
(The last property is called the \textbf{pumpability} of the
substring occurrences of $\vec{x}$ and $\vec{y}$.) Alternatively, 
in place of \ding{193} one may require that $|\vec{v}| \leq p_L$. 
Further we can choose $p_L$ in such a way that every derivable
string $\vec{\gamma}$ with designated occurrences of a string
$\vec{\alpha}$ of length $\geq p_S$ can be decomposed in the way
given.
%%
\end{thm}
%%
\proofbeg
Let $G$ be a grammar which generates $L$. Let $p_L$
be the constant defined in Lemma~\ref{lem:halb}. We look at
a $G$--tree of $\vec{z}$ and the designated occurrence of
$\vec{r}$. Suppose that $\vec{r}$ has length at least $p_L$.
Then there are two left or two right constituent parts of
identical category contained in $\vec{r}$. Without loss of
generality we assume that $\vec{r}$ contains two left parts.
Suppose that these parts are not fully contained in $\vec{r}$.
Then $\vec{r} = \vec{s}\,\vec{x}\,\vec{s}_1$ where $\vec{x}\,\vec{s}_1$
and $\vec{s}_1$ are left constituent parts of identical category,
say $X$. Now $|\vec{x}| > 0$. There are $\vec{s}_2$ and $\vec{y}$
such that $\vec{v} := \vec{s}_1\vec{s}_2$ and
$\vec{x}\,\vec{s}_1\vec{s}_2\vec{y}$ are constituents of 
category $X$.

Hence there exists a decomposition
%%
\begin{equation}
\vec{z} = \vec{u} \conc \vec{x} \conc \vec{v}
\conc \vec{y} \conc \vec{w} 
\end{equation}
%%
where $\vec{v}$ is a constituent of the same category
as $\vec{x}\,\vec{v}\,\vec{y}$ satisfying \ding{192} and
\ding{193}. By the Constituent Substitution Theorem we may 
replace the occurrence of $\vec{x}\,\vec{v}\,\vec{y}$ by $\vec{v}$
as well as $\vec{v}$ by $\vec{x}\,\vec{v}\,\vec{y}$.
This yields \ding{194}, after an easy induction.
Now let the smaller constituent part be contained in $\vec{r}$
but not the larger one. Then we have a decomposition
$\vec{r} = \vec{s}\,\vec{x}\,\vec{v}\,\vec{s}_1$ such that $\vec{v}$
is a constituent part of category $X$ and $\vec{x}\,\vec{v}\,\vec{s}_1$
a left constituent part of a constituent of category $X$. Then there
exists a $\vec{s}_2$ such that also $\vec{x}\vec{v}\vec{s}_1\vec{s}_2$
is a constituent of category $X$. Now put $\vec{y} :=
\vec{s}_1\vec{s}_2$. Then we also have $\vec{y} \neq \varepsilon$.
The third case is if both parts are proper substrings of $\vec{r}$.
Also here we find the desired decomposition. If we want to have in 
place of \ding{193} that $\vec{v}$ is as small as possible then 
notice that $\vec{v}$ already is a
constituent. If it has length $\geq (1+ \pi)^{\nu}$ then
there is a decomposition of $\vec{v}$ such that it contains
pumpable substrings. Hence in place of \ding{193} we may require 
that $|\vec{v}| \leq p_G$.
\proofend

The Pumping Lemma can be stated more concisely as follows.
For every large enough derivable string $\vec{x}$ there exist 
contexts $C$, $D$, where $C \neq \auf \varepsilon, \varepsilon\zu$,
and a string $\vec{y}$ such $\vec{x} = D(C(\vec{y}))$,
and $D(C^k(\vec{y})) \in L$ for every $k \in \omega$.
The strongest form of a pumping lemma is the following.  
Suppose that we have two decompositions into pumping pairs 
$\vec{u}_1\conc \vec{x}_1 \conc \vec{v}_1\conc\vec{y}_1\conc\vec{w}_1$, 
$\vec{u}_2\conc \vec{x}_2 \conc \vec{v}_2\conc\vec{y}_2\conc\vec{w}_2$. 
We say that the two pairs are \textbf{independent} 
%%%
\index{independent pumping pair}%%
%%%
if either (1a) $\vec{u}_1\conc \vec{x}_1\conc\vec{v}_1\conc\vec{y}_1$ 
is a prefix of $\vec{u}_2$, or
(1b) $\vec{u}_2\conc \vec{x}_2\conc\vec{v}_2\conc\vec{y}_2$ is a 
prefix of $\vec{u}_1$, or
(1c) $\vec{u}_1\conc\vec{x}_1$ is a prefix of $\vec{u}_2$ and 
$\vec{y}_1\conc\vec{w}_1$ a suffix of $\vec{w}_2$, or
(1d) $\vec{u}_2\conc\vec{x}_2$ is a prefix of $\vec{u}_1$ and 
$\vec{y}_2\conc\vec{w}_2$ a suffix of $\vec{w}_1$ and 
(2) each of them can be pumped any number of times independently 
of the other. 
\nocite{manasterrameretal:ogden}
%%%%
\begin{thm}[Manaster-Ramer \& Moshier \& Zeitman]
\label{thm:multipump}
Let $L$ be a CFL. Then there exists a number 
$m_{L}$ such that if $\vec{x} \in L$ and we are given $k m_L$ 
occurrences of letters in $\vec{x}$ there are $k$ independent 
pumping pairs, each of which contains at least one and at most 
$m_L$ of the occurrences. 
\end{thm}
%%%%
This theorem implies the well--known \textbf{Ogden's Lemma} (see 
\cite{ogden:helpful}), which says that given at least $m_L$ 
occurrences of letters, there exists a pumping pair containing 
at least one and at most $m_L$ of them.

Notice that in all these theorems we may choose $i = 0$ as well. This 
means that not only we can pump `up' the string so that it becomes longer
except if $i = 1$, but we may also pump it `down' ($i = 0$)
so that the string becomes shorter. However, one can pump down 
only once. Using the Pumping Lemma we can show that the language
$\{\mbox{\tt a}^n \mbox{\tt b}^n \mbox{\tt c}^n : n \in \omega\}$
is not context free.

For suppose the contrary. Then there is an $m$ such that
for all $k \geq m$ the string $\mbox{\tt a}^k \mbox{\tt b}^k \mbox{\tt c}^k$
can be decomposed into
%%
\begin{equation}
\mbox{\tt a}^k \mbox{\tt b}^k \mbox{\tt c}^k
= \vec{u} \conc \vec{v} \conc \vec{w} \conc \vec{x}
\conc \vec{y}
\end{equation}
%%
Furthermore there is an $\ell > k$ such that
%%
\begin{equation}
\mbox{\tt a}^{\ell} \mbox{\tt b}^{\ell} \mbox{\tt c}^{\ell} =
\vec{u} \conc {\vec{v}\,}^2 \conc \vec{w} \conc {\vec{x}\,}^2 \conc
\vec{y}
\end{equation}
%%
The string $\vec{v} \conc \vec{x}$ contains exactly $\ell - k$ times
the letters {\tt a}, {\tt b} and {\tt c}. It is clear that we must
have $\vec{v} \subseteq \mbox{\tt a}^{\ast} \cup
\mbox{\tt b}^{\ast} \cup \mbox{\tt c}^{\ast}$.
For if $\vec{v}$ contains two distinct letters, say
{\tt b} and {\tt c}, then $\vec{v}$ contains an occurrence
of {\tt b} before an occurrence of {\tt c} (certainly not the
other way around). But then ${\vec{v}\,}^2$ contains
an occurrence of {\tt c} before an occurrence of {\tt b},
and that cannot be. Analogously it is shown that
$\vec{y} \in \mbox{\tt a}^{\ast} \cup \mbox{\tt b}^{\ast}
\cup \mbox{\tt c}^{\ast}$. But this is a contradiction.
We shall meet this example of a non--CFL quite often in the 
sequel.

The second example of a context free graph grammar shall be
the so--called {\it tree adjunction grammars}. We take an
alphabet $A$ and a set $N$ of nonterminals.
%%%
\index{centre tree}%%
\index{adjunction tree}%%
%%%
A \textbf{centre tree} is an ordered labelled tree over $A \cup N$
such that all leaves have labels from $A$ all other nodes labels
from $N$. An \textbf{adjunction tree} is an ordered labelled tree 
over $A \cup N$ which is distinct from ordinary trees in that of 
the leaves there is exactly one with a nonterminal label; this 
label is the same as that of the root.  Interior nodes have 
nonterminal labels. We require that an adjunction tree has at 
least one leaf with a terminal symbol.
%%%
\index{tree adjunction grammar!unregulated}%%
\index{UTAG}%%
%%%
An \textbf{unregulated tree adjunction grammar}, briefly \textbf{UTAG},
over $N$ and $A$, is a quadruple $\auf \BC, N, A, \BA\zu$ where
$\BC$ is a finite set of centre trees over $N$ and $A$, and $\BA$ a
finite set of adjunction trees over $N$ and $A$. An example of a
tree adjunction is given in Figure~\ref{fig:baumadjunktion}. The
tree to the left is adjoined to a centre tree with root $X$ and
associated string {\tt bXb}; the result is shown to the right.
Tree adjunction can formally be defined as follows.
%%
\begin{figure}
\begin{center}
\begin{picture}(18,18)
\put(9,16){\line(-1,-1){7}}
\put(9,16){\line(1,-1){7}}
\put(2,9){\line(1,0){14}}
\put(9,12.5){\makebox(0,0){\tt X}}
\put(9,12){\line(-1,-1){3}}
\put(9,12){\line(1,-1){3}}
\put(2,8.5){\makebox(0,0){\tt Y}}
\put(4,8.5){\makebox(0,0){\tt c}}
\put(6,8.5){\makebox(0,0){\tt a}}
\put(9,8.5){\makebox(0,0){\tt A}}
\put(12,8.5){\makebox(0,0){\tt a}}
\put(14,8.5){\makebox(0,0){\tt A}}
\put(16,8.5){\makebox(0,0){\tt a}}
\end{picture}
\qquad
\begin{picture}(18,18)
\put(2,8.5){\makebox(0,0){\tt Y}}
\put(4,8.5){\makebox(0,0){\tt c}}
\put(6,2.5){\makebox(0,0){\tt a}}
\put(9,2.5){\makebox(0,0){\tt A}}
\put(12,2.5){\makebox(0,0){\tt a}}
\put(14,8.5){\makebox(0,0){\tt A}}
\put(16,8.5){\makebox(0,0){\tt a}}
\put(6,3){\line(1,0){6}}
\put(6,3){\line(1,1){3}}
\put(12,3){\line(-1,1){3}}
\put(4,5.5){\makebox(0,0){\tt b}}
\put(4,6){\line(1,1){5}}
\put(4,6){\line(1,0){10}}
\put(14,5.5){\makebox(0,0){\tt b}}
\put(14,6){\line(-1,1){5}}
\put(2,9){\line(1,1){7}}
\put(2,9){\line(1,0){5}}
\put(16,9){\line(-1,1){7}}
\put(16,9){\line(-1,0){5}}
\put(9,11.5){\makebox(0,0){\tt X}}
\put(9,6.5){\makebox(0,0){\tt X}}
\end{picture}
\end{center}
\caption{Tree Adjunction}
\label{fig:baumadjunktion}
\end{figure}
%%
Let $\GB = \auf B, <, \sqsubset, \ell\zu$ be a tree and
$\GA = \auf A, <, \sqsubset, m \zu$ an adjunction tree.
We assume that $r$ is the root of $\GA$ and that $s$ is
the unique leaf such that $m(r) = m(s)$. Now let $x$ be a node
of $B$ such that $\ell(x) = m(r)$. Then the replacement
of $x$ by $\GB$ is defined by naming the colour functionals.
These are
%%
\begin{align}
\goth{II}_{\rho}(y,\sqsubset) & :=
    \begin{cases}
        \{\sqsubset, <\} & \text{if $s \sqsubset y$,} \\
                \{\sqsubset\} & \text{else.}
    \end{cases}
     &
\goth{OI}_{\rho}(y, \sqsubset) & := \varnothing \\\notag
\goth{IO}_{\rho}(y, \sqsubset) & := 
    \begin{cases}
    \{<\} & \text{if $y \sqsubset s$,} \\
    \varnothing & \text{else.}
    \end{cases} &
\goth{OO}_{\rho}(y,\sqsubset) & := \{\sqsubset\} \\
%%%
\goth{II}_{\rho}(y, <)         & := 
	\begin{cases} 
		\{<\} & \text{if $y \geq s$,} \\
		\varnothing & \text{else.} 
	\end{cases}
	&
    \goth{IO}_{\rho}(y, <) & := \varnothing \\\notag
\goth{OI}{\rho}(y, <) & := \varnothing &
    \goth{OO}_{\rho}(y,<) & := \{<\}
\end{align}
%%
Two things may be remarked.  First, instead of a single
start graph we have a finite set of them. This can be
remedied by standard means. Second, all vertex colours
are terminal as well as nonterminal. One may end the
derivation at any given moment. We have noticed in
connection with grammars  for strings that this can be
remedied. In fact, we have not defined context free 
$\gamma$--grammars but context free quasi 
$\gamma$--grammars$^{\ast}$. However, we shall refrain
from being overly pedantic. Suffice it to note that the
adjunction grammars do not define the same kind of
generative process if defined exactly as above.

Finally we shall give a graph grammar which generates all
strings of the form $\mbox{\tt a}^n \mbox{\tt b}^n
\mbox{\tt c}^n$, $n>0$. The idea for this grammar is due to
Uwe M\"onnich 
%%%
\index{M\"onnich, Uwe}%%%
%%%
\shortcite{moennich:cloning}. We shall exploit
the fact that we may think of terms as structures. We posit
a ternary symbol, {\tt F}, which is nonterminal, and another
ternary symbol, {\tt f}, which is terminal. Further, there
is a binary terminal symbol $^{\smallfrown}$. The rules are
as follows. (To enhance readability we shall not write terms
in Polish Notation but by means of brackets.)
%%
\begin{equation}
\begin{split}
\mbox{\tt F}(x, y, z) & \pf \mbox{\tt F}(\mbox{\tt a}{^{\smallfrown}x},
\mbox{\tt b}{^{\smallfrown}y},
    \mbox{\tt c}{^{\smallfrown}z}), \\
\mbox{\tt F}(x,y,z) & \pf \mbox{\tt f}(x,y,z). \\
\end{split}
\end{equation}
%%
%%%
\index{term replacement system}%%
%%%
These rules constitute a so--called \textbf{term replacement system}.
The start term is $\mbox{\tt F}(\mbox{\tt a}, \mbox{\tt b}, \mbox{\tt c})$.
Now suppose that $u \pf v$ is a rule and that we have derived
a term $t$ such that $u^{\sigma}$ occurs in $t$ as a subterm.
Then we may substitute this occurrence by $v^{\sigma}$.
Hence we get the following derivations.
%%
\begin{equation}
\begin{split}
\mbox{\tt F}(\mbox{\tt a},\mbox{\tt b},\mbox{\tt c}) &
    \pf \mbox{\tt f}(\mbox{\tt a},\mbox{\tt b},\mbox{\tt c}), \\
\mbox{\tt F}(\mbox{\tt a},\mbox{\tt b},\mbox{\tt c}) &
    \pf \mbox{\tt F}(\mbox{\tt a}{^{\smallfrown}\mbox{\tt a}},
        \mbox{\tt b}{^{\smallfrown}\mbox{\tt b}},
    \mbox{\tt c}{^{\smallfrown}\mbox{\tt c}}) \\
    & \pf \mbox{\tt f}(\mbox{\tt a}{^{\smallfrown}\mbox{\tt a}},
    \mbox{\tt b}{^{\smallfrown}\mbox{\tt b}},
        \mbox{\tt c}{^{\smallfrown}\mbox{\tt c}}) \\
\mbox{\tt F}(\mbox{\tt a},\mbox{\tt b},\mbox{\tt c}) &
    \pf \mbox{\tt F}(\mbox{\tt a}{^{\smallfrown}\mbox{\tt a}},
        \mbox{\tt b}{^{\smallfrown}\mbox{\tt b}},
    \mbox{\tt c}{^{\smallfrown}\mbox{\tt c}}) \\
	&
    \pf \mbox{\tt F}(\mbox{\tt a}{^{\smallfrown}{(\mbox{\tt
        a}^{\smallfrown}\mbox{\tt a})}},
    \mbox{\tt b}{^{\smallfrown}{(\mbox{\tt
        b}^{\smallfrown}\mbox{\tt b})}},
    \mbox{\tt c}{^{\smallfrown}{(\mbox{\tt
        c}^{\smallfrown}\mbox{\tt c})}}) \\
    & \pf \mbox{\tt f}(\mbox{\tt a}{^{\smallfrown}{(\mbox{\tt
        a}^{\smallfrown}\mbox{\tt a})}},
    \mbox{\tt b}{^{\smallfrown}{(\mbox{\tt
        b}^{\smallfrown}\mbox{\tt b})}},
    \mbox{\tt c}{^{\smallfrown}{(\mbox{\tt
        c}^{\smallfrown}\mbox{\tt c})}})
\end{split}
\end{equation}
%%
Notice that the terms denote graphs here. We make use of the
dependency coding. Hence the associated strings to these terms
are {\tt abc}, {\tt aabbcc} and {\tt aaabbbccc}.

In order to write a graph grammar which generates the graphs
for these terms we shall have to introduce colours for edges.
Put $F_E := \{\uli{0}, \uli{1}, \uli{2}, \sqsubset, <\}$,
$F_V := \{\mbox{\tt F}, \mbox{\tt f}, \mbox{\tt a}, \mbox{\tt b}, 
\mbox{\tt c}\}$, and $F_V^T := \{\mbox{\tt f}, \mbox{\tt a}, 
\mbox{\tt b}, \mbox{\tt c}\}$.
The start graph is as follows. It has four vertices,
$p$, $q$, $r$ and $s$. ($<$ is empty (!), and
$q \sqsubset r \sqsubset s$.) The labelling is
$p \mapsto \mbox{\tt F}$, $q \mapsto \mbox{\tt a}$,
$r \mapsto \mbox{\tt b}$ and $s \mapsto \mbox{\tt c}$.
%%
\begin{equation}
\begin{array}{l}
\begin{picture}(6,6)
\put(1,1.5){\makebox(0,0){$\bullet$}}
    \put(1,1.5){\vector(1,1){2.8}}
    \put(1.7,3){\makebox(0,0)[r]{$\uli{0}$}}
    \put(1,.5){\makebox(0,0){\tt a}}
\put(4,1.5){\makebox(0,0){$\bullet$}}
    \put(4,1.5){\vector(0,1){2.8}}
    \put(3.5,3){\makebox(0,0)[r]{$\uli{1}$}}
    \put(4,.5){\makebox(0,0){\tt b}}
\put(7,1.5){\makebox(0,0){$\bullet$}}
    \put(7,1.5){\vector(-1,1){2.8}}
    \put(6.3,3){\makebox(0,0)[l]{$\uli{2}$}}
    \put(7,.5){\makebox(0,0){\tt c}}
\put(4,4.5){\makebox(0,0){$\bullet$}}
    \put(4,5.5){\makebox(0,0){\tt F}}
\end{picture}
\end{array}
\end{equation}
%%
There are two rules of replacement. The first can be written
schematically as follows. The root, $x$, carries the label {\tt F}
and has three incoming edges; their colours are
$\uli{0}$, $\uli{1}$ and $\uli{2}$. These come from three
disjoint subgraphs, $\GG_0$, $\GG_1$ and $\GG_2$, which are
ordered trees with respect to $<$ and $\sqsubset$ and in which
there are no edges with colour $\uli{0}$, $\uli{1}$ and $\uli{2}$.
In replacement, $x$ is replaced by a graph consisting of
seven vertices, $p$, $q_i$, $r_i$ and $s_i$, $i < 2$,
where $q_i \sqsubset r_j \sqsubset s_k$, $i,j,k < 2$,
and $q \stackrel{\uli{0}}{\pf} p$, $r \stackrel{\uli{1}}{\pf} p$
and $s \stackrel{\uli{2}}{\pf} p$. $< = \{\auf q_1, q_0\zu,
\auf r_1, r_0\zu, \auf s_1, s_0\zu\}$. The colouring is
%%
\begin{equation}
\begin{array}{rlrlrlrl}
p & \mapsto \mbox{\tt F} & q_0 & \mapsto ^{\smallfrown} & 
r_0 & \mapsto ^{\smallfrown} & s_0 & \mapsto ^{\smallfrown} 
\\ 
 & & q_1 & \mapsto \mbox{\tt a} & r_1 & \mapsto \mbox{\tt b} 
& s_1 & \mapsto \mbox{\tt c}
\end{array}
\end{equation}
%%
(With $\{p,q_0, r_0, s_0\}$ we reproduce the begin situation.)
The tree $\GG_0$ is attached to $q_0$ to the right of $q_1$,
$\GG_1$ to $r_0$ to the right of $r_1$ and
$\GG_2$ to $s_0$ to the right of $s_1$.
Additionally, we put $x < p$ for all vertices $x$ of the $\GG_i$.
(So, the edge $\auf x, p\zu$ has colour $<$ for all such $x$.)
By this we see to it that in each step the union of the
relations $<$, $\uli{0}$, $\uli{1}$ and $\uli{2}$ is the
intended tree ordering and that there always exists
an ingoing edge with colour $\uli{0}$, $\uli{1}$ and
$\uli{2}$ into the root.

The second replacement rule replaces the root
by a one vertex graph with label {\tt f} at the root.
This terminates the derivation. The edges with label
$\uli{0}$, $\uli{1}$ and $\uli{2}$ are transmitted under the
name $<$. This completes the tree. It has the desired form.
%%
\vplatz
\exercise
Strings can also be viewed as multigraphs with only one edge 
colour. Show that a CFG for strings can also be defined as a 
context free $\gamma$--grammar on strings. We shall show in 
Section~\ref{kap2}.\ref{kap2-5} that CFLs can also be generated by UTAGs,
but that the converse does not hold.
%%
\vplatz
\exercise
Show that for every context free $\gamma$--grammar $\Gamma$
there exists a context free  $\gamma$--grammar $\Delta$
which has no rules of productivity $- 1$ and which generates
the same class of graphs.
%%
\vplatz
\exercise
Show that for every context free $\gamma$--grammar there exists
a context free $\gamma$--grammar with the same yield and no
rules of productivity $\leq 0$.
%%
\vplatz
\exercise
Define unregulated string adjunction grammars in a similar way to 
UTAGs. Take note of the fact that these are quasi--grammars. Characterize 
the class of strings generated  by these grammars in terms of 
ordinary grammars.
%%
\vplatz %%
\exercise %%
Show that the language $\{\vec{w} \conc \vec{w}
: \vec{w} \in A^{\ast}\}$ is not context free but that it
satisfies the Pumping Lemma. 
%%%
\index{Pumping Lemma}%%%
\index{Interchange Lemma}%%%
%%%
(It does not satisfy the Interchange Lemma (\ref{thm:interchange}).)
%%
