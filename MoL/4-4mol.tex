\section{Binding and Quantification}
\label{kap6-4a}
%
%
%
Quantification and binding are one of the most intricate phenomena
of formal semantics. Examples of quantifiers we have seen already:
the English phrases {\tt every} and {\tt some}, and $\forall$ and 
$\exists$ of predicate logic. Examples of binding without quantification 
can be found easily in mathematics. The integral
%%
\begin{equation}
h(\vec{y}) := \int_0^1 f(x,\vec{y})dx
\end{equation}
%%
is a case in point. The integration operator takes a function (which
may have parameters) and returns its integral over the interval 
$[0,1]$. What this has in common with quantification is that the 
function $h(\vec{y})$ does not depend on $x$. Likewise, the limit 
$\lim_{n \pf \infty} a_n$ of a convergent series 
$a \colon \omega \pf \BR$, is independent of $n$.
(The fact that these operations are not everywhere defined shall
not concern us here.) So, as with quantifiers, integration and
limits take entities that depend on a variable $x$ and return an
entity that is independent of it. The easiest way to analyze this
phenomenon is as follows. Given a function $f$ that depends on $x$,
$\lambda x.f$ is a function which is independent of $x$. Moreover,
everything that lies encoded in $f$ is also encoded in $\lambda
x.f$. So, unlike quantification and integration,
$\lambda$--abstraction does not give rise to a loss of
information. This is ensured by the identity $(\lambda x.f)x = f$.
Moreover, extensionality ensures that abstraction also does not
add any information: the abstracted function is essentially
nothing more than the graph of the function. $\lambda$--abstraction
therefore is {\it the\/} mechanism of binding. Quantifiers,
integrals, limits and so on just take the
$\lambda$--abstract and return a value. This is exactly how we
have introduced the quantifiers: $\exists x.\varphi$
was just an abbreviation of {\tt \GSi(\tlambda$x$.$\varphi$)}. 
Likewise, the integral can be decomposed into two steps: first, 
abstraction of a variable and then the actual integration. Notice, 
how the choice of variable matters:
%%
\begin{equation}
y/3 = \int_0^1 x^2y dx \neq x/2 = \int_0^1 x^2y dy
\end{equation}
%%
The notation $dx$ actually does the same as $\lambda x$: it shows
us over which variable we integrate. We may define integration
as follows. First, we define an operation $I \colon {\BR}^{\BR} \pf {\BR}$,
which performs the integration over the interval $[0,1]$ of $f
\in {\BR}^{\BR}$. Then we define
%%
\begin{equation}
\int_0^1 f(x)dx := I(\lambda x.f)
\end{equation}
%%
This definition decouples the definition of the actual integration
from the binding process that is involved. In general, any operator
$O\auf x_i : i < n\zu.M$ which binds the variables $x_i$, $i < n$,
and returns a value, can be defined as
%%
\begin{equation}
\label{eq:integral}
O\auf x_i : i < n\zu.M := \wht{O}(\lambda x_0.\lambda x_1.\dotsb.%
\lambda x_{n-1}.M)
\end{equation}
%%
for a suitable $\wht{O}$. In fact, since $O\vec{x}.M$ does not
depend on $\vec{x}$, we can use \eqref{eq:integral} to define $\wht{O}$.
What this shows is that $\lambda$--calculus can be used as a
general tool for binding. It also shows that we can to some
extent get rid of explicit variables, something that is quite
useful for semantics. The elimination of variables 
removes a point of arbitrariness in the representation that
makes meanings nonunique. In this section, we shall introduce
two different algebraic calculi.  The first is the algebraic
approach to predicate logic using so called {\it cylindric
algebras}, the other an equational theory of $\lambda$--calculus,
which embraces the (marginally popular) variable free approach
to semantics for first order logic.

\index{logic!first--order}%%
\index{first--order logic}%%
\index{FOL}%%
We have already introduced the syntax and semantics of first--order
predicate logic. Now we are going to present an axiomatization. To
this end we expand the set of axioms for propositional logic by
the axioms (a13) -- (a21) in Table~\ref{tab:predlog}. 
%%
\begin{table}
\caption{The Axioms for Predicate Logic (FOL)}
\label{tab:predlog}
\mbox{\sc Axioms.} \\[2mm]
$\begin{array}{ll}
\multicolumn{2}{l}{\mbox{\rm (a0) --- (a12)} +} \\
\mbox{\rm (a13)} & (\forall x)(\varphi\pf\chi)
    \pf ((\forall x)\varphi \pf (\forall x)\chi) \\
\mbox{\rm (a14)} & (\forall x)\varphi \pf [t/x]\varphi \\
\mbox{\rm (a15)} & \varphi \pf (\forall x)\varphi \qquad 
    (x\not\in \fr(\varphi)) \\
\mbox{\rm (a16)} & (\forall x)\varphi \pf
    \nicht (\exists x)\nicht \varphi \\
\mbox{\rm (a17)} & \nicht (\exists x)\nicht \varphi
    \pf (\forall x)\varphi \\
\mbox{\rm (a18)} & (\forall x)(x \doteq x) \\
\mbox{\rm (a19)} & (\forall x)(\forall y)(x \doteq y \pf y \doteq x) \\
\mbox{\rm (a20)} & (\forall x)(\forall y)(\forall z)(x \doteq y 
	\und y \doteq z \pf x \doteq z) \\
\mbox{\rm (a21)} & (\forall x_0)\dotsb (\forall x_{\Xi(R)-1})(\forall 
	y)((\gund_{i < \Xi(R)} x_i
\doteq y_i) \\
    & \qquad \pf  (R(x_0, \dotsc, x_{\Xi(R)-1})
    \dpf [y/x_i]R(x_0, \dotsc, x_{\Xi(R)-1})))
\end{array}$
\\[2mm]
\mbox{\sc Rules.}\\[2mm]
\mbox{\rm (mp)}~~$\begin{array}{c}
    \varphi \quad \varphi\pf\chi \\\hline
    \chi
    \end{array}
    \qquad
  \mbox{\rm (gen)}~~\begin{array}{c}
    \varphi \\\hline
    (\forall x)\varphi
    \end{array}$
\end{table}
%%
\index{$\mathsf{FOL}$, $\vdash^{\mathsf{FOL}}$}%%%
%%%
The calculus (a0) -- (a21) with the rules (mp) and (gen) is called 
%%%
\index{first--order logic}%%
%%%%
$\mathsf{FOL}$. A \textbf{first--order theory} is a set of formulae 
containing (a0) -- (a21) and which is closed under (mp). 
We write $\Delta \vdash^{\mathsf{FOL}} \varphi$ if every theory 
containing $\Delta$ also contains $\varphi$.
In virtue of (a16) and (a17) we get that 
$(\forall x)\varphi \dpf \nicht(\exists x)\nicht\varphi$ as well as 
$(\exists x)\varphi \dpf \nicht(\forall x)\nicht\varphi$ which means 
that one of the quantifiers can be defined from the other. In (a21), 
we assume $i < \Xi(R)$. 

We shall prove its completeness using a more powerful result due to Leon 
Henkin that simple type theory ($\mathsf{STyp}$) is complete with respect 
to Henkin--frames. Notice that the status of (gen) is the same as that 
of (mn) in modal logic. (gen) is admissible with respect to the model 
theoretic consequence $\vDash$ defined in Section~\ref{kap3}.\ref{kap3-6}, 
but it is
not derivable in it. To see the first, suppose that $\varphi$ is a
theorem and let $x$ be a variable. Then $(\forall x)\varphi$ is a
theorem, too. However, $\varphi \vDash (\forall x)\varphi$ does
not follow. Simply take a unary predicate letter $P$ and a
structure consisting of two elements, 0, 1, such that $P$ is true
of 0 but not of 1. Then with $\beta(x) := 0$ we have $\auf \GM,
\beta\zu \vDash P(x)$ but $\auf \GM, \beta\zu \nvDash (\forall
x)P(x)$. Now let $\uli{P}$ be the set of all formulae that can be
obtained from (a0) -- (a21) by applying (gen). Then the following
holds.
%%
\begin{thm}
\label{thm:genelim}
$\vdash^{\mathsf{FOL}} \varphi$ iff $\varphi$ is derivable from 
$\uli{P}$ using only (mp).
\end{thm}
%%
\proofbeg
Let $\Pi$ be a proof of $\chi$ from $\uli{P}$ using only (mp). 
Transform the proof in the following way. First, prefix every 
occurring formula by $(\forall x)$. Further, for every $k$ such 
that $\varphi_k$ follows from $\varphi_i$ 
and $\varphi_i \pf \varphi_j$ for some $i,j < k$, insert in front of 
the formula $(\forall x)\varphi_j$ the sequence
%%%
\begin{equation}
(\forall x)(\varphi_i \pf \varphi_j) \pf ((\forall x)\varphi_i \pf 
	(\forall x)\varphi_j), 
(\forall x)\varphi_i \pf (\forall x)\varphi_j 
\end{equation}
%%%
The new proof is a proof of $(\forall x)\chi$ from $\uli{P}$, 
since the latter is closed under (gen). Hence, the set of formulae 
derivable from $\uli{P}$ using (mp) is closed under (gen). Therefore 
it contains all tautologies of $\mathsf{FOL}$.
\proofend

The next theorem asserts
that this axiomatization is complete.
%%%
\begin{defn}
Let $\Delta$ be a set of formulae of predicate logic over a
signature, $\varphi$ a formula over that same signature. Then
$\Delta \vdash \varphi$ iff $\varphi$ can be proved from 
$\Delta \cup \uli{P}$ using only (mp).
\end{defn}
%%%
\begin{thm}[G\"odel]
%%%
\index{G\"odel, Kurt}%%%
%%%
\label{thm:ent}
$\Delta \vdash^{\mathsf{FOL}} \varphi$ iff $\Delta \vDash \varphi$.
\end{thm}
%%%
Recall from Section~\ref{kap3}.\ref{kap3-6} the definition of the simple
theory of types. 
%%%
\index{simple type theory (STT)}%%%
%%%
There we have also defined the class of models,
the so called {\it Henkin--frames}. Recall further that this
theory has operators $\Pi^{\alpha}$, which allow to define the
universal quantifiers in the following way.
%%
\begin{equation}
(\forall x_{\alpha})N_t := (\Pi^{\alpha} (\lambda
x_{\alpha}.N_t))
\end{equation}
%%
The simple theory of types is axiomatized as follows. We define a
calculus exclusively on the terms of type $t$ (truth values).
However, it will also be possible to express that two terms are
equal. This is done as follows. Two terms $M_{\alpha}$ and
$N_{\alpha}$ of type $\alpha$ are equal if for every term
$O_{\alpha\pf t}$ the terms $O_{\alpha\pf t}M_{\alpha}$ and
$O_{\alpha\pf t}N_{\alpha}$ are equivalent.
%%
\newcommand{\neweq}{\triangleq}
%%
\begin{equation}
M_{\alpha} \neweq N_{\alpha}  :=
    (\forall z_{\alpha\pf t})(z_{\alpha\pf t}M_{\alpha}
    \dpf z_{\alpha\pf t}N_{\alpha})
\end{equation}
%%
For this definition we assume that $z_{\alpha\pf t}$ is free
neither in $M_{\alpha}$ nor in $N_{\alpha}$. If one dislikes 
the side conditions, one can prevent the accidental capture of 
$z_{\alpha\pf t}$ using the following more refined version:
%%
\begin{equation}
%%%
\index{$M_{\alpha} \neweq N_{\alpha}$}%%%
%%%
M_{\alpha} \neweq N_{\alpha}  :=
    (\lambda x_{\alpha}.\lambda y_{\alpha}.(\forall %
    z_{\alpha\pf t})(z_{\alpha\pf t}x_{\alpha} \dpf %
    z_{\alpha\pf t}y_{\alpha}))M_{\alpha}N_{\alpha}
\end{equation}
%%
However, if $z_{\alpha\pf t}$ is properly chosen, no problem will
ever arise. Now, let (s0) -- (s12) be the formulae (a0) -- (a12)
appropriately translated into the language of types.
%%
\begin{table}
\caption{The Simple Theory of Types}
\index{$\mathsf{STyp}$}%%%
\label{tab:styp}
\mbox{\sc Axioms.} \\[2mm]
\begin{tabular}{ll}
\multicolumn{2}{l}{(s0) --- (s12) +} \\
(s13) & $((\forall x_{\alpha})(y_t \pf M_{\alpha\pf t}x_{\alpha}))
    \pf (y_t \pf \Pi^{\alpha}M_{\alpha\pf t})$ \\
(s14) & $(\Pi^{\alpha} x_{\alpha \pf t}) \pf x_{\alpha\pf t}y_{\alpha}$ \\
(s15) & $(x_{t} \dpf y_{t}) \pf x_t \neweq y_t$ \\
(s16) & $((\forall z_{\alpha})(x_{\alpha\pf\beta}z_{\alpha}
    \neweq y_{\alpha\pf\beta}z_{\alpha})) \pf
    (x_{\alpha\pf\beta} \neweq y_{\alpha\pf\beta})$ \\
(s17) & $x_{\alpha\pf t}y_{\alpha} \pf x_{\alpha\pf t}
    (\iota^{\alpha}x_{\alpha \pf t})$
\end{tabular}
\\[2mm]
\mbox{\sc Rules.} \\
$\begin{array}{l@{\;}l@{\qquad}l@{\;}l}
\mbox{\rm (mp)} & \begin{array}{c}
              M_t \pf N_t \quad M_t \\\hline
          N_t
          \end{array}
        &
\mbox{\rm (ug)} & \begin{array}{c}
           M_{\alpha \pf t}x_{\alpha} \\\hline
           \Pi^{\alpha}M_{\alpha}
           \end{array}
           \quad x_{\alpha} \not\in \fr(M_{\alpha\pf t})
         \\
\mbox{\rm (conv)} & \begin{array}{c}
           M_t \quad M_t \equiv_{\alpha\beta} N_t \\\hline
           N_t
           \end{array}
          &
\mbox{\rm (sub)} & \begin{array}{c}
           M_{\alpha \pf t}x_{\alpha} \\\hline
           M_{\alpha\pf t}N_{\alpha}
           \end{array}
           \quad
           x_{\alpha} \not\in \fr(M_{\alpha\pf t})
\end{array}$
%%
\end{table}
%%
We call the Hilbert--style calculus consisting of (s0) -- (s17) and 
the rules given in Table~\ref{tab:styp} $\mathsf{STyp}$. 
%%%
\index{$\mathsf{STyp}$}%%%
%%%%
All instances of theorems of $\mathsf{PC}$ 
are theorems of $\mathsf{STyp}$. For predicate logic this will also
be true, but the proof of that requires work. The rule (gen) is a 
derived rule of this calculus. To see this, assume that $M_{\alpha \pf
t}z_{\alpha}$ is a theorem. Then, by (conv), $(\lambda
z_{\alpha}.M_{\alpha\pf t}z_{\alpha})z_{\alpha}$ also is a
theorem. Using (ug) we get $\Pi^{\alpha}(\lambda
z_{\alpha}.M_{\alpha\pf t}z_{\alpha})$, which by abbreviatory
convention is $(\forall z_{\alpha})M_{\alpha\pf t}$. We will also
show that (a14) and (a15) are theorems of $\mathsf{STyp}$.
%%
\begin{lem}
\label{lem:a14}%%
$\mathsf{STyp} \vdash (\forall x_{\alpha})y_t \pf
[N_{\alpha}/x_{\alpha}]y_t$.
\end{lem}
%%%
\proofbeg %
By convention, $(\forall x_{\alpha})y_t = \Pi^{\alpha}(\lambda
x_{\alpha}.y_t)$. Moreover, by (s14), $\mathsf{STyp} \vdash ((\forall
x_{\alpha})y_t) \pf (\lambda x_{\alpha}.y_t)x_{\alpha} = 
((\forall x_{\alpha})y_t) \pf y_t$. 
Using (sub) we get
%%
\begin{equation}
\vdash^{\mathsf{STyp}}
[N_{\alpha}/x_{\alpha}]((\forall x_{\alpha})y_t \pf y_t) =
(\forall x_{\alpha})y_t \pf [N_{\alpha}/x_{\alpha}]y_t
\end{equation}
%%
as required. \proofend
%%%
\begin{lem}
Assume that $x_{\alpha}$ is not free in $N_t$. Then 
%%
\begin{equation}
\mathsf{STyp} \vdash N_t \pf (\forall x_{\alpha})N_t\
\end{equation}
\end{lem}
%%
\proofbeg %
With $N_t \pf N_t \equiv_{\alpha\beta} N_t \pf (\lambda
x_{\alpha}.N_t)x_{\alpha}$ and the fact that $(\forall
x_{\alpha})(N_t \pf N_t)$ is derivable (using (gen)), we get with
(conv) $(\forall x_{\alpha})(N_t \pf ((\lambda%
x_{\alpha}.N_t)x_{\alpha}))$ and with (s13) and (mp) we get 
%%%
\begin{equation}
\vdash^{\mathsf{STyp}} N_t \pf \Pi^{\alpha}(\lambda x_{\alpha}.N_t) = 
N_t \pf (\forall x_{\alpha})N_t
\end{equation}
%%%
(The fact that $x_{\alpha}$ is 
not free in $N_t$ is required when using (s13). In order for the 
replacement of $N_t$ for $y_t$ in the scope of $(\forall x_{\alpha})$ 
to yield exactly $N_t$ again, we need that $x_{\alpha}$ is not free 
in $N_t$.) %%
\proofend
%%%
\begin{lem}
\label{lem:lambdaeta} 
If $\mbox{\sf\textgreek{lh}} \vdash
M_{\alpha} \boldsymbol{\doteq} N_{\alpha}$ then $\mathsf{STyp}\vdash
M_{\alpha} \neweq N_{\alpha}$.
\end{lem}
%%
\proofbeg 
{\sf\textgreek{lh}} = {\sf\textgreek{l}} + (ext), and
(ext) is the axiom (s16). Hence it remains to show that 
$\mbox{\sf\textgreek{l}} \vdash M_{\alpha} \boldsymbol{\doteq} 
N_{\alpha}$ implies $\vdash^{\mathsf{STyp}} M_{\alpha} 
\neweq N_{\alpha}$. So, assume the first. Then we have $M_{\alpha}
\equiv_{\alpha\beta} N_{\alpha}$. Hence
%%%
\begin{equation}
z_{\alpha\pf t}M_{\alpha} \pf z_{\alpha\pf t}M_{\alpha}
\equiv_{\alpha\beta} z_{\alpha\pf t}M_{\alpha} \pf z_{\alpha\pf
t}N_{\alpha}
\end{equation}
%%%
Hence, using (conv), and $\vdash^{\mathsf{STyp}} 
z_{\alpha \pf t}M_{\alpha} \pf z_{\alpha\pf t}M_{\alpha}$
we get 
%%
\begin{equation}
\vdash^{\mathsf{STyp}} z_{\alpha\pf t}M_{\alpha}
\pf z_{\alpha\pf t}N_{\alpha}
\end{equation}
%%
By symmetry, $\vdash^{\mathsf{STyp}} z_{\alpha\pf t}M_{\alpha}
\dpf z_{\alpha\pf t}N_{\alpha}$.
%%
Using (gen)  we get
%%%
\begin{equation}
\vdash^{\mathsf{STyp}} (\forall z_{\alpha \pf t})%
(z_{\alpha\pf t}M_{\alpha} \dpf z_{\alpha\pf t}N_{\alpha})
\end{equation}
%%%
By abbreviatory convention, $\vdash^{\mathsf{STyp}} 
M_{\alpha} \neweq N_{\alpha}$. 
\proofend

We shall now show that $\mathsf{STyp}$ is complete with respect to
Henkin--frames where $\neweq$ simply is interpreted as identity.
To do that, we first prove that $\neweq$ is a congruence relation.
%%%
\begin{lem}
\label{lem:neweq}%%
The following formulae are provable in $\mathsf{STyp}$.
%%
\begin{subequations}
\begin{align}
\label{eq:eq1}
& x_{\alpha} \neweq x_{\alpha} \\
\label{eq:eq2}
& x_{\alpha} \neweq y_{\alpha} \pf
    y_{\alpha} \neweq x_{\alpha} \\
\label{eq:eq3}
& x_{\alpha} \neweq y_{\alpha} \und y_{\alpha}
\neweq z_{\alpha}
    \pf x_{\alpha} \neweq z_{\alpha} \\
\label{eq:eq4}
& x_{\alpha\pf\beta} \neweq x'_{\alpha\pf\beta}
\und
    y_{\alpha} \neweq y'_{\alpha} .\pf.
    x_{\alpha\pf\beta}y_{\alpha} \neweq
    x'_{\alpha\pf\beta}y'_{\alpha}
\end{align}
\end{subequations}
\end{lem}
%%
\proofbeg%%
\eqref{eq:eq1} Let $z_{\alpha\pf t}$ be a variable of type $\alpha\pf t$.
Then $z_{\alpha\pf t}x_{\alpha} \dpf z_{\alpha\pf t}x_{\alpha}$ is
provable in $\mathsf{STyp}$ (as $p \dpf p$ is in $\mathsf{PC}$). Hence, 
$(\forall z_{\alpha\pf t})(z_{\alpha\pf t}x_{\alpha} \dpf z_{\alpha\pf
t}x_{\alpha})$ is provable, which is $x_{\alpha} \neweq x_{\alpha}$. 
\eqref{eq:eq2} and \eqref{eq:eq3} are shown using 
predicate logic. \eqref{eq:eq4} Assume $x_{\alpha\pf\beta} \neweq
x'_{\alpha\pf\beta}$  and $y_{\alpha} \neweq y'_{\alpha}$. 
Now,
%%%%
\begin{equation}
z_{\beta\pf t}(x_{\alpha\pf\beta}y_{\alpha}) \equiv_{\alpha\beta}
(\lambda u_{\alpha}.z_{\beta\pf t}(x_{\alpha\pf\beta}u_{\alpha}))
(y_{\alpha})
\end{equation}
%%%
Put $M_{\alpha\pf t} := (\lambda u_{\alpha}.z_{\beta\pf t}(%
x_{\alpha\pf\beta}u_{\alpha}))$. Using the rule (conv), we get
%%
\begin{equation}
\begin{split}
y_{\alpha} \neweq y'_{\alpha} & \vdash^{\mathsf{STyp}} M_{\alpha}y_{\alpha} \dpf
    M_{\alpha}y'_{\alpha} \\
    & \vdash^{\mathsf{STyp}} z_{\beta\pf t}(x_{\alpha\pf\beta}y_{\alpha}) \dpf
    z_{\beta\pf t}(x_{\alpha\pf\beta}y'_{\alpha})
\end{split}
\end{equation}
%%
Likewise, one can show that
%%
\begin{equation}
y_{\alpha} \neweq y'_{\alpha} \vdash^{\mathsf{STyp}}
    z_{\beta\pf t}(x'_{\alpha\pf\beta}y_{\alpha}) \dpf
    z_{\beta\pf t}(x'_{\alpha\pf\beta}y'_{\alpha})
\end{equation}
%%
Similarly, using $N_{(\alpha\pf\beta) \pf t} := (\lambda
u_{\alpha\pf\beta}.z_{\beta\pf t}(u_{\alpha\pf \beta}
    y_{\alpha}))$ one shows that
%%
\begin{equation}
x_{\alpha\pf\beta} \neweq x'_{\alpha\pf\beta} \vdash^{\mathsf{STyp}}
    z_{\beta\pf t}(x_{\alpha\pf\beta}y_{\alpha}) \dpf
    z_{\beta\pf t}(x'_{\alpha\pf\beta}y_{\alpha}) 
\end{equation}
%%
This allows to derive the desired conclusion. %%
\proofend

Now we get to the construction of the frame. Let $C_{\alpha}$ be
the set of closed formulae of type $\alpha$. Choose a maximally
consistent $\Delta \supseteq C_t$. Then, for each type
$\alpha$, define $\approx^{\alpha}_{\Delta}$ by $M_{\alpha}
\approx^{\alpha}_{\Delta} N_{\alpha}$ iff $M_{\alpha}
\neweq N_{\alpha} \in \Delta$. By Lemma~\ref{lem:neweq} this
is an equivalence relation for each $\alpha$, and, moreover, if
$M_{\alpha\pf\beta} \approx^{\alpha\pf\beta}_{\Delta}
M'_{\alpha\pf\beta}$ and $N_{\alpha} \approx^{\alpha}_{\Delta}
N'_{\alpha}$, then also $M_{\alpha\pf\beta}N_{\alpha}
\approx^{\beta}_{\Delta} M'_{\alpha\pf\beta}N'_{\alpha}$. For
$M_{\alpha} \in C_{\alpha}$ put
%%
\begin{equation}
[M_{\alpha}] := \{N_{\alpha} : M_{\alpha} \approx^{\alpha}_{\Delta}
N_{\alpha}\}
\end{equation}
%%
Finally, put $D_{\alpha} := \{[M_{\alpha}] : M_{\alpha} \in
C_{\alpha}\}$. Next, $\bullet$ is defined as usual, $- :=
[\mbox{\mtt\symbol{5}}]$, $\cap := [\mbox{\mtt\symbol{4}}]$, 
$\pi^{\alpha} := [\GPi^{\alpha}]$ and 
$\iota^{\alpha} := [\Giota^{\alpha}]$. This 
defines a structure (where $\CS := \Typ_{\pf}(B)$).
%%
\begin{equation}
\goth{Hfr}_{\Delta} := \auf \{D_{\alpha}:
\alpha \in \CS\}, \bullet, -, \cap, \auf
\pi^{\alpha} : \alpha \in \CS\}\zu, \auf
\iota^{\alpha} : \alpha \in \CS\}\zu\zu
\end{equation}
%%
\begin{lem}[Witnessing Lemma]
\begin{equation*}
\mathsf{STyp} \vdash^{\mathsf{STyp}} M_{\alpha\pf t}(\iota^{\alpha}(\lambda
x_{\alpha}.\nicht M_{\alpha})) \pf \Pi^{\alpha}M_{\alpha}
\end{equation*}
\end{lem}
%%%
\proofbeg
Write $\nicht N_{\alpha\pf t} := \lambda x_{\alpha}.\nicht %
(N_{\alpha\pf t}x_{\alpha})$. Now, by (s17)
%%
\begin{equation}
\vdash^{\mathsf{STyp}} (\nicht N_{\alpha\pf
t})y_{\alpha} \pf (\nicht N_{\alpha\pf t})(\iota^{\alpha}(\nicht %
N_{\alpha \pf t}))
\end{equation}
%%
Using classical logic we obtain
%%
\begin{equation}
\vdash^{\mathsf{STyp}} \nicht ((\nicht N_{\alpha\pf t})(\iota^{\alpha} %
(\nicht N_{\alpha\pf t}))) \pf \nicht ((\nicht N_{\alpha\pf
t})y_{\alpha})
\end{equation}
%%
Now, $\nicht ((\nicht N_{\alpha\pf t})y_t) \triangleright_{\beta}
\nicht \nicht (N_{\alpha\pf t}y_{\alpha})$, the latter being
equivalent to $N_{\alpha\pf t}y_{\alpha}$. Similarly,
$\nicht ((\nicht N_{\alpha\pf t})(\iota^{\alpha}(\nicht N_{\alpha\pf t})))$
is equivalent with $N_{\alpha\pf t}(\iota^{\alpha}(\nicht
N_{\alpha\pf t}))$. Hence
%%
\begin{equation}
\vdash^{\mathsf{STyp}} N_{\alpha\pf t}(\iota^{\alpha}(\nicht
N_{\alpha\pf t})) \pf N_{\alpha\pf t}y_{\alpha} 
\end{equation}
%%
Using (gen), (s13) and (mp) we get
%%
\begin{equation}
\vdash^{\mathsf{STyp}} (N_{\alpha\pf t}(\iota^{\alpha}(\nicht
N_{\alpha\pf t}))) \pf \Pi^{\alpha}N_{\alpha\pf t}
\end{equation}
%%
This we had to show.
\proofend
%%
\begin{lem}
$\goth{Hrf}_{\Delta}$ is a Henkin--frame.
\end{lem}
%%%
\proofbeg%%
By Lemma~\ref{lem:lambdaeta}, if $\mbox{\sf\textgreek{lh}} \vdash
M_{\alpha} \neweq N_{\alpha}$, then $[M_{\alpha}] = [N_{\alpha}]$.
So, the axioms of the theory {\sf\textgreek{lh}} are valid, and
$\auf \{D_{\alpha} : \alpha \in S\}, \bullet\zu$ is a functionally
complete (typed) applicative structure. Since $\Delta$ is
maximally consistent, $D_t$ consists of two elements, which we now
call $0$ and $1$. Furthermore, we may arrange it that $[M_t] = 1$
iff $M_t \in \Delta$. It is then easily checked that
the interpretation of {\mtt\symbol{5}} is complement, and the 
interpretation of {\mtt\symbol{4}} is intersection. Now we treat 
$\pi^{\alpha} := [\GPi^{\alpha}]$. We have to
show that for $a \in D_{\alpha\pf t}$
$\pi^{\alpha} \bullet a = 1$ iff
for every $b \in D_{\alpha}$: $a \bullet b = 1$. Or,
alternatively, $\Pi^{\alpha}M_{\alpha\pf t} \in \Delta$ iff 
$M_{\alpha\pf t}N_{\alpha} \in \Delta$ for every closed
term $N_{\alpha}$. Suppose that $\Pi^{\alpha}M_{\alpha\pf t} \in
\Delta$. Using Lemma~\ref{lem:a14} and the fact that $\Delta$ is
deductively closed, $M_{\alpha\pf t}N_{\alpha} \in \Delta$.
Conversely, assume $M_{\alpha\pf t}N_{\alpha} \in \Delta$ for
every constant term $N_{\alpha}$. Then
$M_{\alpha}(\iota^{\alpha}(\lambda x_{\alpha}.\nicht M_{\alpha\pf
t}x_{\alpha}))$ is a constant term, and it is in $\Delta$.
Moreover, by the Witnessing Lemma, $\Pi^{\alpha}M_{\alpha\pf t}
\in \Delta$. Finally, we have to show that for every $a \in
D_{\alpha\pf t}$: if there is a $b \in D_{\alpha}$ such that $a
\bullet b = 1$ then $a \bullet (\iota^{\alpha} \bullet a) = 1$. 
This means that for $M_{\alpha\pf t}$: if there
is a constant term $N_{\alpha}$ such that $M_{\alpha\pf
t}N_{\alpha} \in \Delta$ then $M_{\alpha\pf
t}(\iota^{\alpha}M_{\alpha\pf t}) \in \Delta$.
This is a consequence of (s17). %%
\newline\mbox{}
\proofend

Now, it follows that $\goth{Hfr}_{\Delta} \vDash N_t$ iff
$N_t \in \Delta$. More generally, let $\beta$ an assignment of
constant terms to variables. Let $M_{\alpha}$ be a term. Write
$M^{\beta}_{\alpha}$ for the result of replacing a free occurrence
of a variable $x_{\gamma}$ by $\beta(x_{\gamma})$. Then
%%%
\begin{equation}
\auf \goth{Hfr}_{\Delta}, \beta\zu \vDash M_{\alpha}
\quad\Dpf\quad M^{\beta}_{\alpha} \in \Delta
\end{equation}
%%
This is shown by induction.
%%
\begin{lem}
\label{lem:wohl}
Let $\Delta_0$ be a consistent set of constant terms. Then there
exists a maximally consistent set $\Delta$ of constant terms
containing $\Delta_0$.
\end{lem}
%%%
\proofbeg%%
Choose a well--ordering $\{N^{\delta} : \delta < \mu\}$ on the set
of constant terms. Define $\Delta_i$ by induction as follows.
$\Delta_{\kappa+1} := \Delta_{\kappa} \cup \{N^{\delta}\}$ if the
latter is consistent. Otherwise, $\Delta_{\kappa+1} :=
\Delta_{\kappa}$. If $\kappa < \mu$ is a limit ordinal,
$\Delta_{\kappa} := \bigcup_{\lambda < \kappa} \Delta_{\lambda}$.
We shall show that $\Delta := \Delta_{\mu}$ is maximally
consistent. Since it contains $\Delta_0$, this will complete the
proof. (a) It is consistent. This is shown inductively. By
assumption $\Delta_0$ is consistent, and if $\Delta_{\kappa}$ is
consistent, then so is $\Delta_{\kappa+1}$. Finally, let $\lambda$
be a limit ordinal and suppose that $\Delta_{\lambda}$ is
inconsistent. Then there is a finite subset $\Gamma$ which is
inconsistent. There exists an ordinal $\kappa < \lambda$ such that
$\Gamma \subseteq \Delta_{\kappa}$. $\Delta_{\kappa}$ is
consistent, contradiction. (b) There is no consistent superset.
Assume that there is a term $M \not\in \Delta$ such that $\Delta
\cup \{M\}$ is consistent. Then for some $\delta$, $M =
N^{\delta}$. Then $\Delta_{\delta} \cup \{N^{\delta}\}$ is
consistent, whence by definition
$N^{\delta} \in \Delta_{\delta+1}$. Contradiction. %%
\proofend
%%%
\begin{thm}[Henkin]
%%%
\index{Henkin, Leon}%%
%%%
(a) A term $N_t$ is a theorem of $\mathsf{STyp}$ iff it is
valid in all Henkin--frames. (b) An equation $M_{\alpha} \neweq
N_{\alpha}$ is a theorem of $\mathsf{STyp}$ iff it holds in
all Henkin--frames iff it is valid in the many sorted 
sense.
\end{thm}
%%%
We sketch how to prove Theorem~\ref{thm:ent}. Let $\auf \Omega, \Xi\zu$ 
be a signature for predicate logic. Define a translation into 
$\mathsf{STyp}$:
%%%
\begin{subequations}
\begin{align}
\label{eq:fdef}
f^{\heartsuit} & := (\lambda x_0)(\lambda x_1)\dotso
	(\lambda x_{\Omega(f)-1})f(x_0, \dotsc, x_{\Omega(f)-1}) \\
\label{eq:rdef}
r^{\heartsuit} & := (\lambda x_0)(\lambda x_1)\dotso
	(\lambda x_{\Xi(r)-1})r(x_0, \dotsc, x_{\Xi(r)-1}) 
\end{align}
\end{subequations}
%%%
This is extended to all formulae. Now we look at the signature for 
$\mathsf{STyp}$ with the constants $f^{\heartsuit}$, $r^{\heartsuit}$ 
of type 
%%
\begin{subequations}
\begin{align}
f^{\heartsuit} : & \ (e \pf (e \pf \dotso (e \pf e) \dotso )) \\
r^{\heartsuit} : & \ (e \pf (e \pf \dotso (e \pf t) \dotso )) 
\end{align}
\end{subequations}
%%
Now, given a first--order model $\GM$, we can construct a Henkin--frame 
for $\GM^{\heartsuit}$ with $D_e = M$ and $D_{\alpha\pf\beta} 
:= D_{\alpha} \pf D_{\beta}$, by interpreting $f^{\heartsuit}$ 
and $r^{\heartsuit}$ as given by \eqref{eq:fdef} and \eqref{eq:rdef}.
%%
\begin{lem}
\label{lem:predtostyp}
Let $\beta$ be a valuation on $\GM$. Extend $\beta$ to 
$\beta^{\heartsuit}$. Then
%%%
\begin{equation}
\auf \GM, \beta\zu \vDash \varphi \quad\Dpf\quad
\auf \GM^{\heartsuit}, \beta\zu \vDash \varphi^{\heartsuit}
\end{equation}
\end{lem}
%%
\begin{lem}
\label{lem:pred}
$\vdash^{\mathsf{STyp}} \varphi^{\heartsuit}$ iff 
$\vdash^{\mathsf{FOL}} \varphi$.
\end{lem}
%%%
Right to left is by induction. Now if $\nvdash^{\mathsf{FOL}}
\varphi$ then there is a model $\auf \GM, \beta\zu \vDash 
\nicht \varphi$, from which we get a Henkin--frame 
$\auf \GM^{\heartsuit}, \beta\zu \nvDash \nicht \varphi^{\heartsuit}$.
The proof of Theorem~\ref{thm:ent} is as follows. Clearly, if
$\varphi$ is derivable it is valid. Suppose that it is not 
derivable. Then $\varphi^{\heartsuit}$ is not derivable   
in $\mathsf{STyp}$. There is a Henkin--frame $\GM \nvDash 
\varphi^{\heartsuit}$. This allows to define a first--order model 
$\GM_{\heartsuit} \nvDash \varphi$.
%%
\vplatz
\exercise
The set of typed $\lambda$--terms is defined over a finite alphabet 
if the set $B$ of basic types is finite. Define from this a 
well--ordering on the set of terms. {\it Remark.} This shows 
that the proof of Lemma~\ref{lem:wohl} does not require the 
use of the Axiom of Choice for obtaining the well--ordering.
%%%
\vplatz
\exercise
Show Lemma~\ref{lem:predtostyp}.
%%%
\vplatz
\exercise
Complete the details of the proof of Theorem~\ref{thm:ent}. 
%%%
\vplatz
\exercise
Let $L$ be a normal modal logic. Show that $\Delta \Vdash_L \chi$ 
iff $\Delta^b \vdash_L \chi$, where $\Delta^b := \{\qu^n \delta : 
\delta \in \Delta, n \in \omega\}$. {\it Hint.} This is analogous 
to Theorem~\ref{thm:genelim}.
