\section{Algebraization}
\label{kap6-4b}
%%%
Now that we have shown completeness with respect to models and
frames, we shall proceed to investigate the possibility of
algebraization of predicate logic and simple type theory. Apart
from methodological reasons, there are also practical reasons for
preferring algebraic models over frames. If $\varphi$ is a
sentence and $\GM$ a model, then either $\GM \vDash \varphi$ or
$\GM \vDash \nicht\varphi$. Hence, the theory of a single model is
maximally consistent, that is, complete. One may argue that this
is as it should be; but notice that the base logic ($\mathsf{FOL}$,
$\mathsf{STyp}$) is not complete --- neither is the knowledge
ordinary people have. Since models are not enough for representing
incomplete theories, something else must step in their place.
These are {\it algebras\/} for some appropriate signature, for the
product of algebras is an algebra again, and the logic of the
product is the intersection of the logics of the factors. Hence,
for every logic there is an adequate algebra. However,
algebraization is not straightforward. The problem is that there
is no notion of binding in algebraic logic. Substitution always is
replacement of an occurrence of a variable by the named string,
there is never a preparatory replacement of variables being
performed. Hence, what creates in fact big problems is those
axioms and rules that employ the notion of a free or bound
variable. In predicate logic this is the axiom (a15). (Unlike 
$\mathsf{PC}$, $\mathsf{FOL}$ has no rule of substitution.)

It was once again Tarski who first noticed the analogy between
modal operators and quantifiers. Consider a language $L$ of first
order logic without quantifiers. We may interpret the atomic
formulae of this language as propositional atoms, and formulae
made from them using the boolean connectives. Then we have a
somewhat more articulate version of our propositional boolean
language. We can now introduce a quantifier $Q$ simply as a unary
operator. For example, {\mtt (\symbol{20}x0)} is a unary operator
on formulae. Given a formula $\varphi$, {\mtt (\symbol{20}x0)$\varphi$}
is again a formula. (Notice that the way we write the formulae is 
somewhat different, but this can easily be accounted for.) In this 
way we get an extended language: a language of formulae extended by 
a single quantifier. Moreover, the laws of {\mtt (\symbol{20}x0)} turn 
the logic exactly into a normal modal logic. The quantifier 
{\mtt (\symbol{21}x0)} then corresponds to $\wD$, the dual of $\qu$. 
Clearly, in order to reach full expressive power of predicate logic 
we need to add infinitely many such operators, one for each variable. 
The resulting algebras are called \textbf{cylindric algebras}. The
principal reference is to \cite{henkinmonktarski:cylindric1}.

We start with the intended models of cylindric algebras. A formula
may be seen as a function from models, that is, pairs $\auf \GM,
\beta\zu$, to $2$, where $\GM$ is a structure and $\beta$ an
assignment of values to the variables.  First of all, we shall
remove the dependency on the structure, which allows us to focus
on the assignments. There is a general first order model for any
complete (= maximal consistent) theory, in which exactly those
sentences are valid that belong to the theory. Moreover, this model
is countable. Suppose a theory $T$ is not complete. Then let $\Delta_i$, 
$i \in I$, be its completions. For each $i \in I$, let $\GM_i$ be the 
canonical structure associated with $\Delta_i$. If $\GA_i$ is the
cylindrical algebra associated with $\GM_i$ (to be defined below),
the algebra associated with $T$ will $\prod_{i \in I} \GA_i$. In this 
way, we may reduce the study to that of a cylindric algebra of a single 
structure.

Take a first order structure $\auf M, \GI\zu$, where $M$ is the
universe and $\GI$ the interpretation function. For simplicity, we
assume that there are no functions. (The reader shall see in the
exercises that there is no loss of expressivity in renouncing
functions.) Let $V := \{x_i : i \in \omega\}$ be the set of
variables. Let $\GV(V;M)$ be the boolean algebra of sets of
functions into $M$. Then for every formula $\varphi$ we
associate the following set of assignments:
%%
\begin{equation}
[\varphi] := \{\beta : \auf \GM, \beta\zu \vDash \varphi\}
\end{equation}
%%
Now, for each number $i$ we assume the following operation $\mathsf{A}_i$.
%%
\begin{equation}
\mathsf{A}_i(S) := \{\beta : \mbox{ for all }\gamma
\sim_{x_i} \beta: \gamma \in S\}
\end{equation}
%%
Then $\mathsf{E}_i(S) := - \mathsf{A}_i(- S)$. (The standard
notation for $\mathsf{E}_i$ is $\mathsf{c}_i$. The letter
$\mathsf{c}$ here is suggestive for `cylindrification'. We have decided
to stay with a more logical notation.) Furthermore, for every pair of
numbers $i, j \in \omega$ we assume the element $\mathsf{d}_{i,j}$.
%%
\begin{equation}
\mathsf{d}_{i,j} := \{\beta :  \beta(x_i) = \beta(x_j)\}
\end{equation}
%%
It is interesting to note that with the help of these elements
substitution can be defined. Namely, put
%%
\begin{equation}
\mathsf{s}^{i}_{j}(x) := 
    \begin{cases}
    x & \text{ if $i = j$,} \\
    \mathsf{E}_i(\mathsf{d}_{i,j} \cap x)
    & \text{ otherwise.}
    \end{cases}
\end{equation}
%%
\begin{lem}
\label{lem:subst}
Let $y$ be a variable distinct from $x$. Then $[y/x]\varphi$
is equivalent with {\mtt (\symbol{21}$x$).(($y$\symbol{61}$%
x$)\symbol{4}$\varphi$)}.
\end{lem}
%%
Thus, equality and quantification alone
can define substitution. The relevance of this observation for
semantics has been nicely explained in
\cite{dresner:tarski}. For example, in applications
it becomes necessary to introduce constants for the relational
symbols. Suppose, namely that {\tt taller} is a binary relation
symbol. Its interpretation is a binary relation on the domain.
If we want to replace the structure by its associated cylindric
algebra, the relation is replaced by an element of that algebra,
namely
%%
\begin{equation}
[\textsf{taller}'(x_0,x_1)] :=
    \{\beta : \auf \beta(x_0), \beta(x_1)\zu \in
    \GI(\mbox{\tt taller})\}
\end{equation}
%%
However, this allows us prima facie only to assess the meaning of
`$x_0$ is taller than $x_1$'. We do not know, for example, what
happens to `$x_2$ is taller than $x_7$'. For that we need the
substitution functions. Now that we have the unary substitution
functions, any finitary substitution becomes definable. In this
particular case,
%%
\begin{equation}
[\textsf{taller}'(x_2,x_7)] = \mathsf{s}^7_1
    \mathsf{s}^2_0 [\textsf{taller}'(x_0,x_1)]\
\end{equation}
%%
Thus, given the definability of substitutions, to define the 
interpretation of $R$ we only need to give the element
$[R(x_0, x_1, \ldots, x_{\Xi(R)-1})]$.

The advantage in using this formulation of predicate logic is
that it can be axiomatized using equations. It is directly
verified that the equations listed in the next definition
are valid in the intended structures.
%%%
\begin{defn}
A \textbf{cylindric algebra} 
%%%
\index{cylindric algebra}%%
%%%
of dimension $\kappa$, $\kappa$ a cardinal number, is a structure
%%
\begin{equation}
\GA = \auf A, 0, 1, -, \cap, \cup, \auf \mathsf{E}_{\lambda}:
    \lambda < \kappa\zu, \auf \mathsf{d}_{\lambda, \mu} :
    \lambda, \mu < \kappa\zu\zu
\end{equation}
%%
such that the following holds for all $x, y \in A$ and $\lambda,
\mu, \nu < \kappa$:
\begin{equation}
\begin{tabular}{ll}
\mbox{\rm (ca1)} &
    $\auf A, 0, 1, -, \cap, \cup\zu$ is a boolean algebra. \\
\mbox{\rm (ca2)} &
    $\mathsf{E}_{\lambda} 0 = 0$. \\
\mbox{\rm (ca3)} &
    $x \cup \mathsf{E}_{\lambda} x = \mathsf{E}_{\lambda} x$.
    \\
\mbox{\rm (ca4)} &
    $\mathsf{E}_{\lambda}(x \cap \mathsf{E}_{\lambda} y) =
    \mathsf{E}_{\lambda} x \cap \mathsf{E}_{\lambda} y$. \\
\mbox{\rm (ca5)} &
    $\mathsf{E}_{\lambda}\mathsf{E}_{\mu} x =
    \mathsf{E}_{\mu}\mathsf{E}_{\lambda} x$. \\
\mbox{\rm (ca6)} &
    $\mathsf{d}_{\lambda,\mu} = 1$. \\
\mbox{\rm (ca7)} &
    If $\lambda \neq \mu, \nu$ then
        $\mathsf{d}_{\mu,\nu} =
    \mathsf{E}_{\lambda}(\mathsf{d}_{\mu,\lambda}
        \cap \mathsf{d}_{\lambda, \nu})$. \\
\mbox{\rm (ca8)} &
    If $\lambda \neq \mu$ then
    $\mathsf{E}_{\lambda}(\mathsf{d}_{\lambda,\mu} \cap x)
    \cap \mathsf{E}_{\lambda}(\mathsf{d}_{\lambda,\mu}
    \cap (-x)) = 0$.
\end{tabular}
\end{equation}
\end{defn}
%%
We shall see that this definition allows to capture the effect of
the axioms above, with the exception of (a15). Notice first the
following. $\dpf$ is a congruence in $\mathsf{FOL}$ as well. For
if $\varphi \dpf \chi$ is a tautology then so is $(\exists
x_i)\varphi \dpf (\exists x_i)\chi$. Hence, we can encode the
axioms of $\mathsf{FOL}$ as equations of the form $\varphi \dpf
\top$ as long as no side condition concerning free or bound
occurrences is present. We shall not go into the details. For
example, in $\varphi = (\forall x)\chi$ $x$ occurs trivially bound.
It remains to treat the rule (gen). It corresponds to the rule
(mn) of modal logic. In equational logic, it is implicit anyway.
For if $x \doteq y$ then $O(x) \doteq O(y)$ for any unary operator
$O$.
%%%
\begin{defn}
Let $\GA$ be a cylindric algebra of dimension $\kappa$, and
$a \in A$. Then
%%
\begin{equation}
\Delta a := \{i : i < \kappa, \mathsf{E}_i a \neq a\}
\end{equation}
%%
is called the \textbf{dimension of} $a$. 
%%%
\index{dimension}%%%
\index{$\Delta a$}%%%
%%%
$\GA$ is said to be \textbf{locally finite dimensional} 
%%%
\index{cylindric algebra!locally finite dimensional}%%%
%%%
if $|\Delta a| < \aleph_0$ for all $a \in A$.
\end{defn}
%%
A particular example of a cylindric algebra is
$\CL_{\kappa}/\!\equiv$, $\CL_{\kappa}$ the formulae of pure
equality based on the variables $\mbox{\tt x}_i$, $i < \kappa$,
and $\varphi \equiv \chi$ iff $\varphi \dpf \chi$ is a
theorem. (If additional function or relation symbols are needed,
they can be added with little change to the theory.) This algebra
is locally finite dimensional and is freely $\kappa$--generated.

The second approach we are going to elaborate is one which takes
substitutions as basic functions. For predicate logic this has
been proposed by Halmos~\shortcite{halmos:polyadic}, 
%%%
\index{Halmos, Paul}%%%
%%%
but most people credit Quine~\shortcite{quine:variables} 
%%%
\index{Quine, Willard van Orman}%%%
%%%
for this idea. For an exposition see \cite{pigozzisalibra:vb}.  
Basically, Halmos takes substitution as primitive. This has
certain advantages that will become apparent soon. Let us agree
that the index set is $\kappa$, again called the \textbf{dimension}.
%%%
\index{dimension}%%%
%%%
Halmos defines operations $\mathsf{S}(\tau)$ for every function
$\tau \colon \kappa \pf \kappa$ such that there are only finitely many
$i$ such that $\tau(i) \neq i$. The theory of such functions is
axiomatized independently of quantification. Now, for every finite
set $I \subset \kappa$ Halmos admits an operator $\mathsf{E}(I)$, 
which represents quantification over each of variables $\mbox{\tt x}_i$, 
where $i \in I$. If $I = \varnothing$, $\mathsf{E}(I)$ is the 
identity, otherwise $\mathsf{E}(I)(\mathsf{E}(K)x) = 
\mathsf{E}(I \cup K)x$. Thus, it is immediately clear
that the ordinary quantifiers $\mathsf{E}(\{i\})$ suffice to
generate all the others. However, the axiomatization is somewhat
easier with the polyadic quantifiers. Another problem, noted in
\cite{sainthompson:scheme}, is the fact that the axioms for polyadic 
algebras cannot be schematized using letters for elements of the 
index set. However, Sain and Thompson \shortcite{sainthompson:scheme} 
%%%
\index{Sain, Ildik\'o}%%%
\index{Thompson, Richard S.}%%%
%%%
also note that the addition of transpositions is actually enough to 
generate the same functions. To see this, here are some definitions.
%%%
\begin{defn}
%%%
\index{support}%%
\index{$\supp(\pi)$}%%
\index{transformation}%%
\index{permutation}%%
\index{transposition}%%
%%
Let $I$ be a set and $\pi \colon I \pf I$. The \textbf{support} of 
$\pi$, $\supp(\pi)$, is the set $\{i : \pi(i) \neq i\}$. A function 
of finite support is called a \textbf{transformation}.
$\pi$ is called a \textbf{permutation of} $I$ if it is bijective. If
the support contains exactly two elements, $\pi$ is called a 
\textbf{transposition}.
\end{defn}
%%%
The functions whose support has at most two elements are of
special interest. Notice first the case when $\supp(\pi)$
has exactly one element. In that case, $\pi$ is called an 
\textbf{elementary substitution}. Then there are $i, j \in I$ such that
$\pi(i) = j$ and $\pi(k) = k$ if $k \neq i$. If $i$ and $j$ are in
$I$, then denote by $(i,j)$ the permutation that sends $i$ to $j$
and $j$ to $i$. Denote by $[i,j]$ the elementary substitution that
sends $i$ to $j$.
%%%
\begin{prop}
\label{prop:support}
Let $I$ be a set. The set $\Phi(I)$ of functions $\pi \colon I \pf I$
of finite support is closed under concatenation. Moreover,
$\Phi(I)$ is generated by the elementary substitutions and the
transpositions.
\end{prop}
%%%
The proof of this theorem is left to the reader. So,  it is enough 
if we take only functions corresponding to $[i,j]$ and $(i,j)$. The 
functions of the first kind are already known: these are 
the $\mathsf{s}^j_i$. For the functions of the second kind, 
write $\mathsf{p}_{i,j}$. Sain and Thompson effectively axiomatize
cylindric algebras that have these additional operations. They
call them \textbf{finitary polyadic algebras}. 
%%%
\index{polyadic algebra!finitary}%%%
%%%
Notice also the following useful fact, which we also leave as an exercise.
%%%
\begin{prop}
\label{prop:approx} Let $\pi \colon I \pf I$ be an arbitrary function,
and $M \subseteq I$ finite. Then there is a product $\gamma$ of
elementary substitutions such that $\gamma \restriction M = \pi
\restriction M$.
\end{prop}
%%%
This theorem is both stronger and weaker than the previous one. It
is stronger because it does not assume $\pi$ to have finite
support. On the other hand, $\gamma$ only approximates $\pi$ on a
given finite set. (The reader may take notice of the fact that
there is no sequence of elementary substitutions that equals the
transformation $(0\; 1)$ on $\omega$. However, we can approximate
it on any finite subset.)

Rather than developing this in detail for predicate logic we shall
do it for the typed $\lambda$--calculus, as the latter is more rich 
and allows to encode arbitrarily complex abstraction (for example, by
way of using $\mathsf{STyp}$). Before we embark on the project let us 
outline the problems that we have to deal with. Evidently, we wish 
to provide an algebraic axiomatization that is equivalent to the 
rules \eqref{eq:lea} -- \eqref{eq:leg} and \eqref{eq:lei}. First, 
the signature we shall choose has function application and abstraction 
as its primitives. However, we cannot have a single abstraction 
symbol corresponding to $\tlambda$, rather, for each variable 
(and each type) we must assume a different unary function symbol 
$\tlambda_i$, corresponding to $\tlambda \mbox{\tt x}_i$. Now, 
\eqref{eq:lea} -- \eqref{eq:lee} and \eqref{eq:lei} are already 
built into the Birkhoff--Calculus. Hence, our only concern are the 
rules of conversion. These are, however, quite 
tricky. Notice first that the equations make use of the substitution 
operation $[N/x]$. This operation is in turn defined with the 
definitions \eqref{eq:suba} -- \eqref{eq:subf}. Already \eqref{eq:suba}
for $N = \mbox{\tt x}_i$ can only be written down if we have an operation 
that performs an elementary substitution. So, we have to add the unary 
functions $\mathsf{s}_{i,j}$, to denote this substitution. Additionally, 
\eqref{eq:suba} needs to be broken down into an inductive definition.
To make this work, we need to add correlates of the variables. 
That is, we add zeroary function symbols $\mbox{\tt x}_i$ for 
every $i \in \omega$. Symbols for the functions $\mathsf{p}_{i,j}$ permuting 
$i$ and $j$ will also be added to be able to say that the variables 
all range over the same set. Unfortunately, this is not all. Notice 
that \eqref{eq:lef} is not simply an equation: it has a side condition, 
namely that $y$ is not free in $M$. In order to turn this into an 
equation we must introduce {\it sorts}, which will help us keep track 
of the free variables. Every term will end up having a unique sort, 
which will be the set of $i$ such that $\mbox{\tt x}_i$ is free in 
it. $B$ is the set of basic types. Call a member of 
$\CI := \Typ_{\pf}(B) \times \omega$ an \textbf{index}. 
%%%
\index{index}%%
\index{type}%%%
%%%%
If $\auf \alpha, i\zu$ is an index, $\alpha$ is its \textbf{type} 
and $i$ its numeral. Let $\CF$ be the set of pairs 
$\auf \alpha, \delta\zu$ where $\alpha$ is a type and $\delta$ 
a finite set of indices.

We now start with the signature. Let $\delta$ and $\delta'$ be finite 
sets of indices, $\alpha$, $\beta$ types, and $\iota = \auf \gamma, i\zu$, 
$\kappa = \auf \gamma', i'\zu$ indices. We list the symbols together with 
their type:
%%%
\begin{subequations}
\begin{align}
\mbox{\tt x}_{\iota} & :  \auf\auf\gamma, \{\iota\}\zu\zu  \\
\tlambda^{\auf\beta,\delta\zu}_{\iota} & : 
\auf \auf \beta, \delta\zu, \auf \gamma\pf\beta, \delta - \{\iota\}\zu\zu \\
\mbox{\tt p}_{\iota,\kappa}^{\auf \alpha, \delta\zu} & :
\auf \auf \alpha, \delta\zu, \auf \alpha, (\iota, \kappa)[\delta]\zu\zu\qquad 
\text{where $\gamma = \gamma'$} \\
\mbox{\tt s}_{\iota, \kappa}^{\auf \alpha, \delta\zu} & :  
\auf \auf \alpha, \delta\zu, \auf \alpha, [\iota, \kappa][\delta]\zu\zu \\
\bullet^{\auf \alpha\pf\beta, \delta\zu, \auf \alpha, \delta'\zu} 
   & : \auf \auf \alpha\pf\beta, \delta\zu, 
\auf \alpha, \delta'\zu, \auf \beta, \delta \cup \delta'\zu\zu
\end{align}
\end{subequations}
%%%
Here $(\iota, \kappa)[\delta]$ is the result of exchanging $\iota$ and 
$\kappa$ in $\delta$ and  $[\iota, \kappa][\delta]$ is the result of 
replacing $\kappa$ by $\iota$ in $\delta$. 
Notice that $\mbox{\tt x}_{\iota}$ is now a constant! 
We may also have additional functional symbols stemming from an 
underlying (sorted) algebraic signature. The reader is asked to 
verify that nothing is lost if we assume that additional function 
symbols only have arity 0, and signature $\auf \auf \alpha, 
\varnothing\zu\zu$ for a suitable $\alpha$. This greatly simplifies 
the presentation of the axioms. 

This defines the language. Notice that in addition to the constants 
$\mbox{\tt x}_{\iota}$ we also have variables $x_i^{\sigma}$ for each 
sort $\sigma$. The former represent the variable $\mbox{\tt x}_i$ (where 
$\iota = \auf \gamma, i\zu$) of the $\lambda$--calculus and the latter 
range over terms of sort $\sigma$. Now, in order to keep the notation 
perspicuous we shall drop the sorts whenever possible. That this is 
possible is assured by the following fact. If $t$ is a term without 
variables, and we hide all the sorts except for those of the variables, 
still we can recover the sort of the term uniquely. For the types this 
is clear, for the second component we observe the following. 
%%%%
\begin{lem}
If a term $t$ has sort $\auf \alpha, \delta\zu$ then
$\fr(t) = \{\mbox{\tt x}_{\iota} : \iota \in \delta\}$.
\end{lem}
%%
The proof of this fact is an easy induction.

\nocite{pigozzisalibra:vb}
For the presentation of the equations we therefore omit the sorts.
They have in fact only been introduced to ensure that we may talk 
about the set of free variables of a term. The equations (vb1) --- 
(vb6) of Table~\ref{tab:vb} characterize the behaviour of the 
substitution and permutation function with respect to the 
indices. We assume that $\iota$, $\mu$, $\nu$ all have the same 
type. (We are dropping the superscripts indicating the sort.)
The equations (vb7) -- (vb11) characterize the pure binding 
by the unary operators $\tlambda_{\iota}$. 
%%
\begin{table}
\caption{The Variable Binding Calculus $\mathsf{VB}$}
\label{tab:vb}
$$\begin{array}{ll@{\quad \doteq \quad}l}
\mbox{\rm (vb1)} & 
    \mbox{\tt s}_{\iota, \mu}\mbox{\tt x}_{\nu} & \left\{\begin{array}{ll}
    \mbox{\tt x}_{\iota} & \mbox{if } \nu \in \{\iota,\mu\}, \\
    \mbox{\tt x}_{\nu}    & \mbox{otherwise.}
    \end{array}\right. \\
    \multicolumn{3}{l}{}
    \\
\mbox{\rm (vb2)} & 
\mbox{\tt p}_{\iota,\mu}\mbox{\tt x}_{\nu} & \left\{\begin{array}{ll}
    \mbox{\tt x}_{\iota} & \mbox{if } \nu = \mu, \\
    \mbox{\tt x}_{\mu} & \mbox{if } \nu = \iota, \\
    \mbox{\tt x}_{\nu}    & \mbox{otherwise.}
    \end{array}\right. \\
    \multicolumn{3}{l}{}
    \\
\mbox{\rm (vb3)} & 
\mbox{\tt p}_{\iota,\iota}x & x \\
\mbox{\rm (vb4)} & 
\mbox{\tt p}_{\iota,\mu}x & \mbox{\tt p}_{\mu,\iota}x \\
\mbox{\rm (vb5)} & 
\mbox{\tt p}_{\iota,\mu}\mbox{\tt p}_{\mu,\nu}x & 
	\mbox{\tt p}_{\mu,\nu}
    \mbox{\tt p}_{\iota,\mu} x \qquad\quad\; \mbox{if } 
	|\{\iota, \mu, \nu\}| = 3
\\
\mbox{\rm (vb6)} & 
\mbox{\tt p}_{\iota, \mu}\mbox{\tt s}_{\mu, \iota} x & 
	\mbox{\tt s}_{\iota, \mu} x \\
\mbox{\rm (vb7)} & 
\mbox{\tt s}_{\iota,\mu}\tlambda_{\nu} x &
    \tlambda_{\nu} \mbox{\tt s}_{\iota,\mu} x
    \qquad\qquad\mbox{ if } |\{\iota,\mu,\nu\}| = 3 \\
\mbox{\rm (vb8)} & 
\mbox{\tt s}_{\mu,\iota} \tlambda_{\iota} x & \tlambda_{\iota} x \\
\mbox{\rm (vb9)} & 
\mbox{\tt p}_{\iota,\mu}\tlambda_{\nu} x &
    \tlambda_{\nu} \mbox{\tt p}_{\iota,\mu} x
    \qquad\qquad\mbox{ if } |\{\iota, \mu, \nu\}| = 3 \\
\mbox{\rm (vb10)} & 
\mbox{\tt p}_{\mu,\iota} \tlambda_{\mu} x & 
	\tlambda_{\iota} \mbox{\tt p}_{\mu,\iota} x \\
\mbox{\rm (vb11)} & 
\tlambda_{\iota}(y \bullet \mbox{\tt x}_{\iota}) & y \\
\mbox{\rm (vb12)} & 
(\tlambda_{\iota}(x \bullet y)) \bullet z &
    ((\tlambda_{\iota} x)\bullet z) \bullet ((\tlambda_{\iota}
    y) \bullet z) \\
\multicolumn{3}{l}{} \\
\mbox{\rm (vb13)} & 
(\tlambda_{\iota}\tlambda_{\mu} x) \bullet y & 
	\left\{\begin{array}{ll}
		\tlambda_{\mu} ((\tlambda_{\iota} x) \bullet y) & 
		\mbox{if }{\iota} \neq {\mu}, \mbox{\tt x}_{\mu} 
		\not\in \fr(y) \\
	\tlambda_{\iota} x & \mbox{if }{\iota} = {\mu} 
	\end{array}\right. \\
\multicolumn{3}{l}{} \\
\mbox{\rm (vb14)} & 
(\tlambda_{\iota} \mbox{\tt x}_{\mu}) \bullet y & 
	\left\{\begin{array}{ll}
		y & \mbox{if }{\iota} = {\mu} \\
		\mbox{\tt x}_{\mu} & \mbox{if }{\iota} \neq {\mu}
	\end{array}\right.
\end{array}$$
\end{table}
%%
The set of equations is invariant under permutation of the indices. 
Moreover, we can derive the invariance under replacement of bound 
variables, for example. Thus, effectively, once the interpretation 
of $\tlambda_{\auf \alpha,0\zu}$ is known, the interpretation of all 
$\tlambda_{\auf \alpha, i\zu}$, $i\in \omega$, is known as well. 
For using the equations we can derive that $\mbox{\tt p}_{\auf \alpha, 
i\zu, \auf \alpha, 0\zu}$ is the inverse of $\mbox{\tt p}_{\auf \alpha, 
0\zu, \auf \alpha, i\zu}$, 
and so 
%%
\begin{equation}
\tlambda_{\auf \alpha, i\zu} x \doteq \mbox{\tt p}_{\auf \alpha, 0\zu,
\auf\alpha, i\zu} \tlambda_{\auf \alpha, 0\zu} \mbox{\tt p}_{\auf \alpha, 
i\zu, \auf \alpha, 0\zu}
x
\end{equation}
%%
The equivalent of \eqref{eq:lef} now turns out to be derivable. 
However, we still need to take care of \eqref{eq:leg}. Since we 
do not dispose of the full substitution $[N/x]$, we need to break 
down \eqref{eq:leg} into an inductive definition (vb12) --- (vb14).
The condition $\mbox{\tt x}_{\mu} \not\in \fr(y)$ is 
just a shorthand; all it says is that we take only those 
equations where the term $y$ has sort $\auf \alpha, \delta\zu$ and 
${\mu} \not\in \delta$. Notice that the disjunction in (vb13) is 
not complete. From these equations we deduce that 
$\mbox{\tt x}_{\auf \alpha, k\zu} \doteq \mbox{\tt p}_{\auf \alpha, k\zu,
\auf \alpha, 0\zu} \mbox{\tt x}_{\auf \alpha, 0\zu}$, 
so we could in principle dispense with all but one variable symbol 
for each type. 

The theory of sorted algebras now provides us with a class of models 
which is characteristic for that theory. We shall not spell out a 
proof that these models are equivalent to models of the $\lambda$--calculus 
in a sense made to be precise. Rather, we shall outline a procedure 
that turns an $\Omega$--algebra into a model of the above equations.
Start with a signature $\auf F, \Omega\zu$, sorted or unsorted. For ease 
of presentation let it be sorted. Then the set $B$ of basic types is the 
set of sorts. Let $\mathsf{Eq}_{\Omega}$ be the equational theory 
of the functions from the signature alone. For complex types, put 
$A_{\alpha\pf\beta} := A_{\alpha} \pf A_{\beta}$. Now transform 
the original signature into a new signature $\Omega' = \auf F, 
\Omega'\zu$ where $\Omega'(f) = 0$ for all $f \in F$. Namely, for 
$f \colon \prod_{i < n} A_{\sigma_i} \pf A_{\tau}$ set 
%%
\index{$f^{\star}$}%%
%%%%
\begin{equation}
f^{\star} := \lambda x_{\auf \sigma_{n-1}, n-1\zu}.\dotsb .\lambda 
x_{\auf \sigma_0, 0\zu}. f^{\GA}(x_{\auf \sigma_{n-1}, n-1\zu},\dotsc, 
x_{\auf \sigma_0, 0\zu})
\end{equation}
%%
This is an element of $A_{\pi}$ where 
%%%
\begin{equation}
\pi := (\sigma_0 \pf (\sigma_1 \pf \dotsb (\sigma_{n -1} 
\pf \tau)\dotsb ))
\end{equation}
%%%
We blow up the types in the way described above. 
This describes the transition from the signature $\Omega$ to a new 
signature $\Omega^{\lambda}$. The original equations are turned into 
equations over $\Omega^{\lambda}$ as follows. 
%%%
\begin{subequations}
\begin{align}
\mbox{\tt x}_{\alpha, i}^{\lambda} & := \mbox{\tt x}_{\auf \alpha, i\zu} \\
(f(\vec{s}))^{\lambda} & := 
	(\dotsb ((f^{\ast} \bullet s_0^{\lambda}) \bullet 
s_1^{\lambda}) \bullet \dotsb \bullet s^{\lambda}_{n-1}) \\
(s \doteq t)^{\lambda} & := s^{\lambda} \doteq t^{\lambda} 
\end{align}
\end{subequations}
%%%
Next, given an $\Omega$--theory, $T$, let $T^{\lambda}$ be the 
translation of $T$, with the postulates (vb1) -- (vb14) added. 
It should be easy to see that if $T \vDash s \boldsymbol{\doteq} t$ 
then also $T^{\lambda} \vDash s^{\lambda} \boldsymbol{\doteq} t^{\lambda}$. 
For the converse we provide a general model construction that for each 
multisorted $\Omega$--structure for $T$ gives a multisorted 
$\Omega^{\lambda}$--structure for $T^{\lambda}$ in which that 
equation fails. 

An \textbf{environment} 
%%%
\index{environment}%%
%%%
is a function $\beta$ from $\CI = \CS \times \omega$ 
($\CS = \Typ_{\pf}(B)$) into $\bigcup \auf A_{\alpha} : \alpha \in \CS\zu$ 
such that for every index $\auf i, \alpha\zu$, 
$\beta(\auf \alpha, i\zu) \in A_{\alpha}$. We denote the set of 
environments by $\CE$. Now let $C_{\auf \alpha, \delta\zu}$
be the set of functions from $\CE$ to $A_{\alpha}$ which depend 
at most on $\delta$. That is to say, if $\beta$ and $\beta'$ are 
environments such that for all $\iota \in \delta$, $\beta(\iota) = 
\beta'(\iota)$, and if $f \in C_{\auf \alpha, \delta\zu}$ then 
$f(\beta) = f(\beta')$.

The constant $f$ is now interpreted by the function 
$[f^{\star}] \colon \beta \mapsto f^{\star}$. For the `variables' 
we put $[\mbox{\tt x}_{\iota}] \colon \beta \mapsto \beta(\iota)$. 
A transformation $\tau \colon \CI \pf \CI$ naturally induces a map 
$\wht{\tau} \colon \CE \pf \CE \colon \beta \mapsto \beta \circ \tau$. 
Further,
%%
\begin{equation}
\begin{split}
\wht{\tau \circ \sigma}(\beta) &
= \beta \circ (\tau \circ \sigma) \\
&
= (\beta \circ \tau) \circ \sigma \\
&
= \wht{\sigma}(\wht{\tau}(\beta)) \\
&
= (\wht{\sigma} \circ \wht{\tau})(\beta)
\end{split}
\end{equation}
%%
Let $\sigma = \auf \alpha, \delta\zu$, $\tau = \auf \alpha, 
[\iota, \mu][\delta]\zu$ and $\upsilon = \auf \alpha, 
(\iota, \mu)[\delta]\zu$.
%%
\begin{align}
\mbox{}[\mbox{\tt s}^{\sigma}_{\iota, \mu}] & \colon 
C_{\sigma} \pf C_{\tau} \colon f \mapsto f \circ \wht{[\iota, \mu]} \\
\mbox{}[\mbox{\tt p}^{\sigma}_{\iota, \mu}] & \colon 
C_{\sigma} \pf C_{\upsilon} \colon f \mapsto f \circ \wht{(\iota, \mu)}
\end{align}
%%
Next, $\bullet^{\auf \alpha\pf\beta, \delta\zu, \auf \alpha, \delta'\zu}$ 
is interpreted as follows.
%%
\begin{multline}
[\bullet^{\auf \alpha\pf\beta, \delta\zu, \auf \alpha, \delta'\zu}] \colon
    C_{\auf \alpha, \delta\zu} \times C_{\auf \beta, \delta'\zu} 
	\pf C_{\auf \alpha, \delta\cup\delta'\zu} \colon \\
	\qquad\qquad
	\auf f, g\zu \mapsto
    \{\auf \beta, f(\beta) \bullet g(\beta)\zu : \beta \in \CE\}
\end{multline}
%%
Finally, we define abstraction. Let $\iota = \auf \gamma, i\zu$.
%%
\begin{multline}
[\tlambda^{\auf \alpha, \delta\zu}_{\iota}] \colon 
	C_{\auf \alpha, \delta\zu} \pf C_{\auf \gamma\pf\beta, 
	\delta - \{\iota\}\zu} \colon \\
\qquad\qquad 
\Gx \mapsto
    \{\auf \beta, \{\auf y, f([y/\beta(\iota)]\beta)\zu 
	: y \in A_{\gamma}\}\zu : \beta \in \CE\}
\end{multline}
%%
It takes some time to digest this definition. Basically, given 
$f$, $g_{f}(\beta) := \{\auf y, f([y/\beta(\iota)]%
\beta)\zu : y \in A_{\gamma}\}$ is a function from $A_{\gamma}$ to 
$A_{\alpha}$ with parameter $\beta \in \CE$. Hence it is a member of 
$A_{\gamma\pf\alpha}$. It assigns to $y$ the value of $f$ on 
$\beta'$, which is identical to $\beta$ except that now 
$\beta(\iota)$ is replaced by $y$. This is the abstraction from $y$. 
Finally, for each $f \in C_{\auf \alpha, \delta\zu}$, 
$[\tlambda^{\auf \alpha, \delta\zu}_{\iota}](f)$ assigns to 
$\beta \in \CE$ the value $g_{f}(\beta)$. (Notice 
that the role of abstraction is now taken over by the set 
formation operator $\{x : \quad\}$.)
%%%
\begin{thm}
Let $\Omega$ a multisorted signature, and $T$ an equational theory over 
$\Omega$. Furthermore, let $\Omega^{\lambda}$ be the signature of 
the $\lambda$--calculus with 0--ary constants $f^{\star}$ for every 
$f \in F$. The theory $T^{\lambda}$ consisting of the translation of 
$T$ and the equations (vb1) --- (vb14) is conservative over $T$. This 
means that an equation $s \doteq t$ valid in the $\Omega$--algebras 
satisfying $T$ iff its translation is valid in all 
$\Omega^{\lambda}$--algebras satisfying $T^{\lambda}$. 
\end{thm}

{\it Notes on this section.} The theory of cylindric algebras has
given rise to a number of difficult problems. First of all, the
axioms shown above do not fully characterize the cylindric
algebras that are representable, that is to say, have as their
domain $U^{\kappa}$, $\kappa$ the dimension, and where relation
variables range over $n$--ary relations over $U$. Thus, although
this kind of cylindric algebra was the motivating example, the
equations do not fully characterize it. As Donald
Monk~\shortcite{monk:cylindric} 
%%%
\index{Monk, Donald}%%%
%%%
has shown, there is no finite set of
schemes (equations using variables for members of set of variable
indices) axiomatizing the class of representable cylindric
algebras of dimension $\kappa$ if $\kappa \geq \aleph_0$;
moreover, for finite $\kappa$, the class of representable algebras
is not finitely axiomatizable. J.~S.~Johnson has shown in
%%%
\index{Johnson, J.~S.}%%%
%%%
\cite{johnson:fpa} an analogue of the second result for polyadic
algebras, Sain and Thompson \shortcite{sainthompson:scheme} 
%%%
\index{Sain, Ildik\'o}%%%
\index{Thompson, Richard S.}%%%
%%%%
an analogue of the first.

The model construction for the model of the $\lambda$--calculus is
called a syntactical model in \cite{barendregt:lambda}. It is due 
to Hindley and Longo from \cite{hindleylongo:models}. 
%%%
\index{Hindley, J.~R.}%%%
\index{Longo, G.}%%
%%%
The approach of using functions from the set of variables into the algebra 
as the carrier set is called a {\it functional environment model}, 
and has been devised by Koymans (see \cite{koymans:lambda}). 
%%%
\index{Koymans, J.~P.~C.}%%%
%%%
A good overview over the different types of models is found in 
\cite{meyer:model} and \cite{koymans:lambda}.
%%
\vplatz
\exercise
For $f$ an $n$--ary function symbol let $R_f$ be an $n+1$--ary
relation symbol. Define a translation from terms to formulae as
follows. First, for a term $t$ let $x_t$ be a variable such
that $x_t \neq x_s$ whenever $s \neq t$.
%%
\begin{subequations}
\begin{align}
(\mbox{\mtt x$_i$})^{\dagger} & := \mbox{\mtt x$_{x_i}$\symbol{61}x$_{x_i}$} \\
f(t_0,\dotsc, t_{n-1})^{\dagger} & :=
    R_f(x_{t_0}, \dotsc, x_{t_{n-1}},
    x_{f(\vec{t})}) \und \gund_{i < n} t_i^{\dagger}
\end{align}
\end{subequations}
%%
Finally, extend this to formulae as follows.
%%
\begin{subequations}
\begin{align}
R(t_0, \dotsc, t_{n-1})^{\dagger} & := R(x_{t_0}, \dotsc, x_{t_{n-1}})
    \und\gund_{i < n} t_i^{\dagger} \\
\mbox{\mtt (\symbol{5}$\varphi$)}^{\dagger} & :=
    \mbox{\mtt (\symbol{5}$\varphi^{\dagger}$)} \\
\mbox{\mtt ($\varphi$\symbol{4}$\chi$)}^{\dagger} & :=
    \mbox{\mtt ($\varphi^{\dagger}$\symbol{4}$\chi^{\dagger}$)} \\
(\mbox{\mtt (\symbol{21}$x$)$\varphi$})^{\dagger} & :=
    \mbox{\mtt (\symbol{21}$x$)$\varphi^{\dagger}$}
\end{align}
\end{subequations}
%%
Now, let $\GM = \auf M, \Pi, \GI\zu$ be a signature. We replace
the function symbols by relation symbols, and let $\GI^+$ be the
extension of $\GI$ such that
%%
\begin{equation}
\GI^+(R_f) = \{\auf \vec{x},y\zu \in M^{\Xi(f)+1} :
\Pi(f)(\vec{x}) = y\}
\end{equation}
%%
Then put $\GM^{\dagger} := \auf M, \GI^+\zu$.
Show that $\auf \GM, \beta\zu \vDash \varphi$
iff $\auf \GM^+, \beta\zu \vDash \varphi^{\dagger}$.
%%%
%%
\vplatz 
\exercise 
Show that if $\GA$ is a cylindric algebra of
dimension $\kappa$, every $\mathsf{E}_{\lambda}$, $\lambda <
\kappa$, satisfies the axioms of $\mathsf{S5}$. Moreover, show that if
$\GA \vDash \varphi$ then $\GA \vDash \nicht\mathsf{E}_{\lambda}
\nicht\varphi$.
%%
\vplatz
\exercise
Prove Lemma~\ref{lem:subst}.
%%
\vplatz
\exercise
Show Proposition~\ref{prop:approx}.
%%
\vplatz
\exercise
Show that $\CL_{\kappa}/\!\equiv$ is a cylindric algebra of
dimension $\kappa$ and that it is locally finite dimensional.
%%%
\vplatz
\exercise
Prove Proposition~\ref{prop:support}.
