\documentclass[12pt]{mouton}
\usepackage[greek,hungarian,english]{babel}
\usepackage[safe]{tipa}
\usepackage{latexsym,amssymb,named,mol,abb2e,eucal,makeidx}
\usepackage[leqno,fleqn]{amsmath}
\usepackage{pifont,bm}
\usepackage[clock]{ifsym}
%%
%% Seitenlayout
%%
\newcommand{\fslash}{\symbol{124}$^{\mbox{\smtt\symbol{43}}}$}
\newcommand{\bslash}{\symbol{124}$^{\mbox{\smtt\symbol{45}}}$}
\newcommand{\bvdash}{\boldsymbol{\vdash}}
\allowdisplaybreaks[1]
\mathindent=5em
\DeclareMathOperator{\subst}{sub}
\DeclareMathOperator{\Suc}{suc}
\DeclareMathOperator{\ein}{in}
\DeclareMathOperator{\aus}{out}
\DeclareMathOperator{\fr}{fr}
\DeclareMathOperator{\var}{var}
\DeclareMathOperator{\FL}{FL}
\DeclareMathOperator{\Taut}{Taut}
\DeclareMathOperator{\son}{son}
\DeclareMathOperator{\der}{der}
\DeclareMathOperator{\supp}{supp}
\DeclareMathOperator{\Typ}{Typ}
\DeclareMathOperator{\Cat}{Cat}
\DeclareMathOperator{\Tm}{Tm}
\DeclareMathOperator{\Clo}{Clo}
\DeclareMathOperator{\Pol}{Pol}
\DeclareMathOperator{\PN}{PN}
\DeclareMathOperator{\focc}{focc}
\DeclareMathOperator{\pt}{Pt}
\DeclareMathOperator{\At}{At}
\DeclareMathOperator{\lub}{lub}
\DeclareMathOperator{\glb}{glb}
\DeclareMathOperator{\CCG}{CCG}
\DeclareMathOperator{\Sent}{Sent}
\DeclareMathOperator{\Cont}{Cont}
\DeclareMathOperator{\im}{im}
%%%%
%%%%
\renewcommand{\doteq}{=}
\newcommand{\nameindex}[1]{}
\newcommand{\sbot}{\boldsymbol{\bot}}
\renewcommand{\labelitemi}{$\ast$}
%\renewcommand{\thesection}{\thechapter.\arabic{section}}
%\renewcommand{\thefigure}{\thechapter.\arabic{figure}}
\newcommand{\hrn}{\leftarrow}
\newcommand{\lauf}{[\![}
\newcommand{\lzu}{]\!]}
\newcommand{\real}[1]{\ulcorner#1\urcorner}
\newcommand{\triangleup}{\vartriangle}
\newcommand{\tlambda}{\mbox{\tt\textgreek{l}}}
\newcommand{\stlambda}{\textgreek{l}}
\newcommand{\conc}{{^{\smallfrown}}}
\newcommand{\oconc}{\diamond}
%%
%% Seitenlayout
%%
%\pagestyle{fancy}
%\fancyhead[LE,RO]{\thepage}
%% \fancyhead[RE]{\bfseries \leftmark}
%% \fancyhead[LO]{\rightmark}
%\fancyfoot{}%
%\renewcommand{\headheight}{15pt}
%\renewcommand{\headsep}{8mm}
%\renewcommand{\sectionmark}[1]{\markright{\footnotesize\thesection.\ #1}}
\clearpage{\pagestyle{empty}\cleardoublepage}
%\renewcommand{\headrulewidth}{0.0pt}
%\renewcommand{\headwidth}{11.8cm}
%\renewcommand{\textheight}{18cm}
%\renewcommand{\textwidth}{11.8cm}
%\setlength{\parindent}{0.5cm}
%\setlength{\parskip}{0cm}
%%
%% Ende Seitenlayout
%%
%%
\makeindex
\setlength{\unitlength}{1em}
%\makeindex
%%
\renewcommand{\phi}{\varphi}
\newcommand{\goth}[1]{\mathfrak{#1}}
%
%\oddsidemargin=2truecm\relax%
%\evensidemargin=2truecm\relax%
%\marginparwidth=2truecm\relax%
%\marginparsep=0.2truecm\relax%
%\topmargin=2truecm\relax%
% \headheight=1truecm\relax%                        %Seite DIN A4
%\headsep=0.4truecm\relax%
%\topskip=\baselineskip%
%\footskip=1truecm%%
%
%
%
\newcommand{\zeichen}[3]{\langle #1 : #2 : #3 \rangle}
\newcommand{\lzeichen}[2]{\langle #1 : #2 : \rangle}
\newcommand{\rzeichen}[2]{\langle : #1 : #2\rangle}
\newcommand{\dzeichen}[2]{\langle #1 :: #2\rangle}
%
\newcommand{\punkt}[1]{\put(#1){\makebox(0,0){$\bullet$}}}
\newcommand{\ubla}[2]{\put(#1){\makebox(0,0){#2}}}
%
%
%
\newcommand{\platz}{\hspace{4em}}
%
% Baeume
%
\newcommand{\binbaum}[3]{
\setlength{\unitlength}{0.8em}
\begin{picture}(10,6)
                        \put(2.2,1.7){\line(1,1){2.8}}
                        \put(7.8,1.7){\line(-1,1){2.8}}
                        \put(5,5.2){\makebox(0,0){$#1$}}
                        \put(2,1){\makebox(0,0){$#2$}}
                        \put(8,1){\makebox(0,0){$#3$}}
                        \end{picture}
\setlength{\unitlength}{1em}}
\newcommand{\recbinbaum}[4]{\begin{picture}(10,6)(#1)
                        \put(2,1.5){\line(1,1){3}}
                        \put(8,1.5){\line(-1,1){3}}
                        \put(5,5){\makebox(0,0){$#2$}}
                        \put(2,1){\makebox(0,0){$#3$}}
                        \put(8,1){\makebox(0,0){$#4$}}
                        \end{picture}}
\newcommand{\unbaum}[2]{\begin{picture}(3,6)
                        \put(1.5,5){\makebox(0,0){#1}}
                        \put(1.5,4.5){\line(0,-1){3}}
                        \put(1.5,1){\makebox(0,0){#2}}
                        \end{picture}}
\newcommand{\recunbaum}[3]{\begin{picture}(3,6)(#1)
                        \put(1.5,5){\makebox(0,0){#2}}
                        \put(1.5,4.5){\line(0,-1){3}}
                        \put(1.5,1){\makebox(0,0){#3}}
                        \end{picture}}
\newcommand{\nbaum}[2]{\begin{picture}(3,6)
                        \put(1.5,5){\makebox(0,0){#2}}
                        \put(1.5,4.5){\line(1,-3){1}}
                        \put(1.5,4.5){\line(-1,-3){1}}
                        \put(0.5,1.5){\line(1,0){2}}
                        \put(1.5,1){\makebox(0,0){#1}}
                        \end{picture}}
%
\newcommand{\sch}[2]{#1 \dotfill \` \pageref{#2} \\}
\newcommand{\osch}[1]{#1 \dotfill \`             \\}
\newcommand{\usch}[2]{\hspace{1em} #1 \dotfill \` \pageref{#2}\\}
\newcommand{\faul}{$\stackrel{\circ}{\sim}$}
%
%
%
\newcommand{\down}[1]{\downarrow\! #1 \!}
\newcommand{\wupp}{\begin{picture}(.6,1)
                        \put(.3,.42){\makebox(0,0){$\uparrow$}}
                        \put(.1,0){\line(1,0){.4}}
                      \end{picture}}
\newcommand{\wdown}{\begin{picture}(.6,1)
                        \put(.3,.4){\makebox(0,0){$\downarrow$}}
                        \put(.1,.9){\line(1,0){.4}}
                      \end{picture}}
\newcommand{\wlhd}{\unlhd}
%%
%%
\newcommand{\vplatz}{\\[2mm]}
\newcounter{uebung}
\setcounter{uebung}{0}
\newcommand{\exercise}{\refstepcounter{uebung}
                       {\bf Exercise \arabic{uebung}. }}
\newcommand{\dexercise}{\refstepcounter{uebung}
                       {\bf $^{\ast}$Exercise \arabic{uebung}. }}
\newcommand{\titel}{The Mathematics of Language}
%
\title{\titel}
\author{Marcus Kracht \\
{\em Department of Linguistics} \\
{\em UCLA} \\
{\em PO Box 951543} \\
{\em 450 Hilgard Avenue} \\
{\em Los Angeles, CA 90095--1543} \\
{\em USA} \\
{\tt kracht@humnet.ucla.edu} \\
\\
Printed Version}
%
\date{September 16, 2003}
%
%
%
\begin{document}
%%
%% Seitenlayout
%%
 \pagenumbering{roman}

 \maketitle
 \mbox{}
 \newpage
 \thispagestyle{empty}
 \setcounter{page}{5}
 \thispagestyle{empty}
 \mbox{}
\vspace{5cm}
%%%
\begin{flushright}
\begin{tabular}{l}
{\it Was dann nachher so sch\"on fliegt . . .} \\
{\it wie lange ist darauf rumgebr\"utet worden.} \\
\\
Peter R\"uhmkorf: {\sl Ph\"onix voran}
\end{tabular}
\end{flushright}
%%%

 \newpage
 \thispagestyle{empty}
 \mbox{}
 \newpage
 \thispagestyle{empty}
 \chapter*{Preface}
%
%
%
The present book developed out of lectures and seminars held over many 
years at the Department of Mathematics of the Freie Universit\"at Berlin, 
the Department of Linguistics of the Universit\"at Potsdam and the 
Department of Linguistics at UCLA. I wish to thank in particular 
the Department of Mathematics at the Freie Universit\"at Berlin 
as well as the Freie Universit\"at Berlin for their support and 
the always favourable conditions under which I was allowed to work.
Additionally, I thank the DFG for providing me with a 
Heisenberg--Stipendium, a grant that allowed me to continue 
this project in between various paid positions.

I have had the privilege of support by Hans--Martin G\"artner, 
%%%
\index{G\"artner, Hans--Martin}%%%
\index{Keenan, Edward L.}%%
\index{Kolb, Hap}%%%
\index{M\"onnich, Uwe}%%%
%%%
Ed Keenan, Hap Kolb and Uwe M\"onnich. Without them I would 
not have had the energy to pursue this work and fill so many 
pages with symbols that create so much headache. They always 
encouraged me to go on. 

Lumme Erilt, Greg Kobele and Jens Michaelis 
%%%
\index{Erilt, Lumme}%%%
\index{Kobele, Greg}%%%
\index{Michaelis, Jens}%%%
%%%
have given me invaluable help by scrupulously reading earlier 
versions of this manuscript. Further, I wish to thank Helmut Alt, 
Christian Ebert, 
%%%
\index{Alt, Helmut}%%%
\index{Ebert, Christian}%%%
\index{Fabian, Benjamin}%%%
\index{Gehrke, Stefanie}%%%
\index{Hanke, Timo}%%%
\index{Hodges, Wilfrid}%%%
\index{J\"ager, Gerhard}%%%
\index{Kanazawa, Makoto}%%%
\index{Koniecny, Franz}%%%
\index{Kosiol, Thomas}%%%
\index{Lin, Ying}%%%
\index{Lipt\'ak, Zsuzsanna}%%
\index{N\'emeti, Istv\'an}%%%
\index{Parsons, Terry}%%%
\index{Manaster--Ramer, Alexis}%%%
\index{Riggle, Jason}%%%
\index{Salinger, Stefan}%%%
\index{Stabler, Edward P.}%%%
\index{Sternefeld, Wolfgang}%%%
\index{Stamm, Harald}%%
\index{Staudacher, Peter}%%%
\index{Tchao, Ngassa}%%%
%%%
Benjamin Fabian, Stefanie Gehrke, Timo Hanke, Wilfrid Hodges, Gerhard 
J\"ager, Makoto Kanazawa, Franz Koniecny, Thomas Kosiol, Ying Lin, 
Zsu\-zsan\-na Lipt\'ak, Istv\'an N\'emeti, Terry Parsons, Alexis--Manaster 
Ramer, Jason Riggle, Stefan Salinger, Ed Stabler, Harald Stamm, Peter 
Staudacher, Wolfgang Sternefeld and Ngassa Tchao for their help.
%%%
\\[1cm]
Los Angeles and Berlin, September 2003
\hfill 
{\it Marcus Kracht}

 \newpage
 \thispagestyle{empty}
 \mbox{}
 \newpage
 \thispagestyle{empty}
 \chapter*{Introduction}
\markboth{Introduction}{}
%
%
%
This book is --- as the title suggests --- a book about
the mathematical study of language, that is, about the
description of language and languages with mathematical
methods. It is intended for students of mathematics,
linguistics, computer science, and computational linguistics,
and also for all those who need or wish to understand the
formal structure of language. It is a mathematical book; it 
cannot and does not intend to replace a genuine introduction
to linguistics. For those who are not acquainted with
general linguistics we recommend \cite{lyons:linguistik},
which is a bit outdated but still worth its while. For a more 
recent book see \cite{fromkin:introduction}. No linguistic 
theory is discussed here in detail. This text only provides the 
mathematical background that will enable the reader to fully 
grasp the implications of these theories and understand them 
more thoroughly than before. Several topics of mathematical 
character have been omitted: there is for example no statistics, 
no learning theory, and no optimality theory. All these 
topics probably merit a book of their own. On the linguistic 
side the emphasis is on syntax and formal semantics, though 
morphology and phonology do play a role. These omissions are 
mainly due to my limited knowledge. However, this book is 
already longer than I intended it to be. No more material 
could be fitted into it.

The main mathematical background is algebra and logic on
the semantic side and strings on the syntactic side. In
contrast to most introductions to formal semantics we do not
start with logic --- we start with strings and develop
the logical apparatus as we go along. This is only a
pedagogical decision. Otherwise, the book would start with
a massive theoretical preamble after which the reader is
kindly allowed to see some worked examples. Thus we have decided
to introduce logical tools only when needed, not as
overarching concepts.

We do not distinguish between natural and formal languages. These
two types of languages are treated completely alike. I believe
that it should not matter in principle whether what we have is a
natural or an artificial product. Chemistry applies to naturally
occurring substances as well as artificially produced ones. All I
will do here is study the structure of language. Noam Chomsky
%%%
\index{Chomsky, Noam}%%%
%%%%
has repeatedly claimed that there is a fundamental difference
between natural and nonnatural languages. Up to this 
moment, conclusive evidence for this claim is missing. Even if 
this were true, this difference should not matter for this book. 
To the contrary, the methods established here might serve as a tool
in identifying what the difference is or might be. The present
book also is not an introduction to the theory of formal
languages; rather, it is an introduction to the mathematical
theory of linguistics. The reader will therefore miss a few topics
that are treated in depth in books on formal languages on the
grounds that they are rather insignificant in linguistic theory.
On the other hand, this book does treat subjects that are hardly
found anywhere else in this form. The main characteristic of our
approach is that we do not treat languages as sets of strings but
as algebras of signs. This is much closer to the linguistic
reality. We shall briefly sketch this approach, which will be
introduced in detail in Chapter~\ref{kap3}.

%%
\index{sign}
%%%
A \textbf{sign} $\sigma$ is defined here as a triple $\auf e, c, m\zu$,
%%
\index{sign!exponent}%%
\index{sign!category}%%
\index{sign!meaning}%%
%%%
where $e$ is the \textbf{exponent of} $\sigma$, which typically is
a string, $c$ the (\textbf{syntactic}) \textbf{category of} $\sigma$, 
and $m$ its \textbf{meaning}. By this convention a string is connected
via the language with a set of meanings. Given a set $\Sigma$ of signs, 
$e$ \textbf{means} $m$ \textbf{in} $\Sigma$ if and only if (= iff) 
there is a category $c$ such that $\auf e,c,m\zu \in \Sigma$. Seen 
this way, the task of language theory is not only to say which are 
the legitimate exponents of signs (as we find in the theory of
formal languages as well as many treatises on generative
linguistics which generously define language to be just syntax)
but it must also say which string can have what meaning.
The heart of the discussion is formed by the principle of
compositionality,
%%%
\index{compositionality}%%
%%%
which in its weakest formulation says that the meaning of a
string (or other exponent) is found by homomorphically mapping
its analysis into the semantics. Compositionality shall be
introduced in Chapter~\ref{kap3} and we shall discuss at
length its various ramifications. We shall also deal with
Montague Semantics, which arguably was the first to 
implement this principle. Once again, the discussion will be
rather abstract, focusing on mathematical tools rather than
the actual formulation of the theory. Anyhow, there are good
introductions to the subject which eliminate the need to
include details. One such book is \cite{dowtywallpeters} and 
the book by the collective of authors \cite{gamut:teil2}.
%%
\nocite{gamut:teil1}
%%
A \textbf{system of signs} is a partial algebra of signs.
This means that it is a pair $\auf \Sigma, M\zu$,
where $\Sigma$ is a set of signs and $M$  a finite set,
%%%
\index{mode}%%
%%%
the set of so--called \textbf{modes} (\textbf{of composition}).
Standardly, one assumes $M$ to have only one nonconstant mode, 
a binary function $\bullet$, which allows one to form a sign
$\sigma_1 \bullet \sigma_2$ from two signs $\sigma_1$ and
$\sigma_2$. The modes are generally partial operations.
The action of $\bullet$ is explained by defining its
action on the three components of the respective signs.
We give a simple example. Suppose we have the following
signs.
%%
$$\begin{array}{l@{\quad = \quad}l}
\mbox{\tt `runs'} & \auf \mbox{\tt runs}, v, \rho\zu \\
\mbox{\tt `Paul'} & \auf \mbox{\tt Paul}, n, \pi\zu
\end{array}$$
%%
Here, $v$ and $n$ are the syntactic categories {\it (intransitive)
verb\/} and {\it proper name}, respectively. $\pi$ is a constant,
which denotes an individual, namely Paul, and $\rho$ is a  function
from individuals to the set of truth values, which typically is 
the set $\{0,1\}$. (Furthermore, $\rho(x) = 1$ if and only if $x$ 
is running.) On the level of exponents we choose word concatenation,
which is string concatenation (denoted by $^{\smallfrown}$) with an 
intervening blank. (Perfectionists will also add the period at the 
end...) On the level of meanings we choose function application.
Finally, let $\circ$ be a partial function which is only defined if 
the first argument is $n$ and the second is $v$ and which in this 
case yields the value $t$. Now we put
%%
$$\auf e_1, c_1, m_1\zu \bullet
    \auf e_2, c_2, m_2\zu :=
\auf e_1^{\smallfrown}\Box^{\smallfrown}e_2, c_1 \circ c_2,
m_2(m_1)\zu$$
%%
Then $\mbox{\tt `Paul'} \bullet \mbox{\tt `runs'}$
is a sign, and it has the following form.
%%
$$\mbox{\tt `Paul'} \bullet \mbox{\tt `runs'} :=
\auf \mbox{\tt Paul runs}, t, \rho(\pi)\zu$$
%%
We shall say that this sentence is true if and only if 
$\rho(\pi) = 1$; otherwise we say that it is false. We hasten 
to add that $\mbox{\tt `Paul'} \bullet \mbox{\tt `Paul'}$ is 
{\it not\/} a sign. So, $\bullet$ is indeed a partial operation.

The key construct is the free algebra generated by the constant
modes alone. This algebra
%%%
\index{algebra of structure terms}
%%%
is called the \textbf{algebra of structure terms}. The structure
terms can be generated by a simple context free grammar. However,
not every structure term names a sign. Since the algebras of
exponents, categories and meanings are partial algebras, it is in
general not possible to define a homomorphism from the algebra of
structure terms into the algebra of signs. All we can get is a
partial homomorphism. In addition, the exponents are not always
strings and the operations between them not only concatenation.
Hence the defined languages can be very complex (indeed, every
recursively enumerable language $\Sigma$ can be so generated).

Before one can understand all this in full detail it is
necessary to start off with an introduction into classical
formal language theory using semi Thue systems and grammars
in the usual sense. This is what we shall do in Chapter~\ref{kap1}. 
It constitutes the absolute minimum one must know about these 
matters. Furthermore, we have
added some sections containing basics from algebra, set theory,
computability and linguistics. In Chapter~\ref{kap2} we study
regular and context free languages in detail. We shall deal
with the recognizability of these languages by means of automata,
recognition and analysis problems, parsing, complexity, and
ambiguity. At the end we shall discuss semilinear languages and 
Parikh's Theorem.

In Chapter~\ref{kap3} we shall begin to study languages as 
systems of signs. Systems of signs and grammars of signs are 
defined in the first section.  Then we shall concentrate on 
the system of categories and
the so--called categorial grammars. We shall introduce
both the Ajdukiewicz--Bar Hillel Calculus and the
Lambek--Calculus. We shall show that both can generate exactly
the context free string languages. For the Lambek--Calculus,
this was for a long time an open problem, which was solved in
the early 1990s by Mati Pentus.
%%%
\index{Pentus, Mati}%%%
%%%

Chapter~\ref{kap6} deals with formal semantics. We shall develop
some basic concepts of algebraic logic, and then deal with boolean
semantics. Next we shall provide a completeness theorem for simple 
type theory and discuss various possible algebraizations. Then we 
turn to the possibilities and limitations of Montague Semantics. 
Then follows a section on partiality and presupposition. 

In the fifth chapter we shall treat so--called PTIME languages.
These are languages for which the parsing problem is decidable
deterministically in polynomial time. The question whether or not
natural languages are context free was considered settled negatively
until the 1980s. However, it was shown that most of the arguments were
based on errors, and it seemed that none of them was actually tenable.
Unfortunately, the conclusion that natural languages are actually all
context free turned out to be premature again. It now seems that natural
languages, at least some of them, are not context free. However, all 
known languages seem to be PTIME languages. Moreover, the so--called 
weakly context sensitive languages also belong to this class. 
A characterization of this class in terms of a generating device was 
established by William Rounds, 
%%%
\index{Rounds, William}%%%
%%%
and in a different way by Annius Groenink, 
%%%
\index{Groenink, Annius}%%%
%%%
who introduced the notion of a literal movement grammar. We shall 
study these types of grammars in depth. 
In the final two sections we shall return to the question
of compositionality in the light of Leibniz' Principle,
and then propose a new kind of grammars, de Saussure grammars,
which eliminate the duplication of typing information
found in categorial grammar.

The sixth chapter is devoted to the logical description of language. 
This approach has been introduced in the 1980s and is currently 
enjoying a revival. The close connection between this approach and
the so--called constraint--programming is not accidental. It was 
proposed to view grammars not as generating devices but as theories 
of correct syntactic descriptions.  This is very far away from the 
tradition of generative grammar advocated by Chomsky, 
%%%
\index{Chomsky, Noam}%%%
%%%
who always insisted that language contains a generating device (though 
on the other hand he characterizes this as a theory of competence). 
However, it turns out that there is a method to convert descriptions 
of syntactic structures into syntactic rules. This goes back
to ideas by B\"uchi, 
%%%
\index{B\"uchi, J.}%%%
\index{Thatcher, J.~W.}%%%
\index{Doner, J.~E.}%%%
\index{Wright, J.~B.}%%%
%%%
Wright as well as Thatcher and Doner on theories of strings and 
theories of trees in monadic second order logic. However, the reverse 
problem, extracting principles out of rules, is actually very hard, 
and its solvability depends on the strength of the description
language. This opens the way into a logically based language
hierarchy, which indirectly also reflects a complexity
hierarchy. Chapter~\ref{kap5} ends with an overview of the
major syntactic theories that have been introduced in the
last 25 years.

{\sc Notation.} Some words concerning our notational conventions.
We use typewriter font for true characters in print. For example:
{\tt Maus} is the German word for `mouse'. Its English counterpart
appears in (English) texts either as {\tt mouse} or as {\tt Mouse}, 
depending on whether or not it occurs at the beginning of
a sentence. Standard books on formal linguistics often ignore
these points, but since strings are integral parts of signs we
cannot afford this here. In between true characters in print we
also use so--called {\it metavariables\/} (placeholders) such as
$a$ (which denotes a single letter) and $\vec{x}$ (which denotes a
string). The notation $\mbox{\tt c}_i$ is also used, which is
short for the true letter {\tt c} followed by the binary code of
$i$ (written with the help of appropriately chosen characters, 
mostly {\tt 0} and {\tt 1}).  When defining languages as sets of
strings we distinguish between brackets that appear in print
(these are {\tt (} and {\tt )}) and those which are just used to
help the eye. People are used to employ abbreviatory conventions,
for example {\tt 5+7+4} in place of {\tt (5+(7+4))}. Similarly, in 
logic one uses {\mtt p$_{\snull}$\symbol{4}(\symbol{5}p$_{\seins}$)} 
or even {\mtt p$_{\snull}$\symbol{4}\symbol{5}p$_{\seins}$} in place 
of {\mtt (p$_{\snull}$\symbol{4}(\symbol{5}p$_{\seins}$))}. We shall 
follow that usage when the material shape of the formula is immaterial, 
but in that case we avoid using the true function symbols and 
the true brackets `{\mtt (}' and `{\tt )}', and use `$($' and 
`$)$' instead. For {\mtt p$_{\snull}$\symbol{4}(\symbol{5}p$_{\seins}$)} 
is actually {\it not\/} the same as 
{\mtt (p$_{\snull}$\symbol{4}(\symbol{5}p$_{\seins}$))}. To the 
reader our notation may appear overly pedantic. However, 
since the character of the representation is part of what we are 
studying, notational issues become syntactic issues, and syntactical 
issues simply cannot be ignored. Notice that 
`$\auf$' and `$\zu$' are truly metalinguistic symbols that
are used to define sequences. We also use sans serife fonts for terms
in formalized and computer languages, and attach a prime to refer
to its denotation (or meaning). For example, the computer code for 
a while--loop is written semi--formally as \textsf{while $i < 100$ do 
$x := x \times (x + i)$ od}. This is just a string of symbols.
However, the notation $\textsf{see}'(\textsf{john}', \textsf{paul}')$ 
denotes the proposition that John sees Paul, not the
sentence expressing that.

 \newpage
 \thispagestyle{empty}
 \tableofcontents
 \newpage
 \thispagestyle{empty}
 \pagenumbering{arabic}
 \setcounter{page}{1}
 \chapter{Fundamental Structures}
\thispagestyle{empty}
\label{kap1}
%%%
\section{Algebras and Structures}
\label{kap1-1}
%
%
%
In this section we shall provide definitions of basic terms
and structures which we shall need throughout this book.
Among them are the notions of {\it algebra\/} and {\it structure}.
Readers for whom these are entirely new are advised
to read this section only cursorily and return to it only when
they hit upon something for which they need background
information.

We presuppose some familiarity with mathematical thinking,
in particular some knowledge of elementary set theory
%%%
\index{set}%%
%%%
and proof techniques such as induction. For basic concepts
in set theory see \cite{vaught:set} or
\cite{justweese:set1,justweese:set2}; for background
in logic see \cite{goldsternjudah}.
Concepts from algebra (especially universal algebra) can
be found in \cite{burris} and \cite{graetzer:algebra}, and in 
\cite{burmeister:partial} and \cite{burmeister:lecturenotes} 
for partial algebras; for general background on
lattices and orderings see \cite{graetzer:lattice}
and \cite{davey}.

We use the symbols
%%%
\index{$\cap$, $\cup$, $-$, $\varnothing$}%%
\index{$\wp$, $\wp_{fin}$}%%
%%%
$\cup$ for the union, $\cap$ for the intersection of two sets.
Instead of the difference symbol $M\backslash N$ we use $M - N$.
$\varnothing$ denotes the empty set. $\wp(M)$ denotes the set of
subsets of $M$, $\wp_{fin}(M)$ the set of finite subsets of $M$.
Sometimes it is necessary to take the union of two sets that does
not identify the common symbols from the different sets. In that
case one uses $+$. %
%%%
\index{$+$}%%
\index{$\in$, $\doteq$, $\subseteq$}%%
%%%%
We define $M + N := M \times \{0\} \cup N \times
\{1\}$ ($\times$ is defined below). This is called the 
\textbf{disjoint union}. For reference, we
fix the background theory of sets that we are using. This is the theory
{\sf ZFC} (Zermelo Fraenkel Set Theory with Choice). It is
essentially a first order theory with only two two place relation
symbols, $\in$ and $\doteq$. (See Section~\ref{kap3}.\ref{kap3-6} for a
definition of first order logic.) We define $x \subseteq y$ 
by $(\forall z)(z \in x \pf x \in y)$. Its axioms are as follows.
%%%
\begin{enumerate}
\item {\sl Singleton Set Axiom}.
    $(\forall x)(\exists y)(\forall z)(z \in y
    \dpf z \doteq x)$.  \\
    This makes sure that for every $x$ we have a set $\{x\}$.
\item {\sl Powerset Axiom}. $(\forall x)(\exists y)(\forall z)(%
    z \in y \dpf z \subseteq x)$. \\
    This ensures that for every $x$ the power set $\wp(x)$
    of $x$ exists.
\item {\sl Set Union}. $(\forall x)(\exists y)(%
    \forall z)(z \in y \dpf (\exists u)(z \in u \und u \in x))$.
    \\
    $u$ is denoted by $\bigcup_{z \in x} z$ or simply by
    $\bigcup x$. The axiom guarantees its existence.
\item {\sl Extensionality}. $(\forall x)(\forall y)(x \doteq y \dpf
    (\forall z)(z \in x \dpf z \in y))$.
\item {\sl Replacement.} If $f$ is a function with domain
    $x$ then the direct image of $x$ under $f$ is a set.
    (See below for a definition of {\it function}.)
\item {\sl Weak Foundation}. 
	$$(\forall x)(x \neq \varnothing \pf
    (\exists y)(y \in x \und (\forall z)(z \in x \pf
    z \not\in y)))$$
    This says that in every set there exists an element that
    is minimal with respect to $\in$.
\item {\sl Comprehension}. If $x$ is a set and $\varphi$ a
    first order formula with only $y$ occurring free, then 
	$\{y : y \in x \und \varphi(y)\}$ also is a set.
\item {\sl Axiom of Infinity}. There exists an $x$ and an injective
    function $f \colon x \pf x$ such that the direct image of
    $x$ under $f$ is not equal to $x$.
\item {\sl Axiom of Choice}. For every set of sets $x$ there
    is a function $f : x \pf \bigcup x$ with
    $f(y) \in y$ for all $y \in x$.
\end{enumerate}
%%%
We remark here that in everyday discourse, comprehension is
generally applied to all collections of sets, not just elementarily
definable ones. This difference will hardly matter here; we only
mention that in monadic second order logic this stronger from of 
comprehension is expressible and also the axiom of foundation.
%%%
\begin{quote}
%%%
\index{comprehension}
%%%%
{\sl Full Comprehension.} For every class $P$ and every set $x$, 
$\{y : y \in x \text{ and }x \in P\}$ is a set.
\end{quote}
%%%
Foundation is usually defined as follows 
%%%
\begin{quote}
%%%
\index{foundation}%%%
%%%
{\sl Foundation.} There is no infinite
chain $x_0 \ni x_1 \ni  x_2 \ni \dotsb$.
\end{quote}
%%%
In mathematical usage, one often forms certain collections of
sets that can be shown not to be sets themselves. One example
is the collection of all finite sets. The reason that it is not a
set is that for every set $x$, $\{x\}$ also is a set. The
function $x \mapsto \{x\}$ is injective (by extensionality), and so
there are as many finite sets as there are sets. If the collection of
finite sets were a set, say $y$, its powerset has strictly more elements
than $y$ by a theorem of Cantor. But this is impossible, since $y$
has the size of the universe. Nevertheless, mathematicians do use these
collections (for example, the collection of $\Omega$--algebras).
This is not a problem, if the following is observed. A collection 
of sets is called a \textbf{class}. %%
%%%
\index{class}%%
%%%%
A class is a set iff it is contained in a set as an element. (We 
use `iff' to abbreviate `if and only if'.)

In set theory, numbers are defined as follows.
%%
\begin{equation}
\begin{split}
0   & := \varnothing \\
n+1 & := \{k : k < n\} = \{0, 1, 2, \dotsc, n-1\}
\end{split}
\end{equation}
%%
\index{$\omega$, $\aleph_0$}%%
\index{$\wp(M)$}%%
%%%
The set of so--constructed numbers is denoted by $\omega$. It
is the set of \textbf{natural numbers}.
%%%
\index{natural number}%%
\index{ordinal}%%
%%%
In general, an \textbf{ordinal} (\textbf{number}) is a set that is
transitively and linearly ordered by $\in$. (See below for these
concepts.) For two ordinals $\kappa$ and $\lambda$, either $\kappa
\in \lambda$ (for which we also write $\kappa < \lambda$) or
$\kappa = \lambda$ or $\lambda \in \kappa$. 
%%%
\begin{thm}
For every set $x$ there exists an ordinal $\kappa$ and a bijective 
function $f \colon \kappa \pf x$.
\end{thm}
%%%
\index{well--ordering}%%%
%%%
$f$ is also referred to as a \textbf{well--ordering} of $x$.
The finite ordinals are exactly the natural numbers defined above.
%%%
\index{cardinal}%%
%%%
A \textbf{cardinal} (\textbf{number}) is an ordinal $\kappa$ such 
that for every ordinal $\lambda < \kappa$ there is no onto map 
$f \colon \lambda \pf \kappa$. It is not hard to see that every set 
can be well--ordered by a cardinal number, and this cardinal is unique. 
It is denoted by $|M|$ and called the \textbf{cardinality of} $M$. The 
smallest infinite cardinal is denoted by $\aleph_0$. The following is 
of fundamental importance.
%%%
\begin{thm}
For two sets $x$, $y$ exactly one of the following holds: $|x| < 
|y|$, $|x| = |y|$ or $|x| > |y|$.
\end{thm}
%%%
By definition, $\aleph_0$ is actually identical to $\omega$ so that 
it is not really necessary to distinguish the two. However, we shall 
do so here for reasons of clarity. (For example, infinite cardinals 
have a different arithmetic than ordinals.) If $M$ is finite, its 
cardinality is a natural number. If $|M| = \aleph_0$, $M$ is called 
\textbf{countable}; it is \textbf{uncountable} otherwise.
%%%
\index{set!countable}%%
%%%
If $M$ has cardinality $\kappa$, the cardinality of $\wp(M)$
is denoted by $2^{\kappa}$. $2^{\aleph_0}$ is the cardinality
of the set of all real numbers. $2^{\aleph_0}$ is strictly
greater than $\aleph_0$ (but need not be the smallest uncountable 
cardinal). We remark here that the set of finite sets of natural 
numbers is countable.

If $M$ is a set, a \textbf{partition} 
%%%
\index{partition}%%
%%%
of $M$ is a set $P \subseteq \wp(M)$ such that every member of $P$ 
is nonempty, $\bigcup P = M$ and for all $A, B \in P$ such that 
$A \neq B$, $A \cap B = \varnothing$. 
If $M$ and $N$ are sets, $M \times N$ denotes the set of all
pairs $\auf x, y\zu$, where $x \in M$ and $y \in N$.
A definition of $\auf x,y\zu$, which goes back to Kuratowski 
and Wiener, is as follows.
%%
\begin{equation}
\auf x, y\zu := \{x, \{x,y\}\}
\end{equation}
%%
\begin{lem}
$\auf x, y\zu = \auf u, v\zu$ iff $x = u$ and $y = v$.
\end{lem}
%%
\proofbeg
By extensionality, if $x = u$ and $y = v$ then $\auf x, y\zu =
\auf u, v\zu$. Now assume that $\auf x, y\zu = \auf u, v\zu$.
Then either $x = u$ or $x = \{u, v\}$, and $\{x, y\} = u$ or $\{x,y\} =
\{u,v\}$. Assume that $x = u$. If $u = \{x,y\}$ then $x = \{x, y\}$, 
whence $x \in x$, in violation to foundation. Hence we have 
$\{x,y\} = \{u, v\}$. Since $x = u$, we must have $y = v$. This 
finishes the first case. Now assume that $x = \{u,v\}$. Then 
$\{x,y\} = u$ cannot hold, for then $u = \{\{u,v\},y\}$, whence 
$u \in \{u,v\} \in u$. So, we must have $\{x,y\} = \{u,v\}$. 
However, this gives $x = \{x,y\}$, once again a contradiction.
So, $x = u$ and $y = v$, as promised.
\proofend

With these definitions, $M \times N$ is a set if $M$ and $N$ are
sets. A
%%%
\index{relation}%%
%%%
\textbf{relation} from $M$ to $N$ is a subset of
$M \times N$. We write $x\, R\, y$ if $\auf x, y\zu
\in R$. Particularly interesting is the case $M = N$.
A relation $R \subseteq M \times M$ is called
%%%
\index{relation!reflexive}%%
\index{relation!symmetric}%%
\index{relation!transitive}%%
\index{equivalence relation}%%
%%%
\textbf{reflexive} if $x \, R\, x$ for all $x \in M$;
\textbf{symmetric} if from $x\, R\, y$ follows that
$y \, R\, x$. $R$ is called \textbf{transitive} if from
$x \, R\, y$ and $y \, R\, z$ follows $x \, R \, z$.
An \textbf{equivalence relation} on $M$ is a reflexive,
symmetric and transitive relation on $M$.
%%%
\index{ordered set}%%
%%%
A pair $\auf M, <\zu$ is called an \textbf{ordered set}
if $M$ is a set and $<$ a transitive, irreflexive binary relation
on $M$. $<$ is then called a
%%%
\index{ordering}%%
%%%
(\textbf{strict}) \textbf{ordering on} $M$ and $M$ is then called
\textbf{ordered by $<$}. $<$ is \textbf{linear} if for any two
elements $x, y \in M$ either $x < y$ or $x = y$ or $y < x$.
A \textbf{partial ordering} is a relation which is reflexive,
transitive and antisymmetric; the latter means that
from $x\, R \, y$ and $y\, R\, x$ follows $x = y$.

If $R \subseteq M \times N$ is a relation, we write
$R^{\smallsmile} := \{\auf x, y\zu : y\, R\, x\}$ for the
so--called \textbf{converse of} $R$. This is a relation from
$N$ to $M$. If $S \subseteq N \times P$ and $T \subseteq M \times N$
are relations, put
%%%%
\index{$R \circ S$, $R \cup S$}%%%
%%%%
\begin{align}
R \circ S & := \{\auf x,y\zu : \text{ for some } 
	z\colon x\, R\, z \, S\, y\} \\\notag
R \cup T  & := \{\auf x,y\zu : x \, R\, y \text{ or } x \, T\, y\}
\end{align}
%%
We have $R \circ S \subseteq M \times P$ and $R \cup T \subseteq
M \times N$. In case $M = N$ we still make further definitions.
We put $\Delta_M := \{\auf x,x\zu : x \in M\}$ and call this set
the \textbf{diagonal on} $M$. Now put
%%%
\index{diagonal}%%
\index{$\Delta$, $R^n$, $R^+$, $R^{\ast}$}%%%
%%
\begin{align}
R^0      & := \Delta_M & R^{n+1}  & := R \circ R^n \\\notag
R^+      & := \bigcup_{0 < i \in \omega} R^i & R^{\ast} 
& := \bigcup_{i \in \omega} R^i
\end{align}
%%
$R^+$ is the smallest transitive relation which contains $R$.
%%%
\index{transitive closure}%%
%%%
It is therefore called the \textbf{transitive closure of} $R$.
$R^{\ast}$ is the smallest reflexive and transitive relation
containing $R$.

%%%
\index{function}%%
\index{function!partial}%%
%%%%
A \textbf{partial function} from $M$ to $N$ is a relation $f \subseteq
M \times N$ such that if $x \, f \, y$ and $x \, f \, z$
then $y = z$. $f$ is a \textbf{function} if for every $x$ there is a 
$y$ such that $x\; f\; y$. We write $y = f(x)$ to say that $x\, f\, y$
and $f \colon M \pf N$ to say that $f$ is a function from $M$ to
$N$. If $P \subseteq M$ then $f \restriction P :=
f \cap (P \times N)$.
%%%
\index{$f \restriction P$}%%
%%%
Further, $f \colon M \epi N$ abbreviates that $f$ is a
%%%
\index{function!surjective}%%
\index{$\epi$, $\mono$}%%
%%%
\textbf{surjective} function, that is, every $y \in N$ is of
the form $y = f(x)$ for some $x \in M$. And we write
%%%
\index{function!injective}%%
%%%
$f \colon M \mono N$ to say that $f$ is \textbf{injective}, that is,
for all $x, x' \in M$, if $f(x) = f(x')$ then $x = x'$.
%%%
\index{function!bijective}%%
%%%
$f$ is \textbf{bijective} if it is injective as well as surjective.
Finally, we write $f \colon x \mapsto y$ if $y = f(x)$. If
$X \subseteq M$ then $f[X] := \{f(x) : x \in X\}$ is the
%%%
\index{direct image}%%
\index{$f[X]$}%%
%%%
so--called \textbf{direct image of} $X$ \textbf{under}
$f$. We warn the reader of the difference between
$f(X)$ and $f[X]$. For example, let
$\Suc \colon \omega \pf \omega \colon x \mapsto x + 1$. Then
according to the definition of natural numbers above
we have $\Suc(4) = 5$ and $\Suc[4] = \{1,2,3,4\}$, since
$4 = \{0,1,2,3\}$. Let $M$ be an arbitrary set. There
is a bijection between the set of subsets of $M$ and the 
set of functions from $M$ to $2 = \{0,1\}$, which is defined 
as follows.  For $N \subseteq M$ we call $\chi_N \colon M \pf \{0,1\}$ 
the \textbf{characteristic function}
%%%%
\index{characteristic function}%%
%%%%
of $N$ if $\chi_N(x) = 1$ iff $x \in N$.
%%%
\index{$\chi_N$}%%
%%
Let $y \in N$ and $Y \subseteq N$; then put
$f^{-1}(y) := \{x : f(x) = y\}$ and
$f^{-1}[Y]  := \{x : f(x) \in Y\}$. If $f$ is injective,
$f^{-1}(y)$ denotes the unique $x$ such that $f(x) = y$
(if that exists). We shall see to it that this overload
in notation does not give rise to confusions.

$M^n$, $n \in \omega$, denotes the set of $n$--tuples
of elements from $M$. 
%%
\begin{align}
M^1     & := M & M^{n+1} & := M^n \times M
\end{align}
%%
In addition, $M^0 := 1 (= \{\varnothing\})$.
Then an $n$--tuple of elements from $M$ is an element of
$M^n$. Depending on need we shall write
$\auf x_i : i < n\zu$ or $\auf x_0, x_1, \dotsc, x_{n-1}\zu$
for a member of $M^n$.

An $n$--\textbf{ary relation} on $M$ is a subset of $M^n$,
an $n$--\textbf{ary function} on $M$ is a function $f \colon M^n \pf M$.
$n = 0$ is admitted. A 0--ary relation is a subset of $1$, hence 
it is either the empty set or the set $1$ itself. A 0--ary function 
on $M$ is a function $c \colon 1 \pf M$. We also call it a 
\textbf{constant}.
%%%
\index{constant}%%
%%%
The \textbf{value} of this constant is the element $c(\varnothing)$.
Let $R$ be an $n$--ary relation and $\vec{x} \in M^n$. Then we
write $R(\vec{x})$ in place of $\vec{x} \in R$.

Now let $F$ be a set and $\Omega \colon F \pf \omega$.
%%%
\index{signature}%%
%%%
The pair $\auf F, \Omega\zu$, also denoted by $\Omega$
alone, is called a \textbf{signature} and $F$ the set of
\textbf{function symbols}.
%%
\begin{defn}
%%%
\index{algebra}%%
\index{algebra!$\Omega$--}%%%
%%%
Let $\Omega \colon F \pf \omega$ be a signature and $A$
a nonempty set. Further, let $\Pi$ be a mapping
which assigns to every $f \in F$ an $\Omega(f)$--ary
function on $A$. Then we call the pair $\GA := \auf
A, \Pi\zu$ an \textbf{$\Omega$--algebra}. $\Omega$--algebras
are in general denoted by upper case German letters.
\end{defn}
%%
In order not to get drowned in notation we write $f^{\GA}$ 
for the function $\Pi(f)$. In place of denoting $\GA$ by the 
pair $\auf A, \Pi\zu$ we shall denote it somewhat ambiguously 
by $\auf A, \{f^{\GA} : f \in F\}\zu$. We warn the reader that 
the latter notation may give rise to confusion since
functions of the same arity can be associated with different
function symbols. However, this problem shall not arise.

%%%
\index{$\Tm_{\Omega}$}%%
\index{term}%%%
%%%
The set of $\Omega$--terms is the smallest set $\Tm_{\Omega}$ 
such that if $f \in F$ and $t_i \in \Tm_{\Omega}$, 
$i < \Omega(f)$, also $f(t_0, \dotsc, t_{\Omega(f)-1}) \in
    \Tm_{\Omega}$.
Terms are abstract entities; they are not to be equated
with functions nor with the strings by which we denote them.
%%%
\index{term!level of a \faul}%
%%%
To begin we define the \textbf{level} of a term. If $\Omega(f) = 0$,
then $f()$ is a term of level 0, which we also denote by `$f$'.
If $t_i$, $i < \Omega(f)$, are terms of level $n_i$,
then $f(t_0, \dotsc, t_{\Omega(f)-1})$ is a term of level
$1 + \max \{n_i : i < \Omega(f)\}$. Many proofs
run by induction on the level of terms, we therefore speak about
{\it induction on the construction of the term}. Two terms $u$ and
$v$ are equal, in symbols $u = v$, if they have identical level and
either they are both of level 0 and there is an $f \in F$ such
$u = v = f()$ or there is an $f \in F$, and terms $s_i$, $t_i$,
$i < \Omega(f)$, such that $u = f(s_0, \dotsc, s_{\Omega(f)-1})$
and $v = f(t_0, \dotsc, t_{\Omega(f) -1})$ as well as $s_i = t_i$
for all $i < \Omega(f)$.

An important example of an $\Omega$--algebra is the so--called {\it
term algebra}. We choose an arbitrary set $X$ of symbols, which
must be disjoint from $F$. The signature is extended to $F \cup X$
such that the symbols of $X$ have arity 0. The terms over this new
signature are called $\Omega$--\textbf{terms over} $X$.
%%%
\index{term!$\Omega$--\faul}%%
%%%
The set of $\Omega$--terms over $X$ is denoted by
$\Tm_{\Omega}(X)$. Then we have $\Tm_{\Omega} = 
\Tm_{\Omega}(\varnothing)$. For many purposes (indeed 
most of the purposes of this book) the terms $\Tm_{\Omega}$ 
are sufficient. For we can always resort to the following trick. 
For each $x \in X$ add a 0--ary function symbol $\uli{x}$ to $F$. 
This gives a new signature $\Omega_X$, also called the 
%%%
\index{signature!constant expansion}%%%
%%%
\textbf{constant expansion of} $\Omega$ by $X$. Then 
$\Tm_{\Omega_X}$ can be canonically identified with 
$\Tm_{\Omega}(X)$.

There is an algebra which has as its objects the terms and 
which interprets the function symbols as follows.
%%
\begin{equation}
\Pi(f) \colon \auf t_i : i < \Omega(f)\zu \mapsto
    f(t_0, \dotsc, t_{\Omega(f)-1}) 
\end{equation}
%%
\index{$\goth{Tm}_{\Omega}(X)$}%%
%%%
Then $\goth{Tm}_{\Omega}(X) := \auf \Tm_{\Omega}(X), \Pi\zu$ is
%%%
\index{term algebra}%%
%%%
an $\Omega$--algebra, called the \textbf{term algebra generated by}
$X$. It has the following property. For any $\Omega$--algebra
$\GA$ and any map $v \colon X \pf A$ there is exactly one homomorphism
$\oli{v} \colon \Tm_{\Omega}(X) \pf \GA$ such that $\oli{v}
\restriction X = v$. This will be restated in 
Proposition~\ref{prop:freegen}.
%%%
\begin{defn}
Let $\GA$ be an $\Omega$--algebra and $X \subseteq A$. We say that
$X$ \textbf{generates} $\GA$ if $A$ is the smallest subset which
contains $X$ and which is closed under all functions $f^{\GA}$. If
$|X| = \kappa$ we say that $\GA$ is $\kappa$--\textbf{generated}.
Let $\CK$ be a class of $\Omega$--algebras and $\GA \in \CK$. We
say that $\GA$ is \textbf{freely generated by $X$ in $\CK$} if for 
every $\GB \in \CK$ and maps $v \colon X \pf B$ there is exactly 
one homomorphism $\oli{v} \colon \GA \pf \GB$ such that $\oli{v} 
\restriction X = v$. If $|X| = \kappa$ we say that $\GA$ is 
%%%%
\index{algebra!freeely ($\kappa$-)generated}%%%
%%%%
\textbf{freely $\kappa$--generated in} $\CK$.
\end{defn}
%%%
\begin{prop}
\label{prop:freegen}
Let $\Omega$ be a signature, and let $X$ be disjoint from $F$. Then
the term algebra over $X$, $\goth{Tm}_{\Omega}(X)$, is freely generated
by $X$ in the class of all $\Omega$--algebras.
\end{prop}
%%%
The following is left as an exercise. It is the justification
for writing $\goth{Fr}_{\CK}(\kappa)$ for the (up to isomorphism
unique) freely $\kappa$--generated algebra of $\CK$. In varieties 
such an algebra always exists.
%%%
\begin{prop}
\label{prop:free}
Let $\CK$ be a class of $\Omega$--algebras and $\kappa$ a
cardinal number. If $\GA$ and $\GB$ are both freely $\kappa$--generated
in $\CK$ they are isomorphic.
\end{prop}
%%%
Maps of the form $\sigma \colon X \pf \Tm_{\Omega}(X)$,
as well as their homomorphic extensions are called
%%%%
\index{substitutions}%%
%%%
\textbf{substitutions}. If $t$ is a term over $X$, we also write
$\sigma(t)$ in place of $\oli{\sigma}(t)$. Another notation,
frequently employed in this book, is as follows. Given terms
$s_i$, $i < n$, we write $[s_i/x_i \colon i < n]t$ in place of
$\sigma(t)$, where $\sigma$ is defined as follows.
%%%
\begin{equation}
\sigma(y) := \begin{cases}
    s_i & \text{ if $y = x_i$,} \\
    y   & \text{ else.}
\end{cases}
\end{equation}
%%%
(Most authors write $t[s_i/x_i \colon i < n]$, but this notation will
cause confusion with other notation that we use.)

Terms induce term functions on a given $\Omega$--algebra $\GA$.
Let $t$ be a term with variables $x_i$, $i < n$. (None of these
variables has to occur in the term.) Then $t^{\GA} \colon A^n \pf A$ is
defined inductively as follows (with $\vec{a} = \auf a_i : i <
\Omega(f)\zu$).
%%%
\index{term function}%%%
\index{term function!clone of}%%%
%%%
\begin{align}
x_i^{\GA}(\vec{a}) & := a_i \\\notag
(f(t_0, \dotsc, t_{\Omega(f)-1}))^{\GA}
    (\vec{a}) & := f^{\GA}(t_0^{\GA}(\vec{a}), \dotsc,
    t_{\Omega(f)-1}^{\GA}(\vec{a}))
\end{align}
%%
We denote by $\Clo_n(\GA)$ the set of $n$--ary term
functions on $\GA$. This set is also called the \textbf{clone of}
$n$--\textbf{ary term functions of} $\GA$. 
%%%
\index{polynomial}%%
%%%
A \textbf{polynomial of}
$\GA$ is a term function over an algebra that is like $\GA$ but
additionally has a constant for each element of $A$. (So, we form
the constant expansion of the signature with every $a \in A$.
Moreover, $\uli{a}$ (more exactly, $\uli{a}()$) shall have value
$a$ in $A$.) The clone of $n$--ary term functions of this algebra 
is denoted by $\Pol_n(\GA)$. For example, $((x_0 + x_1) 
\cdot x_0)$ is a term and
denotes a binary term function in an algebra for the signature
containing only $\cdot$ and $+$. However, $(2 + (x_0 \cdot x_0))$
is a polynomial but not a term. Suppose that we add a constant
\textbf{1} to the signature, with denotation 1 in the natural numbers.
Then $(2 + (x_0 \cdot x_0))$ is still not a term of the expanded
language (it lacks the symbol $2$), but the associated function
actually is a term function, since it is identical with the function
induced by the term $((\mbox{\bf 1}+\mbox{\bf 1}) + (x_0 \cdot x_0))$.
%%
\begin{defn}
%%%
\index{homomorphism}%%
\index{isomorphism}%%
\index{automorphism}%%
\index{endomorphism}%%
%%%
Let $\GA = \auf A, \{f^{\GA} \colon f \in F\}\zu$ and
$\GB = \auf B, \{f^{\GB} \colon f \in F\}\zu$ be $\Omega$--algebras
and $h \colon A \pf B$. $h$ is called a \textbf{homomorphism}
if for every $f \in F$ and every $\Omega(f)$--tuple
$\vec{x} \in A^{\Omega(f)}$ we have
%%
\begin{equation}
h(f^{\GA}(\vec{x})) = f^{\GB}(h(x_0), h(x_1), \dotsc, h(x_{\Omega(f)-1}))
\end{equation}
%%
We write $h \colon \GA \pf \GB$ if $h$ is a homomorphism from
$\GA$ to $\GB$. Further, we write $h \colon \GA \epi \GB$ if $h$
is a surjective homomorphism and $h \colon \GA \mono \GB$ if $h$
is an injective homomorphism. $h$ is an \textbf{isomorphism}
if $h$ is injective as well as surjective. $\GB$ is called
\textbf{isomorphic} to $\GA$, in symbols $\GA \cong \GB$ if there 
is an isomorphism from $\GA$ to $\GB$. If $\GA = \GB$
we call $h$ an \textbf{endomorphism of} $\GA$; if $h$ is 
additionally bijective then $h$ is called an
\textbf{automorphism} of $\GA$.
\end{defn}
%%
If $h \colon A \pf B$ is an isomorphism from $\GA$ to $\GB$ then
$h^{-1} \colon B \pf A$ is an isomorphism from $\GB$ to $\GA$.
%%
\begin{defn}
%%%
\index{congruence relation}%%
%%%
Let $\GA$ be an $\Omega$--algebra and $\Theta$ a
binary relation on $A$. $\Theta$ is called a
\textbf{congruence relation on} $\GA$ if $\Theta$ is an
equivalence relation and for all $f \in F$ and
all $\vec{x}, \vec{y} \in A^{\Omega(f)}$ we have:
%%
\begin{equation}
\label{eq:congr}
\text{If $x_i\, \Theta\, y_i$ for all $i < \Omega(f)$
then $f^{\GA}(\vec{x})\, \Theta\, f^{\GA}(\vec{y})$.}
\end{equation}
%%
\end{defn}
%%
We also write $\vec{x}\; \Theta\; \vec{y}$ in place of `$x_i \; \Theta \; 
y_i$ for all $i < \Omega(f)$'. If $\Theta$ is an equivalence relation put
%%%
\index{$[x]\Theta$}%%%
%%%
\begin{equation}
[x]\Theta := \{y : x \, \Theta\, y\} 
\end{equation}
%%
%%%
\index{equivalence class}%%
%%%
We call $[x]\Theta$ the \textbf{equivalence class of}
$x$. Then for all $x$ and $y$ we have either $[x]\Theta = [y]\Theta$
or $[x]\Theta \cap [y]\Theta = \varnothing$. Further,
we always have $x \in [x]\Theta$. If $\Theta$ additionally is
a congruence relation then the following holds: if
$y_i \in [x_i]\Theta$ for all $i < \Omega(f)$ then
$f^{\GA}(\vec{y}) \in [f^{\GA}(\vec{x})]\Theta$. Therefore
the following definition is independent of representatives.
%%%%
\begin{equation}
[f^{\GA}]\Theta([x_0]\Theta, [x_1]\Theta, \dotsc,
[x_{\Omega(f)-1}]\Theta)  
:= [f^{\GA}(\vec{x})]\Theta 
\end{equation}
%%
Namely, let $y_0 \in [x_0]\Theta,\dotsc, y_{\Omega(f)-1} \in 
[x_{\Omega(f)-1}]\Theta$. Then $y_i \; \Theta\; x_i$ for all 
$i < \Omega(f)$.  Then because of \eqref{eq:congr} we immediately
have $f^{\GA}(\vec{y}) \; \Theta\; f^{\GA}(\vec{x})$.
This simply means $f^{\GA}(\vec{y}) \in [f^{\GA}(\vec{x})]\Theta$.
Put 
%%%%
\index{$\GA/\Theta$}%%%%
%%%%
\begin{align}
A/\Theta & := \{[x]\Theta : x \in A\} \\
 \GA/\Theta & := \auf A/\Theta, \{[f^{\GA}]\Theta : f \in F\}\zu
\end{align}
%%%%
\index{factorization}%%%
%%%%
We call $\GA/\Theta$ the \textbf{factorization of} $\GA$ \textbf{by}
$\Theta$. The map $h_{\Theta} \colon x \mapsto [x]\Theta$ is easily
proved to be a homomorphism.

Conversely, let $h \colon \GA \pf \GB$ be a homomorphism.
Then put 
%%%
\index{$\ker$}%%%
\begin{equation}
\ker(h) := \{\auf x, y\zu \in A^2 :  h(x) = h(y)\}
\end{equation}
%%%
$\ker(h)$ is a congruence relation on $\GA$. Furthermore,
$\GA/\ker(h)$ is isomorphic to $\GB$ if $h$ is surjective. 
A set $B \subseteq A$ is \textbf{closed under} $f \in F$ if for 
all $\vec{x} \in B^{\Omega(f)}$ we have $f^{\GA}(\vec{x}) \in B$.
%%
\begin{defn}
%%%
\index{subalgebra}%%
%%%
Let $\auf A, \{f^{\GA} : f \in F\}\zu$ be an
$\Omega$--algebra and $B \subseteq A$ closed under all
$f \in F$. Put $f^{\GB} := f^{\GA} \restriction B^{\Omega(f)}$.
The pair $\auf B, \{f^{\GB} : f \in F\}\zu$ is called a
\textbf{subalgebra} of $\GA$.
\end{defn}
%%
The product of the algebras $\GA_i$, $i \in I$, is defined as 
follows.  The carrier set is the set of functions $\alpha \colon I % 
\pf \bigcup_{i \in I} A_i$ such that $\alpha(i) \in A_i$ for all 
$i \in I$. Call this set $P$. For an $n$--ary function symbol $f$ 
put
%%%
\index{algebra!product}%%
\index{product of algebras}%%
%%
\begin{multline}
f^{\GP}(\alpha_0, \dotsc, \alpha_{n-1})(i)  \\
      := \auf f^{\GA_i}(\alpha_0(i)), f^{\GA_i}(\alpha_1(i)),
    \dotsc, f^{\GA_i}(\alpha_{n-1}(i))\zu
\end{multline}
%%
The resulting algebra is denoted by $\prod_{i \in I} \GA_i$.
One also defines the product $\GA \times \GB$ in the following
way. The carrier set is $A \times B$ and for an $n$--ary function
symbol $f$ we put
%%
\begin{multline}
f^{\GA\times\GB}(\auf a_0,b_0\zu, \dotsc, \auf a_{n-1}, b_{n-1}\zu)
    \\
    := \auf f^{\GA}(a_0, \dotsc, a_{n-1}), f^{\GB}(b_0, \dotsc,
    b_{n-1})\zu
\end{multline}
%%
The algebra $\GA\times \GB$ is isomorphic to the algebra
$\prod_{i \in 2} \GA_i$, where $\GA_0 := \GA$, $\GA_1 := \GB$.
However, the two algebras are not identical. (Can you verify
this?)

A particularly important concept is that of a {\it variety}
or {\it equationally definable class of algebras}.
%%%
\begin{defn}
%%%
\index{variety}%%
%%%
Let $\Omega$ be a signature. A class of $\Omega$--algebras is
called a \textbf{variety} if it is closed under isomorphic copies,
subalgebras, homomorphic images, and (possibly infinite)
products.
\end{defn}
%%%
Let $V := \{x_i : i \in \omega\}$ be the set of variables.
%%%%
\index{equation}%%%
%%%%
An \textbf{equation} is a pair $\auf s, t\zu$ of $\Omega$--terms
(involving variables from $V$). We introduce a formal symbol 
`$\boldsymbol{\doteq}$' and write $s \boldsymbol{\doteq} t$ 
for this pair. An algebra $\GA$ satisfies the equation 
$s \boldsymbol{\doteq} t$ iff for all maps $v : V \pf A$, 
$\oli{v}(s) = \oli{v}(t)$. We then write 
$\GA \vDash s \boldsymbol{\doteq} t$. A class $\CK$ of 
$\Omega$--algebras satisfies this equation if every algebra of 
$\CK$ satisfies it.  We write $\CK \vDash s \boldsymbol{\doteq} t$.
%%%
\begin{prop}
\label{prop:eqcalc}
The following holds for all classes $\CK$ of $\Omega$--algebras.
%%
\begin{dingautolist}{192}
\item $\CK \vDash s \boldsymbol{\doteq} s$.
\item If $\CK \vDash s \boldsymbol{\doteq} t$ then 
$\CK \vDash t \boldsymbol{\doteq} s$.
\item If $\CK \vDash s \boldsymbol{\doteq} t; t \boldsymbol{\doteq} u$ 
	then $\CK \vDash s \boldsymbol{\doteq} u$.
\item If $\CK \vDash s_i \boldsymbol{\doteq} t_i$ for all $i < \Omega(f)$ 
	then $\CK \vDash f(\vec{s}) \boldsymbol{\doteq} f(\vec{t})$.
\item If $\CK \vDash s \boldsymbol{\doteq} t$ and $\sigma \colon V \pf
    \Tm_{\Omega}(V)$ is a substitution, then
    $\CK \vDash \sigma(s) \boldsymbol{\doteq} \sigma(t)$.
\end{dingautolist}
%%%
\end{prop}
%%%
The verification of this is routine. It follows from the first three
facts that equality is an equivalence relation on the algebra
$\goth{Tm}_{\Omega}(V)$, and together with the fourth that
the set of equations valid in $\CK$ form a congruence
on $\goth{Tm}_{\Omega}(V)$. There is a bit more we can say. Call
%%%
\index{congruence!fully invariant}%%
%%%
a congruence $\Theta$ on $\GA$ \textbf{fully invariant} if for all
endomorphisms $h \colon \GA \pf \GA$: if $x \; \Theta\; y$ then
$h(x) \; \Theta\; h(y)$. The next theorem follows immediately
once we observe that the endomorphisms of $\goth{Tm}_{\Omega}(V)$
are exactly the substitution maps. To this end, let $h\colon 
\goth{Tm}_{\Omega}(V) \pf \goth{Tm}_{\Omega}(V)$. Then
$h$ is uniquely determined by $h \restriction V$, since
$\goth{Tm}_{\Omega}(V)$ is freely generated by $V$. It is easily
computed that $h$ is the substitution defined by $h \restriction V$.
Moreover, every map $v \colon V \pf \goth{Tm}_{\Omega}(V)$ induces
a homomorphism $\oli{v} \colon \goth{Tm}_{\Omega}(V) \pf
\goth{Tm}_{\Omega}(V)$, which is unique. Now write $\mbox{\rm Eq}(\CK) :=
\{\auf s, t\zu : \CK \vDash s \boldsymbol{\doteq} t\}$.
%%%
\begin{cor}
Let $\CK$ be a class of $\Omega$--algebras. Then $\mbox{\rm Eq}(\CK)$
is a fully invariant congruence on $\goth{Tm}_{\Omega}(V)$.
\end{cor}
%%%
Let $E$ be a set of equations. Then put
%%
\begin{equation}
%%%
\index{$\mathsf{Alg}(E)$}%%
%%%
\mathsf{Alg}(E) := \{\GA : \text{ for all }
\auf s, t\zu \in E:  \GA \vDash s \boldsymbol{\doteq} t\}
\end{equation}
%%
This is a class of $\Omega$--algebras. Classes of $\Omega$--algebras
that have the form $\mathsf{Alg}(E)$ for some $E$ are
%%%%
\index{equationally definable class}%%%
%%%
called \textbf{equationally definable}. 
%%%
\begin{prop}
Let $E$ be a set of equations. Then $\mathsf{Alg}(E)$ is
a variety.
\end{prop}
%%%
We state without proof the following result.
%%%
\begin{thm}[Birkhoff]
Every variety is an equationally definable class. Furthermore,
there is a biunique correspondence between varieties and fully
invariant congruences on the algebra $\goth{Tm}_{\Omega}(\aleph_0)$.
\end{thm}
%%%
The idea for the proof is as follows. It can be shown that every
variety has free algebras. For every cardinal number $\kappa$,
$\goth{Fr}_{\CK}(\kappa)$ exists. Moreover, a variety is uniquely
characterized by $\goth{Fr}_{\CK}(\aleph_0)$. In fact, every
algebra is a subalgebra of a direct image of some product of
$\goth{Fr}_{\CK}(\aleph_0)$. Thus, we need to investigate the
equations that hold in the latter algebra. The other algebras will
satisfy these equations, too. The free algebra is the image of
$\goth{Tm}_{\Omega}(V)$ under the map $x_i \mapsto i$. The induced
congruence is fully invariant, by the freeness of
$\goth{Fr}_{\CK}(\aleph_0)$. Hence, this congruence simply {\it is\/}
the set of equations valid in the free algebra, hence in the whole
variety. Finally, if $E$ is a set of equations, we write $E \vDash
t \boldsymbol{\doteq} u$ if $\GA \vDash t \boldsymbol{\doteq} u$ 
for all $\GA \in \mathsf{Alg}(E)$.
%%%
\begin{thm}[Birkhoff]
$E \vDash t \boldsymbol{\doteq} u$ iff $t \boldsymbol{\doteq} u$ 
can be derived from $E$ by means of the rules given in
Proposition~\ref{prop:eqcalc}.
\end{thm}

The notion of an algebra can be extended into two directions, both
of which shall be relevant for us. The first is the concept of a
many--sorted algebra.
%%%
\begin{defn}
\index{signature!sorted}%%
\index{sort}%%
A \textbf{sorted signature} is a triple $\auf F, \CS, \Omega\zu$,
where $F$ and $\CS$ are sets, the set of \textbf{function symbols}
and of \textbf{sorts}, respectively, and $\Omega \colon F \pf \CS^+$ a
function assigning to each element of $F$ its so--called 
\textbf{signature}. We shall denote the signature by the letter 
$\Omega$, as in the unsorted case.%%
\end{defn}%%
%%
So, the signature of a function is a (nonempty) sequence of sorts.
The last member of that sequence tells us what sort the result
has, while the others tell us what sort the individual arguments
of that function symbol have.
%%
\begin{defn}%%
%%%
\index{algebra!many--sorted}%%
%%%
A (\textbf{sorted}) \textbf{$\Omega$--algebra} is a pair $\GA = \auf
\{A_{\sigma} : \sigma \in \CS\}, \Pi\zu$ such that for every
$\sigma \in \CS$ $A_{\sigma}$ is a set and for every $f \in F$
such that $\Omega(f) = \auf \sigma_i : i < n+1\zu$
%%%
\begin{equation}
\Pi(f) \colon A_{\sigma_0} \times A_{\sigma_1} \times \cdots \times
A_{\sigma_{n-1}} \pf A_{\sigma_n} 
\end{equation}
%%
If $\GB = \auf \{B_{\sigma} : \sigma \in \CS\}, \Sigma\zu$ is another 
%%%
\index{homomorphism!sorted}%%
%%%
$\Omega$--algebra, a (\textbf{sorted}) \textbf{homomorphism from} $\GA$ 
\textbf{to} $\GB$ is a set $\{h_{\sigma} \colon A_{\sigma} \pf B_{\sigma} 
: \sigma \in \CS\}$ of functions such that for each $f \in F$ with 
signature $\auf \sigma_i : i < n+1\zu$:
%%%
\begin{equation}
h_{\sigma_n}(f^{\GA}(a_0, \dotsc, a_{n-1})) =
f^{\GB}(h_{\sigma_0}(a_0), \dotsc, h_{\sigma_{n-1}}(a_{n-1}))
\end{equation}
%%%
A \textbf{many--sorted algebra} is an $\Omega$--algebra of some
signature $\Omega$.
\end{defn}
%%
Evidently, if $\CS = \{\sigma\}$ for some $\sigma$, then the
notions coincide (modulo trivial adaptations) with those of
unsorted algebras. Terms are defined as before, but now they are 
sorted. First, for each sort we assume a countably infinite set 
$V_{\sigma}$ of variables. Moreover, $V_{\sigma} \cap V_{\tau} 
= \varnothing$ whenever $\sigma \neq \tau$. Now, every term is 
given a unique sort in the following way.
%%%
\begin{dingautolist}{192}
\item If $x \in V_{\sigma}$, then $x$ has sort $\sigma$.
\item $f(t_0, \dotsc, t_{n-1})$ has sort $\sigma_n$, if
$\Omega(f) = \auf \sigma_i : i < n+1\zu$ and $t_i$ has sort 
$\sigma_i$ for all $i < n$.
\end{dingautolist}
%%
The set of terms over $V$ is denoted by $\Tm_{\Omega}(V)$. This can 
be turned into a sorted $\Omega$--algebra; simply let 
$\Tm_{\Omega}(V)_{\sigma}$
be the set of terms of sort $\sigma$. Again, given a map $v$ that
assigns to a variable of sort $\sigma$ an element of $A_{\sigma}$,
there is a unique homomorphism $\oli{v}$ from the
$\Omega$--algebra of terms into $\GA$. If $t$ has sort $\sigma$,
then $\oli{v}(t) \in A_{\sigma}$. A 
%%%
\index{equation!sorted}%%%
%%%
\textbf{sorted equation} is a pair
$\auf s, t\zu$, where $s$ and $t$ are of equal sort. We denote
this pair by $s \boldsymbol{\doteq} t$. We write $\GA \vDash s 
\boldsymbol{\doteq} t$ if for all maps $v$ into $\GA$, 
$\oli{v}(s) = \oli{v}(t)$. The Birkhoff Theorems have direct 
analogues for the many sorted algebras, and can be proved in the 
same way.

Sorted algebras are one way of introducing partiality. To be able
to compare the two approaches, we first have to introduce partial
algebras. We shall now return to the unsorted notions, although it
is possible --- even though not really desirable --- to introduce
partial many--sorted algebras as well.
%%%
\begin{defn}
Let $\Omega$ be an unsorted signature. A \textbf{partial
$\Omega$--algebra} is a pair $\auf A, \Pi\zu$, where $A$ is a set
and for each $f \in F$: $\Pi(f)$ is a partial function from
$A^{\Omega(f)}$ to $A$.
\end{defn}
%%%
The definitions of canonical terms split into different notions in
the partial case.
%%%
\begin{defn}
%%%
\index{homomorphism}%%
%%%
Let $\GA$ and $\GB$ be partial $\Omega$--algebras, and $h \colon A \pf
B$ a map. $h$ is a 
%%%
\index{homomorphism!weak}%%
%%%%
\textbf{weak homomorphism from} $\GA$ \textbf{to} $\GB$
if for every $\vec{a} \in A^{\Omega(f)}$ we have $h(f^{\GA}(\vec{a})) =
f^{\GB}(h(\vec{a}))$ if both sides are defined. $h$ is a 
\textbf{homomorphism} if it is a weak homomorphism and for every
$\vec{a}\in A^{\Omega(f)}$ if $h(f^{\GA}(\vec{a}))$ is defined
then so is $f^{\GB}(h(\vec{a}))$. Finally, $h$ is a
%%%
\index{homomorphism!strong}%%
%%%%
\textbf{strong homomorphism} if it is a homomorphism and
$h(f^{\GA}(\vec{a}))$ is defined iff $f^{\GB}(h(\vec{a}))$ is. 
$\GA$ is a 
%%%%
\index{subalgebra!strong}%%
%%%%
\textbf{strong subalgebra of} $\GB$ if $A \subseteq B$ 
and the identity map is a strong homomorphism.
\end{defn}
%%%
\begin{defn}
\label{defn:pcongruence}
An equivalence relation $\Theta$ on $A$ is called a 
%%%%
\index{congruence!weak}%%
%%%
\textbf{weak congruence of} $\GA$ if for every $f \in F$ and every 
$\vec{a}, \vec{c} \in A^{\Omega(f)}$ if $\vec{a}\; \Theta\; \vec{c}$ 
and $f^{\GA}(\vec{a})$, $f^{\GA}(\vec{c})$ are both
defined, then $f^{\GA}(\vec{a})\; \Theta\; f^{\GA}(\vec{c})$.
$\Theta$ is 
%%%%
\index{congruence!strong}%%
%%%%
\textbf{strong} if in addition $f^{\GA}(\vec{a})$
is defined iff $f^{\GA}(\vec{c})$ is.
\end{defn}
%%%
It can be shown that the equivalence relation induced by a weak (strong)
homomorphism is a weak (strong) congruence, and that every weak (strong)
congruence defines a surjective weak (strong) homomorphism.

%%%%
\index{$\vDash^w$, $\vDash^s$}%%
%%%%
Let $v \colon V \pf A$ be a function, $t = f(s_0, \dotsc, s_{\Omega(f)-1})$ 
a term. Then $\oli{v}(t)$ is defined iff 
(a) $\oli{v}(s_i)$ is defined for every $i < \Omega(f)$ and (b)
$f^{\GA}$ is defined on $\auf \oli{v}(s_i) : i < n\zu$. Now, we
write  $\auf \GA, v\zu \vDash^w s \boldsymbol{\doteq} t$ if $\oli{v}(s) =
\oli{v}(t)$ in case both are defined and equal; $\auf \GA, v\zu \vDash^s s
\boldsymbol{\doteq} t$ if $\oli{v}(s)$ is defined iff $\oli{v}(t)$
is and if one is defined the two are equal. An equation 
$s \boldsymbol{\doteq} t$ is said to hold in $\GA$ in the \textbf{weak} 
(\textbf{strong}) \textbf{sense}, if 
$\auf \GA, v\zu \vDash^w s \boldsymbol{\doteq} t$ 
($\auf \GA, v\zu \vDash^s s \boldsymbol{\doteq} t$) for all 
$v \colon V \pf A$. Proposition~\ref{prop:eqcalc}
holds with respect to $\vDash^s$ but not with respect to
$\vDash^w$. Also, algebras satisfying an equation in the strong
sense are closed under products, strong homomorphic images and
under strong subalgebras.

The relation between classes of algebras and sets of equations is
called a 
%%%%
\index{Galois correspondence}%%
%%%%
\textbf{Galois correspondence}. It is useful to know a few
facts about such correspondences. Let $A$, $B$ be sets and $R
\subseteq A \times B$ ($A$ and $B$ may in fact also be classes). 
The triple $\auf A, B, R\zu$ is called
%%%
\index{context}%%
%%%
a \textbf{context}. Now define the following operators:
%%
\begin{align}
^{\uparrow} \colon \wp(A) \pf \wp(B) \colon & O \mapsto
    \{y \in B : \mbox{ for all }x \in O: x\; R\; y\} \\
^{\downarrow} \colon \wp(B) \pf \wp(A) \colon & P \mapsto
    \{x \in A : \mbox{ for all }y \in P: x\; R\; y\}
\end{align}
%%
One calls $O^{\uparrow}$ the 
%%%
\index{intent}%%%
%%%%
\textbf{intent of} $O \subseteq A$ and $P^{\downarrow}$ the 
%%%
\index{extent}%%%
%%%%
\textbf{extent} of $P \subseteq B$.
%%%
\begin{thm}
Let $\auf A, B, R\zu$ be a context. Then the following holds
for all $O, O^{\ast} \subseteq A$ and all $P, P^{\ast} \subseteq B$.
%%
\begin{dingautolist}{192}
\item
$O \subseteq P^{\downarrow}$ iff $O^{\uparrow} \supseteq P$.
\item
If $O \subseteq O^{\ast}$ then $O^{\uparrow} \supseteq O^{\ast\uparrow}$.
\item
If $P \subseteq P^{\ast}$ then $P^{\downarrow} \supseteq P^{\ast\downarrow}$.
\item
$O \subseteq O^{\uparrow\downarrow}$.
\item
$P \subseteq P^{\downarrow\uparrow}$.
\end{dingautolist}
\end{thm}
%%%
\proofbeg 
Notice that if $\auf A, B, R\zu$ is a context, $\auf B,
A, R^{\smallsmile}\zu$ also is a context, and so we only need to show
\ding{192}, \ding{193} and \ding{195}. \ding{192}. $O \subseteq 
P^{\downarrow}$ iff every $x \in O$ stands in relation $R$ to every 
member of $P$ iff $P \subseteq O^{\uparrow}$. \ding{193}. If $O
\subseteq O^{\ast}$ and $y \in O^{\ast\uparrow}$, then for every $x \in
O^{\ast}$: $x\; R \; y$. This means that for every $x \in O$: $x\; R\;
y$, which is the same as $y \in O^{\uparrow}$. \ding{195}. Notice that
$O^{\uparrow} \supseteq O^{\uparrow}$ by \ding{192} implies $O \subseteq
O^{\uparrow\downarrow}$. \proofend
%%
\begin{defn}
%%%
\index{closure operator}%%
%%%
Let $M$ be a set and $H \colon \wp(M) \pf \wp(M)$ a function.
$H$ is called a \textbf{closure operator on} $M$ if
for all $X, Y \subseteq M$ the following holds.
%%
\begin{dingautolist}{192}
\item $X \subseteq H(X)$.
\item If $X \subseteq Y$ then $H(X) \subseteq H(Y)$.
\item $H(X) = H(H(X))$.
\end{dingautolist}
%%%
A set $X$ is called \textbf{closed} if $X = H(X)$.
\end{defn}
%%
\begin{prop}
\label{prop:closure} Let $\auf A, B, R\zu$ be a context. Then $O
\mapsto O^{\uparrow\downarrow}$ and $P \mapsto P^{\downarrow\uparrow}$ 
are closure operators on $A$ and $B$, respectively. The closed sets 
are the sets of the form $P^{\downarrow}$ for the first, and 
$O^{\uparrow}$ for the second operator.
\end{prop}
%%
\proofbeg
We have $O \subseteq O^{\uparrow\downarrow}$, from which
$O^{\uparrow} \supseteq O^{\uparrow\downarrow\uparrow}$. On
the other hand, $O^{\uparrow} \subseteq O^{\uparrow\downarrow\uparrow}$,
so that we get $O^{\uparrow} = O^{\uparrow\downarrow\uparrow}$.
Likewise, $P^{\downarrow} = P^{\downarrow\uparrow\downarrow}$ is
shown. The claims now follow easily.
\proofend
%%
\begin{defn}
\index{concept}%%
Let $\auf A, B, R\zu$ be a context. A pair $\auf O, P\zu
\in \wp(A) \times \wp(B)$ is called a \textbf{concept} if
$O = P^{\downarrow}$ and $P = O^{\uparrow}$.
\end{defn}
%%
\begin{thm}
Let $\auf A, B, R\zu$ be a context. The concepts are exactly the
pairs of the form $\auf P^{\downarrow}, P^{\downarrow\uparrow}\zu$,
$P \subseteq B$, or, alternatively, the pairs of the form
$\auf O^{\uparrow\downarrow}, O^{\uparrow}\zu$, $O \subseteq A$.
\end{thm}
%%%
As a particular application we look again at the connection
between classes of $\Omega$--algebras and sets of equations
over $\Omega$--terms. (It suffices to take the set of
$\Omega$--algebras of size $< \kappa$ for a suitable $\kappa$
to make this work.) Let $\mathsf{Alg}_{\Omega}$ denote the
class of $\Omega$--algebras, $\mbox{\sf Eq}_{\Omega}$ the
set of equations. The triple $\auf \mathsf{Alg}_{\Omega},
\mbox{\sf Eq}_{\Omega}, \vDash\zu$ is a context, and the
map $^{\uparrow}$ is nothing but $\mbox{\sf Eq}$ and the map
$^{\downarrow}$ nothing but $\mathsf{Alg}$. The classes
$\mathsf{Alg}(E)$ are the equationally definable classes,
$\mathsf{Eq}(\CK)$ the equations valid in $\CK$. Concepts
are pairs $\auf \CK, E\zu$ such that $\CK = \mathsf{Alg}(E)$
and $E = \mathsf{Eq}(\CK)$.

Often we shall deal with structures in which there are
also relations in addition to functions. The definitions,
insofar as they still make sense, are carried over
analogously. However, the notation becomes more clumsy.
%%
\begin{defn}
Let $F$ and $G$ be disjoint sets and $\Omega \colon F \pf
\omega$ as well as $\Xi \colon G \pf \omega$ functions. A
pair $\GA = \auf A, \GI\zu$ is called an $\auf \Omega,
\Xi\zu$--\textbf{structure} 
%%%%
\index{structure}%%
%%%%
if for all $f \in F$ $\GI(f)$
is an $\Omega(f)$--ary function on $A$ and for each
$g \in G$ $\GI(g)$ is a $\Xi(g)$--ary relation
on $A$. $\Omega$ is called the 
%%%
\index{signature!functional}%%%
\index{signature!relational}%%%
%%%
\textbf{functional signature}, $\Xi$ the \textbf{relational signature}
of $\GA$.
\end{defn}
%%
Whenever we can afford it we shall drop the qualification `$\auf
\Omega, \Xi\zu$' and simply talk of `structures'. If $\auf A,
\GI\zu$ is an $\auf \Omega, \Xi\zu$--structure, then $\auf A, \GI
\restriction F\zu$ is an $\Omega$--algebra. An $\Omega$--algebra
can be thought of in a natural way as a $\auf \Omega,
\varnothing\zu$--structure, where $\varnothing$ is the empty
relational signature. We use a convention similar to that of
algebras. Furthermore, we denote relations by upper case Roman
letters such as $R$, $S$ and so on. Let $\GA = \auf A,
\{f^{\GA} : f \in F\}, \{R^{\GA} : R \in G\}\zu$ and $\GB = \auf
B, \{f^{\GB} : f \in F\}, \{R^{\GB} : R \in G\}\zu$ be structures
of the same signature. A map $h \colon A \pf B$ is called an 
\textbf{isomorphism} from $\GA$ to $\GB$, if $h$ is bijective and for all
$f \in F$ and all $\vec{x} \in A^{\Omega(f)}$ we have
%%
\begin{equation}
h(f^{\GA}(\vec{x})) = f^{\GB}(h(\vec{x})) 
\end{equation}
%%
as well as for all $R \in G$ and all $\vec{x} \in A^{\Xi(R)}$
%%
\begin{equation}
R^{\GA}(\vec{x}) \quad\Dpf\quad
R^{\GB}(h(x_0), h(x_1), \dotsc, h(x_{\Xi(R)-1})) 
\end{equation}
%%%
\vplatz
\exercise
Since $y \mapsto \{y\}$ is an embedding of $x$ into $\wp(x)$, we 
have $|x| \leq |\wp(x)|$. Show that $|\wp(x)| > |x|$ for every set. 
{\it Hint.} Let $f : x \pf \wp(x)$ be any function. Look at the set 
$\{y : y \not\in f(y)\} \subseteq x$. Show that it is not in 
$\im(f)$.
%%%
\vplatz
\exercise
Let $f \colon M \pf N$ and $g \colon N \pf P$. Show that if 
$g \circ f$ is surjective, $g$ is surjective, and that if $g \circ f$ 
is injective, $f$ is injective. Give in each case an example that the 
converse fails.
%%%
\vplatz
\exercise
In set theory, one writes ${^N}M$ for the set of functions
from $N$ to $M$. Show that if $|N| = n$ and $|M| = m$, then
$|{^N}M| = m^n$. Deduce that $|{^N}M| = |M^n|$. Can you find
a bijection between these sets?
%%%
\vplatz
\exercise
Show that for relations $R, R' \subseteq M \times N$,
$S, S' \subseteq N \times P$ we have
%%%
\begin{subequations}
\begin{align}
(R \cup R') \circ S & = (R \circ S) \cup (R' \circ S) \\
R \circ (S \cup S') & = (R \circ S) \cup (R \circ S')
\end{align}
\end{subequations}
%%%
Show by giving an example that analogous laws for $\cap$ do not hold.
%%%
\vplatz 
\exercise 
Let $\GA$ and $\GB$ be $\Omega$--algebras for
some signature $\Omega$. Show that if $h \colon \GA \epi \GB$ is a
surjective homomorphism then $\GB$ is isomorphic to $\GA/\Theta$
with $x \; \Theta\; y$ iff $h(x) = h(y)$.
%%%
\vplatz
\exercise
Show that every $\Omega$--algebra $\GA$ is the homomorphic image
of a term algebra. {\it Hint.} Take $X$ to be the set underlying
$\GA$.
%%
\vplatz
\exercise
Show that $\GA\times\GB$ is isomorphic to $\prod_{i \in \{0,1\}} \GA_i$,
where $\GA_0 = \GA$, $\GA_1 = \GB$. Show also that $(\GA\times\GB)
\times \GC$ is isomorphic to $\GA \times (\GB \times \GC)$.
%%%
\vplatz
\exercise
Prove Proposition~\ref{prop:free}.

 \section{Semigroups and Strings}
\label{einseins}
\label{kap1-2}
%
%
%
In formal language theory, languages are sets of strings over some
alphabet. We assume throughout that an alphabet is a finite,
nonempty set, usually called $A$. It has no further structure (but
see Section~\ref{kap1}.\ref{kap1-3}), it only defines the material of
primitive letters. We do not make any further assumptions on the
size of $A$. The Latin alphabet consists of 26 letters, which
actually exist in two variants (upper and lower case), and we also
use a few punctuation marks and symbols as well as the blank. On
the other hand, the Chinese `alphabet' consists of several
thousand letters!

Strings are very fundamental structures. Without a proper
understanding of their workings one could not read this
book, for example. A string over $A$ is nothing but the result
of successively placing elements of $A$ after each other. It
is not necessary to always use a fresh letter. If, for example,
$A = \{\mbox{\tt a}, \mbox{\tt b}, \mbox{\tt c}, \mbox{\tt d}\}$,
then {\tt abb}, {\tt bac}, {\tt caaba} are strings over $A$.
We agree to use typewriter font to mark actual symbols (= pieces
of ink), while letters in different font are only proxy for
letters (technically, they are variables for letters). Strings
are denoted by a vector arrow, for example $\vec{w}$, $\vec{x}$,
$\vec{y}$ and so on, to distinguish them from individual letters.
Since paper is of bounded length, strings are not really written
down in a continuous line, but rather in several lines, and
on several pieces of paper, depending on need. The way a string is
cut up into lines and pages is actually immaterial for its
abstract constitution (unless we speak of paragraphs and similar
textual divisions). We wish to abstract from these details.
Therefore we define strings formally as follows.
%%
\begin{defn}
%%
\index{string}
\index{segment}
\index{letter}
\index{alphabet}
%%
Let $A$ be a set. A \textbf{string over} $A$ is a function
$\vec{x} \colon n \pf A$ for some natural number $n$. $n$ is called
the \textbf{length of} 
%%%%
\index{string!length}%%%
\index{$"|\vec{x}"|$}%%%
%%%%
$\vec{x}$ and is denoted by $|\vec{x}|$.
$\vec{x}(i)$, $i < n$, is called the $i$\textbf{th segment} or
the $i$\textbf{th letter of} $\vec{x}$. The unique string
of length 0 is denoted by $\varepsilon$. If $\vec{x} \colon m \pf A$ and
$\vec{y} \colon n \pf A$ are strings over $A$ then 
$\vec{x}\conc\vec{y}$ denotes the unique string of 
length $m+n$ for which the following holds:
%%
\index{$\vec{x}\conc\vec{y}$, $\varepsilon$}%%%
%%%
\begin{equation}
(\vec{x}\conc\vec{y})(j) := 
\begin{cases}
    \vec{x}(j) & \text{ if $j < m$,}  \\
    \vec{y}(j - m) & \text{ else.}
\end{cases}
\end{equation}
    %%
We often write $\vec{x}\, \vec{y}$ in place of $\vec{x}\conc\vec{y}$.
In connection with this definition the set $A$ is called the
\textbf{alphabet}, an element of $A$ is also referred to as a 
\textbf{letter}. Unless stated otherwise, $A$ is finite and nonempty.
\end{defn}
%%
So, a string may also be written using simple concatenation. Hence
we have $\mbox{\tt abc} \conc \mbox{\tt baca} = \mbox{\tt abcbaca}$.
Note that there no blank is inserted between the two strings;
for the blank is a {\it letter}. We denote it by $\Box$. Two words
of a language are usually separated by a blank possibly using
additional punctuation marks. That the blank is a symbol is
felt more clearly when we use a typewriter. If we want to
have a blank, we need to press down a key in order to get it.
For purely formal reasons we have added the empty
string to the set of strings. It is not visible (unlike the
blank). Hence, we need a special symbol for it, which is
$\varepsilon$, in some other books also $\lambda$. We have
%%
\begin{equation}
\label{eqn:null}
\vec{x} \conc \varepsilon = \varepsilon \conc \vec{x} =
\vec{x}
\end{equation}
%%
\index{unit}%%
%%%
We say, the empty string is the 
\textbf{unit} with respect to concatenation. For any triple of
strings $\vec{x}$, $\vec{y}$ and $\vec{z}$ we have
%%
\begin{equation}
\label{eqn:eins}
\vec{x} \conc (\vec{y} \conc \vec{z}) =
    (\vec{x} \conc \vec{y}) \conc \vec{z}
\end{equation}
%%
\index{associativity}%%
%%%
We therefore say that concatenation, $\conc$, is \textbf{associative}.
More on that below. We define the notation
${\vec{x}\,}^i$ by induction on $i$.
%%
\index{${\vec{x}\,}^i$, $\prod_{i < n} \vec{x}_n$}%%%
%%%
\begin{equation}
\begin{split}
{\vec{x}\,}^0     & := \varepsilon \\
{\vec{x}\,}^{i+1} & := {{\vec{x}\,}^i}\conc \vec{x} 
\end{split}
\end{equation}
%%
Furthermore, we define $\prod_{i < n} \vec{x}_i$ as follows.
%%
\begin{equation}
\prod_{i < 0} \vec{x}_i := \varepsilon, \qquad
\prod_{i < n+1} := (\prod_{i < n} \vec{x}_i) \conc \vec{x}_n 
\end{equation}
%%
Note that the letter {\tt a} is technically distinct from the
string $\vec{x} \colon 1 \pf A \colon 0 \mapsto \mbox{\tt a}$. They are
nevertheless written in the same way, namely {\tt a}.
If $\vec{x}$ is a string over $A$ and $A \subseteq B$,
then $\vec{x}$ is a string over $B$. The set of all strings
over $A$ is denoted by $A^{\ast}$.

Let $<$ be a linear order on $A$. We define the so--called
\textbf{lexicographical ordering} (\textbf{with respect to} $<$)
as follows.
%%%
\index{ordering!lexicographical}%%
%%%
Put $\vec{x} <_L \vec{y}$ if there exist $\vec{u}$, $\vec{v}$ and
$\vec{w}$ as well as $a$ and $b$ such that $\vec{x} = \vec{u}
\conc a \conc \vec{v}$, $\vec{y} = \vec{u} \conc b \conc \vec{w}$
and $a < b$. Notice that $\vec{x} <_L \vec{y}$ can obtain even if
$\vec{x}$ is longer than $\vec{y}$. Another important ordering
is the following one. Let $\mu(a) := k$ if $a$ is the $k$th symbol
of $A$ in the ordering $<$. Further, put $n := |A|$. For $\vec{x}
= x_0 x_1 \dotsb x_{p-1}$ we associate the following number.
%%
%%%
\index{$Z(\vec{x})$}%%%
%%%
\begin{equation}
Z(\vec{x}) := \sum_{i = 0}^{p-1} (\mu(x_i)+1) (n+1)^{p-i-1}
\end{equation}
%%
Now put $\vec{x} <_N \vec{y}$ if and only if $Z(\vec{x}) <
Z(\vec{y})$. This ordering we call the \textbf{numerical ordering}.
%%%
\index{ordering!numerical}%%
%%%%
Notice that both orderings depend on the choice of $<$. We shall 
illustrate these orderings with $A := \{\mbox{\tt a},\mbox{\tt b}\}$ 
and $\mbox{\tt a} < \mbox{\tt b}$. 
Then the numerical ordering is as follows.
%%
\begin{center}
\begin{tabular}{l|ccccccccccc}
$\vec{x}$ &
$\varepsilon$ & \mbox{\tt a} & \mbox{\tt b} & \mbox{\tt aa} &
\mbox{\tt ab} & \mbox{\tt ba} & \mbox{\tt bb} & \mbox{\tt aaa} &
\mbox{\tt aab} & \mbox{\tt aba} & $\dotsc$ \\\hline
$Z(\vec{x})$ &
0 & 1 & 2 & 4 & 5 & 7 & 8 & 13 & 14 & 16 & \\
\end{tabular}
\end{center}
%%
This ordering is linear. The map sending $i \in \omega$ to the
$i$th element in this sequence is known as the \textbf{dyadic
representation} of the numbers.
%%%
\index{dyadic representation}%%
%%%
In the dyadic representation, $0$ is represented by the
empty string, 1 by {\tt a}, 2 by {\tt b}, 3 by {\tt aa}
and so on. (Actually, if one wants to avoid using the empty
string here, one may start with {\tt a} instead.)

The lexicographical ordering is somewhat more complex. We
illustrate it for words with at most four letters.
%%
\begin{center}
\begin{tabular}{llllll}
$\varepsilon$, & {\tt a},    & {\tt aa},   & {\tt aaa},
    & {\tt aaaa}, & {\tt aaab}, \\
{\tt aab}, & {\tt aaba}, & {\tt aabb}, & {\tt ab},
    & {\tt aba},  & {\tt abaa}, \\
{\tt abab}, & {\tt abb},  & {\tt abba}, & {\tt abbb},
    & {\tt b},    & {\tt ba},  \\
{\tt baa},  & {\tt baaa}, & {\tt baab}, & {\tt bab},  
    & {\tt baba}, & {\tt babb}, \\
{\tt bb},  & {\tt bba},  & {\tt bbaa}, & {\tt bbab}, 
    & {\tt bbb}, & {\tt bbba},  \\
{\tt bbbb}
\end{tabular}
\end{center}
%%
In the lexicographical as well as the numerical ordering
$\varepsilon$ is the smallest element. Now look at the ordered
tree based on the set $A^{\ast}$. It is a tree in which every node 
is $n$--ary branching (cf.\ Section~\ref{kap1}.\ref{kap1-4}). Then the 
lexicographical ordering corresponds to the linearization obtained 
by depth--first search in this tree, while the numerical ordering 
corresponds to the linearization obtained by breadth--first search 
(see Section~\ref{kap2}.\ref{kap2-2}).
%%%
\begin{figure}
\begin{center}
\begin{picture}(24,13)
\put(12,11.5){\makebox(0,0){$\varepsilon$}}
    \put(12,11){\line(-1,-1){3}}
    \put(12,11){\line(1,-1){3}}
\put(9,7.5){\makebox(0,0){\tt a}}
    \put(9,7){\line(-2,-3){2}}
    \put(9,7){\line(2,-3){2}}
\put(15,7.5){\makebox(0,0){\tt b}}
    \put(15,7){\line(-2,-3){2}}
    \put(15,7){\line(2,-3){2}}
\put(7,3.5){\makebox(0,0){\tt aa}}
    \put(7,3){\line(-1,-4){.5}}
    \put(7,3){\line(1,-4){.5}}
    \put(7,0.5){\makebox(0,0){$\dotsc$}}
\put(11,3.5){\makebox(0,0){\tt ab}}
    \put(11,3){\line(-1,-4){.5}}
    \put(11,3){\line(1,-4){.5}}
    \put(11,0.5){\makebox(0,0){$\dotsc$}}
\put(13,3.5){\makebox(0,0){\tt ba}}
    \put(13,3){\line(-1,-4){.5}}
    \put(13,3){\line(1,-4){.5}}
    \put(13,0.5){\makebox(0,0){$\dotsc$}}
\put(17,3.5){\makebox(0,0){\tt bb}}
    \put(17,3){\line(-1,-4){.5}}
    \put(17,3){\line(1,-4){.5}}
    \put(17,0.5){\makebox(0,0){$\ldots$}}
\end{picture}
\end{center}
\caption{The Tree $A^{\ast}$}
\label{fig:praefix}
\end{figure}

%%
\index{monoid}%%
%%%
A \textbf{monoid} is a triple $\GM = \auf M, 1, \circ\zu$
where $\circ$ is a binary operation on $M$ and $1$ an element
such that for all $x, y, z \in M$ the following holds.
%%
\begin{subequations}
\begin{align}
x \circ 1 & = x \\
1 \circ x & = x\\
x \circ (y \circ z) & = (x \circ y) \circ z
\end{align}
\end{subequations}
%%
A monoid is therefore an algebra with signature 
$\Omega \colon 1 \mapsto 0, \cdot \mapsto 2$, which in addition
satisfies the above equations. An example is the algebra 
$\auf 4, 0, \max\zu$ (recall that $4 = \{0,1,2,3\}$), or 
$\auf \omega, 0, +\zu$. 
%%%
\index{$\GZ(A)$}%%
%%
\begin{prop}
Let $\GZ(A) := \auf A^{\ast}, \varepsilon, \cdot\zu$. Then
$\GZ(A)$ is a monoid.
\end{prop}
%%
The function which assigns to each string 
its length is a homomorphism from $\GZ(A)$ onto the monoid $\auf
\omega, 0, +\zu$. It is surjective, since $A$ is always assumed to
be nonempty. $\GZ(A)$ are special monoids: 
%%%
\begin{prop}
The monoid $\GZ(A)$ is freely generated by $A$.
\end{prop}
%%
\proofbeg
Let $\GN = \auf N, 1, \circ\zu$ be a monoid and $v \colon A \pf N$ an
arbitrary map. Then we define a map $\oli{v}$ as follows.
%%
\begin{equation}
\begin{split}
\oli{v}(\varepsilon) & := 1 \\
\oli{v}(\vec{x} \conc a) & := \oli{v}(\vec{x}) \circ v(a)
\end{split}
\end{equation}
%%
This map is surely well defined. For the defining clauses are mutually
exclusive. Now we must show that this map is a
homomorphism. To this end, let $\vec{x}$ and $\vec{y}$ be words.
We shall show that
%%
\begin{equation}
\oli{v}(\vec{x} \conc \vec{y}) = \oli{v}(\vec{x}) \circ
    \oli{v}(\vec{y}) 
\end{equation}
%%
This will be established by induction on the length of $\vec{y}$.
If it is 0, the claim is evidently true. For we have $\vec{y} =
\varepsilon$, and hence $\oli{v}(\vec{x} \conc \vec{y}) =
    \oli{v}(\vec{x}) = \oli{v}(\vec{x}) \circ 1 =
    \oli{v}(\vec{x}) \circ \oli{v}(\vec{y})$.
Now let $|\vec{y}| > 0$. Then $\vec{y} = \vec{w} \conc a$
for some $a \in A$.
%%
\begin{equation}
\begin{split}
\oli{v}(\vec{x} \conc \vec{y}) & = \oli{v}(\vec{x} \conc \vec{w} \conc a) \\
               &  = \oli{v}(\vec{x} \conc \vec{w}) \circ v(a) \\
%    & \text{ by definition} 
               &  = (\oli{v}(\vec{x}) \circ \oli{v}(\vec{w})) \circ v(a) \\
%    & \text{ by induction hypothesis} 
               &  = \oli{v}(\vec{x}) \circ (\oli{v}(\vec{w}) \circ v(a)) \\
%    & \text{ since $\GN$ is a monoid} 
               &  = \oli{v}(\vec{x}) \circ \oli{v}(\vec{y}) \\
%    & \text{ by definition}
\end{split}
\end{equation}
%%
This shows the claim.
\proofend

The set $A$ is the only set that generates $\GZ(A)$ freely. For
a letter cannot be produced from anything longer than a letter.
The empty string is always dispensable, since it occurs anyway
in the signature. Hence any generating set must contain $A$, and
since $A$ generates $A^{\ast}$ it is the only minimal set that
does so. A non--minimal generating set can
never freely generate a monoid. For example, let
$X = \{\mbox{\tt a}, \mbox{\tt b}, \mbox{\tt bba}\}$. 
$X$ generates $\GZ(A)$, but it is not minimal. Hence it does not
generate $\GZ(A)$ freely. For example, let $v \colon \mbox{\tt a} \mapsto
\mbox{\tt a}, \mbox{\tt b} \mapsto \mbox{\tt b}, \mbox{\tt bba}
\mapsto \mbox{\tt a}$. Then there is no homomorphism that extends
$v$ to $A^{\ast}$. For then on the one hand
$\oli{v}(\mbox{\tt bba}) = \mbox{\tt a}$, on the other
$\oli{v}(\mbox{\tt bba}) = v(\mbox{\tt b}) \conc v(\mbox{\tt b})
\conc v(\mbox{\tt a}) = \mbox{\tt bba}$.

The fact that $A$ generates $\GZ(A)$ freely has various noteworthy
consequences. First, a homomorphism from $\GZ(A)$ into an
arbitrary monoid need only be fixed on $A$ in order to be defined.
Moreover, {\it any\/} such map can be extended to a homomorphism
into the target monoid. As a particular application we get that
every map $v \colon A \pf B^{\ast}$ can be extended to a homomorphism
from $\GZ(A)$ to $\GZ(B)$. Furthermore, we get the following result, 
which shows that the monoids $\GZ(A)$ are up to isomorphism the only 
freely generated monoids (allowing infinite alphabets). They reader 
may note that the proof works for algebras of any signature.
%%
\begin{thm}
Let $\GM = \auf M, \circ, 1\zu$ and $\GN = \auf N, \circ, 1\zu$
be freely generated mo\-no\-ids. Then either \ding{192} or \ding{193}
obtains.
%%
\renewcommand{\labelenumi}{(\alph{enumi})}
\begin{dingautolist}{192}
\item
There is an injective homomorphism $i \colon \GM \mono \GN$ and a
surjective homomorphism $h \colon \GN \epi \GM$ such that $h \circ i = 1_M$.
\item
There exists an injective homomorphism $i \colon \GN \mono \GM$ and a
surjective homomorphism $h \colon \GM \epi \GN$ such that $h \circ i = 1_N$.
\end{dingautolist}
\end{thm}
%%
\proofbeg Let $\GM$ be freely generated by $X$, $\GN$ freely
generated by $Y$. Then either $|X| \leq |Y|$ or $|Y| \leq |X|$.
Without loss of generality we assume the first. Then there is an
injective map $p \colon X \mono Y$ and a surjective map $q \colon Y \epi X$
such that $q \circ p = 1_X$. Since $X$ generates $\GM$ freely,
there is a homomorphism $\oli{p} \colon \GM \pf \GN$ with $\oli{p}
\restriction X = p$. Likewise, there is a homomorphism $\oli{q} \colon
\GN \pf \GM$ such that $\oli{q} \restriction Y = q$, since $\GN$
is freely generated by $Y$. The restriction of $\oli{q} \circ
\oli{p}$ to $X$ is the identity. (For if $x \in X$ then $\oli{q}
\circ \oli{p}(x) = \oli{q}(p(x)) = q(p(x)) = x$.) Since $X$ 
freely generates $\GM$, there is only one homomorphism
which extends $1_X$ on $\GM$ and this is the identity. Hence
$\oli{q} \circ \oli{p} = 1_M$. It immediately follows that
$\oli{q}$ is surjective and $\oli{p}$ injective. Hence 
\ding{192} obtains. If $|Y| \leq |X|$ holds, \ding{193} is 
shown in the same way.
\proofend
%%
\begin{thm}
In $\GZ(A)$ the following cancellation laws hold.
%%
\begin{dingautolist}{192}
\item
If $\vec{x} \conc \vec{u} = \vec{y} \conc \vec{u}$,
then $\vec{x} = \vec{y}$.
\item
If $\vec{u} \conc \vec{x} = \vec{u} \conc \vec{y}$,
then $\vec{x} = \vec{y}$.
\end{dingautolist}
\end{thm}
%%
$\vec{x}^T$ is defined as follows.
%%
\begin{equation}
\left(\prod_{i < n} x_i\right)^T := \prod_{i < n} x_{n-1-i}
\end{equation}
%%
\index{$\vec{x}^T$}%%
\index{mirror string}%%
%%%
$\vec{x}^T$ is called the \textbf{mirror string} of $\vec{x}$. It is
easy to see that $(\vec{x}^T)^T = \vec{x}$. The reader is asked to 
convince himself that the map $\vec{x} \mapsto \vec{x}^T$ is 
{\it not\/} a homomorphism if $|A| > 1$.
%%
\begin{defn}
%%%
\index{prefix}%%
\index{postfix}%%
\index{suffix}%%
\index{substring}%%
%%%
Let $\vec{x}, \vec{y} \in A^{\ast}$. Then $\vec{x}$ is a
\textbf{prefix of} $\vec{y}$ if $\vec{y} = \vec{x} \conc
\vec{u}$ for some $\vec{u} \in A^{\ast}$. $\vec{x}$ is called a
\textbf{postfix} or \textbf{suffix of} $\vec{y}$ if $\vec{y} = \vec{u}
\conc \vec{x}$ for some $\vec{u} \in A^{\ast}$.
$\vec{x}$ is called a \textbf{substring of} $\vec{y}$ if
$\vec{y} = \vec{u} \conc \vec{x} \conc \vec{v}$ for some
$\vec{u}, \vec{v} \in A^{\ast}$.
\end{defn}
%%
It is easy to see that $\vec{x}$ is a prefix of $\vec{y}$
exactly if $\vec{x}^T$ is a postfix of $\vec{y}^T$.
Notice that a given string can have several occurrences
in another string. For example, {\tt aa} occurs four
times is {\tt aaaaa}. The occurrences are in addition
not always disjoint. An occurrence of $\vec{x}$ in
$\vec{y}$ can be defined in several ways. We may for
example assign {\it positions\/} to each letters.
%%%
\index{position}%%
%%%
In a string $x_0 x_1 \dotsc x_{n-1}$ the numbers $< n+1$
are called \textbf{positions}. The positions are actually thought
of as the spaces between the letters. The $i$th letter, $x_i$,
occurs between the position $i$ and the position $i+1$. The
substring $\prod_{i \leq j < k} x_i$ occurs between the positions
$i$ and $k$. The reason for doing it this way is that it allows
us to define occurrences of the empty string as well. For each
$i$, there is an occurrence of $\varepsilon$ between position
$i$ and position $i$. We may interpret positions as time points
in between which certain events take place, here the utterance
of a given sound. Another definition of an occurrence is via
the context in which the substring occurs.
%%%
\begin{defn}
%%%
\index{context}%%
\index{substring occurrence}%%
\index{substitution!string}%%
%%%
A \textbf{context} is a pair $C = \auf \vec{y}, \vec{z}\zu$ of 
strings. The  \textbf{substitution of} $\vec{x}$ \textbf{into} 
$C$, in symbols $C(\vec{x})$, is defined to be the string
$\vec{y} \conc \vec{x} \conc \vec{z}$. We say that
$\vec{x}$ \textbf{occurs in} $\vec{v}$ \textbf{in the
context} $C$ if $\vec{v} = C(\vec{x})$. Every occurrence
of $\vec{x}$ in a string $\vec{v}$ is uniquely defined by
its context. We call $C$ a \textbf{substring occurrence of}
$\vec{x}$ \textbf{in} $\vec{v}$.
%%%
\end{defn}
%%%
Actually, given $\vec{x}$ and $\vec{v}$, only one half of the
context defines the other. However, as will become clear, contexts
defined in this way allow for rather concise statements of facts
in many cases. Now consider two substring occurrences $C$, $D$ in
a given word $\vec{z}$. Then there are various ways in which the
substrings may be related with respect to each other.
%%
\begin{defn}
%%%
\index{substring occurrence!overlapping}%%
\index{substring occurrence!contained}%%
%%%
Let $C = \auf \vec{u}_1, \vec{u}_2\zu$ and $D = \auf \vec{v}_1,
\vec{v}_2\zu$ be occurrences in $\vec{z}$ of the strings $\vec{x}$ 
and $\vec{y}$, respectively. We say that $C$ 
%%%%
\index{precedence}%%%
%%%%
\textbf{precedes} $D$ 
if $\vec{u}_1\conc\vec{x}$ is a prefix of $\vec{v}_1$. $C$ and $D$ 
\textbf{overlap} if $C$ does not precede $D$ and $D$ does not precede 
$C$. $C$ \textbf{is contained in} $D$ if $\vec{v}_1$ is a prefix of 
$\vec{u}_1$ and $\vec{v}_2$ is a suffix of $\vec{u}_2$.
\end{defn}
%
Notice that if $\vec{x}$ is a substring of $\vec{y}$ then every
occurrence of $\vec{y}$ contains an occurrence of $\vec{x}$; but
not every occurrence of $\vec{x}$ is contained in a given
occurrence of $\vec{y}$.
%
\begin{defn}
%%%
\index{language}%%
\index{language!string}%%
%%%
\label{defn:sprache} %%
A (\textbf{string}) \textbf{language over the
alphabet} $A$ is a subset of $A^{\ast}$.
\end{defn}
%%
This definition admits that $L = \varnothing$ and that $L = A^{\ast}$.
Moreover, $\varepsilon \in L$ also may occur. The admission of
$\varepsilon$ is often done for technical reasons (like the
introduction of a zero). 
%%
\begin{thm}
Suppose $A$ is not empty, and $|A| \leq \aleph_0$. Then there are 
exactly $2^{\aleph_0}$ languages.
\end{thm}
%%%
\proofbeg
This is a standard counting argument. We establish that $|A^{\ast}| = 
\aleph_0$. The claim then follows since there are as many languages as 
there are subsets of $\aleph_0$, namely $2^{\aleph_0}$. 
If $A$ is finite, we can enumerate $A^{\ast}$ by enumerating the 
strings of length 0, the strings of length 1, the strings of length 2, 
and so on. If $A$ is infinite, we have to use cardinal arithmetic: 
the set of strings of length $k$ of any finite $k$ is countable, 
and $A^{\ast}$ is therefore the countable union of countable sets, 
again countable.
\proofend

One can prove the previous result directly using the following argument.
(The argument works even when $C$ is countably infinite.)
%%
\begin{thm}
\label{bijektion}
Let $C = \{c_i : i < p\}$, $p > 2$, be an arbitrary alphabet
and $A = \{\mbox{\tt a}, \mbox{\tt b}\}$.
Further, let $\oli{v}$ be the homomorphic extension of
$v \colon c_i \mapsto {\mbox{\tt a}^i}\conc \mbox{\tt b}$. The map
$S \mapsto \oli{v}[S] \colon \wp(C^{\ast}) \pf \wp(A^{\ast})$ defined by
$V(S) = \oli{v}[S]$ is a bijection between $\wp(C^{\ast})$ and
those languages which are contained in the direct image of
$\oli{v}$.
\end{thm}
%%
The proof is an exercise. The set of all languages over $A$ is closed
under $\cap$, $\cup$, and $-$, the relative complement with respect
to $A^{\ast}$.  Furthermore, we can define the following operations on
languages.
%%
\index{$L \cdot M$, $L^n$, $L^+$, $L^{\ast}$, $L/M$, $L\backslash M$}%%%%
%%
\begin{subequations}
\begin{align}
L \cdot M & := \{\vec{x} \conc \vec{y} : \vec{x} \in L, \vec{y} \in M\} \\
L^0      & := \{\varepsilon\}\\
L^{n+1}  & := L^n \cdot L\\
L^{\ast} & := \bigcup_{n \in \omega} L^n\\
L^+      & := \bigcup_{0 < n \in\omega} L^n \\
L / M & := \{\vec{y} \in A^{\ast} : (\exists \vec{x} \in M)(\vec{y}\conc
\vec{x} \in L)\} \\
M \backslash L & := \{\vec{y} \in A^{\ast} :
    (\exists \vec{x} \in M)(\vec{x}\conc \vec{y} \in L)\}
\end{align}
\end{subequations}
%%
%%%
\index{Kleene star}%%
%%%
$^{\ast}$ is called the \textbf{Kleene star}. For example,
$L/A^{\ast}$ is the set of all strings which can be extended
to members of $L$; this is exactly the set of prefixes of
members of $L$. We call this set the 
%%%%
\index{prefix closure}%%%
%%%
\textbf{prefix closure} of $L$, in symbols $L^P$.
%%%%
\index{$L^P$, $L^S$}%%
%%%%
Analogously, $L^S := A^{\ast}\backslash L$ is the
\textbf{suffix} or \textbf{postfix closure} 
%%%
\index{postfix closure}%%%
%%%
of $L$. It follows that $(L^P)^S$ is nothing but the substring
closure of $L$.

In what is to follow, we shall often encounter string languages 
with a special distinguished symbol, the \textbf{blank}, 
%%%%
\index{blank}%%%
\index{$\Box$, $\oconc$}%%%
%%%
typically written $\Box$. Then we use the abbreviation 
%%%
\begin{align}
\vec{x}\oconc\vec{y} & := \vec{x}\conc\Box\conc\vec{y} & 
L \oconc M & := \{\vec{x}\oconc\vec{y} : \vec{x} \in L, 
\vec{y} \in M\} 
\end{align}

Let $L$ be a language over $A$, $C = \auf \vec{x}, \vec{y}\zu$
a context and $\vec{u}$ a string. We say that $C$ \textbf{accepts}
$\vec{u}$ \textbf{in} $L$, and write $\vec{u} \dashv_L C$, if 
%%%%
\index{$\dashv_L$}%%
%%%%
$C(\vec{u}) \in L$. The triple 
$\auf A^{\ast}, A^{\ast} \times A^{\ast}, \dashv_L\zu$
is a context in the sense of the previous section. Let $M \subseteq
A^{\ast}$ and $P \subseteq A^{\ast} \times A^{\ast}$. Then denote
by $C_L(M)$ the set of all $C$ which accept all strings from $M$
in $L$ (intent); and denote by $Z_L(P)$ the set of all strings which
are accepted by all contexts from $P$ in $L$ (extent).
%%%%
\index{$C_L(M)$, $Z_L(P)$}%%
%%
%%%%%
\index{Sestier--closure}%
\index{Sestier--operator}%%
\nocite{sestier:contributions}%%
%%%%%
We call $M$ ($L$--)\textbf{closed} if $M = Z_L(C_L(M))$.
The closed sets form the so--called \textbf{distribution classes}
of strings in a language.
%%%%
\index{distribution classes}%%
%%%%%
$Z_L(C_L(M))$ is called the \textbf{Sestier--closure} of $M$ and the map
$S_L \colon M \mapsto Z_L(C_L(M))$ the \textbf{Sestier--operator}. From
Proposition~\ref{prop:closure} we immediately get this result.
%%
\begin{prop}
The Sestier--operator is a closure operator.
\end{prop}
%%

For various reasons, identifying terms with strings that
represent them is a dangerous affair. As is well--known,
conventions for writing down terms can be misleading, since 
they might be ambiguous. Therefore we defined the term as an
entity in itself. The string by which we denote the term is 
only as a representative of that term.
%%
\begin{defn}
%%
\index{representation}%%
\index{representative!unique}%%
\index{string!representing}%%
\index{readability!unique}%%
%%
Let $\Omega$ be a signature. A \textbf{representation of
terms} (\textbf{by means of strings over} $A$) is a relation
$R \subseteq \Tm_{\Omega} \times A^{\ast}$ such that
for each term $t$ there exists a string $\vec{x}$ with
$\auf t, \vec{x}\zu \in R$. $\vec{x}$ is called a
\textbf{representative} or \textbf{representing string}
of $t$ \textbf{with respect to} $R$. $\vec{x}$ is called 
\textbf{unambiguous} if from $\auf t, \vec{x}\zu, \auf u, 
\vec{x}\zu \in R$ it follows that $t = u$. $R$ is called 
\textbf{unique} or \textbf{uniquely readable} if every 
$\vec{x} \in A^{\ast}$ is unambiguous.
\end{defn}
%%
$R$ is uniquely readable iff it is an injective function
from $\Tm_{\Omega}$ to $A^{\ast}$ (and therefore its
converse a partial injective function).  We leave it to the reader
to verify that the representation defined in the previous section
is actually uniquely readable. This is not self evident. It could be
that a term possesses several representing strings. Our usual way of
denoting terms is in fact not uniquely readable.  For example, one 
writes ${\tt 2+3+4}$ even though this could be a representative of the 
term $+(+(2,3),4)$ or of the term $+(2,+(3,4))$. This hardly matters, 
since the two terms denote the same number, but nevertheless they are 
different terms.

There are many more conventions for writing down terms. We give a few
examples.  (a) A binary symbol is typically written in between its
arguments (this is called the \textbf{infix notation}).
%%%
\index{notation!infix}%%
%%%
So, we do not write {\tt +(2,3)} but {\tt (2+3)}. (b) Outermost
brackets may be omitted: {\tt (2+3)} denotes the same term as
{\tt 2+3}. (c) The multiplication sign binds stronger than {\tt +}.
So, the following strings all denote the same term.
%%
\begin{equation}
%%\begin{center}
\mbox{\mtt (2+(3\symbol{42}5))} \quad \mbox{\mtt 2+(3\symbol{42}5)} \quad
\mbox{\mtt (2+3\symbol{42}5)} \quad \mbox{\mtt 2+3\symbol{42}5}
%%\end{center}
\end{equation}
%%
In logic, it was customary to use dots in place of brackets.
In this notation, {\mtt p\symbol{4}q.\symbol{25}.p} means the same 
as the more common {\mtt (p\symbol{4}q)\symbol{25}p}. The dots are 
placed to the left or right (sometimes both) of the operation sign. 
Ambiguity is resolved by using more than one dot, for example `{\mtt :}'. 
(See \cite{curry:logic} on this notation.) Also, let $\circ$ 
be a binary operation symbol, written in infix notation. Suppose 
that $\ell$ defines a string for every term in the following way.
%%%
\begin{align}
\notag
\ell(x) & := x && \text{$x$ basic} \\
\ell(\circ(x,y)) & := \ell(x) \circ y 
	&& \text{$y$ basic}\\
\notag
\ell(\circ(x,t)) & := \ell(x) \circ \mbox{\mtt (}\ell(t)\mbox{\mtt )} 
	&& \text{$t$ complex}
\end{align}
%%
\index{associativity!left--\faul}%%
\index{associativity!right--\faul}%%
If $\ell(t)$ represents $t$, we say that $\circ$ is 
\textbf{left--associative}. If on the other hand $\rho(t)$ 
represents the term $t$, $\circ$ is said to be 
\textbf{right--associative}.
%%%
\begin{align}
\notag
\rho(y) & := x && \text{$y$ basic} \\
\rho(\circ(x,y)) & := x \circ \rho(y) 
	&& \text{$x$ basic}\\
\notag
\ell(\circ(t,y)) & := \mbox{\mtt (}\rho(t)\mbox{\mtt )} \circ \rho(y)
	&& \text{$t$ complex}
\end{align}
%%

Since the string {\mtt (2+3)\symbol{42}5} represents a different term
than {\mtt 2+3\symbol{42}5} (and both have a different value) the
brackets cannot be omitted. That we can do without brackets is an 
insight we owe to the Polish logician Jan {\L}ukasiewicz. In his 
notation, which is also called \textbf{Polish Notation} (\textbf{PN}), 
the function symbol is always placed in front of its arguments.
%%
\index{notation!Polish}%%
\index{notation!Reverse Polish}%%
%%%
Alternatively, the function symbol may be consistently placed
behind its arguments (this is the so--called \textbf{Reverse Polish
Notation}, \textbf{RPN}). There are some calculators (in addition to the
programming language FORTH) 
%%%
\index{FORTH}%%%
%%%%
which have implemented RPN. In place
of the (optional) brackets there is a key called  `{\tt enter}'.
It is needed to separate two successive operands. For in RPN,
the two arguments of a function follow each other immediately.
If nothing is put in between them, both the terms $+(13,5)$
and $+(1,35)$ would both be written {\tt 135+}. To prevent this,
`{\tt enter}' is used to separate the first from the second
input string. You therefore need to enter into the computer
{\tt 13\framebox{enter}5+}. (Here, the box is the usual way
in computer handbooks to turn a sequence into a `key'.
In Chapter~\ref{kap3} we shall deal again with the problem
of writing down numbers.) Notice that in practice (i.e.~as far 
as the tacit conventions go) the choice between Polish and Reverse 
Polish Notation only affects the position of the function symbol, and 
not the way in which arguments are placed with respect to each other. 
For example, suppose there is a key \framebox{\tt exp} for the 
exponential function. Then to get the result of $2^3$, you enter
\mbox{\tt 2\framebox{enter}3\framebox{exp}} on a machine using RPN  
and {\tt \framebox{exp}2\framebox{enter}3=} on a machine using PN. 
Hence, the relative order between base ({\tt 2}) and exponent 
({\tt 3}) remains. (Notice incidentally the need for typing in 
{\tt =} or something else that indicates the end of the second 
operand in PN!) This effect is also noted in natural languages: 
the subject precedes the object in the overwhelming majority of 
languages irrespective of the place of the verb. The mirror image 
of an VSO language is an SOV language, not OSV.

Now we shall show that Polish Notation is uniquely readable.
Let $F$ be a set of symbols and $\Omega$ a signature over $F$.
Each symbol $f \in F$ is assigned an arity $\Omega(f)$. Next, 
we define a set of strings over $F$, which we assign to the various 
%%%%
\index{$\PN_{\Omega}$}%%
%%%%
terms of $\Tm_{\Omega}$. $\PN_{\Omega}$ is the 
smallest set $M$ of strings over $F$ for which the following holds.
%%
\begin{center}
For all $f \in F$ and for all $\vec{x}_i \in M$,
    $i < \Omega(f)$: \\
        $f \conc \vec{x}_0 \conc \dotsb \conc \vec{x}_{\Omega(f)-1}
            \in M$.
\end{center}
%%
(Notice the special case $n = 0$. Further, notice that no special
treatment is needed for variables, by the remarks of the preceding
section.) This defines the set $\PN_{\Omega}$, members
of which are called \textbf{well--formed strings}. Next we
shall define which string represents which term.
The string `$f$', $\Omega(f) = 0$, represents the term `$f$'.
If $\vec{x}_i$ represents $t_i$,  $i < \Omega(f)$, then
$f \conc \vec{x}_0 \conc \dotsb \conc \vec{x}_{\Omega(f)-1}$
represents $f(t_0,\dotsc,t_{\Omega(f)-1})$. We shall
now show that this relation is bijective. (A different proof than 
the one used here can be found in Section~\ref{kap2}.\ref{kap2-4}, 
proof of Theorem~\ref{thm:pn}.) Here we use an important principle, 
namely induction over the length of the string. The following is 
for example proved by induction on $|\vec{x}|$.
%%
\begin{dingautolist}{192}
\item No proper prefix of $|\vec{x}|$ is a well--formed string.
\item If $\vec{x}$  is a well--formed string
then $\vec{x}$ has length at least 1 and the following holds.
%%
\begin{enumerate}
\item If $|\vec{x}| = 1$, then $\vec{x} = f$
    for some $f \in F$ with $\Omega(f) = 0$.
\item If $|\vec{x}| > 1$, then there are $f$ and $\vec{y}$ such that
    $\vec{x} = f \conc \vec{y}$, and $\vec{y}$ is the
    concatenation of exactly $\Omega(f)$ many uniquely
    defined well--formed strings.
\end{enumerate}
\end{dingautolist}
%%
The proof is as follows. Let $t$ and $u$ be terms represented by
$\vec{x}$. Let  $|\vec{x}| = 1$. Then $t = u = f$, for some $f \in F$ 
with $\Omega(f) = 0$. A proper prefix is the empty string, which is 
clearly not well formed.
Now for the induction step. Let $\vec{x}$ have length at least 2.
Then there is an $f \in F$ and a sequence $\vec{y}_i$, $i < \Omega(f)$,
of well--formed strings such that
%%
\begin{equation}
\label{eq:dagger}
\vec{x} = f \conc \vec{y}_0 \conc \dotsb \conc
    \vec{y}_{\Omega(f)-1} 
\end{equation}
    %%
Therefore for each $i < \Omega(f)$ there is a term $u_i$ 
represented by $\vec{y}_i$. By \ding{193}, the $u_i$ are
uniquely determined by the $\vec{y}_i$. Furthermore, the 
symbol $f$ is uniquely determined, too. Now let $\vec{z}_i$, 
$i < \Omega(f)$, be well--formed strings with
%%
\begin{equation}
\vec{x} = f \conc \vec{z}_0 \conc \dotsb \conc
    \vec{z}_{\Omega(f)-1}
\end{equation}
    %%
Then $\vec{y}_0 = \vec{z}_0$. For no proper prefix of $\vec{z}_0$
is a well--formed term, and no proper prefix of $\vec{y}_0$ is a
term. But they are prefixes of each other, so they cannot be
proper prefixes of each other, that is to say, they are equal. If
$\Omega(f) = 1$, we are done. Otherwise we carry on in the same
way, establishing by the same argument that $\vec{y}_1 =
\vec{z}_1$, $\vec{y}_2 = \vec{z}_2$, and so on. The fragmentation
of the string in $\Omega(f)$ many well--formed strings is
therefore unique. By inductive hypothesis, the individual strings
uniquely represent the terms $u_i$. So, $\vec{x}$ uniquely
represents the term $f(\vec{u})$.
This shows \ding{193}.

Finally, we shall establish \ding{192}. Look again at the 
decomposition \eqref{eq:dagger}. If $\vec{u}$ is a well--formed prefix, 
then $\vec{u} \neq \varepsilon$. Hence $\vec{u} = f \conc \vec{v}$ 
for some $\vec{v}$ which can be decomposed into $\Omega(f)$ many 
well--formed strings $\vec{w}_i$. As before we shall argue that 
$\vec{w}_i = \vec{x}_i$  for every $i < \Omega(f)$. Hence 
$\vec{u} = \vec{x}$, which shows that no proper prefix of $\vec{x}$ 
is well--formed.

{\it Notes on this section.} Throughout this book the policy is to 
regard any linguistic object as a string. Strings are considered 
the fundamental structures. This in itself is no philosophical 
commitment, just a matter of convenience. Moreover, when we refer 
to sentences qua material objects (signifiers) we take them to be 
strings over the Latin alphabet. This again is only a matter of 
convenience. Formal language theory very often treats words rather 
than letters as units. If one does so, their composite nature has 
to be ignored. Yet, while most arguments can still be performed 
(since a transducer can be used to switch between these 
representations), some subtleties can get lost in this abstraction. 
We should also point out that since alphabets must be finite, there 
can be no infinite set of variables as a primitive set of letters, 
as is often assumed in logic. 
%%
\vplatz
\exercise
%%
Prove Theorem~\ref{bijektion}.
%%
\vplatz
\exercise
(The `Typewriter Model'.) Fix an alphabet $A$. For each $a \in A$ 
assume a unary symbol $\mbox{\tt s}_a$. Finally, let {\tt 0} be 
a zeroary symbol. This defines the signature $\Psi$. Define a 
map $t : \Tm_{\Psi} \pf A^{\ast}$ as follows. $\tau(\mbox{\tt 0}) := 
\varepsilon$, and $\tau(\mbox{\tt s}_a(s)) := \tau(s)^{\smallfrown}a$.
Show that $\tau$ is bijective. Further, show that there is no 
term $u$ over $\Psi$ such that $\tau(u(x,y)) = 
\tau(x)^{\smallfrown}\tau(y)$, and not even a term 
$v_{\vec{x}}(y)$ such that $\tau(v_{\vec{x}}(y)) = \vec{x}^{\smallfrown}%
\tau(y)$, for any given $\vec{x} \in A^+$. On the other hand 
there {\it does\/} exist a $w_{\vec{y}}$ such that 
$\tau(w_{\vec{y}}(x)) = \tau(x)^{\smallfrown}\vec{y}$ for 
any given $\vec{y} \in A^{\ast}$.
%%%
\vplatz
\exercise
Put $Z^{\ast}(\vec{x}) := \sum_{i < p} \mu(x_i)n^{p-i-1}$. Now put
$\vec{x} <_{N^{\ast}} \vec{y}$ if and only if $Z^{\ast}(\vec{x})
< Z^{\ast}(\vec{y})$. Show that $<_{N^{\ast}}$ is transitive and
irreflexive, but not total.
%%
\vplatz
\exercise
%%
Show that the postfix relation is a partial ordering,
likewise the prefix and the subword relation. Show that
the subword relation is the transitive closure of the
union of the postfix relation with the prefix relation.
%%
\vplatz
\exercise
%%
Let $F$, $X$ and $\{\mbox{\tt (}, \mbox{\tt )}\}$ be three
pairwise disjoint sets, $\Omega$ a signature over $F$.
We define the following function from $\Omega$--terms into
strings over $F \cup X \cup \{\mbox{\tt (},
\mbox{\tt )}\}$:
%%
\begin{equation}
\begin{split}
x^+ & := x \\
f(t_0, \dotsc, t_{\Omega(f)-1})^+ & :=
    f\conc \mbox{\tt (} \conc t_0^+ \conc \dotsb \conc t_{\Omega(f)-1}^+
    \conc\mbox{\tt )}
\end{split}
\end{equation}
%%
(To be clear: we represent terms by the string that we have
used in Section~\ref{kap1}.\ref{kap1-1} already.) Prove the unique
readability of this notation. Notice that this does not already
follow from the fact that we have chosen this notation to
begin with. (We might just have been mistaken ...)
%%
\vplatz
\exercise
%%
Give an exact upper bound on the number of prefixes (postfixes)
of a given string of length $n$, $n$ a natural number. Also give
a bound for the number of subwords. What can you say about the
exactness of these bounds in individual cases?
%%
\vplatz
\exercise
\label{ex:kuerzdef}
%%
Let $L, M \subseteq A^{\ast}$. Define
%%
\index{$M\backslash\backslash L$, $L {//} M$}%%%
\begin{subequations}
\begin{align}
L {/\! /} M & :=
\{\vec{y} : (\forall \vec{x} \in M)(\vec{y} \conc \vec{x} \in L)\} \\
M {\backslash\!\backslash} L & :=
\{\vec{y} : (\forall \vec{x} \in M)(\vec{x} \conc \vec{y} \in L)\}
\end{align}
\end{subequations}
%%
Show the following for all $L, M, N \subseteq A^{\ast}$:
%%
\begin{equation}
M \subseteq L {\backslash\!\backslash} N
\quad\Leftrightarrow\quad
L \cdot M \subseteq N
\quad\Leftrightarrow\quad
L \subseteq N {/\! /} M
\end{equation}
%%
\vplatz
\exercise
Show that not all equivalences are valid if  in place of
${\backslash\! \backslash}$ and ${/\! /}$ we choose
$\backslash$ and $/$. Which implications remain valid, though?

 \section{Fundamentals of Linguistics}
\label{kap1-3}
%
%
%
In this section we shall say some words about our conception of
language and introduce some linguistic terminology. Since we
cannot define all the linguistic terms we are using, this section 
is more or less meant to get those readers acquainted with the 
basic linguistic terminology who wish to read the book without going 
through an introduction into linguistics proper. (However, it is 
recommended to have such a book at hand.)

A central tool in linguistics is that of postulating abstract
units and hierarchization. Language is thought to be more than a
mere relation between sounds and meanings. In between the two
realms we find a rather rich architecture that hardly exists in 
formal languages. This architecture is most clearly
articulated in \cite{harris:structural} and also
\cite{lamb:stratificationalism}. Even though linguists might
disagree with many details, this basic architecture is assumed even
in most current linguistic theories. We shall outline what we
think is minimal consensus.
%%%
\begin{figure}
\begin{center}
\begin{picture}(20,15)
\put(10,1){\makebox(0,0){Phonological Stratum}}
\put(10,1.6){\line(0,1){2.9}}
\put(10,5){\makebox(0,0){Morphological Stratum}}
\put(10,5.7){\line(0,1){2.8}}
\put(10,9){\makebox(0,0){Syntactical Stratum}}
\put(10,9.6){\line(0,1){2.9}}
\put(10,13){\makebox(0,0){Semantical Stratum}}
\end{picture}
\end{center}
\caption{The Strata of Language}
\label{fig:strata}
\end{figure}
%%%%
\index{stratum}%%
\index{stratum!semantical}%
\index{stratum!syntactical}%
\index{stratum!morphological}%
\index{stratum!phonological}%
\index{phonology}%%
\index{morphology}%
\index{syntax}%%
%%%
Language is organized in four levels or layers, which are also
called \textbf{strata}, see Figure~\ref{fig:strata}: the 
\textbf{phonological stratum}, the \textbf{morphological stratum}, 
the \textbf{syntactic stratum} and the \textbf{semantical stratum}. 
Each stratum possesses elementary units and 
rules of combination. The phonological stratum and the morphological 
stratum are adjacent, the morphological stratum and the syntactic 
stratum are adjacent, and the syntactic stratum and the semantic 
stratum are adjacent. Adjacent
%%%
\index{rule of realization}%%
%%%
strata are interconnected by so--called \textbf{rules of realization}.
On the phonological stratum we find the mere representation
of the utterance in its phonetic and phonological form. The
%%%
\index{phone}%%
%%%%
elementary units are the \textbf{phones}. An utterance is composed
from phones (more or less) by concatenation.
%%%
\index{phone}%%
%%%
The terms `phone', `syllable', `accent' and `tone' refer to this stratum. 
In the morphological stratum we find the elementary signs 
of the language (see Section~\ref{kap3}.\ref{kap3-1}),
%%%
\index{morph}%%
%%%
which are called \textbf{morphs}. These are defined to be the smallest
units that carry meaning, although the definition of `smallest'
may be difficult to give. They are different from words.
The word {\tt sees} is a word, but it is the combination
of two morphs, the root {\tt see} and the ending of the third
person singular present, {\tt s}. The units of the syntactical
stratum
%%
\index{lex}%%
\index{seme}%%
%%%
are called \textbf{lexes}, and they more or less are the same as words. 
The units of the semantical stratum are the \textbf{semes}.

On each stratum we distinguish concrete from abstract units. The concrete 
forms represent {\it substance}, while the abstract ones represent the 
{\it form} only. While the relationship between these two levels is 
far from easy, we will simplify the matter as follows. The abstract 
units are seen as sets of concrete ones. The abstraction is done
in such a way that the concrete member of each class that appears
in a construction is defined by its context, and that substitution
of another member results simply in a non well--formed unit (or
else in a virtually identical one). This definition is
deliberately vague; it is actually hard to make precise. The
interested reader is referred to the excellent
\cite{harris:structural} for a thorough discussion of the
structural method. We shall also return to this question in 
Section~\ref{kap5}.\ref{kap5-3}. The abstract counterpart of a phone is a
%%%
\index{phoneme}%%
%%%
\textbf{phoneme}. A phoneme is simply a set of phones. The sounds of
a single language are a subset of the entire space of human
sounds, partitioned into phonemes. This is to say that two
distinct phonemes of a languages are disjoint. We shall deal with
the relationship between phones and phonemes in 
Section~\ref{kap5}.\ref{kap5-3}. We use the following notation. We enclose
phonemes in slashes while square brackets are used to name phones.
So, if [p] denotes a phone then /p/ is a phoneme containing [p].
(Clearly, there are infinitely many sounds that may be called [p],
but we pick just one of them.) An index is used to make clear which
language the phoneme belongs to. For phonemes are strictly language 
bound. It makes little sense to compare phonemes across languages. 
Languages cut up the sound continuum in a different way. For example, 
let [p] and [p\textsuperscript{h}] be two distinct phones, 
where [p] is a phone corresponding to the letter {\tt p} in {\tt spit}, 
[p\textsuperscript{h}] a phone corresponding to the letter {\tt p} 
in {\tt put}. 
%%%%
\index{Hindi}%%
\index{English}%%
%%%
Hindi distinguishes these two phones as instantiations of different
phonemes: $/\mbox{\rm p}/_H \cap /\mbox{\rm p\textsuperscript{h}}/_H 
= \varnothing$.  English does not. So, $/\mbox{\rm p}/_E = 
/\mbox{\rm p\textsuperscript{h}}/_E$.
Moreover, the context determines whether what is written {\tt p}
is pronounced either as [p] or as [p\textsuperscript{h}]. Actually, 
in English there is no context in which both will occur. Finally, 
%%%
\index{French}%%
%%%
French does not even have the sound [p\textsuperscript{h}].  We give 
another example. The combination of the letters {\tt ch} is pronounced 
in two noticeably distinct ways in 
%%%
\index{German}%%%
%%%
German. After [\i], it sounds like 
[\c{c}], for example in {\tt Licht} [l{\i}\c{c}t], but after [a] it 
sounds like [x] as in {\tt Nacht} [naxt]; the choice between these 
two variants is conditioned solely by the preceding vowel. It is 
therefore assumed that German does not possess two phonemes but only 
one, written {\tt ch}, which is pronounced in these two ways depending 
on the context.

In the same way one assumes that German has only one plural 
\textbf{morpheme}
%%%
\index{morpheme}%%
%%%
even though there is a fair number of individual plural morphs.
Table~\ref{tab:gerplur} shows some possibilities of forming the 
plural in German.
%%
\begin{table}
\caption{German Plural Morphs}
\label{tab:gerplur}
\begin{center}
\begin{tabular}{|l|l||l|}
\hline
singular & plural & \\\hline\hline
{\tt Wagen}    & {\tt Wagen}  & `car' \\
{\tt Auto}     & {\tt Autos}  & `car' \\
{\tt Bus}      & {\tt Busse}  &  `bus' \\
{\tt Licht}    & {\tt Lichter}  &  `light' \\
{\tt Vater}    & {\tt V\"ater}  &  `father' \\
{\tt Nacht}    & {\tt N\"achte} &  `night' \\\hline
\end{tabular}
\end{center}
\end{table}
%%
The plural can be expressed either by no change, or by
adding an {\tt s}--suffix, an {\tt e}--suffix (the reduplication of
{\tt s} in {\tt Busse} is  a phonological effect and needs no
accounting for in the morphology), an {\tt er}--suffix, or
by umlaut or a combination of umlaut together with an {\tt e}--suffix.
%%%
\index{umlaut}%%
%%%
(\textbf{Umlaut} is another name for the following change of vowels:
{\tt a} becomes {\tt \"a}, {\tt o} becomes {\tt \"o}, and {\tt u} 
becomes {\tt \"u}. All other vowels remain the same. Umlaut is 
triggered by certain inflectional or derivational suffixes.)
All these are clearly different morphs. But they
%%%
\index{allomorph}%%
%%%
belong to the same morpheme. We therefore call them \textbf{allomorphs} 
of the plural morpheme. The differentiation into
strata allows to abstract away from irregularities.
Moving up one stratum, the different members of an abstraction
class are not distinguished. The different plural morphs for
example, are defined as sequences of phonemes, not of phones. To
decide which phone is to be inserted is the job of the phonological
stratum. Likewise, the word {\tt Lichter} is `known' to the
syntactical stratum only as a plural nominative noun. That it
consists of the root morph {\tt Licht} together with the morph
{\tt er} rather than any other plural morph is not visible in the
syntactic stratum. The difference between concrete and abstract
carries over in each stratum in the distinction
%%%
\index{stratum!surface}%%
\index{stratum!deep}%%
%%%
between a \textbf{surface} and a \textbf{deep} sub--stratum. The
morphotaxis has at deep level only the root {\tt Licht} and the
plural morpheme. At the surface, the latter gets realized as {\tt
er}. The step from deep to surface can be quite complex. For
example, the plural {\tt N\"achte} of {\tt Nacht} is formed by
changing the root vowel and adding the suffix {\tt e}. 
%%%
\index{umlaut}%%
%%% 
(Which of the vowels of the root are subject to umlauted must be 
determined by the phonological stratum. For example, the plural of 
{\tt Altar} `altar' is {\tt Alt\"are} and not {\tt \"Altare} or {\tt
\"Alt\"are}!) As we have already said, on the so--called deep
morphological (sub--)stratum we find only the combination of two
morphemes, the morpheme {\tt Nacht} and the plural morpheme. On
the syntactical stratum (deep or surface) nothing of that
decomposition is visible. We have one lex(eme), {\tt N\"achte}.
On the phonological stratum we find a sequence of 5 (!) phonemes,
which in writing correspond to {\tt n}, {\tt \"a}, {\tt ch}, {\tt
t} and {\tt e}. This is the deep phonological representation. On
the surface, we find the allophone [\c{c}] for the phoneme
(written as) {\tt ch}.

In Section~\ref{kap3}.\ref{kap3-1} we shall propose an approach to language
by means of signs. This approach distinguishes only 3 dimensions: a
sign has a {\it realization}, it has a {\it combinatorics\/} and 
it has a {\it meaning}. While the meaning is uniquely identifiable 
to belong to the semantic stratum, for the other two this is not
clear. The combinatorics may be seen as belonging to the
syntactical stratum. The realization of a sign, finally, could be
spelled out either as a sequence of phonemes, as a sequence of
morphemes or as a sequence of lexemes. Each of these choices is
legitimate and yields interesting insights. However, notice that
choosing sequences of morphemes or lexemes is somewhat incomplete
since it further requires an additional algorithm that realizes
these sequences in writing or speaking.

Language is not only spoken, it is also written. However, one must
distinguish between letters and sounds. The difference between
them is foremost a physical one. They use a different {\it
channel}.
%%
\index{channel}%%
%%
A \textbf{channel} is a physical medium in which the message is 
manifested. Language manifests itself first and foremost acoustically, 
even though a lot of communication is done in writing. We principally
learn a language by hearing and speaking it. Mastery of writing is
achieved only after we are fully fluent just speaking the language, 
even though our views of language are to a large extent shaped by 
our writing culture (see \cite{coulmas:writing} on that). (Sign 
languages form an exception that will not be dealt with here.) 
Each channel allows --- by its mere physical properties 
--- a different means of combination. A piece of paper is a two 
dimensional thing, and we are not forced to write down symbols 
linearly, as we are with acoustical signals. Think for example of 
the fact that Chinese characters are composite entities which contain 
parts in them. These are combined typically by juxtaposition, but 
characters are aligned vertically. Moreover, the graphical composition 
internally to a sign is of no relevance for the actual sound that goes 
with it. To take another example, Hindi is written in a syllabic
script, which
%%%
\index{Devanagari}%%
%%%
is called \textbf{Devanagari}. Each simple consonantal letter denotes
a consonant plus {\tt a}. Vowel letters may be added to these in
case the vowel is different from {\tt a}. (There are special
characters for word initial vowels.) Finally, to denote
consonantal clusters, the consonantal characters are melted into
each other in a particular way. There is only a finite number of
consonantal clusters and the way the consonants are melted is
fixed. The individual consonants are usually recognizable from the
graphical complex. In typesetting there is a similar phenomenon 
%%%%
\index{ligature}%%
%%%%
known as \textbf{ligature}. The graphemes {\tt f} and {\tt i} melt 
into one when the first is before the second: `fi'. (Typewriters 
have no ligature, for obvious reasons. So you get {\tt fi}.) Also, 
in mathematics the possibilities of the graphical channel are widely 
used. We use indices, superscripts, subscripts, underlining, arrows 
and so on. Many diagrams are therefore not so easy to linearize. 
(For example, $\widehat{x}$ is spelled out as {\tt x hat}, $\overline{x}$ 
as {\tt x bar}.) Sign languages also make use of the three--dimensional 
space, which proves to require different perceptual skills than 
spoken language. 

While the acoustic manifestation of language is in some sense
essential for human language, its written manifestation is
typically secondary, not only for the individual human being, as
said above, but also from a cultural historic point of view. The
sounds of the language and the pronunciation of words is something
that comes into existence naturally, and they can hardly be fixed
or determined arbitrarily. Attempts to stop language from changing
are simply doomed to failure. Writing systems, on the other hand,
are cultural products, and subject to sometimes severe regimentation. 
The effect is that writing systems show much greater variety across 
languages than sound systems. The number of primitive letters varies 
between some two dozen and a few thousand. This is so since some 
languages have letters for sounds (more or less) like Finnish 
%%%%
\index{Finnish}%%
%%%
(English is a difficult case), others have letters for syllables 
%%%
\index{Hindi}%%
%%%%
(Hindi, written in Devanagari) and yet others have letters for 
words (Chinese). 
%%%%
\index{Chinese}%%
%%%
It may be objected that in Chinese a character always stands for a 
syllable, but words may consist of several syllables, hence 
of several characters. Nevertheless, the difference with 
Devanagari is clear. The latter shows you how the word sounds like, 
the former does not, unless you know character by character how it 
is pronounced. If you were to introduce a new syllable into Chinese 
you would have to create a new character, but not so in Devanagari. But 
all this has to be taken with care. Although French 
%%%
\index{French}%%
%%%
uses the Latin 
alphabet it becomes quite similar to Chinese. You may still know how 
to pronounce a word that you see written down, but from hearing it 
you are left in the dark as to how to spell it. For example, the 
following words are pronounced completely alike: {\tt au}, {\tt haut}, 
{\tt eau}, {\tt eaux}; similarly {\tt vers}, {\tt vert}, {\tt verre}, 
{\tt verres}.

In what is to follow, language will be written language. This is
the current practice in such books as this one; but it requires 
comment. We are using the so--called Latin alphabet. It is used in 
almost all European countries, while each country typically uses a 
different set of symbols. The difference is slight, but needs 
accounting for (for example, when you wish to produce keyboards 
or design fonts). Finnish, Hungarian and German,
%%%%
\index{Finnish}%%%
\index{Hungarian}%%%
\index{German}%%
%%%%
for example, use {\tt \"a}, {\tt \"o} and {\tt \"u}. The letter
{\tt {\ss}} is used in the German alphabet (but not in
Switzerland). In French, 
%%%
\index{French}%%
%%%
one uses {\tt \c{c}}, also accents, and
so on. The resource of single characters,
%%%
\index{letter}%%
%%%
which we call \textbf{letters}, is for the European languages somewhere 
between 60 and 100. Besides each letter, both in upper and lower case,
we also have the punctuation marks and some extra symbols, not to 
forget the ubiquitous blank. Notice, however, that not all languages 
have a blank (Chinese is a case in point, and also the Romans did not 
use any blanks). 
On the other hand, one blank is not distinct from two. We can either 
decide to disallow two blanks in a row, or postulate that they are 
equal to one. (So, the structure we look at is 
$\GZ(A)/\{\Box \boldsymbol{\doteq} \Box\conc\Box\}$.) A final problem 
area to be considered is our requirement that sign composition is 
additive. This means that every change that occurs is underlyingly viewed 
as adding something that was not there. This can yield awkward 
%%%
\index{German}\index{umlaut}%%
%%%
results. While the fact that German umlaut is graphically 
speaking just the addition of two dots ({\tt a} becomes {\tt \"a}, 
{\tt o} becomes {\tt \"o}, {\tt u} becomes {\tt \"u}), the change 
of a lower case letter to an upper case letter cannot be so 
analysed. This requires another level of representation, one 
at which the process is completely additive. This is harmless, 
if we only change the material aspect (substance) rather than the 
form. 

The counterpart of a letter in the spoken languages is the
phoneme.
%%%
\index{phoneme}%%
%%%
Every language utterance can be analyzed into a sequence of
phonemes (plus some residue about which we will speak briefly
below). There is generally no biunique correspondence between
phonemes and letters. The connection between the visible and the
audible shape of language is everything but predictable or unambiguous 
in either direction. English is a perfect example. There is hardly any 
letter that can unequivocally be related to a phoneme. For example, 
the letter {\tt g} represents in many cases the phoneme [g] unless it 
is followed by {\tt h}, in which case the two typically together
represent a sound that can be zero (as in {\tt sought}
[s{\textopeno}:t]), or {\tt f} (as in {\tt laughter} ([la:ft\textschwa]).
To add to the confusion, the letters represent different sets of 
phones in different languages. (Note that it makes no sense to 
speak of the same {\it phoneme\/} in two different languages, 
as phonemes are abstractions that are formed within a single 
language.) The letter {\tt u} has many different manifestations 
in English, German and French 
%%%
\index{English}\index{German}\index{French}%%%
%%%%
that are hardly compatible. This has prompted the invention
of an international standard, the so--called \textbf{International
Phonetic Alphabet} (\textbf{IPA}, see \cite{ipahandbook}). Ideally,
every sound of a given language can be uniquely transcribed into
IPA such that anyone who is not acquainted with the language can
reproduce the utterances correctly. The transcription of a word
into this alphabet therefore changes whenever its sound
manifestation changes, irrespective of the spelling norm.
Unfortunately, the transcription must ultimately remain 
inconsequential, because even in the IPA letters stand for 
sets of phones, but in every language the width of a phoneme 
(= the set of phones it contains) is different. For example, 
%%%
\index{English}\index{Hindi}%%
%%%%
if (English) $/\mbox{\rm p}/_E$ contains both (Hindi) 
$/\mbox{\rm p}/_H$ and  $/\mbox{\rm p\textsuperscript{h}}/_H$, 
we either have to represent {\tt p} in English by (at least) two 
letters or else give up the exact correspondence. 

The carriers of meaning are however not the sounds or letters
(there is simply not enough of them); it is certain sequences
thereof. Sequences of letters that are not separated by a
blank or a punctuation mark other than `{\tt -}' are called
%%%
\index{word}%%
%%%
\textbf{words}. Words are units which can be analyzed further, for
example into letters, but for the most part we shall treat them
as units. This is the reason why the alphabet $A$ in the technical
sense will often {\it not\/} be the alphabet in the sense of
`stock of letters' but in the sense of `stock of words'. However,
since most languages have infinitely many words (due to
compounding), and since the alphabet $A$ must be finite, some care
must be exercised in choosing the alphabet. Typically, it will 
exclude the compound words, but it will have to include all idioms. 

We have analyzed words into sequences of letters or sounds, and
sentences into sequences of words. This implies that sentences and
words can always be so analyzed. This is what we shall assume
throughout this book. The individual occurrences of
%%%
\index{segment}%%
%%%
sounds (letters) are called \textbf{segments}. For example, the 
(occurrences of the) letters {\tt n}, {\tt o}, and {\tt t} are the 
segments of {\tt not}. The fact that words can be segmented is called
%%%
\index{segmentability}%%
%%%
\textbf{segmentability property}. At closer look it turns out that
segmentability is an idealization. For example, a question differs
from an assertion in its \textbf{intonation contour}, which is 
the rise and fall of the pitch during the utterance. The contour 
shows distribution over the whole sentence but follows
specific rules. It is of course different in different languages.
(Falling pitch at the end of a sentence, for example, may
accompany questions in English, but not in German.) Because of its
nature, intonation contour is called a
%%%
\index{feature!suprasegemental}%%
%%%
\textbf{suprasegmental feature}. There are more, for example emphasis.
Segmentability differs also with the channel.  In writing, a question
is marked by a segmental feature (the question mark), but emphasis is
not. Emphasis is typically marked by underlining or italics. For
example, if we want to emphasize the word `board', we write
$\uli{\mbox{\tt board}}$ or {\it board}. As can be seen, every
letter is underlined or set in italics, but underlining or italics is 
usually not something that is meant to emphasize those letters
that are marked by it; rather, it marks emphasis of the entire
word that is composed from them. We could have used a segmental
symbol, just like quotes, but the fact of the matter is that we
do not. Disregarding this, language typically is segmentable.

However, even if this is true, the idea that the morphemes of the 
language are sequences of letters is largely mistaken. To give an 
extreme example, the plural is formed in Bahasa Indonesia
%%%
\index{Bahasa Indonesia}%%
%%%
by reduplicating the noun. For example, the word {\tt anak} means
`child', the word {\tt anak-anak} therefore means `children', the
word {\tt orang} means `man', and {\tt orang-orang} means `men'.
Clearly, there is no sequence of letters or phonemes that can be
literally said to constitute a plural morph. Rather, it is the
function $f \colon A^{\ast} \pf A^{\ast} \colon \vec{x} \mapsto
\vec{x}\mbox{\tt -}\vec{x}$, sending each string to its duplicate
(with an interspersed hyphen). Actually, in writing the
abbreviation {\tt anak2} and {\tt orang2} is commonplace. Here,
{\tt 2} is a segmentable marker of plurality. However, notice that
the words in the singular or the plural are each fully
segmentable. Only the marker of plurality cannot be identified
with any of the segments. This is to some degree also the case in
German, where the rules are however much more complex, as we have
seen above. The fact that morphs are (at closer look) not simply
strings will be of central concern in this book.

Finally, we have to remark that letters and phonemes are not
unstructured either. Phonemes consist of various so--called
%%%
\index{feature!distinctive}%%
%%%
\textbf{distinctive features}. These are features that distinguish
the pho\-ne\-mes from each other. For example, [p] is distinct from
[b] in that it is voiceless, while [b] is voiced. Other voiceless
consonants are [k], [t], while [g] and [d] are once again voiced.
Such features can be relevant for the description of a language.
There is a rule of German (and other languages, for example
Russian) that forbids voiced consonants to occur at the end of a
syllable. For example, the word {\tt Jagd} `hunting' is pronounced
[\textprimstress ja:kt], not [\textprimstress ja:gd]. This is so 
since [g] and [d] may not occur at the end of the syllable, since 
they are voiced. Now, first of all, why do we not write {\tt Jakt} 
then? This is so since inflection and derivation show that when these 
consonants occur non--finally in the syllable they are voiced: we 
have {\tt Jagden} [\textprimstress ya:kden] `huntings', with [d] 
now in fact being voiced, and also {\tt jagen} 
[\textprimstress ya:g\textschwa n] `to hunt'. Second: why
do we not propose that voiceless consonants become voiced when
syllable initial? Because there is plenty of evidence that this
does not happen. Both voiced and voiceless sounds may appear at
the beginning of the syllable, and those ones that are analyzed as
underlyingly voiceless remain so in whatever position. Third: why
bother writing the underlying consonant rather than the one we
hear? Well, first of all, since we know how to pronounce the word
anyway, it does not matter whether we write [d] or [t]. On the
other hand, if we know how to write the word, we also know a
little bit about its morphological behaviour. What this comes down
to is that to learn how to write a language is to learn how the
language works. Now, once this is granted, we shall explain why we
find [k] in place of [g] and [t] in place of [d]. This is because
of the internal organisation of the phoneme. The phoneme is a set
of distinctive features, one of which (in German) is $[\pm
\mbox{\rm voiced}]$. The rule is that when the voiced consonant
may not occur, it is only the feature $[+ \mbox{\rm voiced}]$ that
is replaced by $[- \mbox{\rm voiced}]$. Everything else remains
the same. A similar situation is the relationship between upper
and lower case letters. The rule says that a sentence may not
begin with a lower case letter. So, when the sentence begins, the
first letter is changed to its upper case counterpart if
necessary. Hence, letters too contain distinctive features. Once
again, in a dictionary a word always appears as if it would normally 
appear elsewhere. Notice by the way that although each letter is by 
itself an upper or a lower case letter, written
language attributes the distinction upper versus lower case to the
word not to the initial letter.  Disregarding some modern
spellings in advertisements (like in Germany {\tt InterRegio},
{\tt eBusiness} and so on) this is a reasonable strategy. However,
it is nevertheless not illegitimate to call it a suprasegmental
feature.

In the previous section we have talked extensively about representations 
of terms by means of strings. In linguistics this is an important 
issue, which is typically discussed in conjunction 
with {\it word order}. Let us give an example.  Disregarding
word classes, each word of the language has one (or several)
arities. The finite verb \textsf{see} has arity 2. The proper
names \textsf{Paul} and \textsf{Marcus} on the other hand have arity
0. Any symbol of arity $> 0$ is called a \textbf{functor} with
respect to its argument.
%%%
\index{functor}%%
\index{argument}%%
%%%
In syntax one also speaks of
%%%
\index{head}%%
\index{complement}%%
%%%
\textbf{head} and \textbf{complement}. These are relative notions. In
the term $\textsf{see}(\textsf{Marcus},\textsf{Paul})$,
the functor or head is \textsf{see}, and its arguments are
\textsf{Paul} and \textsf{Marcus}. To distinguish these arguments from
each other, we use the terms {\it subject\/} and {\it object}.
%%%
\index{subject}%%
\index{object}%%
%%%
\textsf{Marcus} is the \textbf{subject} and \textsf{Paul} is the 
\textbf{object} of the sentence. The notions `subject' and `object' 
denote so--called \textbf{grammatical relations}.
%%%
\index{grammatical relation}%%
%%%
The correlation between argument places and grammatical relations
is to a large extent arbitrary, and is of central concern in
syntactical theory. Notice also that not all arguments are
complements. Here, syntactical theories diverge as to which of
the arguments may be called `complement'. In generative grammar,
for example, it is assumed that only the direct object is a
complement.

Now, how is a particular term represented?
The representation of \textsf{see} is {\tt sees}, that of
\textsf{Marcus} is {\tt Marcus} and that of \textsf{Paul} is {\tt Paul}.
The whole term \eqref{ex:1} is represented by the string
\eqref{ex:2}.
%%
\begin{align}
\label{ex:1}
& \textsf{see}(\textsf{Marcus},\textsf{Paul}) \\
\label{ex:2}
& \mbox{\tt Marcus sees Paul.}
\end{align}
%%
So, the verb appears after the subject, which in turn precedes the
object. At the end, a period is placed. However, to spell out the
relationship between a language and a formal representation is not
as easy as it appears at first sight. For first of all, the term 
should be something that does not depend on the particular
language we choose and which gives us the full meaning of the term
(so it is like a language of thought or an interlingua, if you
wish). So the above term shall mean that Marcus sees Paul. We
could translate the English sentence \eqref{ex:2} by choosing a
different representation language, but the choice between
languages of representation should actually be immaterial as long
as they serve the purpose. This is a very rudimentary picture but
it works well for our purposes. We shall return to the idea of
producing sentences from terms in Chapter~\ref{kap3}. Now look
first at the representatives of the basic symbols in some other
languages.
%%
%%%
\index{Latin}%%%
%%%
\begin{equation}
\mbox{\begin{tabular}{|l||l|l|l|}
\hline
           & \textsf{see} & \textsf{Marcus} & \textsf{Paul} \\\hline
German     & {\tt sieht} & {\tt Marcus} & {\tt Paul} \\
Latin      & {\tt vidit} & {\tt Marcus} & {\tt Paulus} \\
Hungarian  & {\tt l\'{a}tja} & {\tt Marcus} & {\tt P\'al} \\\hline
\end{tabular}}
\end{equation}
%%%
Here is how \eqref{ex:1} is phrased in these languages.
%%
\begin{align}
\label{ex:1349}
& \mbox{\tt Marcus sieht Paul.} \\
\label{ex:1350}
& \mbox{\tt Marcus Paulum vidit.} \\
\label{ex:1351}
& \mbox{\tt Marcus l\'atja P\'alt.}
\end{align}
%%
English is called an \textbf{SVO--language}, since in transitive
%%%
\index{language!SOV--, SVO--, VSO--}%%
\index{language!OSV--, OVS--, VOS--}%%
%%%
constructions the subject precedes the verb, and the verb in turn
the object. This is exactly the infix notation. (However, notice
that languages do not make use of brackets.) One uses the mnemonic
symbols `S', `V' and `O' to define the following basic 6 types of
languages: SOV, SVO, VSO, OSV, OVS, VOS. These names tell us how
the subject, verb and object follow each other in a basic
transitive sentence. We call a language of type VSO or VOS 
%%%
\index{language!verb final}%
\index{language!verb medial}%
\index{language!verb initial}%
%%%
\textbf{verb initial}, a language of type SOV or OSV \textbf{verb
final} and a language of type SVO or OVS \textbf{verb medial}. By
this definition, German 
%%%%
\index{German}%%
\index{Hungarian}%%
%%%
is SVO, Hungarian too, hence both are verb
medial and Latin is SOV, hence verb final. These types are not
equally distributed. Depending on the method of counting, 
40 -- 50~\% of the world's
languages are SOV languages, up to 40~\% SVO languages and another
10~\% are VSO languages. This means that in the vast majority of
languages the order of the two arguments is: subject before
object. This is why one does not generally emphasize the relative
order of the subject with respect to the object. There is a bias
against placing the verb initially (VSO), and a slight bias to put
it finally (SOV) rather than medially (SVO).

One speaks of a \textbf{head final} 
%%%
\index{language!head final}%%
\index{language!head initial}%%
%%%
(\textbf{head initial}) language if
a head is consistently put at the end behind all of its arguments
(at the beginning, before all the arguments). One denotes the type
of order by XH (HX), X being the complement, H the head.
There is no notion of a {\it head medial\/} language for the reason
that most heads only have one complement. It is often understood 
that the direct object is the only complement of the verb.
Hence, the word orders SVO and VOS are head initial, OVS and SOV head
final. (The orders VSO and OSV are problematic since the verb is
not adjacent to its object.) A verb is a head, however a very
important one, since it basically builds the clause. Nevertheless,
different heads may place their arguments differently, so a
language that is verb initial need not be head initial, a language
that is verb final need not be head final. Indeed, there are few
languages that are consistently head initial (medial, final).
%%%
\index{Japanese}%%%
%%%
Japanese is rather consistently head final. Even a
relative clause precedes the noun it modifies. Hungarian 
%%%
\index{Hungarian}%%
%%%
is a mixed case: adjectives precede nouns, there are no prepositions,
only postpositions, but the verb tends to precede its object.

For the interested reader we give some more information on the
languages shown above. First, Latin 
%%%
\index{Latin}%%
%%%
was initially an SOV language,
however word order was not really fixed (see
\cite{lehmann:bases} and \cite{bauer:latin}). In fact, any of the six
permutations of the sentence \eqref{ex:1350} is grammatical.
Hungarian is more complex, again the word order shown in
\eqref{ex:1351} is the least marked, but the rule is that
discourse functions determine word order. (Presumably this is true 
for Latin as well.) German 
%%%
\index{German}%%
%%%
is another special case. Against all appearances there is all reason to
believe that it is actually an SOV language. You can see this by
noting first that only the carrier of inflection appears in second
place, for example only the auxiliary if present. Second, in a
subordinate clause all parts of the verb including the carrier of
inflection are at the end.
%%
\begin{align}
%\begin{tabular}{ll}
%\begin{split}
& \mbox{\tt Marcus sieht Paul.} \\\notag
& \mbox{\it Marcus sees Paul.} \\
%\end{split}
%\begin{split}
& \mbox{\tt Marcus will Paul sehen.} \\\notag
& \mbox{\it Marcus wants to see Paul.} \\
%\end{split}
%\begin{split}
& \mbox{\tt Marcus will Paul sehen k\"onnen.} \\\notag
& \mbox{\it Marcus wants to be able to see Paul.} \\
%\end{split}
%\begin{split}
& \mbox{..., {\tt weil Marcus Paul sieht.}} \\\notag
& \mbox{\it ..., because Marcus sees Paul.} \\
%\end{split}
%\begin{split}
& \mbox{..., {\tt weil Marcus Paul sehen will.}} \\\notag
& \mbox{\it ..., because Marcus wants to see Paul.} \\
%\end{split}
%\begin{split}
& \mbox{..., {\tt weil Marcus Paul sehen k\"onnen will.}} \\\notag
& \mbox{\it ..., because Marcus wants to be able to see Paul.}
%\end{split}
%\end{tabular}
\end{align}
%%
So, the main sentence is not always a good indicator of the
word order. Some languages allow for alternative word orders,
like Latin and Hungarian. This is not to say that all variants
have the same meaning or significance; it is only that they
are equal as representatives of \eqref{ex:1}. We therefore
speak of Latin as having \textbf{free word order}.
%%%
\index{word order!free}%%
%%%
However, this only means that the head and the argument can
assume any order with respect to each other, not that simply
all permutations of the words mean the same.

Now, notice that subject and object are coded by means
%%%
\index{case}%%%
%%%
of so--called \textbf{cases}. In Latin, the object carries accusative
case, so we find {\tt Paulum} instead of {\tt Paulus}. Likewise,
in Hungarian we have {\tt P\'alt} in place of {\tt P\'al},
the nominative. So, the way a representing string is arrived at
is rather complex.  We shall return again to case marking in
Chapter~\ref{kap4}.

Natural languages also display so--called
%%%
\index{polyvalency}%%
%%%
\textbf{polyvalency}. We say that a word is polyvalent if it can have
several arities (even with the same meaning). The verb {\tt to roll}
%%%
\index{verb!transitive}%%
\index{verb!intransitive}%%
%%%
can be unary (= \textbf{intransitive}) as well as binary 
(= \textbf{transitive} if the second argument is accusative, 
\textbf{intransitive} otherwise).
This is not allowed in our definition of signature. However,
it can easily be modified to account for polyvalent symbols.

{\it Notes on this section.} The rule that spells out 
the letters \mbox{\tt ch} in German is more complex than the above 
explications show. For example, it is [x] in {\tt fauchen} 
but [\c{c}] in {\tt Frauchen}. This may have two reasons: 
(a) There is a morpheme boundary between {\tt u} and {\tt ch} 
in the second word but not in the first. This morpheme boundary 
induces the difference. (b) The morpheme {\tt chen} is special 
in that \mbox{\tt ch} will always be realized as [\c{c}]. The 
difference between (a) and (b) is that while (a) defines a 
realization rule that uses only the phonological 
representation, (b) uses morphological information to define 
the realization. Mel'\v{c}uk 
%%%
\index{Mel'\v{c}uk, Igor}%%%
%%%
defines the realization rules 
as follows. In each stratum, there are rules that define how 
deep representations get mapped to surface representations. 
Across strata, going down, the surface representations of the 
higher stratum get mapped into abstract representations of 
the lower stratum. (For example, a sequence of morphemes is 
first realized as a sequence of morphs and then spelled out 
as a sequence of phonemes, until, finally, it gets mapped 
onto a sequence of phones.) Of course, one may also reverse 
the process. However, adjacency between (sub-)strata remains 
as defined. 
%%%
\vplatz
\exercise
%%%
\index{Polish Notation}%%%
%%%
Show that in Polish Notation, unique readability is lost when
there exist polyvalent function symbols.
%%%
\vplatz
\exercise
Show that if you have brackets, unique readability is guaranteed
even if you have polyvalency.
%%%
\vplatz
\exercise
We have argued that German is a verb final language. But is it
strictly head final? Examine the data given in this section
as well as the data given below.
%%
\begin{align}
 & \mbox{\tt Josef pfl\"uckt eine sch\"one Rose f\"ur Maria.} \\\notag
 & \mbox{Josef is.picking a beautiful rose for Mary} \\
 & \mbox{\tt Heinrich ist dicker als Josef.} \\\notag
 & \mbox{Heinrich is fatter than Josef}
\end{align}
%%
%Knowing that long ago the ancestor of German (and English) was
%consistently head final, can you give an explanation for the
%syntactic facts shown in this section?
%%%
\vplatz
\exercise
Even if languages do not have brackets, there are elements that
indicate clearly the left or right periphery of a constituent.
Such elements are the determiners {\tt the} and {\tt a}({\tt n}).
Can you name more? Are there elements in English indicating the
right periphery of a constituent? How about demonstratives like
{\tt this} or {\tt that}?
%%%
\vplatz %%
\exercise %%
By the definitions, Unix is head initial. For example, the
command {\tt lpr} precedes its arguments. Now study the way
in which optional arguments are encoded. (If you are sitting
behind a computer on which Unix (or Linux) is running, 
type {\tt man lpr} and you get a synopsis of the command and its 
syntax.) Does the syntax guarantee unique readability? (For the more 
linguistic minded reader: which type of marking strategy does 
Unix employ?  Which natural language you know of corresponds 
best to it?)

 \section{Trees}
\label{kap1-4}
\label{einszwei}
%
%
%
Strings can also be defined as pairs $\auf \GL, \ell\zu$
where $\GL = \auf L, <\zu$ is a finite linearly ordered set
and $\ell \colon L \pf A$ a function, called the \textbf{labelling
function}.
%%%
\index{labelling function}%%
%%%
Since $L$ is finite we have $\auf L, < \zu \cong \auf n, %
\in\zu$ for $n := |L|$. (Recall that $n$ is a set that is
linearly ordered by $\in$.) Replacing $\auf L, <\zu$ by the 
isomorphic $\auf n, \in\zu$, and eliminating the redundant $\in$, 
a string is often defined as a pair $\auf n, \ell\zu$, where $n$ 
is a natural number. 
In what is to follow, we will very often have to deal with extensions 
of relational structures (over a given signature $\Xi$) by a labelling 
function. They have the general form $\auf M, \GI, \ell\zu$, where 
$M$ is a set, $\GI$ an interpretation and $\ell$ a function from $M$ 
to $A$. These structures shall be called \textbf{structures over} $A$ or 
$A$--\textbf{structures}.
%%%
\index{structure over $A$}%%
\index{structures!$A$--\faul}%%

A very important notion in the analysis of language is that of
a {\it tree}. A tree is a special case of a directed graph.
%%%
\index{graph!directed}%%
%%%
A \textbf{directed graph} is a structure $\auf G, <\zu$, where
$<\; \subseteq\, G^2$ is a binary relation. As is common usage,
we shall write $x \leq y$ if $x < y$ or $x = y$. Also, $x$ and $y$
are called \textbf{comparable} if $x \leq y$ or $y \leq x$. A
%%%
\index{chain}%%
%%%
(\textbf{directed}) \textbf{chain of length} $k$ is  a sequence
$\auf x_i : i < k + 1\zu$ such that $x_i < x_{i+1}$ for all
$i < k$. An \textbf{undirected chain of length} $k$ is a sequence
$\auf x_i : i < k + 1\zu$ where $x_i < x_{i+1}$ or
$x_{i+1} < x_i$ for all $i < k$. A directed graph is called
%%%
\index{graph!connected}%%
%%%
\textbf{connected} if for every two elements $x$ and $y$ there is
an undirected chain from $x$ to $y$. A directed chain
%%
\index{cycle}%%
%%%
of length $k$ is called a \textbf{cycle of length} $k$
%%%
\index{cycle}%%
\index{root}%%
%%%
if $x_k = x_0$. A binary relation is called \textbf{cycle free} if
it only has cycles of length 0. A \textbf{root} is an element $r$
such that for every $x$ $x <^{\ast} r$, where $<^{\ast}$ is the 
reflexive, transitive closure of $<$. 
%%
\begin{defn}
%%%
\index{graph!directed acyclic}%%
\index{graph!directed transitive acyclic (DTAG)}%%
\index{DAG}%%
%%%
A \textbf{directed acyclic graph} (a \textbf{DAG}) 
is a pair $\GG = \auf G, <\zu$ such that $<\; \subseteq G^2$
is an acyclic relation on $G$. If $<$ is transitive, 
$\GG$ is called a \textbf{directed transitive acyclic graph} 
(\textbf{DTAG}).
\end{defn}
%%
\begin{defn}
%%%
\index{forest}%%
%%%
$\GG = \auf G, <\zu$ is called a \textbf{forest} if $<$
is transitive and irreflexive and if $x < y$ and $x < z$ then
$y$ and $z$ are comparable. A forest with a root
is called a \textbf{tree}.
\index{tree}%%
\end{defn}
%%
In a connected rooted DTAG the root is comparable
with every other element since the relation is transitive. Furthermore,
in presence of transitivity $<$ is cycle free iff it is
irreflexive. For if $<$ is not irreflexive it has a cycle of length 1.
Conversely, if there is a cycle $\auf x_i : i < k+1\zu$ of length
$k  > 0$, we immediately have $x_0 < x_k = x_0$, by transitivity.

If $x < y$ and there is no $z$ such that $x < z < y$, $x$ is
called a \textbf{daughter of} $y$, and $y$ the \textbf{mother of} $x$,
and we write $x \prec y$.
%%%
\index{node!daughter}%%
\index{node!mother}%%
\index{$\prec$}%%
%%%
\begin{lem}
\label{lem:tochter}
Let $\auf T, <\zu$ be a finite tree. If $x < y$ then
there exists a $\wht{x}$ such that $x \leq \wht{x} \prec y$
and a $\wht{y}$ such that $x \prec \wht{y} \leq y$.
$\wht{x}$ and $\wht{y}$ are uniquely determined by $x$ and $y$.
\proofend
\end{lem}
%%
The proof is straightforward. In infinite trees this need not hold.
We define
%%%
\index{$x \circ y$}%%
%%%%
$x \circ y$ by $x \leq y$ or $y \leq x$ and say that $x$ and $y$
\textbf{overlap}. The following is also easy.
%%%
\index{overlap}%%
\index{$\circ$}%%
%%%
\begin{lem}[Predecessor Lemma]
\label{lem:vorfahr}
\index{Predecessor Lemma}%%
%%%
Let $\GT$ be a finite tree and $x$ and $y$ nodes which do not
overlap. Then there exist uniquely determined $u$, $v$ and $w$,
such that $x \leq u \prec w$, $y \leq v \prec w$
and $v \neq u$.
\proofend
\end{lem}
%%
\index{branching number}%%
%%%
A node \textbf{branches} $n$ \textbf{times downwards} if it has exactly
$n$ daughters; and it branches $n$ \textbf{times upwards} if it has
exactly $n$ mothers. We say that a node \textbf{branches upwards}
(\textbf{downwards}) if it branches upwards or downwards at least 2
times. A finite forest is characterized by the fact that it is
transitive, irreflexive and no node branches upwards. Therefore,
in connection with trees and forests we shall speak of `branching'
when we mean `downward branching'.
%%%
\index{leaf}%%
%%%
$x$ is called a \textbf{leaf} if there is no $y < x$, that is, if
$x$ branches 0 times. The set of leaves of $\GG$ is denoted by
$b(\GG)$.
%%%
\index{$b(\GG)$}%%

Further, we define the following notation.
%%
\index{$\low{x}$, $\uppx{x}$}%%
%%
\begin{align}
\low{x} & := \{y : y \leq x\} &  \uppx{x} & := \{y : y \geq x\}
\end{align}
%%
By definition of a forest, $\uppx{x}$ is linearly ordered by
$<$. Also, $\low{x}$ together with the restriction of $<$
to $\low{x}$ is a tree.

%%%
\index{path}%%
\index{branch}%%
%%%
A set $P \subseteq G$ is called a \textbf{path} if it is linearly
ordered by $<$ and convex, that is to say, if $x, y \in P$ then
$z \in P$ for every $z$ such that $x < z < y$.  The \textbf{length of}
$P$ is defined to be $|P| - 1$. A \textbf{branch} is a maximal path 
with respect to set inclusion.
%%%
\index{height}%%
\index{depth}%%
\index{$h(x)$, $d(x)$}%%
%%%
The \textbf{height of} $x$ in a DTAG, in symbols $h_{\GG}(x)$ or
simply $h(x)$, is the maximal length of a branch in $\low{x}$.
It is defined inductively as follows.
%%
\begin{equation}
h(x) := \begin{cases}
      0 & \text{if $x$ is a leaf,} \\
      1 + \max \{h(y) : y \prec x\} 
      & \text{otherwise.}
\end{cases}
\end{equation}
%%
%%%
Dually we define the \textbf{depth} in a DTAG.
%%
\begin{equation}
d(x) := 
\begin{cases}
0 &  \text{ if $x$ is a root,} \\
1 + \max \{d(y) : y \succ x\}  &
    \mbox{ otherwise.}
\end{cases}
\end{equation}
%%
For the entire DTAG $\GG$ we set
%%
\begin{equation}
h(\GG) := \{h(x) : x \in T\} 
\end{equation}
%%
and call this the \textbf{height of} $\GG$. (This is an ordinal, 
as is easily verified.)
%%
\begin{defn}
%%%
\index{subgraph}%%
%%%
Let $\GG = \auf G, <_G\zu$ and $\GH = \auf H, <_H\zu$
be directed graphs and $G \subseteq H$. Then $\GG$ is called
a \textbf{subgraph of} $\GH$ if $<_G\; = \; <_H \cap\, G^2$.
\end{defn}
%%
\index{subtree}%%
%%%
If $\GG$ and $\GH$ are DTAG, forests or trees, then $\GG$ is
a \textbf{sub--DTAG}, \textbf{subforest} and \textbf{subtree of}
$\GH$, respectively. A subtree of $\GH$ with underlying set $\low{x}$ 
is called a \textbf{constituent of} $\GH$.
%%%
\index{constituent}%%
%%%
%%
\begin{defn}
Let $A$ be an alphabet. A \textbf{DAG over} $A$ (or an $A$--\textbf{DAG})
is a pair $\auf \GG, \ell\zu$ such that $\GG = \auf G, <\zu$ is
a DAG and $\ell \colon G \pf A$ an arbitrary function.
\end{defn}
%%
Alternatively, we speak of \textbf{DAGs with labels in} $A$, or simply 
of \textbf{labelled DAGs} if it is clear which alphabet is meant. 
Similarly with trees and DTAGs. The notions of substructures are extended 
analogously.

The tree structure in linguistic representations encodes
the hierarchical relations between elements and not their
spatial or temporal relationship. The latter have to be
added explicitly.  This is done by extending the signature
%%%
\index{$\sqsubset$}%%
%%%
by another binary relation symbol, $\sqsubset$. We say that
$x$ is \textbf{before} $y$ and that $y$ is \textbf{after} $x$ if
$x \sqsubset y$ is the case. We say that $x$ \textbf{dominates}
$y$ if $x > y$.
%%%
\index{domination}%%
%%%
The relation $\sqsubset$ articulates the temporal relationship
between the segments. This is first of all defined on the leaves,
and it is a linear ordering. (This reflects the insistance on
segmentability. It will have to be abandoned once we do not
assume segmentability.) Each node $x$ in the tree has the
physical span of its segments. This allows to define an
ordering between the hierarchically higher elements as well.
We simply stipulate that $x \sqsubset y$ iff all
leaves below $x$ are before all leaves below $y$. This is not
unproblematic if nodes can branch upwards, but this situation
we shall rarely encounter in this book.
The following is an intrinsic definition of these structures.
%%
\begin{defn}
%%%
\index{tree!ordered}%%
%%%
An \textbf{ordered tree} is a triple $\auf T, <, \sqsubset\zu$
such that the following holds.
%%%
\renewcommand{\labelenumi}{\mbox{\rm (ot\arabic{enumi})}}
\begin{enumerate}
\item $\auf T, <\zu$ is a tree.
\item $\sqsubset$ is a linear, strict ordering on
    the leaves of $\auf T,<\zu$.
\item If $x \sqsubset z$ and $y < x$ then also $y \sqsubset z$.
    \\
     If $x \sqsubset z$ and $y < z$ then also $x \sqsubset y$.
    \\
\item If $x$ is not a leaf and for all $y < x$
    $y \sqsubset z$ then also $x \sqsubset z$. \\
    If $z$ is not a leaf and for all $y < z$ $x \sqsubset y$
        then also $x \sqsubset z$.
\end{enumerate}
%%%
\renewcommand{\labelenumi}{\alph{enumi}.}
\end{defn}
%%
The condition (ot2) requires that the ordering is coherent
with the ordering on the leaves. It ensures that $x \sqsubset y$
only if all leaves below $x$ are before all leaves below $y$.
(ot3) is a completeness condition ensuring that if the latter
holds, then indeed $x \sqsubset y$.

We agree on the following notation. Let $x \in G$. Put $[x] := 
\low{x} \cap b(\GG)$. We call this the \textbf{extension of} $x$.
%%%
\index{$[x]$}%%
\index{extension of a node}%%
%%%
$[x]$ is linearly ordered by $\sqsubset$. If a labelling function 
$\ell$ is given in addition, we write $k(x) := \auf [x], \sqsubset, 
\ell \restriction [x]\zu$ 
%%%
\index{$k(x)$}%%%
%%%
and call this the \textbf{associated string of} $x$.
%%%
\index{string!associated}%%
%%%
It may happen that two nodes have the same associated string.
The string associated with the entire tree is
%%%
\begin{equation}
k(\GG) := \auf b(\GG), \sqsubset, \ell \restriction b(\GG)\zu
\end{equation}
%%%
\index{constituent!continuous}%%
%%%
A constituent is called \textbf{continuous} if the associated
string is convex with respect to $\sqsubset$. A set $M$ is 
\textbf{convex} (\textbf{with respect to} $\sqsubset$) if for all 
$x, y, z \in M$: if $x \sqsubset z \sqsubset y$ then $z \in M$ 
as well.

For sets $M$, $N$ of leaves put $M \sqsubset N$ iff 
for all $x \in M$ and all $y \in N$ we have $x \sqsubset 
y$. From (ot4) and (ot3) we derive the following:
%%
\begin{equation}
\label{eq:sqsubset}
x \sqsubset y \quad \Dpf \quad
[x] \sqsubset [y]
\end{equation}
%%
This property shows that the orderings on the leaves alone 
determines the relation $\sqsubset$ uniquely. 
%%
\begin{thm}
\label{erweiterung}
Let $\auf T, <\zu$ be a tree and $\sqsubset$ a linear
ordering on its leaves. Then there exists exactly
one relation $\sqsubset'\; \supseteq \;\sqsubset$ such that
$\auf T, <, \sqsubset'\zu$ is an ordered tree.
\end{thm}
%%
We emphasize that the ordering $\sqsubset'$ cannot be linear
if the tree has more than one element. It may even happen that
$\sqsubset'\; = \; \sqsubset$. One can show that overlapping
nodes can never be comparable with respect to $\sqsubset$. For let
$x \circ y$, say $x \leq y$.  Let $u \leq x$ be a leaf. Assume
$x \sqsubset y$; then by (ot3) $u \sqsubset y$ as well as
$u \sqsubset u$. This contradicts the condition that $\sqsubset$
is irreflexive. Likewise $y \sqsubset x$ cannot hold. So, nodes
can only be comparable if they do not overlap. We now ask: is
it possible that they are comparable exactly when they do not
overlap? In this case we call $\sqsubset$ \textbf{exhaustive}.
%%%
\index{ordering!exhaustive}%%
%%%
Theorem~\ref{thm:exhaustive} gives a criterion on the existence 
of exhaustive orderings. Notice that if $M$ and $N$ are convex 
sets, then so is $M \cap N$. Moreover, if $M \cap N = \varnothing$ 
then either $M \sqsubset N$ or $N \sqsubset M$. Also, $M$ is convex 
iff for all $u$: $u \sqsubset M$ or $M \sqsubset u$. 
%%
\begin{thm}
\label{thm:exhaustive}
Let $\auf T, <\zu$ be a tree and $\sqsubset$ a linear
ordering on the leaves. There exists an exhaustive
extension of $\sqsubset$ iff all constituents are 
continuous.
\end{thm}
%%
\proofbeg
By Theorem~\ref{erweiterung} there exists a unique extension, 
$\sqsubset'$. Assume that all constituents are continuous. Let $x$ 
and $y$ are nonoverlapping nodes. Then $[x] \cap [y] = \varnothing$.
Hence $[x] \sqsubset [y]$ or $[y] \sqsubset [x]$. since both sets 
are convex. So, by \eqref{eq:sqsubset} we have $x \sqsubset' y$ or 
$y \sqsubset' x$.  The ordering is therefore exhaustive. Conversely, 
assume that $\sqsubset'$ is exhaustive. Pick $x$. We show that $[x]$ 
is convex. Let $u$ be a leaf and $u \not\in [x]$. Then $u$ does not 
overlap with $x$.  By hypothesis, $u \sqsubset' x$ or $x \sqsubset' u$, 
whence $[u] \sqsubset [x]$ or $[x] \sqsubset [u]$, by \eqref{eq:sqsubset}. 
This means nothing but that either $u \sqsubset y$ for all 
$y \in [x]$ or $y \sqsubset u$ for all $y \in [x]$. So, $[x]$ is
convex.
\proofend
%%
\begin{lem}[Constituent Lemma]
%%%
\index{Constituent Lemma}%%%
%%%
Assume $\auf T, <, \sqsubset, \ell\zu$ is an exhaustively
ordered $A$--tree. Furthermore, let $p < q$. Then there is a
context $C = \auf \vec{u}, \vec{v}\zu$ such that
%%
\begin{equation}
k(q) = C(k(p)) = \vec{u} \conc k(p) \conc \vec{v} 
\end{equation}
%%
\end{lem}
%%
The converse does not hold. Furthermore, it may happen that
$C = \auf\varepsilon, \varepsilon\zu$ --- in which case
$k(q) = k(p)$ --- without $q < p$.
%
%
\begin{prop}
Let $\auf T, <, \sqsubset\zu$ be an ordered tree and $x \in T$.
$x$ is 1--branch\-ing iff $[x] = [y]$ for some $y < x$.
\end{prop}
%%
\proofbeg
Let $x$ be a 1--branching node with daughter $y$. Then we have
$[x] = [y]$ but $x \neq y$. So, the condition is necessary. Let
us show that is sufficient. Let $x$ be minimally 2--branching. Let 
$u < x$. There is a daughter $z \prec x$ such that $u \leq z$, 
and there is $z' \prec x$ different from $z$. Then $[u] \subseteq 
[z] \subseteq [x]$ as well as $[z'] \subseteq [x]$. All sets 
are nonempty and $[z'] \cap [z] = \varnothing$. Hence 
$[z] \subsetneq [x]$ and so also $[u] \subsetneq [x]$.
\proofend

%%%
\index{tree!properly branching}%%
%%%
We say that a tree is \textbf{properly branching} if it has
no 1--branching nodes.

There is a slightly different method of defining trees.
Let $T$ be a set and $\prec$ a cycle free relation on $T$
such that for every $x$ there is at most one $y $ such that
$x \prec y$. And let there be exactly one $x$ which has no
$\prec$--successor (the root). Then put $< := \prec^+$.
$\auf T, <\zu$ is a tree. And $x \prec y$ iff
$x$ is the daughter of $y$. Let $D(x)$ be the set of daughters
of $x$. Now let $P$ be a relation such that (a) $y\; P\; z$ only 
if $y$ and $z$ are sisters, (b) $P^+$, the transitive closure 
of $P$, is a relation that linearly orders $D(x)$ for every $x$, 
(c) for every $y$ there is at most one $z$ such that $y \; P \; z$ 
and at most one $z'$ such that $z' \; P\; y$.  Then put 
$x \sqsubset y$ iff there is $z$ such that 
(a) $x < \wht{x} \prec z$ for some $\wht{x}$, (b)
$y < \wht{y} \prec y$ for some $\wht{y}$, (c)
$\wht{x}\; P^+\; \wht{y}$. $\prec$ and $P$ are the immediate
neighbourhood relations in the tree.
%%
\begin{prop}
Let $\auf T, <, \sqsubset\zu$ be an exhaustively ordered tree.
Then $x \sqsubset y$ iff there are $x' \geq x$ and
$y' \geq y$ which are sisters and $x' \sqsubset y'$.
\end{prop}
%%
Finally we mention a further useful concept, that of a constituent
structure.
%%
\begin{defn}
%%%
\index{constituent structure}%%
%%%
Let $M$ be a set. A \textbf{constituent structure over}
$M$ is a system $\GC$ of subsets of $M$
with the following properties.
%%%
\renewcommand{\labelenumi}{\mbox{\rm (cs\arabic{enumi})}}
%%%
\begin{enumerate}
\item
$\{x\} \in \GC$ for every $x \in M$,
\item
$\varnothing \not\in \GC$, $M \in \GC$,
\item
if $S, T \in \GC$ and $S \nsubseteq T$ as well as
$T \nsubseteq S$ then $S \cap T = \varnothing$.
\end{enumerate}
%%
\renewcommand{\labelenumi}{\arabic{enumi}.}
%%
\end{defn}
%%
\begin{prop}
Let $M$ be a nonempty set.
There is a biunique correspondence between finite constituent
structures over $M$ and finite properly branching trees whose
set of leaves is $\{\{x\} : x \in M\}$.
\end{prop}
%%
\proofbeg
Let $\auf M, \GC\zu$ be a constituent structure.
Then $\auf \GC, \subsetneq\zu$ is a tree. To see this, one has to check
that $\subsetneq$ is irreflexive and transitive and that it has a root.
This is easy. Further, assume that $S \subsetneq T, U$. Then
$U \cap T \supseteq S \neq \varnothing$, because of condition (cs2).
Moreover, because of (cs3) we must have $U \subseteq T$ or $T \subseteq U$.
This means nothing else than that $T$ and $U$ are comparable. The set
of leaves is exactly the set $\{\{x\} : x \in M\}$. Conversely, let
$\GT = \auf T, <\zu$ be a properly branching tree. Put
$M := b(\GT)$ and $\GC := \{[x] : x \in T\}$.  We claim that 
$\auf M, \GC\zu$ is a constituent structure. For (cs1),
notice that for every $u \in b(\GG)$, $[u] = \{u\} \in \GC$.
Further, for every $x$ $[x] \neq \varnothing$, since the tree is finite.
There is a root $r$ of $\GT$, and we have $[r] = M$. This shows
(cs2). Now we show (cs3). Assume that $[x] \nsubseteq [y]$ and
$[y] \nsubseteq [x]$. Then $x$ and $y$ are incomparable (and
different). Let $u$ be a leaf and $u \in [x]$, then we have $u \leq x$.
$u \leq y$ cannot hold since $\uppx{u}$ is linear, and then
$x$ and $y$ would be comparable. Likewise we see that from
$u \leq y$ we get $u \nleq x$. Hence $[x] \cap [y] = \varnothing$.
The constructions are easily seen to be inverses of each other 
(up to isomorphism).
\proofend

In general we can assign to every tree a constituent structure,
but only if the tree is properly branching it can be properly
reconstructed from this structure. The notion of a constituent
structure can be extended straightforwardly to the notion of an 
ordered constituent structure, and we can introduce labellings. 

We shall now discuss the representation of terms by means of trees.
There are two different methods, both widely used. Before we begin,
we shall introduce the notion of a tree domain.
%%
\begin{defn}
%%%
\index{tree domain}%%
%%%
Let $T \subseteq \omega^{\ast}$ be a set of finite sequences of 
natural numbers. $T$ is called a \textbf{tree domain} if the 
following holds.
%%
\renewcommand{\labelenumi}{\mbox{\rm (td\arabic{enumi})}}
%%
\begin{enumerate}
\item If $\vec{x} \conc i \in T$ then $\vec{x} \in T$.
\item If $\vec{x} \conc i \in T$ and $j < i$ then
    also $\vec{x} \conc j \in T$.
\end{enumerate}
%%
\renewcommand{\labelenumi}{\arabic{enumi}.}
%%
\end{defn}
%%
We assign to a tree domain $T$ an ordered tree in the following way.
The set of nodes is $T$, (1) $\vec{x} < \vec{y}$ iff 
$\vec{y}$ is a proper prefix of $\vec{x}$ and (2) $\vec{x} \sqsubset 
\vec{y}$ iff there are numbers $i, j$ and sequences $\vec{u}$, 
$\vec{v}$, $\vec{w}$ such that (a) $i < j$ and 
(b) $\vec{x} = \vec{u} \conc i \conc \vec{v}$, 
$\vec{y} = \vec{u} \conc j \conc \vec{w}$. (This is exactly the 
lexicographical ordering.) Together with these relations, $T$ is an 
exhaustively ordered finite tree, as is easily seen. 
Figure~\ref{fig:baumbereich} shows the tree domain
$T = \{\varepsilon, 0,1,2, 10, 11, 20, 200\}$.
%%
\begin{figure}
\begin{center}
\begin{picture}(10,16)
\put(8,2.5){\makebox(0,0){200}}
    \put(8,3){\line(0,1){3}}
\put(8,6.5){\makebox(0,0){20}}
    \put(8,7){\line(0,1){3}}
\put(8,10.5){\makebox(0,0){2}}
    \put(8,11){\line(-1,1){3}}
\put(5,14.5){\makebox(0,0){$\varepsilon$}}
    \put(5,14){\line(0,-1){3}}
\put(5,10.5){\makebox(0,0){1}}
    \put(5,10){\line(-1,-1){3}}
\put(2,6.5){\makebox(0,0){10}}
    \put(5,10){\line(0,-1){3}}
\put(5,6.5){\makebox(0,0){11}}
    \put(5,14){\line(-1,-1){3}}
\put(2,10.5){\makebox(0,0){0}}
\end{picture}
\end{center}
\caption{A Tree Domain}
\label{fig:baumbereich}
\end{figure}
%%
If $T$ is a tree domain and $\vec{x} \in T$ then put
%%
%%%
\index{$T/\vec{x}$}%%%
%%%
\begin{equation}
T/\vec{x} := \{\vec{y} : \vec{x} \conc \vec{y} \in T\} 
\end{equation}
%%
This is the constituent below $\vec{x}$. (To be exact, it is not
identical to this constituent, it is merely isomorphic to it. The 
(unique) isomorphism from $T/\vec{x}$ onto the constituent 
$\low{\vec{x}}$ is the map $\vec{y} \mapsto \vec{x}\conc\vec{y}$.) 

Conversely, let $\auf T, <, \sqsubset\zu$ be an exhaustively ordered
tree. We define a tree domain $T^{\beta}$ by induction on the depth
of the nodes.  If $d(x) = 0$, let $x^{\beta} := \varepsilon$. In this
case $x$ is the root of the tree. If $x^{\beta}$ is defined, and $y$ 
a daughter of $x$, then put $y^{\beta} := x^{\beta} \conc i$, if $y$ 
is the $i$th daughter of $x$ counting from the left (starting, as usual, 
with 0). (Hence we have $| x^{\beta} | = d(x)$.) We can see quite 
easily that the so defined set is a tree domain. For we have
$\vec{u} \in T^{\beta}$ as soon as $\vec{u} \conc j \in T^{\beta}$
for some $j$. Hence (td1) holds. Further, if $\vec{u} \conc i \in
T^{\beta}$, say $\vec{u} \conc i = y^{\beta}$ then $y$ is the $i$th
daughter of a node $x$. Take $j < i$. Then let $z$ be the $j$th
daughter of $x$ (counting from the left). It exists, and we have
$z^{\beta} = \vec{u} \conc j$. Moreover, it can easily be shown 
that the relations defined on the tree domain are exactly the ones 
that are defined on the tree. In other words the map $x \mapsto % 
x^{\beta}$ is an isomorphism. 
%%
\begin{thm}
Let $\GT = \auf T, <, \sqsubset\zu$ be a finite, exhaustively
ordered tree. The function $x \mapsto x^{\beta}$ is an isomorphism
from $\GT$ onto the associated tree domain $\auf \GT^{\beta}, <, %
\sqsubset\zu$. Furthermore, $\GT \cong \GU$ iff
$\GT^{\beta} = \GU^{\beta}$.
\proofend
\end{thm}
%%
Terms can be translated into labelled tree domains.
Each term $t$ is assigned a tree domain $t^b$ and a labelling function
$t^{\lambda}$. The labelled tree domain associated with $t$
is $t^m := \auf t^b, t^{\lambda}\zu$. We start with the variables.
$x^b := \{\varepsilon\}$, and $x^{\lambda} \colon \varepsilon \mapsto x$.
Assume that the labelled tree domains $t_i^m$, $i < n-1$, are
defined, and put $n := \Omega(f)$. Let $s := f(t_0, \dotsc, t_{n-1})$;
then
%%
\begin{equation}
s^b := \{\varepsilon\} \cup
    \bigcup_{i < n} \{i \conc \vec{x} : \vec{x} \in t_i^b\} 
\end{equation}
%%
Then $s^{\lambda}$ is defined as follows.
%%
\begin{align}
s^{\lambda}(\varepsilon) & := f &
s^{\lambda}(j \conc \vec{x}) & := t_j^{\lambda}(\vec{x})
\end{align}
%%
This means that $s^m$ consists of a root named $f$ which has
$n$ daughters, to which the labelled tree domains of $t_0,
\dotsc, t_{n-1}$ are isomorphic. We call the representation 
which sends $t$ to $t^m$ the \textbf{dependency coding}.
%%%
\index{coding!dependency}%%
%%%
This coding is more efficient that the following, which we call
\textbf{structural coding}.
%%%
\index{coding!structural}%%
%%%
We choose a new symbol, $T$, and define
by induction to each term $t$ a tree domain $t^c$ and a
labelling function $t^{\mu}$. Put $x^c := \{\varepsilon, 0\}$,
$x^{\mu}(\varepsilon) := \mbox{\tt T}$, $x^{\mu}(0) := x$. Further
let for $s = f(t_0, \dotsc, t_{n-1})$
%%
\begin{equation}
\begin{split}
s^c & := \{\varepsilon, 0\} \cup
    \bigcup_{0 < i < n+1} \{i \conc \vec{x} : \vec{x} \in t_i^c\}
\\
s^{\mu}(\varepsilon) & := \mbox{\tt T} \\
s^{\mu}(0) & := f \\
s^{\mu}((j+1) \conc \vec{x}) & := t_j^{\mu}(\vec{x})
\end{split}
\end{equation}
%%
(Compare the structural coding with the associated string in the
notation without brackets.) In Figure~\ref{fig:kodierung} both
codings are shown for the term {\mtt (3+(5\symbol{42}7))} for
comparison. The advantage of the structural coding is that the
string associated to the labelled tree domain is also the string
associated to the term (with brackets dropped, as the tree encodes
the structure anyway).

{\it Notes on this section.} A variant of the dependency coding of 
syntactic structures has been proposed by Lucien Tesni\`ere 
%%%
\index{Tesni\`ere, Lucien}%%%
%%%
in \shortcite{tesniere:elements}. He called tree representations 
%%%
\index{stemma}\index{dependency syntax}%%%
%%%
\textbf{stemmata} (sg.\ {\it stemma}). This notation (and the theory 
surrounding it) became known as \textbf{dependency syntax}. See 
\cite{melcuk:dependency} for a survey. Unfortunately, the stemmata 
do not coincide with the dependency trees defined here, and this 
creates very subtle problems, see \cite{melcuk:dependency}.  
%%%
\index{Mel'\v{c}uk, Igor}%%%
%%%
Noam Chomsky 
%%%
\index{Chomsky, Noam}%%
%%%
on the other hand proposed the more elaborate structural 
coding, which is by now widespread in linguistic theory. 
%%
\begin{figure}
\begin{center}
\begin{picture}(13,15)
\put(2,10.5){\makebox(0,0){\tt 3}}
\put(2,11){\line(1,1){3}}
\put(5,14.5){\makebox(0,0){+}}
\put(5,14){\line(1,-1){3}}
\put(8,10.5){\makebox(0,0){\mtt\symbol{42}}}
\put(5,6.5){\makebox(0,0){\tt 5}}
\put(5,7){\line(1,1){3}}
\put(11,6.5){\makebox(0,0){\tt 7}}
\put(11,7){\line(-1,1){3}}
\end{picture}
\begin{picture}(16,15)
\put(5,6.5){\makebox(0,0){\tt 3}}
    \put(5,7){\line(0,1){3}}
\put(11,2.5){\makebox(0,0){\tt 5}}
    \put(11,3){\line(0,1){3}}
\put(14,2.5){\makebox(0,0){\tt 7}}
    \put(14,3){\line(0,1){3}}
\put(2,10.5){\makebox(0,0){\tt +}}
    \put(2,11){\line(1,1){3}}
\put(5,10.5){\makebox(0,0){\tt T}}
    \put(5,11){\line(0,1){3}}
\put(8,6.5){\makebox(0,0){\mtt\symbol{42}}}
    \put(8,7){\line(0,1){3}}
\put(11,6.5){\makebox(0,0){\tt T}}
    \put(11,7){\line(-1,1){3}}
\put(14,6.5){\makebox(0,0){\tt T}}
    \put(14,7){\line(-2,1){6}}
\put(8,10.5){\makebox(0,0){\tt T}}
    \put(8,11){\line(-1,1){3}}
\put(5,14.5){\makebox(0,0){\tt T}}
\end{picture}
\end{center}
\caption{Dependency Coding and Structural Coding}
\label{fig:kodierung}
\end{figure}
%%
%%
\vplatz
\exercise
Define `exhaustive ordering' on constituent structures.
Show that a linear ordering on the leaves is extensible
to an exhaustive ordering in a tree iff it is in 
the related constituent structure.
%%
\vplatz 
\exercise 
\label{ueb:schwester} 
Let $\GT = \auf T, <\zu$
be a tree and $\sqsubset$ a binary relation such that $x \sqsubset
y$ only if $x, y$ are daughters of the same node (that is, they
are sisters). Further, the daughter nodes of a given node shall be
ordered linearly by $\sqsubset$. No other relations shall hold.
Show that this ordering can be extended to an exhaustive ordering
on $\GT$.
%%
\vplatz
\exercise
Show that the number of binary branching exhaustively ordered
trees over a given string is exactly
%%
\begin{equation}
C_n = \frac{1}{n+1} {2n \choose n}
\end{equation}
%%
These numbers are called \textbf{Catalan numbers}.
%%%
\index{Catalan numbers}%%
%%%
% (Siehe auch Aigner: {\it Diskrete Mathematik}, S.~172.)
%%
\vplatz
\exercise
Show that $C_n < \frac{1}{n+1} 4^n$.
(One can prove that ${2n \choose n}$ approximates the
series $\frac{4^n}{\sqrt{\pi n}}$ in the limit.
%%%
%\nocite{heuser:analysis1}%%
%%%
The latter even majorizes the former. For the exercise there
is an elementary proof.)
%%
\vplatz
\exercise
Let $L$ be finite with $n$ elements and $<$ a linear ordering on
$L$. Construct an isomorphism from $\auf L, <\zu$ onto $\auf n, \in\zu$.

 \section{Rewriting Systems}
\label{kap1-5}
\label{einsdrei}
%
%
%
Languages are by Definition~\ref{defn:sprache} arbitrary sets of
strings over a (finite) alphabet. However, languages that
interest us here are those sets which can be described by finite 
means, particularly by finite processes. These can be processes 
which generate strings directly or by means of some intermediate 
structure (for example, labelled trees). The most popular approach 
is by means of rewrite systems on strings.
%%
\begin{defn}
%%%
\index{semi Thue system}%%
\index{$\Pf_{T}$, $\Pf_{T}^n$, $\Pf_{T}^{\ast}$}%%
%%%
Let $A$ be a set. A \textbf{semi Thue system}
over $A$ is a finite set $T = \{\auf \vec{x}_i,
\vec{y}_i\zu : i < m\}$ of pairs of $A$--strings.
If $T$ is given, write $\vec{u} \Pf^1_T \vec{v}$
if there are $\vec{s}, \vec{t}$ and some $i < m$
such that $\vec{u} = \vec{s} \conc \vec{x}_i \conc \vec{t}$
and $\vec{v} = \vec{s} \conc \vec{y}_i \conc \vec{t}$.
We write $\vec{u} \Pf^0_T \vec{v}$ if $\vec{u} = \vec{v}$, 
and $\vec{u} \Pf^{n+1}_T \vec{v}$ if there is a
$\vec{z}$ such that $\vec{u} \Pf^1_T \vec{z}
\Pf^n_T \vec{v}$. Finally, we write $\vec{u}
\Pf_T^{\ast} \vec{v}$ if $\vec{u} \Pf^n_T \vec{v}$
for some $n \in \omega$, and we say  that $\vec{v}$
is \textbf{derivable in $T$ from $\vec{u}$}.
\end{defn}
%%
We can define $\Pf^1_T$ also as follows.
$\vec{u} \Pf^1_T \vec{v}$ iff
there exists a context $C$ and $\auf \vec{x},
\vec{y}\zu \in T$ such that $\vec{u} = C(\vec{x})$ and
$\vec{v} = C(\vec{y})$. A semi Thue system $T$ is called 
a \textbf{Thue system}
%%%
\index{Thue system}%%
%%%
if from $\auf \vec{x}, \vec{y}\zu \in T$ follows
$\auf \vec{y}, \vec{x}\zu \in T$. In this case
$\vec{v}$ is derivable from $\vec{u}$ iff
$\vec{u}$ is derivable from $\vec{v}$.
%%%
\index{derivation}%%
%%%
A \textbf{derivation of} $\vec{y}$ \textbf{from} $\vec{x}$ 
\textbf{in} $T$ is a finite sequence $\auf \vec{v}_i : i < n+1\zu$ 
such that $\vec{v}_0 = \vec{x}$, $\vec{v}_n = \vec{y}$ and for 
all $i < n$ we have $\vec{v}_i \Pf^1_{T} \vec{v}_{i+1}$.  The 
\textbf{length} of this derivation is $n$. (A more careful 
definition will be given on Page~\pageref{derivation}.) Sometimes 
it will be convenient to admit $\vec{v}_{i+1} = \vec{v}_i$ even 
if there is no corresponding rule. 

A {\it grammar\/} differs from a semi Thue system as follows.
First, we introduce a distinction between the alphabet proper
and an auxiliary alphabet, and secondly,  the language is defined
by means of a special symbol, the so called {\it start symbol}.
%%
\begin{defn}
%%%
\index{grammar}%%
\index{start symbol}%%
\index{rule}%%
\index{symbol!terminal}%%
\index{symbol!nonterminal}%%
%%%
A \textbf{grammar} is a quadruple $G = \auf S, N, A, R\zu$
such that $N, A$ are nonempty disjoint sets, $S \in N$ and $R$ a
semi Thue system over $N \cup A$ such that
$\auf \vec{\gamma}, \vec{\eta}\zu \in R$ only if
$\vec{\gamma} \not\in A^{\ast}$. We call $S$ the
\textbf{start symbol}, $N$ the \textbf{nonterminal alphabet},
$A$ the \textbf{terminal alphabet} and $R$ the \textbf{set of rules}.
\end{defn}
%%
Elements of the set $N$ are also called \textbf{categories}. 
%%%
\index{category}%%
%%%
Notice that often the word `type' is used instead of `category', 
but this usage is dangerous for us in view of the fact that `type' 
is reserved here for types in the $\lambda$--calculus. 
As a rule, we choose $S = \mbox{\tt S}$. This is not necessary.
The reader is warned that {\tt S} need not always be the start
symbol. But if nothing else is said it is. As is common practice,
nonterminals are denoted by upper case Roman letters, terminals by
lower case Roman letters. A lower case Greek letter signifies
a letter that is either terminal or nonterminal. The use of
vector arrows follows the practice established for strings. We
write $G \vdash \vec{\gamma}$ or $\vdash_G \vec{\gamma}$ in case
%%%%
\index{$G \vdash \vec{x}$, $\vdash_G \vec{x}$}%%%
%%%%
that $S \Pf^{\ast}_R \vec{\gamma}$ and say that $G$ \textbf{generates}
$\vec{\gamma}$. Furthermore, we write $\vec{\gamma} \vdash_G
\vec{\eta}$ if $\vec{\gamma} \Pf^{\ast}_R \vec{\eta}$. The
language generated by $G$ is defined by
%%
%%%
\index{$L(G)$}%%
%%%%
\begin{equation}
L(G) := \{\vec{x} \in A^{\ast} : G \vdash \vec{x}\} 
\end{equation}
%%
Notice that $G$ generates strings which may contain terminal
as well as nonterminal symbols. However, those that contain also
nonterminals do not belong to the language that $G$ generates.
A grammar is therefore a semi Thue system which additionally 
defines how a derivation begins and how it ends.

Given a grammar $G$ we call the \textbf{analysis problem}
%%%
\index{analysis problem}%%
\index{parsing problem}%%
%%%
(or \textbf{parsing problem}) for $G$ the problem (1) to say for
a given string whether it is derivable in $G$ and (2) to name
a derivation in case that a string is derivable. The problem (1)
alone is called the \textbf{recognition problem for} $G$.
%%%
\index{recognition problem}%%
%%%

%%%
\index{production}%%
%%%
A rule $\auf \vec{\alpha}, \vec{\beta}\zu$ is often also called a
\textbf{production} and is alternatively written 
%%%
\index{$\vec{\alpha} \pf \vec{\beta}$}%%
%%%
$\vec{\alpha} \pf \vec{\beta}$. We call $\vec{\alpha}$ the 
\textbf{left hand side} and $\vec{\beta}$ the \textbf{right hand 
side} of the production. The 
%%%%
\index{productivity}%%%
%%%%
\textbf{productivity} $p(\rho)$ of a rule $\rho = \vec{\alpha} \pf %
\vec{\beta}$ is the
difference $|\vec{\beta}| - |\vec{\alpha}|$. $\rho$ is called
%%%
\index{production!expanding}%%
\index{production!strictly expanding}%%
\index{production!contracting}%%
\index{rule!terminal}%%
%%%
\textbf{expanding} if $p(\rho) \geq 0$, \textbf{strictly expanding}
if $p(\rho) > 0$ and \textbf{contracting} if $p(\rho) < 0$.
A rule is \textbf{terminal} if it has the form $\vec{\alpha} \pf
\vec{x}$ (notice that by our convention, $\vec{x} \in A^{\ast}$).

This notion of grammar is very general. There are only countably
many grammars over a given alphabet --- and hence only countably
many languages generated by them ---; nevertheless, the variety
of these languages is bewildering. We shall see that every
recursively enumerable language can be generated by some grammar.
So, some more restricted notion of grammar is called for. Noam
Chomsky has proposed the following hierarchy of grammar types.
%%%
\index{$X_{\varepsilon}$}%%
%%%
(Here, $X_{\varepsilon}$ is short for $X \cup \{\varepsilon\}$.)
%%
\begin{dinglist}{43}
%%
\index{grammar!of Type 0,1,2,3}%%
\index{grammar!context sensitive}%%
\index{grammar!context free}%%
\index{grammar!regular}%%
%%%
\item
Any grammar is of \textbf{Type 0}.
\item
A grammar is said to be of \textbf{Type 1} or \textbf{context sensitive}
if all rules are of the form $\vec{\delta}_1 X \vec{\eta}_2 \pf
\vec{\eta}_1 \vec{\alpha} \vec{\eta}_2$ and either (i) always
$\vec{\alpha} \neq \varepsilon$ or (ii) $\mbox{\tt S} \pf \varepsilon$ 
is a rule and {\tt S} never occurs on the right hand side of a production.
\item
A grammar is said to be of \textbf{Type 2} or \textbf{context free}
if it is context sensitive and all productions are of the form
$X \pf \vec{\alpha}$.
\item
A grammar is said to be of \textbf{Type 3} or \textbf{regular} if it is
context free and all productions are of the form $X \pf
\vec{\alpha}$ where $\vec{\alpha} \in A_{\varepsilon} \cdot
N_{\varepsilon}$.
\end{dinglist}
%%
A context sensitive rule $\vec{\eta}_1 X\vec{\eta}_2 \pf
\vec{\eta}_1\vec{\alpha}\vec{\eta}_2$ is also written
%%
\begin{equation}
X \pf \vec{\alpha} /\vec{\eta}_1\underline{\quad}\vec{\eta}_2
\end{equation}
%%
One says that $X$ can be rewritten into $\vec{\alpha}$ \textbf{in 
the context} $\vec{\eta}_1\underline{\quad}\vec{\eta}_2$.
A {\it language\/} is said to be of \textbf{Type} $i$ 
%%%%
\index{language!of Type 0,1,2,3}%%%
%%%
if it can be
generated by a grammar of Type $i$. It is not relevant if there
also exists a grammar of Type $j$, $j \neq i$, that generates this
language in order for it to be of Type $i$. We give examples of 
grammars of Type 3, 2 and 0.
%%%
\index{language!context free}%%%
\index{language!context sensitive}%%%
\index{language!regular}%%

{\sc Example 1.}
There are regular grammars which generate number expressions.
Here a number expression is either a number, with or without
sign, or a pair of numbers separated by a dot, again
with or without sign. The grammar is as follows. The set of
terminal symbols is $\{\mbox{\tt 0}, \mbox{\tt 1}, \mbox{\tt 2},
\mbox{\tt 3}, \mbox{\tt 4}, \mbox{\tt 5}, \mbox{\tt 6}, \mbox{\tt 7},
\mbox{\tt 8}, \mbox{\tt 9}, \mbox{\tt +}, \mbox{\tt -}, \mbox{\tt .}\}$,
the set of nonterminals is $\{\mbox{\tt V},
\mbox{\tt Z}, \mbox{\tt F}, \mbox{\tt K}, \mbox{\tt M}\}$.
The start symbol is {\tt V} and the productions are
%%
\begin{equation}
\begin{split}
\mbox{\tt V} & \pf \mbox{\tt +Z} \mid \mbox{\tt -Z} \mid 
	\mbox{\tt Z} \\
\mbox{\tt Z} & \pf \mbox{\tt 0Z} \mid \mbox{\tt 1Z} \mid 
	\mbox{\tt 2Z} \mid \dotsb \mid \mbox{\tt 9Z} \mid 
	\mbox{\tt F} \\
\mbox{\tt F} & \pf \mbox{\tt 0} \mid \mbox{\tt 1} \mid \mbox{\tt 2}
    \mid \dotsb \mid \mbox{\tt 9} \mid \mbox{\tt K} \\
\mbox{\tt K} & \pf \mbox{\tt .M} \\
\mbox{\tt M} & \pf \mbox{\tt 0M} \mid \mbox{\tt 1M} \mid 
	\mbox{\tt 2M} \mid \dotsb \mid \mbox{\tt 9M}
    \mid \mbox{\tt 0} \mid \mbox{\tt 1}
    \mid \mbox{\tt 2} \mid \dotsb \mid \mbox{\tt 9}
\end{split}
\end{equation}
%%
Here, we have used the following convention. The symbol `$\mid$' on
the right hand side of a production indicates that the part on
the left of this sign and the one to the right are alternatives.
So, using the symbol `$\mid$' saves us from writing two rules
expanding the same symbol. For example, {\tt V} can be expanded
either by {\tt +Z}, {\tt -Z} or by {\tt Z}. The syntax of the
language {\rm ALGOL} 
%%%
\index{ALGOL}%%
%%%
has been written down in this notation,
which became to be known as the \textbf{Backus--Naur Form}. The
arrow was written `$::=$'. (The Backus--Naur form actually allowed 
for context--free rules.)
%%
\index{Backus--Naur form}%%
%%%

{\sc Example 2.}
The set of strings representing terms over a finite signature with
finite set $X$ of variables can be generated by a context free
grammar. Let $F = \{\mbox{\tt F}_i : i < m\}$ and
$\Omega(i) := \Omega(\mbox{\tt F}_i)$.
%%
\begin{equation}
\mbox{\tt T} \quad\pf\quad \mbox{\tt F}_i\mbox{\tt T}^{\Omega(i)} 
\qquad (i < m)
\end{equation}
%%
Since the set of rules is finite, so must be $F$.
The start symbol is {\tt T}. This grammar generates the associated
strings in 
%%%
\index{Polish Notation}%%%
%%%
Polish Notation. Notice that this grammar reflects
exactly the structural coding of the terms. More on that later.
If we want to have dependency coding, we  have to choose instead
the following grammar.
%%
\begin{equation}
\begin{split}
\mbox{\tt S} & \pf \mbox{\tt F}_{j_0}\mbox{\tt F}_{j_1}\dotso
\mbox{\tt F}_{j_{\Omega(i)-1}} \\
\mbox{\tt F}_i & \pf \mbox{\tt F}_{j_0}\mbox{\tt F}_{j_1}\dotso
\mbox{\tt F}_{j_{\Omega(i)-1}}
\end{split}
\end{equation}
%%
This is a scheme of productions. Notice that for technical reasons 
the root symbol must be {\tt S}. We could dispense with the first 
kind of rules if we are allowed to have several start symbols. We 
shall return to this issue below.

{\sc Example 3.}
Our example for a Type 0 grammar is the following, taken from
\cite{salomaa:formal}.
%%
\begin{equation}
\begin{array}{l@{\qquad}r@{\quad \pf \quad}l@{\qquad\qquad}%
r@{\quad \pf \quad}l}
{\rm (a)} & \mbox{\tt X}_{\snull}   & \mbox{\tt a}
    & \mbox{\tt X}_{\snull}   & \mbox{\tt aXX}_{\szwei} \mbox{\tt Z} \\
{\rm (b)} & \mbox{\tt X}_{\szwei} \mbox{\tt Z} & \mbox{\tt aa}
    & \multicolumn{2}{c}{} \\
{\rm (c)} & \mbox{\tt Xa}    & \mbox{\tt aa}
    & \mbox{\tt Ya}    & \mbox{\tt aa} \\
{\rm (d)} & \mbox{\tt X}_{\szwei} \mbox{\tt Z} & \mbox{\tt Y}_{\seins} \mbox{\tt YXZ}
    & \multicolumn{2}{c}{} \\
{\rm (e)} & \mbox{\tt XY}_{\seins} & \mbox{\tt Y}_{\seins} \mbox{\tt YX}
    & \mbox{\tt YY}_{\seins} & \mbox{\tt Y}_{\seins} \mbox{\tt Y} \\
%{\rm (f)} & \mbox{\tt XY}_{\seins} & \mbox{\tt X}_{\seins} \mbox{\tt Y}
%    & \mbox{\tt YY}_{\seins} & \mbox{\tt Y}_{\seins} \mbox{\tt Y} \\
{\rm (f)} & \mbox{\tt aY}_{\seins} & \mbox{\tt aXXYX}_{\szwei}
    & \multicolumn{2}{c}{} \\
{\rm (g)} & \mbox{\tt X}_{\szwei}\mbox{\tt Y}  & \mbox{\tt XY}_{\szwei}
    & \mbox{\tt Y}_{\szwei} \mbox{\tt Y} & \mbox{\tt YY}_{\szwei} \\
\multicolumn{3}{c}{}
    & \mbox{\tt Y}_{\szwei} \mbox{\tt X} & \mbox{\tt YX}_{\szwei}
\end{array}
\end{equation}
%%
$\mbox{\tt X}_{\snull}$ is the start symbol. This grammar generates
the language $\{\mbox{\tt a}^{n^2} : n > 0\}$. This can be seen
as follows. To start, with (a) one can either generate the string
{\tt a} or the string $\mbox{\tt aXX}_{\szwei}\mbox{\tt Z}$. Let
$\vec{\gamma}_i = \mbox{\tt aX}\vec{\delta}_i %
\mbox{\tt X}_{\szwei} \mbox{\tt Z}$,
$\vec{\delta}_i \in \{\mbox{\tt X}, \mbox{\tt Y}\}^{\ast}$.
We consider derivations which go from $\vec{\gamma}_i$ to a
terminal string. At the beginning, only (b) or (d) can be applied.
Let it be (b). Then we can only continue with (c) and then we
create a string of length $4 + |\vec{\delta}_i|$. Since we
have only one letter, the string is uniquely determined.
Now assume that (d) has been chosen. Then we get the string
$\mbox{\tt aX}\vec{\delta}_i\mbox{\tt Y}_{\seins} \mbox{\tt YXZ}$.
The only possibility to continue is using (e). This moves the
index 1 stepwise to the left and puts {\tt Y} before every occurrence
of an {\tt X}. Finally, it hits {\tt a} and we use (f) to get 
$\mbox{\tt aXXYX}_{\szwei} \vec{\delta}'_i\mbox{\tt YYXZ}$. Now there 
is no other choice but to move the index 2 to the right with the help 
of (g). This gives a string $\vec{\gamma}_{i+1} = 
\mbox{\tt aX}\vec{\delta}_{i+1}\mbox{\tt X}_{\szwei}\mbox{\tt Z}$ with 
$\vec{\delta}_{i+1} = \mbox{\tt XYX}\vec{\delta}'_i\mbox{\tt YY}$. 
We have
%%
\begin{equation}
|\vec{\delta}_{i+1}| = |\vec{\delta}_i| + \ell_x(\vec{\delta}_i)
+ 5
\end{equation}
%%
where $\ell_x(\delta_i)$ counts the number of {\tt X} in
$\vec{\delta}_i$. Since $\ell_x(\vec{\delta}_{i+1}) = \ell_x(\vec{\delta}_i)
+ 2$, $\vec{\delta}_0 = \varepsilon$, we conclude that 
$\ell_x(\vec{\delta}_i) = 2i$ and so $|\vec{\delta}_i| = (i+1)^2 - 4$, 
$i > 0$.  Hence, $|\vec{\gamma}_i| = (i+1)^2$, as promised. 

In the definition of a context sensitive grammar the following
must be remembered. By intention, context sensitive grammars only
consist of noncontracting rules. However, since we must begin with a
start symbol, there would be no way to derive the empty string if
no rule is contracting. Hence, we do admit the rule $\mbox{\tt S} %
\pf \varepsilon$. But in order not to let other contracting uses of 
this rule creep in we require that {\tt S} is not on the right hand 
side of any rule whatsoever. Hence, $\mbox{\tt S} \pf \varepsilon$ 
can only be applied once, at the beginning of the derivation. The 
derivation immediately terminates. This condition is also in force 
for context free and regular grammars although without it no more 
languages can be generated (see the exercises). For assume that in a 
grammar $G$ with rules of the form $X \pf \vec{\alpha}$ there
are rules where {\tt S} occurs on the right hand side of a production,
and nevertheless replace {\tt S} by $Z$ in all rules which are not not
of the form $\mbox{\tt S} \pf \varepsilon$. Add also all rules 
$\mbox{\tt S} \pf \vec{\alpha}'$, where $\mbox{\tt S} \pf \vec{\alpha}$ 
is a rule of $G$ and $\vec{\alpha}'$ results from $\vec{\alpha}$ by 
replacing {\tt S} by $Z$. This is a context free grammar which generates 
the same language, and even the same structures. (The only difference 
is with the nodes labelled {\tt S} or $Z$.)

%%%
\index{RG, CFG, CSG, GG}%%
\index{RL, CFL, CSL, GL}%%
%%%
The class of regular grammars is denoted by RG,  the class of all
context free grammars by CFG, the class of context sensitive
grammars by CSG and the class of Type 0 grammars by GG.
The languages generated by these grammars is analogously denoted by
RL, CFL, CSL and GL. The grammar classes form a
proper hierarchy.
%%
\begin{equation}
\mbox{\rm RG} \subsetneq \mbox{\rm CFG} \subsetneq \mbox{\rm CSG}
\subsetneq \mbox{\rm GG}
\end{equation}
%%
This is not hard to see. It follows immediately that the languages
generated by these grammar types also form a hierarchy, but not 
that the inclusions are proper. However, the hierarchy is once again 
strict.
%%
\begin{equation}
\mbox{\rm RL} \subsetneq \mbox{\rm CFL} \subsetneq \mbox{\rm CSL}
\subsetneq \mbox{\rm GL}
\end{equation}
%%
We shall prove each of the proper inclusions. In 
Section~\ref{kap1}.\ref{einsfuenf}
(Theorem~\ref{0-1-echt}) we shall show that there are languages of
Type 0 which are not of Type 1. Furthermore, from the Pumping Lemma
(Theorem~\ref{thm:pumplemma}) for CFLs it follows
that $\{\mbox{\tt a}^n \mbox{\tt b}^n \mbox{\tt c}^n : n \in \omega\}$
is not context free. However, it is context sensitive (which is left
as an exercise in that section). Also, by Theorem~\ref{thm:noncontracting} 
below, the language $\{\mbox{\tt a}^{n^2} : n \in \omega\}$ has a grammar 
of Type 1. However, this language is not semilinear, whence it is not 
of Type 2 (see Section~\ref{kap2}.\ref{kap2-5}). Finally, it will be shown that
$\{\mbox{\tt a}^n \mbox{\tt b}^n : n \in \omega\}$
is context free but not regular.  (See Exercise~\ref{ue:index}.)

Let $\rho = \vec{\gamma} \pf \vec{\eta}$.
We call a triple $A = \auf \vec{\alpha}, C, \vec{\zeta}\zu$
%%%
\index{rule!instance of a \faul}%%
%%%
an \textbf{instance of} $\rho$ if $C$ is an occurrence of
$\vec{\gamma}$ in $\vec{\alpha}$ and also an occurrence of
$\vec{\eta}$ in $\vec{\zeta}$.  This means that there exist 
$\vec{\kappa}_1$ and $\vec{\kappa}_2$ such that
$C = \auf \vec{\kappa}_1, \vec{\kappa}_2\zu$ and
$\vec{\alpha} = \vec{\kappa}_1 \conc \vec{\gamma}
\conc \vec{\kappa}_2$ as well as $\vec{\zeta} =
\vec{\kappa}_1 \conc \vec{\eta} \conc \vec{\kappa}_2$.
%%%
\index{rule instance!domain of a \faul}%%
\label{derivation}
%%%
We call $C$ the \textbf{domain of} $A$. A \textbf{derivation of length} 
$n$ is a sequence $\auf A_i : i < n\zu$ of {\it instances\/} of 
rules from $G$ such that $A_i = \auf \vec{\alpha}_i, C_i,
\vec{\zeta}_i\zu$ for $i < n$ and for every $j < n-1$ 
$\vec{\alpha}_{j+1} = \vec{\zeta}_j$. $\vec{\alpha}_0$ is 
called the \textbf{start} of the derivation, $\vec{\zeta}_{n-1}$ 
the \textbf{end}. 
%%%
\index{derivation}%%
\index{derivation!start of a \faul}%%
\index{derivation!end of a \faul}%%
\index{$\der(G,\vec{\alpha})$, $\der(G)$}%%%
%%%
We denote by $\der(G,\vec{\alpha})$ the set of derivations
$G$ from the string $\vec{\alpha}$ and $\der(G) := \der(G,S)$. 

This definition has been carefully chosen. Let $\auf A_i : i < n\zu$
be a derivation in $G$, where $A_i = \auf \vec{\alpha}_i, C_i, 
\vec{\alpha}_{i+1}\zu$ ($i < n$). Then we call
$\auf \vec{\alpha}_i : i < n+1\zu$ the (\textbf{associated})
%%%
\index{string sequence}%%
\index{string sequence!associated}%%
%%%%
\textbf{string sequence}. Notice that the string sequence has one
more element than the derivation. In what is to follow we shall 
often also call the string sequence a derivation. However, this is not
quite legitimate, since the string sequence does not determine the
derivation uniquely. Here is an example. Let $G$ consist of the rules
$\mbox{\tt S} \pf \mbox{\tt AB}$, $\mbox{\tt A} \pf \mbox{\tt AA}$ 
and $\mbox{\tt B} \pf \mbox{\tt AB}$.  Take the string sequence 
$\auf \mbox{\tt S}, \mbox{\tt AB}, \mbox{\tt AAB}\zu$. There are 
two derivations for this sequence. 
%%
\begin{subequations}
\begin{align}
& \auf\auf \mbox{\tt S}, \auf \varepsilon, \varepsilon\zu,
    \mbox{\tt AB}\zu, \auf \mbox{\tt AB},
    \auf \varepsilon, \mbox{\tt B}\zu, \mbox{\tt AAB}\zu\zu \\
& \auf\auf \mbox{\tt S}, \auf \varepsilon, \varepsilon\zu, 
	\mbox{\tt AB}\zu, \auf \mbox{\tt AB}, \auf \mbox{\tt A}, 
	\varepsilon\zu, \mbox{\tt AAB}\zu\zu
\end{align}
\end{subequations}
%%
After application of a rule $\rho$, the left hand side $\vec{\gamma}$
is replaced by the right hand side, but the context parts $\vec{\kappa}_1$
and $\vec{\kappa}_2$ remain as before. It is intuitively clear that
if we apply a rule to parts of the context, then this application
could be permuted with the first. This is clarified in the following
definition and theorem.
%%
\begin{defn}
%%%%
\index{domains!disjoint}%%
%%%%
Let $\auf \vec{\alpha}, \auf \vec{\kappa}_1, \vec{\kappa}_2\zu,
\vec{\beta}\zu$ be an instance of the rule 
$\rho = \vec{\eta} \pf \vec{\vartheta}$, and let 
$\auf \vec{\beta}, \auf \vec{\mu}_1, \vec{\mu}_2\zu, \vec{\gamma}\zu$
be an instance of $\sigma = \vec{\zeta} \pf \vec{\xi}$. We call the
domains of these applications \textbf{disjoint} if either
(a) $\vec{\kappa}_1\conc\vec{\vartheta}$ is a prefix
of $\vec{\mu}_1$ or (b) $\vec{\vartheta}\conc \vec{\kappa}_2$ is
a suffix of $\vec{\mu}_2$.
\end{defn}
%%%
\begin{lem}[Commuting Instances]
%%%
\index{Commuting Instances Lemma}%%%
%%%
\label{vertausch}
Let $\auf \vec{\alpha}, C, \vec{\beta}\zu$ be an instance of 
$\rho = \vec{\eta} \pf \vec{\vartheta}$, and 
$\auf \vec{\beta}, D, \vec{\gamma}\zu$
an instance of $\sigma = \vec{\zeta} \pf \vec{\xi}$. Suppose that the
instances are disjoint. Then there exists an instance
$\auf \vec{\alpha}, D', \vec{\delta}\zu$ of $\sigma$ as well as an
instance $\auf \vec{\delta}, C', \vec{\gamma}\zu$ of $\rho$,
and both have disjoint domains.
\end{lem}
%%
The proof is easy and left as an exercise. Analogously, suppose that
to the {\it same\/} string the rule $\rho$ can be applied with context 
$C$ and the rule $\sigma$ can be applied with context $D$. Then if 
$C$ precedes $D$, after applying one of them the domains remain 
disjoint, and the other can still be applied (with the context modified 
accordingly).

We give first an example where the instances are not disjoint. Let the 
following rules be given.
%%
\begin{equation}
\begin{array}{l@{\quad\pf\quad}l@{\qquad\qquad}l@{\quad\pf\quad}l}
\mbox{\tt AX} & \mbox{\tt XA} & \mbox{\tt XA} & \mbox{\tt Xa} \\
\mbox{\tt XB} & \mbox{\tt Xb} & \mbox{\tt Xa} & \mbox{\tt a}
\end{array}
\end{equation}
%%
There are two possibilities to apply the rules to {\tt AXB}.
The first has domain $\auf \varepsilon, \mbox{\tt B}\zu$,
the second the domain $\auf \mbox{\tt A}, \varepsilon\zu$. The
domains overlap and indeed the first rule when applied destroys
the domain of the second. Namely, if we apply the rule
$\mbox{\tt AX} \pf \mbox{\tt XA}$ we cannot reach a terminal
string.
%%
\begin{equation}
\mbox{\tt AXB} \Pf \mbox{\tt XAB} \Pf \mbox{\tt XaB}
\end{equation}
%%
If on the other hand we first apply the rule 
$\mbox{\tt XB} \pf \mbox{\tt Xb}$ we do get one.
%%
\begin{equation}
\mbox{\tt AXB} \Pf \mbox{\tt AXb} \Pf \mbox{\tt XAb}
 \Pf \mbox{\tt Xab} \Pf \mbox{\tt ab}
\end{equation}
%%
So much for noncommuting instances. Now take the string {\tt AXXB}.
Again, the two rules are in competition. However, this time none
destroys the applicability of the other.
%%
\begin{equation}
\mbox{\tt AXXB} \Pf \mbox{\tt AXXb} \Pf \mbox{\tt XAXb}
\end{equation}
%%
\begin{equation}
\mbox{\tt AXXB} \Pf \mbox{\tt XAXB} \Pf \mbox{\tt XAXb}
\end{equation}
%%
As before we can derive the string {\tt ab}.
Notice that in a CFG every pair of rules that are
in competition for the same string can be used in succession
with either order on condition that they do not compete for the
same occurrence of a nonterminal.
%%%
\begin{defn}
%%%
\index{standard form}%%%
%%%
A grammar is in \textbf{standard form} if all rules are of the
form $\vec{X} \pf \vec{Y}$, $X \pf \vec{x}$.
\end{defn}
%%%
In other words, in a grammar in standard form the right hand side 
either consists of a string of nonterminals or a string of terminals. 
Typically, one restricts terminal strings to a single symbol or the 
empty string, but the difference between these requirements is 
actually marginal.
%%%
\begin{lem}
For every grammar $G$ of Type $i$ there exists a grammar $H$ of
Type $i$ in standard form such that $L(G) = L(H)$.
\end{lem}
%%%
\proofbeg
Put $N' := \{\mbox{\tt N}_a : a \in A\} \cup N$ and $h :
a \mapsto \mbox{\tt N}_a, X \mapsto X : N \cup A \pf N^1$.
For each rule $\rho$ let $h(\rho)$ be the result of
applying $\oli{h}$ to both strings. Finally, let
$R' := \{h(\rho) : \rho \in R\} \cup \{\mbox{\tt N}_a \pf a : a
\in A\}$, $H := \auf \mbox{\tt S}, N', A, R'\zu$. It is easy to 
verify, using the Commuting Instances 
%%%
\index{Commuting Instances Lemma}%%%
%%%
Lemma, that $L(H) = L(G)$. (See also 
below for proofs of this kind.) 
\proofend

We shall now proceed to show that the conditions on Type 0 grammars
are actually insignificant as regards the class of generated languages.
First, we may assume a set of start symbols rather than a single one.
%%%
\index{grammar$^{\ast}$}%%
%%%%
Define the notion of a \textbf{\bf grammar}$^{\ast}$ (\textbf{of Type} 
$i$) to be a quadruple $G = \auf \Sigma, N, A, R\zu$ such that 
$\Sigma \subseteq N$ and for all $S \in \Sigma$, $\auf S, N, A, R\zu$ 
is a grammar (of Type $i$). Write $G \vdash \vec{\gamma}$ if there is 
an $S \in \Sigma$ such that $S \Pf^{\ast}_R \vec{\gamma}$. We shall 
see that grammars$^{\ast}$
are not more general than grammars with respect to languages.
Let $G$ be a grammar$^{\ast}$. Define $G^{\heartsuit}$ as follows.
Let $S^{\heartsuit} \not\in A \cup N$ be a new nonterminal and
add the rules $S^{\heartsuit} \pf X$ to $R$ for all
$X \in \Sigma$. It is easy to see that $L(G^{\heartsuit}) = L(G)$.
(Moreover, the derivations differ minimally.) Notice also that
we have not changed the type of the grammar.

The second simplification concerns the requirement that the set of
terminals and the set of nonterminals be disjoint. We shall show
that it too can be dropped without increasing the generative power.
We shall sometimes work without this condition, as it can be
cumbersome to deal with.
%%
\begin{defn}
%%%
\index{quasi--grammar}%%
%%%
A \textbf{quasi--grammar} is a quadruple $\auf \mbox{\tt S}, N, A, R\zu$
such that $A$ and $N$ are finite and nonempty sets, $\mbox{\tt S} \in N$,
and $R$ a semi Thue system over $N \cup A$ such that if
$\auf \vec{\alpha}, \vec{\beta}\zu \in R$ then
$\vec{\alpha}$ contains a symbol from $N$.
\end{defn}
%%
%%
\begin{prop}
For every quasi--grammar there exists a grammar which generates
the same language.
\end{prop}
%%
\proofbeg
Let $\auf \mbox{\tt S}, N, A, R\zu$ be a quasi--grammar.
Put $N_1 := N \cap A$. Then assume for every
$a \in N_1$ a new symbol $\mbox{\tt Y}_a$.
Put $Y := \{\mbox{\tt Y}_a : a \in N_1\}$,
$N^{\circ} := (N - N_1) \cup Y$, $A^{\circ} := A$.
Now $N^{\circ} \cap A^{\circ} = \varnothing$.
We put $\mbox{\tt S}^{\circ} := \mbox{\tt S}$
if $\mbox{\tt S} \not\in A$ and $\mbox{\tt S}^{\circ}
:= \mbox{\tt Y}_{\tt S}$ if $\mbox{\tt S} \in A$. Finally, we
define the rules. Let $\vec{\alpha}^{\circ}$ be the result of
replacing every occurrence of an $a \in N_1$ by
the corresponding $\mbox{\tt Y}_a$. Then let
%%
\begin{equation}
R^{\circ} := \{\vec{\alpha}^{\circ} \pf \vec{\beta}^{\circ} :
\vec{\alpha} \pf \vec{\beta} \in R\} \cup
\{\mbox{\tt Y}_a \pf a : a \in N_1\} 
\end{equation}
%%
Put $G^{\circ} := \auf \mbox{\tt S}^{\circ}, N^{\circ}, A^{\circ},
R^{\circ}\zu$. We claim that $L(G^{\circ}) = L(G)$.
To that end we define a homomorphism
$h \colon (A \cup N)^{\ast} \pf (A^{\circ} \cup N^{\circ})^{\ast}$
by $h(a) := a$ for $a \in A - N_1$, $h(a) := \mbox{\tt Y}_a$ for
$a \in N_1$ and $h(X) := X$ for all $X \in N - N_1$. Then
$h(\mbox{\tt S}) = \mbox{\tt S}^{\circ}$ as well as $h(R) %
\subseteq R^{\circ}$. From this it immediately follows that 
if $G \vdash \vec{\alpha}$ then $G^{\circ} \vdash h(\vec{\alpha})$. 
(Induction on the length of a derivation.)  
Since we can derive $\vec{\alpha}$ in $G^{\circ}$ from 
$h(\vec{\alpha})$, we certainly have $L(G) \subseteq L(G^{\circ})$. 
For the converse we have to convince ourselves that an instance of 
a rule $\mbox{\tt Y}_a \pf a$ can always be moved to the end of 
the derivation. For if
$\vec{\alpha} \pf \vec{\beta}$ is a rule then it is of type
$\mbox{\tt Y}_b \pf b$ and replaces a  $\mbox{\tt Y}_b$ by $b$;
and hence it commutes with that instance of the first rule.
Or it is of a different form, namely $\vec{\alpha}^{\circ} \pf %
\vec{\beta}^{\circ}$; since $a$ does not occur in $\vec{\alpha}^{\circ}$,
these two instances of rules commute. Now that this is shown,
we conclude from $G^{\circ} \vdash \vec{\alpha}$ already
$G^{\circ} \vdash \vec{\alpha}^{\circ}$.
This implies $G \vdash \vec{\alpha}$.
\proofend

The last of the conditions, namely that the left hand side of a
production must contain a nonterminal, is also no restriction.
For let $G = \auf \mbox{\tt S}, N, A, R\zu$ be a grammar which does
not comply with this condition. Then for every terminal $a$
let $a^1$ be a new symbol and let $A^1 := \{a^1 : a \in A\}$.
Finally, for each rule $\rho = \vec{\alpha} \pf \vec{\beta}$
let $\rho^1$ be the result of replacing every occurrence of
an $a \in A$ by $a^1$ (on every side of the production).
Now set $\mbox{\tt S}':= \mbox{\tt S}$ if $\mbox{\tt S} \not\in A$
and $\mbox{\tt S}' := \mbox{\tt S}^{\seins}$ otherwise, 
$R' := \{\rho^1 : \rho \in R\} \cup \{a^{\seins} \pf a : a \in A\}$.
Finally put $G' := \auf \mbox{\tt S}', N \cup A^1, A, R'\zu$.
It is not hard to show that $L(G') = L(G)$. These steps have 
simplified the notion of a grammar considerably. Its most general 
form is $\auf \Sigma, N, A, R\zu$, where $\Sigma \subseteq 
N$ is the set of start symbols and $R \subseteq (N \cup A)^{\ast} 
\times (N \cup A)^{\ast}$ a finite set.

%%%
\index{grammar!noncontracting}%%
%%%
Next we shall show a general theorem for context sensitive languages.
A grammar is called \textbf{noncontracting} if either no rule is
contracting or only the rule $\mbox{\tt S} \pf \varepsilon$ is contracting
and in this case the symbol {\tt S} never occurs to the right of a
production. Context sensitive grammars are contracting. However, 
not all noncontracting grammars are context sensitive. It turns out, 
however, that {\it all\/} noncontracting grammars generate context 
sensitive languages. (This can be used also to show that the context 
sensitive languages are exactly those languages that are recognized 
by a linearly space bounded Turing machine.)
%%
\begin{thm}
\label{thm:noncontracting}
A language is context sensitive iff there is a noncontracting
grammar that generates it.
\end{thm}
%%
\proofbeg
($\Pf$) Immediate. ($\Leftarrow$) Let $G$ be a noncontracting grammar. 
We shall construct a grammar $G^{\spadesuit}$ which is context sensitive 
and such that $L(G^{\spadesuit}) = L(G)$. To this end, let
$\rho = X_0 X_1 \dotsb X_{m-1} \pf Y_0 Y_1 \dotsb Y_{n-1}$, $m \leq n$, 
be a production. (As remarked above, we can reduce
attention to such rules and rules of the form $X \pf a$. Since the
latter are not contracting, only the former kind needs attention.)
We assume $m$ new symbols, $Z_0$, $Z_1, \dotsc, Z_{m-1}$. Let 
$\rho^{\spadesuit}$ be the following set of rules. 
%%
\begin{equation}
\begin{split}
X_0 X_1 \dotsb X_{m-1} & \pf Z_0 X_1 \dotsb X_{m-1} \\
Z_0 X_1 X_2 \dotsb X_{m-1} & \pf Z_0 Z_1 X_2 \dotsb X_{m-1} \\
     & \dotso \\
Z_0 Z_1 \dotsb Z_{m-2} X_{m-1} & \pf Z_0 Z_1 \dotsb Z_{m-1} \\
Z_0 Z_1 \dotsb Z_{m-1} & \pf Y_0 Z_1 \dotsb Z_{m-1} \\
Y_0 Z_1 Z_2 \dotsb Z_{m-1} & \pf Y_0 Y_1 Z_2 \dotsb Z_{m-1} \\
     & \dotso \\
Y_0 Y_1 \dotsb Y_{m-2} Z_{m-1} & \pf Y_0 Y_1 \dotsb Y_{n-1}
\end{split}
\end{equation}
%%
Let $G^{\spadesuit}$ be the result of replacing all non context 
sensitive rules $\rho$ by $\rho^{\spadesuit}$. The new grammar is
context sensitive. Now let us be given a derivation in $G$.
Then replace every instance of a rule $\rho$ by the given
sequence of rules in $\rho^{\spadesuit}$. This gives a derivation
of the same string in $G^{\spadesuit}$. Conversely, let us be
given a derivation in $G^{\spadesuit}$. Now look at the following.
If somewhere the rule $\rho^{\spadesuit}$ is applied, and then a
rule from $\rho_{1}^{\spadesuit}$ then the instances commute unless
$\rho_{1} = \rho$ and the second instance is inside that of
that rule instance of $\rho^{\spadesuit}$. Thus, by suitably
reordering the derivation is a sequence of segments, where each
segment is a sequence of the rule $\rho^{\spadesuit}$ for some
$\rho$, so that it begins with $\vec{X}$ and ends with
$\vec{Y}$. This can be replaced by $\rho$. Do this for every segment.
This yields a derivation in $G$.
\proofend

Given that there are Type 0 languages that are not Type 0 
(Theorem~\ref{0-1-echt}) the following theorem shows that the 
languages of Type 1 are not closed under arbitrary homomorphisms. 
%%
\begin{thm}
\label{thm:erase}
Let $\mbox{\tt a}, \mbox{\tt b} \not\in A$ be (distinct) symbols.
For every language $L$ over $A$ of Type 0 there is a language
$M$ over $A \cup \{\mbox{\tt a}, \mbox{\tt b}\}$ of Type 1 such that
for every $\vec{x} \in L$ there is an $i$ with
$\mbox{\tt a}^i \mbox{\tt b} \vec{x} \in M$ and
every $\vec{y} \in M$ has the form $\mbox{\tt a}^i \mbox{\tt b} \vec{x}$
with $\vec{x} \in L$.
\end{thm}
%%
\proofbeg
We put $N^{\clubsuit} := N \cup \{\mbox{\tt A}, \mbox{\tt B},
\mbox{\tt S}^{\clubsuit}\}$. Let 
%%
\begin{equation}
\rho = X_0 X_1 \dotsb X_{m-1} \pf Y_0 Y_1 \dotsb Y_{n-1}
\end{equation}
%%
be a contracting rule. Then put 
%%
\begin{equation}
\rho^{\clubsuit} := X_0 X_1 \dotsb X_{m-1} \pf \mbox{\tt A}^{m-n} 
Y_0 Y_1 \dotsb Y_{n-1}
\end{equation}
%%
$\rho^{\clubsuit}$ is certainly not contracting. If
$\rho$ is not contracting then put $\rho^{\clubsuit} :=
\rho$. Let $R^{\clubsuit}$ consist of all rules of the form
$\rho^{\clubsuit}$ for $\rho \in R$ as well as the following rules.
%%
\begin{equation}
\begin{split}
\mbox{\tt S}^{\clubsuit}  & \pf \mbox{\tt BS}  \\
X \mbox{\tt A} & \pf \mbox{\tt A}X \qquad (X \in N^{\clubsuit})  \\
\mbox{\tt BA} & \pf \mbox{\tt aB} \\
\mbox{\tt B}  & \pf \mbox{\tt b}
\end{split}
\end{equation}
%%
Let $M := L(G^{\clubsuit})$. Certainly, $\vec{y} \in M$ only if
$\vec{y} = \mbox{\tt a}^i \mbox{\tt b} \vec{x}$ for some
$\vec{x} \in A^{\ast}$. For strings contain {\tt B} (or {\tt b})
only once. Further, {\tt A} can be changed into {\tt a} only if
it occurs directly before {\tt B}. After that we get {\tt B} followed
by {\tt a}. Hence {\tt b} must occur after all occurrences of
{\tt a} but before all occurrences of {\tt B}. Now consider the
homomorphism $\oli{v}$ defined by $v \colon
\mbox{\tt A}, \mbox{\tt a}, \mbox{\tt B}, \mbox{\tt b},
\mbox{\tt S}^{\clubsuit} \mapsto \varepsilon$ and
$v \colon X \mapsto X$ for $X \in N$, $v \colon a \mapsto a$ for
$a \in A$. If $\auf \vec{\alpha}_i : i < n\zu$ is a
derivation in $G^{\clubsuit}$ then
$\auf \oli{v}(\vec{\alpha}_i) : 0 < i < n\zu$ is a derivation
in $G$ (if we disregard repetitions).  In this way one shows
that $\mbox{\tt a}^i \mbox{\tt b} \vec{x} \in M$
implies $\vec{x} \in L(G)$. Next, let $\vec{x} \in L(G)$.
Let $\auf \vec{\alpha}_i : i < n\zu$ be a derivation of
$\vec{x}$ in $G$. Then do the following. Define
$\vec{\beta}_0 := S^{\clubsuit}$ and $\vec{\beta}_1
= \mbox{\tt BS}$. Further, let $\vec{\beta}_{i+1}$ be of the form
$\mbox{\tt BA}^{k_i}\vec{\alpha}_i$ for some $k_i$ which is
determined inductively. It is easy to see that $\vec{\beta}_{i+1}
\vdash_{G^{\clubsuit}} \vec{\beta}_{i+2}$, so that
one can complete the sequence  $\auf \vec{\beta}_i : i < n+1\zu$
to a derivation. From $\mbox{\tt BA}^{k_n} \vec{x}$ one can derive
$\mbox{\tt a}^{k_n} \mbox{\tt b} \vec{x}$.
This shows that $\mbox{\tt a}^{k_n} \mbox{\tt b} \vec{x} \in M$, 
as desired.
\proofend
%%

Now let $v \colon A \pf B^{\ast}$ be a map. $v$ (as well as the generated
homomorphism $\oli{v}$) is called $\varepsilon$--\textbf{free}
%%%
\index{homomorphism!$\varepsilon$--free}%%
%%%
if $v(a) \neq \varepsilon$ for all $a \in A$.
%%
\begin{thm}
\label{thm:afl}
Let $L_1$ and  $L_2$ be languages of Type $i$, $0 \leq i \leq 3$.
Then the following are also languages of Type $i$.
%%
\begin{dingautolist}{192}
%%
\item $L_1 \cup L_2$, $L_1 \cdot L_2$, $L_1^{\ast}.$
\item $\oli{v}[L_1]$, where $v$ is $\varepsilon$--free.
\end{dingautolist}
%%
If $i \neq 1$ then  $\oli{v}[L_1]$ also is of Type $i$ even if
$v$ is not $\varepsilon$--free.
%%
\end{thm}
%%
\proofbeg
Before we begin, we remark the following. If $L \subseteq A^{\ast}$
is a language and $G = \auf \mbox{\tt S}, N, A, R\zu$ a grammar over $A$
which generates $L$ then for an arbitrary $B \supseteq A$
$\auf \mbox{\tt S}, N, B, R\zu$ is a  grammar over $B$ which generates $L
\subseteq B^{\ast}$. Therefore we may now assume that
$L_1$ and $L_2$ are languages over the same alphabet.
\ding{192} is seen as follows. We have $G_1 =
\auf \mbox{\tt S}_1, N_1, A, R_1\zu$ and $G_2 = \auf \mbox{\tt S}_2,
N_2, A, R_2\zu$ with $L(G_1) = L(G_2)$. By renaming the nonterminals
of $G_2$ we can see to it that $N_1 \cap N_2 = \varnothing$.
Now we put $N_3 := N_1 \cup N_2 \cup \{\mbox{\tt S}^{\diamondsuit}\}$
(where $\mbox{\tt S}^{\diamondsuit} \not\in N_1 \cup N_2$)
and $R := R_1 \cup R_2 \cup \{\mbox{\tt S}^{\diamondsuit} \pf \mbox{\tt S}_1,
\mbox{\tt S}^{\diamondsuit} \pf \mbox{\tt S}_2\}$. This defines
$G_3 := \auf \mbox{\tt S}^{\diamondsuit}, N_3, A, R_3\zu$.
This is a grammar which generates $L_1 \cup L_2$. We introduce a new 
start symbol
$\mbox{\tt S}^{\times}$ together with the rules $\mbox{\tt S}^{\times} \pf
\mbox{\tt S}_1 \mbox{\tt S}_2$ where $\mbox{\tt S}_1$
is the  start symbol of $G_1$ and $G_2$ the start symbol of $G_2$.
This yields a grammar of Type $i$ except if $i = 3$.
In this case the fact follows from the results of
Section~\ref{kap2}.\ref{zweieins}. It is however not difficult to construct
a grammar which is regular and generates the language
$L_1 \cdot L_2$.  Now for $L_1^{\ast}$. Let {\tt S} be the
start symbol for a grammar $G$ which generates $L_1$.
Then introduce a new symbol $\mbox{\tt S}^+$ as well as a new
start symbol $\mbox{\tt S}^{\ast}$ together with the rules
%%
\begin{equation}
\begin{split}
\mbox{\tt S}^{\ast} & \pf \varepsilon \mid  \mbox{\tt S}
    \mid  \mbox{\tt S}\mbox{\tt S}^+ \\
\mbox{\tt S}^+      & \pf \mbox{\tt S} \mid \mbox{\tt SS}^+
\end{split}
\end{equation}
%%
This grammar is of Type $i$ and generates $L_1^{\ast}$.
(Again the case $i = 3$ is an exception that can be dealt with
in a  different way.) Finally, \ding{193}. Let $v$ be 
$\varepsilon$--free. We extend it by putting $v(X) := X$ for all 
nonterminals $X$. Then replace the rules $\rho = \vec{\alpha} \pf \vec{\beta}$
by $\oli{v}(\rho) := \oli{v}(\vec{\alpha}) \pf \oli{v}(\vec{\beta})$.
If $i = 0, 2$, this does not change the type. If $i = 1$ we must
additionally require that $v$ is $\varepsilon$--free.
For if $\vec{\gamma} X \vec{\delta} \pf
\vec{\gamma} \vec{\alpha} \vec{\delta}$ is a rule
and $\vec{\alpha}$ is a terminal string we may have
$\oli{v}(\alpha) = \varepsilon$. This is however not the case
if $v$ is $\varepsilon$--free. If $i = 3$ again a different
method must be used.  For now --- after applying the replacement
--- we have rules of the form $X \pf \vec{x} Y$ and $X \pf \vec{x}$,
$\vec{x} = x_0 x_1 \dotsb x_{n-1}$. Replace the latter by $X \pf
x_0 Z_0$, $Z_i \pf x_i Z_{i+1}$ and $Z_{n-2} \pf x_{n-1} Y$ and
$Z_{n-2} \pf x_{n-1}$, respectively. \proofend
%%
\begin{defn}
%%%
\index{abstract family of languages}%%
\index{AFL}%%
%%%%
Let $A$ be a (possibly infinite) set. A nonempty set $\CS \subseteq 
\wp(A^{\ast})$ is called an \textbf{abstract family of languages} 
(\textbf{AFL}) \textbf{over} $A$ if the following holds.
%%
\begin{dingautolist}{192}
\item For every $L \in \CS$ there is a finite
    $B \subseteq A$ such that $L \subseteq B^{\ast}$.
\item If $h \colon A^{\ast} \pf A^{\ast}$ is a homomorphism
    and $L \in \CS$ then also $h[L] \in \CS$.
\item If $h \colon A^{\ast} \pf A^{\ast}$ is a homomorphism
    and $L \in \CS$, $B \subseteq A$ finite, then also
    $h^{-1}[L] \cap B^{\ast} \in \CS$.
\item If $L \in \CS$ and $R$ is a  regular language then
    $L \cap R \in \CS$.
\item If $L_1, L_2 \in \CS$ then also $L_1 \cup L_2 \in \CS$ and
    $L_1 \cdot L_2 \in \CS$.
\end{dingautolist}
%%
\end{defn}
%%
We still have to show that the languages of Type $i$ are closed
with respect to intersections with regular languages. A proof for
the Types 3 and 2 is found in Section~\ref{kap2}.\ref{kap2-1},
Theorem~\ref{thm:cfintersekt}.
This proof can be extended to the other types without problems.

The regular, the context free and the Type 0 languages over a fixed
alphabet form an abstract family of languages. The context sensitive
languages fulfill all criteria except for the closure under
homomorphisms. It is easy to show that the regular languages over
$A$ form the smallest abstract family of languages. More on this
subject can be found in \cite{ginsburg:formal}.

{\it Notes on this section.} It is a gross simplification to view 
languages as sets of strings. The idea that they can be defined by 
means of formal processes did not become apparent until
the 1930s. The idea of formalizing rules for transforming strings
was first formulated by Axel Thue \shortcite{thue:zeichenreihen}. 
%%%
\index{Thue, Axel}%%%
%%%
The observation that languages (in his case formal languages) could be seen as
generated from semi Thue systems, is due to Emil Post. 
%%%
\index{Post, Emil}%%
%%%
Also, he has
invented independently what is now known as the Turing machine and
has shown that this machine does nothing but string transformations.
The idea was picked up by Noam Chomsky 
%%%
\index{Chomsky, Noam}%%
%%%
and he defined the
hierarchy which is now named after him (see for example
\cite{chomsky:properties}, but the ideas have been circulating
earlier). In view of Theorem~\ref{thm:erase} it is unclear,
however, whether grammars of Type 0 or 1 have any relevance for
natural language syntax, since there is no notion of a constituent
that they define as opposed to context free grammars. There are 
other points to note about these types of grammars. 
\cite{langholm:indexed} voices clear discontentment with 
the requirement of a single start symbol, which is in practice 
anyway not complied with.
%%
\vplatz
\exercise
Let $T$ be a semi Thue system over $A$ and $A \subseteq B$.
Then $T$ is also a semi Thue system $T'$ over $B$. Characterize
$\Pf^{\ast}_{T'} \subseteq B^{\ast} \times B^{\ast}$ by means
of $\Pf_T^{\ast} \subseteq A^{\ast}\times A^{\ast}$.
{\it Remark.} This exercise shows that with the Thue system
we also have to indicate the alphabet on which it is based.
%%
\vplatz
\exercise
Let $A$ be a finite alphabet. Every string $\vec{x}$ is the value 
of a constant term $\vec{x}^E$ composed  from constants $\uli{a}$ 
for every $a \in A$, the symbol $\varepsilon$, and $^{\smallfrown}$. 
Let $T$ be a Thue system over $A$. Write $T^E := \{\vec{x}^E%
\boldsymbol{\doteq}\vec{y}^E : \auf \vec{x}, \vec{y}\zu \in T\}$. 
Let $M$ be consist of Equations~\eqref{eqn:null} and 
\eqref{eqn:eins}. $T^E$ is an equational theory. Show that 
$\vec{x} \Pf^{\ast}_T \vec{y}$ iff $\vec{y} \Pf^{\ast}_T \vec{x}$ 
iff $T^E \cup M \vdash \vec{x}^E\boldsymbol{\doteq}\vec{y}^E$.  
%%%
\vplatz
\exercise
\index{Commuting Instances Lemma}%%%
%%%%
Prove the Commuting Instances Lemma.
%%
\vplatz
\exercise
Show that every finite language is regular.
%%
\vplatz
\exercise
Let $G$ be a grammar with rules of the form $X \pf \vec{\alpha}$.
Show that $L(G)$ is context free. Likewise show that
$L(G)$ is regular if all rules have the form
$X \pf \alpha_0 \conc \alpha_1$ where
$\alpha_0 \in A \cup \{\varepsilon\}$ and $\alpha_1
\in N \cup \{\varepsilon\}$.
%%
\vplatz
\exercise
Let $G$ be a grammar in which every rule distinct from
$X \pf a$ is strictly expanding. Show that a derivation
of a string of length $n$ takes at most $2n$ steps.
%%
\vplatz
\exercise
Show that the language
$\{\mbox{\tt a}^n\mbox{\tt b}^n : n \in  \omega\}$
is context free.
%%
\vplatz
\exercise
Write a Type 1 grammar for the language
$\{\mbox{\tt a}^n \mbox{\tt b}^n \mbox{\tt c}^n : n \in \omega\}$
and one for $\{ \vec{x} \conc \vec{x} : \vec{x} \in A^{\ast}\}$.
%%

 \section{Grammar and Structure}
\label{einsvier}
%
%
%
Processes that replace strings by strings can often be considered 
as processes that successively replace parts of structures by 
structures. In this section we shall study processes of structure 
replacement. They can in principle operate on any kind of structure.
But we will restrict our attention to algorithms that generate ordered 
trees. There are basically two kinds of algorithms: the first is 
like the grammars of the previous section, generating intermediate 
structures that are not proper structures of the language; and the 
second, which generates in each step a structure of the language.

%%%
\index{multigraph}%%
%%%
Instead of graphs we shall deal with so--called {\it multigraphs}. 
A \textbf{directed multigraph} is
a structure $\auf V, \auf K_i : i < n\zu\zu$ where is $V$
%%%
\index{vertex}%%
%%%
a set, the set of \textbf{vertices}, and $K_i \subseteq V \times V$
%%%
\index{edge}%%
%%%
a disjoint set, the set of \textbf{edges} of type $i$. In our case
edges are always directed. We shall not mention this fact explicitly
later on. Ordered trees are one example
among many of (directed) multigraphs. For technical reasons we shall 
not exclude the case $V = \varnothing$, so that $\auf \varnothing, 
\auf \varnothing : i < n\zu\zu$ also is a multigraph. Next we shall 
introduce a
%%%
\index{vertex colouring}%%
%%%
colouring on the vertices. A \textbf{vertex--colouring} is a
function $\mu_V \colon V \pf F_V$ where $F_V$ is a nonempty set, the
set of \textbf{vertex colours}.
%%%
\index{vertex colour}%%
%%%
Think of the labelling as being a vertex colouring on the graph.
The principal structures are therefore vertex coloured multigraphs.
However, from a technical point of view the different edge relations
can also be viewed as colourings on the edges. Namely, if
$v$ and $w$ are vertices, we colour the edge $\auf v,w\zu$ by
the set $\{i : \auf v,w\zu \in K_i\}$. This set may be empty.
%%
\begin{defn}
%%%
\index{multigraph}%%%
\index{$\gamma$--graph}%%%
%%%%
An $\auf F_V, F_E\zu$--\textbf{coloured multigraph}
or simply a $\gamma$--\textbf{graph} (\textbf{over}
$F_V$ \textbf{and} $F_E$) is a triple $\auf V, \mu_V, \mu_E\zu$,
where $V$ is a (possibly empty) set and $\mu_V \colon V \pf F_V$ as
well as $\mu_E \colon V \times V \pf \wp(F_E)$ are functions.
\end{defn}
%%
\begin{figure}
\begin{center}
\begin{picture}(10,10)
    \put(1.5,2){\makebox(0,0)[r]{$w$}}
    \put(1.5,8){\makebox(0,0)[r]{$x$}}
    \put(8.5,8){\makebox(0,0)[l]{$y$}}
    \put(8.5,2){\makebox(0,0)[l]{$z$}}
        \put(4,5){\makebox(0,0)[r]{$p$}}
\put(2,2){\makebox(0,0){$\bullet$}}
\put(8,2){\makebox(0,0){$\bullet$}}
    \put(8,2){\vector(-1,0){5.8}}
        \put(5,1.8){\makebox(0,0)[t]{\tiny 1}}
    \put(8,2){\vector(0,1){5.8}}
        \put(8.2,5){\makebox(0,0)[l]{\tiny 2}}
\put(2,8){\makebox(0,0){$\bullet$}}
    \put(2,8){\vector(0,-1){5.8}}
        \put(1.8,5){\makebox(0,0)[r]{\tiny 2}}
    \put(2,8){\vector(1,-1){2.8}}
        \put(3.5,6.8){\makebox(0,0)[l]{\tiny 1}}
    \put(2,8){\vector(1,0){5.8}}
        \put(5,8.2){\makebox(0,0)[b]{\tiny 1}}
\put(8,8){\makebox(0,0){$\bullet$}}
\put(5,5){\makebox(0,0){$\bullet$}}
    \put(5,5){\circle{1}}
    \put(5,5){\vector(-1,-1){2.8}}
        \put(3.5,3.8){\makebox(0,0)[b]{\tiny 2}}
    \put(5,5){\vector(1,1){2.8}}
        \put(6.5,6.7){\makebox(0,0)[b]{\tiny 2}}
\put(5,0.5){\makebox(0,0){$\GG_1$}}
\end{picture}
\begin{picture}(10,10)
\put(3,3){\makebox(0,0){$\bullet$}}
    \put(2.5,3){\makebox(0,0)[r]{$a$}}
    \put(3,3){\vector(0,1){3.8}}
        \put(2.8,5){\makebox(0,0)[r]{\tiny 1}}
\put(3,7){\makebox(0,0){$\bullet$}}
    \put(2.5,7){\makebox(0,0)[r]{$b$}}
    \put(3,7){\vector(1,0){3.8}}
        \put(5,7.2){\makebox(0,0)[b]{\tiny 2}}
\put(7,7){\makebox(0,0){$\bullet$}}
    \put(7.5,7){\makebox(0,0)[l]{$c$}}
\put(5,0.5){\makebox(0,0){$\GG_2$}}
\end{picture}
%%
\begin{picture}(13,13)
    \put(1.5,2){\makebox(0,0)[r]{$w$}}
    \put(1.5,11){\makebox(0,0)[r]{$x$}}
    \put(11.5,11){\makebox(0,0)[l]{$y$}}
    \put(11.5,2){\makebox(0,0)[l]{$z$}}
\put(2,2){\makebox(0,0){$\bullet$}}
    \put(2,2){\vector(1,1){2.8}}
        \put(3.5,3.8){\makebox(0,0)[b]{\tiny 1}}
\put(11,2){\makebox(0,0){$\bullet$}}
    \put(11,2){\vector(-1,0){8.8}}
        \put(6.5,1.8){\makebox(0,0)[t]{\tiny 1}}
    \put(11,2){\vector(0,1){8.8}}
        \put(11.2,6.5){\makebox(0,0)[l]{\tiny 2}}
\put(2,11){\makebox(0,0){$\bullet$}}
    \put(2,11){\vector(0,-1){8.8}}
        \put(1.8,6.5){\makebox(0,0)[r]{\tiny 2}}
    \put(2,11){\vector(1,-1){2.8}}
        \put(3.5,9.8){\makebox(0,0)[l]{\tiny 1}}
    \put(2,11){\vector(1,0){8.8}}
        \put(6.5,11.2){\makebox(0,0)[b]{\tiny 1}}
\put(11,11){\makebox(0,0){$\bullet$}}
\put(5,5){\makebox(0,0){$\bullet$}}
    \put(4.5,5){\makebox(0,0)[r]{$a$}}
    \put(5,5){\vector(0,1){2.8}}
        \put(4.8,6.5){\makebox(0,0)[r]{\tiny 1}}
\put(5,8){\makebox(0,0){$\bullet$}}
    \put(4.5,8){\makebox(0,0)[r]{$b$}}
    \put(5,8){\vector(1,0){2.8}}
        \put(6.5,8.2){\makebox(0,0)[b]{\tiny 2}}
\put(8,8){\makebox(0,0){$\bullet$}}
    \put(8.5,8){\makebox(0,0)[l]{$c$}}
    \put(8,8){\vector(1,1){2.8}}
        \put(9.5,9.8){\makebox(0,0)[b]{\tiny 2}}
\put(6.5,0.5){\makebox(0,0){$\GG_3$}}
\end{picture}
\end{center}
\caption{Graph Replacement}
\label{fig:graphersetzung}
\end{figure}
%%%%
Now, in full analogy to the string case we shall distinguish
terminal and nonterminal colours. For simplicity, we shall study 
only replacements of a single vertex by a graph. Replacing a vertex 
by another structure means embedding a structure into some other 
structure. We need to be told how to do so. Before 
we begin we shall say something about the graph replacement in 
general. The reader is asked to look at
Figure~\ref{fig:graphersetzung}. The graph $\GG_3$ is the result of 
replacing in $\GG_1$ the encircled dot by $\GG_2$. The edge colours 
are $1$ and $2$ (the vertex colours pose no problems, so they are
omitted here for clarity).

Let $\GG = \auf E, \mu_E, \mu_K\zu$ be a $\gamma$--graph and
$M_1$ and $M_2$ be disjoint subsets of $E$ with $M_1 \cup M_2 = E$. 
Put $\GM_i = \auf M_i, \mu^i_V, \mu^i_E\zu$, where 
$\mu^i_V := \mu_V \restriction M_i$ and
$\mu^i_E := \mu_E \restriction M_i \times M_i$.
These graphs do not completely determine $\GG$
since there is no information on the edges between them.
We therefore define functions $\ein, \aus\colon
M_2 \times F_E \pf \wp(M_1)$, which for every vertex of $M_2$
and every edge colour name the set of all vertices of $M_1$
which lie on an edge with a vertex that either is directed
into $M_1$ or goes outwards from $M_1$.
%%
%%%
\index{$\ein(x,f)$, $\aus(x,f)$}%%%
%%%
\begin{subequations}
\begin{align}
\ein(x,f) & := \{y \in M_1 : f \in \mu_E(\auf y,x\zu)\} \\
\aus(x,f) & := \{y \in M_1 : f \in \mu_E(\auf x,y\zu)\}
\end{align}
\end{subequations}
%%
It is clear that $\GM_1$, $\GM_2$ and the functions
$\ein$ and $\aus$ determine $\GG$ completely.
In our example we have
%%
\begin{align}
\ein(p,1) & = \{x\} & \ein(p,2) & = \varnothing \\\notag
\aus(p,1) & = \varnothing & \aus(p,2) & = \{w,y\}
\end{align}
%%
Now assume that we want to replace $\GM_2$ by a different graph
$\GH$. Then not only do we have to know $\GH$  but also the 
functions $\ein, \aus \colon H \times F_E \pf
\wp(M_1)$. This, however, is not the way we wish to proceed
here. We want to formulate rules of replacement that are
general in that they do not presuppose exact knowledge about the
embedding context. We shall only assume that the functions
$\ein(x,f)$ and $\aus(x,f)$, $x \in H$, are systematically defined 
from the sets $\ein(y,g)$, $\aus(y,g)$, $y \in M_2$. We shall 
therefore only allow to specify how the sets of the first kind are 
formed from the sets of the second kind. This we do by means of
four so--called {\it colour functionals}. A \textbf{colour functional 
from} $\GH$ \textbf{to} $\GM_2$ is a map
%%%
\index{colour functional}%%
%%%
%%
\begin{equation}
\GF \colon H \times F_E \pf \wp(M_2 \times F_E)
\end{equation}
%%
In our case a functional is a function
from $\{a,b,c\} \times \{1,2\}$ to $\wp(\{p\} \times \{1,2\})$.
We can simplify this to a function from $\{a,b,c\} \times \{1,2\}$
to $\wp(\{1,2\})$. The colour functionals are called $\goth{II}$,
$\goth{IO}$, $\goth{OI}$ and $\goth{OO}$. 
%%%
\index{$\goth{II}$, $\goth{IO}$, $\goth{OI}$, $\goth{OO}$}%%%
%%%
For the example of Figure~\ref{fig:graphersetzung} we get the 
following colour functionals (we only give values when the 
functions do not yield $\varnothing$).
%%
\begin{align}
\goth{II} & \colon \auf b, 1\zu \mapsto \{1\} &
\goth{OI} & \colon \auf a, 2\zu \mapsto \{1\} \\\notag
\goth{IO} & \colon \varnothing &
\goth{OO} & \colon \auf c, 2\zu \mapsto \{2\}
\end{align}
%%
The result of substituting $\GM_2$ by $\GH$ by means of the
colour functionals from $\GF$ is denoted by $\GG[\GH/\GM_2 : \GF]$.
%%%
\index{$\GG[\GH/\GM : \GF]$}%%
%%%
This graph is the union of $\GM_1$ and $\GH$ together with the
functions $\ein^+$ and $\aus^+$, which are defined as follows.
%%
\begin{equation}
\begin{split}
\ein^+(x,f) := & \phantom{\cup}\, \bigcup \auf \ein(x,g) :
    g \in \goth{II}(x,f)\zu \\
    & \cup \bigcup \auf \aus(x,g) :
    g \in \goth{OI}(x,f) \zu \\
\aus^+(x,f) := & \phantom{\cup}\, \bigcup \auf \aus(x,g) :
    g \in \goth{OO}(x,f)\zu \\
    & \cup \bigcup \auf \ein(x,g) :
    g \in \goth{IO}(x,f) \zu
\end{split}
\end{equation}
%%
If $g \in \goth{II}(x,f)$ we say that an edge with
colour $g$ into $x$ is {\it transmitted as an ingoing edge of
colour $f$ to $y$}. If $g \in \goth{OI}(x,f)$ we say
that an edge with colour $g$ going out from $x$ is {\it transmitted
as an ingoing edge with colour $f$ to $y$}. Analogously for
$\goth{IO}$ and $\goth{OO}$. So, we do allow for an edge to
change colour and direction when being transmitted.
If edges do not change direction, we only need the functionals
$\goth{II}$ and $\goth{OO}$, which are then denoted simply by
$\goth{I}$ and $\goth{O}$. Now we look at the special case where
$M_2$ consists of a single element, say $x$. In this case a
colour functional simply is a function $\GF \colon H \times F_E \pf %
\wp(F_E)$.
%%
\begin{defn}
%%%
\index{graph grammar}%%
\index{graph grammar!context free}%%
\index{start graph}%%
%%%
A \textbf{context free graph grammar with edge replacement} ---
a \textbf{context free $\gamma$--grammar} for short --- is a
quintuple of the form
%%
\begin{equation}
\Gamma = \auf \GS, F_V, F^T_V, F_E, R\zu
\end{equation}
%%
in which $F_V$ is a finite set of vertex colours, $F_E$
a finite set of edge colours, $F_V^T \subseteq F_V$
a set of so--called \textbf{terminal vertex colours}, $\GS$ a
$\gamma$--graph over $F_V$ and $F_E$, the so--called
\textbf{start graph}, and finally $R$ a finite set of triples
$\auf X, \GH, \BF\zu$ such that $X \in F_V - F_V^T$ is a
nonterminal vertex colour, $\GH$ a $\gamma$--graph over $F_V$
and $F_E$ and $\BF$ is a matrix of colour functionals.
\end{defn}
%%
\index{derivation}%%
%%%%
A \textbf{derivation} in a $\gamma$--grammar $\Gamma$ is defined as 
follows.  For $\gamma$--graphs $\GG$ and $\GH$ with the colours
$F_V$ and $F_E$, $\GG \Pf^1_R \GH$ means that there is
$\auf X, \GM, \BF\zu \in R$ such that $\GH = \GG [\GM/\GX : \BF]$,
where $\GX$ is a subgraph consisting of a single vertex $x$
having the colour $X$. Further we define $\Pf^{\ast}_R$ to be
the reflexive and transitive closure of $\Pf^1_R$ and finally we put
$\Gamma \vdash \GG$ if $\GS \Pf^{\ast}_R \GG$. A derivation
\textbf{terminates} if there is no vertex with a nonterminal colour.
%%%
\index{$L_{\gamma}(\Gamma)$}%%%%
%%%
We write $L_{\gamma}(\Gamma)$ for the class of $\gamma$--graphs
that can be generated from $\Gamma$. Notice that the edge colours
only the vertex colours are used to steer the derivation.

We also define the productivity of a rule as the difference
between the cardinality of the replacing graph and the cardinality
of the graph being replaced. The latter is 1 in context free
$\gamma$--grammars, which is the only type we shall study here. So, 
the productivity is always $\geq - 1$. It equals $-1$ if the replacing 
graph is the empty graph. A rule has productivity $0$ if the replacing 
graph consists of a single vertex. In the exercises the reader will 
be asked to verify that we can dispense with rules of this kind.

Now we shall define two types of context free $\gamma$--grammars.
Both are context free as $\gamma$--grammars but the second type
can generate non--CFLs. This shows that the concept of 
$\gamma$--grammar is more general. We shall begin with ordinary 
CFGs. We can view them alternatively as
grammars for string replacement or as grammars that replace trees
by trees. For that we shall now assume that there are no rules of
the form $X \pf \varepsilon$. (For such rules generate trees
whose leaves are not necessarily marked by letters from $A$. This
case can be treated if we allow labels to be in $A_{\varepsilon}
= A \cup \{\varepsilon\}$, which we shall not do here.) Let 
$G = \auf \mbox{\tt S}, A, N, R\zu$ be such a grammar.
We put $F_V := A \cup (N \times 2)$.  We write $X^0$ for
$\auf X,0\zu$ and $X^1$ for $\auf X,1\zu$. $F_V^T := A \cup
N \times \{0\}$. $F_E := \{<, \sqsubset\}$. Furthermore, the
start graph consists of a single vertex labelled $\mbox{\tt S}^1$ 
and no edge. The rules of replacement are as follows.
Let $\rho = X \pf \alpha_0 \alpha_1 \dotsb \alpha_{n-1}$
be a rule from $G$, where none of the $\alpha_i$ is $\varepsilon$.
Then we define a $\gamma$--graph $\GH_{\rho}$
as follows. $H_{\rho} := \{y_i : i < n\} \cup \{x\}$.
$\mu_V(x) = X^0$, $\mu_V(y_i) = \alpha_i$ if
$\alpha_i \in A$ and $\mu_V(y_i) = \alpha_i^1$ if
$\alpha_i \in N$. 
%%
\begin{equation}
\begin{split}
\mu_E^{-1}(\{<\}) & := \{\auf y_i, x\zu : i < n\},  \\
\mu_E^{-1}(\{\sqsubset\}) & := \{\auf y_i, y_j \zu : i < j < n\}.
\end{split}
\end{equation}
%%
This defines $\GH_{\rho}$. Now we define the colour functionals.
For $u \in n$ we put
%%
\begin{equation}
\begin{split}
\GI_{\rho}(u, \sqsubset) & := \{\sqsubset\} & 
	 \GO_{\rho}(u, \sqsubset) & := \{\sqsubset\} \\
\GI_{\rho}(u, <)         & := \{<\} &
	\GO_{\rho}(u, <) & := \{<\}
\end{split}
\end{equation}
%%
Finally we put $\rho^{\gamma} := \auf X, \GH_{\rho}, \{\GI_{\rho}, %
\GO_{\rho}\}\zu$. $R^{\gamma} := \{\rho^{\gamma} : \rho \in R\}$.
%%
\begin{equation}
\gamma G := \auf \GS, F_E, F_E^T, F_T, R^{\gamma}\zu 
\end{equation}
%%
We shall show that this grammar yields exactly those trees that
we associate with the grammar $G$. Before we do so, a few remarks
are in order. The nonterminals of $G$ are now from a technical
viewpoint terminals since they are also part of the structure
that we are generating. In order to have any derivation at all
we define two equinumerous sets of nonterminals. Each nonterminal
$N$ is split into the nonterminal $N^1$ (which is nonterminal
in the new grammar) and $N^0$ (which is now a terminal vertex
colour). We call the first kind \textbf{active}, \textbf{nonactive}
%%%
\index{node!active}%%
\index{node!nonactive}%%
%%%
the second. Notice that the rules are formulated in such a way
that only the leaves of the generated trees carry active
nonterminals. A single derivation step is displayed in
Figure~\ref{fig:cfggraph}. In it, the rule $\mbox{\tt X} \pf %
\mbox{\tt AcA}$ has been applied to the tree to the left. The
result is shown on the right hand side.
%%
\begin{figure}
\begin{center}
\begin{picture}(13,13)
\put(2,5){\line(1,1){6}}
\put(2,5){\line(1,0){12}}
\put(14,5){\line(-1,1){6}}
\put(8,5){\makebox(0,0){$\bullet$}}
\put(8,5.5){\makebox(0,0)[b]{$\mbox{\tt X}^1$}}
\end{picture}
\qquad
\begin{picture}(13,13)
\put(2,5){\line(1,1){6}}
\put(2,5){\line(1,0){12}}
\put(14,5){\line(-1,1){6}}
\put(8,5){\makebox(0,0){$\bullet$}}
\put(8,5.5){\makebox(0,0)[b]{$\mbox{\tt X}^0$}}
\put(5,2){\line(1,0){6}}
\put(8,5){\line(-1,-1){3}}
% \put(8,5){\line(0,-1){3}}
\put(8,5){\line(1,-1){3}}
\put(5,1.5){\makebox(0,0){$\mbox{\tt A}^1$}}
\put(8,1.5){\makebox(0,0){\tt c}}
\put(11,1.5){\makebox(0,0){$\mbox{\tt A}^1$}}
\end{picture}
\end{center}
\caption{Replacement in a Context Free Grammar}
\label{fig:cfggraph}
\end{figure}
%%
It is easy to show that in each derivation only leaves
carry active nonterminals. This in turn shows that the
derivations of the $\gamma$--grammar are in one to one
correspondence with the derivations of the CFGs.
We put 
%%%
\begin{equation}
L_B(G) := h[L_{\gamma}(\gamma G)]
\end{equation}
\index{$L_B(G)$}%%%%
%%%
This is the class of trees generated by $\gamma G$, with $X^0$ and 
$X^1$ mapped to $X$ for each $X \in N$. 
The rules of $G$ can therefore be interpreted as conditions
on labelled ordered trees in the following way.
$\GC$ is called a \textbf{local subtree} of $\GB$
%%%
\index{subtree!local}%%
%%%
if (i) it has height 2 (so it does not possess inner nodes)
and (ii) it is maximal with respect to inclusion.
For a rule $\rho = X \pf Y_0 Y_1 \dotsb Y_{n-1}$ we define
$L_{\rho} := \{y_i : i < n\} \cup \{x\}$, $<_{\rho}\; :=
\{ \auf y_i, x\zu : i < n\}$, $\sqsubset_{\rho} := \{\auf y_i, y_j\zu
: i < j < n\}$, and finally $\ell_{\rho}(x) := X$, $\ell(y_i) := Y_i$.
$\GL_{\rho} := \auf L_{\rho}, <_{\rho}, \sqsubset_{\rho},
\ell_{\rho}\zu$. Now, an isomorphism between labelled ordered
trees $\GB = \auf B, <_{\GB}, \sqsubset_{\GB}, \ell_{\GB}\zu$
and $\GC = \auf C, <_{\GC}, \sqsubset_{\GC}, \ell_{\GC}\zu$
is a bijective map $h \colon B \pf C$ such that $h[<_{\GB}] 
= \; <_{\GC}$, $h[\sqsubset_{\GB}] = \sqsubset_{\GC}$
and $\ell_{\GC}(h(x)) = \ell_{\GB}(x)$ for all $x \in B$.
%%
\begin{prop}
Let $G = \auf \mbox{\tt S}, N, A, R\zu$. $\GB \in L_B(G)$ iff 
every local tree of $\GB$ is isomorphic to an $\GL_{\rho}$ such 
that $\rho \in R$.
\end{prop}
%%
\begin{thm}
Let \textbf{B} be a set of trees over an alphabet $A \cup N$
with terminals from $A$. Then $\textbf{B} = L_B(G)$ for a
CFG $G$ iff there is a finite set $\{\GL_i : i < n\}$ of trees 
of height 2 and an $S$ such that $\GB \in \textbf{B}$ exactly if
%%%
\begin{dingautolist}{192}
\item the root carries label $S$,
\item a label is terminal iff the node is a leaf, and
\item every local tree is isomorphic to some $\GL_i$.
\end{dingautolist}
\end{thm}
%%
We shall derive a few useful consequences from these considerations.
It is clear that $\gamma G$ generates trees that do not necessarily
have leaves with terminal symbols. However, we do know that the
leaves carry labels either from $A$ or from $N^1 := N \times \{1\}$
while all other nodes carry labels from $N^0 := N \times \{0\}$.
For a labelled tree we define the associated string sequence $k(\GB)$
in the usual way. This is an element of $(A \cup N^1)^{\ast}$.
Let $v \colon A \cup (N \times 2) \pf A \cup N$ be defined by
$v(a) := a$, $a \in A$ and $v(X^0) := v(X^1) := X$ for $X \in N$.
%%
\begin{lem}
Let $\gamma G \vdash \GB$ and $\vec{\alpha} = k(\GB)$.
Then $\vec{\alpha} \in (A \cup N^1)^{\ast}$
and $G \vdash \oli{v}(\vec{\alpha})$.
\end{lem}
%%
\proofbeg
Induction over the length of the derivation. If the length is
0 then $\vec{\alpha} = \mbox{\tt S}^1$ and $\oli{v}(\mbox{\tt S}^1) 
= \mbox{\tt S}$. Since $G \vdash \mbox{\tt S}$ this case is settled. 
Now let $\GB$ be the result of an application of some rule $\rho^{\gamma}$
on $\GC$ where $\rho = X \pf \vec{\gamma}$. We then have
$k(\GC) \in (A \cup N^1)^{\ast}$. The rule $\rho^{\gamma}$
has been applied to a leaf; this leaf corresponds to an 
occurrence of $X^1$ in $k(\GC)$. Therefore we have
$k(\GC) = \vec{\eta}_1 \conc X^1 \conc \vec{\eta}_2$. Then
$k(\GB) = \vec{\eta}_1 \conc \vec{\gamma} \conc \vec{\eta}_2$.
$k(\GB)$ is the result of a single application of the rule
$\rho$ from $k(\GC)$.
\proofend
%%
\begin{defn}
%%%
\index{cut}%%
%%%
Let $\GB$ be a labelled ordered tree. A \textbf{cut through} $\GB$ is
a maximal set that contains no two elements comparable by $<$. If 
$\GB$ is exhaustively ordered, a cut is linearly ordered and labelled, 
and then we also call the string associated to this set a 
\textbf{cut}.
\end{defn}
%%
\begin{prop}
Let $\gamma G \vdash \GB$ and let $\vec{\alpha}$ be a cut through
$\GB$. Then $G \vdash \oli{v}(\vec{\alpha})$.
\end{prop}
%%
This theorem shows that the tree provides all necessary information.
If you have the tree, all essential details of the derivation can
be reconstructed (up to commuting applications of rules). Now let
us be given a tree $\GB$ and let $\vec{\alpha}$ be a cut.
We say that an occurrence $C$ of $\vec{\gamma}$ in $\vec{\alpha}$
is a \textbf{constituent of category} $X$ \textbf{in} $\GB$ if 
this occurrence of $\vec{\gamma}$ in $\vec{\alpha}$ is that cut 
defined by $\vec{\alpha}$ on $\low{x}$ where $x$ carries the label 
$X$.
%%%
\index{constituent}%%
%%%
This means that $\vec{\alpha} = \vec{\eta}_1 \conc \vec{\gamma}
\conc \vec{\eta}_2$, $C = \auf \vec{\eta}_1, \vec{\eta}_2\zu$,
and $\low{x}$ contains exactly those nodes that do not belong
to $\vec{\eta}_1$ or $\vec{\eta}_2$. Further, let $G$ be a 
CFG. A substring occurrence of $\vec{\gamma}$ is 
a $G$--constituent of category $X$ in $\vec{\alpha}$ if there is a 
$\gamma G$--tree for which there exists a cut $\vec{\alpha}$ such that 
the occurrence $\vec{\gamma}$ is a constituent of category $X$. If 
$G$ is clear from the context, we shall omit it. 
%%
\begin{lem}
\label{wegmachen}
Let $\GB$ be a $\gamma G$--tree and $\vec{\alpha}$ a cut through
$\GB$. Then there exists a tree $\GC$ with associated string
$\vec{\gamma}$ and $\oli{v}(\vec{\gamma}) = \oli{v}(\vec{\alpha})$.
\end{lem}
%%
\begin{lem}
Let $G \vdash \vec{\alpha}_1 \conc \vec{\gamma} \conc \vec{\alpha}_2$, 
$C = \auf \vec{\alpha}_1, \vec{\alpha}_2\zu$ an occurrence 
of $\vec{\gamma}$ as a $G$--constituent of category $X$. Then $C$ is a
$G$--constituent occurrence of $X$ in $C(X) = \vec{\alpha}_1 \conc X \conc %
\vec{\alpha}_2$.
\end{lem}
%%
For a proof notice that if $\vec{\alpha}_1 \conc
\vec{\gamma} \conc \vec{\alpha}_2$ is a cut and
$\vec{\gamma}$ is a constituent of category $X$ therein then
$\vec{\alpha}_1 \conc X \conc \vec{\alpha}_2$ also is a cut.
%%
\begin{thm}[Constituent Substitution]
%%%
\index{Constituent Substitution Theorem}%%
%%%
Suppose that $C$ is an occurrence of $\vec{\beta}$ as
a $G$--constituent of category $X$. Furthermore, let
$X \vdash_G \vec{\gamma}$. Then $G \vdash C(\vec{\gamma}) =
\vec{\alpha}_1 \conc \vec{\gamma} \conc \vec{\alpha}_2$
and $C$ is a $G$--constituent occurrence of $\vec{\gamma}$ 
of category $X$.
\end{thm}
%%
\proofbeg
By assumption there is a tree in which $\vec{\beta}$
is a constituent of category $X$ in $\vec{\alpha}_1 \conc \vec{\beta} %
\conc \vec{\alpha}_2$.  Then there exists a cut $\vec{\alpha}_1 \conc X
\conc \vec{\alpha}_2$ through this tree, and by
Lemma~\ref{wegmachen} there exists a tree with associated
string $\vec{\alpha}_1 \conc X \conc \vec{\alpha}_2$.
Certainly we have that $X$ is a constituent in this tree.
However,  a derivation $X \vdash_G \vec{\gamma}$ can in
this case be extended to a $\gamma G$--derivation of
$\vec{\alpha}_1 \conc \vec{\gamma}\conc \vec{\alpha}_2$
in which $\vec{\gamma}$ is a constituent.
\proofend
%%
\begin{lem}
Let $G$ be a CFG. Then there exists a number $k_G$ such that 
for each derivation tree of a string of length $\geq k_G$ there 
are two constituents $\low{y}$ and $\low{z}$ of identical 
category such that $y \leq z$ or $z \leq y$,
and the associated strings are different.
\end{lem}
%%
\proofbeg
To begin, notice that nothing changes in our claim if we
eliminate the unproductive rules. This does not change the
constituent structure. Now let $\pi$ be the maximum of all
productivities of rules in $G$, and $\nu := |N|$. Then
let $k_G := (1 + \pi)^{\nu} + 1$. We claim that this is the
desired number.  (We can assume that $\pi > 0$. Otherwise
$G$ only generates strings of length 1, and then $k_G := 2$
satisfies our claim.) For let $\vec{x}$ be given such that
$|\vec{x}| \geq k_G$.  Then there exists in every derivation
tree a branch of length $> \nu$. (If not, there can be no more
than $\pi^{\nu}$ leaves.) On this branch we have two nonterminals
with identical label. The strings associated to these
nodes are different since we have no unproductive rules.
\proofend

%%%
\index{constituent part!left, right}%%
%%%
We say, an occurrence $C$ is a \textbf{left constituent part} 
(\textbf{right constituent part}) if $C$ is an occurrence of 
a prefix (suffix) of a constituent.  An occurrence of $\vec{x}$ 
contains a left constituent part $\vec{z}$ if some suffix of 
$\vec{x}$ is a left constituent part.  We also remark that 
if $\vec{u}$ is a left constituent part and a proper substring 
of $\vec{x}$ then $\vec{x} = \vec{v}\,\vec{v}_1\vec{u}$ with 
$\vec{v}_1$ a possibly empty sequence of constituents and 
$\vec{v}$ a right constituent part. This will 
be of importance in the sequel.
%%
\begin{lem}
\label{lem:halb}
Let $G$ be a  CFG. Then there exists a number $k'_G$ such that 
for every derivation tree of a string $\vec{x}$ and every occurrence 
in $\vec{x}$ of a string $\vec{z}$ of length $\geq k'_G$ 
$\vec{z}$ contains two different left or two different right 
constituent parts $\vec{y}$ and $\vec{y}_1$ of constituents that 
have the same category. Moreover, $\vec{y}$ is a prefix 
of $\vec{y}_1$ or $\vec{y}_1$ a prefix of $\vec{y}$ in case that 
both are left constituent parts, and $\vec{y}$ is a suffix of 
$\vec{y}_1$ or $\vec{y}_1$ a suffix of $\vec{y}$ in case that both 
are right constituent parts.
\end{lem}
%%
\proofbeg
Let $\nu := |N|$ and let $\pi$ be the maximal productivity of a rule from
$G$.  We can assume that $\pi \geq 2$. Put $k'_G := (2 + 2\pi)^{\nu}$.
We show by induction on the number $m$ that a string of length
$\geq (2+ 2\pi)^m$ has at least $m$ left or at least $m$ right
constituent parts that are contained in each other. If $m = 1$
the claim is trivial. Assume that it holds for $m \geq 1$. We shall 
show that it also holds for $m+1$.  Let $\vec{z}$ be of length 
$\geq (2+2\pi)^{m+1}$. Let $\vec{x} = \prod_{i < 2\pi+2} \vec{x}_i$ 
for certain $\vec{x}_i$ with length at least $(2 + 2\pi)^m$. By 
induction hypothesis each $\vec{x}_i$ contains at least $m$ constituent
parts. Now we do not necessarily have $(2\pi +2)m$ constituent
parts in $\vec{x}$. For if $\vec{x}_i$ contains a left part then
$\vec{x}_j$ with $j > i$ may contain the corresponding right part.
(There is only one. The sections in between contain subwords of
that constituent occurrence.) For each left constituent part
we count at most one (corresponding) right constituent part.
In total we have at least $(1 + \pi)m \geq m+1$ constituent parts.
However,  we have to verify  that at least $m+1$ of these are
contained inside each other. Assume  this is not the case, for all
$i$. Then $\vec{x}_i$, $i < 2\pi + 2$, contains exactly $m$ left or 
exactly $m$ right constituent parts.
Case 1. $\vec{x}_0$ contains $m$ left constituent parts inside
each other. If $\vec{x}_1$ also contains $m$ left constituent
parts inside each other, we are done. Now suppose that this is
not the case. Then $\vec{x}_1$ contains $m$ right constituent
parts inside each other.  Then we obviously get $m$ entire
constituents stacked inside each other. Again, we would be done
if $\vec{x}_2$ contained $m$ right constituent parts inside
each other. If not, then $\vec{x}_2$ contains exactly $m$ left
constituent parts. And again we would be done if these would
not correspond to exactly $m$ right part that $\vec{x}_3$
contains. And so on. Hence we get a sequence of length $\pi$
of constituents which each contain $m$ constituents stacked
inside each other. Now three cases arise: (a) one of the 
constituents is a left part of some constituent, (b) one of 
the constituent is a right part of some constituent. (For if 
neither is the case, we have a rule of arity $> \pi$, a 
contradiction.) In Case (a) we evidently have $m+1$ left 
constituent parts stacked inside each other, and in Case (b) 
$m+1$ right constituent parts. Case 2. $\vec{x}_0$ contains $m$
right hand constituents stacked inside each other. Similarly.
This shows our auxiliary claim. Putting $m := \nu + 1$ the main
claim now follows.
\proofend
%%
\begin{thm}[Pumping Lemma]
%%%
\index{Pumping Lemma}%%%
\label{thm:pumplemma}
%%%
Given a CFL $L$ there exists a $p_L$ such that
for every string $\vec{z} \in L$ of length at least $p_L$ and an
occurrence of a string $\vec{r}$  of length at least $p_L$ in
$\vec{z}$, $\vec{z}$ possesses a decomposition
%%
\begin{equation}
\vec{z} = \vec{u} \conc \vec{x} \conc \vec{v}
\conc \vec{y} \conc \vec{w}
\end{equation}
%%
such that the following holds.
%%
\begin{dingautolist}{192}
\item
$\vec{x} \conc \vec{y} \neq \varepsilon$. 
%$\vec{u} \conc \vec{w} \neq \varepsilon$.
\item
Either the occurrence of $\vec{x}$ or the occurrence of
$\vec{y}$ is contained in the specified occurrence of $\vec{r}$.
\item
$\{\vec{u} \conc {\vec{x}\,}^i \conc \vec{v} \conc {\vec{y}\,}^i
\conc \vec{w} : i \in \omega\} \subseteq L$.
\end{dingautolist}
%%
(The last property is called the \textbf{pumpability} of the
substring occurrences of $\vec{x}$ and $\vec{y}$.) Alternatively, 
in place of \ding{193} one may require that $|\vec{v}| \leq p_L$. 
Further we can choose $p_L$ in such a way that every derivable
string $\vec{\gamma}$ with designated occurrences of a string
$\vec{\alpha}$ of length $\geq p_S$ can be decomposed in the way
given.
%%
\end{thm}
%%
\proofbeg
Let $G$ be a grammar which generates $L$. Let $p_L$
be the constant defined in Lemma~\ref{lem:halb}. We look at
a $G$--tree of $\vec{z}$ and the designated occurrence of
$\vec{r}$. Suppose that $\vec{r}$ has length at least $p_L$.
Then there are two left or two right constituent parts of
identical category contained in $\vec{r}$. Without loss of
generality we assume that $\vec{r}$ contains two left parts.
Suppose that these parts are not fully contained in $\vec{r}$.
Then $\vec{r} = \vec{s}\,\vec{x}\,\vec{s}_1$ where $\vec{x}\,\vec{s}_1$
and $\vec{s}_1$ are left constituent parts of identical category,
say $X$. Now $|\vec{x}| > 0$. There are $\vec{s}_2$ and $\vec{y}$
such that $\vec{v} := \vec{s}_1\vec{s}_2$ and
$\vec{x}\,\vec{s}_1\vec{s}_2\vec{y}$ are constituents of 
category $X$.

Hence there exists a decomposition
%%
\begin{equation}
\vec{z} = \vec{u} \conc \vec{x} \conc \vec{v}
\conc \vec{y} \conc \vec{w} 
\end{equation}
%%
where $\vec{v}$ is a constituent of the same category
as $\vec{x}\,\vec{v}\,\vec{y}$ satisfying \ding{192} and
\ding{193}. By the Constituent Substitution Theorem we may 
replace the occurrence of $\vec{x}\,\vec{v}\,\vec{y}$ by $\vec{v}$
as well as $\vec{v}$ by $\vec{x}\,\vec{v}\,\vec{y}$.
This yields \ding{194}, after an easy induction.
Now let the smaller constituent part be contained in $\vec{r}$
but not the larger one. Then we have a decomposition
$\vec{r} = \vec{s}\,\vec{x}\,\vec{v}\,\vec{s}_1$ such that $\vec{v}$
is a constituent part of category $X$ and $\vec{x}\,\vec{v}\,\vec{s}_1$
a left constituent part of a constituent of category $X$. Then there
exists a $\vec{s}_2$ such that also $\vec{x}\vec{v}\vec{s}_1\vec{s}_2$
is a constituent of category $X$. Now put $\vec{y} :=
\vec{s}_1\vec{s}_2$. Then we also have $\vec{y} \neq \varepsilon$.
The third case is if both parts are proper substrings of $\vec{r}$.
Also here we find the desired decomposition. If we want to have in 
place of \ding{193} that $\vec{v}$ is as small as possible then 
notice that $\vec{v}$ already is a
constituent. If it has length $\geq (1+ \pi)^{\nu}$ then
there is a decomposition of $\vec{v}$ such that it contains
pumpable substrings. Hence in place of \ding{193} we may require 
that $|\vec{v}| \leq p_G$.
\proofend

The Pumping Lemma can be stated more concisely as follows.
For every large enough derivable string $\vec{x}$ there exist 
contexts $C$, $D$, where $C \neq \auf \varepsilon, \varepsilon\zu$,
and a string $\vec{y}$ such $\vec{x} = D(C(\vec{y}))$,
and $D(C^k(\vec{y})) \in L$ for every $k \in \omega$.
The strongest form of a pumping lemma is the following.  
Suppose that we have two decompositions into pumping pairs 
$\vec{u}_1\conc \vec{x}_1 \conc \vec{v}_1\conc\vec{y}_1\conc\vec{w}_1$, 
$\vec{u}_2\conc \vec{x}_2 \conc \vec{v}_2\conc\vec{y}_2\conc\vec{w}_2$. 
We say that the two pairs are \textbf{independent} 
%%%
\index{independent pumping pair}%%
%%%
if either (1a) $\vec{u}_1\conc \vec{x}_1\conc\vec{v}_1\conc\vec{y}_1$ 
is a prefix of $\vec{u}_2$, or
(1b) $\vec{u}_2\conc \vec{x}_2\conc\vec{v}_2\conc\vec{y}_2$ is a 
prefix of $\vec{u}_1$, or
(1c) $\vec{u}_1\conc\vec{x}_1$ is a prefix of $\vec{u}_2$ and 
$\vec{y}_1\conc\vec{w}_1$ a suffix of $\vec{w}_2$, or
(1d) $\vec{u}_2\conc\vec{x}_2$ is a prefix of $\vec{u}_1$ and 
$\vec{y}_2\conc\vec{w}_2$ a suffix of $\vec{w}_1$ and 
(2) each of them can be pumped any number of times independently 
of the other. 
\nocite{manasterrameretal:ogden}
%%%%
\begin{thm}[Manaster-Ramer \& Moshier \& Zeitman]
\label{thm:multipump}
Let $L$ be a CFL. Then there exists a number 
$m_{L}$ such that if $\vec{x} \in L$ and we are given $k m_L$ 
occurrences of letters in $\vec{x}$ there are $k$ independent 
pumping pairs, each of which contains at least one and at most 
$m_L$ of the occurrences. 
\end{thm}
%%%%
This theorem implies the well--known \textbf{Ogden's Lemma} (see 
\cite{ogden:helpful}), which says that given at least $m_L$ 
occurrences of letters, there exists a pumping pair containing 
at least one and at most $m_L$ of them.

Notice that in all these theorems we may choose $i = 0$ as well. This 
means that not only we can pump `up' the string so that it becomes longer
except if $i = 1$, but we may also pump it `down' ($i = 0$)
so that the string becomes shorter. However, one can pump down 
only once. Using the Pumping Lemma we can show that the language
$\{\mbox{\tt a}^n \mbox{\tt b}^n \mbox{\tt c}^n : n \in \omega\}$
is not context free.

For suppose the contrary. Then there is an $m$ such that
for all $k \geq m$ the string $\mbox{\tt a}^k \mbox{\tt b}^k \mbox{\tt c}^k$
can be decomposed into
%%
\begin{equation}
\mbox{\tt a}^k \mbox{\tt b}^k \mbox{\tt c}^k
= \vec{u} \conc \vec{v} \conc \vec{w} \conc \vec{x}
\conc \vec{y}
\end{equation}
%%
Furthermore there is an $\ell > k$ such that
%%
\begin{equation}
\mbox{\tt a}^{\ell} \mbox{\tt b}^{\ell} \mbox{\tt c}^{\ell} =
\vec{u} \conc {\vec{v}\,}^2 \conc \vec{w} \conc {\vec{x}\,}^2 \conc
\vec{y}
\end{equation}
%%
The string $\vec{v} \conc \vec{x}$ contains exactly $\ell - k$ times
the letters {\tt a}, {\tt b} and {\tt c}. It is clear that we must
have $\vec{v} \subseteq \mbox{\tt a}^{\ast} \cup
\mbox{\tt b}^{\ast} \cup \mbox{\tt c}^{\ast}$.
For if $\vec{v}$ contains two distinct letters, say
{\tt b} and {\tt c}, then $\vec{v}$ contains an occurrence
of {\tt b} before an occurrence of {\tt c} (certainly not the
other way around). But then ${\vec{v}\,}^2$ contains
an occurrence of {\tt c} before an occurrence of {\tt b},
and that cannot be. Analogously it is shown that
$\vec{y} \in \mbox{\tt a}^{\ast} \cup \mbox{\tt b}^{\ast}
\cup \mbox{\tt c}^{\ast}$. But this is a contradiction.
We shall meet this example of a non--CFL quite often in the 
sequel.

The second example of a context free graph grammar shall be
the so--called {\it tree adjunction grammars}. We take an
alphabet $A$ and a set $N$ of nonterminals.
%%%
\index{centre tree}%%
\index{adjunction tree}%%
%%%
A \textbf{centre tree} is an ordered labelled tree over $A \cup N$
such that all leaves have labels from $A$ all other nodes labels
from $N$. An \textbf{adjunction tree} is an ordered labelled tree 
over $A \cup N$ which is distinct from ordinary trees in that of 
the leaves there is exactly one with a nonterminal label; this 
label is the same as that of the root.  Interior nodes have 
nonterminal labels. We require that an adjunction tree has at 
least one leaf with a terminal symbol.
%%%
\index{tree adjunction grammar!unregulated}%%
\index{UTAG}%%
%%%
An \textbf{unregulated tree adjunction grammar}, briefly \textbf{UTAG},
over $N$ and $A$, is a quadruple $\auf \BC, N, A, \BA\zu$ where
$\BC$ is a finite set of centre trees over $N$ and $A$, and $\BA$ a
finite set of adjunction trees over $N$ and $A$. An example of a
tree adjunction is given in Figure~\ref{fig:baumadjunktion}. The
tree to the left is adjoined to a centre tree with root $X$ and
associated string {\tt bXb}; the result is shown to the right.
Tree adjunction can formally be defined as follows.
%%
\begin{figure}
\begin{center}
\begin{picture}(18,18)
\put(9,16){\line(-1,-1){7}}
\put(9,16){\line(1,-1){7}}
\put(2,9){\line(1,0){14}}
\put(9,12.5){\makebox(0,0){\tt X}}
\put(9,12){\line(-1,-1){3}}
\put(9,12){\line(1,-1){3}}
\put(2,8.5){\makebox(0,0){\tt Y}}
\put(4,8.5){\makebox(0,0){\tt c}}
\put(6,8.5){\makebox(0,0){\tt a}}
\put(9,8.5){\makebox(0,0){\tt A}}
\put(12,8.5){\makebox(0,0){\tt a}}
\put(14,8.5){\makebox(0,0){\tt A}}
\put(16,8.5){\makebox(0,0){\tt a}}
\end{picture}
\qquad
\begin{picture}(18,18)
\put(2,8.5){\makebox(0,0){\tt Y}}
\put(4,8.5){\makebox(0,0){\tt c}}
\put(6,2.5){\makebox(0,0){\tt a}}
\put(9,2.5){\makebox(0,0){\tt A}}
\put(12,2.5){\makebox(0,0){\tt a}}
\put(14,8.5){\makebox(0,0){\tt A}}
\put(16,8.5){\makebox(0,0){\tt a}}
\put(6,3){\line(1,0){6}}
\put(6,3){\line(1,1){3}}
\put(12,3){\line(-1,1){3}}
\put(4,5.5){\makebox(0,0){\tt b}}
\put(4,6){\line(1,1){5}}
\put(4,6){\line(1,0){10}}
\put(14,5.5){\makebox(0,0){\tt b}}
\put(14,6){\line(-1,1){5}}
\put(2,9){\line(1,1){7}}
\put(2,9){\line(1,0){5}}
\put(16,9){\line(-1,1){7}}
\put(16,9){\line(-1,0){5}}
\put(9,11.5){\makebox(0,0){\tt X}}
\put(9,6.5){\makebox(0,0){\tt X}}
\end{picture}
\end{center}
\caption{Tree Adjunction}
\label{fig:baumadjunktion}
\end{figure}
%%
Let $\GB = \auf B, <, \sqsubset, \ell\zu$ be a tree and
$\GA = \auf A, <, \sqsubset, m \zu$ an adjunction tree.
We assume that $r$ is the root of $\GA$ and that $s$ is
the unique leaf such that $m(r) = m(s)$. Now let $x$ be a node
of $B$ such that $\ell(x) = m(r)$. Then the replacement
of $x$ by $\GB$ is defined by naming the colour functionals.
These are
%%
\begin{align}
\goth{II}_{\rho}(y,\sqsubset) & :=
    \begin{cases}
        \{\sqsubset, <\} & \text{if $s \sqsubset y$,} \\
                \{\sqsubset\} & \text{else.}
    \end{cases}
     &
\goth{OI}_{\rho}(y, \sqsubset) & := \varnothing \\\notag
\goth{IO}_{\rho}(y, \sqsubset) & := 
    \begin{cases}
    \{<\} & \text{if $y \sqsubset s$,} \\
    \varnothing & \text{else.}
    \end{cases} &
\goth{OO}_{\rho}(y,\sqsubset) & := \{\sqsubset\} \\
%%%
\goth{II}_{\rho}(y, <)         & := 
	\begin{cases} 
		\{<\} & \text{if $y \geq s$,} \\
		\varnothing & \text{else.} 
	\end{cases}
	&
    \goth{IO}_{\rho}(y, <) & := \varnothing \\\notag
\goth{OI}{\rho}(y, <) & := \varnothing &
    \goth{OO}_{\rho}(y,<) & := \{<\}
\end{align}
%%
Two things may be remarked.  First, instead of a single
start graph we have a finite set of them. This can be
remedied by standard means. Second, all vertex colours
are terminal as well as nonterminal. One may end the
derivation at any given moment. We have noticed in
connection with grammars  for strings that this can be
remedied. In fact, we have not defined context free 
$\gamma$--grammars but context free quasi 
$\gamma$--grammars$^{\ast}$. However, we shall refrain
from being overly pedantic. Suffice it to note that the
adjunction grammars do not define the same kind of
generative process if defined exactly as above.

Finally we shall give a graph grammar which generates all
strings of the form $\mbox{\tt a}^n \mbox{\tt b}^n
\mbox{\tt c}^n$, $n>0$. The idea for this grammar is due to
Uwe M\"onnich 
%%%
\index{M\"onnich, Uwe}%%%
%%%
\shortcite{moennich:cloning}. We shall exploit
the fact that we may think of terms as structures. We posit
a ternary symbol, {\tt F}, which is nonterminal, and another
ternary symbol, {\tt f}, which is terminal. Further, there
is a binary terminal symbol $^{\smallfrown}$. The rules are
as follows. (To enhance readability we shall not write terms
in Polish Notation but by means of brackets.)
%%
\begin{equation}
\begin{split}
\mbox{\tt F}(x, y, z) & \pf \mbox{\tt F}(\mbox{\tt a}{^{\smallfrown}x},
\mbox{\tt b}{^{\smallfrown}y},
    \mbox{\tt c}{^{\smallfrown}z}), \\
\mbox{\tt F}(x,y,z) & \pf \mbox{\tt f}(x,y,z). \\
\end{split}
\end{equation}
%%
%%%
\index{term replacement system}%%
%%%
These rules constitute a so--called \textbf{term replacement system}.
The start term is $\mbox{\tt F}(\mbox{\tt a}, \mbox{\tt b}, \mbox{\tt c})$.
Now suppose that $u \pf v$ is a rule and that we have derived
a term $t$ such that $u^{\sigma}$ occurs in $t$ as a subterm.
Then we may substitute this occurrence by $v^{\sigma}$.
Hence we get the following derivations.
%%
\begin{equation}
\begin{split}
\mbox{\tt F}(\mbox{\tt a},\mbox{\tt b},\mbox{\tt c}) &
    \pf \mbox{\tt f}(\mbox{\tt a},\mbox{\tt b},\mbox{\tt c}), \\
\mbox{\tt F}(\mbox{\tt a},\mbox{\tt b},\mbox{\tt c}) &
    \pf \mbox{\tt F}(\mbox{\tt a}{^{\smallfrown}\mbox{\tt a}},
        \mbox{\tt b}{^{\smallfrown}\mbox{\tt b}},
    \mbox{\tt c}{^{\smallfrown}\mbox{\tt c}}) \\
    & \pf \mbox{\tt f}(\mbox{\tt a}{^{\smallfrown}\mbox{\tt a}},
    \mbox{\tt b}{^{\smallfrown}\mbox{\tt b}},
        \mbox{\tt c}{^{\smallfrown}\mbox{\tt c}}) \\
\mbox{\tt F}(\mbox{\tt a},\mbox{\tt b},\mbox{\tt c}) &
    \pf \mbox{\tt F}(\mbox{\tt a}{^{\smallfrown}\mbox{\tt a}},
        \mbox{\tt b}{^{\smallfrown}\mbox{\tt b}},
    \mbox{\tt c}{^{\smallfrown}\mbox{\tt c}}) \\
	&
    \pf \mbox{\tt F}(\mbox{\tt a}{^{\smallfrown}{(\mbox{\tt
        a}^{\smallfrown}\mbox{\tt a})}},
    \mbox{\tt b}{^{\smallfrown}{(\mbox{\tt
        b}^{\smallfrown}\mbox{\tt b})}},
    \mbox{\tt c}{^{\smallfrown}{(\mbox{\tt
        c}^{\smallfrown}\mbox{\tt c})}}) \\
    & \pf \mbox{\tt f}(\mbox{\tt a}{^{\smallfrown}{(\mbox{\tt
        a}^{\smallfrown}\mbox{\tt a})}},
    \mbox{\tt b}{^{\smallfrown}{(\mbox{\tt
        b}^{\smallfrown}\mbox{\tt b})}},
    \mbox{\tt c}{^{\smallfrown}{(\mbox{\tt
        c}^{\smallfrown}\mbox{\tt c})}})
\end{split}
\end{equation}
%%
Notice that the terms denote graphs here. We make use of the
dependency coding. Hence the associated strings to these terms
are {\tt abc}, {\tt aabbcc} and {\tt aaabbbccc}.

In order to write a graph grammar which generates the graphs
for these terms we shall have to introduce colours for edges.
Put $F_E := \{\uli{0}, \uli{1}, \uli{2}, \sqsubset, <\}$,
$F_V := \{\mbox{\tt F}, \mbox{\tt f}, \mbox{\tt a}, \mbox{\tt b}, 
\mbox{\tt c}\}$, and $F_V^T := \{\mbox{\tt f}, \mbox{\tt a}, 
\mbox{\tt b}, \mbox{\tt c}\}$.
The start graph is as follows. It has four vertices,
$p$, $q$, $r$ and $s$. ($<$ is empty (!), and
$q \sqsubset r \sqsubset s$.) The labelling is
$p \mapsto \mbox{\tt F}$, $q \mapsto \mbox{\tt a}$,
$r \mapsto \mbox{\tt b}$ and $s \mapsto \mbox{\tt c}$.
%%
\begin{equation}
\begin{array}{l}
\begin{picture}(6,6)
\put(1,1.5){\makebox(0,0){$\bullet$}}
    \put(1,1.5){\vector(1,1){2.8}}
    \put(1.7,3){\makebox(0,0)[r]{$\uli{0}$}}
    \put(1,.5){\makebox(0,0){\tt a}}
\put(4,1.5){\makebox(0,0){$\bullet$}}
    \put(4,1.5){\vector(0,1){2.8}}
    \put(3.5,3){\makebox(0,0)[r]{$\uli{1}$}}
    \put(4,.5){\makebox(0,0){\tt b}}
\put(7,1.5){\makebox(0,0){$\bullet$}}
    \put(7,1.5){\vector(-1,1){2.8}}
    \put(6.3,3){\makebox(0,0)[l]{$\uli{2}$}}
    \put(7,.5){\makebox(0,0){\tt c}}
\put(4,4.5){\makebox(0,0){$\bullet$}}
    \put(4,5.5){\makebox(0,0){\tt F}}
\end{picture}
\end{array}
\end{equation}
%%
There are two rules of replacement. The first can be written
schematically as follows. The root, $x$, carries the label {\tt F}
and has three incoming edges; their colours are
$\uli{0}$, $\uli{1}$ and $\uli{2}$. These come from three
disjoint subgraphs, $\GG_0$, $\GG_1$ and $\GG_2$, which are
ordered trees with respect to $<$ and $\sqsubset$ and in which
there are no edges with colour $\uli{0}$, $\uli{1}$ and $\uli{2}$.
In replacement, $x$ is replaced by a graph consisting of
seven vertices, $p$, $q_i$, $r_i$ and $s_i$, $i < 2$,
where $q_i \sqsubset r_j \sqsubset s_k$, $i,j,k < 2$,
and $q \stackrel{\uli{0}}{\pf} p$, $r \stackrel{\uli{1}}{\pf} p$
and $s \stackrel{\uli{2}}{\pf} p$. $< = \{\auf q_1, q_0\zu,
\auf r_1, r_0\zu, \auf s_1, s_0\zu\}$. The colouring is
%%
\begin{equation}
\begin{array}{rlrlrlrl}
p & \mapsto \mbox{\tt F} & q_0 & \mapsto ^{\smallfrown} & 
r_0 & \mapsto ^{\smallfrown} & s_0 & \mapsto ^{\smallfrown} 
\\ 
 & & q_1 & \mapsto \mbox{\tt a} & r_1 & \mapsto \mbox{\tt b} 
& s_1 & \mapsto \mbox{\tt c}
\end{array}
\end{equation}
%%
(With $\{p,q_0, r_0, s_0\}$ we reproduce the begin situation.)
The tree $\GG_0$ is attached to $q_0$ to the right of $q_1$,
$\GG_1$ to $r_0$ to the right of $r_1$ and
$\GG_2$ to $s_0$ to the right of $s_1$.
Additionally, we put $x < p$ for all vertices $x$ of the $\GG_i$.
(So, the edge $\auf x, p\zu$ has colour $<$ for all such $x$.)
By this we see to it that in each step the union of the
relations $<$, $\uli{0}$, $\uli{1}$ and $\uli{2}$ is the
intended tree ordering and that there always exists
an ingoing edge with colour $\uli{0}$, $\uli{1}$ and
$\uli{2}$ into the root.

The second replacement rule replaces the root
by a one vertex graph with label {\tt f} at the root.
This terminates the derivation. The edges with label
$\uli{0}$, $\uli{1}$ and $\uli{2}$ are transmitted under the
name $<$. This completes the tree. It has the desired form.
%%
\vplatz
\exercise
Strings can also be viewed as multigraphs with only one edge 
colour. Show that a CFG for strings can also be defined as a 
context free $\gamma$--grammar on strings. We shall show in 
Section~\ref{kap2}.\ref{kap2-5} that CFLs can also be generated by UTAGs,
but that the converse does not hold.
%%
\vplatz
\exercise
Show that for every context free $\gamma$--grammar $\Gamma$
there exists a context free  $\gamma$--grammar $\Delta$
which has no rules of productivity $- 1$ and which generates
the same class of graphs.
%%
\vplatz
\exercise
Show that for every context free $\gamma$--grammar there exists
a context free $\gamma$--grammar with the same yield and no
rules of productivity $\leq 0$.
%%
\vplatz
\exercise
Define unregulated string adjunction grammars in a similar way to 
UTAGs. Take note of the fact that these are quasi--grammars. Characterize 
the class of strings generated  by these grammars in terms of 
ordinary grammars.
%%
\vplatz %%
\exercise %%
Show that the language $\{\vec{w} \conc \vec{w}
: \vec{w} \in A^{\ast}\}$ is not context free but that it
satisfies the Pumping Lemma. 
%%%
\index{Pumping Lemma}%%%
\index{Interchange Lemma}%%%
%%%
(It does not satisfy the Interchange Lemma (\ref{thm:interchange}).)
%%

 \section{Turing machines}
\label{einsfuenf}
%
%
%
We owe to \cite{turing:computable} and \cite{post:combinatory} the 
%%%
\index{Turing, Alan}\index{Post, Emil}%%
%%%
concept of a machine which is very simple and nevertheless capable 
of computing all functions that are believed to be computable. 
Without going into the details of what makes a function computable, 
it is nowadays agreed that there is no loss
if we define `computable' to mean {\it computable by a Turing
machine}. The essential idea was that computations on objects
can be replaced by computations on strings. The number $n$ can
for example be represented by $n+1$ successive strokes on a piece of
paper. (So, the number $0$ is represented by a single stroke.
This is really necessary.) In addition to the stroke we have a
blank, which is used to separate different numbers. The Turing
machine, however powerful, takes a lot of time to compute even the
most basic functions. Hence we agree from the start that it has an
arbitrary, finite stock of symbols that it can use in addition to 
the blank. A Turing machine is a physical device, consisting of a 
tape which is infinite in both directions. That is, it contains 
cells numbered by the set of integers (but the numbering is irrelevant 
for the computation). Each cell may carry a symbol 
from an alphabet $A$ or a blank. The machine possesses a read and 
write head, which can move between the cells, one at a time. 
Finally, it has finitely many states, and can be programmed
in the following way. We assign instructions for the machine 
that tell it what to do on condition that it is in state $q$ 
and reads a symbol $a$ from the tape. These instruction tell the 
machine whether it should write a symbol, then move the head one 
step or leave it at rest, and subsequently change to a state 
$q'$.
%%
\begin{defn}
%%%
\index{Turing machine}%%
\index{Turing machine!deterministic}%%
\index{blank}%%
\index{state}%%
\index{state!initial}%%
\index{state!accepting}%%
%%%
A \textbf{(nondeterministic) Turing machine} is a quintuple
%%%
\begin{equation}
T = \auf A, L, Q, q_0, f\zu
\end{equation}
%%%
where $A$ is a finite set, the 
\textbf{alphabet}, $L \not\in A$ is the so--called \textbf{blank},
$Q$ a finite set, the set of (\textbf{internal}) \textbf{states},
$q_0 \in Q$ the  \textbf{initial state} and 
%%
\begin{equation}
f \colon A_L \times Q 
\pf \wp(A_L\times \{-1,0,1\} \times Q)
\end{equation}
%%%
the \textbf{transition function}.  If for all $b \in A_L$ and 
$q \in Q$ $|f(b,q)| \leq 1$, the machine is called 
\textbf{deterministic}.
\end{defn}
%%
Here, we have written $A_L$ in place of $A \cup \{L\}$. Often,
we use {\tt L} or even $\square$ as particular blanks. What
this describes physically is a machine that has a two--sided
infinite tape (which we can think of as a function $\tau \colon
\BZ \pf A_L$), with a read/write head positioned on one of the cells.
A \textbf{computation step} 
%%%
\index{computation step}%%%
%%%
is as follows. Suppose the machine
scans the symbol $a$ in state $q$ and is on cell $i \in \BZ$.
Then if $\auf b, 1, q'\zu \in f(a,q)$, the machine may write
$b$ in place of $a$, advance to cell $i+1$ and change to state
$q'$. If $\auf b, 0, q'\zu \in f(a,q)$ the machine may
write $b$ in place of $a$, stay in cell $i$ and change
to state $q'$. Finally, if $\auf b,-1,q'\zu \in f(a,q)$,
the machine may write $b$ in place of $a$, move to cell $i-1$
and switch to state $q'$. Evidently, in order to describe
the process we need (i) the tape, (ii) the position of the
head of that tape, (iii) the state the machine is currently
in. We assume throughout that the tape is almost everywhere 
filled by a blank. (The locution `almost all' and `almost everywhere' 
is often used in place `all but finitely many' and `all but finitely 
many places', respectively.) This means that 
the content of the tape plus the information on the machine may be 
coded by a single string, called {\it configuration}. Namely, if the 
tape is almost everywhere filled by a blank, there is a unique 
interval $[m, n]$ which contains all non--blank squares and 
the head of the machine. Suppose that the machine head is 
on Tape~$\ell$. Then let $\vec{x}_1$ be the string defined 
by the interval $[m, \ell -1]$ (it may be empty), and 
$\vec{x}_2$ the string defined by the interval $[\ell, n]$. 
Finally, assume that the machine is in state $q$. Then the 
string $\vec{x}_1\conc q \conc \vec{x}_2$ is the configuration 
corresponding to that phyical configuration.  So, the state
of the machine is simply written behind the symbol of the
cell that is being scanned. (Obviously, $A$ and $Q$ are assumed 
to be disjoint.)
%%%
\begin{defn}
%%
\label{defn:configuration}
\index{configuration}%%
%%%
Let $T = \auf A, \mbox{\tt L}, Q, q_0, f\zu$ be a Turing machine.
A $T$--\textbf{con\-fi\-gu\-ra\-tion} is a string $\vec{x}q\vec{y} \in
A_L^{\ast} \times Q \times A_L^{\ast}$ such that
$\vec{x}$ does not begin and $\vec{y}$ does not end
with a blank.
\end{defn}
%%%
This configuration corresponds to a situation that the tape
is almost empty (that is, almost all occurrences of symbols on
it are blanks). The nonempty part is a string $\vec{x}$, with
the head being placed somewhere behind the prefix $\vec{u}$.
Since $\vec{x} = \vec{u}\, \vec{v}$ for some $\vec{v}$, we insert
the state the machine is in between $\vec{u}$ and $\vec{v}$.
The configuration omits most of the blanks, whence we have
agreed that $\vec{u}q\vec{v}$ is the same configuration as
$\square\vec{u}q\vec{v}$ and the same $\vec{u}q\vec{v}\square$.

We shall now describe the working of the machine using configurations.
We say,
%%%
\index{$\vec{x} \conc q \conc \vec{y} \vdash_T
    \vec{x}_1 \conc q_1 \conc \vec{y}_1$}%%
%%%
$\vec{x} \conc q \conc \vec{y}$ is \textbf{transformed by} $T$
\textbf{in one step into} $\vec{x}_1 \conc q_1 \conc \vec{y}_1$
and write $\vec{x} \conc q \conc \vec{y} \vdash_T
\vec{x}_1 \conc q_1 \conc \vec{y}_1$ if one of the following
holds.
%%
\begin{dingautolist}{192}
\item $\vec{x}_1 = \vec{x}$, and for some $\vec{v}$ and
    $b$ and $c$ we have
    $\vec{y} = b \conc \vec{v}$ and $\vec{y}_1 = c \conc  \vec{v}$,
    as well as $\auf c, 0, q_1\zu  \in f(b,q)$.
\item We have $\vec{x}_1 = \vec{x} \conc c$ and
    $\vec{y} = b \conc \vec{y}_1$ as well as
     $\auf c, 1, q_1\zu \in f(b,q)$.
\item We have $\vec{x} = \vec{x}_1 \conc c$ and
    $\vec{y}_1 = b \conc \vec{y}$
    as well as $\auf c, -1, q_1\zu \in f(b,q)$.
\end{dingautolist}
%%
Now, for $T$--configurations
$Z$ and $Z'$ we define $Z \vdash^n_T Z'$ inductively by
(a) $Z \vdash_T^0 Z'$ iff $Z = Z'$ and
(b) $Z \vdash_T^{n+1} Z'$ iff for some $Z''$
we have $Z \vdash_T^n Z'' \vdash_T Z'$.

It is easy to see that we can define a semi Thue system on
configurations that mimicks the computation of $T$. The canonical 
Thue system, $C(T)$, 
%%%
\index{$C(T)$}%%
%%%
is shown in Table~\ref{tab:canThue}. 
($x$ and $y$ range over $A_L$ and $q$ and $q'$ over $Q$.)
%%
\begin{table}
\caption{The Canonical Thue System}
\label{tab:canThue}
$$\begin{array}{lll}
C(T) := &  & \{\auf \vec{u}qx\vec{v}, \vec{u}yq'\vec{v}\zu :
    \auf y,1,q'\zu \in f(x,q); \vec{u} \neq \varepsilon \mbox{ or }
    y \in A; \\
    & & \multicolumn{1}{r}{ \vec{v} \neq \varepsilon \mbox{ or } x \in A\}} \\
& \cup & \{\auf \vec{u}q, \vec{u}yq'\zu : \auf y,1,q'\zu \in
    f(\square,q); \vec{u} \neq \varepsilon
    \mbox{ or } y \in A\} \\
& \cup & \{\auf qx\vec{v}, q'\vec{v}\zu : \auf \square,1,q'\zu
    \in f(x,q); \vec{v} \neq \varepsilon \mbox{ or } x\in A\} \\
& \cup & \{\auf q, q'\zu : \auf \square,\alpha,q'\zu
    \in f(\square,q), \alpha \in \{-1,0,1\}\} \\
& \cup & \{\auf \vec{u}xq\vec{v}, \vec{u}q'y\vec{v}\zu :
    \auf y,-1,q'\zu \in f(x,q); \\
    & & \multicolumn{1}{r}{
    \vec{u} \neq \varepsilon \mbox{ or } x \in A; 
    \vec{v} \neq \varepsilon \mbox{ or } y \in A\}} \\
& \cup & \{\auf q\vec{v}, q'y\vec{v}\zu : \auf y,-1,q'\zu \in
    f(\square,q); \vec{v} \neq \varepsilon \mbox{ or }
    y \in A\} \\
& \cup & \{\auf \vec{u}xq, \vec{u}q'\zu :
    \auf \square,-1,q'\zu \in f(x,q); \vec{u} \neq
    \varepsilon \mbox{ or } x \in A\} \\
& \cup & \{\auf \vec{u}qx\vec{v}, \vec{u}q'y\vec{v}\zu :
    \auf y,0,q'\zu \in f(x,q); \\
    & & \multicolumn{1}{r}{\vec{v} \neq \varepsilon\mbox{ or }
    x, y \in A\}}
\end{array}$$
\end{table}
%%
Notice that we have to take care not to leave a blank at the
left and right end of the strings. This is why the definition is
more complicated than expected. The alphabet of the semi Thue system
is $(Q \cup A_L)^{\ast}$. The following is easily shown
by induction.
%%
\begin{prop}
Let $T$ be a Turing machine, $C(T)$ be its associated semi Thue
system. Then for all $T$--configurations $Z$ and $Z'$ and for all 
$n >0$: $Z \vdash^n_T Z'$ iff $Z \Pf_{C(T)}^n Z'$.
Moreover, if $Z$ is a $T$--configuration and $Z \Pf_{C(T)}^n
\vec{u}$ for an arbitrary string $\vec{u} \in (Q \cup A_L)^{\ast}$,
then $\vec{u}$ is a $T$--configuration and $Z \vdash^n_T \vec{u}$.
\end{prop}
%%
Of course, the semi Thue system defines transitions on strings
that are not configurations, but this is not relevant for the
theorem.
%%
\begin{defn}
%%%
\index{end configuration}%%
\index{language!accepted}%%
\index{$L(T)$}%%
%%%
Let $T$ be a Turing machine, $Z$ a configuration
and $\vec{x} \in A^{\ast}$. $Z$ is called an
\textbf{end configuration} if there is no configuration $Z'$
such that $Z \vdash_T Z'$. $T$ \textbf{accepts} $\vec{x}$
if there is an end configuration $Z$ such that
$q_0 \conc \vec{x} \vdash^{\ast}_T Z$. The \textbf{language
accepted by} $T$, $L(T)$, is the set of all strings from
$A^{\ast}$ which are accepted by $T$.
\end{defn}
%%
It takes time to get used to the concept of a Turing
machine and the languages that are accepted by such
machines. We suggest to the interested reader to play
a little while with these machines and see if he can
program them to compute a few very easy functions.
A first example is the machine which computes the successor 
function on binary strings. Assume our alphabet is $\{\mbox{\tt 0}, %
\mbox{\tt 1}\}$.  We want to build a machine which computes the 
next string for $\vec{x}$ in the numerical encoding (see 
Section~\ref{kap1}.\ref{einseins} for its definition). This means that if 
the machine starts with $q_0 \conc \vec{x}$ it shall halt in the 
configuration $q_0 \conc \vec{y}$ where $\vec{y}$ is the word 
immediately following $\vec{x}$ in the numerical ordering. (If in the
sequel we think of numbers rather than strings we shall simply
think instead of the string $\vec{x}$ of the number $n$, where
$\vec{x}$ occupies the $n$th place in the numerical ordering.)

How shall such a machine be constructed? We need four states,
$q_i$, $i < 4$. First, the machine advances the head to the right
end of the string, staying in $q_0$ until it reads $\square$.
Finally, when it hits $\square$, it changes to state $q_1$
and starts moving to the left. As long as it reads {\tt 1},
it changes {\tt 1} to {\tt 0} and continues in state $q_1$,
moving to the left. When it hits {\tt 0}, it replaces it by
{\tt 1}, moves left and changes to state $q_2$. When it sees
a blank, that blank is filled by {\tt 0} and the machine
changes to state $q_3$, the final state. In $q_2$, the machine
simply keeps moving leftwards until it hits a blanks and then
stops in state $q_3$. The machine is shown in
Table~\ref{tab:successormachine}.
%%%
\begin{table}
\caption{The Successor Machine}
\label{tab:successormachine}
$$\begin{array}{ll@{\quad\mapsto\quad}l}
q_0 & \mbox{\tt 0} & \auf \mbox{\tt 0}, 1, q_0\zu \\
    & \mbox{\tt 1} & \auf \mbox{\tt 1}, 1, q_0\zu \\
    & \square      & \auf \square, -1, q_1\zu \\
q_1 & \mbox{\tt 0} & \auf \mbox{\tt 1}, -1, q_2\zu \\
    & \mbox{\tt 1} & \auf \mbox{\tt 0}, -1, q_1\zu \\
    & \square      & \auf \mbox{\tt 0}, -1, q_3\zu \\
q_2 & \mbox{\tt 0} & \auf \mbox{\tt 0}, -1, q_2\zu \\
    & \mbox{\tt 1} & \auf \mbox{\tt 1}, -1, q_2\zu \\
    & \square      & \auf \square, 0, q_3\zu \\
q_3 & \multicolumn{2}{c}{}
\end{array}$$
\end{table}
%%
(If you want a machine that computes the successor in the binary 
encoding, you have to replace Line 6 by 
$\square \mapsto \auf \mbox{\tt 1}, -1, q_3\zu$.)
In recursion theory the notions of computability are defined
for functions on the set of natural numbers. By means of the
function $Z$, which is bijective, these notions can be
transferred to functions on strings.
%%
\begin{defn}
%%%
\index{function!computable}%%
%%%
Let $A$ and $B$ be alphabets and $f \colon A^{\ast} \pf B^{\ast}$ a 
function. $f$ is called \textbf{computable} if there is a
deterministic Turing machine $T$ such that for every
$\vec{x} \in A^{\ast}$ there is a $q_t \in Q$ such that 
$q_0 \conc \vec{x} \vdash_T^{\ast} q_t \conc f(\vec{x})$ 
and $q_t \conc f(\vec{x})$ is
an end configuration. Let $L \subseteq A^{\ast}$. $L$
is called \textbf{recursively enumerable} 
%%%
\index{language!recursively enumerable}%%
%%%
if $L = \varnothing$ or
there is a computable function $f \colon \{\mbox{\tt 0},
\mbox{\tt 1}\}^{\ast} \pf A^{\ast}$ such that
$f[\{\mbox{\tt 0}, \mbox{\tt 1}\}^{\ast}] = L$.
$L$ is \textbf{decidable} if both $L$ and $A^{\ast} - L$ are 
recursively enumerable.
%%%
\index{language!decidable}%%
%%%
\end{defn}
%%
\begin{lem}
Let $f \colon A^{\ast} \pf B^{\ast}$ and $g \colon B^{\ast} \pf C^{\ast}$
be computable functions. Then $g \circ f \colon A^{\ast} \pf C^{\ast}$
is computable as well.
\end{lem}
%%%
The proof is a construction of a machine $U$ from machines $T$ and
$T'$ computing $f$ and $g$, respectively. Simply write $T$ and
$T'$ using disjoint sets of states, and then take the union of
the transition functions. However, make the transition function of
$T$ first such that it changes to the starting state of $T'$ as
soon as the computation by $T$ is finished (that is, whenever $T$ 
does not define any transitions).
%%%
\begin{lem}
Let $f \colon A^{\ast} \pf B^{\ast}$ be computable and bijective. Then
$f^{-1} \colon B^{\ast} \pf A^{\ast}$ also is computable (and bijective).
\end{lem}
%%%
Write a machine that generates all strings of $A^{\ast}$ in successive
order (using the successor machine, see above), and computes
$f(\vec{x})$ for all these strings. As soon as the target string
is found, the machine writes $\vec{x}$ and deletes everything else.
%%%
\begin{lem}
\label{lem:bij}
Let $A$ and $B$ be finite alphabets. Then there are computable
bijections $f \colon A^{\ast} \pf B^{\ast}$ and $g \colon 
B^{\ast} \pf A^{\ast}$ such that $f = g^{-1}$.
\end{lem}
%%
In this section we shall show that the recursively enumerable sets 
are exactly the sets which are accepted  by a Turing machine.
Further, we shall show that these are exactly the Type 0
languages. This establishes the first correspondence result
between types of languages and types of automata. Following
this we shall show that the recognition problem for Type 0
languages is in general not decidable. The proofs proceed
by a series of reduction steps for Turing machines. First,
we shall generalize the notion of a Turing machine.
%%%
\index{Turing machine!multitape}%%
%%%
A $k$--\textbf{tape Turing machine} is a quintuple
$\auf A, L, Q, q_0, f\zu$ where $A$,
$L$, $Q$, and $q_0$ are as before but now
%%
\begin{equation}
f \colon A_L^k \times Q \pf
    \wp(A_L^k \times \{-1, 0, 1\} \times Q) 
\end{equation}
%%
This means, intuitively speaking, that the Turing machine
manipulates $k$ tapes in place of a single tape. There is
a read and write head on each of the tapes. In each step
the machine can move only one of the heads. The next state depends
on the symbols read on all the tapes plus the current internal
state. The initial configuration is as follows. All tapes
except the first are empty. The heads are anywhere on these
tapes (we may require them to be in position 0). On the first
tape the head is immediately to the left of the input.
The $k$--tape machine has $k-1$ additional tapes for
recording intermediate results. The reader may verify that we
may also allow such configurations as initial configurations
in which the other tapes are filled with some finite string,
with the head immediately to the left of it. This does not
increase the recognition power. However, it makes the definition
of a machine easier which computes a function of several 
variables. We may also allow that the information to the right of the
head consists in a sequence of strings each separated by
a blank (so that when two successive blanks follow the machine 
knows that the input is completely read). Again, there is a
way to recode these machines using a basic multitape Turing
machine, modulo computable functions. We shall give a little
more detail concerning the fact that also $k$--tape Turing
machines (in whatever of the discussed forms) cannot compute
more functions than 1--tape machines. For this define the
following coding of the $k$ tapes using a single tape.
We shall group $2k$ cells together to a macro cell.
The (micro) cell $2kp + 2m$ corresponds to the entry on
cell $p$ on Tape~$m$. The (micro) cell number $2kp + 2m +1$
only contains {\tt 1} or {\tt 0} depending on whether the head 
of the machine is placed on cell $p$ on tape $m$. (Hence, every 
second micro cell is filled only with {\tt 1} or {\tt 0}.)
Now given a $k$--tape Turing machine $T$ we shall define a
machine $U$ that simulates $T$ under the given coding.
This machine operates as follows. For a single step of $T$
it scans the actual string for the positions of the read and
write heads and remembers the symbols on which they are
placed (they can be found in the adjacent cell). Remembering
this information requires only finite amount of memory, and
can be done using the internal states. The machine scans the
tape again for the head that will have to be changed in position. 
(To identify it, the machine must be able to do calculations 
modulo $2k$. Again finite memory is sufficient.) It adjusts its 
position and the content of the adjacent cell. Now it changes
into the appropriate state. Notice that each step of $T$
costs $2k \cdot |\vec{x}|$ time for $U$ to simulate, where
$\vec{x}$ is the longest string on the tapes. If there is
an algorithm taking $f(n)$ steps to compute then the
simulating machine needs at most $2k(f(n)+n)^2$ time to
compute that same function under simulation. (Notice
that in $f(n)$ steps the string(s) may acquire length at most
$f(n) + n$.)

We shall use this to show that the nondeterministic Turing
machines cannot compute more functions than the deterministic
ones.
%%
\begin{prop}
Let $L = L(T)$ for a Turing machine. Then there is a
deterministic Turing machine $U$ such that
$L = L(U)$.
\end{prop}
%%
\proofbeg
Let $L = L(T)$. Choose a number $b$ such that $|f(q,x)| < b$ for 
all $q \in Q$, $x \in A$. We fix an ordering on $f(q,x)$ for all 
$x$ and $q$. $V$ is a 3--tape machine that does the following. On the 
first tape $V$ writes the input $\vec{x}$. On the second tape we 
generate all sequences $\vec{p}$ of numbers $< b$ of length $n$, 
for increasing $n$. These sequences describe the action sequences 
of $T$. For each sequence $\vec{p} = a_0 a_1 \dotsb a_{n-1}$ we 
copy $\vec{x}$ from Tape~1 onto Tape~3 and let $V$ work as follows. 

The head on Tape~2 is to the left of the sequence $\vec{a}$. In the 
first step $V$ follows the $a_0$th alternative for machine $T$ on 
the 3rd tape and advances head number 2 one step to the right. In the 
second step it follows the alternative $a_1$ in the transition set of 
$T$ and executes it on Tape~3. Then the head of Tape~2 is advanced
one step to the right. If $a_{n-1} < b$ and the $a_{n-1}$st alternative 
does not exist for $T$ but there is a computation for 
$a_0a_1 \dotsb a_{n-2}a'$ for some $a' < a_{n-1}$, $V$ exits the 
computation on Tape~3 and deletes $\vec{p}$ on Tape~2. If $a_{n-1} = b$, 
the $a_{n-1}$st alternative does not exist for $T$, and none exists 
for any $a' < b$, then $V$ halts. In this way $V$ executes on Tape~3 
a single computation of $T$ for the input and checks the prefixes 
for paths for which a computation exists. 
Clearly, $V$ is deterministic. It halts iff for some $n$ $T$ halts 
on some alternative sequences of length $n-1$. 
\proofend

It is easy to see that we can also write a machine that enumerates 
all possible outputs of $T$ for a given input.
%%
\begin{lem}
\label{lem:aufz}
$L$ is recursively enumerable iff $L = L(T)$
for a Turing machine $T$.
\end{lem}
%%
\proofbeg
The case $L = \varnothing$ has to be dealt with separately.
It is easy to construct a machine that halts on no word. This
shows the equivalence in this case.  Now assume that $L \neq
\varnothing$. Let $L$ be recursively enumerable. Then there
exists a function $f \colon \{\mbox{\tt 0}, \mbox{\tt 1}\}^{\ast} 
\pf A^{\ast}$ such that $f[\{\mbox{\tt 0}, \mbox{\tt 1}\}^{\ast}] 
= L$ and a Turing machine $U$ which computes $f$. Now we construct 
a (minimally) 3--tape Turing machine $V$ as follows. The input 
$\vec{x}$ will be placed on the first tape. On the second tape $V$ 
generates all strings $\vec{y} \in \{\mbox{\tt 0},\mbox{\tt 1}\}^{\ast}$
starting with $\varepsilon$, in the numerical order. In order to do 
this we use the machine computing the successors in this ordering. 
If we have computed the string $\vec{y}$ on the second tape the 
machine computes the value $f(\vec{y})$ on
the third tape. (Thus, we emulate machine $T$ on the third tape,
with input given on the second tape.) Since $f$ is computable,
$V$ halts on Tape~3. Then it compares the string on Tape~3,
$f(\vec{y})$, with $\vec{x}$. If they are equal, it halts, if
not it computes the successor of $\vec{y}$ and starts the process 
over again. It is easy to see that $L = L(V)$. By the previous 
considerations, there is a one tape Turing machine $W$ such that 
$L = L(W)$. Now conversely, let $L = L(T)$ for some Turing machine $T$.
We wish to show that $L$ is recursively enumerable. We may
assume, by the previous theorem, that $T$ is deterministic.
We leave it to the reader to construct a machine $U$ which 
computes a function $f \colon \{\mbox{\tt 0},\mbox{\tt 1}\}^{\ast} 
\pf A^{\ast}$ whose image is $L$. 
\proofend
%%
\begin{thm}
\label{thm:typ0}
The following are equivalent.
\begin{dingautolist}{192}
\item
$L$ is of Type 0.
\item
$L$ is recursively enumerable.
\item
$L = L(T)$ for a Turing machine $T$.
\end{dingautolist}
\end{thm}
%%
\proofbeg
We shall show \ding{192} $\Pf$ \ding{193} and \ding{194} $\Pf$ \ding{192}. 
The theorem then follows with Lemma~\ref{lem:aufz}. Let $L$ be of Type 0. 
Then there is a grammar $\auf \mbox{\tt S}, N, A, R\zu$ which generates 
$L$. We have to construct a Turing machine which lists all strings 
that are derivable from {\tt S}. To this end it is enough to construct 
a nondeterministic machine that matches the grammar. This machine 
always starts at input {\tt S} and in each cycle it scans the string 
for a left hand side of a rule and replaces that substring by the 
right hand side. This shows \ding{193}. Now let $L = L(T)$ for some 
Turing machine. Choose the following grammar $G$: in addition to the 
alphabet let {\tt X} be the start symbol, {\tt 0} and {\tt 1} two 
nonterminals, and let each $q \in Q$ $\mbox{\tt Y}_q$ be a 
nonterminal. The rules are as follows.
%%
\begin{align}\notag
\mbox{\tt X}\pf & \mbox{\tt X0} \mid \mbox{\tt X1}
    \mid \mbox{\tt Y}_{q_0} \\\notag
\mbox{\tt Y}_q b \pf & c \mbox{\tt Y}_r 
	&& \text{if $\auf c, 1, r\zu \in f(b,q)$} \\
\mbox{\tt Y}_q b \pf & \mbox{\tt Y}_r c 
	&& \text{if $\auf c, 0, r\zu \in f(b,q)$} \\\notag
\mbox{\tt Y}_q b \pf & \mbox{\tt Y}_r c b 
	&& \text{if $\auf c, -1, r\zu \in f(b,q)$} \\\notag
\mbox{\tt Y}_q b \pf & b 
	&& \text{if $f(b,q) = \varnothing$}
\end{align}
%%
Starting with {\tt X} this grammar generates strings of the form
$\mbox{\tt Y}_{q_0} \vec{x}$, where $\vec{x}$ is a binary string.
This codes the input for $T$. The additional rules code in a
transparent way the computation of $T$ on the string. If the
computation stops, it is allowed to eliminate $\mbox{\tt Y}_q$.
If the string is terminal it will be generated by $G$. In this
way it is seen that $L(G) = L(T)$.
\proofend
%%

Now we shall derive an important fact, namely that there exist undecidable
languages of Type 0. We first of all note that Turing machines can be
regarded as semi Thue systems, as we have done earlier. Now one can
design a machine $U$ which takes two inputs, one being the code
of a Turing machine $T$ and the other a string $\vec{x}$, and
$U$ computes what $T$ computes on $\vec{x}$.
Such a machine is called a \textbf{universal Turing machine}%
%%%
\index{Turing machine!universal}%%
%%%
. The coding of Turing machines can
be done as follows. We only use the letters {\tt a}, {\tt b} and
{\tt c}, which are, of course, also contained in the alphabet $B$.
Let $A = \{\mbox{\tt a}_i : i < n\}$. Then let $\gamma(\mbox{\tt
a}_i)$ be the number $i$ in dyadic coding (over $\{\mbox{\tt
a},\mbox{\tt b}\}$, where {\tt a} replaces {\tt 0} and {\tt b}
replaces {\tt 1}). The number 0 is coded by {\tt a} to distinguish
it from $\varepsilon$. Furthermore, we associate the number $n$
with the blank, {\tt L}. The states are coded likewise; we assume
that $Q = \{0, 1, \dotsc, n-1\}$ for some $n$ and that $q_0 = 0$.
Now we still have to write down $f$. $f$ is a subset of
%%
\begin{equation}
A_{\mbox{\smtt L}} \times Q \times A_{\mbox{\smtt L}} \times
\{-1,0, 1\} \times Q 
\end{equation}
%%
Each element $\auf a, q, b, m, r\zu$ of $f$ can be written down as
%%
\begin{equation}
\vec{x} \conc \mbox{\tt c} \conc \vec{u} \conc \mbox{\tt c} \conc
\vec{\mu} \conc \mbox{\tt c} \conc \vec{y} \conc \mbox{\tt c} \conc
    \vec{v} \conc \mbox{\tt c} 
\end{equation}
%%
where $\vec{x} = \gamma(a)$, $\vec{u} = Z^{-1}(q)$,
$\vec{y} = \gamma(b)$, $\vec{v} = Z^{-1}(r)$.
Further, we have $\vec{\mu} = \mbox{\tt a}$ if $m = -1$,
$\vec{\mu} = \mbox{\tt b}$ if $m = 0$ and
$\vec{\mu} = \mbox{\tt ab}$ if $m = 1$.
Now we simply write down $f$ as a list, the entries being
separated by {\tt cc}. (This is not necessary, but is easier to
handle.) We call the code of $T$ $T^{\spadesuit}$. The set of
all codes of Turing machines is decidable. (This is essential
but not hard to see.) It should not be too hard to see that
there is a machine $U$ with two tapes, which for two strings
$\vec{x}$ and $\vec{y}$ does the following. If $\vec{y} =
T^{\spadesuit}$ for some $T$ then $U$ computes on $\vec{x}$
exactly as $T$ does. If $\vec{y}$ is not the code of a machine,
$U$ moves into a special state and stops.

Suppose that there is a Turing machine $V$ which decides for
given $\vec{x}$ and $T^{\spadesuit}$ wether or not $\vec{x} \in L(T)$.
Now we construct a two tape machine $W$ as follows. The input is
$\vec{x}$, and it is given on both tapes. If $\vec{x} =
T^{\spadesuit}$ for some $T$ then $W$ computes $T$ on $\vec{x}$.
(This is done by emulating $V$.) If $T$ halts on $\vec{x}$,
we send $W$ into an infinite loop. If $T$ does not halt,
$W$ shall stop. (If $\vec{x}$ is not the code of a machine,
the computation stops right away.) Now we have the following:
$W^{\spadesuit} \in L(W)$ exactly if $W^{\spadesuit} \not\in
L(W)$. For $W^{\spadesuit} \in L(W)$ exactly when $W$ stops
if applied to $W^{\spadesuit}$. This however is the case exactly
if $W$ does {\it not\/} stop. If on the other hand $W^{\spadesuit}
\not\in L(W)$ then $W$ does not stop if applied to $W^{\spadesuit}$,
which we can decide with the help of machine $V$, and then $W$
does halt on the input  $W^{\spadesuit}$.  Contradiction. Hence,
$V$ cannot exist. There is, then, no machine that can decide for
any Turing machine (in code) and any input whether that machine
halts on that string. It is still conceivable that this is
decidable for every $T$, but that we simply do
not know how to extract such an algorithm for given $T$.
Now, in order to show that this too fails, we use the universal
Turing machine $U$, in its single tape version. Suppose that $L(U)$
is decidable.  Then we can decide whether $U$ halts on
$\vec{x}\conc \mbox{\tt L}\conc T^{\spadesuit}$. Since $U$ 
is universal, this means that we can decide for given $T$ and 
given $\vec{x}$ whether $T$ halts on $\vec{x}$. We have seen 
above that this is impossible.
%%
\nocite{markov:impossibility}%%
\nocite{post:thue}%%
%%
\index{Post, Emil}%%
\index{Markov, A.~A.}%%
%%%
\begin{thm}[Markov, Post]
There is a recursively enumerable set which is not
decidable. 
\end{thm}
%%
So we also shown that the Type 1 languages are properly contained
in the Type 0 languages. For it turns out that the Type 1
languages are all decidable.
%%%
\index{Chomsky, Noam}%%%
%%%
\begin{thm}[Chomsky]
Every Type 1 language is decidable.
\end{thm}
%%
\proofbeg
Let $G$ be of Type 1 and let $\vec{x}$ be given. Put
$n := |\vec{x}|$ and $\alpha := |A \cup N|$. If there
is a derivation of $\vec{x}$ that has length $>
\alpha^n$, there is a string that occurs twice in it,
since all occurring strings must have length $\leq n$.
Then there exists a shorter derivation for $\vec{x}$.
So, $\vec{x} \in L(G)$ iff it has a
$G$--derivation of length $\leq \alpha^n$. This is
decidable.
\proofend
%%%
\begin{cor}
\label{0-1-echt}
$\mbox{\rm CSL} \subsetneq \mbox{\rm GL}$.
\end{cor}
%%%
Chomsky~\shortcite{chomsky:properties} credits Hilary Putnam 
%%%
\index{Putnam, Hilary}%%%
%%%
with the observation that not all decidable languages are of Type 1.
Actually, we can give a characterization of context sensitive
languages as well. Say that a Turing machine is \textbf{linearly
space bounded} 
%%%%
\index{Turing machine!linearly space bounded}%%
%%%%
if given input $\vec{x}$ it may use only
$O(|\vec{x}|)$ on each of its tapes. Then the following holds.
%%%
\nocite{myhill:lba}
\nocite{landweber:three}
\nocite{kuroda:classes}
\index{Landweber, Peter S.}%%
\index{Kuroda, S.--Y.}%%
%%%
\begin{thm}[Landweber, Kuroda]
A language $L$ is context sensitive iff $L = L(T)$
for some linear space bounded Turing machine $T$.
\end{thm}
%%
The proof can be assembled from Theorem~\ref{thm:noncontracting}
and the proof of Theorem~\ref{thm:typ0}.

We briefly discuss so--called {\it word problems}. Recall from
Section~\ref{kap1}.\ref{kap1-5} the definition of a Thue process $T$. Let
$A$ be an alphabet. Consider the monoid $\GZ(A)$. The set of
pairs $\auf s, t\zu \in A^{\ast} \times A^{\ast}$ such that
$s \Pf^{\ast}_T t$ is a congruence on $\GZ(A)$. Denote the
factor algebra by $\goth{Mon}(T)$. (One calls the pair 
$\auf A, T\zu$ a \textbf{presentation} 
%%%
\index{presentation}%%%
%%%
of $\goth{Mon}(T)$.)
It can be shown to be undecidable whether $\goth{Mon}(T)$ is
the one element monoid. From this one deduces that it is
undecidable whether or not $\goth{Mon}(T)$ is a finite
monoid, whether it is isomorphic to a given finite monoid,
and many more.

Before we close this chapter we shall introduce a few measures for
the complexity of computations. In what is to follow we shall
often have to deal with questions of how fast and with how much
space a Turing machine can compute a given problem.  Let $f \colon
\omega \pf \omega$ be a function, $T$ a Turing machine which
computes a function $g \colon A^{\ast} \pf B^{\ast}$. We say that $T$
needs $O(f)$--\textbf{space} if there is a constant $c$ such that for
all but finitely many $\vec{x} \in A^{\ast}$ there is a
computation of an accepting configuration $q_t \conc g(\vec{x})$ 
from $q_0 \conc \vec{x}$ in which every configuration has length 
$\leq c \times f(|\vec{x}|)$. 
For a multi tape machine we simply add the
lengths of all words on the tapes. We say that $T$ needs
$O(f)$--\textbf{time} if for almost all $\vec{x} \in A^{\ast}$ there
is a $k \leq c \times f(|\vec{x}|)$ such that $q_0 \conc \vec{x}
\vdash_T^{k} q_0 \conc g(\vec{x})$. We denote by $\mbox{\bf
DSPACE}(f)$ ($\mbox{\bf DTIME}(f)$)  the set of all functions
which for some $k$ are computable by a deterministic $k$--tape
Turing machine in $O(f)$--space ($O(f)$--time). Analogously the
notation $\mbox{\bf NSPACE}(f)$ and $\mbox{\bf NTIME}(f)$ is
defined for nondeterministic machines. We always have
%%
\begin{equation}
\mathbf{DTIME}(f) \subseteq \mathbf{NTIME}(f)
\subseteq \mathbf{NSPACE}(f)
\end{equation}
%%
as well as
%%
\begin{equation}
\mathbf{DSPACE}(f) \subseteq \mathbf{NSPACE}(f) 
\end{equation}
%%
For a machine can fill at most $k$ cells in $k$ steps, regardless 
of whether it is deterministic or nondeterministic. This applies 
as well to multi tape machines, since they can only write on one 
cell and move one head at a time.

The reason for not distinguishing  between the time complexity 
$f(n)$ and the $cf(n)$ ($c$ a constant) is the following result.
%%
\begin{thm}[Speed Up Theorem]
\label{thm:speedup}
Let $f$ be a computable function and let $T$ be a Turing machine
which computes $f(\vec{x})$  in at most
$g(|\vec{x}|)$ steps (using at most $h(|\vec{x}|)$ cells) where
$\inf_{n \pf \infty} g(n)/n = \infty$.
Further, let $c$ be an arbitrary real number $> 0$. Then there
exists a Turing machine $U$ which computes $f$ in at most
$c \cdot g(|\vec{x}|)$ steps (using at most $c \cdot h(|\vec{x}|)$
cells).
\end{thm}
%%
The proof results from the following fact. In place of the original 
alphabet $A_L$ we may introduce a new alphabet $B'_{L'} :=  A \cup B 
\cup \{L'\}$, where each symbol from $B$ corresponds to a sequence 
of length $k$ of symbols from $A_L$. The symbol $L'$ then corresponds 
to $L^k$. The alphabet $A_L$ is still used for giving the input. The 
new machine, upon receiving $\vec{x}$ recodes the input and calculates 
completely inside $B'_L$.

Since to each single letter corresponds a block of $k$
letters in the original alphabet, the space requirement
shrinks by the factor $k$. (However, we need to ignore the
length of the input.) Likewise, the time is cut by a factor
$k$, since one move of the head simulates up to $k$ moves.
However, the exact details are not so easy to sum up.
They can be found in \cite{hopcroftullman:formal}.

Typically, one works with the following complexity classes.
%%
\begin{defn}
%%%%
\index{PTIME}\index{NP}\index{PSPACE}\index{EXPTIME}\index{NEXPTIME}%%
%%%
\textbf{PTIME} is the class of functions computable in
deterministic polynomial time, \textbf{NP} the class
of functions computable in nondeterministic polynomial
time. \textbf{PSPACE} is the class of functions computable
in polynomial space, \textbf{EXPTIME} (\textbf{NEXPTIME}) the
class of functions computable in deterministic (nondeterministic)
exponential time.
\end{defn}
%%
\begin{defn}
\label{defn:complang}
A language $L \subseteq A^{\ast}$ is in a complexity class $\CP$
iff $\chi_L \in \CP$.
\end{defn}
%%
\mbox{}

{\it Notes on this section.} In the mid 1930s, several people have
independently studied the notion of feasibility. Alonzo Church
%%%
\index{Church, Alonzo}%%
%%%
and Stephen Kleene 
%%%
\index{Kleene, Stephen C.}%%%
%%%
have defined the notion of $\lambda$--definablity 
and of a general recursive function, Emil Post and Alan Turing
%%%
\index{Post, Emil}\index{Turing, Alan}%%%
%%%
the notion of computability by a certain machine, now called the
Turing machine. All three notions can be shown to identify the
same class of functions, as these people have subsequently shown.
It is known as Church's Thesis that these are all the functions that
humans can compute, but for the purpose of this book it is
irrelevant whether it is correct. We shall define the
$\lambda$--calculus later in Chapter~\ref{kap3}, without
going into the details alluded to here, however. It is to be 
kept in mind that the Turing machine is a physical device. 
Hence, its computational capacities depend on the structure of 
the space--time continuum. This is not any more a speculation. 
Quantum computing exploits the different physical behaviour of 
quantum physics to do parallel computation. This radically 
changes the time complexity of problems (see
\cite{deutschetal:machines}). This asks us to be cautious 
not to attach too much significance to complexity 
results in connection with human behaviour since we do not know 
too well how the brain works. 
%
\vplatz
\exercise
Construct a Turing machine which computes the lexicographic
predecessor of a string, and which returns $\varepsilon$ for
input $\varepsilon$.
%%
\vplatz
\exercise
Construct a Turing machine which, given a list of strings 
(each string  separated from the next by a single blank),
moves the first string onto the end of the list.
%%
\vplatz
\exercise
Let $T$ be a Turing machine over $A$. Show how to write a Turing 
machine over $\{\mbox{\tt 0}, \mbox{\tt 1}\}$ which computes the
same partial function over $A$ under a coding that
assigns each letter of $A$ a unique block of fixed length.
%%
\vplatz
\exercise
\label{ex:oneside}
In many definitions of a Turing machine the tape is only
one sided. Its cells can be numbered by natural
numbers. This requires the introduction
of a special symbol {\tt \#} that marks the left end
of the tape, or of a predicate {\sf left-end}, which is
true each time the head is at the left end of the tape.
The transitions are different depending on whether the
machine is at the left end of the tape or not.
(There is an alternative, namely to stop the computation
once that the left end is reached, but this is not recommended.
Such a machine can compute only very uninteresting functions.)
Show that for a Turing machine with a one sided tape
there is a corresponding Turing machine in our sense computing
the same function, and that for each Turing machine in our sense
there is a one sided machine computing the same function.
%%
\vplatz
\exercise
Prove Lemma~\ref{lem:bij}. {\it Hint.} Show first that it is
enough to look at the case $|A| = 1$.
%%
\vplatz
\exercise
Show that $L \subseteq A^{\ast}$ is decidable iff 
$\chi_L : A^{\ast} \pf \{\mbox{\tt 0}, \mbox{\tt 1}\}$ 
is computable.
%%
%\vplatz
%\exercise
%Show that context sensitive languages are in {\bf PSPACE}.

 \chapter{Context Free Languages}
\thispagestyle{empty}
\label{kap2}
%
%
%
\section{Regular Languages}
\label{zweieins}
\label{kap2-1}
%
%
%
\index{language!regular}%%%
%%%%
Type 3 or regular grammars are the most simple grammars in the
Chomsky Hierarchy.  There are several characterizations of regular
languages: by means of finite state automata, by means of 
equations over strings, and by means of so--called regular expressions.
Before we begin, we shall develop a simple form for regular
grammars. First, all rules of the form $X \pf Y$ can be eliminated.
To this end, the new set of rules will be
%%
\begin{equation}
\begin{split}
R^{\heartsuit} := & \phantom{\mbox{}\cup\mbox{}} \{X \pf a Y :
X \vdash_G a Y\} \\ 
& \cup \{X \pf \vec{x} : X \vdash_G \vec{x}, \vec{x} \in A_{\varepsilon}\}
\end{split}
\end{equation}
%%
It is easy to show that the grammar with $R^{\heartsuit}$ in place
of $R$ generates the same strings. We shall introduce another
simplification.  For each $a \in A$ we introduce a new nonterminal
$U_a$. In place of the rules $X \pf a$ we now add the rules $X \pf a U_a$
as well as $U_a \pf \varepsilon$. Now every rule with the
exception of $U_a \pf \varepsilon$ is strictly expanding.
This grammar is therefore not regular if $\varepsilon \in 
L(G)$ but it generates the same language. However, the last kind 
of rules can be used only once, at the end of the derivation.
For the derivable strings all have the form $\vec{x} \conc Y$
with $\vec{x} \in A^{\ast}$ and $Y \in N$. If one applies a
rule $Y \pf \varepsilon$ then the nonterminal disappears
and the derivation is terminated. We call a regular grammar
\textbf{strictly binary}
%%%
\index{grammar!strictly binary}%%
%%%
if there are only rules of the form $X \pf a Y$ or $X \pf \varepsilon$.
%%
\begin{defn}
%%%
\index{automaton!finite state}%%
\index{automaton!deterministic finite state}%%
\index{transition function}%%
\index{state!initial}%%
\index{state!accepting}%%
%%%
Let $A$ be an alphabet. A (\textbf{partial}) \textbf{finite state
automaton} is a quintuple $\GA = \auf A, Q, i_0, F, \delta\zu$
such that $Q$ is a finite set, $i_0 \in Q$, $F \subseteq Q$ and 
$\delta \colon Q \times A \pf \wp(Q)$. $Q$ is the set of \textbf{states}, 
$i_0$ is called \textbf{the initial state}, $F$ the set of 
\textbf{accepting states} and $\delta$ the \textbf{transition function}. 
$\GA$ is called \textbf{deterministic} if $\delta(q,a)$ contains exactly 
one element for each $q \in Q$ and $a \in A$.
\end{defn}
%%
$\delta$ can be extended to sets of states and strings in the
following way ($S \subseteq Q$, $a \in A$).
%%
\begin{subequations}
\label{eq:sextend}
\begin{align}
\delta(S,\varepsilon) & := S \\
\delta(S,a) & := \bigcup \auf \delta(q,a) : q \in S\zu \\
\delta(S, \vec{x} \conc a) & := \delta(\delta(S, \vec{x}), a)
\end{align}
\end{subequations}
%%
With this defined, we can now define the accepted language.
%%
\index{$L(\GA)$}%%%
%%%
\begin{equation}
L(\GA) = \{\vec{x} : \delta(\{i_0\}, \vec{x}) \cap F \neq
\varnothing\}
\end{equation}
%%
$\GA$ is strictly partial if there is a state $q$
and some $a \in A$ such that $\delta(q,a) = \varnothing$.
An automaton can always be transformed into an equivalent
automaton which is not partial. Just add another state
$q_{\varnothing}$ and add to the transition function the
following transitions.
%%
\begin{equation}
\delta^+(q,a) := 
\begin{cases}
\delta(q,a) & \text{if $\delta(q,a) \neq \varnothing$ and 
	$q \neq q_{\varnothing}$,} \\
q_{\varnothing} & \text{if $\delta(q,a) = \varnothing$ or 
$q = q_{\varnothing}$.}
\end{cases}
\end{equation}
%%
Furthermore, $q_{\varnothing}$ shall {\it not\/} be an
accepting state. In the case of a deterministic automaton
we have $\delta(q, \vec{x}) = \{q'\}$ for some $q'$. In this
case we think of the transition function as yielding
states from states plus strings, that is, we now have
$\delta(q, \vec{x}) = q'$. Then the definition of the
language of an automaton $\GA$ can be refined as follows.
%%
\begin{equation}
L(\GA) = \{\vec{x} : \delta(i_0, \vec{x}) \in F\}
\end{equation}
%%
For every given automaton there is a deterministic automaton 
that accepts the same language. Put
%%
\index{$\GA^d$}%%
%%%
\begin{equation}
\GA^d := \auf A, \wp(Q), \{i_0\}, F^d, \delta\zu 
\end{equation}
%%
where $F^d := \{G \subseteq Q : G \cap F \neq \varnothing\}$
and $\delta$ is the transition function of $\GA$ extended to
sets of states.
%%
\begin{prop}
$\GA^d$ is deterministic and $L(\GA^d) = L(\GA)$.
Hence every language accepted by a finite state automaton is
a language accepted by a deterministic finite state
automaton.
\end{prop}
%%
The proof is straightforward and left as an exercise. Now we
shall first show that a regular language is a language
accepted by a finite state automaton. We may assume that
$G$ is (almost) strictly binary, as we have seen above. So, let
$G = \auf \mbox{\tt S}, N, A, R\zu$. We put $Q_G := N$, $i_0 := 
\mbox{\tt S}$, $F_G := \{X : X \pf \varepsilon \in R\}$ as well as
%%
\begin{equation}
\delta_G(X,a) :=  \{Y : X \pf a Y \in R\} 
\end{equation}
%%
Now put $\GA_G := \auf A, Q_G, i_0, F_G, \delta_G\zu$.
%%
\begin{lem}
\label{lem:labelchar}
For all $X, Y \in N$ and $\vec{x}$ we have
$Y \in \delta(X,\vec{x})$ iff $X \Pf^{\ast}_R \vec{x} \conc Y$.
\end{lem}
%%
\proofbeg
Induction over the length of $\vec{x}$. The case $|\vec{x}| =
\varepsilon$ is evident. Let $\vec{x} = a \in A$.
Then $Y \in \delta_G(X,a)$ by definition iff
$X \pf a Y \in R$, and from this we get $X \Pf_R^{\ast} a Y$.
Conversely, from $X \Pf^{\ast}_R a Y$ follows that $X \pf a Y \in
R$. For since the derivation uses only strictly expanding rules 
except for the last step, the derivation of $a Y$ from $X$ must 
be the application of a single rule. This finishes the case of 
length 1. Now let $\vec{x} = \vec{y} \conc a$. By definition of 
$\delta_G$ we have
%%
\begin{equation}
\delta_G(X, \vec{x}) = \delta_G(\delta_G(X,\vec{y}),a) 
\end{equation}
%%
Hence there is a $Z$ such that $Z \in \delta_G(X, \vec{y})$ and
$Y \in \delta_G(Z,a)$. By induction hypothesis this is equivalent
with $X \Pf_R^{\ast} \vec{y}\conc  Z$ and $Z \Pf^{\ast}_R a Y$. 
From this we get $X \Pf^{\ast}_R \vec{y}\conc  a \conc Y = \vec{x}\conc Y$.
Conversely, from $X \Pf^{\ast}_R \vec{x}\conc Y$ we get 
$X \Pf^{\ast}_R \vec{y} \conc Z$ and $Z \Pf^{\ast}_R a Y$ for some $Z$,
since $G$ is regular. Now, by induction hypothesis,
$Z \in \delta_G(X, \vec{y})$ and $Y \in \delta_G(Z,a)$, and
so $Y \in \delta_G(\vec{x},X)$.
\proofend
%%
\begin{prop}
$L(\GA_G) = L(G)$.
\end{prop}
%%
\proofbeg
It is easy to see that
$L(G) = \{\vec{x} : G \vdash \vec{x}\conc Y, Y \pf \varepsilon \in R\}$.
By Lemma~\ref{lem:labelchar} $\vec{x} \conc Y \in L(G)$ iff
$S \Pf^{\ast}_R \vec{x} \conc Y$. The latter is equivalent with
$Y \in \delta_G(\mbox{\tt S}, \vec{x})$. And this is nothing but
$\vec{x} \in L(\GA_G)$. Hence $L(G) = L(\GA_G)$.
\proofend

Given a finite state automaton $\GA = \auf A, Q, i_0, F, \delta\zu$ 
put $N_{\GA} := Q$, $S_{\GA} := i_0$. $R_{\GA}$ consists of all 
rules of the form $X \pf a Y$ where $Y \in \delta(X,a)$ as well 
as all rules of the form $X \pf \varepsilon$ for $X \in F$. Finally, 
$G_{\GA} := \auf S_{\GA}, N_{\GA}, A, R_{\GA}\zu$. 
$G_{\GA}$ is strictly binary and $\GA_{G_{\GA}} = \GA$. 
Therefore we have $L(G_{\GA}) = L(\GA)$.
%%
\begin{thm}
The regular languages are exactly those languages that are
accepted by some deterministic finite state automaton.
\proofend
\end{thm}
%%
\newcommand{\nll}{0}
%%
Now we shall turn to a further characterization of regular
languages.
%%%
\index{term!regular}%%
%%%
A \textbf{regular term over} $A$ is a term which is composed from $A$
with the help of the symbols $\nll$ (0--ary), $\varepsilon$
(0--ary), $\cdot$ (binary), $\cup$ (binary) and $^{\ast}$ (unary).
A regular term defines a language over $A$ as follows.
%%
\index{$\cdot$, $\cup$, $^{\ast}$}%%%
\begin{subequations}
\begin{align}
L(\nll)        & := \varnothing \\
L(\varepsilon)  & := \{\varepsilon\} \\
L(a)         & := \{a\} \\
L(R \cdot S) & := L(R) \cdot L(S) \\
L(R \cup S)  & := L(R) \cup L(S) \\
L(R^{\ast})  & := L(R)^{\ast}
\end{align}
\end{subequations}
%%
(Commonly, one writes $R$ in place of $L(R)$, a usage that we will
follow in the sequel to this section.) Also, 
%%%
\index{$^+$}%%%
%%%
$R^+ := R^{\ast} \cdot R$ is an often used abbreviation.
Languages which are defined by a regular term can also be viewed
as solutions of some very simple systems of equations. We introduce
variables (say $X$, $Y$ and $Z$) which are variables for subsets
of $A^{\ast}$ and we write down equations for the terms over
these variables and the symbols $\nll$, $\varepsilon$,
$a$ ($a \in A$), $\cdot$, $\cup$ and $^{\ast}$. An example is the
equation $X = \mbox{\tt b} \cup \mbox{\tt a} X$, whose solution is
$X = \mbox{\tt a}^{\ast}\mbox{\tt b}$.
%%
\begin{lem}
Assume $R \neq 0$ and $\varepsilon \not\in L(R)$.
Then $R^{\ast}$ is the unique solution of
$X = \varepsilon \cup R \cdot X$.
\end{lem}
%%
\proofbeg
The proof is by
induction over the length of $\vec{x}$. $\vec{x} \in X$ means by
definition that $\vec{x} \in \varepsilon \cup R \cdot X$.
If $\vec{x} = \varepsilon$ then $\vec{x} \in R^{\ast}$.
Hence let $\vec{x} \neq \varepsilon$; then $\vec{x} \in R
\cdot X$ and so it is of the form $\vec{u}_0 \conc \vec{x}_0$
where $\vec{u}_0 \in R$ and $\vec{x}_0 \in X$.  Since
$\vec{u}_0 \neq \varepsilon$, $\vec{x}_0$ has smaller length
than $\vec{x}$. By induction hypothesis we therefore have
$\vec{x}_0 \in R^{\ast}$. Hence $\vec{x} \in R^{\ast}$. The
other direction is as easy.
\proofend
%%
\begin{lem}
\label{lem:regsolution}
Let $C, D$ be regular terms, $D \neq 0$ and $\varepsilon \not\in L(D)$.
The equation 
%%
\begin{equation}
X = C \cup D \cdot X
\end{equation}
%%%
has exactly one solution, namely $X = D^{\ast} \cdot C$.
\proofend
\end{lem}
%%
We shall now show that regular languages can be seen as solutions
of systems of equations. A general system of string equations is a 
set of equations of the form $X_j = Q \cup \bigcup_{i < m} T^i$ where 
$Q$ is a regular term and the $T^i$ have the form $R \cdot X_k$ where
$R$ is a regular term. Here is an example.
%%
\begin{equation}
\begin{array}{lr@{\; \cup\; }r}
X_0 & = \mbox{\tt a}^{\ast} & \mbox{\tt c} \cdot
    \mbox{\tt a} \cdot \mbox{\tt b} \cdot X_1   \\
X_1 & = \mbox{\tt c}        & \mbox{\tt c}
    \cdot \mbox{\tt b}^3 \cdot X_0
\end{array}
\end{equation}
%%
Notice that like in other systems of equations a variable
need not occur to the right in every equation. Moreover,
a system of equations contains any given variable only once
on the left. The system is called \textbf{proper} if for all
$i$ and $j$ we have $\varepsilon \not\in L(T^i_j)$. We shall call
a system of equations \textbf{simple} if it is proper and $Q$ as well
as the $T^i_j$ consist only of terms made from elements of
$A$ using $\varepsilon$ and $\cup$.
%%%
\index{system of equations!proper}%%
\index{system of equations!simple}%%
%%%
The system displayed above is proper but not simple.

Let now $\auf \mbox{\tt S}, N, A, R\zu$ be a strictly binary
regular grammar. Introduce for each nonterminal
$X$ a variable $Q_X$. This variable $Q_X$ shall stand for the
set of all strings which can be generated from $X$ in this
grammar, that is, all strings $\vec{x}$ for which
$X \Pf^{\ast}_R \vec{x}$. This latter set we denote by
$[X]$. We claim that the $Q_X$ so interpreted satisfy the
following system of equations.
%%
\begin{equation}
\begin{split}
Q_Y = & \phantom{\mbox{}\cup\mbox{}} 
	\bigcup \{\varepsilon : Y \pf \varepsilon \in R \} \\
    & \cup \bigcup \{a \cdot Q_X : Y \pf aX \in R\}
\end{split}
\end{equation}
%%
This system of equations is simple. We show $Q_Y = [Y]$ for all $Y
\in N$. The proof is by induction over the length of the string.
To begin, we show that $Q_Y \subseteq [Y]$. For let $\vec{y} \in
Q_Y$. Then either $\vec{y} = \varepsilon$ and $Y \pf \varepsilon
\in R$ or we have $\vec{y} = a \conc \vec{x}$ with $\vec{x} \in
Q_X$ and $Y \pf a \conc \vec{x} \in R$. In the first case $Y \pf
\varepsilon \in R$, whence $\varepsilon \in [Y]$. In the second
case $|\vec{x}| < |\vec{y}|$ and so by induction hypothesis
$\vec{x} \in [X]$, hence $X \Pf^{\ast}_R \vec{x}$. Then we have 
$Y \Pf^{\ast}_R a \conc \vec{x} = \vec{y}$, from which $\vec{y} 
\in [Y]$. This shows the first inclusion. Now we show that 
$[Y] \subseteq Q_Y$. To this end let $Y \Pf^{\ast}_R \vec{y}$. 
Then either $\vec{y} = \varepsilon$ and so $Y \pf \varepsilon 
\in R$ or $\vec{y} = a \conc \vec{x}$ for some $\vec{x}$.  In 
the first case $\vec{y} \in Q_Y$, by definition. In the second 
case there must be an $X$ such that $Y \pf a X \in R$ and 
$X \Pf^{\ast}_R \vec{x}$. Then $|\vec{x}| < |\vec{y}|$ and 
therefore by induction hypothesis $\vec{x} \in Q_X$. Finally, 
by definition of $Q_Y$, $\vec{y} \in Q_Y$, which had to be shown.

So, a regular language is the solution of a simple system of
equations. Conversely, every simple system of equations can be
rewritten into a regular grammar which generates the solution
of this system. Finally, it remains to be shown that regular terms
describe nothing but regular languages. What we shall establish is
more general and derives the desired conclusion. We shall show
that every proper system of equations which has as many
equations as it has variables has as its solution for each
variable a regular language. To this end, let such a system
$X_j = \bigcup_{i < m_j} T_j^i$ be given. We begin by eliminating
$X_0$ from the system of equations. We distinguish two cases.
(1) $X_0$ appears in the equation $X_0 = \bigcup_{i < m_0}
T^i_j$ only to the left. This equation is fixed, and called the
\textbf{pivot equation for} $X_0$. Then we can replace $X_0$
in the other equations by $\bigcup_{i < m_0} T^i_j$. 
(2) The equation is of the form $X_0 = C \cup D \cdot X_0$, 
$C$ a regular term, which does not contain $X_0$, $D$
free of variables and $\varepsilon \not\in L(D)$. Then
$X_0 = D^{\ast} \cdot C$ by Lemma~\ref{lem:regsolution}.
Now $X_0$ does not occur and we can replace $X_0$ in the other
equations as in (1). The system of equations
that we get is not simple, even if it was simple at the beginning.
We can proceed in this fashion and eliminate step by step the
variables from the right hand side (and putting aside the
corresponding pivot equations) until we reach the last equation.
The solution for $X_{n -1}$ does not contain any variables at all
and is a regular term. The solution can be inserted into the
other equations, and then we continue with $X_{n-2}$, then
with $X_{n-3}$, and so on.
%%
As an example, we take the following system of equations.
%%
$$\begin{array}{ll@{\quad = \quad}rrr}
\mbox{(I)} & X_0 & \mbox{\tt a} \cup\; \mbox{\tt a}\cdot X_0
    & \cup\; \mbox{\tt b} \cdot X_1
    & \cup\; \mbox{\tt c} \cdot X_2 \\
    & X_1 &   \mbox{\tt c} \cdot X_0 &
    & \cup \; \mbox{\tt a} \cdot X_2 \\
    & X_2 & \mbox{\tt b} \cup\; \mbox{\tt a} \cdot X_0
    & \cup \; \mbox{\tt b} \cdot X_1
    & \\
    \multicolumn{5}{c}{} \\
\mbox{(II)} & X_0 & \mbox{\tt a}^+ & \cup \; \mbox{\tt a}^{\ast}
    \mbox{\tt b} \cdot X_1
    & \cup \; \mbox{\tt a}^{\ast}\mbox{\tt c} \cdot X_2 \\
     & X_1 & \mbox{\tt ca}^+  & \cup \; \mbox{\tt ca}^{\ast}%
    \mbox{\tt b} \cdot X_1
    & \cup \; (\mbox{\tt ca}^{\ast}\mbox{\tt c} \cup %
    \mbox{\tt a}) \cdot X_2 \\
     & X_2 & \mbox{\tt b} \cup \mbox{\tt aa}^+ & \cup\;
     (\mbox{\tt a}^+ \mbox{\tt b} \cup \mbox{\tt b})
    \cdot X_1 & \cup \; \mbox{\tt a}^{\ast} \mbox{\tt c}
    \cdot X_2 \\
    \multicolumn{5}{c}{} \\
\mbox{(III)} & X_1 & 
    \multicolumn{3}{r}{%
	(\mbox{\tt ca}^{\ast}\mbox{\tt b})^{\ast} \mbox{\tt ca}^+ 
    \cup (\mbox{\tt ca}^{\ast}\mbox{\tt b})^{\ast}%
    (\mbox{\tt ca}^{\ast}\mbox{\tt c} \cup \mbox{\tt a})
    \cdot X_2} \\
       & X_2 &  
    \multicolumn{3}{r}{%
       (\mbox{\tt b} \cup \mbox{\tt aa}^+) 
    \cup [\mbox{\tt a}^{\ast}\mbox{\tt b}
    (\mbox{\tt ca}^{\ast}\mbox{\tt b})^{\ast}
    (\mbox{\tt ca}^{\ast}\mbox{\tt c} \cup \mbox{\tt a}) %
    \cup \mbox{\tt a}^{\ast}\mbox{\tt c}] \cdot X_2} \\
    \multicolumn{5}{c}{} \\
\mbox{(IV)} & X_2 & \multicolumn{3}{l}{%
    [\mbox{\tt a}^{\ast}\mbox{\tt b} 
    (\mbox{\tt ca}^{\ast}\mbox{\tt b})^{\ast}
    (\mbox{\tt ca}^{\ast}\mbox{\tt c} \cup \mbox{\tt a}) %
    \cup \mbox{\tt a}^{\ast}\mbox{\tt c}]^{\ast} %
    (\mbox{\tt b} \cup \mbox{\tt aa}^+)}
    \end{array}$$
%%
Now that $X_2$ is known, $X_1$ can be determined by inserting the 
regular term for $X_2$, and, finally, $X_0$ is obtained by inserting 
the values for $X_2$ and $X_1$.
%%
\nocite{kleene:regular}
\index{Kleene, Stephen C.}%%
\begin{thm}[Kleene]
\label{thm:reg}
Let $L$ be a language over $A$. Then the following are equivalent:
%%
\begin{dingautolist}{192}
\item
$L$ is regular.
\item
$L = L(\GA)$ for a finite, deterministic automaton
$\GA$ over $A$.
\item
$L = L(R)$ for some regular term $R$ over $A$.
\item
$L$ is the solution for $X_0$ of a simple system of equations
over $A$ with variables $X_i$, $i < m$.
\end{dingautolist}
%%
Further, there exist algorithms which (i) for a given automaton
$\GA$ compute a regular term $R$ such that $L(\GA) = L(R)$; (ii)
for a given regular term $R$ compute a simple system of equations
$\Sigma$ over $\vec{X}$ whose solution for a given variable $X_0$
is exactly $L(R)$; and (iii) which for a given simple system
of equations $\Sigma$ over $\{X_i : i < m\}$ compute an automaton 
$\GA$ such that $\vec{X}$ is its set of states and the solution for
$X_i$ is exactly the set of strings which send the automaton from
state $X_0$ into $X_i$. \proofend
\end{thm}
%%
This is the most important theorem for regular languages.
We shall derive a few consequences. Notice we can turn a 
finite state automaton $\GA$ into a Turing machine $T$ 
accepting the same language in linear time and no additional 
space.  Therefore, the recognition problem for regular languages 
is in $\textbf{DTIME}(n)$ and in $\textbf{DSPACE}(n)$. This 
also applies to the parsing problem, as is easily seen.
%%
\begin{cor}
The recognition and the parsing problem are in
$\textbf{DTIME}(n)$ and $\textbf{DSPACE}(n)$.
\end{cor}
%%
\begin{cor}
\label{cor:con} The set of regular languages over $A$ is closed
under intersection and relative complement. Further, for given
regular terms $R$ and $S$ one can determine terms $U$ and $V$ such
that $L(U) = A^{\ast} - L(R)$ and $L(V) = L(R) \cap L(S)$.
\end{cor}
%%
\proofbeg
It is enough to do this construction for automata.  Using
Theorem~\ref{thm:reg} it follows that we can do it also for
the corresponding regular terms. Let $\GA = \auf A, Q, i_0, F,
\delta\zu$. Without loss of generality we may assume that
$\GA$ is deterministic. Then let $\GA^- := \auf A, Q, i_0, Q - F,
\delta\zu$. We then have $L(\GA^-) = A^{\ast} - L(\GA)$.
This shows that for given $\GA$ we can construct an automaton
which accepts the complement of $L(\GA)$.
Now let $\GA' = \auf A, Q', i_0', F', \delta'\zu$.
Put 
%%
\begin{equation}
\GA  \times \GA' := \auf A, Q\times Q', \auf i_0, i_0'\zu,
F \times F', \delta \times \delta'\zu
\end{equation}
%%%
where
%%%
\begin{equation}
(\delta \times \delta')(\auf q,q'\zu, a) := \{\auf
r,r'\zu : r \in \delta(q,a), r' \in \delta'(q',a)\} 
\end{equation}
%%
It is easy to show that $L(\GA \times \GA') =
L(\GA) \cap L(\GA')$.
\proofend
%%

\noindent
The proof of the next theorem is an exercise.
%%
\begin{thm}
\label{thm:abschluss}
Let $L$ and $M$ be regular languages. Then so are $L/M$ and
$M\backslash L$. Moreover, $L^T$, $L^P := L/A^{\ast}$
as well as $L^S := A^{\ast}\backslash L$ are regular.
\end{thm}
%%
Furthermore, the following important consequence can be
established.
%%
\begin{thm}
Let $\GA$ and $\GB$ be finite state automata.
Then it is decidable whether $L(\GA) = L(\GB)$.
\end{thm}
%%
\proofbeg
Let $\GA$ and $\GB$ be given. By Theorem~\ref{thm:reg}
we can compute a regular term $R$ with $L(R) = L(\GA)$
as well as a regular term $S$ with $L(S) = L(\GB)$.
Then $L(\GA) = L(\GB)$ iff $L(R) = L(S)$
iff $(L(R) - L(S)) \cup (L(S) - L(R))
= \varnothing$. By Corollary~\ref{cor:con} we can compute
a regular term $U$ such that 
%%
\begin{equation}
L(U) = (L(R) - L(S)) \cup (L(S) - L(R))
\end{equation}
%%%
Hence $L(\GA) = L(\GB)$ iff $L(U) = \varnothing$. This is 
decidable by Lemma~\ref{lem:leer}.
\proofend
%%
\begin{lem}
\label{lem:leer}
The problem `$L(R) = \varnothing$', where $R$ is a regular 
term, is decidable.
\end{lem}
%%
\proofbeg
By induction on $R$. If $R = \varepsilon$
or $R = a$ then $L(R) \neq  \varnothing$. If $R = \nll$
then by definition $L(R) = \varnothing$. Now assume that the
problems `$L(R) = \varnothing$' and `$L(S) = \varnothing$'
are decidable. Notice that (a) $L(R \cup S) = \varnothing$ iff
$L(R) = \varnothing$ and $L(S) = \varnothing$, 
(b) $L(R \cdot S) = \varnothing$ iff $L(R) = \varnothing$ or 
$L(S) = \varnothing$ and (c) $L(R^{\ast}) = \varnothing$ iff
$L(R) = \varnothing$. All three problems are decidable.
\proofend
%%

We conclude with the following theorem, which we have used
already in Section~\ref{kap1}.\ref{kap1-5}.
%%
\begin{thm}
\label{thm:cfintersekt} Let $L$ be context free and $R$
regular. Then $L \cap R$ is context free.
\end{thm}
%%
\proofbeg
Let be $G = \auf S, N, A, R\zu$ be a CFG
with $L(G) = L$ and $\GA = \auf n, 0, F, \delta\zu$
a deterministic automaton consisting of $n$  states such that
$L(\GA) = R$. We may assume that rules of $G$ are of the form
$X \pf a$ or $X \pf \vec{Y}$. We define new nonterminals, which
are all of the form $^i X^j$, where $i, j < n$ and $X \in N$. The
interpretation is as follows. $X$ stands for the set of all
strings $\vec{\alpha} \in A^{\ast}$ such that $X \vdash_G
\vec{\alpha}$. $^i X^j$ stands for the set of all
$\vec{\alpha}$ such that $X \vdash_G \vec{\alpha}$ and
$\delta(i,\vec{\alpha}) = j$. We have a set of start symbols,
consisting of all $^0 S^{j}$ with $j \in F$. As we already know,
this does not increase the generative power. A rule 
$X \pf Y_0 Y_1 \dotsb Y_{k-1}$ is now replaced by the set of all
rules of the form
%%
\begin{equation}
^i X^j \pf ^i Y_0^{i_0}\conc ^{i_0}Y_1^{i_1} \conc \dotso
\conc ^{i_{k-2}}Y_{k-1}^j
\end{equation}
%%
Finally, we take all rules of the form $^i X^j \pf a$,
$\delta(i,a) = j$. This defines the grammar $G_r$. We shall show:
$\vdash_{G^r} \vec{x}$ iff $\vdash_G \vec{x}$ and
$\vec{x} \in L(\GA)$. ($\Pf$) Let $\GB$ be a $G^r$--tree with
associated string $\vec{x}$. The map ${^i X}^j \mapsto X$ turns
$\GB$ into a $G$--tree. Hence $\vec{x} \in L(G)$. Further, it
is easily shown that $\delta(0,x_0x_1\dotsb x_j) = k_j$, where
$^{k_{j-1}} X^{k_j}$ is the node dominating $x_j$.
Also, if $|\vec{x}| = n$, then $^0 S^{k_n}$ is the top node
and by construction $k_n \in F$. Hence $\delta(\vec{x}, 0)
\in F$ and so $\vec{x} \in L(\GA)$.
($\Leftarrow$) Let $\vec{x} \in L(G)$ and $\vec{x} \in L(\GA)$.
We shall show that $\vec{x} \in L(G^r)$. We take a $G$--tree
$\GB$ for $\vec{x}$.  We shall now prove that one can replace the
$G$--nonterminals in $\GB$ in such a way by $G^r$--nonterminals
that we get a $G_r$--tree. The proof is by induction on the height
of a node. We begin with nodes of height 1. Let
$\vec{x} = \prod_{i < n} x_i$; and let $X_i$ be the nonterminal
above $x_i$. Further let $\delta(0,\prod_{i < j} x_i) =
j_i$. Then $p_0 = 0$ and $p_n \in F$. We replace $X_i$ by
$^{p_i} X^{p_{i+1}}$. We say that two nodes $x$ and
$y$ \textbf{connect} if they are adjacent and for the labels
${^i X^j}$ of $x$ and $^k Y^{\ell}$ of $y$ we have $j = k$.
Let $x$ be a node of height $n+1$ with label $X$ and let $x$ be
mother of the nodes with labels $Y_0 Y_1 \dotsb Y_{n-1}$ in $G$.
We assume that below $x$ all nodes carry labels from $G^r$ in such
a way that adjacent nodes connect. Then there exists  a rule
in $G_r$ such that $X$ can be labelled with superscripts,
the left hand superscript of $Y_0$ to its left and the right hand
superscript of $Y_{n-1}$ to its right. All adjacent nodes of height
$n+1$ connect, as is easily seen. Further, the leftmost node
carries the left superscript 0, the rightmost node carries a
right superscript $p_n$, which is an accepting state. Eventually,
the root has superscripts as well. It carries the label
$^0 S^{p_n}$, and so we have a $G_r$--tree.
\proofend
%%
\vplatz
\exercise
Prove Theorem~\ref{thm:abschluss}.
%%
\vplatz
\exercise
Show that a language is regular iff it can be
generated by a grammar with rules of the form $X \pf Y$,
$X \pf Ya$, $X \pf a$ and $X \pf \varepsilon$. Such a
grammar is called
%%%
\index{grammar!left regular}%%
\index{grammar!right regular}%%
%%%
\textbf{left regular}, in contrast to the grammars of
Type 3, which we also call \textbf{right regular}.
Show also that it is allowed to add rules of the form
$X \pf \vec{x}$ and $X \pf Y \vec{x}$.
%%
\vplatz
\exercise
Show that there is a grammar with rules of the form
$X \pf a$, $X \pf aY$ and $X \pf Ya$ which generates a
nonregular language. This means that a Type 3 grammar may 
contain (in general) only left regular rules or only right 
regular rules, but not both.
%%
\vplatz
\exercise
Show that if $L$ and $M$ are regular, then so are
$L/M$ and $M\backslash L$.
%%
\vplatz
\exercise
Let $L$ be a language over $A$. Define an equivalence relation
$\sim_S$ over $A^{\ast}$ as follows.  $\vec{x} \sim_S \vec{y}$ iff
for all $\vec{z} \in A^{\ast}$ we have
$\vec{x} \conc \vec{z} \in L \Dpf \vec{y} \conc
\vec{z} \in L$. $L$ is said to have \textbf{finite index}
%%%%
\index{language!finite index}%%
%%%%
if there are only finitely many equivalence classes
with respect to $\sim_S$. Show that $L$ is regular iff
it has finite index.
%%
\vplatz
\exercise
\label{ue:index}
Show that the language $\{\mbox{\tt a}^n\mbox{\tt b}^n :
n \in \omega\}$ does not have finite index. Hence it is
not regular.
%%
\vplatz
\exercise
Show that the intersection of a context sensitive language with
a regular language is again context sensitive.
%%%
\vplatz
\exercise
Show that $L$ is regular iff it is accepted by a read
only 1--tape Turing machine.

 \section{Normal Forms}
\label{zweizwei}
\label{kap2-2}
%
%
%
\index{language!context free}%%%
In the remaining sections of this chapter we shall deal with
CFGs and their languages. In view of the extensive
literature about CFLs it is only possible to
present an overview. In this section we shall deal in particular
with normal forms. There are many normal forms for CFGs,
each having a different purpose. However, notice that
the transformation of a grammar into a normal form necessarily
destroys some of its properties. So, to say that a grammar can
be transformed into another is meaningless unless we specify
exactly what properties remain constant under this transformation.
If, for example, we are only interested in the language generated
then we can transform any CFG into Chomsky Normal Form. However, 
if we want to maintain the constituent
structures, then only the so--called standard form is possible.
A good exposition of this problem area can be found in
\cite{miller:capacity}.

Before we deal with reductions of grammars we shall study the
relationship between derivations, trees and sets of rules.
To be on the safe side, we shall assume that every symbol
occurs at least once in a tree, that is, that the grammar is
slender in the sense of Definition~\ref{def:schlank}. From the 
considerations of Section~\ref{kap1}.\ref{einsvier} we conclude
that for any two CFGs $G = \auf \mbox{\tt S}, N, A, R\zu$
and $G' = \auf \mbox{\tt S}', N', A, R'\zu$ $L_B(G) = L_B(G')$ iff 
$\der(G) = \der(G')$. Likewise we
see that for all $X \in N \cup N'$ $\der(G,X) =
\der(H,X)$ iff $R = R'$. Now let
$G = \auf \mbox{\tt S}, N, A, R\zu$ and  a sequence
$\Gamma = \auf \vec{\alpha}_i : i < n\zu$ be given.
In order to test whether $\Gamma$ is a $G$--string sequence
we have to check for each $i < n-1$ whether $\vec{\alpha}_{i+1}$
can be derived from $\vec{\alpha}_i$ with a single application
of a rule. To this end we have to choose an $\vec{\alpha}_i$
and apply a rule and check whether the string obtained equals
$\vec{\alpha}_{i+1}$. Checking this needs $a_G \times |\vec{\alpha}_i|$
steps, where $a_G$ is a constant which depends only on $G$.
Hence for the whole derivation we need $\sum_{i < n} a_G |\vec{\alpha}_i|$
steps. This can be estimated from above by $a_G \times n \times
|\vec{\alpha}_{n-1}|$ and if $G$ is strictly expanding also by
$a_G \times |\vec{\alpha}_{n-1}|^2$. It can be shown that there are
grammars for which this is the best possible bound. In order to
check for an ordered labelled tree whether it can be generated
by $\gamma G$ we need less time. We only need to check for each node
whether the local tree at $x$ conforms to some rule of
$G$. This can be done in constant time. The time therefore only 
linearly depends on the size of the tree.

There is a tight connection between derivations and trees.
To begin, a derivation has a unique tree corresponding to it.
Simply translate the derivation in $G$ into a derivation
in $\gamma G$. Conversely, however, there may exist many
derivations for the same tree. Their number can be very large.
However, we can obtain them systematically in the following way.
Let $\GB$ be an (exhaustively ordered, labelled) tree.
%%%
\index{linearisation}%%
%%%
Call $\lhd \subseteq B^2$ a \textbf{linearisation} if $\lhd$ is an
irreflexive, linear ordering and from $x > y$ follows $x \lhd y$.
Given a linearisation, a derivation is found as follows. We begin
with the element which is smallest with respect to $\lhd$. This
is, as is easy to see, the root. The root carries the label {\tt S}.
Inductively, we shall construct cuts $\vec{\alpha}_i$ through
$\GB$ such that the sequence $\auf \vec{\alpha}_i : i < n\zu$ is a
derivation of the associated string. (Actually, the derivation is
somewhat more complex than the string sequence, but we shall not
complicate matters beyond need here.) The beginning is clear: we put 
$\vec{\alpha}_0 := \mbox{\tt S}$. Now assume that $\vec{\alpha}_i$ has
been established, and that it is not identical to the associated
string of $\GB$. Then there exists a node $y$ with nonterminal
label in $\vec{\alpha}_i$. (There is a unique correspondence
between nodes of the cut and segments of the strings
$\vec{\alpha}_i$.) We take the smallest such node with respect to
$\lhd$. Let its label be $Y$. Since we have a $G$--tree, the local
tree with root $y$ corresponds to a rule of the form $Y \pf
\vec{\beta}$ for some $\vec{\beta}$. In $\vec{\alpha}_i$ $y$
defines a unique instance of that rule. Then $\vec{\alpha}_{i+1}$
is the result of replacing that occurrence of $Y$ by
$\vec{\beta}$. The new string is then the result of applying a
rule of $G$, as desired.

It is also possible to determine for each derivation a
linearisation of the tree which yields that derivation in the
described manner. However, there can be several linearisations
that yield the same derivation.
%%
\begin{thm}
%%%
\index{$\der(\lhd)$}%%%%
%%%
Let $G$ be a CFG and $\GB \in L_B(G)$. Further, let
$\lhd$ be a linearisation of $\GB$. Then $\lhd$ determines
a $G$--derivation $\der(\lhd)$ of the string
which is associated to $\GB$. If $\shd$ is another linearisation
of $\GB$ then $\der(\shd) = \der(\lhd)$
is the case iff $\shd$ and $\lhd$ coincide on the
interior nodes of $\GB$.
\proofend
\end{thm}
%%
Linearisations can also be considered as top down search strategies
on a tree. We shall present examples. The first is a particular
case of the so--called
%%%
\index{search!depth--first}%%
\index{linearisation!leftmost}%%
%%%
\textbf{depth--first} search and the linearisation shall be called
\textbf{leftmost linearisation}.  It is as follows.
$x \lhd y$  iff $x > y$ or $x \sqsubset y$.
For every tree there is exactly one leftmost linearisation.
We shall denote the fact that there is a leftmost
derivation of $\vec{\alpha}$ from $X$ by $X \vdash_G^{\ell}
\vec{\alpha}$. We can generalize the situation as follows.
Let $\shd$ be a linear ordering uniformly defined on the
leaves of local  subtrees. That is to say, if $\GB$ and $\GC$
are isomorphic local trees (that is, if they correspond to the
same rule $\rho$) then $\shd$ orders the leaves $\GB$ linearly
in the same way as $\lhd$ orders the leaves of $\GC$ (modulo
the unique (!) isomorphism). In the case of the leftmost
linearisation the ordering is the one given by $\sqsubset$.
Now a minute's reflection reveals that every linearisation of
the local subtrees of a tree induces a linearisation of the entire
tree but not conversely (there are orderings which do not
proceed in this way, as we shall see shortly).
$X \vdash_G^{\shd} \vec{\alpha}$ denotes the fact that there
is a derivation of $\vec{\alpha}$ from $X$ determined by
$\shd$. Now call $\pi$
%%%
\index{priorisation}%%
%%%
a \mbox{priorisation for} $G = \auf \mbox{\tt S}, N, A, R\zu$ if
$\pi$ defines a linearisation on the local tree $\GH_{\rho}$,
for every $\rho \in R$. Since the root is always the first
element in a linearisation, we only need to order the daughters
of the root node, that is, the leaves. Let this ordering be
$\shd$. We write $X \vdash_G^{\pi} \vec{\alpha}$ if
$X \vdash_G^{\shd} \vec{\alpha}$ for the linearisation $\shd$
defined by $\pi$.
%%
\begin{prop}
Let $\pi$ be a priorisation. Then $X \vdash_G^{\pi} \vec{x}$
iff $X \vdash_G \vec{x}$.
\end{prop}
%%%
\index{search!breadth--first}%%
%%%
A different strategy is the {\it breadth--first search}. This
search goes through the tree in increasing depth. Let $S_n$ be the
set of all nodes $x$ with $d(x) = n$. For each $n$, $S_n$ shall be
ordered linearly by $\sqsubset$. The \textbf{breadth--first search}
is a linearisation $\Delta$, which is defined as follows. (a) If
$d(x) = d(y)$ then $x\; \Delta\; y$ iff $x\sqsubset y$,
and (b) if $d(x) < d(y)$ then $x\; \Delta\; y$. The difference
between these search strategies, depth--first and breadth--first,
can be made very clear with tree domains (see
Section~\ref{kap1}.\ref{kap1-4}). The depth--first search traverses 
the tree domain in the lexicographical order, the breadth--first 
search in the numerical order. Let the following tree domain be given.
%%
\begin{center}
\begin{picture}(8,11)
\put(1,1.5){\makebox(0,0){00}}
\put(3,1.5){\makebox(0,0){10}}
\put(5,1.5){\makebox(0,0){11}}
\put(7,1.5){\makebox(0,0){20}}
\put(1,5.5){\makebox(0,0){0}}
\put(4,5.5){\makebox(0,0){1}}
\put(7,5.5){\makebox(0,0){2}}
\put(4,9.5){\makebox(0,0){$\varepsilon$}}
\put(1,2){\line(0,1){3}}
\put(3,2){\line(1,3){1}}
\put(5,2){\line(-1,3){1}}
\put(7,2){\line(0,1){3}}
\put(1,6){\line(1,1){3}}
\put(4,6){\line(0,1){3}}
\put(7,6){\line(-1,1){3}}
\end{picture}
\end{center}
%%
The depth--first linearisation is
%%
\begin{equation}
\varepsilon, 0, 00, 1, 10, 11, 2, 20
\end{equation}
%%
The breadth--first linearisation, however, is
%%
\begin{equation}
\varepsilon, 0, 1, 2, 00, 10, 11, 20
\end{equation}
%%
Notice that with these linearisations the tree domain
$\omega^{\ast}$ cannot be enumerated. Namely, the
depth--first linearisation begins as follows. 
%%
\begin{equation}
\varepsilon, 0, 00, 000, 0000, \dotsc
\end{equation}
%%
So we never reach 1. The breadth--first linearisation goes like this.
%%
\begin{equation}
\varepsilon, 0, 1, 2, 3, \dotsc
\end{equation}
%%
So, we never reach 00.
On the other hand, $\omega^{\ast}$ is countable, so we do
have a linearisation, but it is more complicated than the
given ones.

The first reduction of grammars we look at is the elimination
of superfluous symbols and rules. Let 
$G = \auf \mbox{\tt S}, A, N, R\zu$
be a CFG. Call $X \in N$ \textbf{reachable}
%%%
\index{nonterminal!reachable}%%
\index{nonterminal!completable}%%
%%%
if $G \vdash \vec{\alpha} \conc X \conc \vec{\beta}$ for some
$\vec{\alpha}$ and $\vec{\beta}$. $X$ is called \textbf{completable}
if there is an $\vec{x}$ such that $X \Pf^{\ast}_R \vec{x}$.
%%
\begin{equation}
\begin{array}{l@{\quad \pf \quad}l@{\qquad}l@{\quad \pf \quad}l}
\mbox{\tt S} & \mbox{\tt AB} & \mbox{\tt A} & \mbox{\tt CB} \\
\mbox{\tt B} & \mbox{\tt AB} & \mbox{\tt A} & \mbox{\tt x}  \\
\mbox{\tt D} & \mbox{\tt Ay} & \mbox{\tt C} & \mbox{\tt y}
\end{array}
\end{equation}
%%
In the given grammar {\tt A}, {\tt C} and {\tt D}
are completable, and {\tt S}, {\tt A}, {\tt B} and {\tt C}
are reachable. Since {\tt S}, the start symbol, is not completable,
no symbol is both reachable and completable. The grammar generates
no terminal strings.

Let $N'$ be the set of symbols which are both reachable and
completable. If $\mbox{\tt S} \not\in N'$ then $L(G) = \varnothing$. 
In this case we put $N' := \{\mbox{\tt S}\}$ and $R' := \varnothing$. 
Otherwise, let $R'$ be the restriction of $R$ to the symbols from 
$A \cup N'$. This defines $G' = \auf \mbox{\tt S}, N', A, R'\zu$. 
It may be that throwing away rules may make some nonterminals 
unreachable or uncompletable. Therefore, this process must be 
repeated until $G' = G$, in which case every element is both 
reachable and completable. Call the resulting grammar $G^s$. It 
is clear that $G \vdash \vec{\alpha}$ iff $G^s \vdash \vec{\alpha}$. 
Additionally, it can be shown that every derivation in $G$ is a 
derivation in $G^s$ and conversely.
%%
\begin{defn}
\label{def:schlank}
%%%
\index{grammar!slender}%%
%%%
A CFG is called \textbf{slender} if either
$L(G) = \varnothing$ and $G$ has no nonterminals except for
the start symbol and no rules; or $L(G) \neq \varnothing$ and
every nonterminal is both reachable and completable.
\end{defn}
%%
Two slender grammars have identical sets of derivations iff
their rule sets are identical.
%%
\begin{prop}
Let $G$ and $H$ be slender. Then $G = H$ iff
$\der(G) = \der(H)$.
\end{prop}
%%
\begin{prop}
For every CFG $G$ there is an effectively 
constructable slender CFG $G^s = \auf\mbox{\tt S}, %
N^s, A, R^s\zu$ such that $N^s \subseteq N$,
which has the same set of derivations as $G$. In this case it also
follows that $L_B(G^s) = L_B(G)$.
\proofend
\end{prop}
%%
Next we shall discuss the role of the nonterminals. Since these
symbols do not occur in $L(G)$, their name is irrelevant for the
purposes of $L(G)$. To make this precise we shall introduce the
notion of a rule simulation. Let $G$ and $G'$ be grammars with
sets of nonterminals $N$ and $N'$. Let $\sim\; \subseteq\; N
\times N'$ be a relation. This relation can be extended to a
relation $\approx \; \subseteq \; (N \cup A)^{\ast} \times (N'
\cup A)^{\ast}$ by putting $\vec{\alpha} \approx \vec{\beta}$ if
$\vec{\alpha}$ and $\vec{\beta}$ are of equal length and $\alpha_i
\sim \beta_i$ for every $i$. A relation $\sim\; \subseteq N \times
N'$ is called a \textbf{forward rule simulation} or
an \textbf{R--simulation}
%%%
\index{rule simulation!forward}%%
\index{R--simulation}%%
%%%
if (0) $\mbox{\tt S} \sim \mbox{\tt S}'$, (1) if $X \pf
\vec{\alpha} \in R$ and $X \sim Y$ then there exists a
$\vec{\beta}$ such that $\vec{\alpha} \approx \vec{\beta}$ and $Y
\pf \vec{\beta} \in R'$, and (2) if $Y \pf \vec{\beta} \in R'$
and $X \sim Y$ then there exists an $\vec{\alpha}$ such that
$\vec{\alpha} \approx \vec{\beta}$ and $X \pf \vec{\alpha} \in R$.
%%%
\index{rule simulation!backward}%%
%%%
A \textbf{backward simulation} is defined thus.
(0) From $\mbox{\tt S} \sim X$ follows $X = \mbox{\tt S}'$
and from $Y \sim \mbox{\tt S}'$ follows $Y = \mbox{\tt S}$,
(1) if $X \pf \vec{\alpha} \in R$ and $\vec{\alpha} \approx
\vec{\beta}$ then $Y \pf \vec{\beta} \in R'$ for some $Y$
such that $X \sim Y$, and (2) if $Y \pf \vec{\beta} \in R'$ and
$\vec{\beta} \approx \vec{\alpha}$ then $X \pf \vec{\alpha} \in R$
for some $X$ such that $X \sim Y$.

We give an example of a forward simulation. Let
$G$ and $G'$ be the following grammars.
%%
\begin{equation}%{l@{\quad\pf\quad}l@{\qquad\qquad}l@{\quad\pf\quad}l}
\begin{array}{ll@{\qquad}ll}
\mbox{\tt S} & \pf \mbox{\tt ASB} \mid \mbox{\tt AB} 
	& \mbox{\tt S} & \pf \mbox{\tt ATB} \mid \mbox{\tt ASC}
        \mid \mbox{\tt AC} \\
\mbox{\tt A} & \pf \mbox{\tt b}  
	& \mbox{\tt T} & \pf \mbox{\tt ATC} \mid \mbox{\tt AC} \\
\mbox{\tt B} & \pf \mbox{\tt b}  
	& \mbox{\tt A} & \pf \mbox{\tt a} \\
             & & \mbox{\tt B} & \pf \mbox{\tt b} \\
             & & \mbox{\tt C} & \pf \mbox{\tt b} \\
\end{array}
\end{equation}
%%
The start symbol is {\tt S} in both grammars.
Then the following is an  R--simulation.
%%
\begin{equation}
\sim := \{\auf \mbox{\tt A}, \mbox{\tt A}\zu,
\auf \mbox{\tt B}, \mbox{\tt B}\zu, \auf \mbox{\tt S}, \mbox{\tt S}\zu,
\auf \mbox{\tt B}, \mbox{\tt C}\zu, \auf \mbox{\tt S}, \mbox{\tt T}\zu\}
\end{equation}
%%
Together with $\sim$ also the converse relation $\sim^{\smallsmile}$
is an R--simulation. If $\sim$ is an R--simulation and
$\auf \vec{\alpha}_i : i < n+1\zu$ is a $G$--derivation
there exists a $G'$--derivation $\auf \vec{\beta}_i : i < n+1\zu$
such that $\vec{\alpha}_i \approx \vec{\beta}_i$ for every $i < n+1$.
We can say more exactly that if
$\auf \vec{\alpha}_i, C, \vec{\alpha}_{i+1}\zu$ is an instance
of a rule from $G$ where $C = \auf \kappa_1, \kappa_2\zu$ then
there is a context $D = \auf \lambda_1, \lambda_2\zu$ such that
$\auf \vec{\beta}_i, D, \vec{\beta}_{i+1}\zu$ is an instance
of a rule from $G'$. In this way we get that for every
$\GB = \auf B, <,\sqsubset, \ell\zu \in L_B(G)$ there is a
$\GC = \auf B, <, \sqsubset, \mu\zu \in L_B(G')$ such that
$\ell(x) = \mu(x)$ for every leaf and $\ell(x) \sim \mu(x)$
for every nonleaf. Analogously to a rule simulation we can
define a simulation of derivation by requiring that for every
$G$--derivation $\Gamma$ there is a $G'$--derivation
$\Delta$ which is equivalent to it.
%%
\begin{prop}
Let $G_1$ and $G_2$ be slender CFGs and 
$\sim\; \subseteq \; N_1 \times N_2$ be an R--simulation. 
Then for every $G_1$--derivation
$\auf \vec{\alpha}_i : i < n\zu$ there exists a $G_2$--derivation
$\auf \vec{\beta}_i : i < n\zu$ such that $\vec{\alpha}_i \approx
\vec{\beta}_i$, $i < n$.
\proofend
\end{prop}
%%
We shall look at two special cases of simulations.
Two grammars $G$ and $G'$ are called
%%%
\index{equivalence}%%
%%%
\textbf{equivalent} if there is a bijection $b \colon N \cup A \pf
N' \cup A$ such that $b(x) = x$ for every $x \in A$, $b(S) = S'$
and $\oli{b}$ induces a bijection between $G$--derivations and
$G'$--derivations. This notion is more restrictive than the
one which requires that $\oli{b}$ is a bijection between the
sets of rules.  For it may happen that certain rules can never be
used in a derivation. For given CFGs we
can easily decide whether they are equivalent. To begin, we
bring them into a form in which all rules are used in a
derivation, by removing all symbols that are not reachable
and not completable. Such grammars are equivalent if there is a
bijection $b$ which puts the rules into correspondence.
The existence of such a bijection is easy to check.

The notion of equivalence just proposed is too strict in
one sense.  There may be nonterminal symbols which cannot
be distinguished.  We say $G$ is \textbf{reducible to} $G'$ if
%%%
\index{reducibility}%%
%%%
there is a surjective function $b \colon N \cup A \epi N' \cup A'$
such that $b(S) = S'$, $b(x) = x$ for every $x \in A$ and
such that $\oli{b}$ maps every $G$--derivation onto a
$G'$--derivation, while every preimage under $\oli{b}$
of a $G'$--derivation is a $G$--derivation.
(We do not require however that the preimage of the start symbol
from $G'$ is unique; only that the start symbol from
$G$ has {\it one\/} preimage which is a start symbol
of $G'$.)
%%
\begin{defn}
%%%
\index{grammar!reduced}%%
%%%
$G$ is called \textbf{reduced} if every grammar $G'$ such that 
$G$ is reducible onto $G'$ can itself be reduced onto $G$.
\end{defn}
%%
Given $G$ we can effectively construct a reduced grammar
onto which it can be reduced. We remark that in our example above
$G'$ is not reducible onto $G$. For even though $\sim^{\smallsmile}$
is a function (with $\mbox{\tt A} \mapsto \mbox{\tt A},
\mbox{\tt B} \mapsto \mbox{\tt B},
\mbox{\tt C} \mapsto \mbox{\tt B},
\mbox{\tt S} \mapsto \mbox{\tt S},
\mbox{\tt T} \mapsto \mbox{\tt S}$)
and {\tt ASB} can be derived from {\tt S} in one step, {\tt ATB}
cannot be derived from {\tt S} in one step. Given $G$ and the
function $\sim^{\smallsmile}$ the following grammar is reduced
onto $G$.
%%
\begin{align}
\begin{split}
%\begin{array}{l@{\quad \pf\quad}l}
\mbox{\tt S} & \pf \mbox{\tt ASB} \mid \mbox{\tt ATB} \mid \mbox{\tt ASC}
    \mid \mbox{\tt ATC} \mid \mbox{\tt AB} \mid \mbox{\tt AC} \\
\mbox{\tt T} & \pf \mbox{\tt ASB} \mid \mbox{\tt ATB} \mid \mbox{\tt ASC}
    \mid \mbox{\tt ATC} \mid \mbox{\tt AB} \mid \mbox{\tt AC} \\
\mbox{\tt A} & \pf \mbox{\tt a} \\
\mbox{\tt B} & \pf \mbox{\tt b} \\
\mbox{\tt C} & \pf \mbox{\tt b}
%\end{array}$$
\end{split}
\end{align}
%%
Now let $G$ be a CFG. We add to $A$ two more symbols,
namely {\tt (} and {\tt )}, not already contained in $A$. Subsequently, 
we replace every rule $X \pf \vec{\alpha}$ by the rule 
$X \pf \mbox{\tt (}\conc %
\vec{\alpha} \conc \mbox{\tt )}$. The so--constructed grammar
is denoted by $G^b$.
%%
\begin{equation}
\begin{array}{l@{\quad \pf\quad}l@{\qquad\qquad}l@{\quad\pf\quad}l}
\multicolumn{2}{c}{G} & \multicolumn{2}{c}{G^b} \\
\mbox{\tt S} & \mbox{\tt AS} \mid \mbox{\tt SB} \mid \mbox{\tt AB}
    & \mbox{\tt S} & \mbox{\tt (AS)} \mid \mbox{\tt (SB)}
        \mid \mbox{\tt (AB)} \\
\mbox{\tt A} & \mbox{\tt a}        & \mbox{\tt A} & \mbox{\tt (a)} \\
\mbox{\tt B} & \mbox{\tt b}        & \mbox{\tt B} & \mbox{\tt (b)}
\end{array}
\end{equation}
%%
The grammar $G$ generates the language 
$\mbox{\tt a}^+ \mbox{\tt b}^+$. The string {\tt aabb} has several 
derivations, which correspond to different trees.
%%
\begin{equation}
\begin{array}{l}
\auf \mbox{\tt S}, \mbox{\tt AS}, \mbox{\tt ASB},
\mbox{\tt AABB}, \dotsc, \mbox{\tt aabb}\zu \\
\auf \mbox{\tt S}, \mbox{\tt SB}, \mbox{\tt ASB},
\mbox{\tt AABB}, \dotsc, \mbox{\tt aabb}\zu
\end{array}
\end{equation}
%%
If we look at the analogous derivations in $G^b$ we get the
strings
%%
\begin{equation}
\mbox{\tt ((a)(((a)(b))(b)))}, \qquad
\mbox{\tt (((a)((a)(b)))(b))}
\end{equation}
%%
These are obviously distinct. Define a homomorphism $\oli{e}$ by
$\oli{e}(a) := a$, if $a \in A$, $\oli{e}\colon \mbox{\tt )} \mapsto %
\varepsilon$ and  $\oli{e} \colon \mbox{\tt )} \mapsto \varepsilon$.
Then it is not hard to see that
%%
\begin{equation}
L(G) = \oli{e}[L(G^b)]
\end{equation}
%%
Now look at the class of trees $L(G)$ and forget the labels of
all nodes which are not leaves. Then the structure obtained
shall be called a \textbf{bracketing analysis}
%%%
\index{bracketing analysis}%%
%%%
of the associated strings. The reason is that the bracketing
analyses are in one--to--one correspondence with the strings which
$L(G^b)$ generates. Now we will ask ourselves whether for two
given grammars $G$ and $H$ it is decidable whether they generate
the same bracketing analyses. We ask ourselves first what the
analogon of a derivation of $G$ is in $G^b$. Let $\vec{\gamma} X
\vec{\eta}$ be derivable in $G$, and let the corresponding
$G^b$--string in this derivation be $\vec{\gamma}^b X
\vec{\eta}^b$. In the next step $X$ is replaced by $\alpha$. Then
we get $\vec{\gamma}\vec{\alpha}\vec{\eta}$,
and in $G^b$ the string $\vec{\gamma}^b \mbox{\tt (}\vec{\alpha}%
\mbox{\tt )}\vec{\delta}^b$. If we have an R--simulation to $H$
then it is also an R--simulation from $G^b$ to $H^b$ provided
that it sends the opening bracket of $G^b$ to the opening
bracket of $H^b$ and the closing bracket of $G^b$ to the closing
bracket of $H^b$. It follows that if there is an R--simulation
from $G$ to $H$ then not only we have $L(G) = L(H)$ but also
$L(G^b) = L(H^b)$.
%%
\begin{thm}
We have $L(G^b) = L(H^b)$ if there is an R--si\-mu\-la\-tion
from $G$ to $H$.
\end{thm}
%%
The bracketing analysis is too strict for most purposes.
First of all it is not customary to put a single symbol into
brackets. Further, it makes no sense to distinguish between
$\mbox{\tt ((}\vec{x}\mbox{\tt ))}$ and
$\mbox{\tt (}\vec{x}\mbox{\tt )}$,
since both strings assert that $\vec{x}$ is a
constituent. We shall instead use what we call 
\textbf{constituent analyses}.
%%%
\index{constituent analysis}%%
%%%
These are pairs $\auf \vec{x}, \GC\zu$ in which $\vec{x}$
is a string and $\GC$ an exhaustively ordered constituent
structure defined over $\vec{x}$. We shall denote by
%%%%
\index{$L_c(G)$}%%
%%%%
$L_c(G)$ the class of all constituent analyses generated by
$G$. In order to switch from bracketing analyses to
constituent analyses we only have to eliminate the unary
rules. This can be done as follows. Simply replace every rule
$\rho = Y \pf \vec{\alpha}$, where $|\vec{\alpha}| > 1$, 
by the set $\rho^2 := \{Z \pf \vec{\alpha} : 
Z \Pf^{\ast} Y\}$. $R^> := \bigcup \auf \rho^2 : \rho \in R\zu$. 
Finally, let $G^> := \auf \mbox{\tt S}, N, A, R^>\zu$.
Every rule is strictly productive and we have $L_c(G) = L_c(G^>)$. 
(Exception needs to be made for $\mbox{\tt S} \pf \varepsilon$, 
as usual. Also, if necessary, we shall assume that $G^{>}$ is slender.)
%%
\begin{defn}
%%%
\index{grammar!standard form}%%
\index{grammar!Chomsky Normal Form}%%
\index{standard form}%%
\index{Chomsky Normal Form}%%
%%%
A CFG is in \textbf{standard form} if every
rule different from $\mbox{\tt S} \pf \varepsilon$ has the form
$X \pf \vec{Y}$ with $|\vec{Y}| > 1$ or the form $X \pf a$.
A grammar is in \textbf{2--standard form} or
\textbf{Chomsky Normal Form} if every rule is of the form
$\mbox{\tt S} \pf \varepsilon$, $X \pf Y_0 Y_1$ or $X \pf a$.
\end{defn}
%%
(Notice that by our conventions a CFG in
standard form contains the rule $X \pf \varepsilon$  for
$X = \mbox{\tt S}$, but this happens only if {\tt S} is not on the
right hand side of a rule.) We already have proved that
the following holds.
%%
\begin{thm}
For every CFG $G$ one can construct a slender
CFG $G^n$ in standard form which generates
the same constituent structures as $G$.
\end{thm}
%%
\begin{thm}
For every CFG $G$ we can construct a slender
CFG $G^c$ in Chomsky Normal Form such that
$L(G^c) = L(G)$.
\end{thm}
%%
\proofbeg
We may assume that $G$ is in standard form.
Let $\rho = X \pf Y_0 Y_1 \dotsb Y_{n-1}$ be a rule with $n > 2$.
Let $Z^{\rho}_0, Z^{\rho}_1, \dotsc, Z^{\rho}_{n-2}$
be new nonterminals. Replace $\rho$ by the rules
%%
\begin{multline}
\rho^c_0 := X \pf Y_0 Z^{\rho}_0, \;
\rho^c_1 := Z^{\rho}_0 \pf Y_1 Z^{\rho}_1, 
\dotsc,  \\
\rho^c_{n-2} := Z^{\rho}_{n-3} \pf Y_{n-2} Y_{n-1}
\end{multline}
%%
Every derivation in $G$ of a string $\vec{\alpha}$ can be
translated into a derivation in $G^c$ by replacing every
instance of $\rho$ by a sequence $\rho^c_0, \rho^c_1, \dotsc, %
\rho^c_{n-1}$. For the converse we introduce the following
priorisation $\pi$ on the rules. Let $Z_i^{\rho}$ be always
before $Y_i$. However, in $Z_{n-3}^{\rho} \pf Y_{n-2} Y_{n-1}$
we choose the leftmost priorisation. We show $G \vdash^{\ell}
\vec{x}$ iff $G^c \vdash^{\pi} \vec{x}$.
For if $\auf \alpha_i : i < p+1\zu$ is a leftmost
derivation of $\vec{x}$ in $G$, then replace every
instance of a rule $\rho$ by the sequence $\rho^c_0$,
$\rho^c_1$, and so on until $\rho^c_{n-2}$.
This is a $G^c$--derivation, as is easily checked.
It is also a $\pi$--derivation. Conversely, let
$\auf \beta_j : j < q+1\zu$ be a $G^c$--derivation
which is priorized with $\pi$. If $\beta_{i+1}$ is the
result of an application of the rule $\rho^c_k$,
$k < n-2$, then $i +2 < q+1$ and $\beta_{i+2}$ is the
result of an application of $\rho^c_{k+1}$ on $\beta_{i+1}$,
which replaced exactly the occurrence $Z_k$ of the previous
instance. This means that every $\rho^c_k$ in a block
of instances of $\rho^c_0$, $\rho^c_1, \dotsc, \rho^c_{n-2}$ 
corresponds to a single instance of $\rho$. There exists a 
$G$--derivation of $\vec{x}$, which can be obtained
by backward replacement of the blocks. It is a leftmost
derivation.
\proofend
%%

For example, the right hand side grammar is the result of the
conversion of the left hand grammar into Chomsky Normal Form.
%%
\begin{equation}
\begin{array}{ll@{\qquad\qquad}ll}
\mbox{\tt S} & \pf \mbox{\tt ASBBT} \mid \mbox{\tt ABB}
    & \mbox{\tt S} & \pf \mbox{\tt AX} \mid \mbox{\tt AV} \\
& & \mbox{\tt V} & \pf \mbox{\tt BB} \\
& & \mbox{\tt X} & \pf \mbox{\tt SY} \\
& & \mbox{\tt Y} & \pf \mbox{\tt BZ} \\
& & \mbox{\tt Z} & \pf \mbox{\tt BT} \\
\mbox{\tt T} & \pf \mbox{\tt CTD} \mid \mbox{\tt CD}
    & \mbox{\tt T} & \pf \mbox{\tt CW} \mid \mbox{\tt CD} \\
& & \mbox{\tt W} & \pf \mbox{\tt TD} \\
\mbox{\tt A} & \pf \mbox{\tt a} & \mbox{\tt A} & \pf \mbox{\tt a} \\
\mbox{\tt B} & \pf \mbox{\tt b} & \mbox{\tt B} & \pf \mbox{\tt b} \\
\mbox{\tt C} & \pf \mbox{\tt c} & \mbox{\tt C} & \pf \mbox{\tt c} \\
\mbox{\tt D} & \pf \mbox{\tt d} & \mbox{\tt D} & \pf \mbox{\tt d}
\end{array}
\end{equation}
%%
\begin{defn}
%%%
\index{grammar!invertible}%%
%%%
A CFG is called \textbf{invertible}
if from $X \pf \vec{\alpha} \in R$ and $Y \pf \vec{\alpha}
\in R$ it follows that $X = Y$.
\end{defn}
%%
For an invertible grammar the labelling on the leaves uniquely
determines the labelling on the entire tree. We propose an algorithm
which creates an invertible grammar from a CFG.
For simplicity a rule is of the form $X \pf \vec{Y}$ or
$X \pf \vec{x}$. Now we choose our
nonterminals from the set $\wp(N) - \{\varnothing\}$. The
terminal rules are now of the form $\SX \pf \vec{x}$, where
$\SX = \{X : X \pf \vec{x} \in R\}$.  The nonterminal rules
are of the form $\SX \pf \SY_0 \SY_1 \dotsb \SY_{n-1}$ with
%%
\begin{equation}
\SX = \{X : X \pf Y_0 Y_1 \dotsb Y_{n-1} \in R
\text{ for some }Y_i \in \SY_i\} 
\end{equation}
%%
Further, we choose a start symbol, $\Sigma$, and we take
the rules $\Sigma \pf \vec{\SX}$ for every $\vec{X}$, for
which there are $X_i \in \SX_i$ with $S \pf \vec{X} \in R$.
This grammar we call $G^i$. It is not difficult to show
that $G^i$ is invertible.  For let $\SY_0 \SY_1 \dotsb \SY_{n-1}$
be the right hand side of a production. Then there exist
$Y_i \in \SY_i$, $i < n$, and an $X$ such that $X \pf \vec{Y}$
is a rule in $G$. Hence there is an $\SX$ such that $\SX \pf \vec{\SY}$
is in $G^i$. $\SX$ is uniquely determined. Further, $G^i$ is
in standard form (Chomsky  Normal Form), if this is the case
with $G$.
%%
\begin{thm}
\label{thm:invertierbar}
Let $G$ be a CFG. Then we can construct an
invertible CFG $G^i$ which generates the same
bracketing analyses as $G$.
\proofend
\end{thm}
%%
The advantage offered by invertible grammars is that the labelling
can be reconstructed from the labellings on the leaves. The reader
may reflect on the fact that $G$ is invertible exactly if
$G^b$ is.
%%
\begin{defn}
%%%
\index{grammar!perfect}%%
%%%
A CFG is called \textbf{perfect} if it is in
standard form, slender, reduced and invertible.
\end{defn}
%%
It is instructive to see an example of a grammar which is
invertible but not reduced.
%%
\begin{equation}
\begin{array}{l@{\quad\pf\quad}l@{\qquad}l@{\quad\pf\quad}l}
\multicolumn{2}{c}{G} & \multicolumn{2}{c}{H} \\
\mbox{\tt S} & \mbox{\tt AS} \mid \mbox{\tt BS} \mid \mbox{\tt A}
    \mid \mbox{\tt B} & \mbox{\tt S} & \mbox{\tt CS} \mid
        \mbox{\tt C} \\
\mbox{\tt A} & \mbox{\tt a} & \mbox{\tt C} & \mbox{\tt a} \mid
    \mbox{\tt b} \\
\mbox{\tt B} & \mbox{\tt b} & \multicolumn{2}{c}{}
\end{array}
\end{equation}
%%
$G$ is invertible but not reduced. To this end look at
$H$ and the map $\mbox{\tt A} \mapsto \mbox{\tt C}$, 
$\mbox{\tt B} \mapsto \mbox{\tt C}$, 
$\mbox{\tt S} \mapsto \mbox{\tt S}$.
This is an $R$--simulation. $H$ is reduced and
invertible.
%%
\begin{thm}
For every CFG we can construct a perfect CFG which generates 
the same constituent structures.
\end{thm}
%%
Finally we shall turn to the so--called {\it Greibach Normal Form}.
This form most important for algorithms recognizing languages by
reading the input from left to right. Such algorithms have
problems with rules of the form $X \pf Y \conc \vec{\alpha}$,
in particular if $Y = X$.
%%
\begin{defn}
%%%
\index{Greibach Normal Form}%%
%%%
Let $G = \auf \mbox{\tt S}, N, A, R\zu$ be a CFG. $G$ is in 
\textbf{Grei\-bach} (\textbf{Nor\-mal}) \textbf{Form} if every 
rule is of the form $\mbox{\tt S} \pf \varepsilon$ or of the form
$X \pf x \conc \vec{Y}$.
\end{defn}
%%
\begin{prop}
Let $G$ be in Greibach Normal Form. If $X \vdash_G \vec{\alpha}$ 
then $\vec{\alpha}$ has a leftmost derivation from $X$ in $G$ 
iff $\vec{\alpha} = \vec{y} \conc \vec{Y}$ for some 
$\vec{y} \in A^{\ast}$ and $\vec{Y} \in N^{\ast}$ and 
$\vec{y} = \varepsilon$ only if $\vec{Y} = X$.
\end{prop}
%%
The proof is not hard.  It is also not hard
to see that this property characterizes the Greibach form
uniquely. For if there is a rule of the form
$X \pf Y \conc \vec{\gamma}$ then there is a leftmost
derivation of $Y \conc \vec{\gamma}$ from $X$, but not in the
desired form. Here we assume that there are no rules of the form
$X \pf X$.
%%
\nocite{greibach:normal}
\begin{thm}[Greibach]
\label{greibach}
For every CFG one can effectively construct a
grammar $G^g$ in Greibach Normal Form with $L(G^g) = L(G)$.
\end{thm}
%%
Before we start with the actual proof we shall prove some
auxiliary statements. We call $\rho$ an $X$--\textbf{production}
if $\rho = X \pf \vec{\alpha}$
%%%
\index{production!$X$--\faul}%%
\index{production!left recursive}%%
%%%
for some $\vec{\alpha}$. Such a production is called
\textbf{left recursive} if it has the form
$X \pf X \conc \vec{\beta}$. Let $\rho = X \pf \vec{\alpha}$
be a rule; define $R^{- \rho}$ as follows. For every
factorisation $\vec{\alpha} = \vec{\alpha}_1 \conc Y \conc %
\vec{\alpha}_2$ of $\vec{\alpha}$ and every rule $Y \pf \vec{\beta}$
add the rule $X \pf \vec{\alpha}_1 \conc \vec{\beta} \conc %
\vec{\alpha}_2$ to $R$ and finally remove the rule $\rho$.
Now let $G^{-\rho} := \auf \mbox{\tt S}, N, A, R^{- \rho}\zu$.
Then $L(G^{- \rho}) = L(G)$. We call this construction
as \textbf{skipping} the rule $\rho$.
%%%
\index{rule!skipping of a \faul}%%
%%%
The reader may convince himself that the tree for
$G^{-\rho}$ can be obtained in a very simple way from
trees for $G$ simply by removing all nodes $x$ which
dominate a local tree corresponding to the rule
$\rho$, that is to say, which are isomorphic to $\GH_{\rho}$.
(This has been defined in Section~\ref{kap1}.\ref{einsvier}.)
This technique works only if $\rho$ is not an
{\tt S}--production. In this case we proceed as follows.
Replace $\rho$ by all rules of the form $\mbox{\tt S} \pf %
\vec{\beta}$ where $\vec{\beta}$ derives from $\vec{\alpha}$
by applying a rule. Skipping a rule does not necessarily
yield a new grammar. This is so if there are rules of the form
$X \pf Y$ (in particular rules like $X \pf X$).
%%
\begin{lem}
\label{lem:linksrek}
Let $G = \auf \mbox{\tt S}, N, A, R\zu$ be a CFG 
and let $X \pf X \conc \vec{\alpha}_i$, $i < m$, be all left recursive
$X$--productions as well as $X \pf \vec{\beta}_j$, $j < n$, all
non left recursive $X$--productions.  Now let $G^1 := \auf \mbox{\tt S},
N \cup \{Z\}, A, R^1\zu$, where $Z \not\in N %
\cup A$ and $R^1$ consists of all $Y$--productions from
$R$ with $Y \neq X$ as well as the productions
%%
\begin{equation}
\begin{array}{l@{\quad}l@{\qquad\qquad}l@{\quad}l}
X \pf \vec{\beta}_j & j < n, &
    Z \pf \vec{\alpha}_i & i < m, \\
X \pf \vec{\beta}_j \conc Z & j < n, &
    Z \pf \vec{\alpha}_i \conc Z & i < m.
    \end{array}
\end{equation}
%%
Then $L(G^1) = L(G)$.
\end{lem}
%%
\proofbeg
We shall prove this lemma rather extensively since the method is
relatively tricky. We consider the following priorisation on
$G^1$. In all rules of the form $X \pf \vec{\beta}_j$ and
$Z \pf \vec{\alpha}_i$ we take the natural ordering (that is,
the leftmost ordering) and in all rules $X \pf \vec{\beta}_jZ$
as well as $Z \pf \vec{\alpha}_iZ$ we also put the left to right
ordering except that $Z$ precedes all elements from $\vec{\alpha}_j$
and $\vec{\beta}_i$, respectively. This defines the linearisation
$\shd$. Now, let $M(X)$ be the set of all
$\vec{\gamma}$ such that there is a leftmost derivation from
$X$ in $G$ in such a way that $\vec{\gamma}$ is the first element
not of the form $X \conc \vec{\delta}$.  Likewise, we define
$P(X)$ to be the set of all $\vec{\gamma}$ which can be derived
from $X$ priorized by $\shd$ in $G^1$ such that $\vec{\gamma}$ is
the first element which does not contain $Z$. We claim that
$P(X) = M(X)$. It can be seen that
%%
\begin{equation}
M(X) = \bigcup_{j < n} \vec{\beta}_j \cdot (\bigcup_{i < m}
\vec{\alpha}_i)^{\ast} = P(X)
\end{equation}
%%
From this the desired conclusion follows thus.  Let
$\vec{x} \in L(G)$. Then there exists a leftmost derivation
$\Gamma = \auf A_i : i < n+1\zu$ of $\vec{x}$. (Recall that 
the $A_i$ are instances of rules.) This derivation is cut into 
segments $\Sigma_i$, $i < \sigma$, of length $k_i$, such that
%%
\begin{equation}
\Sigma_i = \auf A_j :
\sum_{p < i} k_p \leq j  < 1 + \sum_{p < i+1} k_i \zu
\end{equation}
%%
This partitioning is done in such a way that
each $\Sigma_i$ is a maximal portion of $\Gamma$ of
$X$--productions or a maximal portion of $Y$--productions
with $Y \neq X$. The $X$--segments can be replaced by a
$\shd$--derivation $\wht{\Sigma}_i$ in $G^1$, by the previous
considerations. The segments which do not contain $X$--productions
are already $G^1$--derivations.  For them we put
$\wht{\Sigma}_i := \Sigma_i$. Now let $\wht{\Gamma}$ be 
result of stringing together the $\wht{\Sigma}_i$. This is 
well--defined, since the first string of $\wht{\Sigma}_i$ 
equals the first string of $\Sigma_i$, as the last string 
of $\wht{\Sigma}_i$ equals the last string of
$\Sigma_i$. $\wht{\Gamma}$ is a $G^1$--derivation,
priorized by $\shd$. Hence $\vec{x} \in L(G^1)$.
The converse is analogously proved, by beginning with a
derivation priorized by $\shd$.
\proofend
%%

Now to the proof of Theorem~\ref{greibach}. We may assume at the
outset that $G$ is in Chomsky Normal Form. We choose an
enumeration of $N$ as $N = \{X_i : i < p\}$. We claim first that
by taking in new nonterminals we can see to it that we get a
grammar $G^1$ such that $L(G^1) = L(G)$ in which the
$X_i$--productions have the form $X_i \pf x \conc \vec{Y}$
or $X_i \pf X_j \conc \vec{Y}$ with $j > i$. This we prove by
induction on $i$. Let $i_0$ be the smallest $i$ such that
there is a rule $X_i \pf X_j \conc \vec{Y}$ with $j \leq i$.
Let $j_0$ be the largest $j$ such that $X_{i_0} \pf X_j \conc %
\vec{Y}$ is a rule. We distinguish two cases. The first is
$j_0 = i_0$. By the previous lemma we can eliminate the
production by introducing some new nonterminal symbol $Z_{i_0}$.
The second case is $j_0 < i_0$. Here we apply the induction
hypothesis on $j_0$. We can skip the rule $X_{i_0} \pf X_{j_0} %
\conc \vec{Y}$ and introduce rules of the form (a) $X_{i_0} \pf %
X_k \conc \vec{Y'}$ with $k > j_0$. In this way the second case
is either eliminated or reduced to the first.

Now let $P := \{Z_i : i < p\}$ be the set of newly introduced
nonterminals. It may happen that for some $j$ $Z_j$ does not
occur in the grammar, but this does not disturb the proof.
Let finally $P_i := \{Z_j : j < i\}$. At the end of this
reduction we have rules of the form
%%
\begin{subequations}
\begin{align}
\label{eq:rule21a}
& X_i \pf X_j \conc \vec{Y} & (j > i) \\
\label{eq:rule21b}
& X_i \pf x \conc \vec{Y} & (x \in A) \\
\label{eq:rule21c}
& Z_i \pf \vec{W} & (\vec{W} \in (N \cup P_i)^+
    \conc (\varepsilon \cup Z_i)) 
\end{align}
\end{subequations}
%%
It is clear that every $X_{p-1}$--production already has the form
$X_{p-1} \pf x \conc \vec{Y}$. If some $X_{p-2}$--production
has the form \eqref{eq:rule21a} then we can skip
this rule and get rules of the form $X_{p-2} \pf \vec{x}\vec{Y'}$.
Inductively we see that all rules of the form can be eliminated in
favour of rules of the form \eqref{eq:rule21b}. Now finally the rules 
of type \eqref{eq:rule21c}. Also these rules can be skipped, and then 
we get rules of the form $Z \pf x \conc \vec{Y}$ for some $x \in A$, 
as desired.

For example, let the following grammar be given.
%%
\begin{equation}
\begin{array}{l@{\quad\pf\quad}l@{\qquad}l@{\quad\pf\quad}l}
\mbox{\tt S} & \mbox{\tt SDA} \mid \mbox{\tt CC} &
	\mbox{\tt A} & \mbox{\tt a} \\
\mbox{\tt D} & \mbox{\tt DC} \mid \mbox{\tt AB} &
	\mbox{\tt B} & \mbox{\tt b} \\
\multicolumn{2}{c}{} &
	\mbox{\tt C} & \mbox{\tt c}
\end{array}
\end{equation}
%%
The production $\mbox{\tt S} \pf \mbox{\tt SDA}$ is
left recursive. We replace it according to the above lemma
by
%%
\begin{equation}
\mbox{\tt S} \pf \mbox{\tt CCZ},
\quad \mbox{\tt Z} \pf \mbox{\tt DA},
\quad \mbox{\tt Z} \pf \mbox{\tt DAZ} 
\end{equation}
%%
Likewise we replace the production $\mbox{\tt D} \pf\mbox{\tt DC}$ by
%%
\begin{equation}
\mbox{\tt D} \pf \mbox{\tt ABY},
\quad \mbox{\tt Y} \pf \mbox{\tt C},
\quad \mbox{\tt Y} \pf \mbox{\tt CY}
\end{equation}
%%
With this we get the grammar
%%
\begin{equation}
\begin{array}{l@{\quad\pf\quad}l@{\qquad}l@{\quad\pf\quad}l}
\mbox{\tt S} & \mbox{\tt CC} \mid \mbox{\tt CCZ} &
	\mbox{\tt A} & \mbox{\tt a} \\
\mbox{\tt Z} & \mbox{\tt DA} \mid \mbox{\tt DAZ} &
	\mbox{\tt B} & \mbox{\tt b} \\
\mbox{\tt D} & \mbox{\tt AB} \mid \mbox{\tt ABY} &
	\mbox{\tt C} & \mbox{\tt c} \\
\mbox{\tt Y} & \mbox{\tt C} \mid \mbox{\tt CY} & 
	\multicolumn{2}{c}{} \\
\end{array}
\end{equation}
%%
Next we skip the {\tt D}--productions.
%%
\begin{equation}
\begin{array}{l@{\quad\pf\quad}l@{\qquad}l@{\quad\pf\quad}l}
\mbox{\tt S} & \mbox{\tt CC} \mid \mbox{\tt CCZ} &
	\mbox{\tt A} & \mbox{\tt a} \\
\mbox{\tt Z} & \mbox{\tt ABA} \mid \mbox{\tt ABYA}
    \mid \mbox{\tt ABAZ} \mid \mbox{\tt ABYAZ} & 
	\mbox{\tt B} & \mbox{\tt b} \\
\mbox{\tt D} & \mbox{\tt AB} \mid \mbox{\tt ABY} &
	\mbox{\tt C} & \mbox{\tt c} \\
\mbox{\tt Y} & \mbox{\tt C} \mid \mbox{\tt CY} & 
	\multicolumn{2}{c}{} 
\end{array}
\end{equation}
%%
Next {\tt D} can be eliminated (since it is not reachable)
and we can replace on the right hand side of the productions
the first nonterminals by terminals.
%%
\begin{equation}
\begin{array}{l@{\quad\pf\quad}l}
\mbox{\tt S} & \mbox{\tt cC} \mid \mbox{\tt cCZ} \\
\mbox{\tt Z} & \mbox{\tt aBA} \mid \mbox{\tt aBYA}
    \mid \mbox{\tt aBAZ} \mid \mbox{\tt aBYZ} \\
\mbox{\tt Y} & \mbox{\tt c} \mid \mbox{\tt cY}
\end{array}
\end{equation}
%%
Now the grammar is in Greibach Normal Form.
%%
\vplatz
\exercise
Show that for a CFG $G$ it is decidable
(a) whether $L(G) = \varnothing$, (b) whether $L(G)$ is
finite, (c) whether $L(G)$ is infinite.
%%
\vplatz
\exercise
Let $G^i$ be the invertible grammar constructed from
$G$ as defined above.  Show that the relation $\sim$
defined by
%%
\begin{equation}
\SX \sim Y \quad \Dpf \quad Y \in \SX
\end{equation}
%%
is a backward simulation from $G^i$ to $G$.
%%
\vplatz
\exercise
\label{uebung:zweig}
%%%
\index{branch expression}%%
%%%
Let $\auf B, <, \sqsubset, \ell\zu$ be an ordered labelled
tree. If $x$ is a leaf then $\uppx{x}$ is a branch and can
be thought of in a natural way as a string $\auf \uppx{x},
>, \ell\zu$. Since the leaf $x$ plays a special role, we
shall omit it. We say, a \textbf{branch expression of} $\GB$ is a
string of the form $\auf \uppx{x} - \{x\}, >, \ell\zu$, $x$
a leaf of $\GB$. We call it $\zeta(x)$. Show that the set of 
all branch expressions of trees from $L_B(G)$ is regular.
%%
\vplatz
\exercise
Let $G$ be in Greibach Normal Form and $\vec{x}$ a terminal
string of length $n > 0$. Show that every derivation of
$\vec{x}$ has exactly the length $n$. How long is a derivation 
for an arbitrary string $\vec{\alpha}$?

 \section{Recognition and Analysis}
%
%
%
CFLs can be characterized by special classes of automata, just 
like regular languages. Since there are CFLs that are not regular,
automata that recognize them cannot all be finite state
automata. They must have an infinite memory. The special way
such a memory is organized and manipulated differentiates
the various kinds of nonregular languages. CFLs
can be recognized by so--called {\it pushdown
automata}. These automata have a memory in the form of
a stack onto which they can put symbols and remove (and
read them) one by one. However, the automaton only has
access to the symbol added most recently. A {\it stack\/}
over the alphabet $D$ is a string over $D$. We shall agree
that the first letter of the string is the highest entry
in the stack and the last letter corresponds to the lowest
entry. To denote the end of the stack, we need a special
symbol, which we denote by {\tt \#}. (See Exercise~\ref{ex:oneside} 
for the necessity of an end--of--stack marker.)

A {\it pushdown automaton\/} steers its actions by means of
the highest entry of the stack and the momentary memory state.
Its actions consist of three successive steps.
(1) The disposal or removal of a symbol on the stack.
(2) The moving or not moving of the read head to the right.
(3) The change into a memory state (possibly the same one).
If the automaton does not move the head in (2) we call the
action an $\varepsilon$--\textbf{move}.
%%%
\index{move!$\varepsilon$--\faul}%%
%%%
We write $A_{\varepsilon}$ in place of $A \cup \{\varepsilon\}$.
%%
\begin{defn}
%%%
\index{pushdown automaton}%%
\index{automaton!pushdown}%%
\index{stack alphabet}%%
\index{transition function}%%
%%%
A \textbf{pushdown automaton over} $A$ is a septuple
%%
\begin{equation}
\GK = \auf Q, i_0, A, F, D, \mbox{\tt \#}, \delta\zu
\end{equation}
%%
where $Q$ and $D$ are finite sets, $i_0 \in Q$, $\mbox{\tt \#} \in D$ 
and $F \subseteq Q$, as well as 
%%%
\begin{equation}
\delta \colon Q \times D \times A_{\varepsilon} \pf \wp(Q \times D^{\ast})
\end{equation}
%%%
a function such that $\delta(q,a,d)$ is always finite. We call $Q$ 
the set of \textbf{states}, $i_0$ the \textbf{initial state}, $F$ 
the set of \textbf{accepting states}, $D$ the \textbf{stack alphabet}, 
{\tt \#} the \textbf{beginning of the stack} and $\delta$ the 
\textbf{transition function}.
\end{defn}
%%
\index{configuration}%%
%%%
We call $\Gz := \auf q, \vec{d}\zu$, where $q \in Q$ and $\vec{d}
\in D^{\ast}$, a \textbf{configuration}. We now write
%%
\begin{equation}
\auf p, \vec{d}\zu \stackrel{x}{\pf} \auf p', \vec{d'}\zu
\end{equation}
%%
if for some $\vec{d}_1$ $\vec{d} =  Z \conc \vec{d}_1$, $\vec{d'} =
\vec{e} \conc \vec{d}_1$ and $\auf p', \vec{e}\zu \in
\delta(p,Z,x)$. We call this a \textbf{transition}.
%%%
\index{transition}%%
%%%
We extend the function $\delta$ to configurations. $\auf p',\vec{d'}\zu 
\in \delta(p, \vec{d}, x)$ is also used. Notice that in contrast to a 
pushdown automaton a 
finite state automaton may not change into a new state without reading 
a new symbol. For a pushdown automaton this is necessary in particular 
if the automaton wants to clear the stack. If the stack is empty then the
automaton cannot work further. This means, however, that the
pushdown automaton is necessarily partial. The transition function
can now analogously be extended to strings. Likewise, we can
define it for sets of states. 
%%
\begin{equation}
\Gz \stackrel{\vec{x} \conc \vec{y}}{\longrightarrow} \Gz'
\quad\Dpf\quad
\text{ there exists }\Gz'' \text{ with }
\Gz \stackrel{\vec{x}}{\pf} \Gz'' \stackrel{\vec{y}}{\pf} \Gz'
\end{equation}
%%%
\index{computation}%%
%%%
If $\Gz \stackrel{\vec{x}}{\pf} \Gz'$ we say that there is a
$\GK$--\textbf{computation for} $\vec{x}$ \textbf{from} $\Gz$ 
\textbf{to} $\Gz'$. Now
%%
\begin{equation}
L(\GK) := \{\vec{x} : \text{\it for some }
q \in F, \vec{z} \in D^{\ast} \colon
\auf i_0, \mbox{\tt \#} \zu \stackrel{\vec{x}}{\pf} 
\auf q, \vec{z}\zu\}
\end{equation}
%%
%%%
\index{language!{\faul} accepted by state}%%
\index{pushdown automaton!simple}%%
%%%
We call this the language which is \textbf{accepted by} $\GK$ 
\textbf{by state}. We call a pushdown automaton \textbf{simple} if from
$\auf q, \vec{z}\zu \in \delta(p,Z,a)$ follows $|\vec{z}\conc a| \leq 2$.
It is an exercise to prove the next theorem.
%%
\begin{prop}
\label{prop:einfach}
For every pushdown automaton $\GK$ there is a simple
pushdown automaton $\GL$ such that $L(\GL) = L(\GK)$.
\end{prop}
%%
For this reason we shall tacitly assume that the automaton
does not write arbitrary strings but a single symbol. In addition
to $L(\GK)$ there also is a language which is \textbf{accepted  by}
$\GK$ \textbf{by stack}.
%%%
\index{language!{\faul} accepted by stack}%%
%%%
\begin{equation}
L^s(\GK) := \{\vec{x} : \text{for some } q \in Q \colon
\auf i_0, \mbox{\tt \#}\zu \stackrel{\vec{x}}{\pf} \auf q, \varepsilon\zu\}
\end{equation}
%%
The languages $L(\GK)$ and $L^s(\GK)$ are not necessarily
identical for given $\GK$. However, the set of all languages of the
form $L(\GK)$ for some pushdown automaton equals the set of all
languages of the form $L^s(\GK)$ for some pushdown automaton.
This follows from the next theorem.
%%
\begin{prop}
For every pushdown automaton $\GK$ there is an $\GL$ with
$L(\GK) = L^s(\GL)$ as well as a pushdown automaton $\GM$ with
$L^s(\GK) = L(\GM)$.
\end{prop}
%%
\proofbeg
Let $\GK = \auf Q, i_0, A, F, D, \mbox{\tt \#}, \delta\zu$ be given.
We add to $Q$ two states, $q_i$ and $q_f$. $q_i$ shall be the
new initial state and $F^{\GL} := \{q_f\}$. Further, we add a
new symbol $\flat$ which is the beginning of the stack of
$\GL$. We define $\delta^{\GL}(q_i, \flat, \varepsilon) :=
\{\auf i_0, \mbox{\tt \#} \conc \flat\zu\}$. There are no more
$\delta^{\GL}$--transitions exiting $q_i$.  For $q \neq q_i, q_f$ 
and $Z \neq \flat$ $\delta^{\GL}(q,Z,\vec{x}) := \delta^{\GK}(q,Z,x)$, 
$x \in A$.  Further, if $q \in F$ and $Z \neq \flat$, we put
$\delta^{\GL}(q, Z, \varepsilon) := 
\delta^{\GK}(q,Z, \varepsilon) \cup \{\auf q_f, \varepsilon\zu\}$ 
and otherwise $\delta^{\GL}(q, Z, \varepsilon) := 
\delta^{\GK}(q, Z, \varepsilon)$. Finally, let
$\delta^{\GL}(q_f, Z, x) := \varnothing$ for
$x \in A$ and $\delta^{\GL}(q_f, Z, \varepsilon) :=
\{\auf q_f, \varepsilon\zu\}$ for $Z \in D \cup \{\flat\}$.
Assume now $\vec{x} \in L(\GK)$. Then there is a $\GK$--computation
$\auf i_0, \mbox{\tt \#}\zu \stackrel{\vec{x}}{\pf} %
\auf q, \vec{d}\zu$ for some $q \in F$ and so we also have an 
$\GL$--computation 
$\auf q_i, \flat\zu \stackrel{\vec{x}}{\pf} \auf q_f,\vec{d}\zu$. 
Since $\auf q_f, \vec{d}\zu \stackrel{\varepsilon}{\pf}
\auf q_f, \varepsilon\zu$ we have $\vec{x} \in L^s(\GL)$.
Hence $L(\GK) \subseteq L^s(\GL)$. Now, conversely, let
$\vec{x} \in L^s(\GL)$. Then $\auf q_i, \flat\zu \stackrel{\vec{x}}{\pf}
\auf p, \varepsilon\zu$ for a certain $p$. Then
$\flat$ is deleted only at last since it happens only in
$q_f$ and so $p = q_f$. Further, we have
$\auf q_i, \flat\zu \stackrel{\vec{x}}{\pf} \auf q,
\vec{d}\conc \flat\zu$ for some state $q \in F$.
This means that there is an $\GL$--com\-pu\-ta\-tion
$\auf i_0, \mbox{\tt \#} \conc \flat\zu
\stackrel{\vec{x}}{\pf} \auf q, \vec{d} \conc \flat\zu$.
This, however, is also a $\GK$--computation.
This shows that $L^s(\GL) \subseteq L(\GK)$ and so also the
first claim. Now for the construction of $\GM$. We add two
new states, $q_f$ and $q_i$, and a new symbol, $\flat$,
which shall be the begin of stack of $\GM$, and we put
$F^{\GM} := \{q_f\}$. Again we put $\delta^{\GM}(q_i, \flat, x) %
:= \varnothing$ for $x \in A$ and $\delta^{\GM}(q_i, \flat, \varepsilon) %
:= \{\auf i_0, \mbox{\tt \#} \conc \flat\zu\}$.
Also, we put $\delta^{\GM}(q, Z, x) := \delta^{\GK}(q,Z,x)$
for $Z \neq \flat$ and $\delta^{\GM}(q, \flat, \varepsilon) :=
\{\auf q_f, \varepsilon\zu\}$, as well as
$\delta^{\GM}(q, \flat, x) := \varnothing$ for $x \in A$.
Further, $\delta^{\GM}(q_f, Z, x) := \varnothing$.
This defines $\delta^{\GM}$. Now consider an
$\vec{x} \in L^s(\GK)$. There is a $\GK$--computation
$\auf i_0, \mbox{\tt \#}\zu \stackrel{\vec{x}}{\pf} \auf p, \varepsilon\zu$
for some $p$. Then there exists an $\GL$--computation
%%
\begin{equation}
\auf q_i, \flat\zu \stackrel{\vec{x}}{\pf} \auf p, \flat\zu
\stackrel{\varepsilon}{\pf} \auf q_f, \varepsilon\zu
\end{equation}
%%
Hence $\vec{x} \in L(\GM)$. Conversely, let $\vec{x} \in L(\GM)$.
Then there exists an $\GL$--com\-pu\-ta\-tion $\auf q_i, \flat\zu
\stackrel{\vec{x}}{\pf} \auf q_f, \vec{d}\zu$ for some $\vec{d}$.
One can see quite easily that $\vec{d} = \varepsilon$. Further,
this computation factors as follows.
%%
\begin{equation}
\auf q_i, \flat\zu \stackrel{\varepsilon}{\pf}
\auf i_0, \mbox{\tt \#} \conc \flat\zu \stackrel{\vec{x}}{\pf}
\auf p, \flat\zu \stackrel{\varepsilon}{\pf}
\auf q_f, \varepsilon\zu
\end{equation}
%%
Here $p \in Q$, whence $p \neq q_f, q_i$. But every
$\GM$--transition from $i_0$ to $p$ is also a $\GK$--transition.
Hence there is a $\GK$--computation
$\auf i_0, \mbox{\tt \#}\zu \stackrel{\vec{x}}{\pf}
\auf p, \varepsilon\zu$. From this follows $\vec{x} \in L^s(\GK)$,
and so $L^s(\GK) = L(\GM)$.
\proofend
%%
\begin{lem}
Let $L$ be a CFL over $A$. Then there exists
a pushdown automaton $\GK$ such that $L = L^s(\GK)$.
\end{lem}
%%
\proofbeg
We take a CFG $G = \auf \mbox{\tt S}, N, A, R\zu$
in Greibach Form with $L = L(G)$. We assume that
$\varepsilon \not\in G$. (If $\varepsilon \in L(G)$, then we 
construct an automaton for $L(G) - \{\varepsilon\}$ and then
modify it slightly.) The automaton possesses only one state, $i_0$,
and uses $N$ as its stack alphabet. The beginning of the stack
is {\tt S}.
%%
\begin{equation}
\delta(i_0, X, x) := \{\auf i_0, \vec{Y}\zu :
X \pf x \conc \vec{Y} \in R\}
\end{equation}
%%
This defines $\GK := \auf \{i_0\}, i_0, A, \{i_0\},
N, \mbox{\tt S}, \delta\zu$. We show that $L = L^s(\GK)$. To
this end recall that for every $\vec{x} \in L(G)$ there is a
leftmost derivation. In a grammar in Greibach Form every
leftmost derivation derives strings of the form
$\vec{y} \conc \vec{Y}$. Now one shows by induction that
$G \vdash \vec{y} \conc \vec{Y}$ iff $\auf i_0, %
\vec{Y}\zu \in \delta(i_0, \mbox{\tt S}, \vec{y})$.
\proofend
%%
\begin{lem}
\label{prop:stapel}
Let $\GK$ be a pushdown automaton. Then $L^s(\GK)$
is context free.
\end{lem}
%%
\proofbeg
Let $\GK = \auf Q, i_0, A, F, D, \mbox{\tt \#}, \delta\zu$ be 
a pushdown automaton. We may assume that it is simple. Put 
$N := Q \times D \times (Q \cup \{\mbox{\tt S}\})$,
where \mbox{\tt S} is a new symbol. \mbox{\tt S} shall also be the
start symbol. We write a general element of $N$ in the form
$[q,A,p]$. Now we define $R := R^s \cup R^0 \cup R^{\delta} \cup %
R^{\varepsilon}$, where
%%
\begin{equation}
\label{eq:gvona}
\begin{array}{lll}
R^s & := \{\mbox{\tt S} \pf [i_0, \mbox{\tt \#}, q] :\! 
	&\! q \in Q\} \\
R^0 & := \{[p, Z, q] \pf x :\! 
	& \! \auf r,\varepsilon\zu \in \delta(p,Z,x)\} 
\\
R^{\delta} & := \{[p, Z, q] \pf x [r, Y, q] :\! 
	& \! \auf r, Y\zu \in \delta(p,Z,x)\} \\
R^{\varepsilon} & := \{[p, Z, q] \pf [p', X, r] [r, Y, q] :\! 
	& \! \auf p', XY\zu \in \delta(p,Z,\varepsilon)\}
\end{array}
\end{equation}
%%
The grammar thus defined is called $G(\GA)$. We claim that for every 
$\vec{x} \in A^{\ast}$, every $p, q \in Q$ and every $Z \in D$
%%
\begin{equation}
\label{eq:spr}
[p, Z, q] \vdash_G \vec{x} \quad \Dpf \quad
\auf q, \varepsilon\zu \in \delta(p, Z, \vec{x})
\end{equation}
%%
This suffices for the proof. For if $\vec{x} \in L(G)$ then we have 
$[i_0, \mbox{\tt \#}, q] \vdash_G \vec{x}$ and so because of 
\eqref{eq:spr} $\auf q, \varepsilon\zu \in \delta(i_0, \mbox{\tt \#}, %
\vec{x})$, which means nothing but $\vec{x} \in L^s(\GK)$. And if the
latter holds then we have $[i_0, \mbox{\tt \#}, q] \vdash_G \vec{x}$ and
so $\mbox{\tt S} \vdash_G \vec{x}$, which is nothing else but
$\vec{x} \in L(G)$.

Now we show \eqref{eq:spr}. It is clear that \eqref{eq:spr} 
follows from \eqref{eq:spr2}.
%%
\begin{multline}
\label{eq:spr2}
[p, Z, q] \vdash^{\ell}_G \vec{y} \conc
[q_0, Y_0, q_1][q_1, Y_1, q_2]\dotsb [q_{m-1},Y_{m-1},q] \\
	\qquad \Dpf \qquad
\auf q_0, Y_0Y_1\dotsb Y_{m-1}\zu \in \delta(p, Z, \vec{y})
\end{multline}
%%
\eqref{eq:spr2} is proved by induction.
\proofend
%%

%%
On some reflection it is seen that for every automaton $\GK$ there
is an automaton $\GL$ with only one accepting state which accepts
the same language. If one takes $\GL$ in place of $\GK$ then there
is no need to use the trick with a new start symbol. Said in
another way, we may choose $[i_0, \mbox{\tt \#}, q]$ as a start symbol where
$q$ is the accepting state of $\GL$.
%%
\nocite{chomsky:pushdown}
%%
\begin{thm}[Chomsky]
The CFLs are exactly the languages
which are accepted by a pushdown automaton, either by
state or by stack.
\end{thm}
%%
From this proof we can draw some further conclusions.
The first conclusion is that for every pushdown automaton
$\GK$ we can construct a pushdown automaton $\GL$ for which
$L^s(\GL) = L^s(\GK)$ and which contains no $\varepsilon$--moves.
Also, there exists a pushdown automaton $\GM$ such that
$L^s(\GM) = L^s(\GK)$ and which contains only one state, which is
at the same time an initial and an accepting state. For such
an automaton these definitions reduce considerably. Such an
automaton possesses as a memory only a string. The transition
function can be reduced to a function $\zeta$ from
$A \times D^{\ast}$ into finite subsets of $D^{\ast}$.
(We do not allow $\varepsilon$--transitions.)

The pushdown automaton runs along the string from left to right.
It recognizes in linear time whether or not a string is in the
language. However, the automaton is nondeterministic. 
%%
\begin{defn}
%%%
\index{pushdown automaton!deterministic}%%
\index{language!context free deterministic}%%
%%%
A pushdown automaton $\GK = \auf Q, i_0, A, F, D, \mbox{\tt \#}, %
\delta\zu$ is \textbf{deterministic} if for every $q \in Q$, $Z \in D$ 
and $x \in A_{\varepsilon}$ we have $|\delta(q, Z, x)| \leq 1$ and 
for all $q \in Q$ and all $Z \in D$
either (a) $\delta(q,Z,\varepsilon) = \varnothing$ or
(b) $\delta(q,Z,a) = \varnothing$ for all $a \in A$.
A language $L$ is called \textbf{deterministic} if
$L = L(\GA)$ for a deterministic automaton $\GA$.
%%%
\index{$\Delta$}%%
%%%%
The set of deterministic languages is denoted by $\Delta$.
\end{defn}
%%
Deterministic languages are such languages which are accepted
by a deterministic automaton by state. Now, is it possible to
build a deterministic automaton accepting that language just
like regular languages? The answer is negative. To this end we
consider the \textbf{mirror language} $\{\vec{x}\, \vec{x}^T :
\vec{x} \in A^{\ast}\}$.
%%%
\index{language!mirror}%%
%%%
This language is surely context free. There are, however, no
deterministic automata that accept it. To see this one has to
realize that the automaton will have to put into the stack the
string $\vec{x} \, \vec{x}^T$ at least up to $\vec{x}$ in order
to compare it with the remaining word, $\vec{x}^T$. The machine,
however, has to guess when the moment has come to change from
putting onto stack to removing from stack. The reader may reflect
that this is not possible without knowing the entire word.
%%
\begin{thm}
\label{thm:dtime}
Deterministic languages are in $\textbf{DTIME}(n)$.
\end{thm}
%%
The proof is left as an exercise.

We have seen that also regular languages are in
$\mathbf{DTIME}(n)$. However, there are deterministic languages
which are not regular. Such a language is
$L = \{\vec{x}\mbox{\tt c}\vec{x}^T : \vec{x} \in \{\mbox{\tt a},
\mbox{\tt b}\}^{\ast}\}$. In contrast to the mirror language $L$
is deterministic. For now the machine does not have to guess
where the turning point is: it is right after the symbol
{\tt c}.

Now there is the question whether a deterministic automaton
can recognize languages using the stack. This is not the case.
For let $L = L^s(\GK)$, for some deterministic automaton $\GK$. 
Then, if $\vec{x}\, \vec{y} \in L$ for some $\vec{y} \neq \varepsilon$ 
then $\vec{x} \not\in L$. We say that $L$ is \textbf{prefix free} 
if it has this property.
%%%
\index{language!prefix free}%%
%%%
For if $\vec{x} \in L$ then there exists a  $\GK$--computation
from $\auf q_0, \mbox{\tt \#}\zu$ to $\auf q, \varepsilon\zu$. Further,
since $\GK$ is deterministic: if $\auf q_0,
\mbox{\tt \#}\zu \stackrel{\vec{x}}{\pf} \Gz$ then $\Gz = \auf q,
\varepsilon\zu$. However, if the stack has been emptied
the automaton cannot work further. Hence $\vec{x}\, \vec{y}
\not\in L$. There are deterministic languages which are not
prefix free. We present an important class of such languages,
the {\it Dyck--languages}. Let $A$ be an alphabet. For each $x \in A$ 
let $\uli{x}$ be another symbol. We write $\uli{A} := \{\uli{x} : x \in A\}$.
We introduce a congruence $\theta$ on $\GZ(A \cup \uli{A})$. It is 
generated by the equations
%%
\begin{equation}
a\uli{a}\; \theta\; \varepsilon
\end{equation}
%%
for all $a \in A$.
(The analogous equations $\uli{a}a\; \theta\; \varepsilon$ are
{\it not\/} included.) A string $\vec{x} \in (A \cup \uli{A})^{\ast}$
is called \textbf{balanced} if $\vec{x}\; \theta\; \varepsilon$.
$\vec{x}$ is balanced iff $\vec{x}$ can be rewritten
into $\varepsilon$ by successively replacing substrings of the
form $x\uli{x}$ into $\varepsilon$.
%%
\begin{defn}
%%%
\index{language!Dyck--}%%
%%%
$\textbf{D}_r$ denotes the set of balanced strings over
an alphabet consisting of $2r$ symbols.  A language is
called a \textbf{Dyck--language} if it has the form
$D_r$ for some $r$ (and some alphabet $A \cup \uli{A}$).
\end{defn}
%%
The language XML 
%%%
\index{XML}%%%
%%%
(\textbf{Extensible Markup Language}, an outgrowth
of HTML) embodies like no other language the features of
Dyck--languages. For every string $\vec{x}$ it allows to form a pair
of tags $\mbox{\tt <}\vec{x}\mbox{\tt >}$ (opening tag) and
$\mbox{\tt </}\vec{x}\mbox{\tt >}$ (closing tag). The syntax of
XML is such that the tags always come in pairs. The
tags alone (not counting the text in between) form a Dyck Language.
What distinguishes XML from other languages is that tags can be
freely formed.
%%
\begin{prop}
Dyck--languages are deterministic but not prefix free.
\end{prop}
%%
The following grammars generate the Dyck--languages:
%%
\begin{equation}
\mbox{\tt S} \pf \mbox{\tt SS} \mid x\mbox{\tt S}\uli{x}
\mid \varepsilon
\end{equation}
%%
Dyck--languages are therefore context free. It is easy to see
that together with $\vec{x}, \vec{y} \in D_r$ also
$\vec{x}\vec{y} \in D_r$. Hence Dyck--languages are not prefix
free. That they are deterministic follows from some general
results which we shall establish later. We leave it to the
reader to construct a deterministic automaton which recognizes
$D_r$. This shows that the languages which are accepted by
a deterministic automaton by empty stack are a proper subclass
of the languages which are accepted by an automaton by
state.  This justifies the following definition.
%%
\begin{defn}
%%%
\index{language!strict deterministic}%%
\index{$\Delta^s$}%%
%%%
A language $L$ is called \textbf{strict deterministic} if there
is a deterministic automaton $\GK$ such that
$L = L^s(\GK)$. The class of strict deterministic languages
is denoted by $\Delta^s$.
\end{defn}
%%
\begin{thm}
\label{thm:praefixfrei}
$L$ is strict deterministic if $L$ is
deterministic and prefix free.
\end{thm}
%%
\proofbeg
We have seen that strict deterministic languages are prefix free.
Now let $L$ be deterministic and prefix free. Then there exists an
automaton  $\GK$ which accepts $L$ by state. Since $L$ is prefix
free, this holds for every $\vec{x} \in L$, and for every proper
prefix $\vec{y}$ of $\vec{x}$ we have that if $\auf q_0, \mbox{\tt \#}\zu
\stackrel{\vec{y}}{\pf} \auf q, \vec{Y}\zu$ then $q$ is not an
accepting state. Thus we shall rebuild $\GK$ in the following way.
Let $\delta_1(q,Z,x) := \delta^{\GK}(q,Z,x)$ if $q$ is not
accepting. Further, let $\delta_1(q,Z,x) := \varnothing$
if $q \in F$ and $x \in A$; let $\delta_1(q,Z,\varepsilon) :=
\{\auf q,\varepsilon\zu\}$, $Z \in D$. Finally, let $\GL$ be
the automaton which results from $\GK$ by replacing
$\delta^{\GK}$ with $\delta_1$. $\GL$ is deterministic as
is easily checked. Further, an $\GL$--computation can be
factored into an $\GK$--computation followed by a deletion
of the stack. We claim that $L(\GK) = L^s(\GL)$. The claim
then follows. So let $\vec{x} \in L(\GK)$.
Then there exists a $\GK$--computation using $\vec{x}$
from $\auf q_0, \mbox{\tt \#}\zu$ to $\auf q, \vec{Y}\zu$ where
$q \in F$. For no proper prefix $\vec{y}$ of $\vec{x}$ there is
a computation into an accepting state since $L$ is prefix
free. So there is an $\GL$--computation with $\vec{x}$ from
$\auf q_0, \mbox{\tt \#}\zu$ to $\auf q, \vec{Y}\zu$. Now 
$\auf q, \vec{Y}\zu \stackrel{\varepsilon}{\pf} \auf q, \varepsilon\zu$
and so $\vec{x} \in L^s(\GL)$. Conversely, assume
$\vec{x} \in L^s(\GL)$. Then there is a computation
$\auf q_0, \mbox{\tt \#}\zu \stackrel{\vec{x}}{\pf}
\auf q, \varepsilon\zu$. Let $\vec{Y} \in D^{\ast}$
be the longest string such that
$\auf q_0, \mbox{\tt \#}\zu \stackrel{\vec{x}}{\pf}
\auf q, \vec{Y}\zu$. Then the $\GL$--step before reaching
$\auf q, \vec{Y}\zu$ is a $\GK$--step. So there is a
$\GK$--computation for $\vec{x}$ from $\auf q_0, \mbox{\tt \#}\zu$ to
$\auf q,\vec{Y}\zu$, and so $\vec{x} \in L(\GK)$.
\proofend
%%

The proof of this theorem also shows the following.
%%
\begin{thm}
\label{thm:prffrei}
Let $U$ be a deterministic CFL.
Let $L$ be the set of all $\vec{x} \in U$ for which
no proper prefix is in $U$. Then $L$ is strict
deterministic.
\end{thm}
%%
For the following definition we make the following agreement,
which shall be used quite often in the sequel.
%%%
\index{$^{(k)}\vec{x}$}%%%
%%%%
We denote by ${^{(k)}\vec{\alpha}}$ the prefix of $\vec{\alpha}$
of length $k$ in case $\vec{\alpha}$ has length at least
$k$; otherwise ${^{(k)}\vec{\alpha}} := \vec{\alpha}$.
%%
\begin{defn}
%%%
\index{partition!strict}%%
\index{$\equiv$}%%
%%%
Let $G = \auf \mbox{\tt S}, N, A, R\zu$ be a grammar and
$\Pi \subseteq \wp(N \cup A)$ a partition. We write
$\alpha \equiv \beta$ if there is an $M \in \Pi$ such that
$\alpha, \beta \in M$. $\Pi$ is called \textbf{strict for G}
if the following holds.
%%
\begin{dingautolist}{192}
\item
    $A \in \Pi$
\item
    For $C, C' \in N$ and $\vec{\alpha}, \vec{\gamma}_1,
    \vec{\gamma}_2 \in (N \cup A)^{\ast}$: if $C \equiv C'$
    and $C \pf \vec{\alpha}\, \vec{\gamma}_1$ as well as
    $C' \pf \vec{\alpha}\, \vec{\gamma}_2 \in R$ then either
        \begin{enumerate}
        \item
            $\vec{\gamma}_1, \vec{\gamma}_2 \neq \varepsilon$ and
            ${^{(1)}\vec{\gamma}_1} \equiv {^{(1)}\vec{\gamma}_2}$
            or
        \item
            $\vec{\gamma}_1 = \vec{\gamma}_2 = \varepsilon$ and
            $C = C'$.
        \end{enumerate}
\end{dingautolist}
\end{defn}
%%
\begin{defn}
%%%
\index{grammar!strict deterministic}%%
%%%
A CFG $G$ is called \textbf{strict deterministic}
if there is a strict partition for $G$.
\end{defn}
%%
We look at the following example
(taken from \cite{harrison:formal}):
%%
\begin{equation}
\begin{array}{l@{\quad\pf\quad}l@{\qquad}l@{\quad\pf\quad}l}
\mbox{\tt S} & \mbox{\tt aA} \mid \mbox{\tt aB} &
	\mbox{\tt C} & \mbox{\tt bC} \mid \mbox{\tt a} \\
\mbox{\tt A} & \mbox{\tt aAa} \mid \mbox{\tt bC} &
	\mbox{\tt D} & \mbox{\tt bDc} \mid \mbox{\tt c} \\
\mbox{\tt B} & \mbox{\tt aB} \mid \mbox{\tt bD} 
\end{array}
\end{equation}
%%
$\Pi = \{\{\mbox{\tt a}, \mbox{\tt b}, \mbox{\tt c}\},
\{\mbox{\tt S}\}, \{\mbox{\tt A}, \mbox{\tt B}\},
\{\mbox{\tt C}, \mbox{\tt D}\}\}$ is a strict partition.
The language generated by this grammar is
$\{\mbox{\tt a}^n \mbox{\tt b}^k \mbox{\tt a}^n,
\mbox{\tt a}^n \mbox{\tt b}^k \mbox{\tt c}^k : k, n \geq 1\}$.

We shall now show that the languages generated by strict
deterministic grammars are exactly the strict deterministic
languages. This justifies the terminology in retrospect.
To begin, we shall draw a few conclusions from the definitions.
If $G = \auf \mbox{\tt S}, N, A, R\zu$ is strict deterministic
and $R' \subseteq R$ then $G' = \auf \mbox{\tt S}, N, A, R'\zu$
is strict deterministic as well. Therefore, for a strict
deterministic grammar we can construct a weakly equivalent
strict deterministic slender grammar. We denote by
%%%
\index{$\vec{\alpha} \Pf^n_L \vec{\gamma}$}%%%
%%
$\vec{\alpha} \Pf^n_L \vec{\gamma}$
the fact that there is a leftmost derivation of length $n$ of
$\vec{\gamma}$ from $\vec{\alpha}$.
%%
\begin{lem}
\label{lem:partition}
Let $G$ be a CFG with a strict partition $\Pi$.
Then the following is true.
    For $C, C' \in N$ and $\vec{\alpha}, \vec{\gamma}_1,
    \vec{\gamma}_2 \in (N \cup A)^{\ast}$: if $C \equiv C'$
    and $C \Pf^n_L \vec{\alpha}\, \vec{\gamma}_1$ as well as
    $C' \Pf^n_L \vec{\alpha}\, \vec{\gamma}_2$ then either
        \begin{dingautolist}{192}
        \item
            $\vec{\gamma}_1, \vec{\gamma}_2 \neq \varepsilon$ and
            ${^{(1)}\vec{\gamma}_1} \equiv {^{(1)}\vec{\gamma}_2}$
            or
        \item
            $\vec{\gamma}_1 = \vec{\gamma}_2 = \varepsilon$ and
            $C = C'$.
        \end{dingautolist}
\end{lem}
%%
The proof is an easy induction over the length of the derivation.
%%
\begin{lem}
\label{lem:linksrekursiv}
Let $G$ be slender and strict deterministic. Then if
$C \Pf_L^+ D\, \vec{\alpha}$ we have $C \not\equiv D$.
\end{lem}
%%
\proofbeg
Assume $C \Pf^n_L D\, \vec{\alpha}$. Then because of
Lemma~\ref{lem:partition} we have for all $k \geq 1$:
$C \Pf^{kn}_L D\, \vec{\gamma}$ for some $\vec{\gamma}$.
From this it follows that there is no terminating leftmost
derivation from $C$. This contradicts the fact that $G$ is
slender.
\proofend
%%

It follows that a strict deterministic grammar is not left 
recursive, that is, $A \Pf_L^+ A\, \vec{\alpha}$ cannot hold. 
We can construct a Greibach Normal Form for $G$ in the following way. Let
$\rho = C \pf \alpha \, \vec{\gamma}$  be a rule. If
$\alpha \not\in A$ then we skip $\rho$ by replacing it with the
set of all rules $C \pf \vec{\eta}\, \vec{\gamma}$ such that
$\alpha \pf \vec{\eta} \in R$.  Then Lemma~\ref{lem:partition}
assures us that $\Pi$ is a strict partition also for the new
grammar. This operation we repeat as often as necessary. Since
$G$ is not left recursive, this process terminates.
%%
\begin{thm}
For every strict deterministic grammar $G$ there is a strict 
deterministic grammar $H$ in Greibach Normal Form such that
$L(G) = L(H)$.
\end{thm}
%%
Now for the promised correspondence between strict deterministic
languages and strict deterministic grammars.
%%
\begin{lem}
Let $L$ be strict deterministic. Then there exists a
deterministic automaton with a single accepting state
which accepts $L$ by stack.
\end{lem}
%%
\proofbeg
Let $\GA$ be given. We add a new state $q$ into which the
automaton changes as soon as the stack is empty.
\proofend
%%
\begin{lem}
Let $\GA$ be a deterministic automaton with a single accepting
state. Then $G(\GA)$ is strict deterministic.
\end{lem}
%%
\proofbeg
Let $\GA = \auf Q, i_0, A, F, D, \mbox{\tt \#}, \delta\zu$. By the preceding
lemma we may assume that $F = \{q_f\}$. Now let $G(\GA)$ defined as in
\eqref{eq:gvona}. Put
%%
\begin{equation}
\alpha \equiv \beta \quad:\Dpf\quad
\left\{
    \begin{array}{ll}
                        & \alpha, \beta \in A \\
    \text{ or}   & \alpha = [q,Z,q'], \beta = [q,Z,q''] \\
    & \quad \text{for some } q, q', q'' \in Q, Z \in D.
    \end{array}
\right.
\end{equation}
%%
We show that $\equiv$ is a strict partition. To this end,
let $[q,Z,q'] \pf \vec{\alpha}\vec{\gamma}_1$ and
$[q,Z,q''] \pf \vec{\alpha}\vec{\gamma}_2$ be two rules.
Assume first of all $\vec{\gamma}_1,
\vec{\gamma}_2 \neq \varepsilon$.
Case 1. $\vec{\alpha} = \varepsilon$. Consider
$\zeta_i := {^{(1)}\vec{\gamma}_i}$. If $\zeta_1 \in A$
then also $\zeta_2 \in A$, since $\GA$ is deterministic.
If on the other hand $\zeta_1 \not\in A$ then we have
$\zeta_1 = [q, Y_0, q_1]$ and
$\zeta_2 = [q, Y_0, q_1']$, and so $\zeta_1 \equiv \zeta_2$.
Case 2. $\vec{\alpha} \neq \varepsilon$. Let then
$\eta := {^{(1)}\vec{\alpha}}$. If $\eta \in A$, then we now have
$\zeta_1 = [q_i, Y_i, q_{i+1}]$ and
$\zeta_2 = [q_i, Y_i, q_{i+1}']$ for some
$q_i, q_{i+1}, q_{i+1}' \in Q$. This completes this case.

Assume now $\vec{\gamma}_1 = \varepsilon$.
Then $\vec{\alpha}\, \vec{\gamma}_1$ is a prefix of
$\vec{\alpha}\, \vec{\gamma}_2$.
Case 1. $\vec{\alpha} = \varepsilon$. Then
$\vec{\alpha}\, \vec{\gamma}_2 = \varepsilon$,
hence $\vec{\gamma}_2 = \varepsilon$.
Case 2. $\vec{\alpha} \neq \varepsilon$.
Then it is easy to see that $\vec{\gamma}_2 =
\varepsilon$. Hence in both cases we have
$\vec{\gamma}_2 = \varepsilon$, and so $q' = q''$.
This shows the claim.
\proofend
%%
\begin{thm}
Let $L$ be a strict deterministic language. Then there exists
a strict deterministic grammar $G$ such that $L(G) = L$.
\end{thm}
%%
%%
The strategy to put a string onto the stack and then subsequently
remove it from there has prompted the following definition.
%%%
\index{stack move}%%
\index{turn}%%
%%%
A \textbf{stack move} is a move where the machine writes a symbol
onto the stack or removes a symbol from the stack. (So the stack
either increases in length or it decreases.) The automaton is said
to make a \textbf{turn} if in the last stack move it increased the
stack and now it decreases it or, conversely, in the last stack
move it diminishes the stack and now increases it.
%%
\begin{defn}
%%%
\index{language!$n$--turn}%%
\index{language!ultralinear}%%
%%%
A language $L$ is called an $n$--\textbf{turn language} if there is
a pushdown automaton which recognizes every string from $L$
with at most $n$ turns. $L$ is \textbf{ultralinear} if it is an
$n$--turn language for some $n \in \omega$.
\end{defn}
%%
Notice that a CFL is $n$--turn exactly if there
is an automaton which accepts $L$ and in which for every
string $\vec{x}$ {\it every\/} computation needs at most
$n$ turns. For given any automaton $\GK$ which recognizes
$L$, we build another automaton $\GL$ which has the same
computations as $\GK$ except that they are terminated before
the $n+1$st turn. This is achieved by adding a memory that
counts the number of turns.

We shall not go into the details of ultralinear languages.
One case is worth noting, that of 1--turn languages.
%%%
\index{language!linear}%%
\index{grammar!linear}%%
%%%
A CFG is called \textbf{linear} if in every rule
$X \pf \vec{\alpha}$ the string $\vec{\alpha}$ contains at most one
occurrence of a nonterminal symbol. A language is \textbf{linear}
if it is generated by a linear grammar.
%%
\begin{thm}
A CFL is linear iff it is
1--turn.
\end{thm}
%%
\proofbeg
Let $G$ be a linear grammar. Without loss of generality
a rule is of the form $X \pf aY$ or $X \pf Ya$. Further,
there are rules of the form $X \pf \varepsilon$. We construct
the following automaton. $D := \{\mbox{\tt \#}\} \cup N$, where 
{\tt \#} is the beginning of the stack, $Q := \{+, -, q\}$,
$i_0 := +$, $F := \{q\}$. Further, for
$x \in A$ we put $\delta(+, X, x) := \{\auf +, Y\zu\}$
if $X \pf xY \in R$ and $\delta(+, X, \varepsilon) := 
\{\auf +, Y\zu\}$ if $X \pf Yx \in R$; let
$\delta(-, Y, x) := \{\auf -, \varepsilon\zu\}$
if $X \pf Yx \in R$. And finally $\delta(\auf +, X, x\zu) := 
\{\auf -, \varepsilon\zu\}$ if $X \pf x \in R$. Finally, 
$\delta(-,\mbox{\tt \#},\varepsilon) := \{\auf q, \varepsilon\zu\}$.  
This defines the automaton $\GK(G)$. It is not hard to show that 
$\GK(G)$ only admits computations
without stack moves. For if the automaton is in state 
$+$ the stack may not decrease unless the automaton changes
into the state $-$. If it is in $-$, the stack may not increase 
and it may only be changed into a state $-$, or, finally, into 
$q$. We leave it to the reader to check that
$L(\GK(G)) = L(G)$. Therefore $L(G)$ is a 1--turn language.
Conversely, let $\GK$ be an automaton which allows computations
with at most one turn. It is then clear that if the stack is
emptied the automaton cannot put anything on it.
The automaton may only fill the stack and later empty it.
Let us consider the automaton $G(\GK)$ as defined above.
Then all rules are of the form $X \pf x\vec{Y}$ with
$x \in A_{\varepsilon}$. Let $\vec{Y} =
Y_0 Y_1 \dotsb Y_{n-1}$. We claim that every
$Y_i$--production for $i > 0$ is of the form $Y_i \pf a$
or $Y_i \pf X$. If not,  there is a computation in which
the automaton makes two turns, as we have indicated above.
(This argument makes tacit use of the fact that the automaton
possesses a computation where it performs a transition to
$Y_i = [p,X,q]$ that is to say, that it goes from
$p$ to $q$ where $X$ is the topmost stack symbol. If this
is not the case, however, then the transitions can be
eliminated without harm from the automaton.) Now it is easy
to eliminate the rules of the form $Y_i \pf X$ by skipping them.
Subsequent skipping of the rules $Y_i \pf a$ yields a
linear grammar.
\proofend

The automata theoretic analyses suggest that the recognition
problem for CFLs must be quite hard.
However, this is not the case. It turns out that the recognition
and parsing problem are solvable in $O(n^3)$ steps. To see this,
let a grammar $G$ be given.  We assume without loss of generality
that $G$ is in Chomsky Normal Form.  Let $\vec{x}$ be a string of
length $n$. As a first step we try to list all substrings which
are constituents, together with their category. If $\vec{x}$ is a
constituent of category $S$ then $\vec{x} \in L(G)$; if it is not,
then $\vec{x} \not\in L(G)$. In order to enumerate the substrings
we use an $(n+1) \times (n+1)$--matrix whose entries
are subsets of $N$. Such a matrix is called a \textbf{chart}.
%%%
\index{chart}%%
%%%
Every substring is defined by a pair $\auf i,j\zu$ of numbers,
where $0 \leq i < j \leq n + 1$. In the cell $\auf i,j\zu$ we
enter all $X \in N$ for which the substring $x_i x_{i+1} \dotsb x_{j-1}$
is a constituent of category $X$. In the beginning the matrix is
empty. Put $d := i - j$. Now we start by filling the matrix 
starting at $d = 1$ and counting up to $d = n$. For each $d$, 
we go from $i = 0$ until $i = n - d$. So, we begin
with $d = 1$ and compute for $i = 0$, $i = 1$, $i = 2$ and so on. 
Then we set $d := 2$ and compute for $i = 0$, $i = 1$ etc. We 
consider the pair $\auf d, i\zu$. The substring
$x_i \dotsb x_{i+d}$ is a constituent of category $X$ iff
it decomposes into substrings $\vec{y} = x_i \dotsb x_{i+e}$ and
$\vec{z} = x_{i+e+1} \dotsb x_{i+d}$ such that there is a rule
$X \pf Y Z$ where $\vec{y}$ is a constituent of category $Y$ and
$\vec{z}$ is a constituent of category $Z$. This means that the set of
all $X \in N$ which we enter at $\auf i, i+d\zu$ is computed from
all decompositions into substrings. There are $d - 1 \leq n$
such decomposition. For every decomposition the computational
effort is limited and depends only on a constant $c_G$ whose
value is determined by the grammar. For every pair we need
$c_G \cdot (n+1)$ steps. Now there exist ${n \choose 2}$ proper
subwords. Hence the effort is bounded by $c_G \cdot n^3$.

In Figure~\ref{fig:chart} we have shown the computation of a
chart based on the word {\tt abaabb}. Since the grammar is
invertible any substring has at most one category. In general,
this need not be the case. (Because of Theorem~\ref{thm:invertierbar}
we can always assume the grammar to be invertible.)
%%
\begin{equation}
\begin{array}{l@{\quad\pf\quad}l}
\mbox{\tt S} & \mbox{\tt SS} \mid \mbox{\tt AB} \mid \mbox{\tt BA} \\
\mbox{\tt A} & \mbox{\tt AS} \mid \mbox{\tt SA} \mid \mbox{\tt a} \\
\mbox{\tt B} & \mbox{\tt BS} \mid \mbox{\tt SB} \mid \mbox{\tt b}
\end{array}
\end{equation}
%%
\begin{figure}
\begin{center}
\begin{picture}(20,15)
\put(2.5,1){\makebox(0,0){\tt a}}
\put(5.5,1){\makebox(0,0){\tt b}}
\put(8.5,1){\makebox(0,0){\tt a}}
\put(11.5,1){\makebox(0,0){\tt a}}
\put(14.5,1){\makebox(0,0){\tt b}}
\put(17.5,1){\makebox(0,0){\tt b}}
%%
\put(2.5,2){\line(-1,1){1.5}}
\put(5.5,2){\line(-1,1){3}}
\put(8.5,2){\line(-1,1){4.5}}
\put(11.5,2){\line(-1,1){6}}
\put(14.5,2){\line(-1,1){7.5}}
\put(17.5,2){\line(-1,1){9}}
\put(19,3.5){\line(-1,1){9}}
%%
\put(1,3.5){\line(1,1){9}}
\put(2.5,2){\line(1,1){9}}
\put(5.5,2){\line(1,1){7.5}}
\put(8.5,2){\line(1,1){6}}
\put(11.5,2){\line(1,1){4.5}}
\put(14.5,2){\line(1,1){3}}
\put(17.5,2){\line(1,1){1.5}}
%%
\put(2.5,3.5){\makebox(0,0){\tt A}}
\put(5.5,3.5){\makebox(0,0){\tt B}}
\put(8.5,3.5){\makebox(0,0){\tt A}}
\put(11.5,3.5){\makebox(0,0){\tt A}}
\put(14.5,3.5){\makebox(0,0){\tt B}}
\put(17.5,3.5){\makebox(0,0){\tt B}}
%%
\put(4,5){\makebox(0,0){\tt S}}
\put(7,5){\makebox(0,0){\tt S}}
\put(10,5){\makebox(0,0){$\varnothing$}}
\put(13,5){\makebox(0,0){\tt S}}
\put(16,5){\makebox(0,0){$\varnothing$}}
%%
\put(5.5,6.5){\makebox(0,0){\tt A}}
\put(8.5,6.5){\makebox(0,0){\tt A}}
\put(11.5,6.5){\makebox(0,0){\tt A}}
\put(14.5,6.5){\makebox(0,0){\tt B}}
%%
\put(7,8){\makebox(0,0){$\varnothing$}}
\put(10,8){\makebox(0,0){\tt S}}
\put(13,8){\makebox(0,0){\tt S}}
%%
\put(8.5,9.5){\makebox(0,0){\tt A}}
\put(11.5,9.5){\makebox(0,0){\tt B}}
%%
\put(10,11){\makebox(0,0){\tt S}}
\end{picture}
\end{center}
\caption{A Chart for {\tt abaabb}}
\label{fig:chart}
\end{figure}
%%
The construction of the chart is as follows.
Let $C_{\vec{x}}(i,j)$ be the set of all nonterminals
$X$ such that $X \vdash_G x_i x_{i+1} \dotsb x_{j-1}$.
Further, for two nonterminals $X$ and $Y$
$X \odot Y := \{Z : Z \pf XY \in R\}$ and for sets
$\BU, \BV \subseteq N$ let
%%%
\index{$X \odot Y$, $\BU \odot \BV$}%%
%%
\begin{equation}
\BU \odot \BV := \bigcup \auf X \odot Y : X \in \BU, Y \in \BV\zu
\end{equation}
%%
Now we can compute $C_{\vec{x}}(i,j)$ inductively. The induction
parameter is $j - i$. If $j - i = 1$ then
$C_{\vec{x}}(i,j) = \{X : X \pf x \in R\}$. If $j - i > 1$ then
the following equation holds.
%%
\begin{equation}
C_{\vec{x}}(i,j) = \bigcup_{i < k < j} C_{\vec{x}}(i,k)
\odot C_{\vec{x}}(k,j)
\end{equation}
%%
We always have $j - k, k - i < j - i$.
Now let $\vec{x} \in L(G)$. How can we find a
derivation for $\vec{x}$? To that end we use the fully computed
chart. We begin with $\vec{x}$ and decompose it in an
arbitrary way; since $\vec{x}$ has the category {\tt S}, there must
be a rule $\mbox{\tt S} \pf XY$ and a decomposition
into $\vec{x}$ of category $X$ and $\vec{y}$ of category
$Y$. Or $\vec{x} = a \in A$ and $\mbox{\tt S} \pf a$ is a rule.
If the composition has been found, then we continue with the
substrings $\vec{x}$ and $\vec{y}$ in the same way.
Every decomposition needs some time, which only depends on
$G$. A substring of length $i$ has $i \leq n$ decompositions.
In our analysis we have at most 2$n$ substrings.
This follows from the fact that in a properly branching
tree with $n$ leaves there are at most $2n$ nodes. In total we
need time at most $d_G \cdot n^2$ for a certain constant
$d_G$ which only depends on $G$.

From this it follows that in general even if the grammar is not
in Chomsky Normal Form the recognition and analysis only needs
$O(n^3)$ steps where at the same time we only need $O(n^2)$ cells.
For let $G$ be given. Now transform $G$ into 2--standard form
into the grammar $G^2$. Since $L(G^2) = L(G)$, the recognition
problem for $G$ is solvable in the same amount of time as $G^2$.
One needs $O(n^2)$ steps to construct a chart for $\vec{x}$.
One also needs an additional $O(n^2)$ steps in order to create
a $G$--tree for $\vec{x}$ and $O(n)$ steps to turn this into a
derivation.

However, this is not already a proof that the problem is
solvable in $O(n^3)$ steps and $O(n^2)$ space, for we need
to find a Turing machine which solves the problem in the
same time and space. This is possible; this has been shown
independently by Cocke, Kasami and Younger.
%%%%
\index{Cocke, J.}%%%
\index{Kasami, Tadao}%%%
\index{Younger, D.~H.}%%%
%%
\nocite{younger:cfg}\nocite{kasami:cfg}
%%
\begin{thm}[Cocke, Kasami, Younger]
\label{thm:cky}
CFLs have the following multitape complexity.
%%
\begin{dingautolist}{192}
\item
$\text{CFL} \subseteq \textbf{DTIME}(n^3)$.
\item
$\text{CFL} \subseteq \textbf{DSPACE}(n^2)$.
\end{dingautolist}
\end{thm}
%%
\proofbeg
We construct a deterministic 3 tape Turing machine which only
needs $O(n^2)$ space and $O(n^3)$ time. The essential trick
consists in filling the tape. Also, in addition to the alphabet 
$A$ we need an auxiliary alphabet consisting of
{\tt B} and {\tt Q} as well as for every $U \subseteq N$ 
a symbol $[U]$ and a symbol $[U]^{\surd}$. On Tape 1 we have
the input string, $\vec{x}$. Put $C(i,j) := C_{\vec{x}}(i,j)$.
Let $\vec{x}$ have length $n$. On Tape 1 we construct a
sequence of the following form.
%%
\begin{equation}
\mbox{\tt Q}\mbox{\tt B}^n \mbox{\tt Q}\mbox{\tt B}^{n-1}
\mbox{\tt Q} \dotsb \mbox{\tt QBBQBQ}
\end{equation}
%%
This is the skeleton of the chart. We call a sequence of
{\tt B}s in between two {\tt Q}s a \textbf{block}. The first block
is being filled as follows. The string $\vec{x}$ is deleted
step by step and the sequence $\mbox{\tt B}^n$ is being replaced
by the sequence of the $C(i,i+1)$. This procedure requires
$O(n^2)$ steps. For every $d$ from $1$ to $n-1$ we shall fill
the $d+1$st block. So, let $d$ be given. On Tape 2 we write
the sequence
%%
\begin{equation}
\begin{array}{l}
\mbox{\tt Q}[C(0,1)][C(0,2)]\dotsb [C(0,d)]\conc \\
\conc \mbox{\tt Q} [C(1,2)][C(1,3)]\dotsb[C(1,d+1)]\conc\dotsb \\
\conc\mbox{\tt Q}[C(n-d,n-d+1)][C(n-d,n-d+2)]\dotsb [C(n-d,n)]
\mbox{\tt Q}
\end{array}
\end{equation}
%%
On Tape 3 we write the sequence
%%
\begin{equation}
\begin{array}{l}
\mbox{\tt Q}[C(0,d)][C(1,d)]\dotsb [C(d-1,d)]\conc \\
\conc \mbox{\tt Q}[C(1,d+1)][C(2,d+1)]\dotsb [C(d,d+1)] \conc \dotsb \\
\conc \mbox{\tt Q} [C(n-d,n)][C(n-d+1,n)]\dotsb [C(n-1,n)]\mbox{\tt Q}
\end{array}
\end{equation}
%%
From this sequence we can compute the $d+1$st block quite fast.
The automaton has to traverse the first block on Tape 2 and the
second block on Tape 3 cogradiently and memorize the result of
$C(0,j) \odot C(j,d+1)$. When it reaches the end it has computed
$C(0,d+1)$ and can enter it on Tape 1. Now it moves on to the
next block on the second and the third tape and computes
$C(1,d+2)$. And so on. It is clear that the computation is linear
in the length of the Tape 2 (and the Tape 3) and therefore needs
$O(n^2)$ time. At the end of this procedure Tape 2 and 3 are
emptied. Also this needs quadratic time. At the end we need to
consider that the filling of Tapes 2 and 3 needs
$O(n^2)$ time. Then for every $d$ the time consumption is
at most $O(n^2)$ and in total $O(n^3)$.
For this we first write {\tt Q} and position the head of Tape 1
on the element $[C(0,1)]$. We write $[C(0,1)]$ onto Tape 2
and $[C(0,1)]^{\surd}$ onto Tape 1. (So, we 
`tick off' the symbol. This helps us to remember what we did.)
Now we advance to $[C(1,2)]$ copy the result onto Tape 2
and replace it by $[C(1,2)]^{\surd}$. And so on. This only
needs linear time; for the symbols $[C(i,i+1)]$ we recognize
because they are placed before the {\tt Q}. If we are ready
we write {\tt Q} onto Tape 2 and move on Tape 1 on to the
beginning and then to the first symbol to the right of
a `ticked off' symbol. This is $[C(1,2)]$. We copy this symbol
onto Tape 2 and tick it off. Now we move on to the next symbol
to the right of the symbol which has been ticked off, copy it
and tick it off. In this way Tape 2 is filled in quadratic
time. At last the symbols that have been ticked off are being
ticked `on', which needs $O(n^2)$ time. Analogously the
Tape 3 is filled.
%%
\proofend
%%
\vplatz
\exercise
Prove Proposition~\ref{prop:einfach}.
%%
\vplatz
\exercise
Prove Theorem~\ref{thm:dtime}. {\it Hint.}
Show that the number of $\varepsilon$--moves
of an automaton $\GA$ in scanning of the string
$\vec{x}$ is bounded by $k_{\GA} \cdot |\vec{x}|$,
where $k_{\GA}$ is a number that depends only on
$\GA$. Now code the behaviour of an arbitrary
pushdown automaton using a 2--tape Turing machine and
show that to every move of the pushdown automaton
corresponds a bounded number of steps of the Turing
machine.
%%
\vplatz
\exercise
Show that a CFL is 0--turn iff it is regular.
%%
\vplatz
\exercise
Give an algorithm to code a chart onto the tape of a Turing
machine.
%%
\vplatz
\exercise
Sketch the behaviour of a deterministic Turing machine
which recognizes a given CFL using $O(n^2)$ space.
%%
\vplatz
\exercise
Show that $\{\vec{w}\, \vec{w}^T : \vec{w} \in A^{\ast}\}$ is
context free but not deterministic.
%%
\vplatz
\exercise
Construct a deterministic automaton  which recognizes a given
Dyck--language.
%%
\vplatz
\exercise
Prove Theorem~\ref{thm:prffrei}.

 \section{Ambiguity, Transparency and Parsing Strategies}
\label{kap2-4}
%
%
%
In this section we will deal with the relationship between strings
and trees. As we have explained in Section~\ref{kap1}.\ref{einsvier}, 
there is a bijective correspondence between derivations in $G$ and
derivations in the corresponding graph grammar $\gamma G$.
Moreover, every derivation $\Delta = \auf A_i : i < p\zu$
of $G$ defines an exhaustively ordered tree $\GB$ with labels in
$N \cup A$ whose associated string is exactly $\vec{\alpha}_{p}$, 
where $A_{p-1} = \auf \vec{\alpha}_{p-1}, C_{p-1}, \vec{\alpha}_p\zu$.
If $\vec{\alpha}_p$ is not a terminal string, the labels of the 
leaves are also not all terminal. We call such a tree a 
\textbf{partial} $G$--\textbf{tree}.
%%%
\index{tree!partial $G$--\faul}%%%
%%
\begin{defn}
%%%
\index{constituent!$G$--\faul}%%%
\index{constituent!accidental}%%
%%%
Let $G$ be a CFG. $\vec{\alpha}$ is called a
$G$--\textbf{constituent of category} $A$ if $A \vdash_G \vec{\alpha}$.
Let $\GB$ be a $G$--tree with associated string $\vec{x}$ and
$\vec{y}$ a substring of $\vec{x}$. Assume further that $\vec{y}$
is a $G$--constituent of category $A$ and $\vec{x} = D(\vec{y})$.
The occurrence $D$ of $\vec{y}$ in $\vec{x}$ is called an
\textbf{accidental G--constituent of category} $A$ 
\textbf{in} $\GB$ if it is not a $G$--constituent of category 
$A$ in $\GB$.
\end{defn}
%%
We shall illustrate this terminology with an example. Let $G$
be the following grammar.
%%
\begin{equation}
\begin{array}{l@{\quad\pf\quad}l}
\mbox{\tt S} & \mbox{\tt SS} \mid \mbox{\tt AB} \mid \mbox{\tt BA} \\
\mbox{\tt A} & \mbox{\tt AS} \mid \mbox{\tt SA} \mid \mbox{\tt a} \\
\mbox{\tt B} & \mbox{\tt BS} \mid \mbox{\tt SB} \mid \mbox{\tt b}
\end{array}
\end{equation}
%%
The string $\vec{x} = \mbox{\tt abaabb}$ has several derivations,
which generate among other the following bracketing analyses.
%%
\begin{equation}
(\mbox{\tt a}(\mbox{\tt b}(\mbox{\tt a}((%
\mbox{\tt ab})\mbox{\tt b})))), \quad
((\mbox{\tt ab})(((\mbox{\tt a}(\mbox{\tt ab}))\mbox{\tt b})))
\end{equation}
%%
We now list all $G$--constituents which occur in $\vec{x}$:
%%
\begin{align}\notag
\mbox{\tt A} & : \mbox{\tt a}, \mbox{\tt aab}, \mbox{\tt aba},
    \mbox{\tt baa}, \mbox{\tt abaab} \\
\mbox{\tt B} & : \mbox{\tt b}, \mbox{\tt abb} \\\notag
\mbox{\tt S} & : \mbox{\tt ab}, \mbox{\tt aabb}, \mbox{\tt abaabb}
\end{align}
%%
Some constituents occur several times, for example
{\tt ab} in $\auf \varepsilon, \mbox{\tt aabb}\zu$ and also in
$\auf \mbox{\tt aba}, \mbox{\tt b}\zu$.
Now we look at the first bracketing, $(\mbox{\tt a}(\mbox{\tt b}%
(\mbox{\tt a}((\mbox{\tt ab})\mbox{\tt b}))))$.
The constituents are {\tt a} (contexts:
$\auf \varepsilon, \mbox{\tt baabb}\zu$, $\auf \mbox{\tt ab},
\mbox{\tt abb}\zu$, $\auf \mbox{\tt aba}, \mbox{\tt bb}\zu$),
{\tt b}, {\tt ab} (for example in the context: $\auf \mbox{\tt aba}, %
\mbox{\tt b}\zu$), {\tt abb} in the context $\auf \mbox{\tt aba}, %
\varepsilon\zu$, {\tt aabb}, {\tt baabb} and {\tt abaabb}. These
are the constituents of the tree. The occurrence $\auf \varepsilon, %
\mbox{\tt aabb}\zu$ of {\tt ab} in {\tt ababb} is therefore an accidental
occurrence of a $G$--constituent of category {\tt S} in that tree.
For although {\tt ab} is a $G$--constituent, this occurrence in the 
tree is not a constituent occurrence of it. Notice that it may happen 
that $\vec{y}$ is a constituent of the tree $\GB$ but that as a 
$G$--constituent of category $C$ it occurs accidentally since its 
category in $\GB$ is $D \neq C$.
%%
\begin{defn}
%%%
\index{grammar!transparent}%%
\index{transparency}%%
\index{grammar!inherently opaque}%%
%%%
A grammar $G$ is called \textbf{transparent} if no $G$--constituent
occurs accidentally in a $G$--string. A grammar which is not
transparent will be called \textbf{opaque}. A language for which
no transparent grammar exists will be called \textbf{inherently
opaque}.
\end{defn}
%%
An example shall illustrate this. For any given signature 
$\Omega$, Polish Notation 
%%%
\index{Polish Notation}%%%
%%%
can be generated by a transparent grammar. 
%%
\begin{equation}
\mbox{\tt S} \pf \mbox{\tt F}_{\Omega(f)} \mbox{\tt S}^{\Omega(f)}
\qquad
\mbox{\tt F}_{\Omega(f)} \pf f 
\end{equation}
%%
\index{$\Pi_{\Omega}$}%%
%%%
This defines the grammar $\Pi_{\Omega}$ for $\PN_{\Omega}$.
Moreover, given a string $\vec{x}$ generated by this grammar, 
the subterm occurrences of $\vec{x}$ under a given analysis are 
in one to one correspondence with the subcontituents of category 
{\tt S}. An occurrence of an $n$--ary function symbol is a 
constituent of type $\mbox{\tt F}_n$. We shall show that this 
grammar is not only unambiguous, it is transparent. 

Let $\vec{x} = x_0 x_1 \dotsb x_{n-1}$ be a string. Then let
$\gamma(\vec{x}) := \sum_{i < n} \gamma(x_i)$,
where for every $f \in F$, $\gamma(f) := \Omega(f) - 1$.
(So, if $\Omega(f) = 0$, $\gamma(f) = -1$.)
The proof of the following is left as an exercise. 
%%
\begin{lem}
\label{lem:zahl}%% 
$\vec{x} \in \PN_{\Omega}$ iff (a) $\gamma(\vec{x}) = -1$ 
and (b) for every proper prefix $\vec{y}$ of $\vec{x}$ we have 
$\gamma(\vec{y}) \geq 0$.
\end{lem}
%%
It follows from this theorem that no proper prefix of a
term is a term. (However, a suffix of a term may again be a term.) 
The constituents are therefore all the substrings that have the 
properties (a) and (b). We show that the grammar is transparent.
Now suppose that $\vec{x}$ contains an accidental occurrence of a 
term $\vec{y}$. Then this occurrence overlaps properly with a 
constituent $\vec{z}$. Without loss of generality $\vec{y} = 
\vec{u} \conc \vec{v}$ and $\vec{z} = \vec{v} \conc \vec{w}$ 
(with $\vec{u}, \vec{w} \neq \varepsilon$). It follows that
$\gamma(\vec{v}) = \gamma(\vec{y}) - \gamma(\vec{u})
< 0$ since $\gamma(\vec{u}) \geq 0$. Hence there exists a
proper prefix  $\vec{u}_1$ of $\vec{u}$ such that
$\vec{u}_1 = -1$. (In order to show this one must first conclude
that the set $P(\vec{x}) := \{\gamma(\vec{p}) : 
\vec{p} \mbox{ is a prefix of } \vec{x}\}$ is a convex set
for every term $\vec{x}$. See Exercise~\ref{ex:PN}.)
%%
\begin{thm}
\label{thm:pn}
The grammar $\Pi_{\Omega}$ is transparent.
\proofend
\end{thm}
%%
Now look at the languages $\mbox{\tt a}^+\mbox{\tt b}$ and
$\mbox{\tt a}^+$. Both are regular. There is a transparent 
regular grammar for $\mbox{\tt a}^+\mbox{\tt b}$. It has the rules 
$\mbox{\tt S} \pf \mbox{\tt aB}$, $\mbox{\tt B} \pf \mbox{\tt AB} 
\mid \mbox{\tt b}$. $\mbox{\tt a}^+$ is on the other hand inherently 
opaque.  For any CFG must generate at least two constituents of the  
form $\mbox{\tt a}^p$ and $\mbox{\tt a}^q$, $q > p$. 
Now there exist two occurrences of $\mbox{\tt a}^p$ in 
$\mbox{\tt a}^q$ which properly overlap. One of them must be
accidental.
%%
\begin{prop}
$\mbox{\tt a}^+$ is inherently opaque.
\proofend
\end{prop}
%%
It can easily be seen that if $L$ is transparent and
$\varepsilon \in L$, then $L = \{\varepsilon\}$.
Also, a language over an alphabet consisting of a single letter
can only be transparent if it contains no more than
a single string. Many properties of CFGs
are undecidable. Transparency is different in this respect.
%%
\nocite{fine:transparency}
\index{Fine, Kit}%%%
%%
\begin{thm}[Fine]
Let $G$ be a CFG. It is decidable whether or not
$G$ is transparent.
\end{thm}
%%
\proofbeg
Let $k$ be the constant from the Pumping Lemma (\ref{thm:pumplemma}). 
This constant can effectively be determined. By Lemma~\ref{lem:opakred} 
there is an accidental occurrence of a constituent iff there is
an accidental occurrence of a right hand side of a production.
These are of the length $p + 1$ where $p$ is the maximum
productivity of a rule from $G$. Further, because of Lemma~\ref{lem:akz}
we only need to check those constituents for accidental occurrences
whose length does not exceed $p^2 + p$. This can be done in
finite amount of time.
\proofend
%%
\begin{lem}
\label{lem:opakred}
$G$ is opaque iff there is a production $\rho =
A \pf \vec{\alpha}$ such that $\vec{\alpha}$ has an accidental
occurrence in a partial $G$--tree.
\end{lem}
%%
\proofbeg
Let $\vec{\phi}$ be a string of minimal length which occurs
accidentally. And let $C$ be an accidental occurrence of
$\vec{\phi}$. Further, let $\vec{\phi} = \vec{\gamma}_1 %
\vec{\alpha} \vec{\gamma}_2$, and let  $A \pf \vec{\alpha}$
be a rule. Then two cases may occur.
(A) The occurrence of $\vec{\alpha}$ is accidental.
Then we have a contradiction to the minimality of
$\vec{\phi}$. (B) The occurrence of $\vec{\alpha}$ is not
accidental. Then $\vec{\eta} := \vec{\gamma}_1 A \vec{\gamma}_2$
also occurs accidentally in $C(\vec{\eta})$! (We can undo
the replacement $A \pf \vec{\alpha}$ in the string
$C(\vec{\varphi})$ since $\vec{\alpha}$ is a constituent.)
Also this contradicts the minimality of $\vec{\phi}$. So,
$\vec{\phi}$ is the right hand side of a production.
\proofend
%%
\begin{lem}
Let $G$ be a CFG without rules of productivity
$-1$ and let $\vec{\alpha}$, $\vec{\gamma}$ be strings. Further, 
assume that $\vec{\gamma}$ is a $G$--constituent of category $A$ 
in which $\vec{\alpha}$ occurs accidentally and in which 
$\vec{\gamma}$ is minimal in the following sense: there is no 
$\vec{\eta}$ of category $A$ with (1) $|\vec{\eta}| < |\vec{\gamma}|$ 
and (2) $\vec{\eta} \vdash_G \vec{\gamma}$ and (3) $\vec{\alpha}$ 
occurs accidentally in $\vec{\eta}$. Then every constituent of 
length $> 1$ overlaps with the accidental occurrence of $\vec{\alpha}$.
\end{lem}
%%
\proofbeg
Let $\vec{\gamma} = \vec{\sigma}_1 \, \vec{\eta}\, \vec{\sigma}_2$,
$|\vec{\eta}| > 1$, and assume that the occurrence of $\vec{\eta}$
is a constituent of category $A$ which does not overlap with
$\vec{\alpha}$. Then $\vec{\alpha}$ occurs accidentally in
$\vec{\delta} := \vec{\sigma}_1\, A\, \vec{\sigma}_2$. Further,
$|\vec{\delta}| < |\vec{\gamma}|$, contradicting the minimality of
$\vec{\gamma}$.
\proofend
%%
\begin{lem}
\label{lem:akz}
Let $G$ be a CFG where the productivity of rules
is at least 0 and at most $p$, and let $\vec{\alpha}$
be a string of length $n$ which occurs accidentally.
Then there exists a constituent $\vec{\gamma}$
of length $\leq n p$ in which $\vec{\alpha}$ occurs accidentally.
\end{lem}
%%
\proofbeg
Let $A \vdash_G \vec{\gamma}$ be minimal in the sense of the previous
lemma. Then we have that every constituent of $\vec{\gamma}$
of length $> 1$ overlaps properly with $\vec{\alpha}$. Hence
$\vec{\gamma}$ has been obtained by at most $n$ applications
of rules of productivity $> 0$. Hence $|\vec{\gamma}| \leq n p$.
\proofend

The property of transparency is stronger than that of unique
readability, also known as unambiguity, which is defined as follows.
%%
\begin{defn}
%%%
\index{grammar!ambiguous}%%
\index{language!inherently ambiguous}%%
%%%
A CFG $G$ is called \textbf{unambiguous} if for every string
$\vec{x}$ there is at most one $G$--tree whose associated string
is $\vec{x}$. If $G$ is not unambiguous, it is called 
\textbf{ambiguous}. A CFL $L$ is called \textbf{inherently
ambiguous} if every CFG generating it is ambiguous.
\end{defn}
%%
\begin{prop}
Every transparent grammar is unambiguous.
\end{prop}
%%
There exist inherently ambiguous languages. Here is an example.
%%
\index{Parikh, Rohit}%%%
\begin{thm}[Parikh]
The language $L$ is inherently ambiguous.
%%%
\begin{equation}
L := \{\mbox{\tt a}^n \mbox{\tt b}^n \mbox{\tt c}^m : n,
m \in \omega\} \cup \{\mbox{\tt a}^m \mbox{\tt b}^n \mbox{\tt c}^n
: n, m \in \omega\}
\end{equation}
\end{thm}
%%
\proofbeg %%
$L$ is context free and so there exists a CFG $G$ such that 
$L(G) = L$. We shall show that $G$ must be ambiguous. There is a 
number $k$ which satisfies the
Pumping Lemma (\ref{thm:pumplemma}). Let $n := k! (:= 
\prod_{i = 1}^{k} i)$. Then there exists a decomposition of 
$\mbox{\tt a}^{2n} \mbox{\tt b}^{2n} \mbox{\tt c}^{3n}$ into
%%
\begin{equation}
\label{eq:decomp}
\vec{u}_1\conc \vec{x}_1\conc \vec{v}_1\conc \vec{y}_1\conc \vec{z}_1
\end{equation}
%%
in such a way that $|\vec{u}_1| \leq k$. Furthermore, we may also 
assume that $|\vec{v}_1| \leq k$.
It is easy to see that $\vec{x}_1\, \vec{y}_1$ may not contain 
occurrences of {\tt a}, {\tt b} and {\tt c} at the same time.
Since it contains {\tt a}, it may not contain {\tt c}. So we 
have $\vec{x}_1 = \mbox{\tt a}^p$ and $\vec{y}_1 = \mbox{\tt b}^p$
for some $p$. We consider a maximal constituent of \eqref{eq:decomp} 
of the form $\mbox{\tt a}^q \, \mbox{\tt b}^{q'}$. Such a constituent 
must exist. ($\vec{x}_1\conc\vec{v}_1\conc\vec{y}_1$ is of that 
form.) In it there is a constituent of the form $\mbox{\tt a}^{q - i} \,
\mbox{\tt b}^{q' - i}$ for some $i < k$. This follows from the
Pumping Lemma. Hence we can pump up $\mbox{\tt a}^i$ and
$\mbox{\tt b}^i$ at the same time and get strings of the form
%%
\begin{equation}
\mbox{\tt a}^{2p + ki}\, \mbox{\tt b}^{2p+ki} \, \mbox{\tt c}^{3q}
\end{equation}
%%
while there exists a constituent of the form $\mbox{\tt a}^{2p + ki - r}\,
\mbox{\tt b}^{2p + ki - s}$ for certain $r, s \leq k$. In particular,
for $k := p/i$ we get
%%
\begin{equation}
\mbox{\tt a}^{3p}\, \mbox{\tt b}^{3p}\, \mbox{\tt c}^{3q}
\end{equation}
%%
Now we form a decomposition of
$\mbox{\tt a}^{3n} \mbox{\tt b}^{2n} \mbox{\tt c}^{2n}$ into
%%
\begin{equation}
\vec{u}_2\conc \vec{x}_2\conc \vec{v}_2\conc \vec{y}_2\conc \vec{z}_2
\end{equation}
%%
in such a way that $|\vec{z}_2|, |\vec{v}_2| \leq k$. Analogously
we get a constituent of the form $\mbox{\tt b}^{2p - s'}\conc 
\mbox{\tt c}^{2p - r'}$ for certain $r', s' \leq k$. These
occurrences overlap. For the left hand constituent contains $3p - s$ many
occurrences of {\tt b} and the right hand constituent contains $3p
- s'$ many occurrences of {\tt b}. Since $3p - s + 3p - s' = 6p -
(s + s') > 3p$, these constituents must overlap. However, they are
not equal. But this is impossible. So $G$ is ambiguous. Since $G$
was arbitrary, $L$ is inherently ambigous.
%%
\proofend
%%

Now we discuss a property which is in some sense the opposite of the 
property of unambiguity. It says that if a right hand side occurs in a
constituent, then under some different analysis this occurrence is
actually a constituent occurrence.
%%
\begin{defn}
%%%
\index{NTS--property}%%
\index{language!NTS--\faul}%%
%%%
A CFG has the \textbf{NTS--pro\-per\-ty} if from 
$C \vdash_G \vec{\alpha}_1 \conc \vec{\beta} \conc \vec{\alpha}_2$ 
and $B \pf \vec{\beta} \in R$ follows: $C \vdash_G \vec{\alpha}_1
\conc B \conc \vec{\alpha}_2$. A language is called
an \textbf{NTS--language} if it has an NTS--grammar.
\end{defn}
%%
The following grammar is not an NTS--grammar.
%%
\begin{equation}
\mbox{\tt X} \pf \mbox{\tt aX}, \quad \mbox{\tt X} \pf \mbox{\tt a}
\end{equation}
%%
For we have $\mbox{\tt X} \vdash \mbox{\tt aa}$ but it does
not hold that $\mbox{\tt X} \vdash \mbox{\tt Xa}$.
In general, regular grammars are not NTS. However, we have
%%
\begin{thm}
All regular languages are NTS--languages.
\end{thm}
%%
\proofbeg
Assume that $L$ is regular. Then there exists a finite state automaton
$\GA = \auf A, Q, q_0, F, \delta\zu$ such that $L = L(\GA)$.
Put $N := \{S^{\star}\} \cup \{L(p,q) : p, q \in Q\}$. 
Further, put $G := \auf \mbox{\tt S}^{\star}, N, A, R\zu$, where 
$R$ consists of 
%%
\begin{equation}
\begin{array}{l@{\quad \pf\quad}ll}
\mbox{\tt S}^{\star} & \mbox{\tt L}(q_0,q) & (q \in F) \\
\mbox{\tt L}(p,q)    & \mbox{\tt L}(p,r) \mbox{\tt L}(r,q) \\
\mbox{\tt L}(p,q)    & a           & (q \in \delta(p, a))
\end{array}
\end{equation}
%%
Then we have $\mbox{\tt L}(p,q) \vdash_G \vec{x}$  iff
$q \in \delta(p,\vec{x})$, as is checked by induction.
From this follows that $\mbox{\tt S}^{\star} \vdash_G \vec{x}$
iff $\vec{x} \in L(\GA)$. Hence we have $L(G) = L$. It
remains to show that $G$ has the NTS--property.  To this end let
$\mbox{\tt L}(p,q) \vdash_G \vec{\alpha}_1 \conc \vec{\beta}
\conc \vec{\alpha}_2$ and $\mbox{\tt L}(r,s) \vdash_G \vec{\beta}$.
We have to show that $\mbox{\tt L}(p,q) \vdash_G \vec{\alpha}_1 %
\conc \mbox{\tt L}(r,s) \conc \vec{\alpha}_2$.  In order to do
this we extend the automaton $\GA$ to an automaton which reads
strings from $N \cup A$. Here $q \in \delta(p, C)$
iff for every string $\vec{y}$ with $C \vdash_G \vec{y}$
we have $q \in \delta(p, \vec{y})$. Then it is clear that
$q \in \delta(p, \mbox{\tt L}(p,q))$. Then it still holds that
$\mbox{\tt L}(p,q) \vdash_G \vec{\alpha}$ iff
$q \in \delta(p, \vec{\alpha})$. Hence we have
$r \in \delta(p, \vec{\alpha}_1)$ and
$q \in \delta(s, \vec{\alpha}_2)$. From this follows that
$\mbox{\tt L}(p,q) \vdash_G \mbox{\tt L}(p,r) \mbox{\tt L}(r,s)
\mbox{\tt L}(s,q)$ and finally
$\mbox{\tt L}(p,q) \vdash_G \vec{\alpha}_1 \mbox{\tt L}(r,s) \vec{\alpha}_2$.
%%
\proofend
%%

If a grammar has the NTS--property, strings can be recognized
very fast. We sketch a pushdown automaton that recognizes
$L(G)$. Scanning the string from left to right it puts the
symbols onto the stack. Using its states the automaton
memorizes the content of the stack up to $\kappa$ symbols
deep, where $\kappa$ is the length of a longest right hand
side of a production. If the upper part of the stack matches
a right hand side of a production $A \pf \vec{\alpha}$ in
the appropriate order, then $\vec{\alpha}$ is deleted from
the stack and $A$ is put on top of it. At this moment the
automaton rescans the upper part of the stack up to $\kappa$
symbols deep. This is done using a series of empty moves. The 
automaton pops $\kappa$ symbols and then puts them back onto 
the stack.  Then it continues the procedure above. It is 
important that the replacement of a right hand side by a left 
hand side is done whenever first possible.
%%
\begin{thm}
Let $G$ be an NTS--grammar. Then $G$ is deterministic.
Furthermore, the recognition and parsing problem are in
$\textbf{DTIME}(n)$.
\end{thm}
%%
We shall deepen this result. To this end we abstract somewhat
from the pushdown automata and introduce a calculus which
manipulates pairs $\vec{\alpha} \bvdash \vec{x}$ of strings
separated by a turnstile. Here, we think of $\vec{\alpha}$
as the stack of the automaton and $\vec{x}$ as the string
to the right of the reading head.  It is not really necessary
to have terminal strings on the right hand side; however,
the generalization to arbitrary strings is easy to do.
There are several operations. The first is called
\textbf{shift}. It simulates the reading of the first
symbol.
%%%
\index{shift}%%
%%%
%%
\begin{equation}
\mbox{\rm shift:}\qquad \begin{array}{l@{\,\bvdash\,}l}
\vec{\eta} & x \vec{y} \\\hline
\vec{\eta}x & \vec{y}
\end{array}
\end{equation}
%%
Another operation is \textbf{reduce}.
%%%
\index{reduction}%%
%%%
%%
\begin{equation}
\mbox{reduce $\rho$:}\qquad
    \begin{array}{l@{\,\bvdash\,}l}
\vec{\eta}\vec{\alpha} & \vec{x} \\\hline
\vec{\eta}X & \vec{x}
\end{array}
\end{equation}
%%
Here $\rho = X \pf \vec{\alpha}$ must be a $G$--rule.
This calculus shall be called the \textbf{shift--reduce--calculus for}
$G$. The following theorem is easily proved by induction on the
length of a derivation.
%%
\begin{thm}
Let $G$ be a CFG. $\vec{\alpha} \vdash_G \vec{x}$
iff there is a derivation of $\vec{\alpha} \bvdash \varepsilon$
from $\varepsilon \bvdash \vec{x}$ in the shift--reduce--calculus
for $G$.
\end{thm}
%%
This strategy can be applied to every language. We take the
following grammar.
%%
\begin{equation}
\begin{array}{l@{\quad\pf\quad}l}
\mbox{\tt S} & \mbox{\tt ASB} \mid \mbox{\tt c} \\
\mbox{\tt A} & \mbox{\tt a} \\
\mbox{\tt B} & \mbox{\tt b}
\end{array}
\end{equation}
%%
Then we have $S \vdash_G \mbox{\tt aacbb}$. Indeed, we get
a derivation shown in Table~\ref{tab:schieb}.
%%
\begin{table}
\caption{A Derivation by Shifting and Reducing}
\label{tab:schieb}
$$\begin{array}{l@{\,\bvdash\,}l}
\varepsilon & \mbox{\tt aacbb} \\\hline
\mbox{\tt a} & \mbox{\tt acbb} \\\hline
\mbox{\tt A} & \mbox{\tt acbb} \\\hline
\mbox{\tt Aa} & \mbox{\tt cbb} \\\hline
\mbox{\tt AA} & \mbox{\tt cbb} \\\hline
\mbox{\tt AAc} & \mbox{\tt bb} \\\hline
\mbox{\tt AAS} & \mbox{\tt bb} \\\hline
\mbox{\tt AASb} & \mbox{\tt b} \\\hline
\mbox{\tt AASB} & \mbox{\tt b} \\\hline
\mbox{\tt AS} & \mbox{\tt b} \\\hline
\mbox{\tt ASb} & \varepsilon \\\hline
\mbox{\tt ASB} & \varepsilon \\\hline
\mbox{\tt S}   & \varepsilon
\end{array}$$
\end{table}
%%
Of course the calculus does not provide unique solutions.
On many occasions we have to guess whether to shift or whether 
to reduce, and if the latter, then by what rule. Notice namely 
that if some right hand side of a production is a suffix of a 
right hand side of another production we have an option. We call 
a $k$--\textbf{strategy} a function $f$
%%%
\index{strategy}%%
%%%
which tells us for every pair $\vec{\alpha} \bvdash \vec{x}$
whether or not we shall shift or reduce (and by which rule).
Further, $f$ shall only depend (1) on the reduction rules
which can be at all applied to $\vec{\alpha}$
and (2) on the first $k$ symbols of $\vec{x}$.
We assume that in case of competition only one rule is
chosen. So, a $k$--strategy is a map $R \times \bigcup_{i < k}
A^i$ to $\{s, r\}$. If $\vec{\alpha} \bvdash \vec{x}$ is
given then we determine the next rule application
as follows. Let $\vec{\beta}$ be a suffix of
$\vec{\alpha}$ which is reducible. If
$f(\vec{\beta}, {^{(k)}\vec{x}}) = s$, then we shift; if
$f(\vec{\beta}, {^{(k)}\vec{x}}) = r$ then we apply
reduction to $\vec{\beta}$.  This is in fact not really
unambigous. For a right hand side of a production may
be the suffix of a right hand side of another production.
Therefore, we look at another property.
%%
\begin{multline}
\label{eq:24ast}
\text{If $\rho_1 = X_1 \pf \vec{\beta}_1 \in R$
and $\rho_2 = X_2 \pf  \vec{\beta}_2 \in R$, $\rho_1 \neq
\rho_2$,} \\
\text{and if $|\vec{y}| \leq k$ then $f(\vec{\beta}_1, \vec{y})$ 
or $f(\vec{\beta}_2,\vec{y})$
is undefined.}
\end{multline}
%%
\nocite{knuth:lrk}
%%
\begin{defn}
%%%
\index{grammar!$LR(k)$--\faul}%%
\index{language!$LR(k)$--\faul}%%
%%%
A CFG $G$ is called an $\textbf{LR}(k)$--\textbf{grammar} if
not $\mbox{\tt S} \Pf^+ \mbox{\tt S}$ and if for some
$k \in \omega$ there is a $k$--strategy for the shift--and--reduce
calculus for $G$. A language is called an $LR(k)$--\textbf{language} 
if it is generated by some $LR(k)$--grammar.
\end{defn}
%%
\begin{thm}
A CFG is an $LR(k)$--grammar if the following holds:
Suppose that $\vec{\eta}_1 \vec{\alpha}_1 \vec{x}_1$ and
$\vec{\eta}_2 \vec{\alpha}_2 \vec{x}_2$ have a rightmost
derivation and that with $p := |\vec{\eta}_1 \vec{\alpha}_1| +k$
we have
%%
\begin{equation}
{^{(p)}\vec{\eta}_1 \vec{\alpha}_1 \vec{x}_1} =
{^{(p)}\vec{\eta}_2 \vec{\alpha}_2 \vec{x}_2}
\end{equation}
%%
Then $\vec{\eta}_1 = \vec{\eta}_2$,
$\vec{\alpha}_1 = \vec{\alpha}_2$ and ${^{(k)}\vec{x}_1} =
{^{(k)}\vec{x}_2}$.
\end{thm}
%%
This theorem is not hard to show. It says that the strategy
may be based indeed only on the $k$--prefix of the string
which is to be read. This is essentially the property \eqref{eq:24ast}.
One needs to convince oneself that a derivation in the
shift--reduce--calculus corresponds to a rightmost derivation,
provided reduction is scheduled as early as possible.
%%
\begin{thm}
\label{thm:det}
$LR(k)$--languages are deterministic.
\end{thm}
%%
We leave the proof of this fact to the reader. The task is
to show how to extract a deterministic automaton from a
strategy. The following is easy.
%%
\begin{lem}
Every $LR(k)$--language is an $LR(k+1)$--language.
\end{lem}
%%
So we have the following hierarchy.
%%
\begin{equation}
LR(0) \;\subseteq\; LR(1) \; \subseteq\; LR(2) \; \subseteq \;
LR(3) \dotso
\end{equation}
%%
This hierarchy is stationary already from $k = 1$.
%%
\begin{lem}
\label{lem:reduktion}
Let $k > 0$. If $L$ is an $LR(k+1)$--language then
$L$ also is an $LR(k)$--language.
\end{lem}
%%
\proofbeg
For a proof we construct an $LR(k)$--grammar $G^>$ from an 
$LR(k+1)$--grammar $G$. For simplicity we assume that $G$ is
in Chomsky Normal Form. The general case is easily shown in the
same way. The idea behind the construction is as follows. A
constituent of $G^>$ corresponds to a constituent of $G$ which 
has been shifted one letter to the right. To implement this idea 
we introduce new symbols, $[a,X,b]$, where $a, b \in A$, $X \in N$, and
$[a,X, \varepsilon]$, $a \in A$. The start symbol of $G^>$ is
the start symbol of $G$. The rules are as follows, where $a, 
b, c$ range over $A$.
%%
\begin{equation}
\begin{array}{l@{\quad\pf \quad}ll}
\mbox{\tt S} & \varepsilon & \mbox{if }
        \mbox{\tt S} \pf \varepsilon \in R, \\
\mbox{\tt S} & a\; [a,\mbox{\tt S},\varepsilon] & a \in A, \\
\mbox{}[a, X, b]    & [a, Y, c]\; [c, Z, b]
    & \mbox{if }X \pf YZ \in R, \\
\mbox{} [a, X, \varepsilon] & [a, Y, c]\; [c, Z, \varepsilon] &
    \mbox{if }X \pf YZ \in R, \\
\mbox{}[a, X, b]    & b   & \mbox{if }X \pf a \in R, \\
\mbox{}[a, X, \varepsilon] & \varepsilon & \mbox{if }X \pf a \in R.
\end{array}
\end{equation}
%%
By induction on the length of a derivation the following is
shown.
%%
\begin{subequations}
\begin{align}
[a, X, b] \vdash_{G^>} \vec{\alpha}\, b \qquad\Dpf\qquad
& X \vdash_G a \, \vec{\alpha} \\
[a, X, \varepsilon] \vdash_{G^>} \vec{\alpha} \qquad\Dpf\qquad
& X \vdash_G a\, \vec{\alpha}
\end{align}
\end{subequations}
%%
From this we can deduce that $G^>$ is an $LR(k)$--grammar. To
this end let $\vec{\eta}_1 \vec{\alpha}_1 \vec{x}_1$ and
$\vec{\eta}_2 \vec{\alpha}_2 \vec{x}_2$ be rightmost derivable
in $G^>$, and let $p := |\vec{\eta}_1 \vec{\alpha}_1| +k$ as
well as
%%
\begin{equation}
{^{(p)}\vec{\eta}_1 \vec{\alpha}_1 \vec{x}_1} =
{^{(p)}\vec{\eta}_2 \vec{\alpha}_2 \vec{x}_2}
\end{equation}
%%
Then $a\vec{\eta}_1 \vec{\alpha}_1 \vec{x}_1 =
\vec{\eta}_1' \vec{\alpha}_1' b \vec{x}_1$ for some $a, b \in A$
and some $\vec{\eta}_1'$, $\vec{\alpha}_1'$ with
$a \vec{\eta}_1 = \vec{\eta}_1' c$ for $c \in A$
and $c \vec{\alpha}_1 = \vec{\alpha}_1' b$. Furthermore,
we have $a\vec{\eta}_2 \vec{\alpha}_2 \vec{x}_2 =
\vec{\eta}_2' \vec{\alpha}_2' b \vec{x}_2$,
$a \vec{\eta}_2 = \vec{\eta}_2' c$ and $c \vec{\alpha}_2 =
\vec{\alpha}_2'$ for certain $\vec{\eta}_2'$ and
$\vec{\alpha}_2'$. Hence we have
%%
\begin{equation}
{^{(p+1)}\vec{\eta}_1' \vec{\alpha}_1' b \vec{x}_1} =
{^{(p+1)}\vec{\eta}_2' \vec{\alpha}_2' b \vec{x}_2}
\end{equation}
%%
and $p +1 = |\vec{\eta}_1' \vec{\alpha}_1'| + k + 1$.
Furthermore, the left hand and the right hand string
have a rightmost derivation in $G$. From this it follows,
since $G$ is an $LR(k+1)$--grammar, that $\vec{\eta}_1' = %
\vec{\eta}_2'$ and $\vec{\alpha}_1' = \vec{\alpha}_2'$,
as well as ${^{(k+1)}b \vec{x}_1} = {^{(k+1)}b \vec{x}_2}$.
From this we get $\vec{\eta}_1 = \vec{\eta}_2$, $\vec{\alpha}_1 %
= \vec{\alpha}_2$ and ${^{(k)}\vec{x}_1} = {^{(k)}\vec{x}_2}$,
as required.
\proofend

Now we shall prove the following important theorem.
%%
\begin{thm}
\label{thm:detlr1}
Every deterministic language is an $LR(1)$--lan\-gua\-ge.
\end{thm}
%%
The proof is relatively long. Before we begin we shall
prove a few auxiliary theorems which establish that strictly
deterministic languages are exactly the languages that are
generated by strict deterministic grammars, and that they
are unambiguous and in $LR(0)$. This will give us the key
to the general theorem.

We still owe the reader a proof that strict deterministic grammars 
only generate strict deterministic languages. This is essentially the
consequence of a property that we shall call
\textbf{left transparency}.
%%%
\index{left transparency}%%
%%
We say $\vec{\alpha}$ \textbf{occurs in} $\vec{\eta}_1\, \vec{\alpha}\,
\vec{\eta}_2$ \textbf{with left context} $\vec{\eta}_1$.
%%%
\index{context!left}%%
%%%
%%
\begin{defn}
%%%
\index{grammar!left transparent}%%
%%%
Let $G$ be a CFG. $G$ is called \textbf{left transparent}
if a constituent may never occur in a string accidentally
with the same left context. This means that
if $\vec{x}$ is a constituent of category $C$ in $\vec{y}_1 \, \vec{x}
\, \vec{y}_2$ and if $\vec{z} := \vec{y}_1 \, \vec{x} \, \vec{y}_3$
is a $G$--string then $\vec{x}$ also is a constituent
of category $C$ in $\vec{z}$.
\end{defn}
%%
For the following theorem we need a few definitions.
Let $\GB$ be a tree and $n \in \omega$ a natural number.
Then ${^{(n)}\GB}$ denotes the tree which consists of all
nodes above the first $n$ leaves from the left. Let
$P$ the set of leaves of $\GB$, say $P = \{p_i : i < q\}$,
and let $p_i \sqsubset p_j$ iff $i < j$.
Then put $N_n := \{p_i : i < n\}$, and $O_n := \uppx{N_n}$.
${^{(n)}\GB} := \auf O_n, r, <, \sqsubset\zu$, where
$<$ and $\sqsubset$ are the relations relativized to
$O_n$. If $\ell$ is a labelling function and $\GT = \auf \GB, \ell\zu$
a labelled tree then let ${^{(n)}\GT} := \auf {^{(n)}\GB},
\ell \restriction O_n\zu$. Again, we denote $\ell \restriction
O_n$ simply by $\ell$. We remark that the set
$R_n := {^{(n)}\GB} - {^{(n-1)}\GB}$ is linearly ordered
by $<$.
%There exists for every $y \in R_n$, which is not a leaf
%a $z$ with $z \prec y$.
We look at the largest element $z$ from $R_n$. Two
cases arise. (a) $z$ has no right sister.
(b) $z$ has a right sister. In Case (a) the constituent
of the mother of $z$ is closed at the transition from
${^{(n-1)}\GB}$ to ${^{(n)}\GB}$. Say that $y$ is at the
\textbf{right edge} of $\GT$ if there is no $z$ such that
$y \sqsubset z$. Then $\uppx{R_n}$ consists exactly of the
elements which are at the right edge of ${^{(n)}\GB}$
and $R_n$ consists of all those elements which are at
the right edge of ${^{(n)}\GB}$ but not contained in
${^{(n-1)}\GB}$. Now the following holds.
%%
\begin{prop}
\label{prop:lfp}
Let $G$ be a strict deterministic grammar. Then $G$
is left transparent. Furthermore: let $\GT_1 = \auf \GB_1, \ell_1\zu$
and $\GT_2 = \auf \GB_2, \ell_2\zu$ be partial $G$--trees
such that the following holds.
\begin{dingautolist}{192}
\item
    If $C_i$ is the label of the root of $\GT_i$ then
    $C_1 \equiv C_2$.
\item
    ${^{(n)}k(\GT_1)} = {^{(n)}k(\GT_2)}$.
\end{dingautolist}
%%
Then there is an isomorphism $f \colon {^{(n+1)}\GB_1} \epi {^{(n+1)}\GB_2}$
such that $\ell_2(f(x)) = \ell_1(x)$ in case $x$ is not at the
right edge of ${^{(n+1)}\GB_1}$ and $\ell_2(f(x)) \equiv \ell_1(x)$
otherwise.
\end{prop}
%%
\proofbeg
We show the theorem by induction on $n$. We assume that it holds
for all $k < n$. If $n = 0$, it holds anyway. Now we show
the claim for $n$. There exists by assumption an isomorphism
$f_n \colon {^{(n)}\GB_1} \pf {^{(n)}\GB_2}$ satisfying the
conditions given above. Again, put $R_{n+1} := {^{(n+1)}\GB_1} %
- {^{(n)}\GB_1}$. At first we shall show that $\ell_2(f_n(x)) =
\ell_1(x)$ for all $x \not\in \uppx{R_{n+1}}$. From this it
immediately follows that $\ell_2(f_n(x)) \equiv \ell_1(x)$ for all
$x \in \uppx{R_{n+1}} - R_{n+1}$ since $G$ is strict
deterministic. This claim we show by induction on the height of
$x$. If $h(x) = 0$, then $x$ is a leaf and the claim holds because
of the assumption that $\GT_1$ and $\GT_2$ have the same
associated string. If $h(x) > 0$ then every daughter of $x$ is in
$\uppx{R_{n+1}}$. By induction hypothesis therefore
$\ell_2(f_n(y)) = \ell_1(y)$ for every $y \prec x$. Since $G$ is
strict deterministic, the label of $x$ is uniquely fixed by this
for $\ell_2(f_n(x)) \equiv \ell_1(x)$, by induction hypothesis. So
we now have $\ell_2(f_n(x)) = \ell_1(x)$. This shows the first
claim. Now we extend $f_n$ to an isomorphism $f_{n+1}$ from
${^{(n+1)}\GB_1}$ onto ${^{(n+1)}\GB_2}$ and show at the same time
that $\ell_2(f_n(x)) \equiv \ell_1(x)$ for every $x \in
\uppx{R_{n+1}}$. This holds already by inductive hypothesis for
all $x \not\in R_{n+1}$. So, we only have to show this for $x \in
R_{n+1}$. This we do as follows. Let $u_0$ be the largest node in
$R_{n+1}$. Certainly, $u_0$ is not the root. So let $v$ be the
mother of $u_0$. $f_n$ is defined on $v$ and we have
$\ell_2(f_n(v)) \equiv \ell_1(v)$. By assumption, $\ell_2(f_n(x))
= \ell_1(x)$ for all $x \sqsubset u$. So, we first of all have
that there is a daughter $x_0$ of $f_n(v)$ which is not in the
image of $f_n$. We choose $x_0'$ minimal with this property. Then
we put $f_{n+1}(u_0) := x_0$. Now we have $\ell_2(f_{n+1}(u_0))
\equiv \ell_1(u_0)$. We continue with $u_0$ in place of $v$. In
this way we obtain a map $f_{n+1}$ from ${^{(n)}\GB_1} \cup
R_{n+1} = {^{(n+1)}\GB_1}$ to ${^{(n+1)}\GB_2}$ with
$\ell_2(f_{n+1}(x))  \equiv \ell_1(x)$, if $x \in R_{n+1}$ and
$\ell_2(f_{n+1}(x)) = \ell_1(x)$ otherwise. That $f_{n+1}$ is
surjective is seen as follows. Suppose that $u_k$ is the leaf of
$\GB_1$ in $R_{n+1}$. Then $x_k = f_{n+1}(u_k)$ is not a leaf in
$\GB_2$, and then there exists a $x_{k+1}$ in ${^{(n+1)}\GB_2} -
{^{(n)}\GB_2}$. We have $\ell_2(f_{n+1}(x_k)) \equiv \ell_1(u_k)$.
Let $x_p$ be the leaf in $L$. By Lemma~\ref{lem:linksrekursiv}
$\ell_2(x_p) \not\equiv \ell_2(x_k)$ and therefore also
$\ell_2(x_p) \not\equiv \ell_1(u_k)$. However, by assumption $x_p$
is the $n+1$st leaf of $\GB_2$ and likewise $u_k$ is the $n+1$st
leaf of $\GB_1$, from which we get $\ell_1(u_k) = \ell_2(x_p)$ in
contradiction to what has just been shown. \proofend
%%
\begin{thm}
Let $G$ be a strict deterministic grammar. Then $L(G)$ is
unambiguous. Further, $G$ is an $LR(0)$--grammar and $L(G)$ is
strict deterministic.
\end{thm}
%%
\proofbeg 
The strategy of shifting and reducing can be applied as
follows: every time we have identified a right hand side of a rule
$X \pf \vec{\mu}$ then this is a constituent of category $X$ and
we can reduce. This shows that we have a $0$--strategy. Hence the
grammar is an $LR(0)$--grammar. $L(G)$ is certainly unambiguous.
Furthermore, $L(G)$ is deterministic, by Theorem~\ref{thm:det}.
Finally, we have to show that $L(G)$ is prefix free for then by
Theorem~\ref{thm:praefixfrei} it follows that $L(G)$ is strict
deterministic. Now let $\vec{x}\, \vec{y} \in L(G)$.  If also
$\vec{x} \in L(G)$, then by Proposition~\ref{prop:lfp} we must
have $\vec{y} = \varepsilon$. \proofend

At first sight it appears that Lemma~\ref{lem:reduktion}
also holds for $k = 0$. The construction can be extended to
this case without trouble. Indeed, in this case we get something
of an $LR(0)$--grammar; however, it is to be noted that
a strategy for $G^>$ does not only depend on the next symbol.
Additionally, it depends on the fact whether or not the string that
is yet to be read is empty. The strategy is therefore not
entirely independent of the right context even though
the dependency is greatly reduced. That $LR(0)$--languages
are indeed more special than $LR(1)$--languages is the
content of the next theorem.
%%
\nocite{gellerharrison:lr0}
\index{Harrison, M.~A.}%%%
\index{Geller, M.~M.}%%
%%%%
\begin{thm}[Geller \& Harrison]
Let $L$ be a deterministic CFL. Then the following are equivalent.
%%
\begin{dingautolist}{192}
\item
$L$ is an $LR(0)$--language.
\item
If $\vec{x} \in L$, $\vec{x}\,  \vec{v} \in L$ and $\vec{y} \in L$
then also $\vec{y}\, \vec{v} \in L$.
\item
There are strict deterministic languages $U$ and $V$ such that
$L = U \cdot V^{\ast}$.
\end{dingautolist}
%%
\end{thm}
%%
\proofbeg
Assume \ding{192}. Then there is an $LR(0)$--grammar $G$ for $L$. Hence, 
if $X \pf \vec{\alpha}$ is a rule and if $\vec{\eta} \, \vec{\alpha} 
\, \vec{y}$ is $G$--derivable then also  $\vec{\eta}\, X \, \vec{y}$ is
$G$--derivable. Using induction, this can also be shown of
all pairs $X$, $\vec{\alpha}$ for which $X \vdash_G \vec{\alpha}$.
Now let $\vec{x} \in L$ and $\vec{x}\, \vec{v} \in L$. Then
$\mbox{\tt S} \vdash_G \vec{x}$, and so by the
previous $\vdash_G \mbox{\tt S} \, \vec{v}$. Therefore, 
since $\mbox{\tt S} \vdash_G \vec{y}$ we have 
$\vdash_G \vec{y}\, \vec{v}$. Hence \ding{193} obtains.
Assume now \ding{193}. Let $U$ be the set of all $\vec{x} \in L$ 
such that $\vec{y} \not\in L$ for every proper prefix $\vec{y}$ of 
$\vec{x}$. Let $V$ be the set of all $\vec{v}$ such that $\vec{x}\,
\vec{v} \in L$ for some $\vec{x} \in U$ but
$\vec{x}\, \vec{w} \not\in L$ for every $\vec{x} \in U$
and every proper prefix $\vec{w}$ of $\vec{v}$.
Now, $V$ is the set of all $\vec{y} \in V^{\ast} -
\{\varepsilon\}$ for which no proper prefix is in $V^{\ast} -
\{\varepsilon\}$. We show that $U \cdot V^{\ast} = L$. To this end
let us prove first that $L \subseteq U \cdot V^{\ast}$. Let
$\vec{u} \in L$. We distinguish two cases.
(a) No proper prefix of $\vec{u}$ is in $L$.
Then $\vec{u} \in U$, by definition of $U$.
(b) There is a proper prefix $\vec{x}$ of $\vec{u}$
which is in $L$. We choose $\vec{x}$ minimally.
Then $\vec{x} \in U$. Let $\vec{u} = \vec{x} \, \vec{v}$.
Now two subcases arise. (A) For no proper prefix
$\vec{w}_0$ of $\vec{v}$ we have $\vec{x}\, \vec{w}_0
\in L$. Then $\vec{v} \in V$, and we are done.
(B) There is a proper prefix $\vec{w}_0$ of $\vec{v}$
with $\vec{x}\, \vec{w}_0 \in L$. Let $\vec{v} =
\vec{w}_0 \, \vec{v}_1$. Then, by \ding{193},
we have $\vec{x} \, \vec{v}_1 \in L$. (In \ding{193}, put
$\vec{x}\, \vec{w}_0$ in place of $\vec{x}$ and in place of
$\vec{y}$ put $\vec{x}$ and for $\vec{w}$ put $\vec{v}_1$.)
$\vec{x}\, \vec{v}_1$ has smaller length than $\vec{x}\, \vec{v}$.
Continue with $\vec{x} \, \vec{v}_1$ in the same way.
At the end we get a partition of $\vec{v} =
\vec{w}_0 \, \vec{w}_1 \dotsb \vec{w}_{n-1}$ such that
$\vec{w}_i \in V$ for every $i < n$. Hence $L \subseteq
U \cdot V^{\ast}$. We now show $U \cdot V^{\ast} \subseteq L$.
Let $\vec{u} = \vec{x} \conc \prod_{i < n} \vec{w}_i$.
If $n = 0$, then $\vec{u} = \vec{x}$ and by definition
of $U$ we have $\vec{u} \in L$. Now let $n > 0$. With
\ding{193} we can show that $\vec{x} \conc \prod_{i < n-1}
\vec{w}_i \in L$. This shows that $\vec{u} \in L$.
Finally, we have to show that $U$ and $V$ are deterministic.
This follows for $U$ from Theorem~\ref{thm:prffrei}. Now let
$\vec{x}, \vec{y} \in U$. Then by \ding{193} 
$P := \{\vec{v} : \vec{x}\, \vec{v} \in L\} =
\{\vec{v} : \vec{y}\, \vec{v} \in L\}$. The reader may convince
himself that $P$ is deterministic. Now let $V$ be the set of
all $\vec{v}$ for which there is no prefix in $P - \{\varepsilon\}$.
Then $P = V^{\ast}$ and because of Theorem~\ref{thm:prffrei}
$V$ is strict deterministic. This shows \ding{194}.
Finally, assume \ding{194}. We have to show that
$L$ is an $LR(0)$--language. To this end, let
$G_1 = \auf \mbox{\tt S}, N_1, A, R_1\zu$ be a strict
deterministic grammar which generates $U$ and
$G_2 = \auf \mbox{\tt S}_2, N_2, A, R_2\zu$
a strict deterministic grammar which generates $V$.
Then let $G_3 := \auf \mbox{\tt S}_3, N_1 \cup N_2 \cup
\{\mbox{\tt S}_3, \mbox{\tt S}_4\}, A, R_3\zu$ be defined
as follows.
%%
\begin{equation}
R_3 := R_1 \cup R_2 \cup \{\mbox{\tt S}_3 \pf \mbox{\tt S}_1,
\mbox{\tt S}_3 \pf \mbox{\tt S}_1 \, \mbox{\tt S}_4,
\mbox{\tt S}_4 \pf \mbox{\tt S}_2, \mbox{\tt S}_4 \pf
\mbox{\tt S}_2 \, \mbox{\tt S}_4\}  
\end{equation}
%%
It is not hard to show that $G_3$ is an $LR(0)$--grammar
and that $L(G_3) = L$.
\proofend
%%

The decomposition in \ding{194} is unique, if we exclude the
possibility that $V = \varnothing$ and if we require that 
$U = \{\varepsilon\}$ shall be the case only if $V = \{\varepsilon\}$. 
In this way we take care of the cases $L = \varnothing$ and $L = U$.
The case $U = V$ may arise. Then $L = U^+$.
The semi Dyck languages are of this kind.

Now we proceed to the proof of Theorem~\ref{thm:detlr1}. Let
$L$ be deterministic. Then put $M := L \cdot \{\$\}$,
where \$ is a new symbol. $M$ is certainly deterministic;
and it is prefix free and so strict deterministic.
It follows that $M$ is an $LR(0)$--language. Therefore there
exists a strict deterministic grammar $G$ which generates $M$.
From the next theorem we now conclude that $L$ is an
$LR(1)$--language.
%%
\begin{lem}
Let $G$ be an $LR(0)$--grammar of the form $G = \auf \mbox{\tt S},
N \cup \{\mbox{\tt \$}\}, A, R\zu$ with $R \subseteq N \times ((N \cup A)^{\ast}
\cup (N \cup A)^{\ast} \cdot \mbox{\tt \$})$ and 
$L(G) \subseteq A^{\ast}\mbox{\tt \$}$, and assume that there is 
no derivation $\mbox{\tt S} \Pf_R \mbox{\tt S\$}$ in $G$. Then 
let $H := \auf \mbox{\tt S}, N, A, R'\zu$, where
%%
\begin{align}
R' := & \phantom{\mbox{}\cup\mbox{}}
    \{A \pf \vec{\alpha} : A \pf \vec{\alpha} \in R, \vec{\alpha} \in
    (N \cup A)^{\ast}\} \\\notag
   & \cup \{A \pf \vec{\alpha} : A \pf \vec{\alpha}\, \mbox{\tt \$} \in R\}
\end{align}
%%
Then $H$ is an $LR(1)$--grammar and $L(H)\cdot\mbox{\tt \$} = L(G)$.
\end{lem}
%%
For a proof consider the following. We do not have
$\mbox{\tt S} \Pf^+_L \mbox{\tt S}$ in $H$. Further: if
$\mbox{\tt S} \Pf_L^+ \vec{\alpha}$ in $H$ then there exists
a $D$ such that $\mbox{\tt S} \Pf^+_L \vec{\alpha}\, D$ in $G$,
and if $\mbox{\tt S} \Pf^+_L \vec{\beta}$ in $G$ then we have
$\vec{\beta} = \vec{\alpha}\, D$ and $\mbox{\tt S} \Pf ^+_L
\vec{\alpha}$ in $H$.  From this we can immediately conclude
that $H$ is an $LR(1)$--grammar.

Finally, let us return to the calculus of shifting and reducing.
We generalize this strategy as follows. For every symbol
$\alpha$ of our grammar we add a symbol $\uli{\alpha}$. This
symbol is a formal inverse of $\alpha$; it signals that at its
place we look for an $\alpha$ but haven't identified it yet.
This means that we admit the following transitions.
%%
\begin{equation}
\begin{array}{l@{\, \bvdash\,}l}
\vec{\eta}\uli{\alpha}\alpha & \vec{x} \\\hline
\vec{\eta}                   & \vec{x}
\end{array}
\end{equation}
%%
We call this rule \textbf{cancellation}.
%%%
\index{cancellation}%%
%%%
We write for strings $\vec{\alpha}$ also
$\uli{\vec{\alpha}}$. This denotes the formal inverse of the
entire string. If $\vec{\alpha} = \prod_{i < n} \alpha_i$ then
$\uli{\alpha} = \prod_{i < n} \uli{\alpha_{n-i}}$. Notice that
the order is reversed.  For example $\uli{\mbox{\tt AB}} =
\uli{\mbox{\tt B}}\conc \uli{\mbox{\tt A}}$.
These new strings allow to perform reductions on the left hand side
even when only part of the right hand side of a production has
been identified. The most general rule is this one.
%%
\begin{equation}
\begin{array}{l@{\, \bvdash\, }l}
\vec{\eta}\uli{X}\vec{\alpha} & \vec{x} \\\hline
\vec{\eta} \uli{\vec{\tau}}  & \vec{x}
\end{array}
\end{equation}
%%
This rule is called the \textbf{LC--rule}.
%%%
\index{LC--rule}%%
\index{LC--calculus}%%
%%%
Here $X \pf \vec{\alpha}\vec{\tau}$ must be a $G$--rule.
This means intuitively speaking that $vec{\alpha}$ is an $X$ 
if followed by $\vec{\tau}$. Since $\vec{\tau}$ is not yet there
we have to write $\uli{\vec{\tau}}$.  The \textbf{LC--calculus} 
consists of the rules shift, reduce and LC. Now the following holds.
%%
\begin{thm}
Let $G$ be a  grammar. $\vec{\alpha} \vdash_G \vec{x}$ holds
iff there is a derivation of $\varepsilon \bvdash
\varepsilon$ from $\uli{\vec{\alpha}} \bvdash \vec{x}$ in the
\textbf{LC}--calculus.
\end{thm}
%%
A special case is $\vec{\alpha} = \varepsilon$. Here no part of the
production has been identified, and one simply guesses a rule.
If in place of the usual rules only this rule is taken, we get
a strategy known as \textbf{top--down strategy}.
%%%
\index{strategy!top--down}%%
%%%
In it, one may shift, reduce and guess a rule.  A grammar is called
an $LL(k)$--grammar if it has a deterministic recognition algorithm
using the top--down--strategy in which the next step depends on the
first $k$ symbols of $\vec{x}$. The case $k = 0$ is of little
use (see the exercises).

This method is however too flexible to be really useful.
However, the following is an interesting strategy. The right
hand side of a production is divided into two parts, which
are separated by a dot.
%%
\begin{equation}
\begin{array}{l@{\quad \pf\quad}l}
\mbox{\tt S} & \mbox{\tt A.SB} \mid \mbox{\tt c.} \\
\mbox{\tt A} & \mbox{\tt a.} \\
\mbox{\tt B} & \mbox{\tt b.}
\end{array}
\end{equation}
%%
This dot fixes the part of the rule that must have been read
when the corresponding LC--rule is triggered. A strategy of this
form is called \textbf{generalized left corner strategy}.
%%%
\index{strategy!generalized left corner}%%%
%%%
If the dot is at the right edge we get the bottom--up strategy,
if it is at the left edge we get the top--down strategy.
%%%
\index{strategy!bottom--up}%%
%%%
%%
\begin{table}
\caption{The Generalized LC--Strategy}
\label{tab:lc}
$$\begin{array}{l@{\, \bvdash\,}l}
\uli{\mbox{\tt S}} & \mbox{\tt aacbb} \\\hline
\uli{\mbox{\tt S}}\, \mbox{\tt a} & \mbox{\tt acbb} \\\hline
\uli{\mbox{\tt S}}\, \mbox{\tt A} & \mbox{\tt acbb} \\\hline
\uli{\mbox{\tt B}}\, \uli{\mbox{\tt S}} & \mbox{\tt acbb} \\\hline
\uli{\mbox{\tt B}}\, \uli{\mbox{\tt S}} \mbox{\tt a}
    & \mbox{\tt cbb} \\\hline
\uli{\mbox{\tt B}}\, \uli{\mbox{\tt S}} \,\mbox{\tt A}
    & \mbox{\tt cbb} \\\hline
\uli{\mbox{\tt B}}\, \uli{\mbox{\tt B}}\, \uli{\mbox{\tt S}}
    & \mbox{\tt cbb} \\\hline
\uli{\mbox{\tt B}}\, \uli{\mbox{\tt B}}\, \uli{\mbox{\tt S}}
    \mbox{\tt c} & \mbox{\tt bb} \\\hline
\uli{\mbox{\tt B}}\, \uli{\mbox{\tt B}}\, \uli{\mbox{\tt S}} \mbox{\tt S}
    & \mbox{\tt bb} \\\hline
\uli{\mbox{\tt B}}\, \uli{\mbox{\tt B}} & \mbox{\tt bb} \\\hline
\uli{\mbox{\tt B}}\, \uli{\mbox{\tt B}}\, \mbox{\tt b}
    & \mbox{\tt b} \\\hline
\uli{\mbox{\tt B}}\, \uli{\mbox{\tt B}}\, \mbox{\tt B} & \mbox{\tt b} \\\hline
\uli{\mbox{\tt B}} & \mbox{\tt b} \\\hline
\uli{\mbox{\tt B}}\, \mbox{\tt b} & \varepsilon \\\hline
\uli{\mbox{\tt B}}\, \mbox{\tt B} & \varepsilon \\\hline
\varepsilon & \varepsilon
\end{array}$$
\end{table}
%%
\vplatz
\exercise
Let $R$ be a set of context free rules, $\mbox{\tt S}$ 
a symbol, $N$ and $A$ finite sets, and $G := 
\auf \mbox{\tt S}, N, A, R\zu$. Show that if 
$\Pf^{\ast}_R \varepsilon$ and $G$ is transparent 
then $G$ is a CFG. {\it Remark.} Transparency can 
obviously be generalized to any grammar that uses 
context free rules.
%%
\vplatz
\exercise
Show Theorem~\ref{thm:det}.
%%
\vplatz
\exercise
\label{ex:PN}
Prove Lemma~\ref{lem:zahl}. Show in addition: {\it If
$\vec{x}$ is a term then the set $P(\vec{x})
:= \{\gamma(\vec{y}) : \vec{y} \text{ is a prefix of }
\vec{x}\}$ is convex.}
%%
\vplatz
\exercise
Show the following: {\it If $L$ is deterministic then also
$L/\{\vec{x}\}$ as well as $\{\vec{x}\}\backslash L$
are deterministic.} (See Section~\ref{kap1}.\ref{kap1-2} for notation.)
%%
\vplatz
\exercise
Show that a grammar is an $LL(0)$--grammar if it generates
exactly one tree.
%%
\vplatz
\exercise
Give an example of an NTS--language which is not an
$LR(0)$--language.

 \section{Semilinear Languages}
%
%
%
In this section we shall study semilinear languages. The notion of 
semilinearity is important in itself as it is widely believed that 
natural languages are semilinear. Whether or not this is case, is 
still open (see Section~\ref{kap2}.\ref{kap2-6}). The issue of 
semilinearity is important, because many grammar formalisms proposed 
in the past only generate semilinear languages (or else are generally 
so powerful that they generate every recursively enumerable set). 
Even though semilinearity in natural languages is the rule rather 
than the exception, the counterexamples show that the grammar 
formalisms do not account for natural language in a satisfactory 
way. 

In this chapter we shall prove a theorem by Ginsburg and Spanier 
%%%
\index{Ginsburg, Seymour}%%%
\index{Spanier, Edwin H.}%%%
%%%%
which says that the semilinear subsets of $\omega^n$ are exactly the 
sets definable in 
%%%
\index{Presburger Arithmetic}%%%
%%%
Presburger Arithmetic. This theorem has numerous consequences, in 
linguistics as well as in mathematics. The proof given here differs 
substantially from the original one. 
%%
\begin{defn}
%%%
\index{monoid!commutative}%%
\index{semigroup!commutative}%%
%%%
A \textbf{commutative monoid} or \textbf{commutative se\-mi\-group 
with unit} is a structure $\auf H, 0, +\zu$ in which the following 
holds for every $x, y, z \in H$.
%%%
\begin{equation}
\label{eq:25commgrp}
\begin{split}
x + 0       & = x \\
x + (y + z) & = (x + y) + z \\
x + y       & = y + x
\end{split}
\end{equation}
\end{defn}
%%
Notice that because of associativity we may dispense with 
brackets. Alternatively, any term can be arbitrarily bracketed 
without affecting its value. We define the notation $\mu \cdot x$ 
as follows: $0 \cdot x := 0$ and $(\mu + 1) \cdot x := \mu \cdot x + x$.
(Later on we shall drop $\cdot$.)
Then $\mu \cdot x_0 + \nu \cdot x_0 = (\mu + \nu) \cdot x_0$, 
and $\mu \cdot (\nu \cdot x_0) = (\mu\nu) \cdot x_0$, simply
by definition. Furthermore, $\mu \cdot (x + y) = 
(\mu \cdot x) + (\mu \cdot y)$, by induction on $\mu$. 
This can be generalized.
%%%
\begin{lem}
\label{lem:commgrpgl}
In a commutative semigroup, the following holds. 
%%%
\begin{align}
\mu \cdot (\sum_{i < m} \nu_i \cdot x_i) & = 
	\sum_{i < m} (\mu\nu_i) \cdot x_i \\
\sum_{i < m} \mu_i \cdot x_i + \sum_{i < m} \nu_i\cdot  x_i & = 
	\sum_{i < m} (\mu_i + \nu_i) \cdot x_i
\end{align}
%%%
\end{lem}
%%%
\proofbeg
Induction on $m$. The case $m = 1$ has been dealt with. Now: 
%%%
\begin{align}
\mu \cdot (\sum_{i < m+1} \nu_i \cdot x_i) & = 
     \mu \cdot (\sum_{i < m} \nu_i \cdot x_i + \nu_m \cdot x_m) \\\notag
   & = \mu \cdot (\sum_{i < m} \nu_i \cdot x_i) + \mu\cdot (\nu_m \cdot x_m)
\\\notag
   & = \sum_{i < m} (\mu\nu_i) \cdot x_i + (\mu\nu_m) \cdot x_m 
\\\notag 
   & = \sum_{i < m+1} (\mu\nu_i) \cdot x_i 
\end{align}
%%%
Also 
%%%
\begin{align}
 & \sum_{i < m+1} \mu_i \cdot x_i + \sum_{i < m+1} \nu_i\cdot  x_i 
\\\notag 
= & (\sum_{i < m} \mu_i \cdot x_i + \mu_m \cdot x_m) 
	+ (\sum_{i < m} \nu_i\cdot  x_i + \nu_m \cdot x_m) \\\notag 
= & (\sum_{i < m} \mu_i \cdot x_i + \sum_{i < m} \nu_i \cdot x_i)  
	+ (\mu_m + \nu_m) \cdot x_m \\\notag 
= & \sum_{i < m} (\mu_i + \nu_i) x_i + (\mu_m + \nu_m) \cdot x_m 
	\\\notag
= & \sum_{i < m+1} (\mu_i + \nu_i) x_i
\end{align}
This finishes the proof.
\proofend

%%%
We shall denote by $M(A)$ set underlying the commutative monoid freely 
generated by $A$. By construction, $\GM(A) := \auf M(A), 0, +\zu$ 
%%%
\index{$\GM(A)$, $\Omega^n$}%%%
%%%
is a commutative
semigroup with unit. What is more, $\GM(A)$ is freely generated by
$A$ as a commutative semigroup. We now look at the set $\omega^n$ of 
all $n$--long sequences of natural numbers, endowed with the operation 
$+$ defined by
%%
\begin{equation}
\auf x_i : i < n\zu + \auf y_i : i< n\zu := \auf x_i + y_i :
i < n\zu 
\end{equation}
%%
This also forms a commutative semigroup with unit. Here the
unit is the sequence $\vec{0}$ consisting of $n$ 0s. We denote this
semigroup by $\Omega^n$. For the following theorem we also need
the so--called
%%%
\index{Kronecker symbol}%%
%%%
\textbf{Kronecker symbol}.
%%
\begin{equation}
\delta^i_j := \begin{cases}
1 & \text{ if i = j,} \\
0 & \text{ otherwise.}
\end{cases}
\end{equation}
%%
\begin{thm}
Let $A = \{a_i : i < n\}$. Let $h$ be the map
which assigns to each element $a_i$ the sequence
$\vec{e}_i = \auf \delta^i_j : j < n\zu$. Then the homomorphism
which extends $h$ is an isomorphism from $\GM(A)$
onto $\Omega^n$.
\end{thm}
%%
\proofbeg
Let $\theta$ be the smallest congruence relation on
$\Tm_{\Omega}(A)$ (with $\Omega \colon 0 \mapsto 0, + \mapsto 2$)
which satisfies \eqref{eq:25commgrp}. It follows from 
Lemma~\ref{lem:commgrpgl} by induction on the level of the 
term $t$ that for $t \in \Tm_{\Omega}(A)$ 
there is a $u\; \theta \; t$ of the form
%%%
\begin{equation}
\label{eq:additive}
u = \sum_{i < n} k_i \cdot a_i
\end{equation}
%%
If \eqref{eq:additive} obtains, put $q(t) := \auf k_i : i < n\zu$. 
Now, it is immediately seen that $\theta = \ker q$, whence 
$\Tm_{\Omega}(A)/\theta \cong \Omega^n$. On the other 
hand, $\Tm_{\Omega}(A)/\theta \cong \GM(A)$, since 
it is easily shown that the first is also freely generated by $A$.
Namely, suppose that $v : a_i \mapsto n_i$ is a map from $A$ into 
$\GN$. Let $\oli{v} : \Tm_{\Omega}(A) \pf N$ be the extension of 
$v$. Then, since $\GN$ is a monoid, $\theta \subseteq \ker \oli{v}$, 
so that we can define a map $q : \goth{Tm}_{\Omega}(A) \pf \GN$ 
such that $\oli{v} = q \circ h_{\theta}$. 
\proofend

This theorem tells us that free commutative semigroups can be
thought of as vectors of numbers. A general element of $M(A)$
can be written down as $\sum_{i < n} k_i \cdot a_i$
where $k_i \in \omega$.

Now we shall define the map $\mu \colon A^{\ast} \pf M(A)$ by
%%
%%%
\index{$\mu$}%%%
%%%%
\begin{equation}
\begin{split}
\mu(\varepsilon)   & = 0    \\
\mu(a_i) & = a_i \\
\mu(\vec{x} \conc \vec{y}) & = \mu(\vec{x}) + \mu(\vec{y})
\end{split}
\end{equation}
%%
This map is a homomorphism of monoids and also surjective.
It is not injective, except in the case where $A$ consists
of one element only. The map
%%%
\index{Parikh map}%%
%%%
$\mu$ is called the \textbf{Parikh map}. We have
%%
\begin{equation}
\mu\left(\prod_{i < k} \vec{x}_i\right) =
\sum_{i < k} \mu(\vec{x}_i) 
\end{equation}
%%
\begin{defn}
%%%
\index{letter equivalence}%%
%%%
Two languages $L, M \subseteq A^{\ast}$ are called
\textbf{letter equivalent} if we have $\mu[L] = \mu[M]$.
\end{defn}
%%
\begin{defn}
\label{defn:semi}
%%%
\index{language!linear}%%
\index{language!semilinear}%%
%%%
Elements of $M(A)$ will also be denoted using vector arrows. Moreover, 
if $\vec{x} \in \omega^n$ we write $\vec{x}(i)$ for the $i^{\text{th}}$ 
component of $\vec{x}$. A set $U \subseteq M(A)$ is called \textbf{linear} 
if for some $\alpha \in \omega$ and some $\vec{u}, \vec{v}_i \in M(A)$
%%
\begin{equation}
U = \{\vec{u} + \sum_{i < \alpha} k_i \cdot \vec{v}_i : k_0, \dotsc, 
    k_{\alpha-1}\in \omega\} 
\end{equation}
%%
The $\vec{v}_i$ are called \textbf{cyclic vectors of} $U$.
%%%
\index{vector!cyclic}%%
\index{dimension}%%
%%%
The smallest $\alpha$ for which $U$ has such a representation
is called the \textbf{dimension of} $U$.  $U$ is said to be
\textbf{semilinear} if $U$ is the finite union of linear sets.
A language $L \subseteq A^{\ast}$ is called \textbf{semilinear}
if $\mu[L]$ is semilinear.  
\end{defn}
%%
We can denote semilinear sets rather compactly as follows.
If $U$ and $V$ are subsets of $M(A)$ then write
$U + V := \{\vec{x} + \vec{y} : \vec{x} \in U, \vec{y} \in V\}$. 
%%%%
\index{$U + V$, $nU$, $\omega U$}%%%
%%%%
Further, let $\vec{x} + U := \{\vec{x} + \vec{y} : \vec{y} \in U\}$. 
So, vectors are treated as singleton sets. Also, we write 
$n U := \{n\vec{x} : n \in \omega\}$. Finally, we denote by 
$\omega U$ the union of all $n U$, $n \in \omega$. With these 
abbreviations we write the set $U$ from Definition~\ref{defn:semi} 
as follows.
%%
\begin{equation}
U = \vec{u} + \omega \vec{v}_0 + \omega \vec{v}_1 + \dotsb + 
\omega \vec{v}_{\alpha-1} 
\end{equation}
%%
This in turn we abbreviate by
%%
\begin{equation}
U = \vec{u} + \sum_{i < \alpha} \omega \vec{v}_i
\end{equation}
%%%
Finally, for $V = \{\vec{v}_i : i < \alpha\}$
%%%
\index{$\Sigma(U;V)$}%%
%%%%
\begin{equation}
\Sigma(U;V) := U + \sum_{i < \alpha}  \omega \vec{v}_i
\end{equation}
%%
\begin{lem}
\label{lem:sigmaeig}
The following holds.
\begin{dingautolist}{192}
\item $\Sigma(U;V) \cup \Sigma(U';V) = \Sigma(U \cup U';V)$.
\item $\Sigma(U;V) + \Sigma(U';V') = \Sigma(U + U'; V\cup V')$.
\item $\omega \Sigma(U;V) = \Sigma(\{\vec{0}\};U \cup V)$. 
\end{dingautolist}
\end{lem}
%%
\index{Parikh, Rohit}%%%
%%%%
\begin{thm}[Parikh]
A language is semilinear iff it is letter equivalent to a
regular language.
\end{thm}
%%
\proofbeg
$(\Pf)$ It is enough to show this for linear languages. Suppose 
that $\pi[L] = \Sigma(\{\vec{u}\};V)$, $V = \{\vec{v}_i : i < n\}$. 
Pick a string $\vec{x}$ and $\vec{y}_i$, $i < n$, such that 
$\pi(\vec{x}) = \vec{u}$ and $\pi(\vec{y}_i) = \vec{v}_i$ for 
all $i < n$. Put  
%%%
\begin{equation}
M := \vec{x}\conc (\bigcup_{i < n} \vec{y}_i)^{\ast}
\end{equation}
%%%
Clearly, $M$ is regular and letter equivalent to $L$.
$(\Leftarrow)$ By induction on the length of the regular term $R$
we shall show that $\mu[L(R)]$ is semilinear. This is clear for
$R = a_i$ or $R = \varepsilon$. It is also clear for
$R = S_1 \cup S_2$. Now let $R = S_1 \cdot S_2$. Using the equations 
$(S \cup T) \cdot U = S \cdot U \cup T \cdot U$
and $U \cdot (S \cup T) = U \cdot S \cup U \cdot T$, we can 
assume that $S_1$ and $S_2$ are linear. Then by definition 
$\mu[L(S_1)] = \Sigma(\{\vec{u}\};C_1)$ and 
$\mu[L(S_2)] = \Sigma(\{\vec{v}\};C_2)$, for certain $\vec{u}$, 
$\vec{v}$, and sets $C_1$ and $C_2$. Then, using 
Lemma~\ref{lem:sigmaeig}, we get
%%
\begin{equation}
\mu[L(R)] = \Sigma(\{\vec{u}\};C_1) + \Sigma(\{\vec{v}\};C_2) = 
\Sigma(\{\vec{u}+\vec{v}\};C_1\cup C_2)
\end{equation}
%%
Now, finally, $R = S^{\ast}$. If $S = T \cup U$,
then $R = (T^{\ast} \cdot U^{\ast})^{\ast}$, so that we may
again assume that $S$ is linear, say, $S = \Sigma(\{\vec{u}\}; C)$ 
for some $\vec{u}$ and $C$. By Lemma~\ref{lem:sigmaeig} 
%%
\begin{equation}
\mu[L(R)] = \omega\Sigma(\{\vec{u}\};C) = \Sigma(\{\vec{0}\};\{\vec{u}\} 
\cup C)
\end{equation}
%%
Hence $R$ too is linear. This ends the proof.
\proofend
%%

We draw a useful conclusion from the definitions.
%%
\begin{thm}
Let $A$ be a (possibly infinite) set. The set of semilinear
languages over $A$ form an AFL with the exception that the
intersection of a semilinear language with a regular language
need not be semilinear.
\end{thm}
%%
\proofbeg
Closure under union, star and concatenation are immediate.
We have to show that semilinear languages are closed under
homomorphisms and inverse homomorphisms. The latter is again
trivial. Now let $v \colon A \pf A^{\ast}$ be a homomorphism.
$v$ induces a map $\kappa_v \colon \GM(A) \pf \GM(A)$.
The image under $\kappa_v$ of a semilinear set is semilinear.
For given a string $\vec{x} \in A^{\ast}$ we have
$\mu(\oli{v}(\vec{x})) = \kappa_v(\mu(\vec{x}))$, as is easily
checked by induction on the length of $\vec{x}$.
Let $M$ be linear, say $M = \vec{u} + \sum_{i < k} \omega \cdot
\vec{v}_i$. Then
%%
\begin{equation}
\kappa_v[M] = \kappa_v(\vec{u}) + \sum_{i < k} \omega 
\kappa_v(\vec{v}_i) 
\end{equation}
%%
From this the claim follows. Hence we have
$\mu[\oli{v}[L]] = \kappa_v[\mu[L]]$. The right hand side is
semilinear as we have seen. Finally, take the language
$L := \{\mbox{\tt a}^{2^i}\mbox{\tt b}^{2^i} : i \in \omega\}
\cup \{\mbox{\tt b}^j\mbox{\tt a}^j : j \in \omega\}$.
$L$ is semilinear. $L \cap \mbox{\tt a}^{\ast}\mbox{\tt b}^{\ast}
= \{\mbox{\tt a}^{2^i}\mbox{\tt b}^{2^i} : i \in \omega\}$
is not semilinear, however.
\proofend
%%

Likewise, a subset of $\BZ^n$ ($\BQ^n$) is called \textbf{linear} if it 
has the form
%%
\begin{equation}
\vec{v}_0 + \BZ \vec{v}_1 + \BZ \vec{v}_2 + \dotsb + \BZ \vec{v}_m
\end{equation}
%%
for subsets of $\BZ^n$ as well as 
%%
\begin{equation}
\vec{v}_0 + \BQ \vec{v}_1 + \BQ \vec{v}_2 + \dotsb + \BQ \vec{v}_m
\end{equation}
%%
for subsets of $\BQ^n$. The linear subsets of $\BQ^n$ are nothing but 
the affine subspaces. A subset of $\omega^n$ ($\BZ^n$, $\BQ^n$) is called 
\textbf{semilinear} if it is the finite union of linear sets. 

Presburger Arithmetic is defined as follows. The basic symbols are 
$\mbox{\mtt 0}$, $\mbox{\mtt 1}$, $\mbox{\mtt +}$, $\mbox{\mtt <}$ 
and $\mbox{\mtt \symbol{30}}_m$, $m \in \omega - \{0,1\}$. 
Then Presburger Arithmetic is the set of first order sentences which 
are valid in $\uli{\BZ} := \auf \BZ, 0, 1, +, <, %%
\auf \equiv_m : 1 < m \in \omega\zu\zu$, where $a \equiv_m b$ iff 
$a - b$ is divisible by $m$ (for FOL see Sections~\ref{kap3}.\ref{kap3-6} 
and \ref{kap6}.\ref{kap6-4a}). 

Negation can be eliminated. Notice namely that $\mbox{\mtt \symbol{5}%
(x}_0$=x$_1\mbox{\mtt )}$ is equivalent to {\mtt (x$_0$<x$_1$)%
\symbol{31}(x$_1$<x$_0$)}, {\mtt \symbol{5}(x$_0$<x$_1$)} to 
{\mtt (x$_0$=x$_1$)\symbol{31}(x$_1$<x$_0$)}
and {\mtt \symbol{5}(x$_0$\symbol{30}$_m$x$_1$)} is equivalent to 
{\mtt $\goder_{0 < n < m}$x$_0$\symbol{30}$_m$(x$_1$+$\uli{n}$)}. 
Here, $\uli{n}$ is defined by $\uli{0} := \mbox{\mtt 0}$, 
$\uli{n+1} := \mbox{\mtt ($\uli{n}$+1)}$. We shall 
use {\mtt x$_0$\symbol{28}x$_1$} for 
{\mtt (x$_0$<x$_1$)\symbol{31}(x$_0$=x$_1$)}. Moreover, multiplication 
by a given natural number also is definable: put $0t := \oli{0}$, and 
$(n+1)t := \mbox{\mtt (}nt \mbox{\mtt +} t\mbox{\mtt )}$. Every term 
in the variables {\mtt x$_i$}, $i < n$, is equivalent to 
a term {\mtt x$_0$+$\sum_{i < n}a_i$x$_i$}, where $b, a_i \in \omega$, 
$i < n$. A subset $S$ of $\BZ^n$ is \textbf{definable} 
%%%%
\index{definability}%%
%%%%
if there is a formula $\varphi(\mbox{\mtt x}_0, \mbox{\mtt x}_1, \dotsc, %
\mbox{\mtt x}_{n-1})$ such that 
%%
\begin{equation}
S = \{\auf k_i : i < n\zu \in \BZ^n : \uli{\BZ} \vDash 
	\varphi[k_0, k_1, \dotsc, k_{n-1}]\}
\end{equation}
%%%
The definable subsets of $\BZ^n$ are closed under union, intersection 
and complement and permutation of the coordinates. Moreover, if 
$S \subseteq \BZ^{n+1}$ is definable, so is its projection 
%%
\begin{multline}
\pi_n[S] := \{\auf k_i : i < n\zu : \text{ there is }
k_n \in \BZ \text{ such that } \\
		\auf k_i : i < n+1\zu \in S\}
\end{multline}
%%%
The same holds for definable subsets of $\omega^n$, which are simply 
those definable subsets of $\BZ^n$ that are included in $\omega^n$.
Clearly, if $S \subseteq \BZ^n$ is definable, so is $S \cap \omega^n$.
%%%
\begin{lem}
Suppose that $a + \sum_{i < n} p_i x_i = b + \sum_{i < n} q_i x_i$ 
is a linear equation with rational numbers $a$, $b$, $p_i$ and 
$q_i$ ($i < n$). Then there is an equation 
%%%
\begin{equation}
g + \sum_{i < n} u_i x_i = h + \sum_{i < n} v_i x_i
\end{equation}
%%% 
with the same solutions, but with positive integer coefficients such 
that $g \cdot h = 0$ and for every $i < n$: $v_i u_i = 0$.
\end{lem}
%%%
\proofbeg
First, multiply with the least common denominator to transform the 
equation into an equation with integer coefficients. Next, add 
$-p_ix_i$ to both sides if $p_i < 0$, unless $q_i < p_i < 0$, in 
which case we add $- q_i x_i$. Now all coefficients are positive. 
Next, for every $i < n$, substract $q_i x_i$ from both sides if 
$p_i > q_i$ and $p_i x_i$ otherwise. These transformations preserve 
the set of solutions.
\proofend
%%%

Call an equation \textbf{reduced} if it has the form 
%%%
\index{equation!reduced}%%%
%%%%
%%
\begin{equation}
g + \sum_{i < m} k_i x_i = \sum_{m \leq i < n} k_i x_i
\end{equation}
%%
with positive integer coefficients $g$ and $k_i$, $i < n$. 
Likewise for an inequation. Evidently, modulo renaming of 
variables we can transform every rational equation into 
reduced form.
%%%
\begin{lem}
\label{lem:eq}
The set of solutions of a reduced equation is semilinear.
\end{lem}
%%%
\proofbeg
Let $\mu$ be the least common multiple of the $k_i$. 
Consider a vector of the form $\vec{c}_{i,j} = (\mu/k_i)\vec{e}_i + 
(\mu/k_j)\vec{e}_j$, where $i < m$ and $m \leq j < n$. Then 
if $\vec{v}$ is a solution, so is $\vec{v} + \vec{c}_{i,j}$ and conversely. 
Put $C := \{\vec{c}_{i,j} : i < m \leq j < n\}$ and 
%%
\begin{equation}
P := \left\{\vec{u} : g + \sum_{i < m} k_i \vec{u}(i) = \sum_{m \leq i 
< n} k_i \vec{u}(i), \vec{u}(i) < \mu/k_i\right\}
\end{equation}
%%
Both $P$ and $C$ are finite. Moreover, the set of solutions is 
exactly $\Sigma(P;C)$. 
\proofend
%%%
\begin{lem}
\label{lem:ineq}
The set of solutions of a reduced inequation is semilinear.
\end{lem}
%%
\proofbeg
Assume that the inequation has the form
%%
\begin{equation}
g + \sum_{i < m} k_i x_i \leq \sum_{m \leq i < n} k_i x_i
\end{equation}
%%%
Define $C$ and $P$ as before. Let $E := \{\vec{e}_i : m \leq i < n\}$.
Then the set of solutions is $\Sigma(P;C \cup E)$. If the inequation 
has the form
%%
\begin{equation}
g + \sum_{i < m} k_i x_i \geq \sum_{m \leq i < n} k_i x_i
\end{equation}
%%%
The set of solutions is $\Sigma(P;C \cup F)$, where 
$F := \{\vec{e}_i : i < m\}$.
\proofend
%%%
\begin{lem}
Let $M \subseteq \BQ^n$ be an affine subspace. Then $M \cap \BZ^n$ 
is a semilinear subset of $\BZ^n$.
\end{lem}
%%%
\proofbeg
Let $\vec{v}_i$, $i < m+1$, be vectors such that 
%%
\begin{equation}
M = \vec{v}_0 + \BQ \vec{v}_1 + \BQ \vec{v}_2 + \dotsb + 
\BQ \vec{v}_{m-1}
\end{equation}
%%
We can assume that the $\vec{v}_i$ are linearly independent. 
Clearly, since $\BQ \vec{w} = \BQ (\lambda \vec{w})$ for any 
nonzero rational number $\lambda$, we can assume that 
$\vec{v}_i \in \BZ^n$, $i < m$.
Now, put 
%%%
\begin{equation}
V := \{\vec{v}_0 + \sum_{0 < i < m} \lambda_i \vec{v}_i : 
0 \leq \lambda_i < 1\}
\end{equation}
%%%
$V \cap \BZ^n$ is finite. Moreover, if 
$\vec{v}_{0} + \sum_{0 < i < m} \kappa_i \vec{v}_i \in \BZ^n$ 
then $\vec{v}_{0} + \sum_{0 < i < m} \kappa_i' \vec{v}_i 
\in \BZ^n$ if $\kappa_i - \kappa_i' \in \BZ$. Hence, 
%%%
\begin{equation}
M = \bigcup_{\vec{w} \in V} \left(\vec{w} + 
	\BZ \vec{v}_1 + \dotsb + \BZ \vec{v}_m\right)
\end{equation}
%%
This is a semilinear set.
\proofend
%%%
\begin{lem}
Let $M \subseteq \BZ^n$ be a semilinear subset of $\BZ^n$. Then 
$M \cap \omega^n$ is semilinear. 
\end{lem}
%%%
\proofbeg
It suffices to show this for linear subsets.
Let $\vec{v}_i$, $i < m+1$, be vectors such that 
%%
\begin{equation}
M = \vec{v}_0 + \BZ \vec{v}_1 + \BZ \vec{v}_2 + \dotsb + 
\BZ \vec{v}_{m-1}
\end{equation}
%%
Put $\vec{w}_i := - \vec{v}_i$, $0 < i < m$. Then 
%%
\begin{equation}
M = \vec{v}_0 + \omega \vec{v}_1 + \omega \vec{v}_2 + \dotsb + 
\omega \vec{v}_{m-1} + \omega \vec{w}_1 + \dotsb + \omega \vec{w}_{m-1}
\end{equation}
%%
Thus, we may without loss of generality assume that 
%%
\begin{equation}
M = \vec{v}_0 + \omega \vec{v}_1 + \omega \vec{v}_2 + \dotsb + 
\omega \vec{v}_{m-1}
\end{equation}
%%
Notice, however, that these vectors are not necessarily in $\omega^n$. 
For $i$ starting at 1 until $n$ we do the following. 

Let $x^i_j := \vec{v}_j(i)$. Assume that for $0 < j < p$ we have 
$x^i_j \geq 0$, and that for $p \leq j < m$ we have $x^i_j > 0$. 
(A renaming of the variables can achieve this.) We introduce new 
cyclic vectors $\vec{c}_{j,k}$ for $0 < j < p$ and $p \leq k < m$. 
Let $\mu$ the least common multiple of the $|x^i_s|$, for all 
$0 < s < m$ where $x^i_s \neq 0$. 
%%%
\begin{equation}
\vec{c}_{i,j} := (\mu/x^i_j) \vec{v}_j + (\mu/x^i_k)\vec{v}_k
\end{equation}
%%%
Notice that the $s$--coordinates of these vectors are positive 
for $s < i$, since this is a positive sum of positive numbers. 
The $i$th coordinate of these vectors is 0. Suppose that the 
$i$th coordinate of 
%%
\begin{equation}
\vec{w} = \vec{v}_0 + \sum_{0 < j < m} \lambda_j \vec{v}_j
\end{equation}
%%
is $\geq 0$, where $\lambda_j \in \omega$ for all $0 < j < m$. Suppose 
further that for some $k \geq p$ we have 
$\lambda_k \geq v^i_0 + m(\mu/|x^i_k|)$. 
Then there must be a $j < p$ such that $\lambda_j \geq 
(\mu/x^i_j)$. Then put $\lambda_r' := \lambda_r$ for 
$r \neq j,k$, $\lambda_j' := \lambda_j - (\mu/x^i_j)$ and 
$\lambda_k' := \lambda_k + (\mu/x^i_k)$. Then
%%
\begin{equation}
\vec{w} = \vec{c}_{j,k} + \sum_{0 < j < m} \lambda_j' \vec{v}_j
\end{equation}
%%
Moreover, $\lambda_j' \leq \lambda_j$ for all $j < p$, and 
$\lambda_k' < \lambda_k$. Thus, by adding these cyclic vectors 
we can see to it that the coefficients of the $\vec{v}_k$ 
for $p \leq k < m$ are bounded. Now define $P$ to be the set of 
all $\vec{w}$ which have a decomposition
%%%
\begin{equation}
\vec{w} = \vec{v}_0 + \sum_{0 < j < m} \lambda_j \vec{v}_j \in \omega^n
\end{equation}
%%
where $\lambda_j < v^j_0 + m |\mu/x^i_j|$ for all $0 < j < m$.
%%%
Then 
%%%
\begin{equation}
M \cap \omega^n = \bigcup_{\vec{u} \in P} \left(\vec{u} + 
	\sum_{0 < j < p} \lambda_j \vec{v}_j + 
	\sum_{0 < j < p \leq k < m} \kappa_{j,k} \vec{c}_{j,k}\right)
\end{equation}
%%%
with all $\lambda_j$, $\kappa_{j,k} \geq 0$. Now we have achieved that 
all $j$th coordinates of vectors are positive. 
%%
\proofend

The following is now immediate. 
%%%
\begin{lem}
\label{lem:QtoN}
Let $M \subseteq \BQ^n$ be an affine subspace. Then $M \cap \omega^n$ 
is a semilinear subset of $\omega^n$.
\end{lem}
%%
\begin{lem}
\label{lem:intersection}
The intersection of semilinear sets is again semilinear.
\end{lem}
%%%
\proofbeg
It is enough to show the claim for linear sets. So, let $S_0$ and 
$S_1$ be linear. Then there are $C_0 = \{\vec{u}_i : i < m\}$ and 
$C_1 = \{\vec{v}_i : i < n\}$ and $\vec{u}$ and $\vec{v}$ such 
that $S_0 = \Sigma(\{\vec{u}\}; C_0)$ and $S_1 := \Sigma(\{\vec{v}\}; C_1)$. 
Notice that $\vec{w} \in S_0 \cap S_1$ iff there are 
natural numbers $\kappa_i$ ($i < m$) and $\lambda_j$ ($j < n$) such that 
%%
\begin{equation}
\vec{w} = \vec{u} + \sum_{i < m} \kappa_i \vec{u}_i = \vec{v} + 
\sum_{i < n} \lambda_i \vec{v}_i
\end{equation}
%%
So, we have to show that the set of these $\vec{w}$ is semilinear. 

The equations are now taken as linear equations with $\kappa_i$, 
$i < m$ and $\lambda_i$, $i < n$, as variables. Thus we have 
equations  for $m + n$ variables. We solve these equations first 
in $\BQ^{m+n}$. The solutions form an affine subspace $V \subseteq \BQ^{m+n} 
\cong \BQ^m \oplus \BQ^n$. By Lemma~\ref{lem:QtoN}, 
$V \cap \omega^{m+n}$ is semilinear, and so is its projection 
onto $\omega^m$ (or to $\omega^n$ for that matter). Let it be 
$\bigcup_{i < p} L_i$, where for each $i < p$, $L_i \subseteq 
\omega^m$ is linear. Thus there is a representation of $L_i$ as 
%%
\begin{equation}
L_i = \vec{\theta} + \omega \vec{\eta}_0 + \dotsb +
\omega \vec{\eta}_{\gamma-1}
\end{equation}
%%
Now put 
%%%
\begin{equation}
W_i := \{\vec{u} + \sum_{i < m} \vec{\kappa}(i) \vec{u}_i : 
	\vec{\kappa} \in L_i\}
\end{equation}
%%%
From the construction we get that 
%%
\begin{equation}
S_0 \cap S_1 = \bigcup_{i < p} W_i
\end{equation}
%%%
Define vectors $\vec{q}_i := \sum_{j < m} \vec{\eta}_i(j) \vec{u}_i$, 
$i < \gamma$ and $\vec{r} := \vec{c} + \sum_{j < m} \vec{\theta}(j)
\vec{u}_i$. Then 
%%%
\begin{equation}
W_i = \vec{r} + \omega \vec{q}_0 + \dotsb + \omega \vec{q}_{\gamma-1}
\end{equation}
%%%
So, the $W_i$ are linear. This shows the claim.
\proofend
%%%
\begin{lem}
If $S \subseteq \omega^n$ is semilinear, so is its projection $\pi_n[S]$.
\end{lem}
%%%
We need one more prerequisite. Say that a first--order theory $T$ has 
\textbf{quantifier elimination} 
%%%
\index{quantifier elimination}%%%
%%%%
if for every formula $\varphi(\vec{x})$ 
there exists a quantifier free formula $\chi(\vec{x})$ such that 
$T \vdash \varphi(\vec{x})\boldsymbol{\dpf}\chi(\vec{x})$. We follow 
the proof of \cite{monk:logic}. 
%%%
\index{Presburger}%%%
%%%
\begin{thm}[Presburger]
\label{thm:qe}
Presburger Arithmetic has quantifier elimination. 
\end{thm}
%%%
\proofbeg
It is enough to show that for every formula $\mbox{\mtt 
(\symbol{21}x}_{\snull}\mbox{\mtt)}\varphi(\vec{y}, %
\mbox{\mtt x}_{\snull})$ with $\varphi(\vec{y},x)$ quantifier free 
there exists a quantifier free formula $\chi(\vec{y})$ such that 
%%%
\begin{equation}
\BZ \vDash \mbox{\mtt (\symbol{20}$\vec{y}$)(\symbol{21}x$_{\snull}$%
)($\varphi(\vec{y},\mbox{\mtt x}_0)\boldsymbol{\dpf}\chi(\vec{y})$)}
\end{equation}
%%%
We may further eliminate negation (see the remarks above) 
and disjunctions inside $\varphi(\vec{y},x)$ (since 
{\mtt (\symbol{21}x$_{\snull}$)($\alpha$\symbol{31}$\beta$)} is 
equivalent with {\mtt ((\symbol{21}x$_{\snull}$)$\alpha$)\symbol{31}%
((\symbol{21}x$_{\snull}$)$\beta$)}. Finally, we may assume that all 
conjuncts contain {\mtt x$_{\snull}$}. For if $\alpha$ does not contain 
{\mtt x$_{\snull}$} free, {\mtt (\symbol{21}x$_{\snull}$)($\alpha$\symbol{4}%
$\beta$)} is equivalent to {\mtt ($\alpha$\symbol{4}(\symbol{21}x$_{\snull}$%
)$\beta$)}. So, $\varphi$ can be assumed to be a conjunction 
of atomic formulae of the following form: 
%%%
\begin{multline}
\mbox{\mtt (\symbol{21}x$_{\snull}$)($\gund_{i < p} n_i$x$_{\snull}$%
\mbox{\mtt\symbol{61}}$t_i$\symbol{4}$\gund_{i < q} 
n'_i$x$_{\snull}$<$t'_i$\symbol{4}$\gund_{i < r} n''_i$x$_{\snull}$>$t''_i$}
\\
\mbox{\mtt \symbol{4}$\gund_{i < s} n'''_i$x$_{\snull}$\symbol{30}$_{m_i}%
t'''_i$)}
\end{multline}
%%%
Since {\mtt $s\,$\symbol{30}$_mt$} is equivalent with 
{\mtt $ns\,$\symbol{30}$_m nt$}, so after suitable 
multiplication we may see to it that all the $n_i$, $n'_i$, $n''_i$ 
and $n'''_i$ are the same number $\nu$. 
%%%
\begin{multline}
\mbox{\mtt (\symbol{21}x$_{\snull}$)($\gund_{i < p} \nu$x$_{\snull}$%
\mbox{\mtt\symbol{61}}$\tau_i$\symbol{4}% 
$\gund_{i < q} \nu$x$_{\snull}$<$\tau'_i$\symbol{4}%
$\gund_{i < r} \nu$x$_{\snull}$>$\tau''_i$} \\
\mbox{\mtt \symbol{4}$\gund_{i < s}\nu$x$_{\snull}%
$\symbol{30}$_{m_i} \tau'''_i$)}
\end{multline}
%%%
We may rewrite the formula in the following way (replacing 
{\mtt $\nu$x$_{\snull}$} by {\mtt x$_{\snull}$} and adding 
instead the condition that {\mtt x$_{\snull}$} is divisible by $\nu$). 
%%%
\begin{multline}
\mbox{\mtt (\symbol{21}x$_{\snull}$)(x$_{\snull}$\symbol{30}$_{\nu}%
$0\symbol{4}%
$\gund_{i < p}$x$_{\snull}$\mbox{\mtt\symbol{61}}$\tau_i$\symbol{4}%
$\gund_{i < q}$x$_{\snull}$<$\tau'_i$\symbol{4}% 
$\gund_{i < r}$x$_{\snull}$>$\tau''_i$} \\%
\mbox{\mtt \symbol{4}$\gund_{i < s}$x$_{\snull}$\symbol{30}$_{m_i}\tau'''_i$)}
\end{multline}
%%%
Assume that $p > 0$. Then the first set of conjunctions is equivalent 
with the conjunction of 
$\gund_{i < j < p} \tau_i\mbox{\mtt\symbol{61}}\tau_j$ (which 
does not contain {\mtt x$_{\snull}$}) and 
{\mtt x$_{\snull}$\mbox{\symbol{61}}$\tau_0$}. We may therefore 
eliminate all occurrences of {\mtt x$_{\snull}$} by $\tau_0$ in 
the formula. 

Thus, from now on we may assume that $p = 0$. Furthermore, notice that 
{\mtt (x$_{\snull}$<$\sigma$\symbol{4}x$_{\snull}$<$\tau$)} is equivalent to
{\mtt (x$_{\snull}$<$\sigma$\symbol{4}$\sigma$\symbol{28}$\tau$)\symbol{31}(x%
$_{\snull}$<$\tau$\symbol{4}$\tau$<$\sigma$)}. This means that we can assume 
$q \leq 1$, and likewise that $r \leq 1$. Next we show that we can 
actually have $s \leq 1$. To see this, notice the following. 
%%%%
\begin{quote}
Let $u,v,w,x$ be integers, $w, x > 1$, and let $p$ be the least 
common multiple of $w$ and $x$. Then $\gcd(p/w, p/x) = 1$, 
and so there exist integers $m, n$ such that  $1 = m \cdot p/w + 
n \cdot p/x$. It follows that the following are equivalent. 
%%%%
\begin{dingautolist}{192}
\item $y \equiv u \pmod{w}$ and $y \equiv v \pmod{x}$ 
\item $u \equiv v \pmod{\gcd(w,x)}$ and 
	$y \equiv m(p/w)u + n(p/x)v \pmod{p}$. 
\end{dingautolist} 
\end{quote}
%%%%
The Euclidean algorithm yields numbers $m$ and $n$ as required 
(see \cite{jones:numbertheory}). Now suppose that the first obtains.  
Then $y - u = ew$ and $y - v = fx$ for some numbers $e$ and $f$. 
Then $u - v = fx - ew$, which is divisible by $\gcd(x,w)$. 
So, $u \equiv v \pmod{\gcd(w,x)}$. Furthermore, 
%%%
\begin{align}
y - m(p/w)u - n(p/x)v & = m(p/w)y + n(p/x)y \\\notag
		& \quad - m(p/w)u - n(p/x)v \\\notag
		& = m(p/w)(y - u) \\\notag
		& \quad + n(p/x)(y - v) \\\notag
		& = m(p/w)em + n(p/x)fn  \\\notag
		& \equiv 0 \pmod{p} 
\end{align}
%%%
So, the second holds. Conversely, if the second holds, then for 
some $k$ we have $u - v = k \gcd(w,x)$. Then 
%%%
\begin{align}
y - u           & = y - m(p/w)u - n(p/x)u \\\notag
		& = y - m(p/w)u - n(p/x)v - 
		n(p/x)k \cdot \gcd(m,n)\\\notag
		& \equiv 0 \pmod{w} 
\end{align}
%%%
Analogously $y \equiv v \pmod{x}$ is shown. 
		  
Using this equivalence we can reduce the congruence statements to 
a conjunction of congruences where only one involves {\mtt x$_{\snull}$}. 

This leaves us with 8 possibilities. If $r = 0$ or $s = 0$ the 
formula is actually trivially true. So, 
{\mtt (\symbol{21}x$_{\snull}$)(x$_{\snull}$<$\tau$)}, 
{\mtt (\symbol{21}x$_{\snull}$)($\upsilon$<x$_{\snull}$)}, 
{\mtt (\symbol{21}x$_{\snull}$)(x$_{\snull}$\symbol{61}$_m\xi$)}, 
as well as
{\mtt (\symbol{21}x$_{\snull}$)(x$_{\snull}$<$\tau$\symbol{4}x%
$_{\snull}$\symbol{30}$_m\xi$)} 
and {\mtt (\symbol{21}x$_{\snull}$)($\upsilon$<x$_{\snull}%
$\symbol{4}x$_{\snull}$\symbol{30}$_m \xi$)}  can all be dropped 
or replaced by $\boldsymbol{\top}$. Finally, {\mtt (\symbol{21}x%
$_{\snull}$)(x$_{\snull}$<$\tau$\symbol{4}$\upsilon$<x$_{\snull}$)} 
is equivalent with  {\mtt $\upsilon$+1<$\tau$} and 
{\mtt (\symbol{21}x$_{\snull}$)(x$_{\snull}$<$\tau$\symbol{4}$\upsilon$%
<x$_{\snull}$\symbol{4}x$_{\snull}$\symbol{61}$_m \xi$)} is equivalent with
{\mtt $\goder_{i < m}$($\tau$+1+$i$<$\upsilon$\symbol{4}% 
$\tau$+1+$i$\symbol{30}$_m \xi$)}.
This shows the claim.
\proofend
%%%
\nocite{ginsburgspanier:presburger}
\nocite{ginsburgspanier:semilinear}
\index{Ginsburg, Seymour}%%
\index{Spanier, Edwin H.}%%%
\begin{thm}[Ginsburg \& Spanier]
\label{thm:semabschluss}
A subset of $\omega^n$ is se\-mi\-li\-ne\-ar iff it is definable 
in Presburger Arithmetic. 
\end{thm}
%%%
\proofbeg
($\Pf$) Every semilinear set is definable in Presburger Arithmetic.
To see this it is enough to show that linear sets are definable. For 
if $M$ is a union of $N_i$, $i < p$, and each $N_i$ is linear and 
hence definable by a formula $\varphi_i(\vec{x})$, then $M$ is 
definable by $\goder_{i < p} \varphi_i(\vec{x})$. Now let 
$M = \vec{v} + \omega \vec{v}_0 + \dotsb + \omega \vec{v}_{m-1}$ be 
linear. Then put 
%%%
\begin{multline}
\varphi(\vec{x}) := \mbox{\mtt (\symbol{21}x$_n$)(\symbol{21}%
x$_{n+1}$)$\dotsb$(\symbol{21}x$_{n+m-1}$)($\gund_{i < m}$0\symbol{28}%
x$_{n+i}$} \\
 	\mbox{\mtt \symbol{4}$\gund_{i < n}$($\vec{v}(i)$%
+$\sum_{j < m}$x$_{n+i}\vec{v}(i)_j$=x$_{i}$))}
\end{multline}
%%%
$\varphi(\vec{x})$ defines $M$. ($\Leftarrow$) Let $\varphi(\vec{x})$ be a 
formula defining $S$. By Theorem~\ref{thm:qe}, there exists a 
quantifier free formula $\chi(\vec{x})$ defining $S$. Moreover, 
as we have remarked above, $\chi$ can be assumed to be negation free. 
Thus, $\chi$ is a disjunction of conjunctions of atomic formulae. 
By Lemma~\ref{lem:intersection}, the set of semilinear subsets of 
$\omega^n$ is closed under intersection of members, and it is also closed  
under union. Thus, all we need to show is that atomic formulae 
define semilinear sets. Now, observe that 
{\mtt x$_{\snull}$\symbol{30}$_m$x$_{\seins}$} is equivalent to 
{\mtt (\symbol{21}x$_{\szwei}$)(x$_{\snull}$=x$_{\seins}$+$m$x$_{\szwei}$)}, 
which is semilinear, as it is the projection of 
{\mtt x$_{\snull}$=x$_{\seins}$+$m$x$_{\szwei}$} 
onto the first two components. 
\proofend
%%%
\vplatz
\exercise
Let $|A| = 1$. Show that $\GZ(A)$ is isomorphic to $\GM(A)$.
Derive from this that there are only countably many semilinear 
languages over $A$.
%%%
\vplatz
\exercise
Let $L \subseteq A^{\ast}$. Call $L$ \textbf{almost periodical} if there 
%%%
\index{language!almost periodical}%%%
%%%
are numbers $p$ (the modulus of periodicity) and $n_0$ such that 
for all $\vec{x} \in L$ with length $\geq n_0$ there is a string 
$\vec{y} \in L$ such that $|\vec{y}| = |\vec{x}| + p$. Show that 
a semilinear language is almost periodical.  
%%%
\vplatz %%
\exercise %%
\label{ex:semilincont}%%
Let $A = \{\mbox{\tt a},
\mbox{\tt b}\}$. Further, let $U := \mbox{\tt a}^{\ast} \cup
\mbox{\tt b}^{\ast}$. Now let $N \subseteq M(A)$ be a set such
that $N - U$ is infinite. Show that there are $2^{\aleph_0}$ many
languages $L$ with $\mu[L] = N$. (The cardinality of
$A^{\ast}$ is $\aleph_0$, hence there can be no more than
$2^{\aleph_0}$ such languages. The exercise consists in 
showing that there are no less of them either.)
%%%%
\vplatz
\exercise
Show that semilinear languages have the following pumping property: 
{\it For every semilinear set $V \subseteq \omega^n$ there exists a 
number $n$ such that if $\vec{v} \in V$ has length $\geq n$, there 
exist $\vec{w}$ and $\vec{x}$ such that $\vec{v} = \vec{w} + \vec{x}$ 
and $\vec{w} + \omega\vec{x} \subseteq V$.}
%%%
\vplatz
\exercise
\label{ex:omega}
Let $\Omega \subseteq \omega$. Let $V_{\Omega} \subseteq \omega^2$ 
be defined by
%%
\begin{equation}
V_{\Omega} := \{\auf m, n\zu : m \neq n \text{ or }m \in \Omega\}
\end{equation}
%%%
Show that $V_{\Omega}$ satisfies the pumping property of the previous 
exercise. Show further that $V_{\Omega}$ is semilinear iff $\Omega$ is. 
%%%
\vplatz
\exercise
%%%
\index{Presburger Arithmetic}%%%
%%%
Show that for every sentence $\varphi$ of Presburger Arithmetic 
it is decidable whether or not it is true in $\BZ$. 
{\it Hint.} Use quantifiers elimination and the fact that the 
elimination is constructive.

 \section{Parikh's Theorem}
\label{kap2-5}
%%%
Now we shall turn to the already announced embedding of context
free tree sets into tree sets generated by UTAGs.
(The reader may wonder why we speak of sets and not
of classes. In fact, we shall tacitly assume that trees are
really tree domains, so that classes of finite trees are automatically
sets.) Let $G = \auf \mbox{\tt S}, N, A, R\zu$  be a CFG. We
want to define a tree adjunction grammar $\mathsf{Ad}_G =
\auf \BC_G, N, A, \BA_G\zu$ such that $L_B(G) = L_B(\mathsf{Ad}_G)$.
We define $\BC_G$ to be the set of all (ordered labelled) tree (domains)
$\GB$ which can be generated by $L_B(G)$ and which are centre trees
and in which on no path not containing the root some nonterminal symbol
occurs twice. Since there are only finitely many symbols and the
branching is finite, this set is actually finite. Now we define
$\BA_G$. Let $\BA_G$ contain all adjunction trees $\GB_X$, $X \in N$, 
(modulo identification of $Y^0, Y^1$ with $Y$ for all $Y \in N$)
such that (1) $\GB_X$ can be derived from $X$ in $\gamma G$,  
(2) no symbol occurs twice along a path that does contain the root.  
Also $\BA_G$ is finite. It is not hard to show that 
$L_B(\mathsf{Ad}_G) \subseteq L_B(G)$. The reverse inclusion 
we shall show by induction on the number of nodes in the tree 
(domain). Let $\GB$ be in $L_B(G)$. Either there is a path
not containing the root along which some symbol occurs twice, or
there is not.  In the second case the tree is in $\BC_G$.
Hence $\GB \in L_B(\mathsf{Ad}_G)$ and we are done.
In the first case we choose an $x \in B$ of minimal height such that
there is a $y < x$ with identical label; let the label be $X$.
Consider the subtree $\GU$ induced by the set
$(\low{x} - \low{y}) \cup \{y\}$. We claim that
$\GU \in \BA_G$.  For this we have to show the following.
(a) $\GU$ is an adjunction tree, (b) $\GU$ can be deduced from $X$,
(c) no symbol symbol occurs twice along a path  which does not contain
$x$. Ad (a). A leaf of $\GU$ is either a leaf of $\GB$ or $= y$. In
the first case the label is a terminal symbol in the second case
it is identical to that of the root. Ad (b). If $\GB$ is a tree of
$\gamma G$ then $\GU$ can be derived from $X$. Ad (c).
Let $\pi$ be a path which does not contain $x$ and let
$u,v \in \pi$ nodes with identical label and $u < v$.
Then $v < x$, and this contradicts the minimality of
$x$. Hence all three conditions are met.  So we can disembed
$\GU$. This means that there is a tree $\GB'$ such that $\GB$
is derived from $\GB'$ by adjoining $\GU$. We have $\GB' \in L_B(G)$
and by induction hypothesis $\GB' \in L_B(\mathsf{Ad}_G)$.
Hence $\GB \in L_B(\mathsf{Ad}_G)$, which had to be shown.
\nocite{joshilevytakahashi:adjunct}
%%
\index{Joshi, Aravind}%%%
\index{Takahashi, Masako}%%
\index{Levy, Leon S.}%%%
%%%
\begin{thm}[Joshi \& Levy \& Takahashi]
Every set of labelled ordered tree domains generated by a
CFG is also one generated by a UTAG.
\proofend
\end{thm}
%%
Now we shall prove Parikh's Theorem for UTAGs.
Let $\alpha$ be a letter and $\GB$ a tree. Then 
$\sigma_{\alpha}(\GB)$ 
%%%%
\index{$\sigma_a(\GB)$, $\mu(\GB)$}%%%
%%%%
is the number of nodes whose label 
is $\alpha$. If $\GB$ is an adjunction tree then the label 
of the root is {\it not counted}. Now let $\auf \BC, N, A, \BA\zu$ 
be a UTAG and $\BC = \{\GC_i : i < \alpha\}$, $\BA = \{\GA_j : j < \beta\}$.
%%
\begin{lem}
Let $\GB'$ result from $\GB$ by adjoining the tree
$\GA$. Then $\sigma_{\alpha}(\GB') = \sigma_{\alpha}(\GB)
+ \sigma_{\alpha}(\GA)$.
\end{lem}
%%
The proof of this lemma is easy. From this it follows that
we only need to know for an arbitrarily derived tree how
many times which tree has been adjoined and what the starting
tree was. So let $\GB$ be a tree which resulted from $\GC_i$
by adjoining $\GA_j$ $p_j$ times, $j < \beta$. Then
%%
\begin{equation}\sigma_{\alpha}(\GB) = \sigma_{\alpha}(\GC_i) +
\sum_{i < \beta} p_j \cdot \sigma_{\alpha}(\GA_j)
\end{equation}
%%
Let now $\mu(\GB) := \sum_{a \in A} \sigma_a(\GB) \cdot a$.
Then 
%%
\begin{equation}
\mu(\GB) = \mu(\GC_i) + \sum_{i < \beta} p_j \cdot \mu(\GB_j)
\end{equation}
%%
We define the following sets
%%
\begin{equation}
\Sigma_i := \mu(\GC_i) + \sum_{j < \beta} \omega \mu(\GA_j) 
\end{equation}
%%
Then $\mu[L_B(\auf \BC, \BA\zu)] \subseteq \bigcup_{i < n} \Sigma_i$.
However, equality need not always hold. We have to notice the following
problem. A tree $\GA_j$ can be adjoined to a tree $\GB$ only if its root
label actually occurs in the tree $\GB$. Hence not all values of
$\bigcup \Sigma_i$ are among the values under $\mu$ of a derived
tree. However, if a tree can be adjoined {\it once\/} it can
be adjoined any number of times and to all trees that result
from this tree by adjunction. Hence we modify our starting set
of trees somewhat. We consider the set $D$ of all pairs
$\auf k, W\zu$ such that $k < \alpha$, $W \subseteq \beta$ and
there is a derivation of a tree that starts with $\GC_k$ and
uses exactly the trees from $W$. For $\auf k, W\zu \in D$
%%
\begin{equation}
L(k,W) = \mu(\GC_i) + \sum_{j \in W} \omega \cdot \mu(\GA_j)
\end{equation}
%%
Then $L := \bigcup \auf L(k,W) : \auf k,W\zu \in D\zu$
is semilinear. At the same time it is the set of
all $\mu(\GB)$ where $\GB$ is derivable from $\auf \BC, N, A, %
\BA\zu$. 
%%
\begin{thm}
Let $L$ be the language of an unregulated tree adjunction grammar
then $L$ is semilinear.
\proofend
\end{thm}
%%
\index{Parikh, Rohit}%%
%%%
\begin{cor}[Parikh]
Let $L$ be context free. Then $L$ is semilinear. \proofend
\end{cor}
%%
This theorem is remarkable is many respects. We shall
meet it again several times. Semilinear sets are closed under
complement (Theorem~\ref{thm:semabschluss}) and hence also
under intersection. We shall show, however, that this does not hold
for semilinear languages.
%%
\begin{prop}
There are CFLs $L_1$ and $L_2$ such that $L_1 \cap L_2$ is not semilinear.
\end{prop}
%%
\proofbeg
Let $M_1 := \{\mbox{\tt a}^n \mbox{\tt b}^n :
n \in \omega\}$ and $M_2 := \{\mbox{\tt b}^n \mbox{\tt a}^{2n} :
n \in \omega\}$. Put
%%
\begin{align}
L_1 & := \mbox{\tt b} M_1^{\ast} \mbox{\tt a}^{\ast} &
L_2 & := M_2^+
\end{align}
%%
Because of Theorem~\ref{thm:afl} $L_1$ and $L_2$ are context free.
Now look at $L_1 \cap L_2$. It is easy to see that the intersection
consists of the following strings.
%%
\begin{equation}
\mbox{\tt ba}^2, \quad \mbox{\tt ba}^2 \mbox{\tt b}^2\mbox{\tt a}^4,
\quad \mbox{\tt ba}^2\mbox{\tt b}^2\mbox{\tt a}^4\mbox{\tt
    b}^4\mbox{\tt a}^8, \quad
\mbox{\tt ba}^2\mbox{\tt b}^2 \mbox{\tt a}^4\mbox{\tt
    b}^4 \mbox{\tt a}^8 \mbox{\tt b}^8 \mbox{\tt a}^{16}, \dotsc
\end{equation}
%%
The Parikh image is $\{(2^{n+2}-2)\mbox{\tt a} +
(2^{n+1} - 1)\mbox{\tt b} : n \in \omega\}$.  This set is not
semilinear, since the result of deleting the symbol {\tt b} (that
is, the result of applying the projection onto
$\mbox{\tt a}^{\ast}$) is not almost periodical.
\proofend

We know that for every semilinear set $N \subseteq M(A)$ there
is a regular grammar $G$ such that $\mu[L(G)] = N$. However
$G$ can be relatively complex. Now the question arises whether
the complete preimage $\mu^{-1}[N]$ under $\mu$ is at least
regular or context free. This is not the case. However, we
do have the following.
%%
\begin{thm}
\label{thm:urbild}
The full preimage of a semilinear set over a single letter
alphabet is regular.
\end{thm}
%%
This is the best possible result. The theorem becomes false as soon
as we have two letters.
%%
\begin{thm}
The full preimage of $\omega (\mbox{\tt a} + \mbox{\tt b})$
is not regular; it is however context free. The full
preimage of $\omega (\mbox{\tt a} + \mbox{\tt b} + \mbox{\tt c})$
is not context free.
\end{thm}
%%
\proofbeg
We show the second claim first. Let
%%
\begin{equation}
W := \mu^{-1}[\omega (\mbox{\tt a} + \mbox{\tt b} +
\mbox{\tt c})] 
\end{equation}
%%
Assume that $W$ is context free. Then the intersection with the
regular language $\mbox{\tt a}^{\ast}\mbox{\tt b}^{\ast}
\mbox{\tt c}^{\ast}$ is again context free. This is precisely
the set $\{\mbox{\tt a}^n \mbox{\tt b}^n \mbox{\tt c}^n : n \in %
\omega\}$. Contradiction. Now for the first claim. Denote by
$b(\vec{x})$ the number of occurrences of {\tt a} in $\vec{x}$
minus the number of occurrences of {\tt b} in $\vec{x}$. Then
$V := \{\vec{x} : b(\vec{x}) = 0\}$ is the full preimage of
$\omega (\mbox{\tt a} + \mbox{\tt b})$. $V$ is not regular;
otherwise the intersection with $\mbox{\tt a}^{\ast} %
\mbox{\tt b}^{\ast}$ is also regular. However, this is 
$\{\mbox{\tt a}^n \mbox{\tt b}^n : n \in \omega\}$.
Contradiction. However, $V$ is context free. To show this we 
shall construct a CFG $G$ over $A = \{\mbox{\tt a}, \mbox{\tt b}\}$ 
which generates $V$. We have three
nonterminals, {\tt S}, {\tt A}, and {\tt B}. The rules are
%%
\begin{equation}
\begin{array}{l@{\quad\pf\quad}l}
\mbox{\tt S} & \mbox{\tt SS} \mid \mbox{\tt AB} \mid \mbox{\tt BA}
    \\
\mbox{\tt A} & \mbox{\tt AS} \mid \mbox{\tt SA} \mid \mbox{\tt a} \\
\mbox{\tt B} & \mbox{\tt BS} \mid \mbox{\tt SB} \mid \mbox{\tt b} \\
\end{array}
\end{equation}
%%
The start symbol is {\tt S}. We claim:
$\mbox{\tt S} \vdash_G \vec{x}$ iff $b(\vec{x}) = 0$,
$\mbox{\tt A} \vdash_G \vec{x}$ iff
$b(\vec{x}) = 1$ and $\mbox{\tt B} \vdash_G \vec{x}$ iff
$b(\vec{x}) = - 1$. The directions from left to right are
easy to verify. It therefore follows that
$V \subseteq L(G)$. The other directions we show by induction
on the length of $\vec{x}$. It suffices to show the following
claim.
%%
\begin{quote}
If $b(\vec{x}) \in \{1,0,-1\}$ there are
$\vec{y}$ and $\vec{z}$ such that $|\vec{y}|, |\vec{z}| <
|\vec{x}|$ and such that $\vec{x} = \vec{y}\,\vec{z}$ as well as
$|b(\vec{y})|, |b(\vec{z})| \leq 1$.
\end{quote}
%%
Hence let $\vec{x} = \prod_{i < n} x_i$  be given.
Define $k(\vec{x}, j) := b(^{(j)}\vec{x})$,
and $K := \{k(\vec{x}, j) : j < n+1\}$. As is easily seen,
$K = [m,m']$ with $m \leq 0$.
Further, $k(\vec{x},n) = b(\vec{x})$. (a) Let $b(\vec{x}) = 0$.
Then put $\vec{y} := x_0$ and $\vec{z} :=
\prod_{0 < i < n} x_i$. This satisfies the conditions.
(b) Let $b(\vec{x}) = 1$. Case 1: $x_0 = \mbox{\tt a}$.
Then put again $\vec{y} := x_0$ and $\vec{z} :=
\prod_{0 < i < n} x_i$. Case 2: $x_0 = \mbox{\tt b}$.
Then $k(\vec{x},1) = -1$ and there is a $j$ such that
$k(\vec{x}, j) = 0$. Put $\vec{y} := \prod_{i < j} x_i$,
$\vec{z} := \prod_{j \leq i < n} x_i$.  Since $0 < j < n$,
we have $|\vec{y}|, |\vec{z}| < |\vec{x}|$.
Furthermore, $b(\vec{y}) = 0$ and $b(\vec{z}) = 1$. (c)
$b(\vec{x}) = -1$. Similar to (b).
\proofend
%%
\vplatz
\exercise
Let $|A| = 1$ and $\SA\Sd$ be a UTAG. Show that the language 
generated by $\SA\Sd$ over $A^{\ast}$ is regular.
%%%
\vplatz
\exercise
Prove Theorem~\ref{thm:urbild}. {\it Hint.} Restrict
your attention first to the case that $A = \{\mbox{\tt a}\}$.
%%
\vplatz
\exercise
Let $N \subseteq M(A)$ be semilinear. Show that the full
preimage is of Type 1 (that is, context sensitive).
{\it Hint.} It is enough to show this for linear sets.
%%
\vplatz
\exercise
In this exercise we sketch an alternative proof of Parikh's Theorem.
Let $A = \{\mbox{\tt a}_i : i < n\}$ be an alphabet. In analogy to the
regular terms we define semilinear terms. (a) $\mbox{\tt a}_i$, $i < n$,
is a semilinear term, with interpretation $\{\vec{e}_i\}$. (b) If
$A$ and $B$ are semilinear terms, so is $A \oplus B$ with
interpretation $\{\vec{u} + \vec{v} : \vec{u} \in A,
\vec{v} \in B\}$, $A \cup B$, with interpretation $\{\vec{u} :
\vec{u} \in A \mbox{ or }\vec{u} \in B\}$ and
$\omega A$ with interpretation $\{k \vec{u} : k \in \omega,
\vec{u} \in A\}$. The first step is to translate a CFG
into a set of equations of the form
$X_i = C_i(X_0, X_1, \dotsc, X_{q-1})$, $q$ the number
of nonterminals, $C_i$ semilinear terms. This is done as follows.
Without loss of generality we can assume that in a rule $X \pf
\vec{\alpha}$, $\vec{\alpha}$ contains a given variable at most
once. Now, for each nonterminal $X$ let $X \pf \vec{\alpha}_i$,
$i < p$, be all the rules of $G$. Corresponding to these rules
there is an obvious equation of the form
%%
\begin{equation}
X = A \cup (B \oplus X) \text{ or } X = A
\end{equation}
%%
where $A$ and $B$ are semilinear terms that do not contain $X$.
The second step is to prove the following lemma:
%%
\begin{quote}
{\it Let $X = A \cup (B \oplus X) \cup (C \oplus \omega X)$,
with $A$, $B$ and $C$ semilinear terms not containing $X$. Then the
least solution of that equation is $A \cup \omega B \cup \omega C$.
If $B \oplus X$ is missing from the equation, the solution is
$A \cup \omega C$, and if $C \oplus \omega X$ is missing the
solution is $A \cup \omega B$.}
\end{quote}
%%
Using this lemma it can be shown that the system of equations
induced by $G$ can be solved by constant semilinear terms for
each variable.
%%%
\vplatz
\exercise
Show that the UTAG $\auf \{\GC\}, \{\mbox{\tt S}\}, \{\mbox{\tt a}, 
\mbox{\tt b}, \mbox{\tt c}, \mbox{\tt d}\}, \{\GA\}\zu$ generates 
exactly the strings of the form $\vec{x}\mbox{\tt d}\mbox{\tt c}^n$, 
where $\vec{x}$ is a string of $n$ {\tt a}'s and $n$ {\tt b}'s such 
that every prefix of $\vec{x}$ has at least as many {\tt a}'s as 
{\tt b}'s. 
%%%
\begin{center}\begin{picture}(7,10)
\put(1,9){\makebox(0,0){$\GC$}}
\put(3,4.5){\makebox(0,0){\makebox{\tt d}}}
\put(3,5.5){\line(0,1){3}}
\put(3,9){\makebox(0,0){\makebox{\tt S}}}
\end{picture}
%%
\begin{picture}(10,10)
\put(1,9){\makebox(0,0){$\GA$}}
\put(1,5){\makebox(0,0){\makebox{\tt a}}}
\put(1.5,5.5){\line(1,1){3}}
\put(5,9){\makebox(0,0){\makebox{\tt S}}}
\put(5.5,8.5){\line(1,-1){3}}
\put(9,5){\makebox(0,0){\makebox{\tt S}}}
\put(8.5,4.5){\line(-1,-1){3}}
	\put(5,1){\makebox(0,0){\makebox{\tt b}}}
\put(9,4.5){\line(0,-2){3}}
	\put(9,1){\makebox(0,0){\makebox{\tt S}}}
\put(9.5,4.5){\line(1,-1){3}}
	\put(13,1){\makebox(0,0){\makebox{\tt c}}}
\end{picture}
\end{center}
%%%
Show also that this language is not context free. (This 
example is due to \cite{joshilevytakahashi:adjunct}.)

 \section{Are Natural Languages Context Free?}
\label{kap2-6}
%
%
%
We shall finish our discussion of CFLs by looking at some naturally 
arising languages. We shall give examples of languages and constructions 
which are definitely not context free. The complexity of natural 
languages has been high on the agenda ever since the introduction 
of this hierarchy. Chomsky's 
%%%
\index{Chomsky, Noam}%%%
%%%%
intention was in part to discredit 
structuralism, which he identified with the view that natural 
languages always are context free. By contrast, he claimed that 
natural languages are not context free and gave many examples. 
It is still widely believed that Chomsky had won his case. (For 
an illuminating discussion read \cite{manasterramerkac:concept}.)

It has emerged over the years
that the arguments given by Noam Chomsky and Paul Postal 
%%%
\index{Postal, Paul}%%%
%%%
against the context freeness of natural languages were faulty. 
Gerald Gazdar, Geoffrey Pullum 
%%%
\index{Gazdar, Gerald}%%%
\index{Pullum, Geoffrey}%%
%%%%
and others have repeatedly found holes in the
argumentation.  This has finally led to the bold claim that 
natural languages are all context free (see 
\cite{gazdarpullumsag:gpsg}). The first to deliver a correct
proof of the contrary was Riny Huybregts, 
%%%
\index{Huybregts, Riny}%%%
%%%
only shortly later followed by Stuart Shieber. 
%%%%
\index{Shieber, Stuart}%%%
%%%
(See \cite{huybregts:overlapping}
and \cite{shieber:evidence}.) Counterevidence from Bambara was 
given by Culy \shortcite{culy:bambara}. Of course, it was hardly
doubted that from structural point of view natural languages are
not context free (see the analyses of Dutch and German within
GB, for example, or \cite{bresnanetal:dutch}),
but it was not shown decisively that they are not even weakly
context free.

How is a proof the non context freeness of a language $L$ possible? 
A typical method is this. Take a suitable regular language $R$ and 
intersect it with $L$. If $L$ is context free, so is $L \cap R$. Now 
choose a homomorphism $h$ and map the language $L \cap R$ onto a known
non--CFL. We give an example from the paper by Stuart Shieber. Look 
at \eqref{ex:271} -- \eqref{ex:273}. If one looks at the nested 
infinitives in Swiss German (first rows) we find that they are 
structured differently from English (last rows) and High German 
%%%
\index{English}%%%
\index{German}%%%
\index{Swiss German}%%%
%%%%
(middle rows). (Instead of a gloss, we offer the following parallels: 
{\tt das} $\bumpeq$ {\tt dass} $\bumpeq$ {\tt that}, 
{\tt h\"alfe} $\bumpeq$ {\tt helfen} $\bumpeq$ {\tt help},
{\tt aastriche} $\bumpeq$ {\tt anstreichen} $\bumpeq$ 
{\tt paint}, {\tt huus} $\bumpeq$ {\tt Haus} $\bumpeq$ 
{\tt house}, {\tt mer} $\bumpeq$ {\tt wir} $\bumpeq$ {\tt we}, 
{\tt l\"ond} $\bumpeq$ {\tt lassen} $\bumpeq$ {\tt let}, 
{\tt chind} $\bumpeq$ {\tt Kinder} $\bumpeq$ {\tt children}.)  
%%
%\begin{table}
%\caption{Infinitives in Germanic Languages}
%\label{tab:swiss}
%\begin{tabular}{ll}
\begin{align}
\label{ex:271}
 & \mbox{\tt Jan s\"ait, das Hans es huus aastricht.} \\\notag
 & \mbox{\tt Jan sagt, dass Hans das Haus anstreicht.} \\\notag
 & \mbox{\tt Jan says that Hans is painting the house.} \\
%%%
\label{ex:272}
 & \mbox{\tt Jan s\"ait, das mer em Hans es huus h\"alfed} \\\notag
    & \quad \mbox{\tt aastriche.}  \\\notag
 & \mbox{\tt Jan sagt, dass wir Hans das Haus anstreichen} \\\notag
    & \quad \mbox{\tt helfen.} \\\notag
 & \mbox{\tt Jan says that we help Hans paint the house.} \\
\label{ex:273}
  & \mbox{\tt Jan s\"ait, das mer d'chind em Hans es huus} \\\notag
    & \quad \mbox{\tt l\"ond h\"alfe aastriche.} \\\notag
& \mbox{\tt Jan sagt, dass wir die Kinder Hans das Haus} \\\notag
    & \quad \mbox{\tt anstreichen helfen lassen.} \\\notag
& \mbox{\tt Jan says that we let the children help Hans} \\\notag
    & \quad \mbox{\tt paint the house.} \\
%%%
\label{ex:274}
 & ^{\ast}\mbox{\tt Jan s\"ait, das mer de Hans es huus h\"alfed} \\\notag
    & \quad \mbox{\tt aastriche.} \\
\label{ex:275}
 & ^{\ast}\mbox{\tt Jan s\"ait, das mer em chind em Hans es huus} \\\notag
    & \quad \mbox{\tt l\"ond h\"alfe aastriche.}
%\end{tabular}
%\end{table}
\end{align}
%%
By asking who does what to whom (we let, the children help, Hans
paints) we see that the constituents are quite different in the
three languages. Subject and corresponding verb are together in
English (see \eqref{eq:english-wo}), in High German they are on 
opposite sides of the embedded infinitive (see \eqref{eq:german-wo}, 
this is called the \textbf{nesting order}).
%%%
\index{order!nesting}%%
%%%
Swiss German, however, is still different. The verbs follow
each other in the reverse order as in German (so, they occur
in the order of the subjects, see \eqref{eq:swiss-wo}). This is 
called the \textbf{crossing order}.
%%%
\index{order!crossing}%%
%%%
\begin{subequations}
\begin{align}
\label{eq:english-wo}
& S_1\; V_1\; S_2\; V_2\; S_3\; V_3\; \dotso \\
\label{eq:german-wo}
& S_1\; S_2\; S_3\; \dotso\; V_3\; V_2\; V_1 \\
\label{eq:swiss-wo}
& S_1\; S_2\; S_3\; \dotso\; V_1\; V_2\; V_1\dotso  
\end{align}
\end{subequations}
%%%%
Now we assume --- this is an empirical assumption, to be
sure --- that this is the general pattern. It shall be
emphasized that the processing of such sentences becomes
difficult with four or five infinitives. Nevertheless, the
resulting sentences are considered grammatical.

Now we proceed as follows. The verbs require accusative
or dative on their complements. The following examples
show that there is a difference between dative and accusative.
In \eqref{ex:274} {\tt de Hans} is accusative and the complement
of {\tt aastriche}, which selects dative. The resulting sentence
is ungrammatical. In \eqref{ex:275}, {\tt em chind} is dative,
while {\tt l\"ond} selects accusative. Again the sentence is
ungrammatical. We now define the following regular language 
(recall the definition of $\diamond$ from Section~\ref{kap1}.\ref{kap1-2}).
%%
\begin{align}
R := & \mbox{\tt Jan}\oconc\mbox{\tt s\"ait}\conc\mbox{\tt ,}\oconc
	\mbox{\tt das} \oconc \mbox{\tt mer} \\\notag 
     & \oconc ((\mbox{\tt em}\conc\Box \cup \mbox{\tt d'} \cup 
	\mbox{\tt de}\conc\Box) \conc (\mbox{\tt chind}\conc\Box \cup 
	\mbox{\tt Hans}\conc\Box))^{\ast} \\\notag 
     & \conc \mbox{\tt es}\oconc \mbox{\tt huus} \\\notag
     & \oconc (\mbox{\tt laa}\conc\Box \cup \mbox{\tt l\"ond}\conc\Box 
	\cup \mbox{\tt h\"alfe}\conc\Box)^{\ast} \conc
	\mbox{\tt aastriche}\conc\mbox{\tt .}
\end{align}
%%
This is defined over the standard alphabet. It is not hard to see 
(invoking the Transducer Theorem, 
%%%
\index{Transducer Theorem}%%%
%%%
\ref{cor:transducer}) that the 
corresponding language over the alphabet of lexemes is also regular. 
We define the following mapping from the lexemes (denoted by their 
strings). $v$ sends {\tt d'}, {\tt de}, {\tt laa} and {\tt l\"ond} 
to {\tt a}, {\tt em} and {\tt h\"alfe} to {\tt d}, everything else 
inculding the blank is mapped to $\varepsilon$. The claim is that
%%
\begin{equation}
h[S \cap R] = \{\vec{x}\, \vec{x} : \vec{x} \in
\mbox{\tt a} \cdot (\mbox{\tt a} \cup \mbox{\tt d})^{\ast}\}
\end{equation}
%%
To this end we remark that a verb is sent to {\tt d} if it has a
dative object and to {\tt a} if it has an accusative object. An
accusative object is of the form $\mbox{\tt de}\; N$ or $\mbox{\tt
d'}\,  N$ ($N$ a noun) and is mapped to {\tt a} by $\oli{v}$. A
dative object has the form $\mbox{\tt em}\;  N$, $N$ a noun, and
is mapped onto {\tt d}. Since the nouns are in the same order as
the associated infinitives we get the desired result.

In mathematics we find a phenomenon similar to Swiss German.
%%%
\index{Swiss German}%%%
%%%
Consider the integral of a function. If $f(x)$ is a function,
the integral of $f(x)$ in the interval $[a,b]$ is denoted by
%%
\begin{equation}
\int_{a}^{b} f(x)dx
\end{equation}
%%
This is not in all cases well formed. For example,
$\int_0^1 x^{-1}dx$ is ill formed, since there Riemann approximation
leads to a sequence which is not bounded, hence has no limit.
Similarly, $\lim_{n \pf \infty} (-1)^n$ does not exist. Notice
that the value range of $x$ is written at the integral sign
without saying with what variable the range is associated.
For example, let us look at
%%
\begin{equation}
\int_{a}^b \int_{c}^d f(x,y)dxdy
\end{equation}
%%
The rectangle over which we integrate the function is
$a \leq x \leq b$ and $c \leq y \leq d$. Hence, the first
integral sign corresponds to the operator $dx$, which occurs
first in the list. Likewise for three integrals:
%%
\begin{equation}
\int_{a_0}^{b_0} \int_{a_1}^{b_1} \int_{a_2}^{b_2}
f(x_0,x_1,x_2) dx_0dx_1dx_2
\end{equation}
%%
where the value range is $a_i \leq x_i \leq b_i$ for all
$i < 3$. Consider the following functions:
%%
\begin{equation}
f(x_0, \dotsc, x_n) := \prod_{i < n} x_i^{\alpha_i}
\end{equation}
%%
with $\alpha_i \in \{-1,1\}$, $i < n$.
Further, we allow for the interval $[a_i,b_i]$ either
$[0,1]$ or $[1,2]$. Then an integral expression
%%
\begin{equation}
\int_{a_0}^{b_0} \int_{a_1}^{b_1} \dotsi \int_{a_{n-1}}^{b_{n-1}}
f(x_0,x_1,\dotsc, x_{n-1}) dx_0dx_1\dotsb dx_{n-1}
\end{equation}
%%
is well formed iff $a_i > 0$ for all
$i < n$ such that $\alpha_i = -1$. The dependencies are
crossing, and the order of elements is exactly as in Swiss
German (considering the boundaries and the variables).
The complication is the mediating function, which determines
which of the boundary elements must be strictly positive.

In \cite{kacmanasterramerrounds:simultaneous}, it is argued that 
%%%
\index{English}%%%
%%%
even English is not context free. The argument applies a theorem 
from \cite{ogdenrosswinkelmann:interchange}. If $L$ is a 
language, let $L_n$ denote the set of strings that are in $L$ and 
have length $n$. The following theorem makes use of the fact that 
a string of length $n$ possesses $n(n+1)/2$ proper substrings and 
that $n(n+1)/2 < n^2$ for all $n > 1$. Denote by $\ulcorner c \urcorner$ 
the smallest integer $\geq c$.  
%%%
\begin{thm}[Interchange Lemma]
\index{Interchange Lemma}%%%
%%%
\label{thm:interchange}
Let $L$ be a CFL. Then there exists a real number 
$c_L$ such that for every natural number $n > 0$ and every set 
$Q \subseteq L_n$ there is a $k \geq \ulcorner |Q|/(c_L n^2)\urcorner$, 
and strings $\vec{x}_i$, $\vec{y}_i$, $\vec{z}_i$, $i < k$, such that 
%%
\begin{dingautolist}{192}
\item for all $i < i < k$: $|\vec{x}_i| = |\vec{x}_j|$, 
$|\vec{y}_i| = |\vec{y}_j|$, 
and $|\vec{z}_i| = |\vec{z}_j|$.
\item for all $i < k$: $|\vec{y}_i|, |\vec{x}_i\vec{z}_i| > 0$, 
\item for all $i < k$: $\vec{x}_i\vec{y}_i\vec{z}_i \in Q$, and 
\item for all $i, j < k$: $\vec{x}_i\vec{y}_j\vec{z}_i \in L_n$.
\end{dingautolist}
\end{thm}
%%%
\proofbeg
Let $G$ be a CFG that generates $L$. Let $c_L := |N|$. We show 
that $c_L$ satisfies the above conditions. Take any set $Q \subseteq 
L_n$. Then there is $E \subseteq Q$ of cardinality $\geq 2 |Q|/(n+1)n$ 
and numbers $k \geq 0$ and $\ell > 0$ such that every member of $E$ 
possesses a decomposition $\vec{x}\,\vec{y}\,\vec{z}$ where $\vec{x}$ has 
length $k$, $\vec{y}$ has length $\ell$, and $\auf \vec{x}, \vec{z}\zu$ 
is a constituent occurrence of $\vec{y}$ in the string. It is 
then clear that there is a subset $F \subseteq E$ of cardinality 
$\geq 2 |Q|/((n+1)n |N|) > |Q|/(c_L n^2)$ such that 
all $\auf \vec{x}, \vec{z}\zu$ are constituent occurrences of 
identical nonterminal category. The above conditions are now 
satisfied for $F$. Moreover, 
$|F| \geq \ulcorner |Q|/(c_L n^2)\urcorner$, which had to be shown.   
\proofend

Note that if the sequence of numbers $L_n/n^2$ is bounded, then 
$L$ satisfies the conditions of the Interchange Lemma. For assume 
that there is a $c$ such that for all $n$ we have $L_{n}/n^2 \leq c$. 
Then $c_L := \sup \{|L_{n}|/n^2 : n \in \BN\} \leq c$. Then for every 
$n$ and every subset $Q$ of $L_n$, $\ulcorner |Q|/(c_L n^2)\urcorner 
\leq \ulcorner |L_n|/(c_L n^2)\urcorner \leq 1$. However, with $k = 1$ 
the conditions above become empty. 
%%%
\begin{thm}
\label{subquadratic}
Let $L \subseteq A^{\ast}$ be a language such that 
$(|L_n|/n^2)_{n \in \BN}$ is a bound\-ed sequence. Then $L$ 
satisfies the conditions of the Interchange Lemma. This is  
always the case if $|A| = 1$. 
\end{thm}
%%%%
Kac, Manaster--Ramer and Rounds use constructions with 
{\tt respectively} shown below, in which 
there is an equal number of nouns and verb phrases to be matched. 
In these constructions, the $n$th noun must agree in number with 
the $n$th verb phrase.  
%%%
%%%\begin{table}
%\caption{`Respectively'--Constructions}
%\label{tab:respectively}
%\begin{tabular}{l@{$\;$}l}
\begin{align}
 & \mbox{\tt This land can be expected to sell itself}/\\\notag 
  & \quad ^{\ast}\mbox{\tt themselves.}  \\\notag
 & \mbox{\tt These woods can be expected to sell }^{\ast}\mbox{\tt %
itself}/ \\\notag
	& \quad \mbox{\tt themselves.} \\
 & \mbox{\tt This land and these woods can be expected to rent} \\\notag
  & \quad \mbox{\tt itself and sell themselves respectively.} \\
 & ^{\ast}\mbox{\tt This land and these woods can be expected to rent} 
\\\notag 
 & \quad \mbox{\tt themselves and sell itself respectively.} \\
\label{ex:279}
 & \mbox{\tt This land and these woods and this land can be} \\\notag 
 & \quad \mbox{\tt expected to sell themselves and rent themselves} \\\notag
	& \quad \mbox{\tt respectively.}
\end{align}
%\end{tabular}
%\end{table}
%%%
The problematic aspect of these constructions is illustrated by 
\eqref{ex:279}. There need not be an exact match of NPs and 
VPs, and when there is no match, agreement becomes obscured (though 
it follows clear rules). Now let 
%%
\begin{align}
A := & (\mbox{\tt this}\oconc \mbox{\tt land} \cup \mbox{\tt these}
	\oconc \mbox{\tt woods})\oconc \mbox{\tt and} \\\notag
& \oconc (\mbox{\tt this}\oconc\mbox{\tt land}\conc\Box \cup 
	\mbox{\tt these}\oconc\mbox{\tt woods}\conc\Box)^+ \\\notag 
	& \conc \mbox{\tt can}\oconc\mbox{\tt be}\oconc\mbox{\tt expected}
	\oconc\mbox{\tt to} \\\notag
	& \oconc (\mbox{\tt rent} \cup \mbox{\tt sell})\oconc
	(\mbox{\tt itself}\conc\Box \cup 
	\mbox{\tt themselves}\conc\Box)^+ \\\notag
	& \conc\mbox{\tt and} \oconc 
	(\mbox{\tt rent} \cup \mbox{\tt sell}) \oconc
	(\mbox{\tt itself}\conc\Box \cup \mbox{\tt themselves}\conc\Box)^+ 
	\\\notag
	& \conc\mbox{\tt respectively}\conc\mbox{\tt .}
\end{align}
%%%
and let $D$ be the set of strings of $A$ that contain as many nouns 
as they contain pronouns. $B$ is that subset of $D$ where the $i$th 
noun is {\tt land} iff the $i$th pronoun is {\tt itself}. The empirical 
fact about English is that the intersection of English with $D$ is exactly 
$B$. Based on this we show that English is not context free. For suppose 
it were. Then 
we have a constant $c_L$ satisfying the Interchange Lemma. (We ignore 
the blanks and the period from now on.) Let $n$ be given. Choose 
$Q := B_n$, the set of strings of length $n$ in $B$. Notice that 
$|B_n| \geq 2^{(n - 8)/2}$ for all $n$. Therefore, for some $n$, 
$|B_n| > 2 n^2 c_L$ so that $\ulcorner |B_n|/c_L n^2\urcorner \geq 2$. 
This means that there are $\vec{x}_1$, $\vec{x}_2$, $\vec{z}_1$, 
$\vec{z}_2$ and $\vec{y}_1$ and $\vec{y}_2$ such that $B_n$ contains 
$\vec{x}_1\vec{y}_1\vec{z}_1$ as well as $\vec{x}_2\vec{y}_2\vec{z}_2$, 
but $\vec{x}_1\vec{y}_2\vec{z}_1$ and $\vec{x}_2\vec{y}_1\vec{z}_2$ 
are also grammatical (and therefore even in $B_n$). It is easy to see 
that this cannot be.

The next example in our series is modelled after the proof of
the non context freeness of ALGOL. 
%%%
\index{ALGOL}%%%
%%%
It deals with a quite
well known language, namely predicate logic. Predicate logic is
defined as a language over a set of relation and function
symbols of varying arity and a set of variables $\{\mbox{\mtt x}_i 
: i \in \omega\}$. In order to be able to conceive of predicate logic 
as a language in our sense, we code the variables as consisting
of sequences $\mbox{\mtt x}\vec{\alpha}$, where $\vec{\alpha} \in 
\{\mbox{\mtt 0}, \mbox{\mtt 1}\}^{\ast}$.
We have $\mbox{\mtt x}\vec{\alpha} = \mbox{\mtt x}\vec{\beta}$
iff $\vec{\alpha} = \vec{\beta}$. (Leading zeros
are not suppressed. The numbers are usually put as subscripts, but 
we shall not do that here.) We restrict ourselves to the language of
pure equality. The alphabet is $\{\mbox{\mtt\symbol{20}}, 
\mbox{\mtt\symbol{21}}, \mbox{\mtt (}, \mbox{\mtt )}, \mbox{\mtt =}, 
\mbox{\mtt x}, \mbox{\mtt 0}, \mbox{\mtt 1}, \mbox{\mtt\symbol{4}}, 
\mbox{\mtt\symbol{5}}, \mbox{\mtt\symbol{25}}\}$. 
The grammar rules are as follows.
%%
\begin{align}
\mbox{\mtt F} & \pf\mbox{\mtt Q(F)} \mid \mbox{\mtt \symbol{5}(F)} \mid
    \mbox{\mtt (F)\symbol{4}(F)} \mid
    \mbox{\mtt (F)\symbol{25}(F)} \mid 
    \mbox{\mtt P} \\\notag
\mbox{\mtt P} & \pf\mbox{\mtt V=V} \\\notag
\mbox{\mtt Q} & \pf\mbox{\mtt (\symbol{20}V)F} 
	\mid \mbox{\mtt (\symbol{21}V)F} \\\notag
\mbox{\mtt V} & \pf \mbox{\tt x} \mid \mbox{\mtt xZ} \\\notag
\mbox{\mtt Z} & \pf\mbox{\mtt 0Z} \mid \mbox{\mtt 1Z} \mid \mbox{\mtt 0}
    \mid \mbox{\mtt 1}
\end{align}
%%
Here {\mtt F} stands for the set of formulae {\mtt P} for the set of
prime formulae {\mtt Q} for the set of quantifier prefixes,
{\mtt V} the set of variables and {\mtt E} for the set of strings
over {\mtt 0} and {\mtt 1}. Let $\vec{x}$ be a formula and $C$ an
occurrence of a variable $\mbox{\mtt x}\vec{\alpha}$. We now
say that this occurrence
%%%
\index{variable!bound}%%
%%%
of a variable is \textbf{bound} in $\vec{x}$ if it is an occurrence
$D$ of a formula {\mtt ($Q$x$\vec{\alpha}$)$\vec{y}$} in $\vec{x}$
with $Q \in \{\mbox{\mtt\symbol{20}}, \mbox{\mtt\symbol{21}}\}$ 
which contains $C$. A formula is called a 
%%%
\index{sentence}%%
%%%
\textbf{sentence} if every occurrence of a variable is bound.
%%
\begin{thm}
The set of sentences of predicate logic of pure equality is not
context free.
\end{thm}
%%
\proofbeg
Let $L$ be the set of sentences of pure equality of predicate
logic. Assume this set is context free. Then by the Pumping Lemma
there is a $k$ such that every string of length $\geq k$ has a
decomposition $\vec{u}\,\vec{x}\,\vec{v}\,\vec{y}\,\vec{w}$ such that
$\vec{u}\,{\vec{x}\,}^i\vec{v}\,{\vec{y}\,}^i\vec{z} \in L$ for all $i$
and $|\vec{x}\,\vec{v}\,\vec{y}| \leq k$.  Define the following
formulae.
%%
\begin{equation}
\mbox{\mtt (\symbol{20}x$\vec{\alpha}$)(x$\vec{\alpha}$=x$\vec{\alpha}$)}
\end{equation}
%%
All these formulae are sentences. If $\vec{\alpha}$ is sufficiently
long (for example, longer than $k$) then there is a decomposition
as given. Since $\vec{x}\,\vec{v}\,\vec{y}$ must have length $\leq k$
$\vec{x}$ and $\vec{y}$ cannot both be disjoint to all occurrences of
$\vec{\alpha}$. On the other hand, it follows from this that
$\vec{x}$ and $\vec{y}$ consist only of {\mtt 0} and {\mtt 1}, and so
necessarily they are disjoint to some occurrence of $\vec{\alpha}$.
If one pumps up $\vec{x}$ and $\vec{y}$, necessarily one occurrence
of a variable will end up being unbound.
\proofend

We can strengthen this result considerably.
%%
\begin{thm}
The set of sentence of predicate logic of pure equality
is not semilinear.
\end{thm}
%%
\proofbeg
Let $P$ be the set of sentences of predicate logic of pure equality.
Assume that $P$ is semilinear. Then let $P_1$ be the set of
sentences which contain only one occurrence of a quantifier,
and let this quantifier be {\mtt\symbol{21}}. $\mu[P_1]$ is the intersection
of $\mu[P]$ with the set of all vectors whose {\mtt\symbol{21}}--component
is 1 and whose {\mtt\symbol{20}}--component is $0$. This is then also
semilinear. Now we consider the image of $\mu[P_1]$ under deletion
of all symbols which are different from {\mtt x}, {\mtt 0}
and {\mtt 1}. The result is denoted by $Q_1$.  $Q_1$ is semilinear.
By construction of $P_1$ there is an $\vec{\alpha} \in \{\mbox{\mtt 0},
\mbox{\mtt 1}\}^{\ast}$ such that every occurrence of a variable is
of the form $\mbox{\mtt x}\vec{\alpha}$. If this variable occurs
$k$ times and if $\vec{\alpha}$ contains $p$ occurrences of
{\mtt 0} and $q$ occurrences of {\mtt 1} we get as a result the
vector $k\mbox{\mtt x} + kp\mbox{\mtt 0} + kq\mbox{\mtt 1}$.
It is easy to see that $k$ must be odd. For a variable occurs
once in the quantifier and elsewhere once to the left and once
to the right of the equation sign. Now we have among others
the following sentences.
%%
\begin{equation}
\begin{array}{l}
\mbox{\mtt (\symbol{21}x$\vec{\alpha}$)(x$\vec{\alpha}$=x$\vec{\alpha}$)} \\
\mbox{\mtt (\symbol{21}x$\vec{\alpha}$)((x$\vec{\alpha}$=x$\vec{\alpha}$)%
\symbol{4}(x$\vec{\alpha}$=x$\vec{\alpha}$))} \\
\mbox{\mtt (\symbol{21}x$\vec{\alpha}$)((x$\vec{\alpha}$=x$\vec{\alpha})$%
\symbol{4}((x$\vec{\alpha}$=x$\vec{\alpha}$)\symbol{4}(x$\vec{\alpha}$%
=x$\vec{\alpha}$)))}
    \end{array}
\end{equation}
%%
Since we may choose any sequence $\vec{\alpha}$ we have
%%
\begin{equation}
Q_1 = \{(2k+3)(\mbox{\mtt x} + p\mbox{\mtt 0} +
q\mbox{\mtt 1}) : k, p, q\in \omega\}
\end{equation}
%%
$Q_1$ is an infinite union of planes of the form
$(2k+3)(\mbox{\mtt x} + \omega\mbox{\mtt 0} + \omega
\mbox{\mtt 1})$. We show: no finite union of linear planes equals
$Q_1$. From this we automatically get a contradiction.
So, assume that $Q_1$ is the union of $U_i$, $i < n$, $U_i$ linear.
Then there exists a $U_i$ which contains infinitely many vectors
of the form $(2k+3)\mbox{\mtt x}$. From this one easily deduces
that $U_i$ contains a cyclic vector of the form $m\mbox{\mtt x}$,
$m > 0$. (This is left as an exercise.) However, it is clear
that if $v \in Q_1$ then we have $m\mbox{\mtt x} + v \not\in Q_1$,
and then we have a contradiction.
\proofend

Now we shall present an easy example of a `natural' language
which is not semilinear. It has been proposed in somewhat different
form by Arnold Zwicky. Consider the number names of English.
%%%
\index{English}%%
\index{Zwicky, Arnold}%%%
%%%%
The stock of primitive names for numbers is finite. It contains
the names for digits ({\tt zero} up to {\tt nine}) the names
for the multiples of ten ({\tt ten} until {\tt ninety}), the numbers
from {\tt eleven} and {\tt twelve} until {\tt nineteen} as well as
some names for the powers of ten: {\tt hundred}, {\tt thousand},
{\tt million}, {\tt billion}, and a few more. (Actually, using
Latin numerals we can go to very high powers, but few people master
these numerals, so they will hardly know more than these.)
Assume without loss of generality that {\tt million} is the
largest of them. Then there is an additional recipe for naming
higher powers, namely by stacking the word {\tt million}. The
number $10^{6k}$ is represented by the $k$--fold iteration of
the word {\tt million}. For example, the sequence 
%%%
\begin{equation}
\mbox{\tt one million million million million} 
\end{equation}
%%%
names the number $10^{24}$. (It is also called {\tt octillion}, 
from Latin {\tt octo} `eight', because there are eight blocks 
of three zeros.) For arbitrary numbers the schema is as follows. A number 
in digital expansion is divided from right to left into blocks of six. 
So, it is divided as follows:
%%
\begin{equation}
\alpha_0 + \alpha_1 \times 10^6 + \alpha_2 \times 10^{12} \dotsb
\end{equation}
%%
where $\alpha_i < 10^6$ for all $i$. The associated number name is 
then as follows.
%%
\begin{equation}
\dotsb\oconc\vec{\eta}_2\oconc\mbox{\mtt million}\oconc%
\mbox{\mtt million}\oconc\vec{\eta}_1\oconc\mbox{\mtt million}
\oconc\vec{\eta}_0
\end{equation}
%%
where $\vec{\eta}_i$ is the number name of $\alpha_i$.
If $\alpha_i = 0$ the $i$th block is omitted. Let $Z$ be the set 
of number names. We define a function $\phi$ as follows. 
$\phi({\tt million}) = \mbox{\tt b}$; $\phi(\Box) := \varepsilon$, 
all other primitive names are mapped onto {\tt a}. The Parikh image 
of $\phi[Z]$ is denoted by $W$. Now we have
%%
\begin{equation}
W = \left\{ k_0\mbox{\tt a} + k_1 \mbox{\tt b} : k_1 \geq {{\llcorner
k_0/9\lrcorner}\choose 2}\right\} 
\end{equation}
%%
Here, $\llcorner k\lrcorner$ is the largest integer $\leq k$.
We have left the proof of this fact to the reader.
We shall show that $W$ is not semilinear. This shows that $Z$
is also not semilinear. Suppose that $W$ is semilinear, say
$W = \bigcup_{i < n} N_i$ where all the $N_i$ are linear. Let
%%
\begin{equation}
N_i = u_i + \sum_{j < p_i} \omega v^i_j
\end{equation}
%%
for certain $u_i$ and $v^i_j = \lambda^i_j \mbox{\tt a} +
\mu^i_j \mbox{\tt b}$. Suppose further that for some $i$ and $j$
we have $\lambda^i_j \neq 0$. Consider the set
%%
\begin{equation}
P := u_i + \omega v^i_j = \{ u_i + k\lambda^i_j \mbox{\tt a} +
    k\mu^i_j \mbox{\tt b} : k \in \omega\} 
\end{equation}
%%
Certainly we have $P \subseteq N_i \subseteq W$. Furthermore, we
surely have $\mu^i_j \neq 0$. Now put $\zeta :=
\lambda^i_j/\mu^i_j$. Then
%%
\begin{equation}
P = \{u_i + k\mu^i_j (\mbox{\tt a} + \zeta \mbox{\tt b}) : k \in \omega\}
\end{equation}
%%
\begin{lem}
For every $\varepsilon > 0$ almost all elements of $P$
have the form $p\mbox{\tt a} + q\mbox{\tt b}$ where
$q/p \leq \zeta + \varepsilon$.
\end{lem}
%%
\proofbeg
Let $u_i = x\mbox{\tt a} + y\mbox{\tt b}$. Then a general element
of the set $P$ is of the form $(x + k\lambda^i_j)\mbox{\tt a} +
(y + k\mu^i_j)\mbox{\tt b}$.  We have to show that for almost all
$k$ the inequality
%%
\begin{equation}
\frac{x+ k\lambda^i_j}{y + k\mu^i_j} \leq \varepsilon + \zeta
\end{equation}
%%
is satisfied. Indeed, if $k > \frac{x}{\mu^i_j \varepsilon}$, then
%%
\begin{equation}
\frac{x + k\lambda^i_j}{y + k\mu^j_i} \leq
\frac{x + k\lambda^i_j}{k\mu^i_j} =
\zeta + \frac{x}{k \mu^i_j}
< \zeta + \frac{x}{\mu^i_j x/\mu^i_j \varepsilon}
= \zeta + \varepsilon
\end{equation}
%%
This holds for almost all $k$.
%%
\proofend
%%
\begin{lem}
Almost all points of $P$ are outside of $W$.
\end{lem}
%%
\proofbeg
Let $n_0$ be chosen in such a way that
${\llcorner n_0/9\lrcorner \choose 2} > n_0(\zeta + 1)$.
Then for all $n \geq n_0$ we also have
${\llcorner n/9\lrcorner \choose 2} > n(\zeta + 1)$.
Let $p\mbox{\tt a} + q \mbox{\tt b} \in W$ with $p \geq n_0$.
Then we have $\frac{q}{p} > \zeta + \varepsilon$, and therefore
$p\mbox{\tt a} + q \mbox{\tt b} \not\in P$. Put $H := \{p\mbox{\tt a} 
+ q\mbox{\tt b} : p \geq n_0\}$. Then $P \cap H = \varnothing$. 
However $W \cap - H$ is certainly finite. Hence $W \cap P$ is 
finite, as required.
\proofend
%%

Now have the desired contradiction. For on the one hand no vector
is a multiple of {\tt a}; on the other hand there can be no vector
$m\mbox{\tt a} + n \mbox{\tt b}$ with $n \neq 0$. Hence $W$ is not
semilinear.

{\it Notes on this section.} 
The question concerning the complexity of variable binding is discussed in 
\cite{marshpartee:binding}. It is shown there that the language of 
sentences of predicate logic is not context free (a result that was 
`folklore') but that it is at least an indexed language. (Indexed 
languages neeed not be semilinear.) On the other hand, it has been
conjectured that if we take $V$ to the set of formulae in which every 
quantifier binds at least one free occurrence of a variable, the 
language $V$ is not even an indexed language. See also 
Section~\ref{kap4}.\ref{kap4-6}. Philip
Miller~\shortcite{miller:scandinavian} 
%%%
\index{Miller, Philip H.}%%
%%%%
argues that 
Swedish and Norwegian are not context free, and if right branching 
analyses are assumed, they are not even indexed languages. 
%%
\vplatz
\exercise
Formalize the language of functions and integral expressions.
Prove that the language of proper integral expressions is not
context free.
%%
\vplatz
\exercise
Show the following: {\it Let $U$ be a linear set which contains infinitely
many vectors of the form $k\mbox{\tt a}$. Then there exists a cyclic
vector of the form $m\mbox{\tt a}$, $m > 0$.} {\it Hint.} Notice
that the alphabet may consist of more than one letter.
%%
\vplatz
\exercise
Show that $W$ has the claimed form.
%%
\vplatz
\exercise
Show that the set $V$ is not semilinear.
%%
\begin{equation}
V := \left\{ k_0 \mbox{\tt a} + k_1 \mbox{\tt b} :
k_1 \leq {{k_0}\choose 2}\right\}
\end{equation}
%%
{\it Hint.} Evidently, no linear set $\subseteq V$ may contain
a vector $k \mbox{\tt b}$. Therefore the following is 
well--defined.
%%
\begin{equation}
\gamma := \max \left\{\frac{\mu^i_j}{\lambda^i_j} :
i< n, j < p_i\right\}
\end{equation}
%%
Show now that for every $\varepsilon > 0$ almost all elements of
$W$ are of the form $x\mbox{\tt a} + y \mbox{\tt b}$ where $y \leq
(\gamma + \varepsilon)x$. If we put for example $\varepsilon = 1$
we now get a contradiction.
%%
\vplatz 
\exercise 
Prove the unique readability of predicate logic.
{\it Hint.} Since we have strictly speaking not defined terms,
restrict yourself to proving that the grammar given above is 
unambiguous. You might try to show that it is also transparent.
%%%
\vplatz
\exercise
Let $\Omega \subseteq \omega$. Put $L_{\Omega} := \{\mbox{\tt a}^m%
\mbox{\tt b}^n : m \neq n \text{ or } m \in \Omega\}$. Then 
$\pi[L_{\Omega}] = V_{\Omega}$, as defined in Exercise~\ref{ex:omega}. 
Show that $L_{\Omega}$ satisfies the properties of 
Theorem~\ref{thm:multipump} and of Theorem~\ref{thm:interchange}. 
It follows that there are $2^{\aleph_0}$ many languages over 
{\tt a} and {\tt b} that satisfy these criteria for context freeness 
and are not even semilinear. 

% \newpage 
%	\thispagestyle{empty}
%	\mbox{}
 \chapter{Categorial Grammar and Formal Semantics}
\thispagestyle{empty}
\label{kap3}
%
%
%
\section{Languages as Systems of Signs}
\label{kap3-1}
%
%
%
Languages are certainly not sets of strings. They are systems for
communication. This means in particular that the strings have
meaning, a meaning which all speakers of the language more or 
less understand. And since natural languages have potentially
infinitely many strings, there must be a way to find out what meaning
a given string has on the basis of finite information. An important
principle in connection with this is the so--called
%%%%
\index{compositionality}%%
%%%%
{\it Principle of Compositionality\/}. It says in simple words that
the meaning of a string only depends on its derivation. For a CFG this 
means: if $\rho = \beta \pf \alpha_0 \alpha_1 \dotsb \alpha_{n-1}$ 
is a rule and $\vec{u}_i$ a string of category $\alpha_i$ then 
$\vec{v} := \vec{u}_0 \vec{u}_1 \dotsb \vec{u}_{n-1}$ is a
string of category $\beta$ and the meaning of $\vec{v}$ depends only 
on the meaning of the $\vec{u}_i$ and $\rho$. In this form the principle 
of compositionality is still rather vague, and we shall refine
and precisify it in the course of this section. However, for now we
shall remain with this definition. It appears that we have admitted
only context free rules. This is a restriction, as we know. We shall
see later how we can get rid of it.

To begin, we shall assume that meanings come from some set
$M$, which shall not be specified further. As before, exponents
are members of $A^{\ast}$, where $A$ is a finite alphabet.
(Alternatives to this assumption will be discussed later.)
%%
\begin{defn}
\index{interpreted string language}%%
\index{language!interpreted}%%
%%%
An \textbf{interpreted} (\textbf{string}) \textbf{language over the
alphabet} $A$ and \textbf{with meanings in} $M$ is a relation
$\CI \subseteq A^{\ast} \times M$. The \textbf{string language
associated with} $\CI$ is
%%
\index{$L(\CI)$, $M(\CI)$}%%
%%%
\begin{equation}
L(\CI) := \{\vec{x} : \text{ there is }
m \in M \text{ such that } \auf \vec{x}, m\zu \in \CI\}
\end{equation}
%%
The meanings expressed by $\CI$ are
%%
\begin{equation}
M(\CI) := \{m  : \text{ there is } \vec{x} \in A^{\ast}
\text{ such that }\auf \vec{x}, m\zu \in \CI\}
\end{equation}
\end{defn}
%%
Alternatively, we may regard a language as a function from 
$A^{\ast}$ to $\wp(M)$. Then $L(f) := \{\vec{x} : f(\vec{x})
\neq \varnothing\}$ is the string language associated with
$f$ and $M(f) := \bigcup_{\vec{x} \in A^{\ast}} f(\vec{x})$
the set of expressed meanings of $f$. These definitions are 
not equivalent when it comes to compositionality. In the 
original definition, any particular meaning of a composite 
expression is derived from some particular meanings of its 
parts, in the second the totality of meanings is derived from 
the totality of the meanings of the parts. 

We give an example. We consider the number terms as known from
everyday life as for example {\mtt ((3+5)\symbol{42}2)}. We 
shall write a grammar with which we can compute the value of a 
term as soon as its analysis is known. This means that we regard 
an interpreted language as a set of pairs $\auf t, x\zu$ where 
$t$ is an arithmetical term and $x$ its value. Of course, the 
analysis does not directly reveal the value but we must in addition 
to the rules of the grammar specify in which way the value of the 
term is computed inductively over the analysis. Since the nodes 
correspond to the subterms this is straightforward. Let $T$ be the 
following grammar.
%%
\begin{equation}
\label{eq:grammT}
\begin{array}{l@{\quad \pf\quad}l}
\mbox{\mtt T} & \mbox{\mtt (T+T)} \mid
    \mbox{\mtt (T-T)} \mid \mbox{\mtt (T\symbol{42}T)} 
	\mid \mbox{\mtt (T\symbol{47}T)} \\
\mbox{\mtt T} & \mbox{\mtt Z} \mid \mbox{\mtt (-Z)} \\
\mbox{\mtt Z} & \mbox{\mtt 0} \mid \mbox{\mtt 1} \mid
    \mbox{\mtt 2} \mid \dotsb \mid \mbox{\mtt 9}
\end{array}
\end{equation}
%%
(This grammar only generates terms which have ciphers
in place of decimal strings. But see Section~\ref{kap3}.\ref{kap3-2}.)
Let now an arbitrary term be given. To this term corresponds
a unique number (if for a moment we disregard division by 0).
This number can indeed be determined by induction over the
term. To this end we define a partial interpretation map $I$,
which if defined assigns a number to a given term.
%%
\begin{equation}
\begin{array}{l@{\quad := \quad}l}
I(\mbox{\mtt ($\vec{x}$+$\vec{y}$)})
    & I(\vec{x}) + I(\vec{y}) \\
I(\mbox{\mtt ($\vec{x}$-$\vec{y}$)})
    & I(\vec{x}) - I(\vec{y}) \\
I(\mbox{\mtt ($\vec{x}$\symbol{42}$\vec{y}$)})
    & I(\vec{x}) \times I(\vec{y}) \\
I(\mbox{\mtt ($\vec{x}$\symbol{47}$\vec{y}$)})
    & I(\vec{x}) \div I(\vec{y}) \\
I(\mbox{\mtt (-$\vec{x}$)}) & - I(\vec{x}) \\
I(\mbox{\mtt 0}) & 0 \\
I(\mbox{\mtt 1}) & 1 \\
\multicolumn{2}{c}{\dotsb} \\
I(\mbox{\mtt 9}) & 9
\end{array}
\end{equation}
%%
If a function $f$ is undefined on $x$ we write $f(x) = \star$.
We may also regard $\star$ as a value. The rules for $\star$
are then as follows. If at least one argument is $\star$, so is the value.
Additionally, $a / 0 = \star$ for all $a$. If $\vec{x}$ is a
term, then $I(\vec{x})$ is uniquely defined. For either $\vec{x}$
is a cipher from {\mtt 0} to {\mtt 9} or it is a negative cipher,
or $\vec{x} = \mbox{\mtt ($\vec{y}_1\odot\vec{y}_2$)}$ for some 
uniquely determined $\vec{y}_1$, $\vec{y}_2$ and 
$\odot \in \{\mbox{\mtt +}, \mbox{\mtt -}, \mbox{\mtt\symbol{42}}, 
\mbox{\mtt\symbol{47}}\}$. In this way one can calculate $I(\vec{x})$ 
if one knows $I(\vec{y}_1)$ and $I(\vec{y}_2)$. The value of a term 
can be found by naming a derivation and then computing the value of 
each of its subterms. Notice that the grammar is transparent so that 
only one syntactical analysis can exist for each string.

The method just described has a disadvantage: the interpretation
of a term is in general not unique, for example if a string is
ambiguous. (For example, if we erase all brackets then the
term {\mtt 3+5\symbol{42}2} has two values, 13 or 16.) As explained
above, we could take the meaning of a string to be a set of numbers.
If the language is unambiguous this set has at most one member.
Further, we have $I(\vec{x}) \neq \varnothing$ only if $\vec{x}$
is a constituent. However, in general we wish to avoid taking this
step. Different meanings should arise only from different
analyses. There is a way to implement this idea no matter what the
grammar is. Let $U$ be the grammar which results from $T$ by
deleting the brackets of $T$.
%%
\begin{equation}
\begin{array}{l@{\quad \pf\quad}l}
\mbox{\mtt T} & \mbox{\mtt T+T} \mid \mbox{\mtt T-T} \mid
    \mbox{\mtt T\symbol{42}T} \mid
    \mbox{\mtt T\symbol{47}T} \\
\mbox{\mtt T} & \mbox{\mtt Z} \mid \mbox{\mtt -Z} \\
\mbox{\mtt Z} & \mbox{\mtt 0} \mid \mbox{\mtt 1} \mid
    \mbox{\mtt 2} \mid \dotsb \mid \mbox{\mtt 9}
\end{array}
\end{equation}
%%
The strings of $U$ can be viewed as images of a canonical transparent 
grammar. This could be \eqref{eq:grammT}. However, for some 
reason that will become clear we shall choose a different 
grammar. Intuitively, we think of the string as the image 
of a term which codes the derivation tree. This tree differs 
from the structure tree in that the intermediate symbols are 
not nonterminals but symbols for rules. The derivation tree 
is coded by term in Polish Notation. 
%%%%
\index{Polish Notation}%%%
%%%
For each rule $\rho$ 
we add a new symbol $\mbox{\tt R}_{\rho}$. In place of the 
rule $\rho = A \pf \vec{\alpha}$ we now take the rule
$A \pf \mbox{\mtt R}_{\rho} \vec{\alpha}$. This grammar,
call it $V$, is transparent (see Exercise~\ref{ex:transparent}). 
$\vec{x} \in L(V)$
is called a \textbf{derivation term}. 
%%%
\index{derivation term}%%%
%%%
We define two maps
$\zeta$ and $\iota$. $\zeta$ yields a string for each
derivation term, and $\iota$ yields an interpretation.
Both maps shall be homomorphisms from the
term algebra, though the concrete definition is defined
over strings. $\zeta$ can be uniformly defined by deleting
the symbols $\mbox{\mtt R}_{\rho}$. However, notice that the
rules below yield values only if the strings are derivation
terms.
%%
\begin{equation}
\begin{split}
\zeta(\mbox{\mtt R}_{\rho}\vec{\alpha}_0 \dotsb
    \vec{\alpha}_{n-1}) & := \zeta(\alpha_0) \conc
        \zeta(\alpha_1) \conc \dotsb \conc
        \zeta(\alpha_{n-1}) \\
\zeta(\alpha) & := \alpha
\end{split}
\end{equation}
%%
In the last line, $\alpha$ is different from all $\mbox{\mtt R}_{\rho}$.
We have assumed here that the grammar has no rules of the form
$A \pf \varepsilon$ even though a simple adaptation can
help here as well. Now on to the definition of $\iota$.
In the case at hand this is without problems.
%%
\begin{equation}
\begin{array}{l@{\quad := \quad}l}
\iota(\mbox{\mtt R}_{\mbox{\smtt +}}\vec{\alpha}_0\mbox{\mtt +}\vec{\alpha}_1) 
	& \iota(\vec{\alpha}_0) + \iota(\vec{\alpha}_1) \\
\iota(\mbox{\mtt R}_{\mbox{\smtt -}^2}\vec{\alpha}_0\mbox{\mtt -}%
\vec{\alpha}_1) &
    \iota(\vec{\alpha}_0) - \iota(\vec{\alpha}_1) \\
\iota(\mbox{\mtt R}_{\mbox{\smtt\symbol{42}}}\vec{\alpha}_0%
\mbox{\mtt\symbol{42}}\vec{\alpha}_1) &
    \iota(\vec{\alpha}_0) \times \iota(\vec{\alpha}_1) \\
\iota(\mbox{\mtt R}_{\mbox{\smtt\symbol{47}}}\vec{\alpha}_0%
\mbox{\mtt\symbol{47}}\vec{\alpha}_1) &
    \iota(\vec{\alpha}_0) \div \iota(\vec{\alpha}_1) \\
\iota(\mbox{\mtt R}_{\mbox{\smtt -}^1}\mbox{\mtt -}\vec{\alpha}) 
	& - \iota(\vec{\alpha})
\end{array}
\end{equation}
%%
Here we have put the derivation term into Polish Notation, 
%%%
\index{Polish Notation}%%%
%%%%
since it is uniquely readable. However, this only holds under
the condition that every symbol is unique. Notice, namely, that
some symbols can have different meanings --- as in our example the
minus symbol. To this end we have added an additional annotation
of the symbols. Using a superscript we have distinguished between
the unary minus and the binary one. Since the actual language does
not do so (we write `{\mtt -}' without distinction), we have written
$\mbox{\mtt R}_{\mbox{\smtt -}^1}$ if the rule for the unary symbol 
has been used, and $\mbox{\mtt R}_{\mbox{\smtt -}^2}$ if the one for 
the binary symbol has been used.

The mapping $\iota$ is a homomorphism of the algebra of derivation
terms into the algebra of real numbers with $\star$, which is
equivalent to a partial homomorphism from the algebra of terms to
the algebra of real numbers. For example the symbol $\mbox{\mtt
R}_{\mbox{\smtt +}}$ is interpreted by the function $+ \colon
\BR_{\star}\times\BR_{\star} \pf \BR_{\star}$, where $\BR_{\star}
:= \BR \cup \{\star\}$ and $\star$ satisfies the laws specified
above. In principle this algebra can be replaced by any other
which allows to interpret unary and binary function symbols. We
emphasize that it is not necessary that the interpreting
functions are basic functions of the algebras.  It is enough 
if they are polynomial functions (see \cite{hendriks:compositionality} 
on this point). For example, we can introduce a 
unary function symbol {\mtt d} whose interpretation is
duplication. Now $2x = x + x$, and hence the duplication is a
polynomial function of the algebra $\auf \BR, +, \cdot, 0, 1\zu$,
but not basic. However, the formal setup is easier if we interpret 
each function symbol by a basic function. (It can always be added, 
if need be.)

This exposition motivates a terminology which sees meanings and
strings as images of abstract signs under a homomorphism. We shall
now develop this idea in full generality. The basis is formed by
an algebra of signs. Recall from Section~\ref{kap1}.\ref{kap1-1} the notion 
of a strong (partial) subalgebra. A strong subalgebra is determined 
by the set $B$. The functions on $B$ are the restrictions of the 
respective functions on $A$. Notice that it is not allowed to 
partialize functions additionally. For example, $\auf A, \Xi\zu$ 
with $\Xi(f) = \varnothing$ is not a strong subalgebra of $\GA$ 
unless $\Pi(f) = \varnothing$.

A \textbf{sign} 
%%%%%
\index{sign}%%
%%%%%
is a triple $\sigma = \auf e, c, m\zu$
where $e$ is the exponent of $\sigma$, usually some kind of string
over an alphabet $A$, $c$ the category of $\sigma$ and $m$ its
meaning. Abstractly, however, we shall set this up differently.
We shall first define an algebra of signs as such, and introduce
exponent, category and meaning as values of the signs under some
homomorphisms. This will practically amount to the same, however.
So, we start by fixing a signature $\auf F, \Omega\zu$. In this
connection the function symbols from $F$ are called \textbf{modes}.
%%%
\index{mode}%%
\index{$\auf F,\Omega\zu$}%%%
%%%
Over this signature we shall define an algebra of signs, of
exponents, of categories and meanings. An algebra of signs over
$\auf F, \Omega\zu$ is simply a 0--generated partial algebra
$\GA$ over this signature together with certain homomorphisms,
which will be defined later.
%%%
\begin{defn}
%%%
\index{algebra!$n$--generated}%%
%%%
A (partial) $\Omega$--algebra $\GA = \auf A, \Pi\zu$ is called
$n$--\textbf{generated} if there is an $n$--element subset $X
\subseteq A$ such that the smallest strong subalgebra containing
$X$ is $\GA$.
\end{defn}
%%
\begin{defn}
The quadruple $\auf \GA, \varepsilon, \gamma, \mu\zu$ 
%%%
\index{$\varepsilon$, $\gamma$, $\mu$}%%%
%%%%
is called a \textbf{sign grammar over the  signature} $\Omega$
if $\GA$ is a 0--generated partial $\Omega$--algebra
and $\varepsilon \colon \GA \pf \GE$, $\gamma \colon \GA \pf \GC$
and $\mu \colon \GA \pf \GM$ homomorphisms to certain partial
$\Omega$--algebras such that the homomorphism
$\auf \varepsilon, \gamma, \mu\zu$ is injective and strong.
$\GA$ is called the \textbf{algebra of signs}, $\GE$ the
\textbf{algebra of exponents}, $\GC$ the \textbf{algebra of categories}
and $\GM$ the \textbf{algebra of meanings}.
\end{defn}
%%
This means in particular:
%%
\begin{dinglist}{43}
\item
Every sign $\sigma$ is uniquely characterized by three things:
\begin{itemize}
\item
its so--called \textbf{exponent} $\varepsilon(\sigma)$,
%%%
\index{exponent}%%
%%%
\item
its (\textbf{syntactical}) \textbf{category} $\gamma(\sigma)$ (which 
is also often called its \textbf{type}),
%%%
\index{category}%%
\index{type}%%
%%%
\item
its meaning $\mu(\sigma)$.
\end{itemize}
\item
To every function symbol $f \in F$ corresponds an
$\Omega(f)$--ary function $f^{\GE}$ in $\GE$, an 
$\Omega(f)$--ary function $f^{\GC}$ in $\GC$ and an 
$\Omega(f)$--ary function $f^{\GM}$ in $\GM$.
\item
Signs can be combined with the help of the function $f^{\GA}$
any time their respective exponents can be combined with the help
of $f^{\GE}$, their respective categories can be combined
with $f^{\GC}$ and their respective meanings with $f^{\GM}$.
(This corresponds to the condition of strongness.)
\end{dinglist}
%%
\index{$f^{\varepsilon}$, $f^{\gamma}$, $f^{\mu}$}%%
%%%
In the sequel we shall write $f^{\varepsilon}$ in place of 
$f^{\GE}$, $f^{\gamma}$ in place of $f^{\GC}$ and $f^{\mu}$ in 
place of $f^{\GM}$. This will allow us to suppress mentioning 
which actual algebras are chosen. If $\sigma$ is a sign, then 
$\auf \varepsilon(\sigma), \gamma(\sigma), \mu(\sigma)\zu$ is 
uniquely defined by $\sigma$, and on the other hand it uniquely 
defines $\sigma$ as well. We shall call this triple the 
%%%
\index{realization}%%%
%%%
\textbf{realization} of $\sigma$. Additionally,
we can represent $\sigma$ by a term in the free $\Omega$--algebra.
We shall now deal with the correspondences between these
viewpoints.

Let $\goth{Tm}_{\Omega} := \auf \PN_{\Omega}, \{g^{\goth{Tm}_{\Omega}} 
: g \in F\}\zu$, where $\PN_{\Omega}$ is the set of constant $\Omega$--terms
written in Polish Notation and 
%%
\begin{equation}
g^{\goth{Tm}_{\Omega}}(\vec{x}_0, \dotsc, \vec{x}_{\Omega(g)-1}) :=
    g \conc \prod_{i < \Omega(g)} \vec{x}_i
\end{equation}
%%
$\goth{Tm}_{\Omega}$ is a freely 0--generated $\Omega$--algebra.
The elements of $\PN_{\Omega}$ are called 
%%%
\index{structure term}%%
%%%%
\textbf{structure terms}.  We use $\Gs$, $\Gt$, $\Gu$ and so on 
as metavariables for structure terms. We give an example. Suppose 
that $\mbox{\tt N}$ is a
0--ary mode and $\mbox{\tt S}$ a unary mode. Then we have 
$\mbox{\tt N}^{\goth{Tm}_{\Omega}} = \mbox{\tt N}$ and 
$\mbox{\tt S}^{\goth{Tm}_{\Omega}} %
\colon \vec{x} \mapsto \mbox{\tt S}\conc \vec{x}$. This yields
the following strings as representatives of structure terms.
%%
\begin{equation}
\mbox{\tt N}, \mbox{\tt SN}, \mbox{\tt SSN},
\mbox{\tt SSSN}, \dotsc
\end{equation}
%%

We denote by $h \colon M \stackrel{p}{\pf} N$ the fact that $h$
is a partial function from $M$ to $N$. We now define
partial maps $\dot{\varepsilon} \colon \PN_{\Omega} 
\stackrel{p}{\pf} E$, $\dot{\gamma} \colon \PN_{\Omega} 
\stackrel{p}{\pf} C$ and $\dot{\mu} \colon \PN_{\Omega}
\stackrel{p}{\pf} M$ in the following way.
%%%
\begin{equation}
\dot{\varepsilon}(g^{\goth{Tm}_{\Omega}}(\Gs_0, \dotsc, \Gs_{\Omega(g)-1}))
    := g^{\varepsilon}(\dot{\varepsilon}(\Gs_0), \dotsc,
    \dot{\varepsilon}(\Gs_{\Omega(g)-1}))
\end{equation}
%%%
Here, the left hand side is defined iff the right hand
side is and then the two are equal. If we have a 0--ary mode $g$,
then it is a structure term $\dot{\varepsilon}(g) = g^{\varepsilon}
\in E$. Likewise we define the other maps.
%%
\begin{align}
\dot{\gamma}(g^{\goth{Tm}_{\Omega}}(\Gs_0, \dotsc, \Gs_{\Omega(g)-1}))
    & := g^{\gamma}(\dot{\gamma}(\Gs_0), \dotsc,
    \dot{\gamma}(\Gs_{\Omega(g)-1})) \\
\dot{\mu}(g^{\goth{Tm}_{\Omega}}(\Gs_0, \dotsc, \Gs_{\Omega(g)-1}))
    & := g^{\mu}(\dot{\mu}(\Gs_0), \dotsc,
    \dot{\mu}(\Gs_{\Omega(g)-1}))
\end{align}
%%
As remarked above, for every sign there is a structure term.
The converse need not hold.
%%
\begin{defn}
We say, a structure term $\Gs$ is
%%%
\index{structure term!orthographically definite}%%
%%%
\textbf{orthographically definite} if $\dot{\varepsilon}(\Gs)$
is defined. $\Gs$ is \textbf{syntactically definite} %%
%%%
\index{structure term!syntactically definite}%%
%%%
if $\dot{\gamma}(\Gs)$ is defined and \textbf{semantically definite}
%%%
\index{structure term!semantically definite}%%
%%%
if $\dot{\mu}(\Gs)$ is defined. Finally, $\Gs$ is
\textbf{definite}
%%%
\index{structure term!definite}%%
%%%
if $\Gs$ is orthographically, syntactically as well as semantically
definite.
\end{defn}
%%
\begin{defn}
%%%
\index{$\upsilon$}%%
\index{unfolding map}%%
%%%
The partial map $\upsilon := \auf \dot{\varepsilon},
\dot{\gamma}, \dot{\mu}\zu$ is called the \textbf{unfolding map}.
\end{defn}
%%%
The reader is referred to Figure~\ref{fig:synopsis} for a synopsis 
of the various algebras and maps between them.
%%
\begin{figure}
\begin{center}
\begin{picture}(30,22)
\put(15,3){\makebox(0,0){$\GA$}}
	\put(14.5,3.5){\vector(-1,1){3}}
		\put(12.5,5){\makebox(0,0)[r]{$\varepsilon$}}
	\put(15,3.5){\vector(0,1){3}}
		\put(14.8,5.5){\makebox(0,0)[r]{$\gamma$}}
	\put(15.5,3.5){\vector(1,1){3}}
		\put(17.5,5){\makebox(0,0)[l]{$\mu$}}
\put(14,3){\line(-1,0){7}}
\put(7,3){\line(0,1){8}}
	\put(6,7){\makebox(0,0)[r]{$\auf \varepsilon, \gamma, \mu\zu$}}
\put(7,11){\vector(1,0){5}}
\put(11,7){\makebox(0,0){$\GE$}}
\put(15,7){\makebox(0,0){$\GC$}}
\put(19,7){\makebox(0,0){$\GM$}}
\put(15,11){\makebox(0,0){$\GE \times \GC \times \GM$}}
	\put(14.5,10.5){\vector(-1,-1){3}}
		\put(12.5,9){\makebox(0,0)[r]{$\pi_0$}}
	\put(15,10.5){\vector(0,-1){3}}
		\put(14.8,8.5){\makebox(0,0)[r]{$\pi_1$}}
	\put(15.5,10.5){\vector(1,-1){3}}
		\put(17.5,9.5){\makebox(0,0){$\pi_2$}}
\put(15,15){\makebox(0,0){$\goth{Def}$}}
	\put(15,15.5){\vector(0,1){3}}
		\put(15.5,17){\makebox(0,0)[l]{$\mbox{\it id\/}$}}
	\put(15,14.5){\vector(0,-1){3}}
		\put(15.5,13){\makebox(0,0)[l]{$\upsilon = 
		\auf\dot{\varepsilon}, \dot{\gamma}, \dot{\mu}\zu$}}
\put(16,15){\line(1,0){7}}
\put(23,15){\line(0,-1){12}}
\put(23,3){\vector(-1,0){7}}
\put(15,19){\makebox(0,0){$\goth{Tm}_{\Omega}$}}
\end{picture}
\end{center}
\caption{Synopsis}
\label{fig:synopsis}
\end{figure}
%%%
In the sequel we shall often identify the structure term $\Gs$ 
with its image under the unfolding map. This will result in rather 
strange types of definitions, where on the left we find a string 
(which {\it is\/} the structure term, by convention) and on the 
right a triple. This abuse of the language shall hopefully present 
no difficulty. $\GA$ is isomorphic to the partial algebra of all 
$\auf \dot{\varepsilon}(\Gs), \dot{\gamma}(\Gs), \dot{\mu}(\Gs)\zu$,
where $\Gs$ is a definite structure term. This we can also look 
at differently. Let $D$ be the set of definite structure terms. This
set becomes a partial $\Omega$--algebra together with the partial
functions $g^{\goth{Tm}_{\Omega}} \restriction D$. We denote this algebra by
$\goth{Def}$. $\goth{Def}$ is usually not a strong subalgebra of 
$\goth{Tm}_{\Omega}$.  For let $j \colon \Gs \mapsto \Gs$ be the 
identity map. Then we have 
$j(g^{\goth{Def}}(\Gs_0, \dotsc, \Gs_{\Omega(g)-1})) = 
g^{\goth{Tm}_{\Omega}}(j(\Gs_0), \dotsc, j(\Gs_{\Omega(g)-1}))$. 
The right hand side is always defined, the left hand side need not be.

The homomorphism $\upsilon \restriction D$ (which we also denote by
$\upsilon$) is however strong. Now look at the relation
$\Theta := \{\auf \Gs_0, \Gs_1\zu : \upsilon(\Gs_0) =
\upsilon(\Gs_1)\}$. $\Theta$ is a congruence on $\goth{Def}$; for it clearly
is an equivalence relation and if $\Gs_i\; \Theta\; \Gu_i$ for all 
$i < \Omega(f)$ then $f(\vec{\Gs})$ is defined iff $f(\vec{\Gu})$ is. 
And in this case we have $f(\vec{\Gs}) \; \Theta\; f(\vec{\Gu})$.
We can now put:
%%
\begin{equation}
f^{\GA}(\auf [\Gs_i]{\Theta} : i < \Omega(f)\zu)
:= [f(\auf \Gs_i : i < \Omega(f)\zu)]{\Theta}
\end{equation}
%%
This is well--defined and we get an algebra, the algebra
$\goth{Def}/\Theta$. The following is easy to see.
%%
\begin{prop}
$\GA \cong\goth{Def}/\Theta.$
\end{prop}
%%
So, $\goth{Def}/\Theta$ is isomorphic to the algebra of signs. 
For every sign there is a structure term, but there might 
also be several. As an instructive example we look at the
sign system of triples of the form $\auf \mbox{\showclock{4}{45}}, 
T, 285\zu$, where \showclock{4}{45} is the arrangement of 
hands of an ordinary clock (here showing 4:45), $T$ a fixed 
letter, and $285$ the number of minutes past midnight/noon 
that is symbolized by this arrangement. So, the above triple 
is a sign of the language, while $\auf \mbox{\showclock{3}{10}}, 
T, 177\zu$ is not, since the hands show 3:10, which equals 190 
minutes, not 177. We propose two modes: {\mtt N} (the zero, 0--ary) 
and {\mtt S} (the successor function, unary). So, the unfolding of 
{\mtt N} is $\auf \mbox{\showclock{0}{0}}, T, 0\zu$, and the unfolding 
of {\mtt S} is the advancement by one minute. Then 
$\upsilon(\mbox{\mtt S})$ is a total function, and we have
%%
\begin{equation}
\upsilon(\mbox{\mtt N}) = \upsilon(\mbox{\mtt S}^{720}\mbox{\mtt N}) 
\end{equation}
%%
From this one easily gets that for every structure term 
$\Gs$, $\upsilon(\Gs) = \upsilon(\mbox{\mtt S}^{720}\Gs)$. 
Hence every sign has infinitely many structure terms, and so is
inherently structurally ambiguous. If instead we take as meanings 
the natural numbers (say, the minutes that elapsed since some fixed 
reference point) and $\mbox{\mtt N}^{\mu} := 0$ as well as 
$\mbox{\mtt S}^{\mu} := \lambda n.n+1$ then every structure term 
represents a different sign! However, still there are only 720 
exponents. Only that every exponent has infinitely many meanings.

We shall illustrate the concepts of a sign grammar by proceeding
with our initial example. Our alphabet is now
%%
\begin{equation}
R := \{\mbox{\mtt 0},
\mbox{\mtt 1}, \dotsc, \mbox{\mtt 9}, \mbox{\mtt +}, \mbox{\mtt -}, 
\mbox{\mtt\symbol{42}}, \mbox{\mtt\symbol{47}}, \mbox{\mtt (},
\mbox{\mtt )}\}
\end{equation}
%%
The algebra $\GE$ consists of $R^{\ast}$ together with some functions
that we still have to determine. We shall now begin to determine the
modes. They are $\mbox{\mtt R}_{\mbox{\smtt +}}$, 
$\mbox{\mtt R}_{\mbox{\smtt -}^2}$,
$\mbox{\mtt R}_{\mbox{\smtt\symbol{42}}}$, 
$\mbox{\mtt R}_{\mbox{\smtt\symbol{47}}}$, which are binary,
$\mbox{\tt R}_{\mbox{\smtt -}^1}$, {\mtt V}, which are unary, and
--- finally --- ten 0--ary modes, namely $\mbox{\mtt Z}_{\snull}$, 
$\mbox{\mtt Z}_{\seins}, \dotsc, \mbox{\mtt Z}_{\sneun}$.

We begin with the 0--ary modes. These are, by definition, signs.
For their identification we only need to know the three components.
For example, to the mode $\mbox{\mtt Z}_{\snull}$ corresponds the triple
$\auf \mbox{\mtt 0}, \mbox{\mtt Z}, 0\zu$. This means: the exponent
of the sign $\mbox{\mtt Z}_{\snull}$ (what we get to see) is the digit
{\mtt 0}; its category is {\mtt Z}, and its meaning the number 0.
Likewise with the other 0--ary modes. Now on to the unary modes.
These are operations taking signs to make new signs.
We begin with $\mbox{\mtt R}_{\mbox{\smtt -}^1}$. On the level of 
strings we get the polynomial 
$\mbox{\mtt R}_{\mbox{\smtt -}^1}^{\varepsilon}$,
which is defined as follows.
%%
\begin{equation}
\mbox{\mtt R}^{\varepsilon}_{\mbox{\smtt -}^1}(\vec{x}) :=
    \mbox{\mtt (-$\vec{x}$)} 
\end{equation}
    %%
On the level of categories we get the function
%%
\begin{equation}
\mbox{\mtt R}^{\gamma}_{\mbox{\smtt -}^1}(c) :=
    \begin{cases}
    \mbox{\mtt T} & \text{ if $c = \mbox{\mtt Z}$,} \\
    \star        & \text{ otherwise.}
    \end{cases}
\end{equation}
%%
Here $\star$ is again the symbol for the fact that the function is
not defined. Finally we have to define 
$\mbox{\mtt R}^{\mu}_{\mbox{\smtt -}^1}$. We put
%%
\begin{equation}
\mbox{\mtt R}^{\mu}_{\mbox{\smtt -}^1}(x) := - x 
\end{equation}
%%
Notice that even if the function $x \mapsto -x$ is iterable,
the mode $\mbox{\mtt R}_{\mbox{\smtt -}^1}$ is not. This is made impossible
by the categorial assignment. This is an artefact of the example.
We could have set things up differently. The mode {\mtt V} finally
is defined by the following functions.
$\mbox{\mtt V}^{\varepsilon}(\vec{x}) := \vec{x}$,
$\mbox{\mtt V}^{\mu}(x) := x$ and $\mbox{\mtt V}^{\gamma}(c) :=
\mbox{\mtt R}_{\mbox{\smtt -}^1}(c)$. Finally we turn to the binary modes.
Let us look at $\mbox{\mtt R}_{\mbox{\smtt\symbol{47}}}$. 
$\mbox{\mtt R}^{\mu}_{\mbox{\smtt\symbol{42}}}$ 
is the partial (!) binary function $\div$ on $\BR$. 
Further, we put
%%
\begin{equation}
\mbox{\mtt R}^{\varepsilon}_{\mbox{\smtt\symbol{47}}}(\vec{x},\vec{y})
    := \mbox{\mtt ($\vec{x}$\symbol{47}$\vec{y}$)}
\end{equation}
    %%
as well as
%%
\begin{equation}
\mbox{\tt R}^{\gamma}_{\mbox{\smtt\symbol{47}}}(c,d) :=
    \begin{cases}
    \mbox{\mtt T} & \text{ if $c = d = \mbox{\mtt T}$,} \\
    \star & \text{ otherwise.}
    \end{cases}
\end{equation}
%%
The string 
{\mtt R$_{\mbox{\smtt\symbol{42}}}$R$_{\mbox{\smtt +}}%
$Z$_{\sdrei}$Z$_{\sfuenf}$Z$_{\ssieben}$}
defines --- as is easily computed --- a sign whose exponent is
{\mtt ((3+5)\symbol{42}7)}. By contrast, 
{\mtt R$_{\mbox{\smtt\symbol{47}}}$Z$_{\szwei}$Z$_{\snull}$}
does {\it not\/} represent a sign. It is syntactically definite
but not semantically, since we may not divide by 0.
%%
\begin{defn}
%%%
\index{sign system!linear}%%
%%%
A \textbf{linear system of signs} over the \textbf{alphabet} $A$,
the set of \textbf{categories} $C$ and the set of \textbf{meanings} 
$M$ is a set $\Sigma \subseteq A^{\ast}\times C\times M$. Further,
let {\mtt S} be a category. Then the interpreted language of
$\Sigma$ with respect to this category {\tt S} is defined by
%%
\begin{equation}
\mbox{\mtt S}(\Sigma) :=
\{\auf \vec{x}, m\zu : \auf \vec{x}, \mbox{\mtt S}, m\zu
\in \Sigma\} 
\end{equation}
%%
\end{defn}
%%
We added the qualifying phrase `linear' to distinguish this
from sign systems which do not generally take strings as
exponents. (For example, pictograms are nonlinear.)

A system of signs is simply a set of signs. The question is
whether one can define an algebra over it. This is always
possible. Just take a 0--ary mode for every sign. Since this
is certainly not as intended, we shall restrict the possibilities
as follows.
%%
\begin{defn}
%%%
\index{sign system!compositional}%%
%%%
Let $\Sigma \subseteq E\times C\times M$ be a
system of signs. We say that $\Sigma$ is
\textbf{compositional} if there is a finite signature
$\Omega$ and partial $\Omega$--algebras
$\GE = \auf E, \{ f^{\GE} : f \in F\}\zu$,
$\GC = \auf C, \{ f^{\GC} : f \in F\}\zu$,
$\GM = \auf M, \{ f^{\GM} : f \in F\}\zu$ such that all
functions are computable and $\Sigma$ is the carrier set 
of the 0--generated partial (strong) subalgebra of  
signs from $\GE \times \GC \times \GM$. $\Sigma$ is
%%%
\index{sign system!weakly compositional}%%
%%%
\textbf{weakly compositional} if there is  a compositional
system $\Sigma'$ such that $\Sigma = \Sigma' \cap E \times 
C \times M$.
\end{defn}
%%
Notice that $\Sigma' \subseteq E' \times C' \times M'$ for certain 
sets $E'$, $C'$ and $M'$. We remark that a partial function 
$f \colon M^n \stackrel{p}{\pf} M$ in the sense of the definition 
above is a computable total function
$f^{\star} \colon M_{\star}^n \pf M_{\star}$ such that
$f^{\star} \restriction M^n = f$. So, the computation always halts,
and we are told at its end whether or not the function is defined
and if so what the value is.

Two conditions have been made: the signature has to be
finite and the functions on the algebras computable. We shall
show that however strong they appear, they do not really
restrict the class of sign systems in comparison to weak
compositionality.

We start by drawing some immediate conclusions from the definitions. 
If $\sigma$ is a sign we say that $\auf {\varepsilon}(\sigma), 
{\gamma}(\sigma), {\mu}(\sigma)\zu$ (no dots!) is its \textbf{realization}. 
%%%
\index{sign!realization}%%%
%%%%
We have introduced the unfolding map $\upsilon$ above.
%%
\begin{prop}
Let $\auf \GA, \varepsilon, \gamma, \mu\zu$ be a compositional
sign grammar. Then the unfolding map is computable.
\end{prop}
%%
Simply note that the unfolding of a structure term can be computed  
inductively. This has the following immediate consequence.
%%
\begin{cor}
\label{cor:recen}
Let $\Sigma$ be compositional. Then $\Sigma$ is recursively
enumerable.
\end{cor}
%%
This is remarkable inasmuch as the set of all signs over
$E \times C\times M$ need not even be enumerable. For
typically $M$ contains uncountably many elements
(which can of course not all be named by a sign)!
%%
\begin{thm}
\label{thm:rekzeichen}
A system of signs is weakly compositional iff it
is recursively enumerable.
\end{thm}
%%
\proofbeg
Let $\Sigma \subseteq E \times C \times M$ be given. If
$\Sigma$ is weakly compositional, it also is recursively
enumerable. Now, let us assume that $\Sigma$ is recursively
enumerable, say $\Sigma = \{\auf e_i, c_i, m_i\zu : 0 < i
\in \omega\}$. (Notice that we start counting with 1.)
Now let {\tt V} be a symbol and $\Delta :=
\{\auf \mbox{\tt V}^n, \mbox{\tt V}^n, \mbox{\tt V}^n\zu :
n \in \omega\}$ a system of signs. By properly choosing {\tt V}
we can see to it that $\Delta \cap \Sigma = \varnothing$
and that no $\mbox{\tt V}^n$ occurs in $E$, $C$ or $M$.
Let $F := \{\mbox{\tt Z}_{\snull}, \mbox{\tt Z}_{\seins}, 
\mbox{\tt Z}_{\szwei}\}$, $\Omega(\mbox{\tt Z}_{\snull}) 
:= 0$, $\Omega(\mbox{\tt Z}_{\seins}) := 1$
and $\Omega(\mbox{\tt Z}_{\szwei}) := 1$.
%%
\begin{equation}
\begin{array}{l@{\quad := \quad}l}
\mbox{\tt Z}_{\snull} & \auf \mbox{\tt V}, \mbox{\tt V}, \mbox{\tt V}\zu, \\
\multicolumn{2}{c}{} \\
\mbox{\tt Z}_{\seins}(\sigma) &
    \begin{cases}
    \auf \mbox{\tt V}^{i+1},
    \mbox{\tt V}^{i+1}, \mbox{\tt V}^{i+1}\zu &
        \text{ if $\sigma = \auf \mbox{\tt V}^i,
        \mbox{\tt V}^i, \mbox{\tt V}^i\zu$,} \\
    \star & \text{ otherwise,}
    \end{cases} \\
\multicolumn{2}{c}{} \\
\mbox{\tt Z}_{\szwei}(\sigma) &
    \begin{cases}
    \auf e_i, c_i, m_i\zu & \text{ if 
    $\sigma = \auf \text{\tt V}^i, \mbox{\tt V}^i, \mbox{\tt V}^i\zu$,} \\
    \star & \mbox{ otherwise.}
    \end{cases}
\end{array}
\end{equation}
%%
This is well--defined. Further, the functions are all computable.
For example, the map $\mbox{\tt V}^i \mapsto e_i$ is computable
since it is the concatenation of the computable functions
$\mbox{\tt V}^i \mapsto i$, $i \mapsto \auf e_i, c_i,
m_i\zu$ with $\auf e_i, c_i, m_i\zu \mapsto e_i$. We claim:
the system of signs generated is exactly $\Delta \cup \Sigma$.
For this we notice first that a structure term is definite iff
it has the following form.
(a) $t = \mbox{\tt Z}_{\seins}^i\mbox{\tt Z}_{\snull}$, or
(b) $t = \mbox{\tt Z}_{\szwei}\mbox{\tt Z}_{\seins}^i\mbox{\tt Z}_{\snull}$.
In Case (a) we get the sign $\auf \mbox{\tt V}^{i+1},
\mbox{\tt V}^{i+1}, \mbox{\tt V}^{i+1}\zu$, in Case
(b) the sign $\auf e_{i+1}, c_{i+1}, m_{i+1}\zu$.
Hence we generate exactly $\Delta \cup \Sigma$. So,
$\Sigma$ is weakly compositional.
\proofend

Notice that the algebra of exponents uses additional symbols
which are only used to create new objects which are like
natural numbers. The just presented algebra is certainly not
very satisfying. (It is also not compositional.) Hence one has
sought to provide a more systematic theory of categories and their
meanings. A first step in this direction are the categorial
grammars. To motivate them we shall give a construction for
CFGs that differs markedly from the one in
Theorem~\ref{thm:rekzeichen}. The starting point is once again
an interpreted language $\CI = \{\auf \vec{x}, f(\vec{x})\zu :
\vec{x} \in L\}$, where $L$ is context free and $f$
computable. Then let $G = \auf \mbox{\tt S}, N, A, R\zu$ be
a CFG with $L(G) = L$. Put
$A' := A$, $C' := N \cup \{\mbox{\tt S}^{\heartsuit}\}$
and $M' := M \cup A^{\ast}$. For simplicity we presuppose 
that $G$ is already in Chomsky Normal Form.
For every rule $\rho$ of the form $\rho = A \pf \vec{x}$
we take a 0--ary mode $\mbox{\tt R}_{\rho}$, which is defined
as follows:
%%
\begin{equation}
\mbox{\tt R}_{\rho} := \auf \vec{x}, A, \vec{x}\zu 
\end{equation}
%%
For every rule $\rho$ of the form $\rho = A \pf B\ C$ we take
a binary mode $\mbox{\tt R}_{\rho}$ defined by
%%
\begin{equation}
\mbox{\tt R}_{\rho}(\auf \vec{x}, B, \vec{x}\zu,
    \auf \vec{y}, C, \vec{y}\zu) :=
    \auf \vec{x}\, \vec{y}, A, \vec{x}\, \vec{y}\,\zu 
\end{equation}
%%
Finally we choose a unary mode {\tt S}:
%%
\begin{equation}
\mbox{\tt S}(\auf \vec{x}, \mbox{\tt S}, \vec{x}\zu) :=
    \auf \vec{x}, \mbox{\tt S}^{\heartsuit}, f(\vec{x})\zu 
\end{equation}
    %%
Then $\CI$ is indeed the set of signs with category
$\mbox{\tt S}^{\heartsuit}$. As one can see, this algebra of signs
is more perspicuous. The strings are just concatenated. The meanings,
however, are not the ones we expect to see. And the category assignment
is unstructured. This grammar is not compositional, since it still
uses nonstandard meanings. Hence once again some pathological examples, 
which will show that there exist nonrecursive compositional systems of
signs.

Suppose that $\Delta$ is a decidable system of signs. This means
that there are countable sets $E$, $C$ and $M$ such that either 
(i) $\Delta = E \times C \times M$, or (ii) $\Delta = \varnothing$, 
or (iii) there are two computable functions, 
%%%
\begin{equation}
d_{\bullet} : \omega \epi \Delta, \qquad 
d_{\circ} : \omega \epi (E \times C \times M - \Delta)
\end{equation}
%%%%
In particular, $E$, $C$ and $M$ are finite or countable. Also, we can 
find a bijection $\delta_{\bullet} : \kappa \pf \Delta$, where 
$\kappa = |\Delta|$. (Simply generate a list $d_{\bullet}(i)$ for 
$i = 0,1,\dotsc$ and skip repeated items.) Its inverse is also 
computable. Now we look at the projections $\pi_0 : \auf e, c, m\zu 
\mapsto e$, $\pi_1 : \auf e,c,m\zu \mapsto c$ and $\pi_2 : \auf e,c,m\zu 
\mapsto m$.
%%%
\begin{defn}
%%%
\label{sign system!enumerative}
%%%
Let $\Delta$ be a system of signs. $\Delta$ is called 
\textbf{enumerative} if the projections $\pi_0$, $\pi_1$, 
and $\pi_2$ are either bijective and computable or constant.
\end{defn}
%%%
Here is an enumerative subsystem of English. Take $E$ to be 
the set of number names of English (see Section~\ref{kap2}.\ref{kap2-6}), 
$C = \{\nu\}$, where $\nu$ is the category of numbers, and 
$M = \omega$. Now let $\CE$ be the set of signs $\auf \vec{x}, 
\nu, n\zu$, where $\vec{x}$ names the number $n$ in English.
It is straightforward to check that $\CE$ is enumerative.

Let $\Delta$ be enumerative. We introduce two modes, {\mtt N} 
(zeroary) and {\mtt S} (unary) and say that 
%%%
\begin{equation}
\begin{split}
\mbox{\mtt N} & := \delta_{\bullet}(0) \\
\mbox{\mtt S}(\sigma) & := \delta_{\bullet}%
(\delta_{\bullet}^{-1}(\sigma) +1)
\end{split} 
\end{equation}
%%%
This generates $\Delta$, as is easily verified. This, however, is 
not compositional, unless we can show that the {\mtt S} can be 
defined componentwise. Therefore put
%%%
\begin{equation}
\mbox{\mtt S}^{\varepsilon}(e) := 
\begin{cases} 
e & \text{if $\pi_0$ is constant,} \\
\pi_0(\mbox{\mtt S}(\pi_0^{-1}(e))) & \text{otherwise.}
\end{cases}
\end{equation}
%%%
This is computable if it is decidable whether or not $e$ is 
in the image of $\pi_0$. So, the set $\pi_0[\Delta]$ must be 
decidable. Similarly $\mbox{\mtt S}^{\gamma}$ and 
$\mbox{\mtt S}^{\mu}$ are defined, and are computable if 
$\pi_1[\Delta]$ and $\pi_2[\Delta]$, respectively, are decidable.
%%%
\begin{defn}
%%%
\index{sign system!modularly decidable}%%%
%%%%
$\Delta$ is called \textbf{modularly decidable} if $\Delta$, 
$\pi_0[\Delta]$, $\pi_1[\Delta]$ and $\pi_2[\Delta]$ are decidable. 
\end{defn}
%%%
\begin{thm}
\label{thm:enum}
Suppose that $\Delta$ is modularly decidable and enumerative. Then 
$\Delta$ is compositional.
\proofend
\end{thm}
%%%
\begin{thm}[Extension]
\label{thm:erweiterung}
Let $\Sigma \subseteq E \times C \times M$ be a recursively 
enumerable set of signs. Let $\Delta \subseteq \Sigma$ be 
modularly decidable and enumerative. Assume that $E$ is finite 
iff $\pi_0$ is constant on $\Delta$; similarly for $C$ and $M$. 
Then $\Sigma$ is compositional.
\end{thm}
%%
\proofbeg
We first assume that $E$, $C$ and $M$ are all infinite. By 
Theorem~\ref{thm:enum}, $\Delta$ is compositional. Further, 
$\Sigma$ is recursively enumerable. So there is a computable function 
$\xi : \omega \epi \Sigma$. Moreover, $\delta^{-1}_{\bullet}$ is 
also computable, and so $\xi \circ \delta^{-1}_{\bullet} : 
\Delta \epi \Sigma$ is computable. Add a unary mode 
{\mtt F} to the signature and let 
%%%
\begin{align}
\notag
\mbox{\mtt F}^{\varepsilon}(e) := & 
\pi_0((\xi \circ \delta^{-1}_{\bullet})(\pi_0^{-1}(e))) \\
\mbox{\mtt F}^{\varepsilon}(c) := & 
\pi_1((\xi \circ \delta^{-1}_{\bullet})(\pi_1^{-1}(c))) \\
\notag
\mbox{\mtt F}^{\varepsilon}(m) := & 
\pi_2((\xi \circ \delta^{-1}_{\bullet})(\pi_2^{-1}(m))) 
\end{align}
%%%
(On all other inputs the functions are not defined.)
This is well--defined and surjective. $\auf \mbox{\tt F}^{\varepsilon}, 
\mbox{\mtt F}^{\gamma}, \mbox{\mtt F}^{\mu}\zu$ is partial, computable, 
and defined only on $\Delta$. Its full image is $\Sigma$.
Now assume that one of the projections, say $\pi_0$, is constant. 
Then $E$ is finite, by assumption on $\Sigma$, say 
$E = \{e_i : i < n\}$ for some $n$. Then put $\Sigma_i := \Sigma 
\cap (\{e_i\} \times C \times M)$. $\Sigma_i$ is also recursively 
enumerable. We do the proof as before, with an enumeration 
$\xi_i : \omega \epi \Sigma_i$ in place of $\xi$. Assume $n$ 
new unary modes, $\mbox{\mtt G}_i$, and put
%%%
\begin{align}
\notag
\mbox{\mtt G}_i^{\varepsilon}(e) := & 
e_i \\
\mbox{\mtt G}_i^{\varepsilon}(c) := & 
\pi_1((\xi_i \circ \delta^{-1}_{\bullet})(\pi_1^{-1}(c))) \\
\notag
\mbox{\mtt G}_i^{\varepsilon}(m) := & 
\pi_2((\xi_i \circ \delta^{-1}_{\bullet})(\pi_2^{-1}(m))) 
\end{align}
%%%
All $\auf \mbox{\mtt G}_i^{\varepsilon}, \mbox{\mtt G}_i^{\gamma}, 
\mbox{\tt G}_i^{\mu}\zu$ are computable, partial, and defined 
exactly on $\Delta$, which they map onto $\Sigma_i$. 
%%%
\proofend

In this construction all occurring signs are in $\Sigma$. Still, we 
do want to say that the grammar just constructed is compositional. 
Namely, if we apply $\mbox{\mtt F}^{\varepsilon}$ to the string 
$\vec{x}$ we may get a string that may have nothing to do with 
$\vec{x}$ at all. Evidently, we need to further restrict our 
operations, for example, by not allowing arbitrary 
string manipulations. We shall deal with this problem in 
Section~\ref{kap4}.\ref{kap4-7}.

Compositionality in the weak sense defines semantics as an
autonomous component of language. When a rule is applied,
the semantics may not `spy' into the phonological form or
the syntax to see what it is supposed to do. Rather, it acts
autonomously, without that knowledge. Its only input is the
semantics of the argument signs and the mode that is being
applied. In a similar way syntax is autonomous from phonology
and semantics. That this is desirable has been repeatedly
argued for by Noam Chomsky. It means that syntactic rules apply
regardless of the semantics or the phonological form. It is
worthwile to explain that our notion of compositionality not
only makes semantics autonomous from syntax and phonology, but
also syntax autonomous from phonology and semantics and phonology
autonomous from syntax and semantics.

{\it Notes on this section.} The notion of sign defined here is
the one that is most commonly found in linguistics. In essence
it goes back to de Saussure \shortcite{desaussure:grundfragen}, 
%%%
\index{de Saussure, Ferdinand}%%%
%%%
published posthumously in 1916, who takes a linguistic 
sign to consist of a signifier and denotatum 
(see also Section~\ref{kap4}.\ref{kap4-8}). De Saussure therewith diverged 
from Peirce, 
%%%
\index{Peirce, Charles S.}%%%
%%%%
for whom a sign was a triadic relation between the signifier, the 
interpreting subject and the denotatum. (See also \cite{lyons:semantics}
for a discussion.) On the other hand, following the mainstream we 
have added to de Saussure signs the category, which is nothing but 
a statement of the combinatorics of that sign. This structure of a 
sign is most clearly employed, for example, in Montague Grammar and 
%%%
\index{Montague Grammar (see Montague Semantics)}%%
\index{Montague Semantics}%%%
in the Meaning--to--Text framework of Igor Mel'\v{c}uk 
%%%
\index{Mel'\v{c}uk, Igor}%%%
%%%
(see for example \cite{melcuk:morphologie}). Other theories, 
for example early HPSG 
and Unification Categorial Grammar also use the tripartite distinction 
between what they call phonology, syntax and semantics, but signs 
are not triples but much more complex in structure.

The distinction between compositionality and weak compositionality
turns on the question whether the generating functions should
work inside the language or whether they may introduce new objects.
We strongly opt for the former not only because it gives us a
stronger notion. The definition in its informal rendering makes
reference to the parts of an expression and their meanings --- and 
in actual practice the parts from which we compose an expression 
do have meanings, and it is these meanings we employ in forming 
the meaning of a complex expression.
%%
%\vplatz
%\exercise
%Call $\Delta$ $\gamma$--{\bf unique} if the following holds:
%%%
%\begin{dingautolist}{192}
%\item
%If $\auf e, \gamma, m\zu, \auf e', \gamma, m\zu \in \Delta$ then
%$e = e'$.
%\item
%If $\auf e, \gamma, m\zu, \auf e, \gamma, m'\zu \in \Delta$ then
%$m = m'$.
%\end{dingautolist}
%%%
%Let $C = \{\gamma\}$ and $\Delta$ be modularly decidable and
%$\gamma$--unique. Show that $\Delta$ is compositional.
%%
%\vplatz
%\exercise
%Let $C$ be finite, $E = A^{\ast}$ and let $\Delta$ be
%$\gamma$--unique for every $\gamma \in C$. Finally, let
%$\Delta$ be modularly decidable. Show that $\Delta$ is
%compositional. {\it Hint.} Define for every $\gamma$ a
%binary mode $\mbox{\tt C}_{\gamma}$.  Let
%$\mbox{\tt C}_{\gamma}^{\GE}$ be the
%concatenation and $\mbox{\tt C}_{\gamma}^{\GC}(t,t') :=
%t$ if $t = t' = \gamma$, and undefined otherwise. The
%trick is the definition of $\mbox{\tt C}_{\GC}^{\mu}$.
%%
\vplatz
\exercise
\label{ex:transparent}
Let $G = \auf \mbox{\tt S}, N, A, R\zu$ be a CFG. Put 
$N' := N \cup \{\mbox{\tt R}_{\rho} : \rho \in R\}$, and
$R" := \{X \pf \mbox{\tt R}_{\rho}\vec{\alpha} : \rho = X 
	\pf \vec{\alpha} \in R\}$, 
$G' := \auf \mbox{\tt S}, N', A, R'\zu$.  
Show that $G'$ is transparent. 
%%%
\vplatz
\exercise
Show that English satisfies the conditions of
Theorem~\ref{thm:erweiterung}. Hence English
is compositional! 
%%
\vplatz
\exercise
Construct an undecidable set $\Delta$ such that its projections 
$\pi_0[\Delta]$, $\pi_1[\Delta]$ and $\pi_2[\Delta]$ are decidable. 
Construct a $\Delta$ which is decidable but not its projection
$\pi_0[\Delta]$.
%%%
\vplatz
\exercise
Show that the functions postulated in the proof of
Theorem~\ref{thm:erweiterung}, $z_{\gamma}$ and $m_{\gamma}$,
do exist if $\Sigma$ is recursively enumerable.
%%%
\vplatz 
\exercise 
Say that $\Sigma \subseteq E \times C \times M$
is \textbf{extra weakly compositional} if there exists a finite
signature $\Omega$ and $\Omega$--algebras $\GE'$, $\GC'$ and
$\GM'$ over sets $E' \supseteq E$, $C' \supseteq C$ and $M'
\supseteq M$, respectively, such that $\Sigma$ is the carrier 
set of the 0--generated partial subalgebra of 
$\GE' \times \GC' \times \GM'$ which belong to the
set $E \times C \times M$. (So, the definition is like that of
weak compositionality, only that the functions are not
necessarily computable.) Show that $\Sigma$ is extra weakly
compositional iff it is countable. (See also
\cite{zadrozny:compositionality}.)

 \newcommand{\strictif}{\supset}
\section{Propositional Logic}
\label{kap:prop}
%
%
%
Before we can enter a discussion of categorial grammar and type
systems, we shall have to introduce some techniques from
propositional logic. We seize the opportunity to present boolean
logic using our notions of the previous section. The alphabet is
defined to be $A_P := \{\mbox{\mtt p}, \mbox{\mtt 0}, \mbox{\mtt 1},
\mbox{\mtt (}, \mbox{\mtt )}, \sbot, \mbox{\mtt\symbol{25}}\}$. 
Further, let $T := \{P\}$, and $M := \{0,1\}$. Next, we define 
the following modes. The zeroary modes are
%%%%
\begin{equation}
\mbox{\mtt X}_{\vec{\alpha}} := \auf \mbox{\mtt p}\vec{\alpha},
    P, 0\zu, \quad 
\mbox{\mtt Y}_{\vec{\alpha}} := \auf \mbox{\mtt p}\vec{\alpha},
    P, 1\zu, \quad 
\mbox{\mtt M}_{\bot} := \auf \sbot, P, 0\zu 
\end{equation}
%%%
Here, $\vec{\alpha}$ ranges over (possibly empty) sequences of
{\mtt 0} and {\mtt 1}. (So, the signature is infinite.) Further, 
let $\strictif$ be the following function:
%%
\begin{equation}
\begin{array}{l|ll}
\strictif & 0 & 1 \\\hline
0       & 1 & 1 \\
1       & 0 & 1
\end{array}
\end{equation}
%%
The binary mode $\mbox{\tt M}_{\mbox{\smtt\symbol{25}}}$ of implication 
formation is spelled out as follows.
%%
\begin{equation}
\mbox{\tt M}_{\mbox{\smtt\symbol{25}}}(\auf \vec{x}, P, \eta\zu,
    \auf \vec{y}, P, \theta\zu)
:= 
    \auf \mbox{\mtt ($\vec{x}$\symbol{25}$\vec{y}$)}, P,
    \eta \strictif \theta\zu
\end{equation}
%%%
The system of signs generated by these modes is called
\textbf{boolean logic}
%%%
\index{logic!boolean}%%
%%%
and is denoted by $\Sigma_{\mathsf{B}}$. To see that this is
indeed so, let us explain in more conventional terms what these
definitions amount to. First, the string language $L$ we have
defined is a subset of $A_P^{\ast}$, which is generated as follows.
%%
\begin{dingautolist}{192}
\item If $\vec{\alpha} \in \{\mbox{\mtt 0}, \mbox{\mtt 1}\}^{\ast}$,
    then $\mbox{\mtt p}\vec{\alpha} \in L$. These sequences are
    called \textbf{propositional variables}.
%%%
\index{variable!propositional}%%
%%%
\item $\sbot\in L$.
\item If $\vec{x}, \vec{y}\in L$ then $\mbox{\mtt ($\vec{x}$\symbol{25}$%
\vec{y}$)}\in L$.
\end{dingautolist}
%%
$\vec{x}$ is also called a \textbf{well--formed formula} (\textbf{wff}) 
or simply a \textbf{formula} 
%%%
\index{formula!well--formed}%%
\index{wff}%%
%%%
iff it belongs to $L$. There are three kinds of wffs.
%%%
\begin{defn}
\index{tautology}%%
\index{contradiction}%%
\index{formula!contingent}%%
%%%
Let $\vec{x}$ be a well--formed formula. $\vec{x}$ is a 
\textbf{tautology} if $\auf \vec{x}, P, 0\zu \not\in \Sigma_{\mathsf{B}}$. 
$\vec{x}$ is a \textbf{contradiction} if $\auf\vec{x}, P, 1\zu
\not\in \Sigma_{\mathsf{B}}$. If $\vec{x}$ is neither a tautology
nor a contradiction, it is called \textbf{contingent}.
\end{defn}
%%%
The set of tautologies is denoted by 
%%%%
\index{$\Taut_{\mathsf{B}}(\mbox{\mtt\symbol{25}},\sbot)$}%% 
%%%
$\Taut_{\mathsf{B}}(\mbox{\mtt\symbol{25}},\sbot)$, 
or simply by $\Taut_{\mathsf{B}}$ if the
language is clear from the context. It is easy to see that
$\vec{x}$ is a tautology iff {\mtt ($\vec{x}$\symbol{25}$\bot)$} 
is a contradiction. Likewise, $\vec{x}$ is a contradiction iff 
{\mtt ($\vec{x}$\symbol{25}$\bot)$} is a tautology. We now agree 
on the following convention. Lower case Greek letters are proxy 
for well--formed formulae, upper case Greek letters are proxy for 
sets of formulae. Further, we write $\Delta; \varphi$ instead of
$\Delta\cup \{\varphi\}$ and $\varphi;\chi$ in place of
$\{\varphi, \chi\}$.

Our first task will be to present a calculus with which we can
generate all the tautologies of $\Sigma_{\mathsf{B}}$. For this
aim we use a so--called {\it Hilbert style calculus}. Define the
following sets of formulae.
%%
\index{Hilbert (style) calculus}
%%
\begin{equation}
\begin{array}{ll}
\mbox{\rm (a0)} & \mbox{\mtt ($\varphi$\symbol{25}($\psi$\symbol{25}%
$\varphi$))} \\
\mbox{\rm (a1)} & \mbox{\mtt (($\varphi$\symbol{25}($\psi$\symbol{25}%
$\chi$))\symbol{25}(($\varphi$\symbol{25}$\psi$)\symbol{25}(%
$\varphi$\symbol{25}$\chi$)))} \\
\mbox{\rm (a2)} & \mbox{\mtt ($\sbot$\symbol{25}$\varphi$)} \\
\mbox{\rm (a3)} & \mbox{\mtt ((($\varphi$\symbol{25}$\sbot$)\symbol{25}%
$\sbot$)\symbol{25}$\varphi$)}
\end{array}
\end{equation}
%%%
The logic axiomatized by (a0) -- (a3) is known as \textbf{classical} 
or \textbf{boolean logic},
%%%
\index{logic!classical}\index{logic!boolean}%%
%%%
the logic axiomatized by (a0) -- (a2) as \textbf{intuitionistic logic}.
%%%%
\index{intuitionistic logic}%%
\index{logic!intuitionistic}%%%
%%%%
To be more precise, (a0) -- (a3) each are sets of formulae. For example:
\begin{equation}
\mbox{\rm (a0)} = \{\mbox{\mtt ($\varphi$\symbol{25}($\psi%
$\symbol{25}$\varphi$))} : \varphi, \psi \in L\}
\end{equation}
%%%
We call (a0) an \textbf{axiom schema} and
%%%
\index{axiom schema}%%%
\index{axiom schema!instance}%%
%%%
its elements \textbf{instances of} (a0). Likewise with (a1) -- (a3).
%%%
\begin{defn}
%%%
\index{proof}
\index{proof!length of a \faul}%%%
\index{$\vdash^{\mathsf{B}} \varphi$}%%%
%%%%
A finite sequence $\Pi = \auf \delta_i : i < n\zu$ of formulae 
is a $\mathsf{B}$--\textbf{proof of} $\varphi$ if (a) $\delta_{n-1}
= \varphi$ and (b) for all $i < n$ either (b1) $\delta_i$ is an
instance of (a0) -- (a3) or (b2) there are $j, k < i$ such that
$\delta_k = \mbox{\mtt ($\delta_j$\symbol{25}$\delta_i$)}$. The 
number $n$ is called the \textbf{length of} $\Pi$. We write 
$\vdash^{\mathsf{B}} \varphi$ if there is a $\mathsf{B}$--proof of 
$\varphi$.
\end{defn}
%%%
The formulae (a0) -- (a3) are called the \textbf{axioms} of this
%%%
\index{axiom}%%
\index{Modus Ponens (MP)}%%
%%%
calculus. Moreover, this calculus uses a single inference rule,
which is known as Modus Ponens. It is the inference from 
{\mtt ($\varphi$\symbol{25}$\chi$)} and $\varphi$ to
$\chi$. The easiest part is to show that the calculus generates
only tautologies.
%%
\begin{lem}
If $\vdash^{\mathsf{B}} \varphi$ then $\varphi$ is a tautology.
\end{lem}
%%
The proof is by induction on the length of the proof. The completeness
part is somewhat harder and requires a little detour. We shall extend 
the notion of proof somewhat to cover proofs from assumptions.
%%%
\begin{defn}
%%%
\index{proof}%%
%%%
A $\mathsf{B}$--\textbf{proof of} $\varphi$ \textbf{from} $\Delta$ is a
finite sequence $\Pi = \auf \delta_i : i < n\zu$ of formulae such
that (a) $\delta_{n-1} = \varphi$ and (b) for all $i < n$ either
(b1) $\delta_i$ is an instance of (a0) -- (a3) or (b2) there are
$j, k < i$ such that $\delta_k = \mbox{\mtt ($\delta_j$\symbol{25}%
$\delta_i$)}$ or (b3) $\delta_i \in \Delta$. We write
$\Delta \vdash^{\mathsf{B}} \varphi$ if there is a 
$\mathsf{B}$--proof of $\varphi$ from $\Delta$.
\end{defn}
%%%
To understand this notion of a hypothetical proof, we shall introduce
the notion of an \textbf{assignment}. It is common to define an assignment
to be a function from variables to the set $\{0,1\}$. Here, we
shall give an effectively equivalent definition.
%%%
\begin{defn}
%%%
\index{assignment}%%
%%%
An \textbf{assignment} is a maximal subset $A$ of
%%
\begin{equation}
\{\mbox{\mtt X}_{\vec{\alpha}}
: \vec{\alpha} \in (\mbox{\mtt 0} \cup \mbox{\mtt 1})^{\ast}\}
\cup
\{\mbox{\mtt Y}_{\vec{\alpha}}
: \vec{\alpha} \in (\mbox{\mtt 0} \cup \mbox{\mtt 1})^{\ast}\}
\end{equation}
%%
such that for no $\vec{\alpha}$ both
$\mbox{\mtt X}_{\vec{\alpha}}, \mbox{\mtt Y}_{\vec{\alpha}}
\in A$.
\end{defn}
%%%
(So, an assignment is a set of zeroary modes.) Each assignment
defines a closure under the modes $\mbox{\mtt M}_{\bot}$ and
$\mbox{\mtt M}_{\mbox{\smtt\symbol{25}}}$, which we denote by 
%%%%
\index{$\Sigma_{\mathsf{B}}(A)$}%%%
%%%%
$\Sigma_{\mathsf{B}}(A)$.
%%%
\begin{lem}
Let $A$ be an assignment and $\varphi$ a well--formed formula. Then
either $\auf\varphi, P,0\zu \in \Sigma_{\mathsf{B}}(A)$ or
$\auf\varphi, P, 1\zu \in \Sigma_{\mathsf{B}}(A)$, but not both.
\end{lem}
%%
The proof is by induction on the length of $\vec{x}$. We say that
an assignment $A$ makes a formula $\varphi$ \textbf{true} if $\auf
\varphi, P, 1\zu \in \Sigma_{\mathsf{B}}(A)$.
%%%
\index{truth}%%
%%%
\begin{defn}
Let $\Delta$ be a set of formulae and $\varphi$ a formula. We say
that $\varphi$ \textbf{follows from} (or \textbf{is a consequence of})
%%%%
\index{consequence}%%
\index{$\vDash$}%%
%%%%
$\Delta$ if for all assignments $A$: if $A$ makes all formulae of
$\Delta$ true then it makes $\varphi$ true as well.
In that case we write $\Delta \vDash \varphi$.
\end{defn}
%%%
Our aim is to show that the Hilbert calculus characterizes this
notion of consequence:
%%%
\begin{thm}
\label{thm:hilbertmain}%%
$\Delta \vdash^{\mathsf{B}} \varphi$ iff $\Delta
\vDash \varphi$.
\end{thm}
%%%
Again, the proof has to be deferred until the matter is
sufficiently simplified. Let us first show the following
fact, known as the \textbf{Deduction Theorem} (\textbf{DT}).
%%%
%%%
\begin{lem}[Deduction Theorem]
\index{Deduction Theorem}
$\Delta; \varphi \vdash^{\mathsf{B}} \chi$ iff $\Delta
\vdash^{\mathsf{B}}
    \mbox{\mtt ($\varphi$\symbol{25}$\chi$)}$.
\end{lem}
%%%
\proofbeg 
The direction from right to left is immediate and left
to the reader. Now, for the other direction suppose that $\Delta;
\varphi \vdash^{\mathsf{B}} \chi$. Then there exists a proof $\Pi
= \auf \delta_i : i < n\zu$ of $\chi$ from $\Delta;\varphi$.  We
shall inductively construct a proof $\Pi' = \auf \delta'_j : j <
m\zu$ of {\mtt ($\varphi$\symbol{25}$\chi$)} from $\Delta$.
The construction is as follows. We define $\Pi_i$ inductively. 
%%
\begin{equation}
\Pi_0 := \varepsilon, \qquad \Pi_{i+1} := \Pi_i^{\smallfrown}\Sigma_i,
\end{equation}
%%
where $\Sigma_i$, $i < n$, is defined as given below. Furthermore,
we will verify inductively that $\Pi_{i+1}$ is a proof of its last
formula, which is {\mtt ($\varphi$\symbol{25}$\delta_i$)}.
Then $\Pi' := \Pi_n$ will be the desired proof, since $\delta_{n-1}
= \chi$.  Choose $i < n$. Then either (1) $\delta_i \in \Delta$ or
(2) $\delta$ is an instance of (a0) -- (a3) or (3) $\delta_i = \varphi$
or (4) there are $j, k < i$ such that $\delta_k = \mbox{\mtt ($\delta_j$%
\symbol{25}$\delta_i$)}$. In the first two cases we put
$\Sigma_i := \auf\delta_i, \mbox{\mtt ($\delta_i$\symbol{25}(%
$\varphi$\symbol{25}$\delta_i$))}, \mbox{\mtt ($\varphi$%
\symbol{25}$\delta_i$)}\zu$. In Case (3) we put
%%
\begin{align}
%\begin{array}{l@{}l}
\Sigma_i := \quad \auf &
\mbox{\mtt (($\varphi$\symbol{25}(($\varphi$\symbol{25}$\varphi$%
)\symbol{25}$\varphi$))\symbol{25}(($\varphi$\symbol{25}(%
$\varphi$\symbol{25}$\varphi$))\symbol{25}($\varphi$\symbol{25}%
$\varphi$)))}, \\\notag
&
    \mbox{\mtt ($\varphi$\symbol{25}(($\varphi$\symbol{25}$\varphi$%
)\symbol{25}$\varphi$))}, \\\notag
&
    \mbox{\mtt (($\varphi$\symbol{25}($\varphi$\symbol{25}$\varphi$))%
\symbol{25}($\varphi$\symbol{25}$\varphi$))}, \\\notag
&
    \mbox{\mtt ($\varphi$\symbol{25}($\varphi$\symbol{25}%
$\varphi$))}, \\\notag
&
    \mbox{\mtt ($\varphi$\symbol{25}$\varphi$)}\zu
\end{align}
%%
$\Sigma_i$ is a proof of {\mtt ($\varphi$\symbol{25}$\varphi$)},
as is readily checked. Finally, Case (4). There are $j, k < i$ such
that $\delta_k = \mbox{\mtt ($\delta_j$\symbol{25}$\delta_i$)}$.
Then, by induction hypothesis, {\mtt ($\varphi$\symbol{25}$\delta_j$)}
and $\mbox{\mtt ($\varphi$\symbol{25}$\delta_k$)} =
\mbox{\mtt ($\varphi$\symbol{25}($\delta_j$\symbol{25}$\delta_i$))}$
already occur in the proof. Then put
%%
\begin{align}
%\begin{array}{l@{}l}
\Sigma_i := \quad \auf &
\mbox{\mtt (($\varphi$\symbol{25}($\delta_j$\symbol{25}$\delta_i$))%
\symbol{25}(($\varphi$\symbol{25}$\delta_j$)\symbol{25}($\varphi$%
\symbol{25}$\delta_i$)))}, \\\notag
&
    \mbox{\mtt (($\varphi$\symbol{25}$\delta_j$)\symbol{25}(%
$\varphi$\symbol{25}$\delta_i$))}, \\\notag
&
    \mbox{\mtt ($\varphi$\symbol{25}$\delta_i$)}
    \zu
\end{align}
%%
It is verified that $\Pi_{i+1}$ is a proof of {\mtt ($\varphi$\symbol{25}%
$\delta_i$)}.
\proofend

A special variant is the following.
%%%
\begin{lem}[Little Deduction Theorem]
%%%
\index{Little Deduction Theorem}%%%
%%%
For all $\Delta$ and $\varphi$: $\Delta\vdash^{\mathsf{B}}\varphi$ 
if and only if $\Delta;\mbox{\mtt ($\varphi$\symbol{25}$\sbot$)}%
\vdash^{\mathsf{B}}\sbot$.
\end{lem}
%%%
\proofbeg 
Assume that $\Delta\vdash^{\mathsf{B}}\varphi$. Then
there is a proof $\Pi$ of $\varphi$ from $\Delta$. It follows that 
$\Pi^{\smallfrown}\auf \mbox{\mtt ($\varphi$\symbol{25}$\sbot$)},
\sbot\zu$ is a proof of $\sbot$ from $\Delta;\mbox{\mtt ($\varphi$
\symbol{25}$\sbot$)}$. Conversely, assume that
$\Delta;\mbox{\mtt ($\varphi$\symbol{25}$\sbot$)} \vdash^{\mathsf{B}}
\sbot$. Applying DT we get $\Delta\vdash^{\mathsf{B}}\mbox{\mtt
(($\varphi$\symbol{25}$\sbot$)\symbol{25}$\sbot$)}$. Using (a3) we
get $\Delta\vdash^{\mathsf{B}}\varphi$. 
\proofend
%%%
\begin{prop}
\label{prop:deduce}
The following holds.
\begin{dingautolist}{192}
\item $\varphi \vdash^{\mathsf{B}} \varphi$.
\item If $\Delta \subseteq  \Delta'$ and
    $\Delta \vdash^{\mathsf{B}}\varphi$
    then also $\Delta' \vdash^{\mathsf{B}} \varphi$.
\item If $\Delta \vdash^{\mathsf{B}} \varphi$ and
    $\Gamma; \varphi \vdash^{\mathsf{B}} \chi$
    then $\Gamma;\Delta \vdash^{\mathsf{B}} \chi$.
\end{dingautolist}
\end{prop}
%%%
This is easily verified. Now we are ready for the proof of
Theorem~\ref{thm:hilbertmain}. An easy induction on the length of
a proof establishes that if $\Delta \vdash^{\mathsf{B}} \varphi$
then also $\Delta \vDash \varphi$. (This is called the {\it
correctness\/} of the calculus.) So, the converse implication,
which is the {\it completeness\/} part needs proof. Assume that
$\Delta \nvdash^{\mathsf{B}} \varphi$. We shall show that also
$\Delta \nvDash \varphi$. Call a set $\Sigma$ \textbf{consistent} 
(\textbf{in} $\vdash^{\mathsf{B}}$)  if
%%%
\index{set!consistent}%%
%%%
$\Sigma \nvdash^{\mathsf{B}} \sbot$.
%%%
\begin{lem}
\label{lem:tableau}
\begin{dingautolist}{192}
\item
Let $\Delta;\mbox{\mtt ($\varphi$\symbol{25}$\chi$)}$ be consistent.
Then either $\Delta;\mbox{\mtt ($\varphi$\symbol{25}$\sbot$)}$ is
consistent or $\Delta;\chi$ is consistent.
\item
Let $\Delta;\mbox{\mtt (($\varphi$\symbol{25}$\chi$)\symbol{25}%
$\sbot$)}$ be consistent. Then also $\Delta;\varphi;\mbox{\mtt 
($\chi$\symbol{25}$\sbot$)}$ is consistent.
\end{dingautolist}
\end{lem}
%%%
\proofbeg 
\ding{192}. Assume that both $\Delta;\mbox{\mtt ($\varphi$\symbol{25}%
$\sbot$)}$ and $\Delta;\chi$ are inconsistent. Then we have $\Delta;\mbox{\mtt
($\varphi$\symbol{25}$\sbot$)} \vdash^{\mathsf{B}} \sbot$ and
$\Delta;\chi\vdash^{\mathsf{B}} \sbot$. So
$\Delta\vdash^{\mathsf{B}} \mbox{\mtt (($\varphi$\symbol{25}%
$\sbot$)\symbol{25}$\sbot$)}$ by DT and, using (a3), 
$\Delta\vdash^{\mathsf{B}} \varphi$. Hence $\Delta;\mbox{\mtt %
($\varphi$\symbol{25}$\chi$)}%
\vdash^{\mathsf{B}} \varphi$ and so $\Delta;\mbox{\mtt
($\varphi$\symbol{25}$\chi$)} \vdash^{\mathsf{B}} \chi$. Because
$\Delta;\chi\vdash^{\mathsf{B}} \sbot$, we also have
$\Delta;\mbox{\mtt ($\varphi$\symbol{25}$\chi$)}\vdash^{\mathsf{B}}
\sbot$, showing that $\Delta;\mbox{\mtt ($\varphi$\symbol{25}$\chi$)}$ 
is inconsistent. \ding{193}. Assume $\Delta;\varphi;\mbox{\mtt ($\chi$%
\symbol{25}$\sbot$)}$ is inconsistent. Then 
$\Delta;\varphi;\mbox{\mtt ($\chi$\symbol{25}%
$\sbot$)}\vdash^{\mathsf{B}} \sbot$. So,
$\Delta;\varphi\vdash^{\mathsf{B}} \mbox{\mtt
(($\chi$\symbol{25}$\sbot$)\symbol{25}$\sbot$)}$, by applying DT.
So, $\Delta;\varphi\vdash^{\mathsf{B}}\chi$, using (a3). Applying
DT we get $\Delta \vdash^{\mathsf{B}} \mbox{\mtt
($\varphi$\symbol{25}$\chi$)}$. Using (a3) and DT once again it is
finally seen that $\Delta;\mbox{\mtt (($\varphi$\symbol{25}$\chi$%
)\symbol{25}$\sbot$)}$ is inconsistent. 
\proofend

Finally, let us return to our proof of the completeness theorem.
We assume that $\Delta \nvdash^{\mathsf{B}} \varphi$. We have to
find an assignment $A$ that makes $\Delta$ true but not
$\varphi$. We may also apply the Little DT and assume that
$\Delta;\mbox{\mtt ($\varphi$\symbol{25}$\sbot$)}$ is consistent and
find an assignment that makes this set true. The way to find such
an assignment is by applying the so--called downward closure of
the set.
%%%
\begin{defn}
\index{set!downward closed}%%
%%
A set $\Delta$ is \textbf{downward closed} iff (1) for
all $\mbox{\mtt ($\varphi$\symbol{25}$\chi$)} \in \Delta$ either
$\mbox{\mtt ($\varphi$\symbol{25}$\sbot$)} \in \Delta$ or $\chi \in
\Delta$ and (2) for all formulae $\mbox{\mtt (($\varphi$\symbol{25}$\chi$%
)\symbol{25}$\sbot$)} \in \Delta$ also $\varphi \in \Delta$ and 
$\mbox{\mtt ($\chi$\symbol{25}$\sbot$)} \in \Delta$.
\end{defn}
%%%
Now, by Lemma~\ref{lem:tableau} every consistent set has a consistent
closure $\Delta^{\ast}$. (It is an exercise for the diligent reader
to show this. In fact, for infinite sets a little work is needed here,
but we really need this only for finite sets.) Define the following
assignment.
%%
\begin{align}
%\begin{array}{l@{}l@{}l}
A \quad := & \phantom{\mbox{}\cup\mbox{}}
    \{\auf \mbox{\mtt p}\vec{\alpha}, P, 1\zu :
    \text{\mtt (p$\vec{\alpha}$\symbol{25}$\sbot$)}
    \text{ does not occur in }\Delta^{\ast}\} \\\notag
        & \cup
    \{\auf \mbox{\mtt p}\vec{\alpha}, P, 0\zu :
    \text{\mtt (p$\vec{\alpha}$\symbol{25}$\sbot$)}
    \text{ does occur in }\Delta^{\ast}\}
\end{align}
%%
It is shown by induction on the formulae of $\Delta^{\ast}$ that
the so--defined assignment makes every formula of $\Delta^{\ast}$
true. Using the correspondence between syntactic derivability
and semantic consequence we immediately derive the following.
%%%
\begin{thm}[Compactness Theorem]
%%%%
\index{Compactness Theorem}%%
%%%%
Let $\varphi$ be a formula and $\Delta$ a set of formulae such
that $\Delta \vDash \varphi$. Then there exists a finite set $\Delta'
\subseteq \Delta$ such that $\Delta' \vDash \varphi$.
\end{thm}
%%
\proofbeg 
Suppose that $\Delta \vDash \varphi$. Then $\Delta
\vdash^{\mathsf{B}} \varphi$. Hence there exists a proof of
$\varphi$ from $\Delta$. Let $\Delta'$ be the set of those
formulae in $\Delta$ that occur in that proof. $\Delta'$ is
finite. Clearly, this proof is a proof of $\varphi$ from
$\Delta'$, showing $\Delta' \vdash^{\mathsf{B}} \varphi$. Hence
$\Delta' \vDash \varphi$. 
\proofend

Usually, one has more connectives than just $\sbot$ and 
{\mtt\symbol{25}}. Now, two effectively equivalent strategies suggest 
themselves, and they are used whenever convenient. The first is to 
introduce a new connective as an abbreviation. So, we might define 
(for well--formed formulae)
%%
\begin{align}
\nicht\varphi & := \mbox{\mtt $\varphi$\symbol{25}$\sbot$} \\
\varphi\oder\chi & := \mbox{\mtt ($\varphi$\symbol{25}$\sbot$)%
\symbol{25}$\chi$} \\
\varphi\und\chi & := \mbox{\mtt ($\varphi$\symbol{25}($\chi$%
\symbol{25}$\sbot$))\symbol{25}$\sbot$} 
\end{align}
%%
After the introduction of these abbreviations, everything is
the same as before, because we have not changed the language,
only our way of referring to its strings. However, we may also
change the language by expanding the alphabet. In the cases at hand
we will add the following unary and binary modes (depending on which
symbol is to be added):
%%
\begin{align}
\mbox{\mtt M}_{\mbox{\smtt\symbol{5}}}(\auf \vec{x}, P, \eta\zu) &
    := \auf \mbox{\mtt (\symbol{5}$\vec{x}$)}, P, -\eta\zu \\
\mbox{\mtt M}_{\mbox{\smtt\symbol{31}}}(\auf \vec{x}, P, \eta\zu,
    \auf \vec{y}, P, \theta\zu)
    &
    := \auf \mbox{\mtt ($\vec{x}$\symbol{31}$\vec{y}$)}, P,
    \eta \cup \theta\zu \\
\mbox{\mtt M}_{\mbox{\smtt\symbol{4}}}(\auf \vec{x}, P, \eta\zu,
    \auf \vec{y}, P, \theta\zu)
    &
    := \auf \mbox{\mtt ($\vec{x}$\symbol{4}$\vec{y}$)}, P,
    \eta \cap \theta\zu
\end{align}
%%%
\begin{equation}
\begin{array}{l|ll}
\cup & 0 & 1 \\\hline
0    & 0 & 1 \\
1    & 1 & 1
\end{array}\qquad
\begin{array}{l|ll}
\cap & 0 & 1 \\\hline
0    & 0 & 0 \\
1    & 0 & 1
\end{array}\qquad
\begin{array}{l|l}
  & - \\\hline
0 & 1 \\
1 & 0
\end{array}
\end{equation}
%%
For {\mtt\symbol{4}}, {\mtt\symbol{25}} and {\mtt\symbol{5}} 
we need the postulates shown in \eqref{eq:p4}, \eqref{eq:p25} 
and \eqref{eq:p5}, respectively:
%%
\begin{align}
\label{eq:p4}
\mbox{\mtt ($\varphi$\symbol{25}($\psi$\symbol{25}($\varphi$%
\symbol{4}$\psi$)))}, &
\mbox{\mtt ($\varphi$\symbol{25}($\psi$\symbol{25}($\psi$%
\symbol{4}$\varphi$)))}, \\\notag
\mbox{\mtt (($\varphi$\symbol{4}$\psi$)\symbol{25}$\varphi$)},
&
\mbox{\mtt (($\varphi$\symbol{4}$\psi$)\symbol{25}$\psi$)} \\
\label{eq:p25}
\mbox{\mtt ($\varphi$\symbol{25}($\varphi$\symbol{31}$\psi$))},
& 
\mbox{\mtt ($\psi$\symbol{25}($\varphi$\symbol{31}$\psi$))}, \\\notag
\mbox{\mtt ((($\varphi$\symbol{31}$\psi$)\symbol{25}$\chi$)%
\symbol{25}($\varphi$\symbol{25}$\chi$))},
& 
\mbox{\mtt ((($\varphi$\symbol{31}$\psi$)\symbol{25}$\chi$)%
\symbol{25}($\psi$\symbol{25}$\chi$))} \\
\label{eq:p5}
\mbox{\mtt (($\varphi$\symbol{25}$\psi$)\symbol{25}%
((\symbol{5}$\psi$)\symbol{25}(\symbol{5}$\varphi$)))},
& 
\mbox{\mtt ($\varphi$\symbol{25}(\symbol{5}(\symbol{5}$\varphi$)))}
\end{align}
%%
Notice that in defining the axioms we have made use of {\mtt\symbol{25}}
alone. The formula \eqref{eq:classneg} is derivable. 
%%%
\begin{equation}
\label{eq:classneg}
\mbox{\mtt ((\symbol{5}(\symbol{5}$\varphi$))\symbol{25}$\varphi$)}
\end{equation}
%%%
If we eliminate the connective $\sbot$ and define $\Delta \vdash \varphi$
as before (eliminating the axioms (a2) and (a3), however) we get
once again intuitionistic logic, unless we add \eqref{eq:classneg}. 
The semantics of intuitionistic logic 
is too complicated to be explained here, so we just use the Hilbert 
calculus to introduce it. We claim that with only (a0) and (a1) it 
is not possible to prove all formulae of $\Taut_{\mathsf{B}}$ 
that use only {\mtt\symbol{25}}. A case in point is the formula
%%
\begin{equation}
\mbox{\mtt ((($\varphi$\symbol{25}$\chi$)\symbol{25}$\varphi$)%
\symbol{25}$\varphi$)}
\end{equation}
%%
which is known as {\it Peirce's Formula}. Together with Peirce's
Formula, (a0) and (a1) axiomatize the full set of tautologies of
boolean logic in {\mtt\symbol{25}}. The calculus based on (a0) and 
(a1) is called \textsf{H} and we write $\Delta \vdash^{\mathsf{H}} \chi$ 
to say that there is a proof in the Hilbert calculus of $\chi$ from
$\Delta$ using (a0) and (a1).

Rather than axiomatizing the set of tautologies we can also axiomatize 
the deducibility relation itself. This idea goes back to Gerhard 
Gentzen, who used it among other to show the consistency of arithmetic 
(which is of no concern here). For simplicity, we stay with the language 
with only the arrow. We shall axiomatize the derivability of intuitionistic 
logic. The statements that we are deriving now have the form 
`$\Delta  \boldsymbol{\vdash} \varphi$' and are called \textbf{sequents}. 
$\Delta$ is called the \textbf{antecedent} and $\varphi$ the 
\textbf{succedent} of that sequent.
%%%
\index{sequent}%%
\index{antecedent}%%
\index{succedent}%%
%%%
The axioms are
%%
\begin{equation}
\mbox{\rm (ax)} \quad \varphi \bvdash \varphi
\end{equation}
%%
Then there are the following rules of introduction of connectives:
%%
\begin{equation}
\mbox{\rm (\textbf{I}{\mtt\symbol{25}})}\quad
\begin{array}{c}
    \Delta;\varphi\bvdash\chi\\\hline
    \Delta\bvdash\mbox{\mtt ($\varphi$\symbol{25}$\chi$)}
    \end{array}
\qquad
\mbox{\rm ({\mtt\symbol{25}}\textbf{I})}\quad
\begin{array}{c}
    \Delta\bvdash\varphi \qquad \Delta;\psi\bvdash\chi\\\hline
    \Delta;\mbox{\mtt ($\varphi$\symbol{25}$\psi$)}\bvdash
        \chi
    \end{array}
\end{equation}
%%
Notice that these rules introduce occurrences of the arrow. The
rule (\textbf{I}{\mtt\symbol{25}}) introduces an occurrence on the 
right hand side of $\bvdash$, while ({\mtt\symbol{25}}\textbf{I}) 
puts an occurrence on the left hand side. (The names of the rules 
are chosen accordingly.) Further, there are the following so--called 
rules of inference:
%%
\begin{equation}
\mbox{\rm (cut)}\quad
\begin{array}{c}
\Delta\bvdash\varphi \qquad \Theta;\varphi \bvdash \chi \\\hline
\Delta;\Theta \bvdash\chi
\end{array}
\qquad
\mbox{\rm (mon)}\quad
\begin{array}{c}
\Delta\bvdash\varphi \\\hline
\Delta;\Theta\bvdash\varphi
\end{array}
\end{equation}
%%%
\index{premiss}%%
\index{conclusion}%%
\index{formula!main}%%
\index{formula!cut--\faul}%%
\index{Gentzen calculus}%%
%%%
The sequents above the line are called the \textbf{premisses}, the sequent
below the lines the \textbf{conclusion} of the rule. Further, the
formulae that are introduced by the rules ({\mtt\symbol{25}}\textbf{I}) 
and (\textbf{I}{\mtt\symbol{25}}) are called \textbf{main formulae}, 
and the formula $\varphi$ in (cut) the \textbf{cut--formula}. Let 
us call this the \textbf{Gentzen calculus}. It is denoted by $\CH$.
%%%
\begin{defn}
%%%
\index{sequent proof}%%
%%%
Let $\Delta \bvdash \varphi$ be a sequent. A (\textbf{sequent}) 
\textbf{proof of length} $n$ of $\Delta \bvdash \varphi$ in $\CH$ is a
sequence $\Pi = \auf \Sigma_i \bvdash \chi_i : i < n+1\zu$ such
that (a) $\Sigma_n = \Delta$, $\chi_n = \varphi$, (b) for all $i <
n+1$ either (ba) $\Sigma_i \bvdash \chi_i$ is an axiom or (bb)
$\Sigma_i \bvdash \chi_i$ follows from some earlier sequents by
application of a rule of $\CH$.
\end{defn}
%%%
It remains to say what it means that a sequent follows from some
other sequents by application of a rule. This, however, is straightforward.
For example, $\Delta \bvdash \mbox{\mtt ($\varphi$\symbol{25}$\chi$%
)}$ follows from the earlier sequents by application of the rule
(\textbf{I}{\mtt\symbol{25}}) if among the earlier sequents we find 
the sequent $\Delta;\varphi \bvdash \chi$. We shall define also a different
notion of proof, which is based on trees rather than sequences.
In doing so, we shall also formulate a somewhat more abstract
notion of a calculus.
%%%
\begin{defn}
%%%
\index{rule}%%
\index{rule!finitary}%%%
\index{sequent calculus}%%
\index{proof tree}%%
%%%
A \textbf{finitary rule} is a pair $\rho = \auf M, \GS\zu$, where
$M$ is a finite set of sequents and $\GS$ a single sequent. (These
rules are written down using lower case Greek letters as schematic
variables for formulae and upper case Greek letters as schematic
variables for sets of formulae.) A \textbf{sequent calculus} $\CS$
is a set of finitary rules. An $\CS$--\textbf{proof tree} is a
triple $\BT = \auf T, \succ, \ell\zu$ such that $\auf T, \prec\zu$
is a tree and for all $x$: if $\{y_i : i < n\}$ are the daughters
of $T$, $\auf \{\ell(y_i) : i < n\}, \ell(x)\zu$ is an instance of
a rule of $\CS$. If $r$ is the root of $\BT$, we say that $\BT$
\textbf{proves} $\ell(r)$ \textbf{in} $\CS$. We write
%%%
\begin{equation}
\stackrel{\CS}{\rightsquigarrow} \Delta \bvdash \varphi
\end{equation}
%%%
to say that the sequent $\Delta\bvdash \varphi$ has a proof in $\CS$.
\end{defn}
%%%
We start with the only rule for $\sbot$, which actually is an axiom.
%%%
\begin{equation}
\mbox{\rm ($\sbot$\textbf{I})}\quad \sbot \bvdash \varphi
\end{equation}
%%
For negation we have these rules.
%%
\begin{equation}
\mbox{\rm ({\mtt\symbol{5}}\textbf{I})}\quad
\begin{array}{c}
\Delta \bvdash \varphi \\\hline
\Delta; \mbox{\mtt (\symbol{5}$\varphi$)} \bvdash \sbot
\end{array}
\qquad
\mbox{\rm (\textbf{I}{\mtt\symbol{5}})}\quad
\begin{array}{c}
\Delta; \varphi \bvdash \sbot \\\hline
\Delta \bvdash \mbox{\mtt (\symbol{5}$\varphi$)}
\end{array}
\end{equation}
%%
The following are the rules for conjunction.
%%
\begin{equation}
\mbox{\rm ({\mtt\symbol{4}}\textbf{I})}\quad
\begin{array}{c}
\Delta;\varphi;\psi \bvdash \chi \\\hline
\Delta;\mbox{\mtt ($\varphi$\symbol{4}$\psi$)} \bvdash \chi
\end{array}
\qquad
\mbox{\rm (\textbf{I}{\mtt\symbol{4}})}\quad
\begin{array}{c}
\Delta \bvdash \varphi \qquad \Delta \bvdash \psi\\\hline
\Delta \bvdash \mbox{\mtt ($\varphi$\symbol{4}$\psi$)}
\end{array}
\end{equation}
%%
Finally, these are the rules for {\mtt\symbol{31}}.
%%
\begin{equation}
\begin{array}{l}
\mbox{\rm ({\mtt\symbol{31}}\textbf{I})}\quad
\begin{array}{c}
\Delta; \varphi \bvdash \chi \qquad \Delta; \psi \bvdash \chi \\\hline
\Delta; \mbox{\mtt ($\varphi$\symbol{31}$\psi$)} \bvdash \chi
\end{array} \\
\mbox{\rm (\textbf{I}$_1${\mtt\symbol{31}})}\quad
\begin{array}{c}
\Delta \bvdash \varphi \\\hline
\Delta \bvdash \mbox{\mtt ($\varphi$\symbol{31}$\psi$)}
\end{array}
\qquad
\mbox{\rm (\textbf{I}$_2${\mtt\symbol{31}})}\quad
\begin{array}{c}
\Delta \bvdash \psi \\\hline
\Delta \bvdash \mbox{\mtt ($\varphi$\symbol{31}$\psi$)}
\end{array}
\end{array}
\end{equation}
%%
Let us return to the calculus $\CH$. We shall first of all show
that we can weaken the rule system without changing the set of
derivable sequents. Notice that the following is a proof tree.
%%%
\begin{equation}
\begin{array}{c}
\varphi \bvdash \varphi \qquad \psi \bvdash \psi \\\hline
\mbox{\mtt ($\varphi$\symbol{25}$\psi$)};\varphi \bvdash \psi \\\hline
\mbox{\mtt ($\varphi$\symbol{25}$\psi$)}
    \bvdash\mbox{\mtt ($\varphi$\symbol{25}$\psi$)}
\end{array}
\end{equation}
%%%
This shows us that in place of the rule (ax) we may actually use a
restricted rule, where we have only $\mbox{\mtt p}_i \bvdash %
\mbox{\mtt p}_i$.
%%%
\index{axiom!primitive}%%
%%%
Call such an instance of (ax) \textbf{primitive}. This fact may be
used for the following theorem.
%%%
\begin{lem}
$\stackrel{\CH}{\rightsquigarrow} \Delta \bvdash 
\mbox{\mtt ($\varphi$\symbol{25}$\chi$)}$ iff 
$\stackrel{\CH}{\rightsquigarrow} \Delta;\varphi \bvdash\chi$.
\end{lem}
%%%
\proofbeg
From right to left follows using the rule (\textbf{I}{\mtt\symbol{25}}).
Let us prove the other direction. We know that there exists a proof
tree for $\Delta \bvdash \mbox{\mtt ($\varphi$\symbol{25}$\chi$)}$
from primitive axioms. Now we trace backwards the occurrence of
{\mtt ($\varphi$\symbol{25}$\chi$)} in the tree from the root
upwards. Obviously, since the formula has not been introduced
by (ax), it must have been introduced by the rule 
(\textbf{I}{\mtt\symbol{25}}). Let $x$ be the node where the 
formula is introduced. Then we remove $x$ from the tree, thereby 
also removing that instance of (\textbf{I}{\mtt\symbol{25}}). 
Going down from $x$, we have to repair our proof as follows. 
Suppose that at $y < x$ we have an instance of
(mon). Then instead of the proof part to the left we use the one
to the right.
%%%
\begin{equation}
\begin{array}{c}
    \Sigma\bvdash\mbox{\mtt ($\varphi$\symbol{25}$\chi$)} \\\hline
    \Sigma;\Theta\bvdash\mbox{\mtt ($\varphi$\symbol{25}$\chi$)}
\end{array}
\quad
\begin{array}{c}
    \Sigma;\varphi\bvdash\chi \\\hline
    \Sigma;\Theta;\varphi\bvdash\chi
\end{array}
\end{equation}
%%%
Suppose that we have an instance of (cut). Then our specified 
occurrence of {\mtt ($\varphi$\symbol{25}$\chi$)} is the one 
that is on the right of the target sequent. So, in place of the 
proof part on the left we use the one on the right.
%%%
\begin{equation}
\begin{array}{c}
\Delta\bvdash\psi \quad \Theta;\psi \bvdash 
	\mbox{\mtt ($\varphi$\symbol{25}$\chi$)} \\\hline
\Delta;\Theta \bvdash\mbox{\mtt ($\varphi$\symbol{25}$\chi$)}
\end{array}
\qquad
\begin{array}{c}
\Delta\bvdash\psi \quad \Theta;\varphi;\psi \bvdash \chi\\\hline
\Delta;\Theta;\varphi \bvdash\chi
\end{array}
\end{equation}
%%%
Now suppose that we have an instance of ({\mtt\symbol{25}}\textbf{I}). 
Then this instance must be as shown to the left. We replace it by 
the one on the right.
%%%
\begin{equation}
\begin{array}{c}
    \Delta\bvdash\tau \quad \Delta;\psi\bvdash%
    \mbox{\mtt ($\varphi$\symbol{25}$\chi$)} \\\hline
    \Delta;\mbox{\mtt ($\tau$\symbol{25}$\psi$)}\bvdash
        \mbox{\mtt ($\varphi$\symbol{25}$\chi$)}
    \end{array}
\qquad
\begin{array}{c}
\Delta\bvdash\tau \quad \Delta;\phi;\psi\bvdash \chi\\\hline
\Delta;\mbox{\mtt ($\tau$\symbol{25}$\psi$)};\varphi \bvdash \chi
\end{array}
\end{equation}
%%%
The rule ({\mtt\symbol{25}}\textbf{I}) does not occur below $x$, as is 
easily seen. This concludes the replacement. It is verified that after 
performing these replacements, we obtain a proof tree for
$\Delta;\varphi\bvdash\chi$.
\proofend
%%%
\begin{thm}
$\Delta \vdash^{\mathsf{H}} \varphi$ iff
 $\stackrel{\CH}{\rightsquigarrow} \Delta \bvdash \varphi$.
\end{thm}
%%%
\proofbeg 
Suppose that $\Delta \vdash^{\mathsf{H}} \varphi$. By induction on the 
length of the proof we shall show that 
$\stackrel{\CH}{\rightsquigarrow} \Delta \bvdash \varphi$. Using DT we may
restrict ourselves to $\Delta = \varnothing$. First, we shall show
that (a0) and (a1) can be derived. (a0) is derived as follows.
%%
\begin{equation}
$$\begin{array}{r@{\;\bvdash\;}l}
\varphi      & \varphi \\\hline
\varphi;\psi & \varphi \\\hline
\varphi      & \mbox{\mtt ($\psi$\symbol{25}$\varphi$)} \\\hline
             &
    \mbox{\mtt ($\varphi$\symbol{25}($\psi$\symbol{25}$\varphi$))}
\end{array}
\end{equation}
%%
For (a1) we need a little more work.
%%
\begin{equation}
\begin{array}{cccc}
& & \psi \bvdash \psi & \chi \bvdash \chi \\\cline{3-4}
& \varphi \bvdash \varphi &
    \multicolumn{2}{c}{\psi;\mbox{\mtt ($\psi$\symbol{25}$\chi$)}
        \bvdash \chi} \\\cline{2-4}
\varphi \bvdash \varphi &
    \multicolumn{3}{c}{\varphi;\mbox{\mtt ($\varphi$\symbol{25}$\psi$)};
    \mbox{\mtt ($\psi$\symbol{25}$\chi$)} \bvdash \chi}
\\\cline{1-4}
\multicolumn{4}{c}{
    \mbox{\mtt ($\varphi$\symbol{25}($\psi$\symbol{25}$\chi$))};%
\mbox{\mtt ($\varphi$\symbol{25}$\psi$)};
    \varphi \bvdash \chi}
\end{array}
\end{equation}
%%%
If we apply (\textbf{I}{\mtt\symbol{25}}) three times we get (a1). Next 
we have to show that if we have $\stackrel{\CH}{\rightsquigarrow} \varnothing 
\bvdash \phi$ and $\stackrel{\CH}{\rightsquigarrow} \varnothing \bvdash 
\mbox{\mtt ($\chi$\symbol{25}$\phi$)}$ then $\stackrel{\CH}{\rightsquigarrow}
\varnothing \bvdash \chi$. By DT, we also have
$\stackrel{\CH}{\rightsquigarrow} \varphi \bvdash \chi$ and then a single
application of (cut) yields the desired conclusion. This proves
that $\stackrel{\CH}{\rightsquigarrow}\varnothing \bvdash \varphi$. Now,
conversely, we have to show that $\stackrel{\CH}{\rightsquigarrow}
\Delta\bvdash\varphi$ implies that $\Delta \vdash^{\mathsf{H}}
\varphi$. This is shown by induction on the height of the nodes in
the proof tree. If it is 1, we have an axiom: however,
$\varphi\vdash^{\mathsf{H}} \varphi$ clearly holds. Now suppose
the claim is true for all nodes of depth $< i$ and let $x$ be of
depth $i$. Then $x$ is the result of applying one of the four
rules. ({\mtt\symbol{25}}\textbf{I}). By induction hypothesis, $\Delta
\vdash^{\mathsf{H}} \varphi$ and $\Delta;\psi \vdash^{\mathsf{H}}
\chi$. We need to show that 
$\Delta;\mbox{\mtt ($\varphi$\symbol{25}$\psi$)} \vdash^{\mathsf{H}} 
\chi$. Simply let $\Pi_1$ be a proof of $\varphi$ from $\Delta$, $\Pi_2$ 
a proof of $\chi$ from $\Delta;\psi$. Then $\Pi_3$ is a proof of $\chi$
from $\Delta;\mbox{\mtt ($\varphi$\symbol{25}$\psi$)}$.
%%
\begin{equation}
\Pi_3 := \Pi_1^{\smallfrown}\auf\mbox{\mtt ($\varphi$\symbol{25}%
$\psi$)},\psi\zu^{\smallfrown}\Pi_2
\end{equation}
%%
(\textbf{I}{\mtt\symbol{25}}). This is straightforward from DT. (cut). 
Suppose that $\Pi_1$ is a proof of $\varphi$ from $\Delta$ and $\Pi_2$ a
proof of $\chi$ from $\Theta;\varphi$. Then
$\Pi_1^{\smallfrown}\Pi_2$ is a proof of $\chi$ from $\Delta;
\varphi$, as is easily seen. (mon). This follows from
Proposition~\ref{prop:deduce}. 
\proofend

Call a rule $\rho$ \textbf{admissible} 
%%%
\index{rule!admissible}%%
%%%
for a calculus $\CS$ if any
sequent $\Delta \bvdash \varphi$ that is derivable in $\CS + \rho$
is also derivable in $\CS$. Conversely, if $\rho$ is admissible
in $\CS$, we say that $\rho$ is \textbf{eliminable from} 
%%%
\index{rule!eliminable}%%
%%%
$\CS + \rho$. We shall show that (cut) is eliminable from $\CH$, so that
it can be omitted without losing derivable sequents. As
cut--elimination will play a big role in the sequel, the reader is
asked to watch the procedure carefully.
%%%
\begin{thm}[Cut Elimination]
\index{Cut Elimination}%%%
\label{thm:cutelimination}
\mbox{\rm (cut)} is eliminable from $\CH$.
\end{thm}
%%%
\proofbeg
Recall that (cut) is the following rule.
%%
\begin{equation}
\label{eq:dgcut}
\mbox{\rm (cut)}\quad
\begin{array}{c}
\Delta\bvdash\varphi \quad \Theta;\varphi \bvdash \chi \\\hline
\Delta;\Theta \bvdash\chi
\end{array}
\end{equation}
%%
\index{cut!degree of a \faul}%%
\index{cut!weight of a \faul}%%
\index{cut--weight}%%
%%%
Two measures are introduced. The \textbf{degree} of \eqref{eq:dgcut}
is
%%%
\begin{equation}
d := |\Delta| + |\Theta| + |\varphi| + |\chi|
\end{equation}
%%%
The \textbf{weight} of \eqref{eq:dgcut} is $2^d$.  The \textbf{cut--weight} 
of a proof tree $\BT$ is the sum over all weights of occurrences of cuts 
(= instances of (cut)) in it. Obviously,
the cut--weight of a proof tree is zero iff there are no
cuts in it. We shall now present a procedure that operates on proof
trees in such a way that it reduces the cut--weight of every given
tree if it is nonzero. This procedure is as follows. Let $\BT$ be 
given, and let $x$ be a node carrying the conclusion of an instance 
of (cut). We shall assume that above $x$ no instances of (cut) exist. 
(Obviously, $x$ exists if there are cuts in $\BT$.) $x$ has two mothers,
$y_1$ and $y_2$. Case (1). Suppose that $y_1$ is a leaf. Then
we have $\ell(y_1) = \varphi \bvdash \varphi$, $\ell(y_2) =
\Theta;\varphi \bvdash \chi$ and $\ell(x) = \Theta;\varphi\bvdash\chi$.
In this case, we may simply skip the application of cut by
dropping the nodes $x$ and $y_1$. This reduces the degree of the
cut by $2 \cdot |\varphi|$, since this application of (cut) has been
eliminated without trace. Case (2). Suppose that $y_2$ is a leaf.
Then $\ell(y_2) = \chi \bvdash \chi$, $\ell(y_1) = \Delta \bvdash
\varphi$, whence $\varphi  = \chi$ and $\ell(x) = \Delta \bvdash
\varphi = \ell(y_1)$. Eliminate $x$ and $y_2$. This reduces the
cut--weight by the weight of that cut. Case (3). Suppose that
$y_1$ has been obtained by application of (mon). Then the proof is
as shown on the left.
%%
\begin{equation}
\begin{array}{r@{\;\bvdash\;}rcc}
\Delta & \varphi & \quad & \\\cline{1-2}
\Delta;\Delta' & \varphi & \Theta;\varphi \bvdash \chi \\\hline
\multicolumn{3}{c}{\Delta;\Delta';\Theta \bvdash \chi}
\end{array}
\qquad
\qquad
\begin{array}{r@{\;\bvdash\;}l}
\multicolumn{2}{c}{\Delta \bvdash \varphi \qquad
    \Theta;\varphi \bvdash \chi } \\\hline
\Delta;\Theta & \chi \\\hline
\Delta;\Delta';\Theta & \chi
\end{array}
\end{equation}
%%%
We may assume that $\Delta' > 0$. We replace the local tree by the 
one on the right. The cut weight is reduced by 
%%%
\begin{equation}
2^{|\Delta| + |\Delta'| + |\Theta| + |\varphi| + |\chi|}  
- 2^{|\Delta| + |\Theta| + |\varphi| + |\chi|} > 0  
\end{equation}
%%%
Case (4). $\ell(y_2)$ has been obtained by application of (mon). This 
is similar to the previous case. Case (5). $\ell(y_1)$ has been obtained 
by ({\mtt\symbol{25}}\textbf{I}). Then the main formula is not the cut 
formula.
%%%
\begin{equation}
\begin{array}{ccc}
\Delta \bvdash \rho \quad \Delta;\tau \bvdash \varphi & \quad &
    \\\cline{1-1}
\Delta;\mbox{\mtt ($\rho$\symbol{25}$\tau$)} \bvdash \varphi & &
    \Theta;\varphi \bvdash \chi \\\hline
\multicolumn{3}{c}{\Delta;\Theta;\mbox{\mtt ($\rho$\symbol{25}$\tau$)}
    \bvdash \chi}
\end{array}
\end{equation}
%%
And the cut can be rearranged as follows.
%%
\begin{equation}
\begin{array}{r@{\;\bvdash\;}l@{\quad}ccc}
\Delta & \rho & & \Delta;\tau \bvdash \varphi &
    \Theta;\varphi \bvdash \chi \\\cline{1-2}\cline{4-5}
\Delta;\Theta & \rho & &
    \multicolumn{2}{c}{\Delta;\Theta;\tau \bvdash \chi}
\\\hline
\multicolumn{5}{c}{\Delta;\Theta;\mbox{\mtt ($\rho$\symbol{25}$\tau$)}
    \bvdash \chi}
\end{array}
\end{equation}
%%
Here, the degree of the cut is reduced by $|\mbox{\mtt ($\rho$\symbol{25}%
$\tau$)}| - |\tau| > 0$. Thus the cut--weight is reduced as well. 
Case (6). $\ell(y_2)$ has been obtained by ({\mtt\symbol{25}}\textbf{I}).
Assume $\varphi \neq \mbox{\mtt ($\rho$\symbol{25}$\tau$)}$.
%%%
\begin{equation}
\begin{array}{r@{\;\bvdash\;}l@{\quad}ccc}
\multicolumn{2}{c}{} & & \Theta;\varphi \bvdash \rho &
    \Theta;\varphi;\tau \bvdash \chi \\\cline{4-5}
\Delta & \varphi & &
    \multicolumn{2}{c}{\Theta;\varphi;\mbox{\mtt ($\rho$\symbol{25}$\tau$)} 
	\bvdash \chi}
\\\hline
\multicolumn{5}{c}{\Delta;\Theta;\mbox{\mtt ($\rho$\symbol{25}$\tau$)}
    \bvdash \chi}
\end{array}
\end{equation}
%%%
In this case we can replace the one cut by two as follows. 
%%%
\begin{equation}
\begin{array}{c}
\Delta\;\bvdash\; \varphi \qquad \Theta;\varphi\;\bvdash\; \rho 
\\\hline
\Delta;\Theta\;\bvdash\;\rho
\end{array}
\qquad
\begin{array}{c}
\Delta\;\bvdash\; \varphi \qquad \Theta;\varphi;\tau\;\bvdash\; \chi
\\\hline
\Delta;\Theta;\tau\;\bvdash\;\chi
\end{array}
\end{equation}
%%%
If we now apply ({\mtt\symbol{25}}\textbf{I}), we get the same sequent. 
The cut--weight has been diminished by
%%%%
\begin{equation}
2^{|\Delta| + |\Theta| + |\rho| + |\tau| + 3} - 
2^{|\Delta| + |\Theta| + |\rho|} - 
2^{|\Delta| + |\Theta| + |\tau|} > 0
\end{equation}
%%%
(See also below for the same argument.) Suppose however 
$\varphi = \mbox{\mtt ($\rho$\symbol{25}$\tau$)} \not\in \Theta$. 
Then either $\varphi$ is not the main formula of $\ell(y_1)$, 
in Case (1), (3), (5), or it actually is the main formula, and 
then we are in Case (7), to which we now turn.
%%%
Case (7). $\ell(y_1)$ has been introduced by (\textbf{I}{\mtt\symbol{25}}). 
If the cut formula is not the main formula, we are in cases 
(2), (4), (6) or (8), which we dealt with separately. Suppose 
however the main formula is the cut formula. Here, we cannot 
simply permute the cut unless $\ell(y_2)$ is the result of applying 
({\mtt\symbol{25}}\textbf{I}). In this case we proceed as follows. 
$\varphi =\mbox{\mtt ($\rho$\symbol{25}$\tau$)}$ for some $\rho$ 
and $\tau$. The local proof is as follows.
%%%
\begin{equation}
\begin{array}{r@{\;\bvdash\;}lccc}
\Delta;\rho & \tau & &
    \Theta \bvdash \rho & \Theta;\tau\bvdash\chi
\\\cline{1-2}\cline{4-5}
\Delta & \mbox{\mtt ($\rho$\symbol{25}$\tau$)}
    & &
\multicolumn{2}{c}{\Theta;\mbox{\mtt ($\rho$\symbol{25}$\tau$)}
    \bvdash \chi} \\\hline
\multicolumn{5}{c}{\Delta;\Theta \bvdash \chi}
\end{array}
\end{equation}
%%
This is rearranged in the following way.
%%
\begin{equation}
\begin{array}{ccr@{\;\bvdash\;}l}
\Delta;\rho \bvdash \tau
\qquad
\Theta \bvdash  \rho & & \Theta;\tau & \chi \\\cline{1-1}\cline{3-4}
\Delta;\Theta \bvdash \tau & & \Delta;\Theta;\tau  & \chi \\\hline
\multicolumn{4}{c}{
\Delta;\Theta \bvdash \chi}
\end{array}
\end{equation}
%%%
This operation eliminates the cut in favour of two cuts. The overall
degree of these cuts may be increased, but the weight has been
decreased. Let $d := |\Delta;\Theta|$, $p := |\mbox{\mtt ($\rho$%
\symbol{25}$\tau$)}|$. Then the first cut has weight $2^{d + p + |\chi|}$.
The two other cuts have weight
%%
\begin{equation}
2^{d + |\rho| + |\tau|} + 2^{d + |\tau| + |\chi|} \leq
2^{d + |\rho| + |\tau| + |\chi|} <
2^{d + p + |\chi|}
\end{equation}
%%
since $p > |\rho| + |\tau| > 0$. (Notice
that $2^{a+c} + 2^{a+d} = 2^a \cdot (2^c + 2^d) \leq 2^a \cdot
2^{c+d} = 2^{a+c+d}$ if $c, d > 0$.) 
Case (8). $\ell(y_2)$ has been obtained by (\textbf{I}{\mtt\symbol{25}}). 
Then $\chi = \mbox{\mtt ($\rho$\symbol{25}$\tau$)}$ for some $\rho$ 
and $\tau$. We replace the left hand proof part by the right hand part, 
and the degree is reduced by $|\mbox{\mtt ($\rho$\symbol{25}$\tau$)}| 
- |\tau| > 0$.
%%
\begin{equation}
\begin{array}{cr@{\;\bvdash\;}l}
    & \Theta;\varphi;\rho & \tau \\\cline{2-3}
\Delta \bvdash \varphi & \Theta;\varphi & 
	\mbox{\mtt ($\rho$\symbol{25}$\tau$)} \\\hline
\multicolumn{3}{c}{\Delta;\Theta \bvdash 
	\mbox{\mtt ($\rho$\symbol{25}$\tau$)}}
\end{array}
\qquad
\begin{array}{c}
\Delta \bvdash \varphi \qquad \Theta;\rho;\varphi \bvdash \tau \\\hline
\begin{array}{r@{\;\bvdash \;}l}
\Delta;\Theta;\rho & \tau \\\hline
\Delta;\Theta & \mbox{\mtt ($\rho$\symbol{25}$\tau$)}
\end{array}
\end{array}
\end{equation}
%%
So, in each case we managed to decrease the cut--weight.
This concludes the proof.
%%%
\proofend

Before we conclude this section we shall mention another deductive
%%%
\index{natural deduction}%%
%%%
calculus, called \textbf{Natural Deduction}. It uses proof trees, but
is based on the Deduction Theorem. First of all notice that we can
write Hilbert style proofs also in tree format. Then the leaves of
the proof tree are axioms, or assumptions, and the only rule we
%%%
\index{Modus Ponens (MP)}%%
%%%
are allowed to use is \textbf{Modus Ponens}.
%%
\begin{equation}
\mbox{\rm (MP)}\quad\begin{array}{c}
    \mbox{\mtt ($\varphi$\symbol{25}$\psi$)}\qquad
    \varphi \\\hline
    \psi
    \end{array}
\end{equation}
%%
This, however, is a mere reformulation of the previous calculus. The
idea behind natural deduction is that we view Modus Ponens as a rule
to {\it eliminate\/} the arrow, while we add another rule that allows
to introduce it. It is as follows.
%%
\begin{equation}
\mbox{\rm (\textbf{I}{\mtt\symbol{25}})}\quad
    \begin{array}{c}\psi \\\hline
    \mbox{\mtt ($\varphi$\symbol{25}$\psi$)}
    \end{array}
\end{equation}
%%
However, when this rule is used, the formula $\varphi$ may be
eliminated from the assumptions. Let us see how this goes. Let
$x$ be a node. Let us call the set $A(x) := \{\auf y, \ell(y)\zu
%%%
\index{assumption}%%
%%%
: y > x, y \mbox{ leaf}\}$ the set of \textbf{assumptions of} $x$.
If (\textbf{I}{\mtt\symbol{25}}) is used to introduce {\mtt ($\varphi$%
\symbol{25}$\psi$)}, any number of assumptions of $x$ that have the 
form $\auf y, \varphi\zu$ may be retracted. In order to know what
assumption has been effectively retracted, we check mark the
retracted assumptions by a superscript (e.~g.~$\varphi^{\surd}$).
%%%
\index{$\varphi^{\surd}$, $[\varphi]$}%%
%%%
Here are the standard rules for the other connectives. The fact
that the assumption $\varphi$ is or may be removed is annotated as
follows:
%%
\begin{equation}
\mbox{\rm (\textbf{I}{\mtt\symbol{25}})}\quad
    \begin{array}{c}[\varphi] \\
    \vdots \\
    \psi \\\hline
    \mbox{\mtt ($\varphi$\symbol{25}$\psi$)}
    \end{array}
\qquad
\mbox{\rm (\textbf{E}{\mtt\symbol{25}})}\quad
    \begin{array}{c} 
    \mbox{\mtt ($\varphi$\symbol{25}$\psi$)} \quad 
	\varphi \\\hline
	\psi 
    \end{array}
\end{equation}
%%
Here, $[\varphi]$ means that any number of assumptions of the form
$\varphi$ above the node carrying $\varphi$ may be check marked
when using the rule. (So, it does {\it not\/} mean that it
requires these formulae to be assumptions.) The rule 
(\textbf{E}{\mtt\symbol{25}}) is nothing but (MP). First, conjunction.
%%%
\begin{equation}
\mbox{\rm (\textbf{I}{\mtt\symbol{4}})}\quad\begin{array}{c}
\varphi \quad\psi \\\hline
\mbox{\mtt ($\varphi$\symbol{4}$\psi$)}
\end{array}
\qquad
\mbox{\rm (\textbf{E}$_1${\mtt\symbol{4}})}\quad
\begin{array}{c}
\mbox{\mtt ($\varphi$\symbol{4}$\psi$)} \\\hline
\varphi
\end{array}\qquad
\mbox{\rm (\textbf{E}$_2${\mtt\symbol{4}})}\quad
\begin{array}{c}
\mbox{\mtt ($\varphi$\symbol{4}$\psi$)} \\\hline
\psi
\end{array}
\end{equation}
%%%
The next is $\sbot$:
%%
\begin{equation}
\mbox{\rm (\textbf{E}$\sbot$)}\quad\begin{array}{c}
    \sbot \\\hline
    \varphi
    \end{array}
\end{equation}
%%%
For negation we need some administration of the check mark.
%%
\begin{equation}
\mbox{\rm (\textbf{I}{\mtt\symbol{5}})}\quad
\begin{array}{c}
[\varphi] \\
\vdots \\
\sbot \\\hline
\mbox{\mtt (\symbol{5}$\varphi$)}
\end{array}
\qquad
\mbox{\rm (\textbf{E}{\mtt\symbol{5}})}\quad
\begin{array}{c}
\varphi \quad\mbox{\mtt (\symbol{5}$\varphi$)} \\\hline
\sbot
\end{array}
\end{equation}
%%
So, using the rule (\textbf{I}{\mtt\symbol{5}}) any number of assumptions 
of the form $\varphi$ may be check marked.
%%
Disjunction is even more complex.
%%%
\begin{equation}
\begin{array}{l}
\mbox{\rm (\textbf{I}$_1${\mtt\symbol{31}})}\quad
\begin{array}{c}
\varphi \\\hline
\mbox{\mtt ($\varphi$\symbol{31}$\psi$)}
\end{array}\qquad
\mbox{\rm (\textbf{I}$_2${\mtt\symbol{31}})}\quad
\begin{array}{c}
\psi \\\hline
\mbox{\mtt ($\varphi$\symbol{31}$\psi$)}
\end{array} \\
\\
\mbox{\rm (\textbf{E}{\mtt\symbol{31}})}\quad
\begin{array}{ccc}
    & [\varphi] & [\psi] \\
    & \vdots & \vdots \\
\mbox{\mtt ($\varphi$\symbol{31}$\psi$)} & \chi  & \chi\\\hline
\multicolumn{3}{c}{\chi}
\end{array}
\end{array}
\end{equation}
%%%
In the last rule, we have three assumptions. As we have indicated,
whenever it is used, we may check mark any number of assumptions
of the form $\varphi$ in the second subtree and any number of
assumptions of the form $\psi$ in the third.

We shall give a characterization of natural deduction trees.
%%%
\index{rule}%%
\index{rule!finitary}%%%
%%%
A \textbf{finitary rule} is a pair $\rho = \auf \{\chi_i[A_i] : i < n\},
\varphi\zu$, where for $i < n$, $\chi_i$ is a formula, $A_i$ a finite
set of formulae and $\varphi$ a single formula. A \textbf{natural
deduction calculus} 
%%%
\index{natural deduction calculus}%%%
%%%%
$\GN$ is a set of finitary rules. A \textbf{proof tree for} 
%%%
\index{proof tree}%%
%%%
$\GN$ is a quadruple $\BT = \auf T, \succ, \ell, \CC\zu$ such that
$\auf T, \prec\zu$ is a tree, $\CC \subseteq T$
a set of leaves and $\BT$ is derived in the following way.
(Think of $\CC$ as the set of leaves carrying discharged
assumptions.)
%%%
\begin{dinglist}{43}
\item
$\BT = \auf \{x\}, \varnothing, \ell, \varnothing\zu$,
where $\ell \colon x \mapsto \varphi$.
\item
There is a rule $\auf \{\chi_i[A_i] : i < n\}, \gamma\zu$, and
$\BT$ is formed from trees $\BS_i$, $i < n$, with roots
$s_i$, by adding a new root node $r$, such that
$\ell_{\BS_i}(y_i) = \chi_i$, $i < n$, $\ell_{\BT}(x) =
\gamma$. Further, $\CC_{\BT} = \bigcup_{i < n} \CC_{\BS_i} \cup
\bigcup_{i < n} N_i$, where $N_i$ is a set of leaves
of $\BS_i$ such that for all $i < n$ and all $x \in N_i$:
$\ell_{\BS_i}(x) \in A_i$.
\end{dinglist}
%%
(Notice that the second case includes $n = 0$, in which case
$\BT = \auf \{x\}, \varnothing, \ell, \{x\}\zu$ where
$\ell(x)$ is simply an axiom.) We say that $\BT$ \textbf{proves}
$\ell(r)$ \textbf{in} $\GN$ from $\{\ell(x) : x \mbox{ leaf}, x
\not\in \CC\}$. Here now is a proof tree ending in (a0).
%%%
%%
\begin{equation}
\begin{array}{c}
\varphi^{\surd} \\\hline
\mbox{\mtt ($\psi$\symbol{25}$\varphi$)} \\\hline
\mbox{\mtt ($\varphi$\symbol{25}($\psi$\symbol{25}$\varphi$))}
\end{array}
\end{equation}
%%
Further, here is a proof tree ending in (a1).
%%
\begin{equation}
\begin{array}{ccc}
\mbox{\mtt ($\varphi$\symbol{25}($\psi$\symbol{25}$\chi$))}^{\surd}
    \qquad \varphi^{\surd} &
    & \mbox{\mtt ($\phi$\symbol{25}$\psi$)}^{\surd} \qquad \phi^{\surd}
    \\\cline{1-1}\cline{3-3}
\mbox{\mtt ($\psi$\symbol{25}$\chi$)} & & \psi \\\cline{1-3}
\multicolumn{3}{c}{\chi} \\\hline
\multicolumn{3}{c}{\mbox{\mtt ($\varphi$\symbol{25}$\chi$)}} \\\hline
\multicolumn{3}{c}{\mbox{\mtt (($\varphi$\symbol{25}$\psi$)\symbol{25}%
($\varphi$\symbol{25}$\chi$))}} \\\hline
\multicolumn{3}{c}{\mbox{\mtt (($\varphi$\symbol{25}($\psi$\symbol{25}%
$\chi$))\symbol{25}(($\varphi$\symbol{25}$\psi$)\symbol{25}(%
$\varphi$\symbol{25}$\chi$)))}}
\end{array}
\end{equation}
%%
A formula depends on all its assumptions that have not been
retracted in the following sense.
%%%
\begin{lem}
Let $\BT$ be a natural deduction tree with root $x$. Let $\Delta$
be the set of all formulae $\psi$ such that $\auf y, \psi\zu$ is an
unretracted assumption of $x$ and let $\varphi := \ell(x)$. Then
$\Delta \vdash^{\mathsf{H}} \varphi$.
\end{lem}
%%%
\proofbeg
By induction on the derivation of the proof tree.
\proofend

The converse also holds. If $\Delta \vdash^{\mathsf{H}} \varphi$
then there is a natural deduction proof for $\varphi$ with
$\Delta$ the set of unretracted assumptions (this is 
Exercise~\ref{ex:unretracted}).

{\it Notes on this section.} Proofs are graphs whose labels are
sequents. The procedure that eliminates cuts can be described
using a graph grammar. Unfortunately, the replacements also
manipulate the labels (that is, the sequents), so either one
uses infinitely many rules or one uses schematic rules.
%%%
\vplatz%%
\exercise%%
Show (a) $\mbox{\mtt ($\varphi$\symbol{25}($\psi$\symbol{25}$\chi$))} 
\vdash^{\mathsf{B}} \mbox{\mtt ($\psi$\symbol{25}($\varphi$%
\symbol{25}$\chi$))}$ and (b) $\mbox{\mtt ($\varphi$\symbol{4}$\psi$)}
\vdash^{\mathsf{H'}} \mbox{\mtt ($\psi$\symbol{4}$\varphi$)}$, 
where $\mathsf{H'}$ is \textsf{H} with the axioms for {\mtt\symbol{4}} 
added.
%%%
\vplatz 
\exercise 
Show that a set $\Sigma$ is inconsistent iff for every $\varphi$:  
$\Sigma \vdash^{\mathsf{B}} \varphi$.
%%%
\vplatz%%
\exercise%%
Show that a Hilbert style calculus satisfies DT for {\mtt\symbol{25}}
iff the formulae (a0) and (a1) are derivable in it.
(So, if we add, for example, the connectives {\mtt\symbol{5}}, 
{\mtt\symbol{4}} and {\mtt\symbol{31}} together with the corresponding 
axioms, DT remains valid.)
%%%
\vplatz%%
\exercise%%
Define $\varphi \approx \psi$ by $\varphi \vdash^{\mathsf{H}}
\psi$ and $\psi \vdash^{\mathsf{H}} \varphi$. Show that if
$\varphi \approx \psi$ then (a) for all $\Delta$ and $\chi$:
$\Delta; \varphi \vdash^{\mathsf{H}} \chi$ iff
$\Delta;\psi \vdash^{\mathsf{H}} \chi$, and (b) for all $\Delta$:
$\Delta \vdash^{\mathsf{H}} \varphi$ iff $\Delta
\vdash^{\mathsf{H}} \psi$.
%%%
\vplatz%%
\exercise%%
Let us call \textsf{Int} 
%%%
\index{\textsf{Int}}%%%
%%%%
the Hilbert calculus for {\mtt\symbol{25}}, 
$\sbot$, {\mtt\symbol{5}}, {\mtt\symbol{31}} and {\mtt\symbol{4}}. 
Further, call the Gentzen calculus for these connectives $\CI$. Show 
that $\Delta \vdash^{\mathsf{Int}} \varphi$ iff 
$\stackrel{\GI}{\rightsquigarrow} \Delta \bvdash
\varphi$.
%%
\vplatz%%
\exercise%%
\label{ex:unretracted}
Show the following claim: {\it If $\Delta \vdash^{\mathsf{H}}
\varphi$ then there is a natural deduction proof for $\varphi$
with $\Delta$ the set of unretracted assumptions}.
%%%
\vplatz
\exercise
Show that the rule of {\it Modus Tollens} is admissible in the natural
deduction calculus defined above (with added negation).
%%%
\begin{equation}
\mbox{\rm Modus Tollens:}\quad
\begin{array}{c}
\mbox{\mtt ($\varphi$\symbol{25}$\psi$)} \qquad 
	\mbox{\mtt (\symbol{5}$\psi$)} \\\hline
\mbox{\mtt (\symbol{5}$\varphi$)}
\end{array}
\end{equation}

 \section{Basics of $\lambda$--Calculus and Combinatory Logic}
\label{kap3-7}
%
%
%
%%%
%\index{$\lambda$--Kalk\"ul}%%
\nocite{hindley:combinatory}%%
\nocite{barendregt:lambda}%%
%%%
There is a fundamental difference between a term and a function.
The {\it term\/} $x^2 + 2xy$ is something that has a concrete
value if $x$ and $y$ have a concrete value. For example, if $x$
has value 5 and $y$ has value $2$ then $x^2 + 2xy = 25 + 20 = 45$.
However, the function $f \colon \BZ \times \BZ \pf \BZ \colon \auf x, y\zu
\mapsto x^2 + 2xy$ does not need any values for $x$ and $y$. It
only needs a pair of {\it numbers\/} to yield a value. That we
have used variables to define $f$ is of no concern here. We would
have obtained the same function had we written $f \colon \auf x, u\zu
\mapsto x^2 + 2xu$. However, the term $x^2 + 2xu$ is different
from the term $x^2 + 2xy$. For if $u$ has value 3, $x$ has
value 5 and $y$ value $2$, then $x^2 + 2xu = 25 + 30 = 55$, while
$x^2 + 2xy = 45$. To accommodate this difference, the
$\lambda$--calculus has been developed. The $\lambda$--calculus
allows to define functions from terms. In the case above we may
write $f$ as
%%
\begin{equation}
f := \lambda xy.x^2 + 2xy
\end{equation}
%%
This expression defines a function $f$ and by saying what it does to 
its arguments. The prefix `$\lambda xy$' means that we are dealing
with a function from pairs $\auf m, n\zu$ and that the function
assigns this pair the value $m^2 + 2mn$. This is the same as
what we have expressed with $\auf x, y\zu \mapsto x^2 + 2xy$. Now
we can also define the following functions.
%%
\begin{equation}
\lambda x.\lambda y.x^2 + 2xy, \qquad \lambda y.\lambda x.
x^2 + 2xy
\end{equation}
%%
The first is a function which assigns to every number $m$ the
function $\lambda y.m^2 + 2my$; the latter yields the value
$m^2 + 2mn$ for every $n$. The second is a function which gives
for every $m$ the function $\lambda x.x^2 + 2xm$; this in turn
yields $n^2 + 2nm$ for every $n$. Since in general $m^2 + 2mn \neq
n^2 + 2nm$, these two functions are different.

In $\lambda$--calculus one usually does not make use of the
simultaneous abstraction of several variables, so one only allows
prefixes of the form `$\lambda x$', not those of the form
`$\lambda xy$'. This we shall also do here. We shall give a
general definition of $\lambda$--terms. Anyhow, by introducing
pairing and projection (see Section~\ref{kap3}.\ref{kap:lambek}) simultaneous 
abstraction can be defined. The alphabet consists of a set 
$F$ of function symbols (for which a signature $\Omega$ needs 
to be given as well), {\tt\stlambda}, the variables 
$V := \{\mbox{\tt x}_i : i \in \omega\}$ the brackets 
{\tt (}, {\tt )} and the period `{\tt .}'.
%%
\begin{defn}
%%%
\index{$\lambda\Omega$--term}%%
%%%
The set of \textbf{$\lambda$--terms over the signature} $\Omega$,
the set of \textbf{$\lambda\Omega$--terms} for short, is the
smallest set $\Tm_{\lambda\Omega}(V)$ for which the
following holds:
%%
\begin{dingautolist}{192}
\item Every $\Omega$--term is in $\Tm_{\lambda\Omega}(V)$.
\item If $M, N \in \Tm_{\lambda\Omega}(V)$ then also
    $\mbox{\tt ($MN$)} \in \Tm_{\lambda\Omega}(V)$.
\item If $M \in \Tm_{\lambda\Omega}$ and $x$ is a
    variable then $\mbox{\tt (\stlambda$x$.$M$)}
    \in \Tm_{\lambda\Omega}(V)$.
\end{dingautolist}
%%
If the signature is empty or clear from the context we shall
simply speak of \textbf{$\lambda$--terms}.
\index{$\lambda$--term}%%
\end{defn}
%%
Since in \ding{193} we do not write an operator symbol, Polish Notation 
is now ambiguous. Therefore we follow standard usage and use the brackets 
{\tt (} and {\tt )}. We agree now that $x$, $y$ and $z$ and so on 
are metavariables for variables (that is, for elements of $V$). 
Furthermore, upper case Roman letters like $M$, $N$ are metavariables 
for $\lambda$--terms. One usually takes $F$ to be $\varnothing$, to
concentrate on the essentials of functional abstraction.
%%%
\index{$\lambda$--term!pure}%%%
%%%
If $F = \varnothing$, we speak of \textbf{pure $\lambda$--terms}.
It is customary to omit the brackets if the term is bracketed
to the left. Hence $MNOP$ is short for {\tt ((($MN$)$O$)$P$)} 
and {\tt \stlambda$x$.$MN$} short for {\tt ((\stlambda$x$.($MN$))}
(and distinct from {\tt ((\stlambda$x$.$M$)$N$)}). However, this 
abbreviation has to be used with care since the brackets are symbols 
of the language. Hence {\mtt x$_{\snull}$x$_{\snull}$x$_{\snull}$} 
is not a string of the language but only a shorthand for 
{\mtt ((x$_{\snull}$x$_{\snull}$)x$_{\snull}$)}, a difference that 
we shall ignore after a while. Likewise, outer brackets are often 
omitted and brackets are not stacked when several $\lambda$--prefixes 
appear. Notice that {\tt (x$_{\snull}$x$_{\snull}$)} is a term.  
It denotes the application of {\tt x$_{\snull}$} to itself. 
We have defined occurrences of a string $\vec{x}$ in a
string $\vec{y}$ as contexts $\auf \vec{u}, \vec{v}\zu$ where 
$\vec{u}\,\vec{x}\,\vec{v} = \vec{y}$. $\Omega$--terms are thought 
to be written down in Polish Notation.
%%%
\begin{defn}
%%%
\index{variable!occurrence}%%%
%%%
Let $x$ be a variable. We define the set of \textbf{occurrences of} 
$x$ in a $\lambda\Omega$--term inductively as follows.
%%
\begin{dingautolist}{192}
\item If $M$ is an $\Omega$--term then the set of occurrences
    of $x$ in the $\lambda\Omega$--term $M$ is the set of
    occurrences of the variable $x$ in the $\Omega$--term $M$.
\item The set of occurrences of $x$ in {\tt ($MN$)}
    is the union of the set of pairs $\auf \mbox{\tt (}\vec{u},
    \vec{v}N\mbox{\tt )}\zu$, where $\auf \vec{u}, \vec{v}\zu$ is
    an occurrence of $x$ in $M$ and the set of pairs
    $\auf \mbox{\tt (}M\vec{u},\vec{v}\mbox{\tt)}\zu$,
    where $\auf \vec{u}, \vec{v}\zu$ is an occurrence of
    $x$ in $N$.
\item The set of occurrences of $x$ in {\tt (\stlambda$x$.$M$)}
    is the set of all $\auf \mbox{\tt (\stlambda$x$.$\vec{u}$},
    \vec{v}\mbox{\tt )}\zu$, where $\auf \vec{u}, \vec{v}\zu$
    is an occurrence of $x$ in $M$.
\end{dingautolist}
\end{defn}
%%
So notice that --- technically speaking --- the occurrence of the
string $x$ in the $\lambda$--prefix of {\tt (\stlambda$x$.$M$)} 
is not an occurrence of the variable $x$. Hence {\tt x$_{\snull}$} does 
not occur in {\tt (\stlambda$x_{\snull}$.x$_{\seins}$)} as a 
$\lambda\Omega$--term although it does occur in it as a string!
%%
\begin{defn}
%%%
\index{variable!free}%%%
\index{variable!bound}%%
\index{$\lambda$--term!closed}%%
%%%
Let $M$ be a $\lambda$--term, $x$ a variable and $C$ an occurrence
of $x$ in $M$. $C$ is a \textbf{free occurrence of} $x$ \textbf{in}
$M$ if $C$ is not inside a term of the form {\tt (\stlambda$x$.$N$)} 
for some $N$; if $C$ is not free, it is called \textbf{bound}. A 
$\lambda$--term is called \textbf{closed} if no variable occurs free 
in it. The set of all variables having a free occurrence in $M$ 
is denoted by $\fr(M)$.  
\end{defn}
%%
A few examples shall illustrate this. In $M = \mbox{\tt ({\stlambda}x%
$_{\snull}$.(x$_{\snull}$x$_{\seins}$))}$ the variable {\tt x$_{\snull}$} 
occurs only bound, since it only occurs inside a subterm of the form 
{\tt ({\stlambda}x$_{\snull}$.$N$)} (for example $N := 
\mbox{\tt (x$_{\snull}$x$_{\seins}$)}$). However, {\tt x$_{\seins}$} 
occurs free. A variable may occur free as well as bound in a term. An 
example is the variable {\tt x$_{\snull}$} in 
{\tt (x$_{\snull}$({\stlambda}x$_{\snull}$.x$_{\snull}$))}.

Bound and free variable occurrences behave differently under
replacement.  If $M$ is a $\lambda$--term and $x$ a variable
then denote by $[N/x]M$ the result of replacing $x$ by $N$.
In this replacement we do not simply replace all occurrences
of $x$ by $N$; the definition of replacement requires some care.
%%
\begin{subequations}
\label{eq:sub}
\begin{align}
\label{eq:suba}
 [N/x]y & := \begin{cases}
	N & \text{if $x = y$,} \\
	y & \text{otherwise.}
	\end{cases} \\
\label{eq:subb}
[N/x]f(\vec{s}) & := 
	f([N/x]s_0, \dotsc, [N/x]s_{\Omega(f)-1}) \\
\label{eq:subc}
 [N/x]\mbox{\tt (}MM'\mbox{\tt )} &
    := \mbox{\tt (}([N/x]M)([N/x]M')\mbox{\tt )} \\
\label{eq:subd}
 [N/x]\mbox{\tt (\stlambda} x\mbox{\tt .}M\mbox{\tt )} &
    := \mbox{\tt (\stlambda} x\mbox{\tt .}M\mbox{\tt )} \\
\label{eq:sube}
 [N/x]\mbox{\tt (\stlambda} y\mbox{\tt .}M\mbox{\tt )} &
    := \mbox{\tt (\stlambda} y\mbox{\tt .}[N/x]M\mbox{\tt )} \\\notag
  & \qquad 
\text{if }y \neq x \text{ and: } y \not\in \fr(N) \text{ or }x \not\in \fr(M) \\
\label{eq:subf}
 [N/x]\mbox{\tt (\stlambda} y\mbox{\tt .}M\mbox{\tt )} &
    := \mbox{\tt (\stlambda} z\mbox{\tt .}[N/x][z/y]M\mbox{\tt )} \\\notag
    & \qquad \text{if }y \neq x, y \in \fr(N)
    \text{ and }x \in \fr(M) 
\end{align}
\end{subequations}
%%
In \eqref{eq:subf} we have to choose $z$ in such a way that it does 
not occur freely in $N$ or $M$. In order for substitution to be uniquely
defined we assume that $z = \mbox{\tt x}_i$, where $i$ is the least 
number such that $z$ satisfies the conditions. 
The precaution in \eqref{eq:subf} of an additional
substitution is necessary. For let $y = \mbox{\tt x}_{\seins}$
and $M = \mbox{\tt x}_{\snull}$. Then without this substitution we would 
get
%%
\begin{equation}
[\mbox{\tt x}_{\seins}/\mbox{\tt x}_{\snull}]\mbox{\tt (\stlambda x}_{\snull}%
\mbox{\tt .x}_{\seins}\mbox{\tt )} = \mbox{\tt (\stlambda x}_{\seins}%
\mbox{\tt .}[\mbox{\tt x}_{\seins}/\mbox{\tt x}_{\snull}]%
\mbox{\tt x}_{\snull}\mbox{\tt )} =
\mbox{\tt (\stlambda x}_{\seins}\mbox{\tt .x}_{\seins}\mbox{\tt )}
\end{equation}
%%
This is clearly incorrect. For
{\tt (\stlambda x$_{\seins}$.x$_{\snull}$)} is the function which for
given $a$ returns the value of $\mbox{\tt x}_{\snull}$.
However, {\tt (\stlambda x$_{\seins}$.x$_{\seins}$)} is the identity
function and so it is different from that function.
Now the substitution of a variable by another variable
shall not change the course of values of a function.
%%
\begin{subequations}
\begin{align}
%\begin{table}
%\caption{Axioms and Rules of the $\lambda$--Calculus}
%\label{tab:lambdaC}
%\begin{array}{llrl}
\label{eq:lea} 
& M \boldsymbol{=} M & \\
\label{eq:leb} 
& M \boldsymbol{=} N \Pf N \boldsymbol{=} M & \\
\label{eq:lec} 
& M \boldsymbol{=} N, N \boldsymbol{=} L \Pf M \boldsymbol{=} L &  \\
\label{eq:led} 
& M \boldsymbol{=} N \Pf \mbox{\tt (}ML\mbox{\tt )} \boldsymbol{=} 
\mbox{\tt (}NL\mbox{\tt )} & \\
\label{eq:lee} 
& M \boldsymbol{=} N \Pf \mbox{\tt (}LM\mbox{\tt )} \boldsymbol{=} 
\mbox{\tt (}LN\mbox{\tt )} & \\
\label{eq:lef} 
& \mbox{\tt (\stlambda}x.M\mbox{\tt )}\boldsymbol{=} 
\mbox{\tt (\stlambda} y.[y/x]M\mbox{\tt )} \quad  
	y \not\in \fr(M) &
    (\alpha\mbox{\rm --conversion)}\\
%%%
\index{conversion!$\alpha$--, $\beta$--, $\eta$--\faul}%%%
%%%
\label{eq:leg} 
& \mbox{\tt (\stlambda} x.M\mbox{\tt )}N \boldsymbol{=} [N/x]M 
	& (\beta\mbox{\rm --conversion}) \\
%%%
\label{eq:leh} 
& \mbox{\tt (\stlambda} x.M\mbox{\tt )} \boldsymbol{=} M \qquad 
        x \not\in \fr(M) &
        (\eta\mbox{\rm --conversion}) \\
%%%
\label{eq:lei} 
& M \boldsymbol{=} N \Pf \mbox{\tt (\stlambda}x.M\mbox{\tt )} 
	\boldsymbol{=} \mbox{\tt (\stlambda} x.N\mbox{\tt )}
	&  (\xi\mbox{\rm --rule}) &&
%%%
\index{$\xi$--rule}%%%
%%%
\end{align}
\end{subequations}
%%
We shall present the theory of $\lambda$--terms which we
shall use in the sequel. It consists in a set of equations
$M \boldsymbol{=} N$, where $M$ and $N$ are terms. These are subject to
the laws above. The theory axiomatized by \eqref{eq:lea} -- 
\eqref{eq:leg} and \eqref{eq:lei} is called $\mbox{\sf\textgreek{l}}$, 
the theory axiomatized by \eqref{eq:lea} --\eqref{eq:lei} 
$\mbox{\sf\textgreek{lh}}$. Notice that \eqref{eq:lea} -- \eqref{eq:lee}
 simply say that $\boldsymbol{=}$ is a congruence. A different rule 
is the following so--called \textbf{extensionality rule}.
%%
\begin{align}
& Mx \boldsymbol{=} Nx \Pf M \boldsymbol{=} N &
\mbox{\rm (ext)}
%%%
\index{(ext)}%%%
%%%
\end{align}
%%
It can be shown that $\mbox{\sf\textgreek{l}} + \mbox{\rm (ext)} = 
\mbox{\sf\textgreek{lh}}$.
The model theory of $\lambda$--calculus is somewhat tricky. Basically,
all that is assumed is that we have a domain $D$ together with a 
binary operation $\bullet$ that interprets function application. 
Abstraction is defined implicitly. Call a function $\beta : V \pf D$
a \textbf{valuation}. %%%
\index{valuation}%%%
%%
Now define $[M]^{\beta}$ inductively as follows.
%%
\begin{subequations}
\begin{align}
[\mbox{\tt x}_i]^{\beta} & := \beta(\mbox{\tt x}_i) \\
[\mbox{\tt (}MN\mbox{\tt )}]^{\beta} & := 
	[M]^{\beta}([N]^{\beta}) \\
\label{eq:func}
[\mbox{\tt (\stlambda} x.M\mbox{\tt )}]^{\beta} \bullet a &
    := [M]^{\beta[x := a]}
\end{align}
\end{subequations}
%%
(Here, $a \in D$.)
\eqref{eq:func} does not fix the interpretation of
$\mbox{\tt (\stlambda} x.M\mbox{\tt )}$ uniquely on the basis of 
the interpretation of $M$. If it does, however, the structure is 
called {\it extensional}. We shall return to that issue below. 
First we shall develop some more syntactic techniques for 
dealing with $\lambda$--terms.
%%
\begin{defn}
%%%
\index{replacement}%%
\index{$\lambda$--term!congruent}%%
\index{$\rightsquigarrow_{\alpha}, \triangleright_{\alpha}, \equiv_{\alpha}$}%%
%%%
Let $M$ and $N$ be $\lambda$--terms. We say, $N$ \textbf{is obtained
from $M$ by replacement of bound variables} or by 
$\alpha$--\textbf{conversion} and write $M \rightsquigarrow_{\alpha} N$ 
if there is a subterm {\tt (\stlambda$y$.$L$)} of $M$ and a variable 
$z$ which does not occur in $L$ such that $N$ is the result of replacing 
an occurrence of {\tt (\stlambda$y$.$L$)} by {\tt (\stlambda$z$.$[z/y]L$)}. 
The relation $\triangleright_{\alpha}$ is the transitive closure of
$\rightsquigarrow_{\alpha}$. $N$ is \textbf{congruent to} $M$, in symbols 
$M \equiv_{\alpha} N$, if both $M \triangleright_{\alpha} N$ and $N
\triangleright_{\alpha} M$.
\end{defn}
%%
Similarly the definition of $\beta$--conversion.
%%
\begin{defn}
%%%
\index{$\lambda$--term!contraction}%%
\index{$\rightsquigarrow_{\beta}, \triangleright_{\beta}, \equiv_{\beta}$}%%
%%%
Let $M$ be a $\lambda$--term. We write $M \rightsquigarrow_{\beta} N$ and
say that $M$ \textbf{contracts to} $N$ if $N$ is the result of a
single replacement of an occurrence of {\tt ((\stlambda$x$.$L$)$P$)}
in $M$ by {\tt ($[P/x]L$)}. Further, we write $M \triangleright_{\beta} N$ 
if $N$ results from $M$ by a series of contractions and 
$M \equiv_{\beta} N$ if $M \triangleright_{\beta} N$ and 
$N \triangleright_{\beta} M$.
\end{defn}
%%%%
\index{redex}%%
\index{contractum}%%
\index{$\lambda$--term!evaluated}%%
\index{normal form}%%
%%%%
A term of the form {\tt ((\stlambda$x$.$M$)$N$)} is called a 
\textbf{redex} and $[N/x]M$ its \textbf{contractum}. The step from 
the redex to the contractum 
represents the evaluation of a function to its argument. A 
$\lambda$--term is \textbf{evaluated} or \textbf{in normal form} 
if it contains no redex.

Similarly for the notation $\rightsquigarrow_{\alpha\beta}$,
$\triangleright_{\alpha\beta}$ and $\equiv_{\alpha\beta}$. Call
$M$ and $N$ $\alpha\beta$--\textbf{equi\-va\-lent}
($\alpha\beta\eta$--\textbf{equivalent}) if $\auf M, N\zu$ is 
contained in the least equivalence relation containing
$\triangleright_{\alpha\beta}$
($\triangleright_{\alpha\beta\eta})$.
%%%
\begin{prop}
$\mbox{\sf\textgreek{l}} \vdash M \boldsymbol{=} N$ iff $M$ and $N$ are
$\alpha\beta$--equivalent. $\mbox{\sf\textgreek{lh}} \vdash %
M \boldsymbol{=} N$ iff $M$ and $N$ are $\alpha\beta\eta$--equivalent.
\end{prop}
%%%
\index{normal form}%%
%%%
If $M \triangleright_{\alpha\beta} N$ and $N$ is in normal form
then $N$ is called a \textbf{normal form of} $M$. Without proof we
state the following theorem.
%%
\begin{thm}[Church, Rosser]
Let $L, M, N$ be $\lambda$--terms such that
$L \triangleright_{\alpha\beta} M$ and $L \triangleright_{\alpha\beta} 
N$. Then there exists a $P$
such that $M \triangleright_{\alpha\beta} P$ and $N
\triangleright_{\alpha\beta} P$.
\end{thm}
%%
The proof can be found in all books on the $\lambda$--calculus.
This theorem also holds for $\triangleright_{\alpha\beta\eta}$.
%%
\begin{cor}
Let $N$ and $N'$ be normal forms of $M$. Then $N \equiv_{\alpha}
N'$.
\end{cor}
%%
The proof is simple. For by the previous theorem there exists a $P$ such
that $N \triangleright_{\alpha\beta} P$ and $N'
\triangleright_{\alpha\beta} P$. But since $N$ as well as $N'$ do
not contain any redex and $\alpha$--conversion does not introduce
any redexes then $P$ results from $N$ and $N'$ by
$\alpha$--conversion. Hence $P$ is $\alpha$--congruent with $N$
and $N'$ and hence $N$ and $N'$ are $\alpha$--congruent.

Not every $\lambda$--term has a normal form. For example
%%
\begin{align}
& \mbox{\tt ((\stlambda x$_{\snull}$.(x$_{\snull}$x$_{\snull}$%
))(\stlambda x$_{\snull}$.(x$_{\snull}$x$_{\snull}$)))} \\\notag
\triangleright_{\beta} & 
\mbox{\tt ((\stlambda x$_{\snull}$.(x$_{\snull}$x$_{\snull}$%
))({\stlambda}x$_{\snull}$.(x$_{\snull}$x$_{\snull}$)))}
\end{align}
%%
Or
%%
\begin{align}
 & \mbox{\tt (({\stlambda}x$_{\snull}$.((x$_{\snull}$x$_{\snull}$%
)x$_{\seins}$))({\stlambda}x$_{\snull}$.((x$_{\snull}$x$_{\snull}$%
)x$_{\seins}$)))}
\\\notag
\triangleright_{\beta} &
\mbox{\tt ((({\stlambda}x$_{\snull}$.((x$_{\snull}$x$_{\snull}$%
)x$_{\seins}$))({\stlambda}x$_{\snull}$.((x$_{\snull}$x$_{\snull}$%
)x$_{\seins}$)))x$_{\seins}$)}
\\\notag
\triangleright_{\beta} &
\mbox{\tt (((({\stlambda}x$_{\snull}$.((x$_{\snull}$x$_{\snull}$%
)x$_{\seins}$))({\stlambda}x$_{\snull}$.((x$_{\snull}$x$_{\snull}$%
)x$_{\seins}$)))x$_{\seins}$)x$_{\seins}$)}
\end{align}
%%
The typed $\lambda$--calculus differs from the calculus which has
just been presented by an important restriction, namely that every
term must have a type.
%%
\begin{defn}
Let $B$ be a set. The set of \textbf{types over} $B$,
$\Typ_{\pf}(B)$, is the smallest set $M$ for which
the following holds.
%%
\begin{dingautolist}{192}
\item $B \subseteq M$.
\item If $\alpha \in M$ and $\beta \in M$ then $\alpha \pf \beta
    \in M$.
\end{dingautolist}
%%
\end{defn}
%%
In other words: types are simply terms in the signature
$\{\pf\}$ with $\Omega(\pf) = 2$ over a set of basic types.
Each term is associated with a type and the structure of terms
is restricted by the type assignment. Further, all $\Omega$--terms
are admitted. Their type is already fixed. The following rules
are valid.
%%
\begin{dingautolist}{192}
\item
If $\mbox{\tt (}MN\mbox{\tt )}$ is a term of type $\gamma$ then
there is a type $\alpha$ such that $M$ has the type
$\alpha \pf \gamma$ and $N$ the type $\gamma$.
\item
If $M$ has the type $\gamma$ and $x_{\alpha}$ is a variable of
type $\alpha$ then $\mbox{\tt (\stlambda} x_{\alpha} \mbox{\tt
.}M\mbox{\tt )}$ is of type $\alpha \pf \gamma$.
\end{dingautolist}
%%
Notice that for every type $\alpha$ there are countably many 
variables of type $\alpha$. More exactly, the set of variables 
of type $\alpha$ is 
$V^{\alpha} := \{\mbox{\tt x}^i_{\alpha} : i \in \omega\}$.
We shall often use the metavariables $x_{\alpha}$,
$y_{\alpha}$ and so on. If $\alpha \neq \beta$ then also
$x_{\alpha} \neq x_{\beta}$ (they represent different variables).
With these conditions the formation of $\lambda$--terms is severely 
restricted. For example $\mbox{\tt (\stlambda x$_{\snull}$.(x%
$_{\snull}$x$_{\snull}$))}$ is not a typed term no matter which 
type $\mbox{\tt x}_{\snull}$ has. One can show that a typed term 
always has a normal form. This is in fact an easy matter. Notice 
by the way that if the term {\tt (x$_{\snull}$+x$_{\seins}$)} has 
type $\alpha$ and {\tt x$_{\snull}$} and {\tt x$_{\seins}$} also 
have the type $\alpha$, the function 
{\tt ({\stlambda}x$_{\snull}$.({\stlambda}x$_{\seins}$%
.(x$_{\snull}$+x$_{\seins}$)))} has the type $\alpha
\pf (\alpha \pf \alpha)$. The type of an $\Omega$--term is the
type of its value, in this case $\alpha$. The types are nothing
but a special version of {\it sorts}. Simply take 
$\Typ_{\pf}(B)$ to be the set of sorts. However, while 
application (written $\bullet$) is a single symbol in the typed 
$\lambda$--calculus, we must now assume in place of it a family of 
symbols $\bullet^{\beta}_{\alpha}$ of signature $\auf \alpha\pf\beta,
\alpha, \beta\zu$ for every type $\alpha, \beta$. Namely,
$M \bullet^{\beta}_{\alpha} N$ is defined iff
$M$ has type $\alpha\pf\beta$ and $N$ type $\alpha$, and the
result is of sort (= type) $\beta$. While the notation within many
sorted algebras can get clumsy, the techniques (ultimately derived
from the theory of unsorted algebras) are very useful, so the
connection is very important for us. Notice that algebraically
speaking it is not $\mbox{\tt \stlambda}$ but 
$\mbox{\tt \stlambda x}_{\alpha}$
that is a member of the signature, and once again, in the many
sorted framework, $\mbox{\tt \stlambda x}_{\alpha}$ turns into a
family of operations $\mbox{\tt \tlambda$^{\beta}$x}_{\alpha}$ of
sort $\auf \beta, \alpha\pf\beta\zu$. That is to say, $\mbox{\tt
\stlambda$^{\beta}$x}_{\alpha}$ is a function symbol that only
forms a term with an argument of sort (= type) $\beta$ and yields
a term of type $\alpha\pf \beta$.

We shall now present a model of the $\lambda$--calculus. We begin 
by studying the purely applicative structures and then turn to 
abstraction after the introduction of combinators. In the untyped 
case application is a function that is everywhere defined.
The model structures are therefore so--called
{\it applicative structures}.
%%%
\begin{defn}
%%%%
\index{applicative structure}%%
\index{applicative structure!partial}
\index{applicative structure!extensional}%%
%%%%
An \textbf{applicative structure} is a pair $\GA = \auf A,
\bullet\zu$ where $\bullet$ is a binary operation on $A$. If
$\bullet$ is only a partial operation, $\auf A, \bullet\zu$ is
called a \textbf{partial applicative structure}. $\GA$ is called
\textbf{extensional} if for all $a, b \in A$:
%%
\begin{equation}
a = b \text{ iff for all }c \in A:
    a \bullet c = b \bullet c 
\end{equation}
%%
\end{defn}
%%%
\begin{defn}
%%%
\index{applicative structure!typed}%%%
%%%%
A \textbf{typed applicative structure} over a given set of basic
types $B$ is a structure $\auf \{A_{\alpha} : \alpha \in \Typ_{\pf}(B)\},
\bullet\zu$ such that (a) $A_{\alpha}$ is a set
for every $\alpha$, (b) $A_{\alpha} \cap A_{\beta} = \varnothing$ 
if $\alpha \neq \beta$ and (c) $a \bullet b$ is defined iff
there are types $\alpha\pf \beta$ and $\alpha$
such that $a \in A_{\alpha \pf \beta}$ and $b \in A_{\alpha}$,
and then $a \bullet b \in A_{\beta}$.
\end{defn}
%%
A typed applicative structure defines a partial applicative
structure. Namely, put $A := \bigcup_{\alpha} A_{\alpha}$; then
$\bullet$ is nothing but a partial binary operation on $A$. The
typing is then left implicit. (Recovering the types of elements is
not a trivial affair, see the exercises.) Not every partial
applicative structure can be typed, though.

One important type of models are those where $A$ consists of sets and
$\bullet$ is the usual functional application as defined in sets.
More precisely, we want that $A_{\alpha}$ is a set of sets for
every $\alpha$. So if the type is associated with the set $S$ then
a variable may assume as value any member of $S$. So, it follows
that if $\beta$ is associated with the set $T$ and $M$ has the
type $\beta$ then
the interpretation of $\mbox{\tt (\stlambda x}_{\alpha}.%
M\mbox{\tt )}$ is a function from $S$ to $T$. We set the
realization of $\alpha \pf \beta$ to be the set of all functions
from $S$ to $T$. This is an arbitrary choice, a different
choice (for example a suitable subset) would do as well.

Let $M$ and $N$ be sets. Then a function from $M$ to $N$
is a subset $F$ of the cartesian product $M \times N$ which
satisfies certain conditions (see Section~\ref{kap1}.\ref{kap1-1}).
Namely, for every $x \in M$ there must be a $y \in N$
such that $\auf x, y\zu \in F$ and if $\auf x, y\zu \in F$
and $\auf x, y'\zu \in F$ then $y = y'$. (For partial
functions the first condition is omitted. Everything else
remains. For simplicity we shall deal only with totally
defined functions.) Normally one thinks of a function
as something that gives values for certain arguments.
This is not so in this case. $F$ is not a function in this sense,
it is just the \textbf{graph} of a function. In set theory
one does not distinguish between a function and its graph.
We shall return to this later. How do we have to picture $F$ as
a set? Recall that we have defined
%%
\begin{equation}
M \times N = \{\auf x,y\zu : x\in M, y\in N\}
\end{equation}
%%
This is a set. Notice that $M \times (N \times O) \neq
(M \times N) \times O$. 
%For $M \times (N\times O)$
%consists of all pairs of the form $\auf x, \auf y, z\zu\zu$,
%where $x \in M$, $y \in N$ and $z \in O$, while
%$(M \times N)\times O$ consists of all pairs of the form
%$\auf \auf x,y\zu, z\zu$. And if
%$\auf x, \auf y, z\zu\zu = \auf \auf x', y'\zu, z'\zu$,
%then we must have $x = \auf x', y'\zu$ and
%$\auf y, z\zu = z'$. From this follows $x' \in x$.
%$M$ possesses an element which is minimal with respect to
%$\in$, namely $x^{\ast}$. (This follows from the axiom of foundation.) 
%If we put this element in place of $x$ we get a contradiction. 
%This shows $M \times (N \times O) \neq (M \times N) \times O$.
However, the mapping %%
%%
\index{$\rtimes$, $\ltimes$}%%
%%
\begin{equation}
\rtimes \colon \auf x, \auf y, z\zu\zu \mapsto
\auf \auf x, y\zu, z\zu \colon
M \times (N\times O) \pf (M\times N)\times O
\end{equation}
%%
is a bijection. Its inverse is the mapping
%%
\begin{equation}
\ltimes\; \colon \auf \auf x, y \zu, z\zu \mapsto
\auf x, \auf y, z\zu\zu \colon
(M \times N)\times O \pf M\times (N\times O)
\end{equation}
%%
Finally we put
%%
\begin{equation}
M \pf N := \{F \subseteq M\times N: F \mbox{ a function }\}
\end{equation}
%%%
\index{$M \pf N$}%%
%%
Elsewhere we have used the notation $N^M$ for that set.
Now functions are also sets and their arguments are sets, too. 
Hence we need a map which applies a function to an argument. 
Since it must be defined for all cases of
functions and arguments, it must by necessity be a partial
function. If $x$ is a function and $y$ an arbitrary object, 
we define $\mathsf{app}(x,y)$ as follows.
%%%%
\index{$\mathsf{app}(x,y)$}%%%
%%
\begin{equation}
\mathsf{app}(x,y) := \begin{cases}
        z & \text{ if $\auf y, z\zu \in x$,} \\
        \star & \text{ if no $z$ exists such that 
$\auf y,z\zu \in x$.}
        \end{cases}
\end{equation}
%%
$\mathsf{app}$ is a partial function. Its graph in the universe
of sets is a proper class, however. It is the class of pairs
$\auf \auf F, x\zu, y\zu$, where $F$ is a function and
$\auf x, y\zu \in F$.

Note that if $F \in M \pf (N \pf O)$ then
%%%
\begin{equation}
F \subseteq M \times (N \pf O) \subseteq M\times (N\times O)
\end{equation}
%%%
Then $\rtimes [F] \subseteq (M \times N) \times O$, and
one calculates that $\rtimes [F] \subseteq (M \times N)
\pf O$. In this way a unary function with values in
$N \pf O$ becomes a unary function from $M\times N$
to $O$ (or a binary function from $M$, $N$ to $O$). Conversely,
one can see that if $F \in (M \times N) \pf O$ then
$\ltimes [F] \in M \pf (N \pf O)$.
%%%
\begin{stz}
Let $V_{\omega}$ be the set of finite sets. Then $\auf V_{\omega},
\mathsf{app}\zu$ is a partial applicative structure.
\end{stz}
%%%
In place of $V_{\omega}$ one can take any $V_{\kappa}$ where
$\kappa$ is an ordinal. However, only if $\kappa$ is a limit 
ordinal (that is, an ordinal without predecessor), the structure 
will be combinatorially complete. A more general result is described 
in the following theorem for the typed calculus. Its proof is 
straightforward.
%%%
\begin{stz}
Let $B$ be the set of basic types and $M_{b}$, $b \in B$,
arbitrary sets. Let $M_{\alpha}$ be inductively defined
by $M_{\alpha \pf \beta} := (M_{\beta})^{M_{\alpha}}$.
Then 
%%%
\begin{equation}
\auf \{M_{\alpha} : \alpha \in \Typ_{\pf}(B)\},
\mathsf{app}\zu
\end{equation}
%%%
is a typed applicative structure. Moreover, it is extensional.
\end{stz}
%%%
For a proof of this theorem one simply has to check the conditions.

In categorial grammar, with which we shall deal in this chapter,
we shall use $\lambda$--terms to name meanings for symbols and
strings. It is important however that the $\lambda$--term is only
a formal entity (namely a certain string), and it is not the
meaning in the proper sense of the word. To give an example,
$\mbox{\tt (\stlambda x}_{\snull}\mbox{\tt .(\stlambda x}_{\seins}%
\mbox{\tt .x}_{\snull}\mbox{\tt +x}_{\seins}\mbox{\tt ))}$ is a string which
names a function. In the set universe, this function is a subset
of $\BN \pf (\BN \pf \BN)$. For this reason one has to distinguish
between equality $=$ and the symbol(s) $\equiv$/$\boldsymbol{=}$. $M = N$ 
means that we are dealing with the same strings (hence literally the
same $\lambda$--terms) while $M \equiv N$ means that $M$ and
$N$ name the same function. In this sense $\mbox{\tt ({\stlambda}%
x$_{\snull}$.({\stlambda}x$_{\seins}$.x$_{\snull}$+x$_{\seins}$%
))(x$_{\snull}$)(x$_{\szwei}$)} \neq \mbox{\tt x$_{\snull}$+x%
$_{\szwei}$}$, but they also denote the same value. Nevertheless, in 
what is to follow we shall not always distinguish between a 
$\lambda$--term and its interpretation, in order not to make the 
notation too opaque.

The $\lambda$--calculus has a very big disadvantage, namely that
it requires some caution in dealing with variables. However, there 
is a way to avoid having to use variables. This is achieved through 
the use of combinators.
%%%%
\index{{\tt S}, {\tt K}, {\tt I}}%%
%%%%
Given a set $V$ of variables and the zeroary constants {\tt S},
{\tt K}, {\tt I}, combinators are terms over the signature that
has only one more binary symbol, $\bullet$. This symbol is
generally omitted, and terms are formed using infix notation with
brackets. Call this signature $\Gamma$.
%%
\begin{defn}
%%%
\index{combinator}%%%
\index{combinatorial term}%%
%%%
An element of $\Tm_{\Gamma}(V)$ is called a \textbf{combinatorial term}. 
A \textbf{combinator} is an element of $\Tm_{\Gamma}(\varnothing)$.
\end{defn}
%%
Further, the redex relation $\triangleright$ is 
defined as follows. 
%%
\begin{subequations}
\begin{align}
\label{eq:reda}
& \mbox{\tt I}X \triangleright X \\
\label{eq:redb}
& \mbox{\tt K}XY \triangleright X \\
\label{eq:redc}
& \mbox{\tt S}XYZ \triangleright XZ(YZ) \\
\label{eq:redd}
& X \triangleright X \\
\label{eq:rede}
& \text{if }X \triangleright Y \text{ and }Y \triangleright Z 
\text{ then }X \triangleright Z \\
\label{eq:redf}
& \text{if }X \triangleright Y\text{ then }\mbox{\mtt ($XZ$)} 
\triangleright \mbox{\mtt ($YZ$)} \\
\label{eq:redg}
& \text{if }X \triangleright Y\text{ then }\mbox{\mtt ($ZX$)} 
\triangleright \mbox{\mtt ($ZY$)} 
\end{align}
\end{subequations}
%
\index{combinatory logic}%%
\index{$\mathsf{CL}$}%%
%%%
\textbf{Combinatory logic} ($\mathsf{CL}$) is \eqref{eq:reda} -- 
\eqref{eq:rede}. It is an equational theory if we read $\triangleright$ 
simply as equality. 
(The only difference is that $\triangleright$ is not symmetric. So, to be 
exact, the rule `if $X \boldsymbol{\doteq} Y$ then $Y \boldsymbol{\doteq} 
X$' needs to be added.) We note that there is a combinator $\mathsf{C}$ 
containing only {\tt K} and {\tt S} such that 
$\mathsf{C} \triangleright \mbox{\tt I}$ (see
Exercise~\ref{ex:ski}). This explains why {\tt I} is sometimes
omitted.

We shall now show that combinators can be defined by
$\lambda$--terms and vice versa. First, define
%%
\begin{subequations}
\begin{align}
\mbox{\sf I} & := \mbox{\tt ({\stlambda}x$_{\snull}$.x$_{\snull}$)} \\
\mbox{\sf K} & := \mbox{\tt ({\stlambda}x$_{\snull}$.({\stlambda}x%
$_{\seins}$.x$_{\snull}$))} \\
\mbox{\sf S} & := \mbox{\tt ({\stlambda}x$_{\snull}$.({\stlambda}%
x$_{\seins}$.({\stlambda}x$_{\szwei}$.x$_{\snull}$x$_{\szwei}$(x%
$_{\seins}$x$_{\szwei}$))))}
\end{align}
\end{subequations}
%%
Define a translation $^{\lambda}$ by $X^{\lambda} := X$ for $X \in
V$, $\mbox{\tt S}^{\lambda} := \mathsf{S}$, $\mbox{\tt
K}^{\lambda} := \mathsf{K}$, $\mbox{\tt I}^{\lambda} :=
\mathsf{I}$. Then the following is proved by induction on the
length of the proof.
%%%
\begin{thm}
\label{thm:CombzuLambda} 
Let $C$ and $D$ be combinators. If $C \triangleright D$ then $C^{\lambda}
\triangleright_{\beta} D^{\lambda}$. Also, if 
$\mathsf{CL} \vdash C \boldsymbol{=} D$ then 
$\mbox{\sf\textgreek{l}} \vdash C^{\lambda} \boldsymbol{=} D^{\lambda}$.
\end{thm}
%%%
The converse translation is more difficult. We shall define first 
a function $[x]$ on combinatory terms. (Notice that there are no 
bound variables, so $\var(M) = \fr(M)$ for any combinatorial term
$M$.) 
%%
\begin{subequations}
\begin{align}
& [x]x := \mbox{\tt I}. \\
\label{eq:abs2} & 
	[x]M := \mbox{\tt K}M, \text{ if $x \not\in \var(M)$.} \\
\label{eq:abs3} & 
	[x]Mx := M, \text{ if $x \not\in \var(M)$.} \\
\label{eq:abs4} & 
	[x]MN := \mbox{\tt S}\mbox{\tt (}[x]M]\mbox{\tt )(}[x]N\mbox{\tt )},
    \text{ otherwise.}
\end{align}
\end{subequations}
%%
(So, \eqref{eq:abs4} is applied only if \eqref{eq:abs2} and 
\eqref{eq:abs3} cannot be applied.) 
For example $[\mbox{\tt x$_{\seins}$}]\mbox{\tt x$_{\seins}$x$_{\snull}$} =
\mbox{\tt S(}[\mbox{\tt x$_{\seins}$}]\mbox{\tt x$_{\seins}$)(}
    [\mbox{\tt x$_{\seins}$}]\mbox{\tt x$_{\snull}$)}
        = \mbox{\tt SI(Kx$_{\snull}$)}$.
Indeed, if one applies this to {\tt x$_{\seins}$}, then one gets
    %%
\begin{equation}
\mbox{\tt SI(Kx$_{\snull}$)x$_{\seins}$}
\triangleright
\mbox{\tt Ix$_{\seins}$(Kx$_{\snull}$x$_{\seins}$)}
\triangleright
\mbox{\tt x$_{\seins}$(Kx$_{\snull}$x$_{\seins}$)} 
\triangleright 
\mbox{\tt x$_{\seins}$x$_{\snull}$}
\end{equation}
%%
Further, one has
%%
\begin{equation}
\mathsf{U} := [\mbox{\tt x$_{\seins}$}]\mbox{\tt (}[\mbox{\tt %
x$_{\snull}$}]\mbox{\tt x$_{\seins}$x$_{\snull}$)} 
= [\mbox{\tt x$_{\seins}$}]\mbox{\tt SI(Kx$_{\snull}$)} 
= \mbox{\tt S(K(SI))K}
\end{equation}
%%
The reader may verify that 
$\mathsf{U}\mbox{\tt x$_{\snull}$x$_{\seins}$}
\triangleright \mbox{\tt x$_{\seins}$x$_{\snull}$}$. Now define
$^{\kappa}$ by $x^{\kappa} := x$, $x \in V$, $\mbox{\tt (}MN%
\mbox{\tt )}^{\kappa} := \mbox{\tt (}M^{\kappa}N^{\kappa}\mbox{\tt )}$
and $\mbox{\tt (\stlambda}x.N\mbox{\tt )}^{\kappa} :=
[x]N^{\kappa}$.
%%
\begin{thm}
Let $C$ be a closed $\lambda$--term. Then
$\mbox{\sf\textgreek{l}} \vdash C \boldsymbol{=} C^{\kappa}$.
\end{thm}
%%
Now we have defined translations from $\lambda$--terms to
combinators and back. It can be shown, however, that the theory
$\mbox{\sf\textgreek{l}}$ is stronger than $\mathsf{CL}$ under 
translation. Curry found a list $\mathbf{A}_{\beta}$ of five equations such
%%%
\index{Curry, Haskell B.}%%%
%%%%
that $\mbox{\sf\textgreek{l}}$ is as strong as $\mathsf{CL} +
\mathbf{A}_{\beta}$ in the sense of
Theorem~\ref{thm:LambdazuComb} below. Also, he gave a list
$\mathbf{A}_{\beta\eta}$ such that $\mathsf{CL} + 
\mathbf{A}_{\beta\eta}$ is equivalent to $\mbox{\sf\textgreek{lh}} =
\mbox{\sf\textgreek{l}} + \mbox{\rm (ext)}$. $\mathbf{A}_{\beta\eta}$ 
also is equivalent to the first--order postulate (ext): $(\forall
xy)((\forall z)(x \bullet z \boldsymbol{=} y \bullet z) \pf x 
\boldsymbol{=} y)$.
%%%
\begin{thm}[Curry]
%%%
\index{Curry, Haskell B.}%%%
%%%%
\label{thm:LambdazuComb}
Let $M$ and $N$ be $\lambda$--terms.
%%%
\begin{dingautolist}{192}
\item
If $\mbox{\sf\textgreek{l}} \vdash M \boldsymbol{=} N$ then
$\mathsf{CL} + \mathbf{A}_{\beta}
\vdash M^{\kappa} \boldsymbol{=} N^{\kappa}$.
\item
If $\mbox{\sf\textgreek{lh}} \vdash M \boldsymbol{=} N$ then $\mathsf{CL} +
\mathbf{A}_{\beta\eta} \vdash M^{\kappa} \boldsymbol{=} N^{\kappa}$.
\end{dingautolist}
\end{thm}
%%%
There is also a typed version of combinatorial logic. There are
two basic approaches. The first is to define typed combinators.
The basic combinators now split into infinitely many typed
versions as follows.
%%
\begin{equation}
$$\begin{array}{ll}
\mbox{\rm Combinator} & \mbox{\rm Type} \\\hline
\mbox{\tt I}_{\alpha} & \alpha \pf \alpha \\
\mbox{\tt K}_{\alpha,\beta} & \alpha \pf (\beta \pf \alpha) \\
\mbox{\tt S}_{\alpha,\beta,\gamma} &
    (\alpha \pf (\beta \pf \gamma)) \pf ((\alpha \pf \beta)
        \pf (\alpha \pf \gamma)) 
\end{array}
\end{equation}
%%
Together with $\bullet$ they form the typed signature $\Gamma^\tau$.
For each type there are countably infinitely many variables
of that type in $V$.
%%%
\index{combinatorial term!typed}%%
\index{combinator!typed}%%
%%%
\textbf{Typed combinatorial terms} are elements
of $\Tm_{\Gamma^{\tau}}(V)$, and \textbf{typed combinators}
are elements of $\Tm_{\Gamma^{\tau}}$. Further, if $M$
is a combinator of type $\alpha\pf\beta$ and $N$ a combinator of
type $\alpha$  then $\mbox{\tt (}MN\mbox{\tt )}$ is a combinator
of type $\beta$. In this way, every typed combinatorial term has
a unique type.

The second approach is to keep the symbols {\tt I}, {\tt K} and
{\tt S} and to let them stand for any of the above typed
combinators. In terms of functions, {\tt I} takes an argument $N$
of any type $\alpha$ and returns $N$ (of type $\alpha$). Likewise,
{\tt K} is defined on any $M$, $N$ of type $\alpha$ and $\beta$,
respectively, and $\mbox{\tt K}MN = M$ of type $\alpha$. Also,
$\mbox{\tt K}M$ is defined and of type $\beta \pf \alpha$. Basically,
the language is the same as in the untyped case. A combinatorial
term is
%%%
\index{combinatorial term!stratified}%%
\index{combinator!stratified}%%
%%%
\textbf{stratified} if for each variable and each occurrence of {\tt
I}, {\tt K}, {\tt S} there exists a type such that if that
(occurrence of the) symbol is assigned that type, the resulting
string is a typed combinatorial term. (So, while each occurrence
of {\tt I}, {\tt K} and {\tt S}, respectively, may be given a
different type, each occurrence of the same variable must have the
same type.) For example, {\sf B} := {\tt S(KS)K} is stratified,
while {\tt SII} is not.

We show the second claim first. Suppose that there are types
$\alpha$, $\beta$, $\gamma$, $\delta$, $\epsilon$ such that
{\mtt ((S$_{\alpha,\beta,\gamma}$I$_{\delta}$)I$_{\epsilon}$)} 
is a typed combinator.
%%
\begin{equation}
$$\begin{array}{ccc}
\mbox{\tt ((S}_{\alpha,\beta,\gamma} & {\tt I}_{\delta} \mbox{\tt
)} & {\tt I}_{\epsilon}\mbox{\tt )}
\\\hline
(\alpha \pf (\beta\pf\gamma)) & \delta \pf\delta & \epsilon\pf\epsilon \\
	\quad \pf ((\alpha \pf\beta) \pf (\alpha\pf\gamma)) 
	& & 
\end{array}
\end{equation}
%%
Then, since $\mbox{\tt S}_{\alpha,\beta,\gamma}$ is applied to 
$\mbox{\tt I}_{\delta}$ we must have $\delta \pf \delta = \alpha %
\pf (\beta\pf \gamma)$, whence $\alpha = (\beta\pf\gamma)$. So, 
{\mtt (S$_{\alpha,\beta,\gamma}$I$_{\delta}$)} has the type 
%%%
\begin{equation}
((\beta \pf \gamma) \pf \beta) \pf ((\beta \pf \gamma) \pf \gamma)
\end{equation}
%%%
This combinator is applied to $\mbox{\tt I}_{\epsilon}$, and so we have 
$(\beta\pf\gamma)\pf\beta = \epsilon\pf\epsilon$, whence
$\beta\pf\gamma = \epsilon = \beta$, which is impossible. So, {\tt
SII} is not stratified. On the other hand, {\sf B} is stratified.
Assume types such that $\mbox{\tt S}_{\zeta,\eta,\theta}\mbox{\tt
(K}_{\alpha,\beta} \mbox{\tt S}_{\gamma,\delta,\epsilon}\mbox{\tt
)}\mbox{\tt K}_{\iota,\kappa}$ is a typed combinator. First, ${\tt
K}_{\alpha,\beta}$ is applied to ${\tt
S}_{\gamma,\delta,\epsilon}$. This means that
%%
\begin{equation}
\alpha = (\gamma \pf (\delta\pf\epsilon)) \pf ((\gamma\pf\delta)
\pf (\gamma\pf\epsilon))
\end{equation}
%%
The result has type
%%
\begin{equation}
\beta \pf ((\gamma \pf (\delta\pf\epsilon)) \pf ((\gamma\pf\delta)
\pf (\gamma\pf\epsilon)))
\end{equation}
%%
This is the argument of ${\tt S}_{\zeta,\eta,\theta}$. Hence we
must have
%%
\begin{multline}
\zeta \pf (\eta \pf \theta) \\
= \beta \pf ((\gamma \pf
(\delta\pf\epsilon)) \pf ((\gamma\pf\delta)
\pf (\gamma\pf\epsilon)))
\end{multline}
%%
So, $\zeta = \beta$, $\eta = \gamma\pf(\delta\pf\epsilon)$,
$\theta = (\gamma\pf\delta)\pf(\gamma\pf\epsilon)$. The resulting
type is $(\zeta\pf\eta)\pf(\zeta\pf\theta)$. This is applied to
$\mbox{\tt K}_{\iota,\kappa}$ of type $\iota\pf(\kappa\pf\iota)$.
For this to be well--defined we must have $\iota\pf(\kappa\pf\iota)
= \zeta\pf\eta$, or $\iota = \zeta = \beta$ and $\kappa\pf\iota =
\eta = \gamma\pf(\delta\pf\epsilon)$. Finally, this results in
$\kappa = \gamma$, $\iota = \beta = \delta\pf\epsilon$. So,
$\alpha$, $\gamma$, $\delta$ and $\epsilon$ may be freely chosen,
and the other types are immediately defined.

It is the second approach that will be the most useful for us
later on. We call combinators \textbf{implicitly typed} if they
are thought of as typed in this way. (In fact, they simply
are untyped terms.) The same can be done with $\lambda$--terms,
giving rise to the notion of a stratified $\lambda$--term. In
the sequel we shall not distinguish between combinators and their
representing $\lambda$--terms.

Finally, let us return to the models of the $\lambda$--calculus.
Recall that we have defined abstraction only implicitly, using
Definition~\eqref{eq:func} repeated below:
%%%
\begin{equation}
[\mbox{\tt (\stlambda} x.M\mbox{\tt )}]^{\beta} \bullet a 
    := [M]^{\beta[x := a]}
\end{equation}
%%%
In general, this object need not exist, in which case we do not have 
a model for the $\lambda$--calculus. 
%%%
\begin{defn}
%%%
\index{applicative structure!combinatorially complete}%%%
%%%
An applicative structure $\GA$ is called \textbf{combinatorially
complete} if for every term $t$ in the language with free
variables from $\{\mbox{\tt x}_i : i < n\}$
there exists a $y$ such that for all $b_i \in A$, $i < n$:
%%
\begin{equation}
(\dotsb ((y \bullet b_0) \bullet b_1) \bullet
\dotsb \bullet b_{n-1}) = t(b_0, \dotsc, b_{n-1})
\end{equation}
%%
\end{defn}
%%
This means that for every term $t$ there exists an element
which represents this term: 
%%
\begin{equation}
\mbox{\tt ({\stlambda}x$_{\snull}$.({\stlambda}x$_{\seins}$.}%
\dotsb\mbox{\tt .({\stlambda}x}_{n-1}\mbox{\tt .}t(\mbox{\tt x}_{\snull},
    \dotsc, \mbox{\tt x}_{n-1})\mbox{\tt )}\dotsb
    \mbox{\tt ))} 
\end{equation}
%%
Thus, this defines the notion of an applicative structure in which 
every element can be abstracted. It is these structures that can 
serve as models of the $\lambda$--calculus. Still, no explicit way 
of generating the functions is provided. One way is to use countably 
many abstraction operations, one for every number $i < \omega$
(see Section~\ref{kap6}.\ref{kap6-4b}). Another way is to translate 
$\lambda$--terms into combinatory logic using $[-]$ for abstraction.
In view of the results obtained above we get the following result.
%%
\begin{thm}[Sch\"onfinkel]
%%%
\index{Sch\"onfinkel, Moses}%%%
%%%
$\GA$ is combinatorially complete iff there are elements
$k$ and $s$ such that
%%
\begin{equation}
((k \bullet a) \bullet b) = a \qquad
(((s \bullet a) \bullet b) \bullet c) =
(a \bullet c) \bullet (b \bullet c)
\end{equation}
%%
\end{thm}
%%
\begin{defn}
\index{combinatory algebra}%%%
\index{combinatory algebra!extensional}%%%
\index{$\lambda$--algebra}%%
%%%
A structure $\GA = \auf A, \bullet,
\mbox{\sf k}, \mbox{\sf s}\zu$ is called a \textbf{combinatory
algebra} if $\GA \vDash \mbox{\sf k} \bullet x \bullet y \boldsymbol{=} x,
\mbox{\sf s} \bullet x \bullet y \bullet z \boldsymbol{=} x \bullet z
\bullet (y \bullet z)$. It is a
$\lambda$--\textbf{algebra} (or \textbf{extensional}) if it satisfies
$\mathbf{A}_{\beta}$ ($\mathbf{A}_{\beta\eta}$) in addition.
\end{defn}
%%
So, the class of combinatory algebras is an equationally definable
class. (This is why we have not required $|A|>1$, as is often
done.) Again, the partial case is interesting. Hence, we can use
the theorems of Section~\ref{kap1}.\ref{kap1-1} to create structures. Two
models are of particular significance. One is based on the algebra
of combinatorial terms over $V$ modulo derivable identity, the
other is the algebra of combinators modulo derivable identity.
Indirectly, this also shows how to create models for the
$\lambda$--calculus. We shall explain a different method below
in Section~\ref{kap6}.\ref{kap6-4b}.

Call a structure $\auf A, \bullet, \mathsf{k}, \mathsf{s}\zu$
%%%%
\index{combinatory algebra!partial}%%
%%%%
a \textbf{partial combinatory algebra} if (i) $\mathsf{s} \bullet x
\bullet y$ is always defined and (ii) the defining equations hold
in the intermediate sense, that is, if one side is defined so is
the other and they are equal (cf. Section~\ref{kap1}.\ref{kap1-1}). 
Consider once again the universe $V_{\omega}$. Define
%%
\begin{align}
\Gk & := \{\auf x, \auf y, x\zu\zu : x, y \in V_{\omega}\} \\
\Gs & := \{\auf x, \auf y, \auf z, \mathsf{app}(\mathsf{app}(x,z), 
\mathsf{app}(y,z))\zu\zu : x, y, z \in V_{\omega}\}
\end{align}
%%
$\auf V_{\omega}, \mbox{\sf app}, \Gk, \Gs\zu$ is not a
partial combinatory algebra because
$\mathsf{app}(\mathsf{app}(\Gk, x), y)$ is not always
defined. So, the equation $(k \bullet x) \bullet y \boldsymbol{=} x$ 
does not hold in the intermediate sense (since the right hand is 
obviously always defined). The defining equations hold
only in the weak sense: if both sides are defined, then they are
equal. Thus, $V_{\omega}$ is a useful model only in the typed 
case.

In the typed case we need a variety of combinators. More exactly:
for all types $\alpha$, $\beta$ and $\gamma$ we need elements
$\mathsf{k}_{\delta} \in A_{\delta}$, $\delta = \alpha \pf %
(\beta \pf \alpha)$ and $\mathsf{s}_{\eta} \in A_{\eta}$, $\eta
= (\alpha \pf (\beta\pf \gamma)) \pf  ((\alpha \pf \beta) \pf (\alpha %
\pf \gamma))$ such that for all $a \in A_{\alpha}$ and %
$b \in A_{\beta}$ we have
%%
\begin{equation}
(\mathsf{k}_{\delta} \bullet a) \bullet b = a
\end{equation}
%%
and for every $a \in A_{\alpha \pf (\beta \pf \gamma)}$,
$b \in A_{\alpha \pf \beta}$ and $c \in A_{\alpha}$ we have
%%
\begin{equation}
((\mathsf{s}_{\eta} \bullet a) \bullet b) \bullet c =
(a \bullet c) \bullet (b \bullet c) 
\end{equation}

We now turn to an interesting connection between intuitionistic
logic and type theory, known as the
{\it Curry--Howard--Isomorphism}.
%%%
\index{Curry, Haskell B.}%%%
\index{Curry--Howard--Isomorphism}%%
\index{Howard, William}%%%
%%%
Write $M : \varphi$ if $M$ is a $\lambda$--term of type $\varphi$.
Notice that while each term has exactly one type, there are
infinitely many terms having the same type. The following
is a Gentzen--calculus for statements of the form $M : \varphi$.
Here, $\Gamma$, $\Delta$, $\Theta$ denote arbitrary sets of such
statements, $x$, $y$ individual variables (of appropriate type),
and $M$, $N$ terms. The rules are shown in Table~\ref{tab:CHI}.
%%
\begin{table}
\caption{Rules of the Labelled Calculus}
\label{tab:CHI}
$$\begin{array}{l}
\mbox{\rm (axiom)} \quad
x : \varphi \bvdash x : \varphi
\qquad
\mbox{\rm (M)} \quad \begin{array}{c}
\Gamma \bvdash M : \varphi \\\hline
\Gamma, x : \chi \bvdash M : \varphi
\end{array} \\
\mbox{\rm (cut)} \quad
\begin{array}{c}
\Gamma \bvdash M : \varphi  \quad \Delta, x : \varphi, \Theta \bvdash
    N : \chi \\\hline
\Delta, \Gamma, \Theta \bvdash [M/x]N : B
\end{array}
\\
\mbox{(\textbf{E}{\mtt\symbol{25}})}\quad
\begin{array}{c}
\Gamma \bvdash M : \mbox{\mtt ($\varphi$\symbol{25}$\chi$)}%
	\quad \Delta \bvdash N : \varphi
\\\hline
\Gamma, \Delta \bvdash \mbox{\tt (}MN\mbox{\tt )} : \chi
\end{array}
    \\
\mbox{(\textbf{I}{\mtt\symbol{25}})}\quad
\begin{array}{c}
\Gamma, x : \varphi \bvdash M : \chi \\\hline
\Gamma \bvdash \mbox{\tt (\stlambda$x$.$M$)} :
    \mbox{\mtt ($\varphi$\symbol{25}$\chi$)}
\end{array}
\end{array}$$
\end{table}
%%
First of all notice that if we strip off the labelling by
$\lambda$--terms we get a natural deduction calculus for
intuitionistic logic (in the only connective {\mtt\symbol{25}}). 
Hence if a sequent $\{M_i : \varphi_i : i < n\} \bvdash N : \chi$ 
is derivable then $\stackrel{\CH}{\rightsquigarrow} 
\{ \varphi_i : i < n\}\bvdash \chi$, whence 
$\{\varphi_i : i < n\} \vdash^{\mathsf{H}} \chi$. 
Conversely, given a natural deduction proof of 
$\{\varphi_i : i < n\} \bvdash \chi$, we can decorate the proof with
$\lambda$--terms by assigning the variables at the leaves of the
tree for the axioms and then descending it until we hit the root.
Then we get a proof of the sequent $\{M_i : \varphi_i : i < n\}
\bvdash N : \chi$ in the above calculus.

Now we interpret the intuitionistic formulae in this proof calculus
as types. For a set $\Gamma$ of $\lambda$--terms over the set $B$
of basic types we put
%%
\begin{equation}
|\Gamma| := \{\varphi \in \Typ_{\pf}(B) :
    \mbox{ there is } M \in \Gamma \mbox{ of type } \varphi\}
\end{equation}
%%
\begin{defn}
For a set $\Gamma$ of types and a single type $\varphi$ over a set $B$ 
of basic types we put $\Gamma \vdash^{\boldsymbol{\lambda}} \varphi$ if 
there is a term $M$ of type $\varphi$ such that every type of a variable 
occurring free in $M$ is in $\Gamma$.
\end{defn}
%%
Returning to our calculus above we notice that if
%%
\begin{equation}
\{M_i : \varphi_i : i < n\} \bvdash N : \chi
\end{equation}
%%
is derivable, we also have 
$\{\varphi_i : i < n\} \vdash^{\boldsymbol{\lambda}}
\chi$. This is established by induction on the proof.
Moreover, the converse also holds (by induction on the
derivation). Hence we have the following result.
%%
\begin{thm}[Curry]
%%%
\index{Curry, Haskell B.}%%%
%%%
$\Gamma \vdash^{\boldsymbol{\lambda}} \varphi$ iff 
$\Gamma \vdash^{\mathsf{H}} \varphi$.
\end{thm}
%%
The correspondence between intuitionistic formulae and types has
also been used to obtain a rather nice characterization of
shortest proofs. Basically, it turns out that a proof of $\Gamma
\bvdash N : \varphi$ can be shortened if $N$ contains a redex.
Suppose, namely, that $N$ contains the redex {\tt ((\stlambda
$x$.$M$)$U$)}. Then, as is easily seen, the proof
contains a proof of $\Delta \bvdash \mbox{\tt (\stlambda$x$.$M$%
)$U$)} : \chi$. This proof part can be shortened. To
simplify the argument here we assume that no use of (cut) and (M)
has been made. Observe that we can assume that this very sequent
has been introduced by the rule (\textbf{I}{\mtt\symbol{25}}) and 
its left premiss by the rule (\textbf{E}{\mtt\symbol{25}}) and 
$\Delta = \Delta' \cup \Delta''$.  
%%
\begin{equation}
\begin{array}{ccc}
\Delta', x : \psi \bvdash M : \chi & \qquad & \\\cline{1-1}
\Delta' \bvdash \mbox{\tt (\stlambda$x$.$M$)} :
    \mbox{\mtt ($\psi$\symbol{25}$\chi$)} & & \Delta'' 
	\bvdash U : \psi \\\hline
\multicolumn{3}{c}{%
\Delta', \Delta'' \bvdash \mbox{\tt ((\stlambda$x$.$M$)$U$)} : \chi}
\end{array}
\end{equation}
%%
Then a single application of (cut) gives this:
%%
\begin{equation}
\begin{array}{c}
\Delta'' \bvdash U : \psi \qquad \Delta', x : \psi \bvdash
    M : \chi \\\hline
\Delta', \Delta'' \bvdash [M/x]U : \chi
\end{array}
\end{equation}
%%
While the types and the antecedent have remained constant, the
conclusion now has a term associated to it that is derived from
contracting the redex. The same can be shown if we take
intervening applications of (cut) and (M), but the proof is more
involved. Essentially, we need to perform more complex proof
transformations. There is another simplification that can be made,
namely when the derived term is explicitly $\alpha$--converted.
Then we have a sequent of the form $\Gamma \bvdash \mbox{\tt (%
\stlambda} x\mbox{\tt .}Mx\mbox{\tt )} : \mbox{\mtt ($\varphi$%\
\symbol{25}$\chi$)}$. Then, again putting aside intervening 
occurrences of (cut) and (M), the proof is as follows.
%%
\begin{equation}
\begin{array}{c}
\Gamma \bvdash \mbox{\tt (\stlambda} x\mbox{\tt .}Mx\mbox{\tt )}
    : \varphi\pf\chi  \qquad y : \varphi \bvdash y : \varphi \\\hline
\Gamma, y : \varphi \bvdash \mbox{\tt (}My\mbox{\tt )} : \chi
\\\hline \Gamma \bvdash \mbox{\tt (\stlambda} y\mbox{\tt
.}My\mbox{\tt )} : \mbox{\mtt ($\varphi$\symbol{25}$\chi$)}
\end{array}
\end{equation}
%%
This proof part can be eliminated completely, leaving only the
proof of the left hand premiss. An immediate corollary of this
fact is that if the sequent 
$\{x_i : \varphi_i : i < n\} \bvdash N : \chi$ is
provable for some $N$, then there is an $N'$ obtained from $N$ by
a series of $\alpha$--/$\beta$-- and $\eta$--normalization steps
such that the sequent $\{x_i : \varphi_i : i < n\} \bvdash N' : \chi$ 
is also derivable. The proof of the latter formula is shorter than the
first on condition that $N$ contains a subterm that can be
$\beta$-- or $\eta$--reduced.

{\it Notes on this section.} $\lambda$--abstraction already
appeared in \cite{frege:funktion} (written in 1891). 
%%%
\index{Frege, Gottlob}%%%
%%%%
Frege wrote $\stackrel{\boldmath{,}}{\mbox{\textgreek{e}}}$.$f$(\textgreek{e}). 
The first to study abstraction systematically was Alonzo Church 
%%%
\index{Church, Alonzo}%%
%%%
(see \cite{church:foundation}). Combinatory logic on the other hand has 
appeared first in the work of Moses Sch\"onfinkel 
%%%
\index{Sch\"onfinkel, Moses}%%%
%%%
\shortcite{schoenfinkel:bausteine} and Haskell
Curry \shortcite{curry:grundlagen}. The typing is reminiscent of 
%%%
\index{Curry, Haskell B.}%%%
%%%
Husserl's 
%%%
\index{Husserl, Edmund}%%%
%%%
semantic categories. More on that in
Chapter~\ref{kap6}. Suffice it to say that two elements are of the
same semantic category iff they can meaningfully occur in the same
terms. There are exercises below on applicative structures that
demonstrate that Husserl's conception characterizes exactly the
types up to renaming of the basic types.
%%
\vplatz
\exercise
Show that in $\mathsf{ZFC}$,  $M \times (N \times O) \neq 
(M \times N) \times O$. 
\vplatz%%
\exercise%%
Find combinators $\mathsf{G}$ and $\mathsf{C}$ such that $\mathsf{G}XYZ
\triangleright X(ZYZ)$ and $\mathsf{C}XYZ \triangleright XZY$.
%%
\vplatz%%
\exercise%%
Determine all types of $\mathsf{G}$ and $\mathsf{C}$ of the previous
exercise.
%%
\vplatz%%
\exercise%%
\label{ex:ski}%%
We have seen in Section~\ref{kap3}.\ref{kap:prop} that 
{\mtt ($\varphi$\symbol{25}$\varphi$)} can be derived from (a0) and 
(a1). Use this proof to give a
definition of {\tt I} in terms of {\tt K} and {\tt S}.
%%
\vplatz
\exercise
Show that any combinatorially complete applicative structure with
more than one element is infinite.
%%
\vplatz
\exercise
Show that $\bullet$, $\Gk$ and $\Gs$ defined on $V_{\omega}$
are proper classes in $V_{\omega}$. {\it Hint.} It suffices
to show that they are infinite. However, there is a proof that 
works for any universe $V_{\kappa}$, so here is a more general 
method. Say that $C \subseteq V_{\kappa}$ is {\it rich\/} if for 
every $x \in V_{\kappa}$, $x \in^+ C$. Show that no set is rich. 
Next show that $\bullet$, $\Gk$ and $\Gs$ are rich.
%%
\vplatz%%
\exercise%%
Let $\auf \{A_{\alpha} : \alpha \in \Typ_{\pf}(B)\}, %
\bullet\zu$ be a typed applicative structure. Now define the 
partial algebra $\auf A, \bullet\zu$ where $A :=
\bigcup_{\alpha} A_{\alpha}$. Show that if the applicative
structure is combinatorially complete, the type assignment is
unique up to permutation of the elements of $B$. Show also that if
the applicative structure is not combinatorially complete,
uniqueness fails. {\it Hint.} First, establish the elements of
basic type, and then the elements of type $b \pf c$, where $b, c
\in C$ are basic. Now, an element of type $b \pf c$ can be applied 
to all and only the elements of type $c$. This allows
to define which elements have the same basic type. 
%%
\vplatz%%
\exercise%%
Let $V := \{\mbox{\tt p}\vec{\alpha} :
\vec{\alpha} \in \{\mbox{\tt 0}, \mbox{\tt 1}\}^{\ast}\}$. Denote
the set of all types of combinators that can be formed over the
set $V$ by $C$. Show that $C$ is exactly the set of
intuitionistically valid formulae, that is, the set of formulae
derivable in $\vdash^{\mathsf{H}}$.
%%

 \newcommand{\circplus}{\,\raisebox{.1em}{\footnotesize{$\oplus$}}\,}
\newcommand{\circminus}{\,\raisebox{.1em}{\footnotesize{$\ominus$}}\,}
%%%%
\section{The Syntactic Calculus of Ca\-te\-go\-ries}
\label{kap3-2}
%
%
%
Categorial grammars --- in contrast to phrase structure grammars ---
specify no special set of rules, but instead associate with
each lexical element a finite set of context schemata. These
context schemata can either be defined over strings or over
structure trees. The second approach is older and leads to the so
called Aj\-du\-kie\-wicz--Bar Hillel--Calculus ($\mathsf{AB}$),
%%%
\index{Ajdukiewicz, Kazimierz}%%
\index{Bar--Hillel, Yehoshua}%%
\index{Ajdukiewicz--Bar Hillel Calculus}%%
%%%
the first to the Lambek--Cal\-cu\-lus ($\mathsf{L}$).
%%%
\nocite{ajdukiewicz:konnexitaet}
\index{Lambek--Calculus}%%
\index{Lambek, Joachim}%%%
%%%
We present first the calculus $\mathsf{AB}$.

We assume  that all trees are strictly binary branching with
exception of the preterminal nodes. Hence, every node whose
daughter is not a leaf has exactly two daughters. The phrase
structure rule $X \pf Y Z$ licenses the expansion of the symbol 
$X$ to the sequence $YZ$.  In categorial grammar, the category 
$Y$ represents the set of trees whose root has label $Y$, and 
the rule says that trees with root label $Y$ and $Z$, respectively, 
may be composed to a tree with root $X$. The approach is therefore 
from bottom to top rather than top to bottom. The fact that a tree 
of the named kind may be composed is coded by the so called
%%%%
\index{category assignment}%%
%%%%
\textbf{category assignment}. To this end we first have to define
{\it categories}. Categories are simply terms over a signature.
If the set of proper function symbols is $M$ and the set of 
0--ary function symbols is $C$ we write $\Cat_M(C)$ 
rather than $\Tm_M(C)$ for the set of terms over this 
signature. The members are called \textbf{categories}
%%%%
\index{$\Cat_{\mbox{\smtt\tb}, \mbox{\smtt\tf}}(C)$}%%
\index{category}%%%%
%%%%
while members of $C$ are called \textbf{basic categories}. %%%
%%%
\index{category!basic}%%
%%%
In the AB--Calculus we have $M = \{\mbox{\mtt\tb}, %
\mbox{\mtt\tf}\}$. ($\mathsf{L}$ also has $\bullet$.) Categories 
are written in infix notation. So, we write {\mtt (a{\tf}b)} in 
place of {\mtt {\tf}ab}. Categories will be denoted by lower 
case Greek letters, basic categories by lower case Latin letters. 
If $C = \{\mbox{\mtt a}, \mbox{\mtt b}, \mbox{\mtt c}\}$ then 
{\mtt ((a{\tf}b){\tb}c)}, {\mtt (c{\tf}a)} are categories. 
Notice that we take the actual strings to be the categories. 
This convention will soon be relaxed. Then we also use left 
associative bracketing as with $\lambda$--terms. So, 
{\mtt a$/$b$/$c$/$b$/$a} will be short for
{\mtt ((((a{\tf}b){\tf}c){\tf}b){\tf}a)}. (Notice the change 
in font signals that the way the functor is written down has 
been changed.) The interpretation of categories in terms of 
trees is as follows.  A {\it tree\/} is understood to be an 
exhaustively ordered strictly binary branching tree with labels in 
$\Cat_{\mbox{\smtt\tb}, \mbox{\smtt\tf}}(C)$, which 
results from a constituent analysis. This means that nonterminal 
nodes branch exactly when
they are not preterminal. Otherwise they have a single daughter, 
whose label is an element of the alphabet. The labelling function 
$\ell$ must be correct in the sense of the following definition.
%%
\begin{equation}
\begin{array}{l}
\binbaum{\delta}{\gamma}{\mbox{\mtt ($\gamma${\tb}$\delta$)}}
\qquad
\binbaum{\delta}{\mbox{\mtt ($\delta${\tf}$\gamma$)}}{\gamma}
\end{array}
\end{equation}
%%
Call a tree \textbf{2--standard} if a node is at most binary branching,
and if it is nonbranching iff it is preterminal.
%%
\begin{defn}
Let $A$ be an alphabet and $\zeta \colon A_{\varepsilon} \pf
\wp(\Cat_{\mbox{\smtt\tb}, \mbox{\smtt\tf}}(C))$ be 
a function for which $\zeta(a)$ is always finite.  Then $\zeta$ 
is called a \textbf{category assignment}.
%%%%
\index{category assignment}%%
%%%
Let $\GT = \auf T, <, \sqsubset, t\zu$ be a 2--standard tree
with labels in $\Cat_{\mbox{\smtt\tb}, \mbox{\smtt\tf}}(C)$.
$\GT$ is \textbf{correctly $\zeta$--labelled} %%
%%%
\index{tree!correctly labelled}%%
%%%
if (1) for every nonbranching $x$ with daughter $y$
$\ell(x) \in \zeta(\ell(y))$, and (2) for every branching $x$
which immediately dominates $y_0$, $y_1$ and $y_0 \sqsubset y_1$
we have: $\ell(y_0) = \mbox{\mtt ($\ell(x)${\tf}$\ell(y_1)$)}$ or 
$\ell(y_1) = \mbox{\mtt ($\ell(y_0)${\tb}$\ell(x)$)}$.
\end{defn}
%%
\begin{defn}
%%%%
\index{categorial grammar!AB--\faul}%%
%%%%
The quadruple $K = \auf S, C, A, \zeta\zu$ is an 
\textbf{AB}--\textbf{grammar} if $A$ and $C$ are finite sets, the 
\textbf{alphabet} and the set of \textbf{basic categories},
respectively, $S \in C$, and $\zeta  \colon A \pf \wp(\Cat_{\mbox{\smtt\tb}, 
\mbox{\smtt\tf}}(C))$ a category assignment. The set of labelled 
trees that is accepted by $K$ is denoted by $L_B(K)$.
It is the set of 2--standard correctly $\zeta$--labelled trees
with labelling $\ell \colon T \pf \Cat_{\mbox{\smtt\tb}, 
\mbox{\smtt\tf}}(C)$ such that the root carries the label $S$.
\end{defn}
%%
We emphasize that for technical reasons also the empty string
must be assigned a category. Otherwise no language which contains
the empty string is a language accepted by a categorial grammar.
We shall ignore this case in the sequel, but in the exercises
will shed more light on it.

AB--grammars only allow to define the mapping
$\zeta$. For given $\zeta$, the set of trees that are correctly
$\zeta$--labelled are then determined and can be enumerated. To 
this end we need to simply enumerate all possible constituents. 
Then for each preterminal $x$ we choose an appropriate label $\gamma 
\in \zeta(\ell(y))$, where $y \prec x$. The labelling function
therefore is fixed on all other nodes. In other words, the 
AB--grammars (which will turn out to be variants of CFGs) 
are invertible. The algorithm for finding analysis trees is not very 
effective. However, despite this we can show that already a CFG generates 
all trees, which allows us to import the results on CFGs.
%%
\begin{thm}
Let $K = \auf \mbox{\tt S}, C, A, \zeta\zu$ be an AB--grammar. 
Then there exists a CFG $G$ such that $L_B(K) = L_B(G)$.
\end{thm}
%%
\proofbeg
Let $N$ be the set of all subterms of terms in $\zeta(a)$,
$a \in A$. $N$ is clearly finite. It can be seen without problem 
that every correctly labelled tree only carries labels from $N$. 
The start symbol is that of $K$. The rules have the form
%%
\begin{align}
\gamma & \pf \mbox{\mtt ($\gamma${\tf}$\delta$)} \quad \delta & \\
\gamma & \pf \delta \quad \mbox{\mtt ($\delta${\tb}$\gamma$)} & \\
\gamma & \pf a           & (\gamma \in \zeta(a))
\end{align}
%%
where $\gamma$, $\delta$ run through all symbols of $N$ and $a$ 
through all symbols from $A$. This defines $G := \auf \mbox{\tt S}, %
N, A, R\zu$. If $\GT \in L_B(G)$ then the labelling is correct, as 
is easily seen. Conversely, if $\GT \in L_B(K)$ then every local tree 
is an instance of a rule from $G$, the root carries the symbol
{\tt S}, and all leaves carry a terminal symbol. Hence $\GT \in L_B(G)$.
\proofend

%%
Conversely every CFG can be converted into an AB--grammar; 
however, these two grammars need not be strongly equivalent.
Given $L$, there exists a grammar $G$ in Greibach Normal Form such that 
$L(G) = L$. We distinguish two cases. Case 1. $\varepsilon \in L$.  
We assume that {\tt S} is never on the right hand side of a 
production. (This can be installed keeping to Greibach Normal Form; 
see the exercises.) Then we choose a category assignment as in 
Case 2 and add $\zeta(\varepsilon) := \{\mbox{\tt S}\}$. Case 2.
$\varepsilon \not\in L$. Now define
%%
\begin{equation}
\zeta_G(a) := \{X/Y_{n-1}/\dotsb/Y_1/Y_0
: X \pf a \conc \prod_{i < n} Y_i \in R\}
\end{equation}
%%
Put $K := \auf \mbox{\tt S}, N_G, A, \zeta_G\zu$. We claim that
$L(K) = L(G)$. To this end we shall transform $G$
by replacing the rules $\rho = X \pf a \conc \prod_{i < n} Y_i$
by the rules
%%
\begin{equation}
Z^{\rho}_0 \pf a Y_0, \quad Z^{\rho}_1 \pf Z_0 Y_1,
    \quad \dotsc,
        \quad Z^{\rho}_{n-1} \pf Y_{n-2} Y_{n-1}
\end{equation}
%%
This defines the grammar $H$. We have $L(H) = L(G)$. Hence
it suffices to show that $L(K) = L(H)$. In place of $K$
we can also take a CFG $F$; the nonterminals
are $N_F$. We show now that that $F$ and $H$ generate the same
trees modulo the R--simulation $\sim\, \subseteq\, N_H \times N_F$,
which is defined as follows. (a) For $X \in N_G$ we have
$X \sim Y$ iff $X = Y$. (b) $Z^{\rho}_i \sim W$
iff $W = X/Y_{n-1}/\dotsb/Y_{i+1}$ and
$\rho = X \pf Y_0 \conc Y_1 \conc \dotsm \conc Y_{n-1}$
for certain $Y_j$, $i < j < n$. To this end it suffices
to show that the rules of $F$ correspond via $\sim$ to the
rules of $H$. This is directly calculated.
%%
\begin{thm}[Bar--Hillel \& Gaifman \& Shamir]
%%%
\index{Bar--Hillel, Yehoshua}%%
\index{Gaifman, Haim}%%%
\index{Shamir, E.}%%%
%%%
Let $L$ be a language. $L$ is context free iff $L = L_B(K)$ for 
some AB--grammar.
\proofend
\end{thm}
%%
Notice that we have used only {\mtt\tf}. It is easy to see that 
{\mtt\tb} alone would also have sufficed.

Now we look at Categorial Grammar from the standpoint of
the sign grammars. We introduce a binary operation `$\cdot$'
on the set of categories which satisfies the following equations.
%%
\begin{equation}
\mbox{\mtt ($\gamma${\tf}$\delta$)} \cdot \delta = \gamma, \qquad
\delta \cdot \mbox{\mtt ($\delta${\tb}$\gamma$)} = \gamma
\end{equation}
%%
Hence $\delta \cdot \eta$ is defined only when
$\eta = \mbox{\mtt ($\delta${\tb}$\gamma$)}$ or 
$\delta = \mbox{\mtt ($\gamma${\tf}$\eta$)}$
for some $\gamma$. Now let us look at the construction of a
sign algebra for CFGs of Section~\ref{kap3}.\ref{kap3-1}.
Because of the results of this section we can assume that
the set $T'$ is a subset of $\Cat_{\mbox{\smtt\tb}, %
\mbox{\smtt\tf}}(C)$ which is closed under $\cdot$. Then for 
our proper  modes we may proceed as follows. If $a$ is of 
category $\gamma$ then there exists a context free rule 
$\rho = \gamma \pf a$ and we introduce a 0--ary mode 
$\mbox{\tt R}_{\rho} := \auf a, \gamma, a\zu$. The other rules 
can be condensed into a single mode
%%
\begin{equation}
\mbox{\tt A}(\auf \vec{x}, \gamma, \vec{x}\zu,
    \auf \vec{y}, \beta, \vec{y}\zu) :=
    \auf \vec{x}\, \vec{y}, \gamma \cdot \beta,
    \vec{x}\, \vec{y}\zu
\end{equation}
    %%
(Notice that {\mtt A} is actually a structure term, so should 
actually write $\upsilon(\mbox{\mtt A})$ is place of it. We will 
not do so, however, to avoid clumsy notation.)

However, this still does not generate the intended meanings.
We still have to introduce $\mbox{\tt S}^{\heartsuit}$
as in Section~\ref{kap3}.\ref{kap3-1}. We do not want to do this,
however. Instead we shall deal with the question whether
one can generate the meanings in a more systematic fashion.
In general this is not possible, for we have only assumed
that $f$ is computable. However, in practice it appears
that the syntactic categories are in close connection to the
meanings. This is the philosophy behind Montague Semantics.
%%%
\index{Montague Semantics}%%%

Let an arbitrary set $C$ of basic categories be given. Further,
let a set $B$ of basic types be given. From $B$ we can form types 
in the sense of the typed $\lambda$--calculus and from $C$ categories 
in the sense of categorial grammar. We shall require that these 
two are connected by a homomorphism from the algebra of 
categories to the algebra of types. Both are realized over
strings. So, for each basic category $c \in C$ we choose a type 
$\gamma_c$. Then we put
%%
\begin{align}
%\begin{array}{l@{\quad := \quad}l}
\notag
\sigma(c) & := \gamma_c \\
\sigma(\mbox{\mtt ($\gamma${\tf}$\delta$)}) & 
	:= \mbox{\mtt ($\sigma(\delta)$\symbol{25}$\sigma(\gamma)$)} \\\notag
\sigma(\mbox{\mtt ($\delta${\tb}$\gamma$)}) & 
	:= \mbox{\mtt ($\sigma(\delta)$\symbol{25}$\sigma(\gamma)$)}
\end{align}
%%
Let now $\GA = \auf \{A_{\alpha} \colon \alpha \in 
\Typ_{\mbox{\smtt\symbol{25}}}(B)\}, \bullet\zu$ be a typed 
applicative structure. $\sigma$ defines a \textbf{realization of} 
$B$ in $\GA$ by assigning to each category $\gamma$ the set 
$A_{\sigma(\gamma)}$, which we also denote by $\real{\gamma}$. 
%%%
\index{$\real{\gamma}$}%%%
%%%
We demonstrate this with our arithmetical terms. The applicative 
structure shall be based on sets, using $\mathsf{app}$ as the interpretation 
of function application. This means that $A_{\mbox{\smtt (}\alpha
\mbox{\smtt\symbol{25}}\beta\mbox{\smtt )}} = A_{\alpha} \pf A_{\beta}$. 
Consequently, $\real{\mbox{\mtt ($\gamma${\tf}$\delta$)}}  
= \real{\mbox{\mtt ($\delta$\tb$\gamma$)}} = \real{\delta} \pf 
\real{\gamma}$. There is the basic category {\tt Z}, and it is 
realized by the set of numbers from $0$ to $9$. Further, there 
is the category {\tt T} which gets realized by the rational 
numbers $\BQ$ --- for example.
%%
\begin{align}
\real{\mbox{\tt Z}} & := \{0, 1, \dotsc, 9\} \\\notag
\real{\mbox{\tt T}} & := \BQ 
\end{align}
%%
$+ \colon \BQ \times \BQ \pf \BQ$ is a binary function. We  can
redefine it as shown in Section~\ref{kap3}.\ref{kap3-7} to an element of
$\BQ \pf (\BQ \pf \BQ)$, which we also denote by $+$. The syntactic 
category which we assign to $+$ has to match this. We choose
{\mtt ((T{\tb}T){\tf}T)}.
Now we have
%%
\begin{equation}
\real{\mbox{\mtt ((T{\tb}T)/T)}}
= \BQ \pf (\BQ \pf \BQ) 
\end{equation}
%%
as desired. Now we have to see to it that the meaning of the
string $\mbox{\tt 5+7}$ is indeed 12. To this end we require
that if {\tt +}  is combined with {\tt 7} to the constituent
{\tt +7} the meaning of {\tt +} (which is a function) is applied
to the number $7$. So, the meaning of {\tt +7} is the function
$x \mapsto x + 7$ on $\BQ$. If we finally group {\tt +7} and
{\tt 5} together to a constituent then we get a constituent
of category {\tt T} whose meaning is 12.

If things are arranged in this way we can uniformly define two
modes for $\mathsf{AB}$, $\mbox{\tt A}_{\sgr}$ and
$\mbox{\tt A}_{\skl}$.
%%%
\index{$\mbox{\tt A}_{\sgr}$, $\mbox{\tt A}_{\skl}$}%%%
%%
\begin{subequations}
\begin{align}
\mbox{\tt A}_{\sgr}(\auf \vec{x}, \alpha, M\zu,
    \auf \vec{y}, \beta, N\zu)
    & := \auf \vec{x}\, \vec{y}, \alpha \cdot \beta,
    MN\zu \\
%%
\mbox{\tt A}_{\skl}(\auf \vec{x}, \alpha, M\zu,
    \auf \vec{y}, \beta, N\zu)
    & := \auf \vec{x}\, \vec{y}, \alpha \cdot \beta, NM\zu
    \end{align}
\end{subequations}
    %%
We further assume that if $a \in A$ has category $\alpha$
then there are only finitely many $M \in \real{\alpha}$ which are
meanings of $a$ of category $\alpha$. For each such meaning
$M$ we assume a 0--ary mode $\auf a, \alpha, M\zu$. Therewith
$\mathsf{AB}$ is completely standardized. In the respective 
algebras $\GZ$, $\GT$ and $\GM$ there is only one binary operation. 
In $\GZ$ it is the concatenation of two strings, in $\GT$ it is 
cancellation, and in $\GM$ function application. The variability 
is not to be found in the proper modes, only in the 0--ary 
modes, that is, the lexicon. Therefore one speaks of Categorial 
Grammar as a `lexical' theory; all information about the
language is in the lexicon.
%%
\begin{defn}
%%%
\index{sign grammar!AB--\faul}%%%
%%%
A sign grammar $\auf \GA, \varepsilon, \gamma, \mu\zu$ is called 
an \textbf{AB--sign grammar} if the signature consists of the two 
modes $\mbox{\tt A}_{\sgr}$ and $\mbox{\tt A}_{\skl}$  and finitely
many 0--ary modes $\mbox{\tt M}_i$, $i < n$ such that 
%%%
\begin{dingautolist}{192}
\item
$\mbox{\tt M}_i^{\upsilon} = \auf \vec{x}_i, \gamma_i, N_i\zu$, 
$i < n$,
\item
$\GZ = \auf A^{\ast}, \conc, \auf \vec{x}_i : i < n\zu\zu$,
\item
$\GT = \auf \Cat_{\mbox{\smtt\tb},\mbox{\smtt\tf}}(C), \cdot, 
\auf \gamma_i : i < n\zu\zu$ for some set $C$,
\item
$\GM = \auf \{M_{\alpha} : \alpha \in 
\Typ_{\pf}(B)\}, \bullet, \auf N_i : i < n\zu\zu$ 
is an expansion of a typed applicative structure by constants, 
\item
and $N_i \in M_{\sigma(\gamma_i)}$, $i <n$.
\end{dingautolist}
%%%
\end{defn}
%%
Notice that the algebra of meanings is partial and has as its 
unique operation function application. (This is not defined if 
the categories do not match.) As we shall see, the concept of 
a categorial grammar is somewhat restrictive with respect to 
the language generated (it has to be context free) and with 
respect to the categorial symbols, but it is not restrictive 
with respect to meanings. 

We shall give an example. We look at our
alphabet of ten digits. Every nonempty string over this
alphabet denotes a unique number, which we name by this
very sequence. For example, the sequence {\tt 721} denotes
the number 721, which in binary is {\tt 101101001} or
{\tt LOLLOLOOL}. We want to write an AB--grammar which
couples a string of digits with its number.  This is not as easy
as it appears at first sight. In order not to let the example
appear trivial we shall write a grammar
for binary numbers, with {\tt L} in place of 1 and {\tt O}
in place of $0$. To start, we need a category
{\tt Z} as in the example above. This category is realized by the
set of natural numbers. Every digit has the category {\tt Z}.
So, we have the following 0--ary modes.
%%
\begin{equation}
\mbox{\tt Z}_{\snull} := \auf \mbox{\tt O}, Z, 0\zu \qquad
\mbox{\tt Z}_{\seins} := \auf \mbox{\tt L}, Z, 1\zu 
\end{equation}
%%
Now we additionally agree that digits have the category {\mtt
Z{\tb}Z}. With this the number {\tt LOL} is analyzed in this way.
%%
\begin{equation}
\begin{array}{ccc}
\mbox{\tt L} & \mbox{\tt O} & \mbox{\tt L} \\
\mbox{\mtt Z} & \mbox{\mtt (Z{\tb}Z)} &
    \mbox{\mtt (Z{\tb}Z)} \\\cline{1-2}
    \multicolumn{2}{c}{\mbox{\mtt Z}} &
        \\\cline{1-3}
            \multicolumn{3}{c}{\mbox{\mtt Z}}
\end{array}
\end{equation}
%%
This means that digits are interpreted as functions
from $\omega$ to $\omega$. As one easily finds out these are the
functions $\lambda x_0.2x_0+k$, $k \in \{0,1\}$. Here $k$ must be 
the value of the digit. So, we additionally need the following 
zeroary modes.
%%
\begin{align}
\mbox{\tt M}_{\snull} & := \auf \mbox{\tt 0}, \mbox{\mtt (Z{\tb}Z)},
    \lambda x_0.2x_0\zu \\
\mbox{\tt M}_{\seins} & := \auf \mbox{\tt 1}, \mbox{\mtt (Z{\tb}Z)},
    \lambda x_0.2x_0+1\zu
\end{align}
%%
(Notice that we write $\lambda x_0.2x_0$ and not 
\mbox{\tt (\stlambda x$_{\snull}$.(2{\mtt\symbol{42}}x$_{\snull}$))}, 
since the latter is a string, while the former is actually a function 
in a particular algebra.) However, the grammar does not have the ideal form.
For every digit has two different meanings which do not need to
have anything to do with each other. For example, we could have
introduced the following mode in place of --- or even in addition 
to --- $\mbox{\tt M}_{\seins}$.
%%
\begin{equation}
\mbox{\tt M}_{\szwei} := \auf \mbox{\tt 0}, \mbox{\mtt (Z{\tb}Z)},
    \lambda x_0.2x_0+1\zu
\end{equation} 
%%
We can avoid this by introducing a second category symbol,
{\tt T}, which stands for a sequence of digits, while
{\tt Z} only stands for digits. In place of $\mbox{\tt M}_{\snull}$
we now define the empty modes $\mbox{\tt N}_{\snull}$,
and $\mbox{\tt N}_{\seins}$:
%%
\begin{align}
\mbox{\tt N}_{\snull} & := \auf \varepsilon, \mbox{\mtt (T{\tf}Z)},
    \lambda x_0.x_0\zu \\
\mbox{\tt N}_{\seins} & := \auf \varepsilon, \mbox{\mtt ((T{\tf}T){\tf}Z)},
    \lambda x_1.\lambda x_0.2x_1+x_0\zu
\end{align}
%%
For example, we get {\tt LOL} as the exponent of the term
%%
\begin{equation}
\mbox{\mtt A$_{\sgr}$A$_{\sgr}$N$_{\seins}$A$_{\sgr}$A$_{\sgr}$N%
$_{\seins}$A$_{\sgr}$N$_{\snull}$Z$_{\seins}$Z$_{\snull}$Z$_{\seins}$} 
\end{equation}
%%
The meaning of this term is calculated as follows.
%%
\begin{align}
\begin{split}
& (\mbox{\mtt A$_{\sgr}$A$_{\sgr}$N$_{\seins}$A$_{\sgr}$A$_{\sgr}$N%
$_{\seins}$A$_{\sgr}$N$_{\snull}$Z$_{\seins}$Z$_{\snull}$Z$_{\seins}$}% 
)^{\mu} \\
	= & \mbox{\tt N}_{\seins}^{\mu}(\mbox{\tt N}_{\seins}^{\mu}(%
    \mbox{\tt N}_{\snull}^{\mu}(\mbox{\tt Z}_{\seins}^{\mu})) %
	(\mbox{\tt Z}_{\snull}^{\mu}))(\mbox{\tt Z}_{\seins}^{\mu}) \\
    = &
    \mbox{\tt N}_{\seins}^{\mu}(\mbox{\tt N}_{\seins}^{\mu}((%
    \mbox{\tt N}_{\snull}^{\mu}(1)) (0))(1) \\
    = & \mbox{\tt N}_{\seins}^{\mu}(\mbox{\tt N}_{\seins}^{\mu}(%
    (\lambda x_0.x_0)(1))(0))(1) \\
    = & \mbox{\tt N}_{\seins}^{\mu}(\mbox{\tt N}_{\seins}^{\mu}(1)(0))(1) \\
    = & \mbox{\tt N}_{\seins}^{\mu}((\lambda x_1.\lambda x_0.(2x_1+x_0
    ))(1)(0))(1) \\
    = & \mbox{\tt N}_{\seins}^{\mu}(2)(1) \\
    = & (\lambda x_1.\lambda x_0.(2x_1+x_0))(2)(1) \\
    = & 5
\end{split}
\end{align}
%%
This solution is far more elegant than the first. Despite of
this, it too is not satisfactory. We had to postulate
additional modes which one cannot see on the string. Also, 
we needed to distinguish strings from digits. For 
comparison we show a solution that involves restricting the 
concatenation function. Put 
%%%
\begin{equation}
\vec{x} \star \vec{y} := 
\begin{cases} 
\vec{x}\conc \vec{y} & \text{if $\vec{y} \in A$,} \\
\text{undefined} & \text{otherwise.}
\end{cases}
\end{equation}
%%%
Now take a binary symbol {\tt P} and set 
%%%
\begin{equation}
\mbox{\tt P}(\auf \vec{x}, \mbox{\tt Z}, m\zu, 
\auf \vec{y}, \mbox{\tt Z}, n\zu) = 
\auf \vec{x} \star \vec{y}, \mbox{\tt Z}, 2m+n\zu 
\end{equation}
%%%
One could also define two unary modes for appending a digit. 
But this would mean making the empty string an exponent for 0, 
or else it requires another set of two digits to get started. 
A further problem is the restricted
functionality in the realm of strings. With the example
of the grammar $T$ of the previous section we shall exemplify
this. We have agreed that every term is enclosed by brackets,
which merely are devices to help the eye. These brackets are
now symbols of the alphabet, but void of real meaning.
To place the brackets correctly, some effort must be made.
We propose the following grammar.
%%
\begin{align}
\notag
\mbox{\tt O}_{\seins} & := \auf \mbox{\mtt +}, \mbox{\mtt ((T{\tb}U){\tf}T)},
    \lambda x_1.\lambda x_0.x_0+x_1 \zu\\
\notag
\mbox{\tt O}_{\szwei} & := \auf \mbox{\mtt -}, \mbox{\mtt ((T{\tb}U){\tf}T)},
    \lambda x_1.\lambda x_0.x_0-x_1 \zu \\
\notag
\mbox{\tt O}_{\sdrei} & := \auf \mbox{\mtt \symbol{47}}, 
	\mbox{\mtt ((T{\tb}U){\tf}T)},
    \lambda x_1.\lambda x_0.x_0/x_1\zu \\
\notag
\mbox{\tt O}_{\svier} & := \auf \mbox{\mtt \symbol{42}}, 
	\mbox{\mtt ((T{\tb}U){\tf}T)},
    \lambda x_1.\lambda x_0.x_0x_1\zu \\
\mbox{\tt O}_{\sfuenf} & := \auf \mbox{\mtt -}, (\mbox{\mtt U{\tf}T)},
    \lambda x_0.-x_0 \zu \\
\notag
\mbox{\tt O}_{\ssechs} & := \auf \mbox{\tt (}, \mbox{\mtt (L{\tf}U)},
    \lambda x_0.x_0 \zu \\
\notag
\mbox{\tt O}_{\ssieben} & := \auf \mbox{\tt )}, \mbox{\mtt (L{\tb}T)},
    \lambda x_0.x_0\zu \\
\notag
\mbox{\tt Z}_{\snull} & := \auf \mbox{\mtt L}, \mbox{\mtt T}, 0\zu \\
\notag
\mbox{\tt Z}_{\seins} & := \auf \mbox{\mtt O}, \mbox{\mtt T}, 1\zu
\end{align}
%%
The conception is that an operation symbol generates an
unbracketed term which needs a left and a right bracket
to become a `real' term. A semantics that fits with this
analysis will assign the identity to all these. We simply
take $\BQ$ for all basic categories. The brackets are interpreted
by the identity function. If we add a bracket, nothing happens
to the value of the term. This is a viable solution. However,
it amplifies the set of basic categories without any increase in
semantic types as well.

The application of a function to an argument is by far not the
only possible rule of composition. In particular Peter Geach has
proposed in \cite{geach:program} to admit further rules of
combination. This idea has been realized on the one hand in
the Lambek--Calculus, which we will study later, and also in
\textbf{combinatory categorial grammars}.
%%
\index{combinatory categorial grammar}%%
%%
The idea to the latter is as follows. Each mode in Categorial
Grammar is interpreted by a semantical typed combinator.
For example, $\mbox{\tt A}_{\skl}$ acts on the semantics like the
combinator $\mathsf{U}$ (defined in Section~\ref{kap3}.\ref{kap3-7}) and
$\mbox{\tt A}_{\sgr}$ is interpreted by the combinator {\tt I}.
This choice of combinators is --- seen from the standpoint of
combinatory logic --- only one of many possible choices. Let
us look at other possibilities. We could add to the ones we
have also the functions corresponding to the following closed 
$\lambda$--term.
%%
\begin{equation}
\mathsf{B} := \mbox{\tt (\stlambda x}_{\snull}\mbox{\tt .(%
\stlambda x}_{\seins}\mbox{\tt .(\stlambda x}_{\szwei}\mbox{\tt %
.(x}_{\snull}\mbox{\tt (x}_{\seins}\mbox{\tt x}_{\szwei}\mbox{\tt )))))} 
\end{equation}
%%
$\mathsf{B}MN$ is nothing but function composition of the functions
$M$ and $N$. For evidently, if $\mbox{\tt x}_{\szwei}$ has type $\gamma$
then $\mbox{\tt x}_{\seins}$ must have the type $\beta \pf \gamma$ for 
some $\beta$ and $\mbox{\tt x}_{\snull}$ the type $\alpha \pf \beta$ for 
some $\alpha$. Then $\mathsf{B}\mbox{\tt x}_{\snull}\mbox{\tt x}_1 
\triangleright \mbox{\tt (\stlambda x}_{\szwei}\mbox{\tt .(x}_{\snull}%
\mbox{\tt (x}_{\seins}\mbox{\tt x}_{\szwei}\mbox{\tt )))}$ is of type 
$\alpha \pf \gamma$. Notice that for each $\alpha$, $\beta$ and
$\gamma$ we have a typed $\lambda$--term
$\mathsf{B}_{\alpha,\beta,\gamma}$.
%%
\begin{equation}
\mathsf{B}_{\alpha,\beta,\gamma} :=
\mbox{\tt (\stlambda x}^{\snull}_{\alpha\pf\beta}\mbox{\tt .(\stlambda %
x}^{\seins}_{\beta\pf\gamma}\mbox{\tt .(\stlambda x}^{\szwei}_{\alpha}%
\mbox{\tt .(x}^{\snull}_{\alpha\pf\beta}\mbox{\tt %
(x}^{\seins}_{\beta\pf\gamma}\mbox{\tt x}^{\szwei}_{\gamma}\mbox{\tt )))))} 
\end{equation}
%%
However, as we have explained earlier, we shall not use the
explicitly typed terms, but rather resort to the implicitly
typed terms (or combinators). We define two new category 
%%%
\index{$\circplus$, $\circminus$}%%
%%%
products $\circplus$ and $\circminus$ by
%%
\begin{subequations}
\begin{align}
\mbox{\mtt ($\gamma${\tf}$\beta$)}   
	\circplus \mbox{\mtt ($\beta${\tf}$\alpha$)} & 
    := \mbox{\mtt ($\gamma${\tf}$\alpha$)} \\
\mbox{\mtt ($\beta${\tf}$\alpha$)}  
	\circminus \mbox{\mtt ($\beta${\tb}$\gamma$)} &
    := \mbox{\mtt ($\gamma${\tf}$\alpha$)} \\
\mbox{\mtt ($\gamma${\tf}$\beta$)}  
	\circplus \mbox{\mtt ($\alpha${\tb}$\beta$)} &
    := \mbox{\mtt ($\alpha${\tb}$\gamma$)} \\
\mbox{\mtt ($\alpha${\tb}$\beta$)}  
	\circminus \mbox{\mtt ($\beta${\tb}$\gamma$)} &
    := \mbox{\mtt ($\alpha${\tb}$\gamma$)}
\end{align}
\end{subequations}
%%
Further, we define two new modes, $\mbox{\tt B}_{\sgr}$ and
$\mbox{\tt B}_{\skl}$, as follows:
%%
\begin{align}
\mbox{\tt B}_{\sgr}(\auf \vec{x},\alpha,M\zu,
\auf \vec{y},\beta,N\zu) & := 
\auf \vec{x}\conc\vec{y}, \alpha \circplus \beta, \mathsf{B}MN\zu \\
\mbox{\tt B}_{\skl}(\auf \vec{x},\alpha,M\zu,
\auf \vec{y},\beta,N\zu) & := 
\auf \vec{x}\conc\vec{y}, \alpha \circminus \beta, \mathsf{B}NM\zu
\end{align}
%%
Here, it is not required that the type of $M$ matches $\alpha$ in
any way, or the type of $N$ the category $\beta$.
In place of $\mathsf{B}NM$ we could
have used $\mathsf{V}MN$, where
%%
\begin{equation}
\mathsf{V} :=
\mbox{\tt (\stlambda x}_{\snull}\mbox{\tt .(\stlambda x}_{\seins}%
\mbox{\tt .(\stlambda x}_{\szwei}\mbox{\tt .(x}_{\seins}\mbox{\tt %
(x}_{\snull}\mbox{\tt x}_{\szwei}\mbox{\tt )))))}
\end{equation}
%%
We denote by $\CCG(\mathsf{B})$ the extension of $\mathsf{AB}$ by
the implicitly typed combinator $\mathsf{B}$.
%%%
\index{$\CCG(\mathsf{B})$}%%
%%
This grammar not only has the modes $\mbox{\tt A}_{\sgr}$ and
$\mbox{\tt A}_{\skl}$ but also the modes $\mbox{\tt B}_{\sgr}$ and
$\mbox{\tt B}_{\skl}$. The resulting tree sets are however of a new
kind. For now, if $x$ is branching with daughters $y_0$ and $y_1$,
$x$ can have the category $\alpha/\gamma$ if $y_0$ has the
category $\alpha/\beta$ and $y_1$ the category $\beta/\gamma$. In
the definition of the products $\circplus$ and $\circminus$ there 
is a certain arbitrariness. What we must expect from the semantic 
typing regime is that the type of $\sigma(\alpha \circplus \beta)$ 
and $\sigma(\beta \circminus \alpha)$ 
equals $\eta \pf \theta$ if $\sigma(\alpha) = \zeta \pf \theta$ 
and $\sigma(\beta) = \eta \pf \zeta$ for some $\eta$, $\zeta$ and 
$\theta$.  Everywhere else the syntactic product should be undefined.
However, in fact the syntactic product has been symmetrified, and
the directions specified. This goes as follows. By applying a rule
a category (here $\zeta$) is cancelled. In the category
$\eta/\theta$ the directionality (here: right) is viewed as a
property of the argument, hence of $\theta$. If $\theta$ is not
cancelled, we must find $\theta$ being selected to the right
again. If, however, it is cancelled from $\eta/\theta$, then the
latter must be to the left of its argument, which contains some
occurrence of $\theta$ (as a result, not as an argument). This
yields the rules as given. We leave it to the reader to show that
the tree sets that can be generated from an initial category
assignment $\zeta$ are again all context free. Hence, not much
seems to have been gained. We shall next study another extension,
$\CCG(\mathsf{P})$. Here
%%
\begin{equation}
\mathsf{P} := \mbox{\tt (\stlambda x}_{\snull}\mbox{\tt .(\stlambda
x}_{\seins}\mbox{\tt .(\stlambda x}_{\szwei}\mbox{\tt .(\stlambda %
x}_{\sdrei}\mbox{\tt .(x}_{\snull}\mbox{\tt (x}_{\seins}\mbox{\tt %
x}_{\szwei}\mbox{\tt )x}_{\sdrei}\mbox{\tt )))))} 
\end{equation}
%%
In order for this to be properly typed we may freely choose
the type of $\mbox{\tt x}_{\szwei}$ and $\mbox{\tt x}_{\sdrei}$, 
say $\beta$ and $\gamma$. Then $\mbox{\tt x}_{\seins}$ is of type
$\gamma \pf (\beta \pf \delta)$ for some $\delta$ and
$\mbox{\tt x}_{\snull}$ of type $\delta \pf \alpha$ for some
$\alpha$. $\mbox{\tt x}_{\seins}$ stands for an at least binary function,
$\mbox{\tt x}_{\snull}$ for a function that needs at least one argument.
If the combinator is defined, the mode is fixed if we
additionally fix the syntactic combinatorics. To this end
%%%
\index{$\gtrdot$, $\lessdot$}%%
%%%
we define the products $\gtrdot$, $\lessdot$ as in 
Table~\ref{tab:bullet}.
%%
\begin{table}
\caption{The Products $\gtrdot$ and $\lessdot$}
\label{tab:bullet}
$$\begin{array}{l@{\quad}c@{\quad}l@{\quad := \quad}l}
\mbox{\mtt ($\alpha$\tf$\delta$)} & \gtrdot
	 & \mbox{\mtt (($\delta$\tf$\beta$)\tf$\gamma$)} 
	& \mbox{\mtt (($\alpha$\tf$\beta$)\tf$\gamma$)} \\
\mbox{\mtt (($\delta$\tf$\beta$)\tf$\gamma$)} & \lessdot
	& \mbox{\mtt ($\delta$\tb$\alpha$)} 
        & \mbox{\mtt (($\alpha$\tf$\beta$)\tf$\gamma$)} \\
\mbox{\mtt ($\alpha$\tf$\delta$)} & \gtrdot
	& \mbox{\mtt (($\beta$\tb$\delta$)\tf$\gamma$)} 
        & \mbox{\mtt (($\beta$\tb$\alpha$)\tf$\gamma$)} \\
\mbox{\mtt (($\beta$\tb$\delta$)\tf$\gamma$)} & \lessdot
	& \mbox{\mtt ($\delta$\tb$\alpha$)}
        & \mbox{\mtt (($\beta$\tb$\alpha$)\tf$\gamma$)} \\
\mbox{\mtt ($\alpha$\tf$\delta$)} & \gtrdot
	& \mbox{\mtt ($\gamma${\tb}($\delta$\tf$\beta$))}
        & \mbox{\mtt ($\gamma${\tb}($\alpha$\tf$\beta$)}) \\
\mbox{\mtt ($\gamma${\tb}($\delta$\tf$\beta$))} & \lessdot
	& \mbox{\mtt ($\delta$\tb$\alpha$)}
        & \mbox{\mtt ($\gamma${\tb}($\alpha$\tf$\beta$))} \\
\mbox{\mtt ($\alpha$\tf$\delta$)}  & \gtrdot
	& \mbox{\mtt ($\gamma${\tb}($\beta$\tb$\delta$))}
        & \mbox{\mtt ($\gamma${\tb}($\beta$\tb$\alpha$))} \\
\mbox{\mtt ($\gamma${\tb}($\beta$\tb$\delta$))} & \lessdot
	& \mbox{\mtt ($\delta$\tb$\alpha$)}
        & \mbox{\mtt ($\gamma${\tb}($\beta$\tb$\alpha$))} 
\end{array}$$
\end{table}
%%
Now we define the following new modes:
%%
\begin{align}
\mbox{\tt P}_{\sgr}(\auf \vec{x},\alpha,M\zu,
\auf \vec{y},\beta,N\zu) & := 
\auf \vec{x}\conc\vec{y}, \alpha \gtrdot \beta, \mathsf{P}MN\zu \\
\mbox{\tt P}_{\skl}(\auf \vec{x},\alpha,M\zu,
\auf \vec{y},\beta,N\zu) & := 
\auf \vec{x}\conc\vec{y}, \alpha \lessdot \beta, \mathsf{P}NM\zu
\end{align}
%%
We shall study this type of grammar somewhat closer. We take the
following modes.
%%
\begin{align}
\notag
\mbox{\tt M}_{\snull} & := \auf \mbox{\tt A}, \mbox{\mtt ((c{\tf}a){\tf}c)}, 
	\lambda x_0.\lambda x_1.x_0+x_1\zu \\
\notag
\mbox{\tt M}_{\seins} & := \auf \mbox{\tt B},\mbox{\mtt ((c{\tf}b){\tf}c)}, 
	\lambda x_0.\lambda x_1.x_0x_1\zu \\
\mbox{\tt M}_{\szwei} & := \auf \mbox{\tt a}, \mbox{\mtt a}, 1\zu \\
\notag
\mbox{\tt M}_{\sdrei} & := \auf \mbox{\tt b}, \mbox{\mtt b}, 2\zu \\
\notag
\mbox{\tt M}_{\svier} & := \auf \mbox{\tt C}, \mbox{\mtt (c{\tf}a)}, 
	\lambda x_0.x_0\zu
\end{align}
%%
Take the string {\tt ABACaaba}. It has two analyses, shown in 
Figure~\ref{fig:abac}. In both analyses the meaning is $5$. In 
the first analysis only the mode $\mbox{\tt A}_{\sgr}$ has been used. 
%%
\begin{figure}
$$\begin{array}{cccccccc}
\mbox{\tt A} & \mbox{\tt B} & \mbox{\tt A} & \mbox{\tt C} &
    \mbox{\tt a} & \mbox{\tt a}
    & \mbox{\tt b} & \mbox{\tt a} \\
\mbox{\mtt ((c{\tf}a){\tf}c)} & \mbox{\mtt ((c{\tf}b){\tf}c)} 
	& \mbox{\mtt ((c{\tf}a){\tf}c)} & \mbox{\mtt (c{\tf}a)} 
	& \mbox{\mtt a} & \mbox{\mtt a} & \mbox{\mtt b} 
	& \mbox{\mtt a} \\\cline{4-5}
\vdots  & \vdots & \mbox{\mtt ((c{\tf}a){\tf}c)} 
	& \multicolumn{2}{c}{\mbox{\mtt c}} & \vdots & \vdots
    & \vdots
        \\\cline{3-5}
\vdots  & \vdots  & \multicolumn{3}{c}{\mbox{\mtt (c{\tf}a)}} 
         & \mbox{\mtt a} & \vdots &
\vdots
        \\\cline{3-6}
\vdots  & \mbox{\mtt ((c{\tf}b){\tf}c)} &  & 
	\multicolumn{3}{c}{\mbox{\mtt c}}   & \vdots  & \vdots
        \\\cline{2-5}
\vdots  &         & \mbox{\mtt (c{\tf}b)} & & &  & \mbox{\mtt b} & \vdots
        \\\cline{3-7}
\mbox{\mtt ((c{\tf}a){\tf}c)} &  
	& \multicolumn{5}{c}{\mbox{\mtt c}}             & \vdots
\\\cline{1-5}
\multicolumn{6}{c}{\mbox{\mtt (c{\tf}a)}} & & \mbox{\mtt a} 
	\\\cline{2-8}
        &       
\multicolumn{7}{c}{\mbox{\mtt c}}
\end{array}$$
%%
$$\begin{array}{ccccccccc}
\mbox{\tt A} & \mbox{\tt B} & \mbox{\tt A} & & \mbox{\tt C} &
    \mbox{\tt a} & \mbox{\tt a}
    & \mbox{\tt b} & \mbox{\tt a} \\
\mbox{\mtt ((c/a)/c)} & \mbox{\mtt ((c/b)/c)} 
	& \mbox{\mtt ((c{\tf}a){\tf}c)} & & \mbox{\mtt (c{\tf}a)} 
	& \mbox{\mtt a} & \mbox{\mtt a} & \mbox{\mtt b} 
	& \mbox{\mtt a} \\\cline{2-3}
\vdots  & \multicolumn{2}{c}{\mbox{\mtt (((c{\tf}b){\tf}a){\tf}c)}} 
	& & \mbox{\mtt (c{\tf}a)} & \mbox{\mtt a}
    & \vdots & \vdots & \vdots
        \\\cline{1-3}\cline{5-6}
\multicolumn{3}{c}{\mbox{\mtt ((((c{\tf}a){\tf}b){\tf}a){\tf}c)}}   
	&  & \multicolumn{2}{c}{\mbox{\mtt c}} &
    \vdots & \vdots & \vdots
        \\\cline{1-6}
\multicolumn{6}{c}{\mbox{\mtt (((c{\tf}a){\tf}b){\tf}a)}}   
	& \mbox{\mtt a} & \vdots & \vdots
        \\\cline{2-7}
        &
\multicolumn{6}{c}{\mbox{\mtt ((c{\tf}a){\tf}b)}} & \mbox{\mtt b} 
	& \vdots \\\cline{3-8}
        & &
\multicolumn{6}{c}{\mbox{\mtt (c{\tf}a)}}             & \mbox{\mtt a}
        \\\cline{4-9}
        & & &
\multicolumn{6}{c}{\mbox{\mtt c}}
\end{array}$$
\caption{Two Analyses of {\tt ABACaaba}}
\label{fig:abac}
\end{figure}
%%
The second analysis uses the mode $\mbox{\tt P}_{\sgr}$. Notice that 
in the course of the derivation the categories get larger and larger 
(and therefore also the types).
%%
\begin{thm}
There exist $\CCG(\mathsf{P})$--grammars which generate non
context free tree sets.
\end{thm}
%%
We shall show that the grammar just defined is of this kind.
To this end we shall make a few more considerations.
%%
\begin{lem}
\label{lem:assoz}
%%
Let $\alpha = \eta_1/\eta_2/\eta_3$, $\beta =
\eta_3/\eta_4/\eta_5$ and $\gamma = \eta_5/\eta_6/\eta_7$. Then
%%
\begin{equation}
\alpha \gtrdot (\beta \gtrdot \gamma) =
(\alpha \gtrdot \beta) \gtrdot \gamma
\end{equation}
%%
\end{lem}
%%
\proofbeg
Proof by direct computation. For example,
$\alpha \gtrdot \beta = \eta_1/\eta_2/\eta_3/\eta_4/\eta_5$.
\proofend

%%
In particular, $\gtrdot$ is associative if defined
(in contrast to `$\cdot$'). Now, let us look at a string
of the form $\vec{x}\mbox{\tt Ca}\vec{y}$,
where $\vec{x} \in (\mbox{\tt A} \cup \mbox{\tt B})^{\ast}$,
$\vec{y} \in (\mbox{\tt a} \cup \mbox{\tt b})^{\ast}$
and $h(\vec{x})= {\vec{y}\,}^T$, where $h \colon \mbox{\tt A}
\mapsto \mbox{\tt a}, \mbox{\tt B} \mapsto \mbox{\tt b}$.
An example is the string {\tt AABACabaaa}. Then 
with the exception of $\vec{x}\mbox{\tt C}$ all prefixes
are constituents. For prefixes of $\vec{x}$ are constituents, 
as one can easily see. It follows easily that the tree sets 
are not context free. For if $\vec{x} \neq \vec{y}$ then 
$\vec{x}\mbox{\tt Ca}h({\vec{y}\,}^T)$ is not derivable. However, 
$\vec{x}\mbox{\tt Ca}h({\vec{x}\,}^T)$ is derivable. If the tree 
set was context free, there cannot be infinitely many such 
$\vec{x}$, a contradiction.

So, we have already surpassed the border of context freeness.
However, we can push this up still further. Let $\GN$ be the 
following grammar.
%%
\begin{align}
\notag
\mbox{\tt N}_{\snull} & := \auf \mbox{\tt A}, \mbox{\mtt (c{\tb}(c{\tf}a))},
    \lambda x_0.\lambda x_1.x_0+x_1\zu \\
\notag
\mbox{\tt N}_{\seins} & := \auf \mbox{\tt B}, \mbox{\mtt (c{\tb}(c/b))},
    \lambda x_0.\lambda x_1.x_0\cdot x_1 \zu \\
\mbox{\tt N}_{\szwei} & := \auf \mbox{\tt a}, \mbox{\mtt a}, 1\zu \\
\notag
\mbox{\tt N}_{\sdrei} & := \auf \mbox{\tt b}, \mbox{\mtt b}, 2\zu \\
\notag
\mbox{\tt N}_{\svier} & := \auf \mbox{\tt C}, \mbox{\mtt c}, 
	\lambda x_0.x_0\zu
\end{align}
%%
\begin{thm}
$\GN$ generates a non context free language.
\end{thm}
%%
\proofbeg
Let $L$ be the language generated by $\GN$. Put
$M := \mbox{\tt C}(\mbox{\tt A} \cup \mbox{\tt B})^{\ast}
(\mbox{\tt a}\cup \mbox{\tt b})^{\ast}$. If $L$ is context free,
so is $L \cap M$ (by Theorem~\ref{thm:cfintersekt}).
Define $h$ by $h(\mbox{\tt A}) := h(\mbox{\tt a}) := \mbox{\tt a}$,
$h(\mbox{\tt B}) := h(\mbox{\tt b}) := \mbox{\tt b}$ as well as
$h(\mbox{\tt C}) := \varepsilon$. We show:
%%
\begin{equation}
\label{eq:ast}
\vec{x} \in L \cap M \text{ iff (a) } \vec{x} \in L 
\text{ and (b) } h(\vec{x}) = \vec{y}\vec{y} \text{ for some } 
\vec{y} \in (\mbox{\tt a}\cup\mbox{\tt b})^{\ast}
\end{equation}
%%
Hence $h[L \cap M] = \{\vec{y}\, \vec{y} : \vec{y} \in
(\mbox{\tt a}\cup \mbox{\tt b})^{\ast}\}$. The latter is not context
free. From this follows by Theorem~\ref{thm:afl} that $L \cap M$
is not context free, hence $L$ is not context free either. Now for
the proof of \eqref{eq:ast}. If $\Delta = \auf \delta_i : i < n\zu$ 
then let $\mbox{\mtt c}/\Delta$ denote the category 
$\mbox{\mtt c}/\delta_0/\delta_1/ \dotsb/\delta_{n-1}$.
Then we have:
%%
\begin{equation}
\mbox{\mtt c}\backslash (\mbox{\mtt c}/\Delta_1) \gtrdot
\mbox{\mtt c}\backslash (\mbox{\mtt c}/\Delta_2) =
\mbox{\mtt c}\backslash (\mbox{\mtt c}/\Delta_2; \Delta_1)
\end{equation}
%%
Now let $\mbox{\tt C}\vec{x}\vec{y}$ be such that
$\vec{x} \in (\mbox{\tt A}\cup\mbox{\tt B})^{\ast}$ and
$\vec{y} \in (\mbox{\tt a}\cup\mbox{\tt b})^{\ast}$.
It is not hard to see that then $\mbox{\tt C}\vec{x}$ is a
constituent. (Basically, one can either multiply or apply. 
The complex categories cannot be applied to the right, they 
can only be applied to the left, so this can happen only with 
{\tt C}. If one applies {\mtt (c{\tb}(c{\tf}a))} to {\mtt c} 
one gets {\mtt (c{\tf}a)}, which cannot be multiplied by 
$\gtrdot$ with any other constituent formed. It 
cannot be applied either (assuming that the string is not 
{\tt CAa}, in which case {\tt CA} does become a constituent under 
this analysis), because nothing on the right of it has category 
{\tt a}. Now let $\vec{x} := x_0x_1\dotsb x_{n-1}$. 
Further, let $d_i := \mbox{\tt a}$ if $x_i = \mbox{\tt A}$ and 
$d_i := \mbox{\tt b}$ if $x_i = \mbox{\tt B}$, $i < n$. Then the 
category of $\vec{x}$ equals 
$\mbox{\mtt c}\backslash (\mbox{\mtt c}/\Delta)$ with 
$\Delta = \auf d_{n-i-1}: i < n\zu$. Hence $\mbox{\tt C}\vec{x}$ 
is a constituent of category $\mbox{\mtt c}/\Delta$. This 
means, however, that $y_0$ has the category $d_0$ (because 
$d_0$ is the last in the list hence the first to be discharged),
$y_1$ the category $d_1$ and so on. But if $y_i$ has the
category $d_i$ then $h(x_i) = y_i$, as is easily checked.
This yields that $h(\vec{x}) = \vec{y}$. If on the other
hand this is the case, the string is derivable.
\proofend

Hence we now have a grammar which generates a non context free
language. CCGs are therefore stronger than AB--grammars.

There is a still different way to introduce CCGs. There we do
not enlarge the set of combinatorial rules but instead introduce
empty modes.
%%
\begin{equation}
\begin{split}
\mbox{\tt B}_{\snull} & := \auf \varepsilon,
   \gamma/\alpha/(\gamma/\beta)/(\beta/\alpha),
   \mathsf{B} \zu \\
\mbox{\tt B}_{\seins} & := \auf \varepsilon,
   (\alpha\backslash \gamma)/(\gamma/\beta)/(\alpha\backslash\beta),
   \mathsf{B}\zu \\
\mbox{\tt B}_{\szwei} & := \auf \varepsilon,
   \gamma/\alpha/(\beta\backslash\gamma)/(\beta/\alpha),
   \mathsf{V}\zu \\
\mbox{\tt B}_{\sdrei} & := \auf \varepsilon,
   (\alpha\backslash \gamma)/(\beta\backslash%
   \gamma)/(\alpha\backslash\gamma),
   \mathsf{V}\zu
\end{split}
\end{equation}
%%
Here we do not have four but infinitely many modes, one for
each choice of $\alpha$, $\beta$ and $\gamma$. Only in this
way it is possible to generate non context free languages.
Lexical elements that have a parametric (= implicitly typed) 
set of categories (together with parametric meanings) are called 
%%%%
\index{polymorphism}%%
%%%%
\textbf{polymorphic}. Particularly interesting cases of polymorphic 
elements are the logical connectors, {\tt and} and {\tt not}. 
Syntactically, they have the category $(\alpha\backslash\alpha)/\alpha$ 
and $\alpha/\alpha$, respectively, where $\alpha$ can assume any
(non parametric) category. This means that two constituents
of identical category can be conjoined by {\tt and} to another
constituent of the same category, and every constituent can be
turned by {\tt not} to a constituent of identical category.

{\it Notes on this section.} Although we have said that the
meanings shall be functions in an applicative structure, we
sometimes put strings in their place. These strings only denote 
these functions. This is not an entirely harmless affair. For example,
the string $\mbox{\tt (\stlambda x}_{\snull}\mbox{\tt .x}_{\snull}%
\mbox{\tt +1)}$ and the string $\mbox{\tt (\stlambda x}_{\seins}%
\mbox{\tt .x}_{\seins}\mbox{\tt +1)}$ denote the same function. In 
fact, for reduced terms terms uniqueness
holds only up to renaming of bound variables. It is standard practice
in $\lambda$--calculus to consider $\lambda$--terms `up to renaming
of bound variables' (see \cite{pigozzisalibra:vb} for a discussion).
A possible remedy might be to use combinators. But here the same
problem arises. Different strings may denote the same function. This
is why normalisation becomes important. On the other hand, strings as
meanings have the advantage to be finite, and thus may function as
objects that can be stored (like codes of a Turing machine, see the
discussion of Section~\ref{kap6}.\ref{kap:feasibility}).
%%
\vplatz
\exercise
Let $\zeta \colon A_{\varepsilon} \pf \wp(\Cat_{\mbox{\smtt\tb}, %
\mbox{\smtt\tf}}(C))$ be a category assignment. Show that the correctly 
labelled trees form a context free tree set.
%%
\vplatz
\exercise
Show that for every CFG there exists a weakly equivalent grammar 
in Greibach Normal Form, where the start symbol {\tt S} does not occur 
on the right hand side of a production.
%%
\vplatz
\exercise
Let $\zeta \colon A_{\varepsilon} \pf \wp(\Cat_{\mbox{\smtt\tb}, %
\mbox{\smtt\tf}}(C))$ be a category assignment. Further, let {\tt S} 
be the distinguished category. $\zeta'$ is called \textbf{normal} if
$\zeta(\varepsilon) = \mbox{\tt S}$ and no $\zeta(a)$ contains
an $\alpha$ of the form $\gamma/\beta_0/\dotsb/\beta_{n-1}$
with $\beta_i = \mbox{\tt S}$ for some $i < n$. Show that
for any $\zeta$ there is a normal $\zeta'$ such that $\zeta'$ and
$\zeta$ have the same language.
%%
\vplatz
\exercise
Let $L \subseteq A^{\ast}$ be context free and $f \colon A^{\ast} \pf M$ 
a computable function. Write an AB--sign grammar whose 
interpreted language is $\{\auf \vec{x}, f(\vec{x})\zu : \vec{x} \in L\}$.
%%
\vplatz
\exercise
Let $\auf \GA, \varepsilon, \gamma,\mu\zu$ be an AB--sign 
grammar. Show for all signs $\auf \vec{x}, \alpha, M\zu$ generated 
by that grammar: $M$ has the type $\sigma(\alpha)$.  {\it Hint.} 
Induction on the length of the structure term.
%%
\vplatz 
\exercise 
\label{ueb:ab} 
Show that the $\CCG(\mathsf{B})$ grammars only generate context free 
string languages, even context free tree sets. {\it Hint.} Show the 
following: if $A$ is an arbitrary finite set of categories, then with 
$\mathsf{B}$ one can generate at most $|A|^n$ many categories.
%%%
\vplatz
\exercise
Suppose we defined a product $\circ$ on categories as follows. 
$\alpha \circ \beta$ is defined whenever (a) $\alpha 
\circplus \beta$ is defined (and has the same value), or 
(b) $\alpha \circminus \beta$ is defined (and has the same value). 
Show that this does not 
allow to unambiguously define the semantics. (Additional question: 
why does this problem not arise with $\cdot$?) {\it Hint.} Take 
$\alpha = \beta = \mbox{\mtt ($\gamma${\tf}$\gamma$)}$. 

 \section{The AB--Calculus}
\label{kap3-3}
%
%
%
We shall now present a calculus to derive all the valid
derivability statements for AB--grammars. 
Notice that the only variable element is the elementary category 
assignment. We choose an alphabet $A$ and an elementary category 
assignment $\zeta$. We write 
%%
\index{$[\alpha]_{\zeta}$, $[\alpha]$}%%
%%%
$[\alpha]_{\zeta}$ for the set of all
unlabelled binary constituent structures over $A$ that have
root category $\alpha$ under some correct $\zeta$--labelling.
As $\zeta$ is completely arbitrary, we shall deal here only 
with the constituent structures obtained by taking away the 
terminal nodes. This eliminates $\zeta$ and $A$, and leaves a class of
purely categorial structures, denoted by $[\alpha]$. Since 
AB--grammars are invertible, for any given constituent structure 
there exists at most one labelling function (with the exception of 
the terminal labels). Now we introduce a binary symbol $\circ$, 
%%%
\index{$\circ$}%%%
%%%
which takes as arguments correctly
$\zeta$--labelled constituent structures. Let $\auf X, \GX, \ell\zu$
and $\auf Y, \GY,m\zu$ such constituent structures
and $X \cap Y = \varnothing$. Then let
%%
\begin{align}
\auf X, \GX,\ell\zu \circ \auf Y, \GY,m\zu := &
        \auf X \cup Y, \GX \cup \GY \cup \{X \cup Y\},n\zu \\
n(z) := & \begin{cases}
\ell(z) & \text{if $z \in \GX$}, \\
m(z) & \text{if $z \in \GY$}, \\
\ell(X) \cdot m(Y) & \text{if $z = X \cup Y$.}
\end{cases}
\end{align}
%%
(In principle, $\circ$ is well defined also if the constituent
structures are not binary branching.) In case where
$X \cap Y \neq \varnothing$ one has to proceed to the disjoint
sum. We shall not spell out the details. With the help of
$\circ$ we shall form terms over $A$, that is, we form the
algebra freely generated by $A$ by means of $\circ$.
To every term we inductively associate a constituent
structure in the following way.
%%
\begin{subequations}
\begin{align}
\alpha^k & := \auf \{0\}, \{\{0\}\}, \auf \{0\}, \alpha\zu\zu \\
(s \circ t)^k & := s^k \circ t^k
\end{align}
\end{subequations}
%%
Notice that $\circ$ has been used with two meanings. Finally, 
we take a look at $[\alpha]$. It denotes classes of binary 
branching constituent structures over $A$. The following holds.
%%%
\begin{subequations}
\begin{align}
[\alpha/\beta] \circ [\beta] & 
	\subseteq [\alpha] \\
[\beta] \circ [\beta\backslash \alpha] &
	\subseteq [\alpha]
\end{align}
\end{subequations}
%%
We abstract now from $A$ and $\zeta$. In place of interpreting $\circ$
as a constructor for constituent structures over $A$ we now
interpret it as a constructor to form constituent structures
over $\Cat_{\mbox{\smtt\tb}, \mbox{\smtt\tf}}(C)$ for some given $C$.
We call a term from categories with the help of $\circ$ a 
\textbf{category complex}.
%%%%%
\index{category complex}%%
%%%%%
Categories are denoted by lower case Greek letters, category
complexes by upper case Greek letters. Inductively, we extend
the interpretation $[-]$ to structures as follows.
%%
\begin{equation}
[\Gamma \circ \Delta] := [\Gamma] \circ
[\Delta] 
\end{equation}
%%
Next we introduce yet another symbol, $\bvdash$. This is a
relation between structures and categories. If $\Gamma$ is
a structure and $\alpha$ a category then $\Gamma \bvdash \alpha$
denotes the fact that for every interpretation in some alphabet
$A$ with category assignment $\zeta$ $[\Gamma] \subseteq [\alpha]$. 
We call the object $\Gamma \bvdash \alpha$ a \textbf{sequent}. %%
%%%
\index{sequent}%%
%%%
The interpretation that we get in this way we call the
\textbf{cancellation interpretation}. %%%
%%%
\index{cancellation interpretation}%%
%%%
Here, categories are inserted as concrete labels which are assigned
to nodes and which are subject to the cancellation interpretation.

We shall now introduce two different calculi, one of which will
be shown to be adequate for the cancellation interpretation. In 
formulating the rules we use the following convention. $\Gamma[\alpha]$ 
above the line means in this connection that $\Gamma$ is a
category complex in which we have fixed a single occurrence of 
$\alpha$. When we write, for example, $\Gamma[\Delta]$ below the line,
then this denotes the result of replacing that occurrence
of $\alpha$ by $\Delta$.
%%
\begin{equation}
\begin{array}{l@{\quad}l@{\qquad}l@{\quad}l}
\mbox{\rm (ax)} & \alpha \bvdash \alpha &
\mbox{\rm (cut)} & \begin{array}{c}
                \Gamma \bvdash \alpha \qquad \Delta[\alpha] \bvdash \beta
                        \\\hline
                    \Delta[\Gamma]  \bvdash \beta
                \end{array} \\
\mbox{\rm (\textbf{I}--{\mtt{\tf}})} & \begin{array}{c}
                \Gamma \circ \alpha \bvdash \beta \\\hline
                \Gamma \bvdash \beta/\alpha
              \end{array} 
	      	&
\mbox{\rm ({\mtt{\tf}}--\textbf{I})} & \begin{array}{c}
        \Gamma \bvdash \alpha \qquad \Delta[\beta] \bvdash \gamma \\\hline
            \Delta[\beta/\alpha \circ \Gamma] \bvdash \gamma
              \end{array} \\
\mbox{\rm (\textbf{I}--{\mtt{\tb}})} & \begin{array}{c}
                \alpha \circ \Gamma \bvdash \beta \\\hline
                \Gamma \bvdash \alpha \backslash \beta
                \end{array} 
		& 
\mbox{\rm ({\mtt{\tb}}--\textbf{I})} & \begin{array}{c}
        \Gamma \bvdash \alpha \qquad \Delta[\beta] \bvdash \gamma \\\hline
            \Delta[\Gamma \circ \alpha \backslash \beta] \bvdash \gamma
                \end{array}
\end{array}
\end{equation}
%%
%%%
\index{$\mathsf{AB}$, $\mathsf{AB} + \mbox{\rm (cut)}$, $\mathsf{AB}^{-}$}%%%
%%%%
We denote the above calculus by $\mathsf{AB} + \mbox{\rm (cut)}$, 
and by $\mathsf{AB}$ the calculus without (cut). Further, the 
calculus consisting of (ax) and the rules ({\mtt{\tb}}--\textbf{I}) and 
({\mtt{\tf}}--\textbf{I}) is called $\mathsf{AB}^{-}$.
%%%
\begin{defn}
%%%
\index{category!basic}%%
\index{category!distinguished}%%
\index{categorial sequent grammar}%%
%%%
Let $M$ be a set of category constructors.
A \textbf{categorial sequent grammar} is a quintuple
%%%%
\begin{equation}
G = \auf \mbox{\tt S}, C, \zeta, A, \CS\zu
\end{equation}
%%%
where $C$ is a finite
set, the set of \textbf{basic categories}, $\mbox{\tt S} \in C$ the
so called \textbf{distinguished category}, $A$ a finite set,
the alphabet, $\zeta \colon A \pf \wp(\Cat_M(C))$ a category 
assignment, and $\CS$ a sequent calculus. We write $\vdash_G \vec{x}$ 
if for some category complex 
$\Gamma$ whose associated string via $\zeta$ is $\vec{x}$ we have
$\stackrel{\CS}{\rightsquigarrow} \Gamma \bvdash \mbox{\tt S}$.
\end{defn}
%%
We stress here that the sequent calculi are calculi to derive sequents. 
A sequent corresponds to a grammatical rule, or, more precisely, 
the sequent $\Gamma \bvdash \alpha$ expresses the fact that a 
category complex of type $\Gamma$ is a constituent that has the 
category $\alpha$ by the rules of the grammar. The rules of the 
sequent calculus can then be seen as {\it metarules}, which allow 
to pass from one valid statement 
concerning the grammar to another. 
%%
\begin{prop}[Correctness]
\label{korrektheit}
If $\Gamma \bvdash \alpha$ is derivable in $\mathsf{AB}^{-}$
then $[\Gamma] \subseteq [\alpha]$.
\end{prop}
%%
$\mathsf{AB}$ is strictly stronger than $\mathsf{AB}^-$. Notice namely
that the following sequent is derivable in $\mathsf{AB}$:
%%
\begin{equation}
\alpha \bvdash (\beta/\alpha)\backslash\beta
\end{equation}
%%
In natural deduction style calculi this corresponds to the
following unary rule:
%%
\begin{equation}
\begin{array}{c}
\alpha \\\hline
(\beta/\alpha)\backslash\beta
\end{array}
\end{equation}
%%
This rule is known as \textbf{type raising},
%%%
\index{type raising}%%
%%%
since it allows to proceed from the category $\alpha$ to the
``raised'' category $(\beta/\alpha)\backslash\beta$. Perhaps 
one should better call it {\it category raising}, but the other name 
is standard. To see that it is not derivable in $\mathsf{AB}^-$
we simply note that it is not correct for the cancellation
interpretation. We shall return to the question of interpretation
of the calculus $\mathsf{AB}$ in the next section.

An important property of these calculi is their decidability.
Given $\Gamma$ and $\alpha$ we can decide in finite time whether or
not $\Gamma \bvdash \alpha$ is derivable.
%%
\begin{thm}[Cut Elimination]
%%%
\index{Cut Elimination}%%%
%%%
There exists an algorithm to construct a proof of a sequent 
$\Gamma \bvdash \alpha$ in $\mathsf{AB}$ from a proof of 
$\Gamma \bvdash \alpha$ in $\mathsf{AB} + \mbox{\rm (cut)}$. 
Hence (cut) is admissible for $\mathsf{AB}$.
\end{thm}
%%%
\proofbeg
We presented a rather careful proof of Theorem~\ref{thm:cutelimination}, 
so that here we just give a sketch to be filled in accordingly.
We leave it to the reader to verify that each of the operations 
reduces the cut--weight.
We turn immediately to the case where the cut is on a main formula 
of a premiss. The first case is that the formula is introduced by
(\textbf{I}--{\mtt{\tf}}).
%%
\begin{equation}
\begin{array}{ccc}
\Gamma \circ \alpha \bvdash \beta & \qquad & \\\cline{1-1}
\Gamma \bvdash \beta/\alpha & & \Delta[\beta/\alpha] \bvdash \gamma
\\\hline
\multicolumn{3}{c}{\Delta[\Gamma] \bvdash \gamma}
\end{array}
\end{equation}
%%
Now look at the rule instance that is immediately above
$\Delta[\beta/\alpha] \bvdash \gamma$. There are several cases.
Case (0). The premiss is an axiom. Then $\gamma = \beta/\alpha$, and
the cut is superfluous. Case (1). $\beta/\alpha$ is a main formula
of the right hand premiss. Then $\Delta[\beta/\alpha]
= \Theta[\beta/\alpha \circ \Xi]$ for some $\Theta$ and $\Xi$,
and the instance of the rule was as follows.
%%
\begin{equation}
\label{eq:schn1}
\begin{array}{c}
\Xi \bvdash \alpha \qquad \Theta[\beta] \bvdash \gamma \\\hline
\Theta[\beta/\alpha \circ \Xi] \bvdash \gamma
\end{array}
\end{equation}
%%
Now we can restructure \eqref{eq:schn1} as follows.
%%
\begin{equation}
\begin{array}{ccc}
\Gamma \circ \alpha \bvdash \beta \qquad \Theta[\beta] \bvdash \gamma &
\qquad & \\\cline{1-1}
\Theta[\Gamma \circ \alpha] \bvdash \gamma & & \Xi \bvdash \alpha \\\hline
\multicolumn{3}{c}{\Theta[\Gamma \circ \Xi] \bvdash \gamma}
\end{array}
\end{equation}
%%
Now we assume that the formula is not a main formula of the right hand 
premiss. Case (2).  $\gamma = \zeta/\varepsilon$ and the premiss is 
obtained by application of (\textbf{I}--{\mtt{\tf}}).
%%
\begin{equation}
\label{eq:schn2}
\begin{array}{rl}
\Delta[\beta/\alpha] \circ \varepsilon & \bvdash \zeta \\\hline
\Delta[\beta/\alpha] & \bvdash \zeta/\varepsilon
\end{array}
\end{equation}
%%
We replace \eqref{eq:schn2} by
%%
\begin{equation}
\begin{array}{ccc}
\Gamma \circ \alpha \bvdash \beta & \qquad & \\\cline{1-1}
\Gamma \bvdash \beta/\alpha & & \Delta[\beta/\alpha] \circ \varepsilon
        \bvdash \zeta \\\hline
\multicolumn{3}{c}{%
\begin{array}{c}
\Delta[\Gamma] \circ \varepsilon \bvdash \zeta \\\hline
\Delta[\Gamma] \bvdash \varepsilon/\zeta
\end{array}}
\end{array}
\end{equation}
%%
Case (3). $\gamma = \varepsilon\backslash \zeta$ and has been obtained
by applying the rule (\textbf{I}--{\mtt{\tb}}). Then proceed as in
Case (2). Case (4). The application of the rule introduces a formula
which occurs in $\Delta$. This case is left to the reader.

Now if the left hand premiss has been obtained by
({\mtt{\tb}}--\textbf{I}), then one proceeds quite analogously.
So, we assume that the left hand premiss is created by an
application of ({\mtt{\tb}}--\textbf{I}).
%%
\begin{equation}
\label{eq:schn3}
\begin{array}{ccc}
\Gamma \bvdash \alpha & \Delta[\beta] \bvdash \gamma & \\\cline{1-2}
\multicolumn{2}{c}{%
\Delta[\beta/\alpha \circ \Gamma] \bvdash \gamma}
 & \Theta[\gamma] \bvdash \delta \\\hline
\multicolumn{3}{c}{\Theta[\Delta[\beta/\alpha \circ \Gamma]] %
\bvdash \delta}
\end{array}
\end{equation}
%%
We can restructure \eqref{eq:schn3} as follows.
%%
\begin{equation}
\begin{array}{ccc}
   & \Delta[\beta] \bvdash \gamma & 
	\Theta[\gamma] \bvdash \delta \\\cline{2-3}
\Gamma \bvdash \alpha & 
	\multicolumn{2}{c}{\Theta[\Delta[\beta]] \bvdash \delta}
\\\hline
\multicolumn{3}{c}{\Theta[\Delta[\beta/\alpha \circ \Gamma]] \bvdash \delta}
\end{array}
\end{equation}
%%
Also here one calculates that the degree of the new cut is less than
the degree of the old cut. The case where the left hand premiss is
created by ({\mtt{\tb}}--\textbf{I}) is analogous. All cases have
now been looked at.
\proofend
%%
\begin{cor}
$\mathsf{AB} + \mbox{\rm (cut)}$ is decidable.
\proofend
\end{cor}
%%
$\mathsf{AB}$ gives a method to test category complexes for their syntactic 
category. We expect that the meanings of the terms are likewise 
systematically connected with a term and that we can determine the 
meaning of a certain string once we have found a derivation for it. 
We now look at the rules of $\mathsf{AB}$ to see how they can be used 
as rules for deducing sign--sequents. Before we start we shall 
distinguish two interpretations of the calculus. The first is the 
intrinsic interpretation: every sequent we derive should be 
correct, with all basic parts of it belonging to the original 
lexicon. The second is the global interpretation: the sequents 
we derive should be correct if the lexicon was suitably expanded. 
This only makes a difference with respect to signs with empty 
exponent. If a lexicon has no such signs the intrinsic interpretation 
bans their use altogether, but the global interpretation leaves 
room for their addition. Adding them, however, will make more 
sequents derivable that are based on the original lexicon only.

We also write $\vec{x} : \alpha : M$ for the sign 
$\auf \vec{x}, \alpha, M\zu$. If $\vec{x}$, $\alpha$ or 
$M$ is irrelevant in the context it is omitted. For the meanings we 
use $\lambda$--terms, which are however only proxy for the `real' 
meanings (see the discussion at the end of the preceding section). 
Therefore we now write $(\lambda x.xy)$ in place of 
{\tt ($\lambda$x$_{\snull}$.(x$_{\snull}$x$_{\seins}$))}. 
A \textbf{sign complex} 
%%%
\index{sign complex}%%%
%%%
is a term made of signs with the help of $\circ$. 
Sequents are pairs $\Gamma \bvdash \tau$ where $\Gamma$ is a sign 
complex and $\tau$ a sign. $\sigma$ maps categories to types, as in 
Section~\ref{kap3}.\ref{kap3-2}. If $\vec{x} : \alpha : M$ is derivable, we 
want that $M$ is of type $\sigma(\alpha)$. Hence, the rules of 
$\mathsf{AB}$ should preserve this property. We define first a relation 
%%%
\index{$\succ$, $\Vdash$}%%
%%%
$\Vdash$ between sign complexes. It proceeds by means of the following 
rules. 
%%%
\begin{align}
\label{eq:leftcancel}
\Gamma[\vec{x} : \alpha/\beta : M \circ \vec{y} : \beta : N] 
 & \succ \Gamma[\vec{x}\conc\vec{y} : \alpha : (MN)]
\\
\label{eq:rightcancel}
\Gamma[\vec{x} : \beta : M \circ \vec{y} : \beta\backslash\alpha : N] 
 & \succ \Gamma[\vec{x}\conc\vec{y} : \alpha : (NM)] 
\end{align}
%%%
Since $M$ and $N$ are actually functions and not $\lambda$--terms, 
one may exchange any two $\lambda$--terms that denote the same 
function. However, if one considers them being actual 
$\lambda$--terms, then the following rule has to be added:
%%%
\begin{equation}
	\Gamma[\vec{x} : \alpha : M] \succ
	\Gamma[\vec{x} : \alpha : N] 
	\qquad \text{if $M \equiv N$}
\end{equation}
%%%
For $\Gamma$ a sign complex and $\sigma$ a sign put 
$\Gamma \Vdash \sigma$ iff $\Gamma \succ^{\ast} \sigma$.
We want to design a calculus that generates all and only 
the sequents $\Gamma \bvdash \sigma$ such that 
$\Gamma \Vdash \sigma$.

To begin, we shall omit the strings and deal with the 
meanings. Later we shall turn to the strings, which pose 
independent problems. The axioms are as follows.
%%
\begin{equation}
\mbox{\rm (ax)} \qquad 
\alpha  : M \bvdash \alpha : M
\end{equation}
%%
where $M$ is a term of type $\sigma(\alpha)$. (cut) looks like this.
%%
\begin{equation}
{\rm (cut)}\quad\begin{array}{c}
\Gamma \bvdash \alpha : N \qquad
    \Delta[\alpha : x_{\eta}]
    \bvdash \beta : M
\\\hline
\Delta[\Gamma] \bvdash \beta : [N/x_{\eta}]M
\end{array}
\end{equation}
%%
So, if $\Delta[\alpha : x_{\eta}]$ is a sign complex containing 
an occurrence of $\alpha : x_{\eta}$, then the occurrence of this 
sign complex is replaced and with it the variable $x_{\eta}$ in 
$M$. So, semantically speaking cut is substitution. Notice that 
since we cannot tell which occurrence in $M$ is to be replaced, 
we have to replace all of them. We will see that there are reasons 
to require that every variable has exactly one occurrence, so that 
this problem will never arise. (We could make this a condition on (cut). 
But see below for the fate of (cut).) The other rules are more 
complex. 
%%
\begin{equation}
\mbox{\rm ({\mtt{\tf}}--\textbf{I})}\quad
\begin{array}{c}
\Gamma \bvdash \alpha : M \qquad \Delta[\beta : x_{\zeta}] \bvdash \gamma :
    N \\\hline
\Delta[\beta/\alpha : x_{\eta \pf \zeta} \circ \Gamma] \bvdash \gamma :
    [(x_{\eta \pf \zeta}M)/x_{\zeta}]N
\end{array}
\end{equation}
%%
This corresponds to the replacement of a primitive constituent by
a complex constituent or the replacement of a value $M(x)$ by the
pair $\auf M,x\zu$. Here, the variable $x_{\eta \pf \zeta}$ is
introduced, which stands for a function from objects of type 
$\eta$ to objects of type $\zeta$. The variable $x_{\zeta}$ has, 
however, disappeared. This is a serious deficit of the calculus 
(which has other advantages, however). We shall below develop a 
different calculus. Analogously for the rule ({\mtt{\tb}}--\textbf{I}). 
%%%
\begin{equation}
\begin{array}{ll}
\mbox{\rm (\textbf{E}--{\mtt{\tf}})} &
\quad\begin{array}{c}
        \Gamma \bvdash \alpha : M
    \quad \Delta \bvdash \beta/\alpha : N \\\hline
            \Delta \circ \Gamma \bvdash \beta : (NM)
              \end{array} \\
\mbox{\rm (\textbf{E}--{\mtt{\tb}})} &
\quad\begin{array}{c}
        \Gamma \bvdash \alpha : M \quad
    \Delta \bvdash \alpha\backslash \beta : N \\\hline
            \Gamma \circ \Delta \bvdash \beta : (NM)
                \end{array}
\end{array}
\end{equation}
%%%
\begin{lem}
\label{lem:e-der}
The rule \mbox{\rm (\textbf{E}--{\mtt{\tf}})} is derivable from 
(cut) and \mbox{\rm ({\mtt{\tf}}--\textbf{I})}. Likewise, the rule 
\mbox{\rm (\textbf{E}--{\mtt{\tf}})} is derivable from 
(cut) and \mbox{\rm ({\mtt{\tf}}--\textbf{I})}. 
\end{lem}
%%%
\proofbeg
The following is an instance of \mbox{\rm ({\mtt{\tf}}--\textbf{I})}. 
%%%
\begin{equation}
\begin{array}{c}
\beta : x_{\zeta} \bvdash \beta : x_{\zeta} \qquad
	\alpha : x_{\eta} \bvdash \alpha :x_{\eta} 
\\\hline
\alpha/\beta : x_{\zeta\pf\eta} \circ \beta : x_{\zeta} 
	\bvdash \alpha : x_{\eta} 
\end{array}
\end{equation}
%%%
Now two cuts, with $\alpha : M \bvdash \alpha : M$ and with 
$\beta : N \bvdash \beta :N$, give \mbox{\rm (\textbf{E}--{\mtt{\tf}})}.
\proofend

Thus, the rules \eqref{eq:leftcancel} and \eqref{eq:rightcancel} 
are accounted for.

The rules (\textbf{I}--{\mtt{\tf}}) and (\textbf{I}--{\mtt{\tb}}) 
can be interpreted as follows. Assume that $\Gamma$ is a constituent of
category $\alpha/\beta$. 
%%
\begin{equation}
\mbox{\rm (\textbf{I}--{\mtt{\tf}})}\qquad
\begin{array}{c}
\Gamma \circ \alpha : x_{\eta} \bvdash \beta : M
    \\\hline
\Gamma \bvdash \beta/\alpha :
    (\lambda x_{\eta}. M)
\end{array} 
\end{equation}
%%
Here, the meaning of $\Gamma$ is of the form $M$ and the
meaning of $\alpha$ is $N$. Notice that $\mathsf{AB}$ forces us 
in this way to view the meaning of a word of category $\alpha/\beta$ 
to be a function from $\eta$--objects to $\zeta$--objects.
For it is formally required that $\Gamma$ has to have the
meaning of a function. We call the rules (\textbf{I}--{\mtt{\tf}}) 
and (\textbf{I}--{\mtt{\tb}}) also \textbf{abstraction rules}.
These rules have to be restricted, however. Define for a variable 
$x$ and a term $M$, 
%%%
\index{$\focc(x,M)$}%%
%%%
$\focc(x,M)$ to be the number of free occurrences of $x$ in $M$.
In the applications of the introduction rules, we add a side 
condition: 
%%%
\begin{equation}
\text{In }\mbox{\rm (\textbf{I}--{\mtt{\tf}})} \text{ and }
\mbox{\rm (\textbf{I}--{\mtt{\tb}})} : 
\focc(x_{\eta},M) \leq 1 
\end{equation}
%%%
(In fact, one can show that from this condition already follows 
$\focc(x_{\eta},M) = 1$, by induction on the proof.) To see the 
need for this restriction, look at the following derivation.
%%%
%\begin{equation}
$$\begin{array}{r@{\;\bvdash\;}l}
(\beta/\alpha/\alpha : M \circ \alpha : x_{\eta}) \circ 
	\alpha : x_{\eta} & \beta : (Mx_{\eta})x_{\eta} \\\hline
\beta/\alpha/\alpha : M \circ \alpha : x_{\eta} & 
	\beta/\alpha : \lambda x_{\eta}.(Mx_{\eta})x_{\eta} \\\hline
\beta/\alpha/\alpha : M & 
	\beta/\alpha/\alpha : \lambda x_{\eta}.\lambda x_{\eta}.%
((Mx_{\eta})x_{\eta})
\end{array}$$
%\end{equation}
%%%
The first is obtained using two applications of the derivable 
(\textbf{E}--{\mtt{\tf}}). 

This rule must be further restricted, however, as the next 
example shows. In the rule ({\mtt{\tb}}--\textbf{I}) put 
$\Delta := \gamma := \beta$.
%%
\begin{equation}
\begin{array}{c}
\alpha : x_{\eta} \bvdash \alpha : x_{\eta} \qquad
    \beta : x_{\zeta} \bvdash \beta : x_{\zeta} \\\hline
\alpha/\beta : x_{\zeta \pf \eta} \circ
    \beta : x_{\zeta} \bvdash \alpha :
    (x_{\zeta \pf \eta}x_{\zeta})
\end{array}
\end{equation}
%%
Using (\textbf{I}--{\mtt{\tf}}), we get 
%%
\begin{equation}
\begin{array}{c}
\alpha/\beta : x_{\zeta \pf \eta} \circ \beta : x_{\zeta}
    \bvdash \alpha : (x_{\zeta \pf \eta}x_{\zeta})
\\\hline
\alpha/\beta : x_{\zeta \pf \eta} \bvdash \alpha/\beta :
    (\lambda x_{\zeta}.(x_{\zeta \pf \eta}x_{\zeta}))
\end{array}
\end{equation}
%%
Now $\lambda x_{\zeta}.x_{\zeta \pf \eta} x_{\zeta}$
is the same function as $x_{\zeta \pf \eta}$. On the other hand, by applying
(\textbf{I}--{\mtt{\tb}}) we get
%%
\begin{equation}
\begin{array}{c}
\alpha/\beta : x_{\zeta \pf \eta} \circ \beta : x_{\zeta}
    \bvdash \alpha : (x_{\zeta \pf \eta}x_{\zeta})
\\\hline
\beta : x_{\zeta}\bvdash (\alpha/\beta)\backslash \alpha :
    (\lambda x_{\zeta \pf \eta}.(%
    x_{\zeta \pf \eta}x_{\zeta}))
\end{array}
\end{equation}
%%
This is the type raising rule which we have discussed above.
A variable $x_{\zeta}$ can also be regarded as a function, which 
for given function $f$ taking arguments of type $\zeta$ returns 
the value $f(x_{\zeta})$. However, $x_{\zeta}$ is not the same 
function as $(\lambda x_{\zeta \pf \eta}.(x_{\zeta \pf\eta}x_{\zeta}))$.
(The latter has the type $(\beta \pf \alpha) \pf \alpha$.)
Therefore the application of the rule is incorrect in this
case. Moreover, in the typed $\lambda$--calculus the equation 
$x_{\zeta} \doteq (\lambda x_{\zeta \pf \eta}.(%
    x_{\zeta \pf \eta}x_{\zeta}))$ is invalid. 

To remedy the situation we must require that the variable
which we have abstracted over appears on the left hand side of
$\bvdash$ in the premiss as an argument variable and not as a
variable of a function that is being applied to something. So,
the final form of the right slash--introduction rule is as follows.
%%
\begin{equation}
\mbox{\rm (\textbf{I}--{\mtt{\tf}})}\quad
\begin{array}{c}
\Gamma \circ \alpha : x_{\eta} \bvdash \beta : M
    \\\hline
\Gamma \bvdash \beta/\alpha :
    (\lambda x_{\eta}. M)
\end{array} \quad
\begin{array}{l}
\mbox{$x_{\eta}$ an argument variable,} \\
\text{and $\focc(x_{\eta},M) \leq 1$}
\end{array}
\end{equation}
%%
How can one detect whether $x_{\eta}$ is an argument variable?
To this end we require that the sequent $\Gamma \bvdash \beta/\alpha$ 
be derivable in categorial $\mathsf{AB}^-$. This seems paradoxical. 
For with this restriction the calculus seems to be as weak as 
$\mathsf{AB}^-$. Why should one make use of the rule
(\textbf{I}--{\mtt{\tf}}) if the sequent is anyway derivable? To 
understand this one should take note of the difference between the
categorial calculus and the interpreted calculus. We allow the use of
the interpreted rule (\textbf{I}--{\mtt{\tf}}) if $\Gamma \bvdash
\beta/\alpha$ is derivable in the categorial calculus; or, to
be more prosaic, if $\Gamma$ has the category $\beta/\alpha$ and
hence the type $\alpha \pf \beta$. That this indeed strengthens
the calculus can be seen as follows. In the interpreted $\mathsf{AB}^-$ 
the following sequent is not derivable (though it is derivable in 
$\mathsf{AB}$). The proof of this claim is left as an exercise. 
%%
\begin{equation}
\alpha/\beta : x_{\zeta \pf \eta} \bvdash
\alpha/\beta : \lambda x_{\zeta}.x_{\zeta\pf\eta}x_{\zeta}
\end{equation}
%%
We assign to a sign complex a sign as follows. 
%%
%%%
\index{$\S(\Gamma)$}%%%
%%%%
\begin{align}
\notag
\S(\vec{x} : \alpha : M) & := \auf \vec{x}, \alpha, M\zu \\
\S(\vec{x} : \alpha/\beta : M \circ \vec{y} : \beta : N) 
        & := \mbox{\mtt A}_{\sgr}(\auf \vec{x}, \alpha/\beta, M\zu, 
	\auf \vec{y}, \beta, N\zu) \\
\notag
\S(\vec{x} : \beta : M \circ \vec{y} : \beta\backslash \alpha : N) 
	& := \mbox{\mtt A}_{\skl}(\auf \vec{x}, \beta, M\zu, 
	\auf \vec{y}, \beta\backslash\alpha, N\zu) 
\end{align}
%%
It is easy to see that if $\Gamma \bvdash \alpha : M$ is derivable 
in the interpreted $\mathsf{AB}$ then $\S(\Gamma) = \auf \vec{x}, \alpha, %
M'\zu$ for some $M' \equiv M$. (Of course, $M$ and $M'$ are just notational 
variants denoting the same object. Thus they are identical qua 
objects they represent.) 

The calculus that has just been defined has drawbacks. We will see 
below that (cut) cannot be formulated for strings. Thus, we have 
to do without it. But then we cannot derive the rules 
\mbox{\rm (\textbf{E}--{\mtt{\tf}})} and 
\mbox{\rm (\textbf{E}--{\mtt{\tb}})}. The calculus $\mathsf{N}$ 
obviates the need for that.
%%
\begin{defn}
%%%
\index{$\mathsf{N}$}%%
%%%
The calculus $\mathsf{N}$ has the rules \mbox{(ax)}, 
\mbox{\rm (\textbf{I}--{\mtt{\tf}})}, \mbox{\rm (\textbf{E}--{\mtt{\tf}})} 
\mbox{\rm (\textbf{I}--{\mtt{\tf}})} and 
\mbox{\rm (\textbf{E}--{\mtt{\tb}})}. 
\end{defn}
%%
In (the interpreted) $\mathsf{N}$, (cut) is admissible. (The 
proof of that is left as an exercise.) 

Now let us turn to strings. Now we omit the interpretation, since
it has been dealt with. Our objects are now written as $\vec{x} :
\alpha$ where $\vec{x}$ is a string and $\alpha$ a category. The
reader is reminded of the fact that $\vec{y}/\vec{x}$ denotes that
string which results from $\vec{y}$ by removing the postfix
$\vec{x}$. This is clearly defined only if $\vec{y} =
\vec{u}\conc\vec{x}$ for some $\vec{u}$, and then we have
$\vec{y}/\vec{x} = \vec{u}$. Analogously for $\vec{x}\backslash
\vec{y}$.
%%
\begin{equation}
$$\begin{array}{l@{\quad}l@{\quad}l@{\quad}l}
\mbox{(ax)} & \vec{x} : \alpha \bvdash \vec{x} : \alpha & \\
\mbox{\rm (\textbf{I}--{\mtt{\tf}})} &
        \begin{array}{c}
                \Gamma \circ \vec{x} : \alpha \bvdash \vec{y} : \beta
        \\\hline
                \Gamma \bvdash \vec{y}/ \vec{x} : \beta/\alpha
              \end{array} &
\mbox{\rm (\textbf{E}--{\mtt{\tf}})} & \begin{array}{c}
        \Gamma \bvdash \vec{x} : \alpha
   	 \quad 
    \Delta \bvdash \vec{y} : \beta/\alpha  \\\hline
            \Delta \circ \Gamma \bvdash
        \vec{y} \conc \vec{x} : \beta
              \end{array} \\
\mbox{\rm (\textbf{I}--{\mtt{\tb}})} &
        \begin{array}{c}
                \vec{x} : \alpha \circ \Gamma \bvdash \vec{y} : \beta
        \\\hline
                \Gamma \bvdash \vec{x}\backslash \vec{y} :
        \alpha \backslash \beta
                \end{array} &
\mbox{\rm (\textbf{E}--{\mtt{\tb}})} & \begin{array}{c}
        \Gamma \bvdash \vec{x} : \alpha \quad
    \Delta \bvdash \vec{y} : \alpha\backslash \beta \\\hline
            \Gamma \circ \Delta \bvdash \vec{x} \conc \vec{y} : \beta
        \end{array}
\end{array}$$
\end{equation}
%%
The cut rule is however no more a rule of the calculus. There
is no formulation of it at all. Suppose we try to formulate a 
cut rule. Then it would go as follows.
%%
\begin{equation}
\begin{array}{c}
                \Gamma \bvdash \vec{x} : \alpha \qquad
        \Delta[\vec{y} : \alpha] \bvdash \vec{z} : \beta
                        \\\hline
                    \Delta[\Gamma]  \bvdash [\vec{y}/\vec{x}]
            \vec{z} : \beta
                \end{array}
\end{equation}
%%
Here, $[\vec{y}/\vec{x}]\vec{z}$ denotes the result of replacing
$\vec{y}$ for $\vec{x}$ in $\vec{z}$. So, on the strings (cut) 
becomes constituent replacement. Notice that only one occurrence 
may be replaced, so if $\vec{x}$ occurs several times, the result 
of the operation $[\vec{y}/\vec{x}]\vec{z}$ is not uniquely defined. 
Moreover, $\vec{x}$ may occur accidentally in $\vec{z}$! Thus, it 
is not clear which of the occurrences is the right one to be replaced. 
So the rule of (cut) cannot even be properly formulated. On 
the other hand, semantically it is admissible, so for the semantics 
we can do without it anyway. However, the same problem of 
substitution arises with the rules ({\mtt{\tf}}--\textbf{I}) and 
({\mtt{\tb}}--\textbf{I}). Thus, they had to be eliminated as 
well.

This completes the definition of the sign calculus $\mathsf{N}$.
Call $\mathsf{E}$ the calculus consisting of just ({\mtt{\tf}}--\textbf{E}) 
%%%%
\index{$\mathsf{E}$}%%%
%%%%
and ({\mtt{\tb}}--\textbf{E}). Based on Lemma~\ref{lem:e-der} the 
completeness of $\mathsf{E}$ for $\Vdash$ is easily established.
%%%
\begin{thm}
$\stackrel{\mathsf{E}}{\leadsto} \Gamma \bvdash \sigma$ iff 
$\Gamma \Vdash \sigma$.
\end{thm}
%%
$\mathsf{N}$ is certainly correct for the global interpretation, 
but it is correct for the intrinsic interpretation? The answer 
is actually yes! The fact is that the introduction rules are 
toothless tigers: they can only eliminate a variable that has 
never had a chance to play a role. For assume that we have an 
$\mathsf{N}$--proof. If (\textbf{I}--{\mtt{\tf}}) is used, let 
the highest application be as follows. 
%%%
\begin{equation}
\begin{array}{c}
\Gamma \circ \varepsilon : \alpha : x_{\eta} \bvdash \vec{y} : \beta : M
    \\\hline
\Gamma \bvdash \vec{y} : \beta/\alpha : (\lambda x_{\eta}. M)
\end{array} 
\end{equation}
%%%
Then the sequent above the line has been introduced by 
(\textbf{E}--{\mtt{\tf}}): 
%%%
\begin{equation}
\begin{array}{c}
\Gamma \bvdash \vec{y} : \beta/\alpha : N 
	\qquad \varepsilon : \alpha : x_{\eta} 
	\bvdash \varepsilon : \alpha : x_{\eta} 
    \\\hline
\Gamma \circ \varepsilon : \alpha : x_{\eta} \bvdash 
	\vec{y} : \beta : Nx_{\eta}
\end{array} 
\end{equation}
%%%
Here, $N x_{\eta} = M$. Since $(\lambda x_{\eta}.M) = 
(\lambda x_{\eta}.Nx_{\eta}) = N$, this part of the proof can 
be eliminated.
%%%
\begin{thm}
The rules \mbox{\rm (\textbf{I}--{\mtt{\tf}})} and 
\mbox{\rm (\textbf{I}--{\mtt{\tb}})} are admissible in $\mathsf{E}$.
\end{thm}
%%%
In the next section, however, we shall study the effect of adding 
associativity. In presence of associativity the introduction 
rules actually do considerable work. In that case, to regain 
correctness, we can either ban the introduction rules, or we 
can restrict the axioms. Given an AB--sign grammar $\GA$ we can 
restrict the set of axioms to 
%%
\begin{equation}
\mbox{\rm (ax$_{\GA}$)} \quad \upsilon(f) \bvdash \upsilon(f)
    \qquad \text{where $f \in F$ and $\Omega(f) = 0$}
\end{equation}
%%
For an AB--grammar does not possess any modes of the form 
$\auf \varepsilon, \alpha, x_{\alpha}\zu$ where $x_{\alpha}$
is a variable. 
%%
\vplatz
\exercise
Prove the correctness theorem, Proposition~\ref{korrektheit}.
%%
\vplatz
\exercise
Define a function $p$ from category complexes to categories as follows.
%%
\begin{equation}
\begin{split}
p(\alpha) & := \alpha \\
p(\Gamma \circ \Delta) & := p(\Gamma) \cdot p(\Delta)
\end{split}
\end{equation}
%%
Show that $\Gamma \bvdash \alpha$ is derivable in $\mathsf{AB}^-$ 
iff $p(\Gamma) = \alpha$. Show that this also holds for $\mathsf{AB}^- + %
\mbox{(cut)}$. Conclude from this that (cut) is admissible 
in (categorial!) $\mathsf{AB}^-$. (This could in principle be extracted 
from the proof for $\mathsf{AB}$, but this proof here is quite 
simple.)
%%
%\vplatz
%\exercise
%Prove the completeness of \mathsf{AB} with respect to the 
%cancellation interpretation. This means that
%if for all $A$ and all $\zeta$ $[\Gamma]_{\zeta} \subseteq
%[\alpha]_{\zeta}$ then $\Gamma \bvdash \alpha$ is derivable
%in \textsf{AB}.
%%
\vplatz
\exercise
Show that every CFL can be generated by an AB--grammar 
using only two basic categories. 
%%
\vplatz
\exercise
Show the following claim: in the interpreted
$\mathsf{AB}^-$--calculus no sequents are derivable
which contain bound variables.
%%
\vplatz
\exercise
\label{ueb:nschnitt}
Show that (cut) is admissible for $\mathsf{N}$.

 \section{The Lambek--Calculus}
\label{kap:lambek}
%
%
%
The Lambek--Calculus, $\mathsf{L}$, is in many respects an extension of
$\mathsf{AB}$. It has been introduced in \cite{lambek:mathematics}. In
contrast to $\mathsf{AB}$, in $\mathsf{L}$ categories are not interpreted 
as sets of labelled trees but as sets of strings. This means that the
calculus has different laws. Furthermore, $\mathsf{L}$ possesses a new
category constructor, {\it pair formation\/}; it is written $\bullet$ 
and has a counterpart on the level of categories, also denoted by that 
symbol. The constructors of the classical Lambek--calculus for 
categories therefore are {\mtt{\tb}}, {\mtt{\tf}} and $\bullet$. 
Given an alphabet $A$ and an elementary category assignment $\zeta$
we denote by $\{\alpha\}_{\zeta}$ the set of all strings over $A$ 
which are of category $\alpha$ with respect to $\zeta$. Then the 
following holds.
%%
\begin{equation}
\begin{split}
\{\alpha \bullet \beta\}_{\zeta}  &
        := \{\alpha\}_{\zeta} \cdot \{\beta\}_{\zeta} \\
\mbox{}\{\Gamma \circ \Delta\}_{\zeta}   &
    := \{\Gamma\}_{\zeta} \cdot \{\Delta\}_{\zeta}
\end{split}
\end{equation}
%%
Since we have the constructor $\bullet$ at our disposal, we
can in principle dispense with the symbol $\circ$. However,
we shall not do so. We shall formulate the calculus as before
using $\circ$, which makes it directly comparable to the ones
we have defined above. Hence as before we distinguish {\it terms\/}
from {\it structures}. We write $\Gamma \bvdash \beta$ if
$\{ \Gamma \}_{\zeta} \subseteq \{ \alpha\}_{\zeta}$.  We shall
axiomatize the sequents of $\bvdash$. In order to do so we add
the following rules to the calculus $\mathsf{AB}$ (without (cut)).
%%
$$\begin{array}{c@{\;}cc@{\;}c}
\mbox{(\textbf{ass1})} & \begin{array}{c}
        \Gamma[\Delta_1 \circ (\Delta_2 \circ \Delta_3)]
    \bvdash \alpha \\\hline
        \Gamma[(\Delta_1 \circ \Delta_2) \circ \Delta_3] \bvdash \alpha
         \end{array} &
\mbox{(\textbf{ass2})} & \begin{array}{c}
        \Gamma[(\Delta_1 \circ \Delta_2) \circ \Delta_3] \bvdash \alpha
    \\\hline
        \Gamma[\Delta_1 \circ (\Delta_2 \circ \Delta_3)] \bvdash \alpha
        \end{array} \\
\mbox{(\textbf{$\bullet$--I})} & \begin{array}{c}
        \Gamma[\alpha \circ \beta] \bvdash \gamma \\\hline
        \Gamma[\alpha \bullet \beta] \bvdash \gamma
        \end{array} &
\mbox{(\textbf{I--$\bullet$})} & \begin{array}{c}
        \Gamma \bvdash \alpha \qquad \Delta \bvdash \beta \\\hline
        \Gamma \circ \Delta \bvdash \alpha \bullet \beta
        \end{array}
\end{array}$$
%%
This calculus is called the  \textbf{Lambek--Calculus}, or simply 
$\mathsf{L}$.
%%%
\index{Lambek--Calculus}
\index{Lambek--Calculus!Nonassociative}
\index{$\mathsf{L}$, $\mathsf{NL}$, $\mathsf{NL}^-$}%%
%%%
Further, we put $\mathsf{NL}  := \mathsf{AB} + \mbox{(\textbf{I}--$\bullet$)} 
+ \mbox{($\bullet$--\textbf{I})}$ and  $\mathsf{NL}^{-} := \mathsf{AB}^{-} + 
\mbox{($\bullet$--\textbf{I})} + \mbox{(\textbf{I}--$\bullet$)}$.
$\mathsf{NL}$ is also called the \textbf{Nonassociative Lambek--Calculus}.
%%
\begin{thm}[Lambek]
%%%
\index{Lambek, Joachim}%%%
%%%
(cut) is admissible for $\mathsf{L}$.
\end{thm}
%%
\begin{cor}[Lambek]
%%%
\index{Lambek, Joachim}%%%
%%%%
$\mathsf{L}$ with or without (cut) is decidable.
\end{cor}
%%
For a proof we only have to look at applications of (cut) following
an application of the new rules.  Assume that the left hand premiss 
has been obtained by an application of (ass1).
%%
\begin{equation}
\begin{array}{ccc}
\Gamma[\Theta_1 \circ (\Theta_2 \circ \Theta_2)] \bvdash \alpha &
        \qquad & \\\cline{1-1}
\Gamma[(\Theta_1 \circ \Theta_2) \circ \Theta_2] \bvdash \alpha &
        & \Delta[\alpha] \bvdash \beta \\\hline
\multicolumn{3}{c}{%
\Delta[\Gamma[(\Theta_1 \circ \Theta_2) \circ \Theta_3]] \bvdash \beta}
\end{array}
\end{equation}
%%
This proof part we reformulate into the following one.
%%
\begin{equation}
\begin{array}{c}
\Gamma[\Theta_1 \circ (\Theta_2 \circ \Theta_3)] \bvdash \alpha
        \qquad \Delta[\alpha] \bvdash \beta \\\hline
\begin{array}{c}
\Delta[\Gamma[\Theta_1 \circ (\Theta_2 \circ \Theta_3)]] \bvdash
        \beta \\\hline
\Delta[\Gamma[(\Theta_1 \circ \Theta_2) \circ \Theta_3]] \bvdash
        \beta
\end{array}
\end{array}
\end{equation}
%%
Analogously if the left hand premiss has been obtained by using
(ass2). We leave it to the reader to treat the case where the right
hand premiss has been obtained by using (ass1) or (ass2). We
have to remark here that by reformulation we do not diminish the
degree of the cut. So the original proof is not easily transported
into the new setting. However, the {\it depth\/} of the application
has been diminished. Here, depth means (intuitively) the length of
a longest path through the proof tree from the top up to the
rule occurrence. If we assume that $\Gamma[\Theta_1 \circ %
(\Theta_2 \circ \Theta_3)] \bvdash \alpha$ has depth $i$ and
$\Delta[\alpha] \bvdash \beta$ depth $j$ then in the first tree
the application of (cut) has depth $\max \{i,j\} + 1$,
in the second however it has depth $\max \{i,j\}$.

Let us look at the cases of introduction of $\bullet$. The case of
($\bullet$--\textbf{I}) on the left hand premiss is easy.
%%
\begin{multline}
\begin{array}{ccc}
\Gamma[\theta_1 \circ \theta_2] \bvdash \alpha & \quad &
\\\cline{1-1} \Gamma[\theta_1 \bullet \theta_2] \bvdash \alpha & &
\Delta[\alpha]
        \bvdash \gamma \\\hline
\multicolumn{3}{c}{\Delta[\Gamma[\theta_1 \bullet \theta_2]]
        \bvdash \gamma}
\end{array} \\
\quad\leadsto\quad
\begin{array}{c}
\Gamma[\theta_1 \circ \theta_2] \bvdash \alpha \quad
        \Delta[\alpha] \bvdash \gamma \\\hline
\begin{array}{c}
\Delta[\Gamma[\theta_1 \circ \theta_2]] \bvdash \gamma \\\hline
\Delta[\Gamma[\theta_1 \bullet \theta_2]] \bvdash \gamma
\end{array}
\end{array}
\end{multline}
%%
Now for the case of (\textbf{I}--$\bullet$) on the right hand premiss.
%%
\begin{equation}
\begin{array}{ccc}
 & \qquad & \Theta_1 \bvdash \theta_1 \qquad \Theta_2 \bvdash \theta_2
        \\\cline{3-3}
\Gamma \bvdash \alpha & & \Delta[\alpha] \bvdash \gamma \\\hline
\multicolumn{3}{c}{\Delta[\Gamma] \bvdash \gamma}
\end{array}
\end{equation}
%%
In this case $\gamma = \theta_1 \bullet \theta_2$. Furthermore,
$\Delta = \Theta_1 \circ \Theta_2$ and the marked occurrence of
$\alpha$ either is in $\Theta_1$ or in $\Theta_2$. Without loss of
generality we assume that it is in $\Theta_1$. Then we can replace
the proof by
%%
\begin{equation}
\begin{array}{ccc}
\Gamma \bvdash \alpha \qquad \Theta_1[\alpha] \bvdash \theta_1 & \qquad
        & \\\cline{1-1}
\Theta_1[\Gamma] \bvdash \theta_1 & & \Theta_2 \bvdash \theta_2 \\\hline
\multicolumn{3}{c}{\Theta_1[\Gamma] \circ \Theta_2 \bvdash
        \theta_1 \bullet \theta_2}
\end{array}
\end{equation}
%%
We have $\Theta_1[\Gamma] \circ \Theta_2 = \Delta[\Gamma]$ by
hypothesis on the occurrence of $\alpha$. Now we look at the case
where the left hand premiss of cut has been introduced by
(\textbf{I}--$\bullet$). We may assume that the right hand premiss has
been obtained through application of ($\bullet$--\textbf{I}). The case
where $\alpha$ is a side formula is once again easy. So let
$\alpha$ be main formula. We get the following local tree.
%%
\begin{multline}
\begin{array}{c}
\begin{array}{c}
\Theta_1 \bvdash \theta_1 \qquad \Theta_2 \bvdash \theta_2 \\\hline
\Theta_1 \circ \Theta_2 \bvdash \theta_1 \bullet \theta_2
\end{array}
        \qquad
\begin{array}{c}
\Delta[\theta_1 \circ \theta_2] \bvdash \gamma \\\hline
\Delta[\theta_1 \bullet \theta_2] \bvdash \gamma
\end{array}
        \\\hline
\Delta[\Theta_1 \circ \Theta_2] \bvdash \gamma
\end{array}
\qquad\leadsto\qquad \\
%%
\begin{array}{ccc}
 & \qquad & \Theta_1 \bvdash \theta_1 \qquad
        \Delta[\theta_1 \circ \theta_2] \bvdash \gamma \\\cline{3-3}
\Theta_2 \bvdash \theta_2 & &
        \Delta[\Theta_1 \circ \theta_2] \bvdash \gamma \\\hline
\multicolumn{3}{c}{\Delta[\Theta_1 \circ \Theta_2] \bvdash \gamma}
\end{array}
\end{multline}
%%
In all cases the cut--weight (or the sum of the depth of the 
cuts) has been reduced. 

We shall also present a different formulation of $\mathsf{L}$ using
natural deduction over ordered DAGs. Here are the rules:
%%
\begin{equation}
$$\begin{array}{l@{\qquad}l}
\mbox{\rm (\textbf{I}--$\bullet$)}\quad
\begin{array}{c}
\alpha\quad\beta \\\hline
\mbox{\mtt ($\alpha\bullet\beta$)}
\end{array}
    &
\mbox{\rm (\textbf{E}--$\bullet$)}\quad
\begin{array}{c}
\mbox{\mtt ($\alpha\bullet\beta$)} \\\hline
\alpha \quad \beta
\end{array}
    \\
    \\
\mbox{\rm (\textbf{I}--{\mtt{\tf}})}\quad
\begin{array}{c}
[\alpha] \\
\vdots   \\\hline \beta \\\hline \mbox{\mtt ($\alpha${\tf}$\beta$)}
\end{array}
    &
\mbox{\rm (\textbf{E}--{\mtt{\tf}})}\quad
\begin{array}{c}
\beta \quad \mbox{\mtt ($\beta${\tb}$\alpha$)} \\\hline
\alpha
\end{array}
    \\
    \\
\mbox{\rm (\textbf{I}--{\mtt{\tb}})}\quad
\begin{array}{c}
[\alpha] \\
\vdots   \\\hline \beta \\\hline 
\mbox{\mtt ($\alpha${\tf}$\beta$)}
\end{array}
    &
\mbox{\rm (\textbf{E}--{\mtt{\tb}})}\quad
\begin{array}{c}
\mbox{\mtt ($\alpha${\tf}$\beta$)} \quad \beta \\\hline
\alpha
\end{array}
\end{array}$$
\end{equation}
%%
These rules are very much like the natural deduction rules for
intuitionistic logic. However, two differences must be noted.
First, suppose we disregard for the moment the rules for $\bullet$. 
(This would incidentally give exactly the natural deduction calculus
corresponding to $\mathsf{AB}$.) The rules must be understood to operate 
on ordered trees. Otherwise, the difference between then rules 
for {\mtt\tf} and the rules for {\mtt\tb} would be obliterated.
Second, the elimination rule for $\bullet$ creates two linearly
ordered daughters for a node, thus we not only create ordered
trees, we in fact create ordered DAGs. We shall not spell out
exactly how the rules are interpreted in terms of ordered DAGs,
but we shall point out a few noteworthy things. First, this style of
presentation is very much linguistically oriented. We may in fact
proceed in the same way as for $\mathsf{AB}$ and define algorithms that
decorate strings with certain categorial labels and proceed
downward using the rules shown above. Yet, it must be clear that
the so created structures cannot be captured by constituency rules
(let alone rules of a CFG) for the simple reason
that they are not trees. The following derivation is
illustrative of this.
%%
\begin{equation}
\begin{array}{ccccc}
\multicolumn{5}{c}{(\alpha \bullet (\alpha\backslash\gamma)/\beta) %
\bullet \beta} \\\hline
\multicolumn{3}{c}{\alpha \bullet (\alpha\backslash\gamma)/\beta}
    & \quad & \beta \\\hline
\alpha & \quad & (\alpha \backslash\gamma)/\beta & \quad & \beta
    \\\cline{3-5}
\vdots  &       & \multicolumn{3}{c}{\alpha\backslash\gamma}
\\\hline \multicolumn{5}{c}{\gamma}
\end{array}
\end{equation}
%%
Notice that if a rule has two premisses, these must be adjacent
and follow each other in the order specified in the rule. No more
is required. This allows among other to derive associativity, that
is, $(\alpha \bullet\beta) \bullet \gamma \dashv\bvdash \alpha
\bullet (\beta \bullet\gamma)$. However, notice the role of the
so--called assumptions and their discharge. Once an assumption is
discharged, it is effectively removed, so that the items to its
left and its right are now adjacent. This plays a crucial role in
the derivation of the rule of function composition.
%%
\begin{equation}
\begin{array}{ccccc}
\alpha/\beta & \quad & \beta/\gamma & \quad & \gamma^{\surd}
    \\\cline{3-5}
\vdots       &       & \multicolumn{3}{c}{\beta} \\\cline{1-5}
             &       &    \alpha    &       & \\\cline{3-3}
             &       & \alpha/\gamma &      &
\end{array}
\end{equation}
%%
As soon as the assumption $\gamma$ is removed, the top sequence
reads $\alpha/\beta, \beta/\gamma$.

The relationship with $\mathsf{L}$ is as follows. Let $\Gamma$ be a
sequence of categories. We interpret this as a labelled DAG, which
is linearly ordered. Now we successively apply the rules above. It
is verified that each rule application preserves the property that
the leaves of the DAG are linearly ordered. Define a category
corresponding to a sequence as follows.
%%
\begin{equation}
\begin{split}
\alpha^{\bullet} & := \alpha \\
(\alpha,\Delta)^{\bullet} & := \alpha \bullet \Delta^{\bullet}
\end{split}
\end{equation}
%%
First of all we say that for two sequences $\Delta$ and $\Delta'$,
$\Delta'$ is \textbf{derivable from} $\Delta$ in the natural deduction 
style calculus if there is a DAG constructed according to the rules 
above, whose topmost sequence is $\Delta$ and whose lowermost sequence
is $\Delta'$. (Notice that assumptions get discharged, so that we
cannot simply say that $\Delta$ is the sequence we started off with.)
The following is then shown by induction.
%%
\begin{thm}
Let $\Delta$ and $\Theta$ be two sequences of categories. $\Theta$
is derivable from $\Delta$ iff $\Delta \bvdash \Theta^{\bullet}$ is 
derivable in $\mathsf{L}$.
\end{thm}
%%
This shows that the natural deduction style calculus is
effectively equivalent to $\mathsf{L}$.

$\mathsf{L}$ allows for a result akin to the
Curry--Howard--Isomorphism. This is an extension of the latter
result in two respects. First, we have the additional type
constructor $\bullet$, which we have to match by some category
constructor, and second, there are different structural rules.
First, the new type constructor is actually the pair--formation.
%%
\begin{defn}
%%%
\index{projection}%%
\index{$\lambda^{\bullet}$--term}%%
%%%
Every $\lambda$--term is a $\lambda^{\bullet}$--\textbf{term}.
Given two $\lambda^{\bullet}$--terms $M$ and $N$, $\auf M, N\zu$, $p_1(M)$ 
and $p_2(M)$ also are $\lambda^{\bullet}$--\textbf{terms}.
Further, the following equations hold.
%%
\begin{equation}
p_1(\auf M, N\zu) = M, \qquad p_2(\auf M,N\zu) = N.
\end{equation}
%%
$p_1(U)$ and $p_2(U)$ are not defined if $U$ is not of the
form $\auf M, N\zu$ for some $M$ and $N$. The functions $p_1$
and $p_2$ are called the \textbf{projections}.
\end{defn}
%%
Notice that antecedents of sequents no longer consist of sets of
sequences. Hence, $\Gamma$, $\Delta$, $\Theta$ now denote
sequences rather than sets. In Table~\ref{tab:seq} we display the
new calculus.
%%
\begin{table}
\caption{$\mathsf{L}$ with $\lambda$--Term Annotation}%%
\label{tab:seq}
$$\begin{array}{ll}
\mbox{\rm (ax)} \quad x : \varphi \bvdash x : \varphi
   &   \\
\multicolumn{2}{l}{\mbox{\rm (cut)} \quad
\begin{array}{c}
\Gamma \bvdash M : \varphi \quad \Delta, x : \varphi, \Theta \bvdash
    N : \beta \\\hline
\Delta, \Gamma, \Theta \bvdash [M/x]N : \beta
\end{array}}
\\
\mbox{(\textbf{E}--{\mtt{\tf}})}\quad
\begin{array}{c}
\Gamma \bvdash M : \alpha/\beta \quad \Delta \bvdash N : \beta
\\\hline
\Gamma, \Delta \bvdash MN : \alpha
\end{array}
    &
\mbox{(\textbf{I}--{\mtt{\tf}})}\quad
\begin{array}{c}
\Gamma, x : \beta \bvdash M : \alpha \\\hline
\Gamma \bvdash \lambda x.M : \alpha/\beta
\end{array}
\\
\mbox{(\textbf{E}--{\mtt{\tb}})}\quad
\begin{array}{c}
\Gamma \bvdash M : \beta \backslash\alpha \quad \Delta \bvdash N : \beta
\\\hline
\Delta, \Gamma \bvdash MN : \alpha
\end{array}
    &
\mbox{(\textbf{I}--{\mtt{\tb}})}\quad
\begin{array}{c}
x : \beta, \Gamma \bvdash M : \alpha \\\hline
\Gamma \bvdash \lambda x.M : \beta \backslash \alpha
\end{array}
\\
\multicolumn{2}{l}{\mbox{(\textbf{E}--$\bullet$)}\quad
\begin{array}{c}
\Gamma \bvdash M : \alpha\bullet\beta \quad \Delta, x : \alpha, y : \beta,
    \Theta
     \bvdash U : \psi
\\\hline
\Delta, \Gamma, \Theta \bvdash [p_1(M)/x][p_2(M)/y]U : \psi
\end{array}} \\
\mbox{(\textbf{I}--$\bullet$)}\quad
\begin{array}{c}
\Gamma \bvdash M : \alpha \quad \Delta \bvdash N : \beta \\\hline
\Gamma, \Delta \bvdash \auf M, N\zu :
    \alpha \bullet \beta
\end{array}
\end{array}$$
\end{table}
%%
We have also put a general constraint on the proofs that variables
may not be used twice. To implement this constraint, we define the
notion of a linear term:
%%
\begin{defn}
%%%
\index{$\lambda^{\bullet}$--term!strictly linear}%%%
\index{$\lambda^{\bullet}$--term!linear}%%%
%%%
A term $M$ is \textbf{strictly linear} if for every variable 
$x$ and every subterm $N$, $\focc(x,N) \leq 1$. A term is \textbf{linear} 
if it results from a strictly linear term $M$ by iterated 
replacement of a subterm $M'$ by $[p_1(N)/x][p_2(N)/y]M'$, 
where $N$ is a linear term.
\end{defn}
%%
The calculus above yields only linear terms if we start with 
variables and require in the rules (\textbf{I}--$\bullet$), 
(\textbf{E}--{\mtt{\tf}}), (\textbf{E}--{\mtt{\tb}}) that the 
sets of free variables be disjoint, and that in 
(\textbf{I}--{\mtt{\tf}}) and (\textbf{I}--{\mtt{\tb}}) the
variable occurs exactly once free in $M$. In this way we can 
ensure that for every sequent derivable in $\mathsf{L}$ there 
actually exists a labelling such that the labelled sequent is 
derivable in the labelled
calculus. This new calculus establishes a close correspondence
between linear $\lambda^{\bullet}$--terms and the so--called
multiplicative fragment of linear logic, which naturally arises
from the above calculus by stripping off the terms and leaving
only the formulae. A variant of proof normalization can be shown,
and all this yields that $\mathsf{L}$ has quite well--behaved
properties.

In presence of the rules (ass1) and (ass2) $\bullet$ behaves exactly
like concatenation, that is, it is a fully associative operation.
Therefore we shall change the notation in what is to follow. In
place of structures consisting of categories we shall consider
finite sequences of categories, that is, strings over 
$\Cat_{\mbox{\smtt{\tb}}, \bullet, \mbox{\smtt{\tf}}}(C)$. 
We denote concatenation by comma, as is commonly done.

Now we return to the theory of meaning. In the previous section we
have seen how to extend $\mathsf{AB}$ by a component for meanings
which computes the meaning in tandem with the category. We shall
do the same here. To this end we shall have to first clarify what
we mean by a realization of $\alpha \bullet \beta$. We shall agree
on the following.
%%
\begin{equation}
\real{\alpha \bullet \beta} := \real{\alpha} \times \real{\beta}
\end{equation}
%%
The rules are tailored to fit this interpretation. They are
as follows.
%%
\begin{equation}
\begin{array}{c}
        \Gamma[\Delta_1 \circ (\Delta_2 \circ \Delta_3)]
    \bvdash \alpha : M \\\hline
        \Gamma[(\Delta_1 \circ \Delta_2) \circ \Delta_3]
    \bvdash \alpha : M
         \end{array}
\end{equation}
%%
This means that the restructuring of the term is without influence
on its meaning. Likewise we have
%%
\begin{equation}
\begin{array}{c}
        \Gamma[(\Delta_1 \circ \Delta_2) \circ \Delta_3]
    \bvdash \alpha : M
    \\\hline
        \Gamma[\Delta_1 \circ (\Delta_2 \circ \Delta_3)]
    \bvdash \alpha : M
        \end{array}
\end{equation}
%%%
So, for $\bullet$ we assume the following rule.
%%
\begin{equation}
\label{eq:bull}
\begin{array}{c}
        \Gamma[\alpha : x_{\alpha} \circ \beta : x_{\beta}]
    \bvdash \gamma : M \\\hline
        \Gamma[\alpha \bullet \beta] \bvdash \gamma :
    [p_1(z_{\alpha\bullet \beta})/x_{\alpha},
    p_2(z_{\alpha\bullet \beta})/x_{\beta}]M
        \end{array}
\end{equation}
%%
\eqref{eq:bull} says that in place of a function of two arguments
$\alpha$ and $\beta$ we can form a function of a single argument
of type $\alpha \bullet \beta$. The two arguments we can
recover by application of the projection functions. The fourth
rule finally tells us how the type/category $\alpha \bullet \beta$
is interpreted.
%%
\begin{equation}
\begin{array}{c}
        \Gamma \bvdash \alpha : M \qquad
        \Delta \bvdash \beta : N \\\hline
        \Gamma \circ \Delta \bvdash \alpha \bullet \beta :
    \auf M, N\zu
        \end{array}
\end{equation}
%%
Here we have the same problem as before with $\mathsf{AB}$. The meaning
assignments that are being computed are not in full
accord with the interpretation.  The term $\mbox{\tt (x}_0\mbox{\tt %
(x}_1\mbox{\tt x}_2\mbox{\tt ))}$ does not denote the same
function as $\mbox{\tt ((x}_0\mbox{\tt x}_1\mbox{\tt
)x}_2\mbox{\tt )}$. (Usually, one of them is not even well
defined.) So this raises the question whether it is at all
legitimate to proceed in this way. We shall avoid the question by
introducing a totally different calculus, sign based $\mathsf{L}$ 
%%%
\index{$\mathsf{L}$}%%%
%%%
(see Table~\ref{tab:lz}), which builds on the calculus $\mathsf{N}$ of the
previous section. The rules (\textbf{ass1}) and (\textbf{ass2}) are
dropped. Furthermore, ($\bullet$--\textbf{I}) is restricted to
$\Gamma = \varnothing$. These restrictions are taken over from
$\mathsf{N}$ for the abstraction rules.
%%
\begin{table}
\caption{The Sign Based Calculus $\mathsf{L}$}%%
\label{tab:lz}
$$\begin{array}{l@{\quad}l}
\mbox{(ax)} & \vec{x} : \alpha : x_{\zeta}
    \bvdash \vec{x} : \alpha : x_{\zeta}, 
\qquad\qquad
\zeta = \sigma(\alpha) 
                \\
\mbox{(\textbf{I}--{\mtt{\tf}})} & \begin{array}{c}
                \Gamma \circ \vec{x} : \alpha : x_{\zeta}
        \bvdash \vec{y} : \beta : N
        \\\hline
                \Gamma \bvdash \vec{y}/ \vec{x} : \beta/\alpha :
        \lambda x_{\zeta}. N
              \end{array} \qquad
    \begin{array}{l} 
	\mbox{$x_{\zeta}$ an argument variable,}  \\
	\focc(x_{\zeta},N) \leq 1 
	\end{array} \\
\mbox{(\textbf{I}--{\mtt{\tb}})} & \begin{array}{c}
                \vec{x} : \alpha : x_{\zeta} \circ \Gamma
        \bvdash  \vec{y} : \beta : N
        \\\hline
                \Gamma \bvdash \vec{x}\backslash \vec{y} :
        \alpha \backslash \beta :
        \lambda x_{\zeta}.N
                \end{array} \qquad
	\begin{array}{l}
              \mbox{$x_{\zeta}$ an argument variable,} \\ 
		\focc(x_{\zeta},N) \leq 1 
	\end{array} \\
\mbox{(\textbf{E}--{\mtt{\tf}})} & \begin{array}{c}
        \Gamma \bvdash \vec{x} : \alpha : M
    \qquad \Delta \bvdash \vec{y} : \beta/\alpha : N \\\hline
            \Delta \circ \Gamma \bvdash \vec{y}\conc \vec{x} : \beta :
        NM
              \end{array} \\
\mbox{(\textbf{E}--{\mtt{\tb}})} & \begin{array}{c}
        \Gamma \bvdash \vec{x} : \alpha : M \qquad
    \Delta \bvdash \vec{y} : \alpha\backslash \beta : N \\\hline
            \Gamma \circ \Delta \bvdash \vec{x} \conc \vec{y} : \beta :
        NM
                \end{array} \\
\mbox{($\bullet$--\textbf{I})} &
    \begin{array}{c}
        \vec{x} : \alpha : x_{\zeta} \circ
        \vec{y} : \beta : y_{\eta} \bvdash \vec{z} : \gamma : M
    \\\hline
            \vec{x}\conc \vec{y} : \alpha \bullet \beta :
        z_{\zeta \bullet \eta} \bvdash \vec{z} : \gamma
        : [p_1(z_{\zeta \bullet \eta})/x_{\zeta},
        p_2(z_{\zeta \bullet \eta})/y_{\eta}]M
                \end{array} \\
\mbox{(\textbf{I}--$\bullet$)} & \begin{array}{c}
        \Gamma \bvdash \vec{x} : \alpha : M \qquad
    \Delta \bvdash \vec{y} : \beta : N \\\hline
            \Gamma \circ \Delta \bvdash \vec{x} \conc \vec{y} :
        \alpha \bullet \beta :
        \auf M, N\zu
                \end{array} \\
\end{array}$$
\end{table}
%%
Sign based $\mathsf{L}$ has the global side condition that no variable 
is used in two different leaves. This condition can be replaced (up to 
$\alpha$--conversion) by the condition that all occurring terms are 
linear. In turn, this can be implemented by the adding suitable side 
conditions on the rules. 

Sign based $\mathsf{L}$ is not as elegant as plain categorial $\mathsf{L}$. 
However, it is semantically correct. If one desperately wants to have 
associativity, one has to introduce combinators at the right hand side. 
So, a use of the associativity rule is accompanied in the semantics by 
a use of $\mathsf{C}$ with $\mathsf{C}MNP = M\mbox{\tt (} NP\mbox{\tt )}$. 
We shall not spell out the details here.
%%
\vplatz
\exercise
Assume that in place of sequents of the form $\alpha \bvdash \alpha$ for
arbitrary $\alpha$ only sequents $c \bvdash c$, $c \in C$,
are axioms. Show that with the rules of $\mathsf{L}$
$\alpha \bvdash \alpha$ can be derived for every
$\alpha$.
%%
%\vplatz
%\exercise
%Prove Theorem~\ref{gruppenwertig}.
%%
\vplatz
\exercise
Let $G = \auf C, S, A, \zeta, \mathsf{NL}^-\zu$ be a
categorial sequent grammar. Show that the language
$\{\vec{x} : \; \vdash_G \vec{x}\}$ is context free.
%%
\vplatz \exercise%%
Show that the sequent $\alpha/\beta \circ \beta/\gamma \bvdash
\alpha/\gamma$ is derivable in $\mathsf{L}$ but not in $\mathsf{AB}$.
What semantics does the structure $\alpha/\beta \circ
\beta/\gamma$ have?
%%%
\vplatz
\exercise
%%%
\index{loop}%%
%%%%
A \textbf{loop} is a structure $\auf L, \cdot, \backslash, /\zu$
where $\Omega(\cdot) = \Omega(\backslash) = \Omega(/)= 2$ and
the following equations hold for all $x, y \in L$.
%%
\begin{equation}
x \cdot (x\backslash y) = y, \quad (y/x) \cdot x = y
\end{equation}
%%
The categories do not form a loop with respect to $\backslash$,
$/$ and $\cdot$ (!), for the reason that $\cdot$ is only
partially defined. Here is a possible remedy. Define
$\approx\,  \subseteq \Cat_{{\smtt{\tb}}, \bullet, {\smtt{\tf}}}(C)^2$ 
to be the least congruence such that
%%
\begin{equation}
(\alpha \bullet \beta)/\beta \approx \alpha, \qquad
\beta\backslash(\beta\bullet \alpha) \approx \alpha
\end{equation}
%%
Show that the free algebra of categories over $C$ factored by
$\approx$ is a loop. What is $\alpha \cdot \beta$ in the factored
algebra?
%%%
\vplatz 
\exercise%%
Show that the following rules are admissible in $\mathsf{L}$.
%%
\begin{equation}
\mbox{($\bullet$--\textbf{E})}\
\begin{array}{c}
\Gamma[\theta_1 \bullet \theta_2] \bvdash \alpha \\\hline
\Gamma[\theta_1 \circ \theta_2]   \bvdash \alpha
\end{array}
\quad \mbox{(\textbf{E}--{\mtt{\tf}})}\
\begin{array}{c}
\Gamma \bvdash \alpha/\beta \\\hline \Gamma \circ \beta \bvdash
\alpha
\end{array}
\quad \mbox{(\textbf{E}--{\mtt{\tb}})}\
\begin{array}{c}
\Gamma \bvdash \beta\backslash \alpha \\\hline \beta \circ \Gamma
\bvdash \alpha
\end{array}
\end{equation}

 \section{Pentus' Theorem}
%
%
%
It was conjectured by Noam Chomsky that the languages generated by 
$\mathsf{L}$ are context free, which means that $\mathsf{L}$ is in effect 
not stronger than $\mathsf{AB}$. This was first shown by Mati Pentus 
%%%
\index{Pentus, Mati}%%
%%%
(see \cite{pentus:lambek}). His proof makes use of the fact that 
$\mathsf{L}$ has interpolation. We start with a simple observation. 
Let $\GG := \auf G, \cdot, ^{-1}, 1\zu$ be a group and 
$\gamma \colon C \pf G$. We extend $\gamma$ 
to all types and structures as follows.
%%
\begin{equation}
\begin{split}
\gamma(\alpha \bullet \beta) & := \gamma(\alpha) \cdot \gamma(\beta) \\
\gamma(\alpha / \beta) & := \gamma(\alpha) \cdot \gamma(\beta)^{-1} \\
\gamma(\beta \backslash \alpha) & := \gamma(\beta)^{-1} \cdot
        \gamma(\alpha) \\
\gamma(\Gamma \circ \Delta) & := \gamma(\Gamma) \cdot \gamma(\Delta)
\end{split}
\end{equation}
%%
We call $\gamma$ a \textbf{group valued interpretation}.
%%%
\index{interpretation!group valued}%%
%%%
%%
\begin{thm}[Roorda]
%%%%
\index{Roorda, Dirk}%%
%%%
\label{gruppenwertig}
If $\Gamma \bvdash \alpha$ is derivable in $\mathsf{L}$ then for all 
group valued interpretations $\gamma$ $\gamma(\Gamma) = \gamma(\alpha)$.
\end{thm}
%%
The proof is performed by induction over the length of the derivation
and is left as an exercise. Let $C$ be given and $c \in C$. For a
category $\alpha$ over $C$ we define
%%
\begin{equation}
\begin{split}
\sigma_c(c') & := 
    \begin{cases}
    1  & \text{ if $c = c'$,} \\
    0  & \text{ otherwise.}
    \end{cases} \\
\sigma_c(\alpha \bullet \beta) & := \sigma_c(\alpha) + \sigma_c(\beta) \\
\sigma_c(\alpha/\beta) & := \sigma_c(\alpha) + \sigma_c(\beta) \\
\sigma_c(\beta\backslash \alpha) & := \sigma_c(\alpha) + \sigma_c(\beta)
\end{split}
\end{equation}
%%
Likewise we define
%%
\begin{align}
| \alpha | & := \sum_{c \in C} \sigma_c(\alpha) \\
\pi(\alpha) & := \{c \in C : \sigma_c(\alpha) > 0\}
\end{align}
%%
These definitions are extended in the canonical way to
structures. Let $\Delta$ be a nonempty structure (that is, $\Delta \neq
\varepsilon$) and $\Gamma[-]$ a structure containing a marked
occurrence of a substructure. An \textbf{interpolant} %%%
%%%
\index{interpolant}%%
%%%
for a sequent $\Gamma[\Delta] \bvdash \alpha$ in a calculus
%%
$\CS$ with respect to  $\Delta$ is a category $\theta$ such that
%%
\begin{dingautolist}{192}
\item $\sigma_c(\theta) \leq \min \{\sigma_c(\Gamma) +
        \sigma_c(\alpha),\sigma_c(\Delta)\}$, for all $c \in C$,
\item $\Delta \bvdash \theta$ is derivable in $\CS$,
\item $\Gamma[\theta] \bvdash \alpha$ is derivable in $\CS$.
\end{dingautolist}
%%
In particular $\pi(\theta) \subseteq \pi(\Gamma \circ \alpha) \cap 
\pi(\Delta)$ if $\theta$ satisfies these conditions. We say that 
$\CS$ has \textbf{interpolation} %%%
%%%
\index{interpolation}%%
%%%
if for every derivable $\Gamma[\Delta] \bvdash \alpha$ there
exists an interpolant with respect to $\Delta$.

We are interested in the calculi $\mathsf{AB}$ and $\mathsf{L}$.
In the case of $\mathsf{L}$ we have to remark that in presence
of full associativity the interpolation property can be
formulated as follows. We deal with sequents of the form
$\Gamma \bvdash \alpha$ where $\Gamma$ is a sequence of categories.
If $\Gamma = \Theta_1, \Delta, \Theta_2$ with $\Delta \neq
\varepsilon$ then there is an interpolant with respect to
$\Delta$. For let $\Delta^{\circ}$ be a structure in $\circ$
which corresponds to $\Delta$ (after omitting all occurrences of
$\circ$). Then there exists a sequent $\Gamma^{\circ} \bvdash \alpha$
which is derivable and in which $\Delta^{\circ}$ occurs as a
substructure.

Interpolation is shown by induction on the derivation. In the case 
of an axiom there is nothing to show.
For there we have a sequent $\alpha \bvdash \alpha$ and the marked
structure $\Delta$ has to be $\alpha$. In this case $\alpha$ is
an interpolant. Now let us assume that the rule (\textbf{I}--{\mtt{\tf}}) 
has been applied to yield the final sequent. Further, assume that the
interpolation property has been shown for the premisses.
Then we have the following constellation.
%%
\begin{equation}
\begin{array}{c}
\Gamma[\Delta] \circ \alpha \bvdash \beta \\\hline
\Gamma[\Delta] \bvdash \beta/\alpha
\end{array}
\end{equation}
%%
We have to find an interpolant with respect to $\Delta$.
By induction hypothesis there is a formula $\theta$ such that
$\Gamma[\theta] \circ \alpha \bvdash \beta$ and
$\Delta \bvdash \theta$ are both derivable and
$\sigma_c(\theta) \leq \min\{\sigma_c(\Gamma \circ \alpha \circ \beta), %
\sigma_c(\Delta)\}$ for all $c \in C$. Then also 
$\Gamma[\theta] \bvdash \beta/\alpha$
and $\Delta \bvdash \theta$ are derivable and we have
$\sigma_c(\theta) \leq \min\{\sigma_c(\Gamma \circ
\beta/\alpha), \sigma_c(\Delta)\}$. Hence $\theta$ also is an
interpolant with respect to $\Delta$ in $\Gamma[\Delta] \bvdash %
\beta/\alpha$. The case of (\textbf{I}--{\mtt{\tb}}) is fully
analogous.

Now we look at the case that the last rule is ({\mtt{\tf}}--\textbf{I}).
%%
\begin{equation}
\begin{array}{c}
\Gamma \bvdash \beta \qquad \Delta[\alpha] \bvdash \gamma \\\hline
\Delta[\alpha/\beta \circ \Gamma] \bvdash \gamma
\end{array}
\end{equation}
%%
Choose a substructure $Z$ from $\Delta[\alpha/\beta \circ \Gamma]$.
Several cases have to be distinguished.
(1) $Z$ is a substructure of $\Gamma$, that is, $\Gamma =
\Gamma'[Z]$. Then there exists an interpolant $\theta$ for
$\Gamma'[Z] \bvdash \beta$ with respect to $Z$. Then
$\theta$ also is an interpolant for $\Delta[\alpha/\beta
\circ \Gamma'[Z]] \bvdash \gamma$ with respect to $Z$.
(2) $Z$ is disjoint with $\alpha/\beta \circ \Gamma$. Then 
we have $\Delta[\alpha] = \Delta'[Z, \alpha]$ (with two marked 
occurrences of structures) and there is an interpolant $\theta$ 
with respect to $Z$ for $\Delta'[Z,\alpha]
\bvdash \gamma$. Also in this case one calculates that
$\theta$ is the desired interpolant. (3) $Z = \alpha/\beta$.
By induction hypothesis there is an interpolant $\theta_{\ell}$
for $\Gamma \bvdash \beta$ with respect to $\Gamma$, as well as
an interpolant $\theta_r$ for $\Delta[\alpha] \bvdash
\gamma$ with respect to $\alpha$. Then $\theta := \theta_r/\theta_{\ell}$
is the interpolant. For we have
%%
\begin{align}
\sigma_c(\theta) & = \sigma_c(\theta_r) + \sigma_c(\theta_{\ell}) \\\notag
                 & \leq 
	\min\{\sigma_c(\Delta \circ \gamma),\sigma_c(\alpha)\} +
\min\{\sigma_c(\beta), \sigma_c(\Gamma)\} \\\notag
                 & \leq 
\min\{\sigma_c(\Delta \circ \Gamma \circ \gamma),
\sigma_c(\alpha/\beta)\}
\end{align}
%%
Furthermore,
%%
\begin{equation}
\begin{array}{c}
\Gamma \bvdash \theta_{\ell} \qquad \Delta[\theta_r] \bvdash \gamma
\\\hline
\Delta[\theta_r/\theta_{\ell} \circ \Gamma] \bvdash \gamma
\end{array}\qquad\qquad
\begin{array}{c}
\theta_{\ell} \bvdash \beta \qquad \alpha \bvdash \theta_r \\\hline
\begin{array}{c}
\alpha/\beta \circ \theta_{\ell} \bvdash \theta_r \\\hline
\alpha/\beta \bvdash \theta_r / \theta_{\ell}
\end{array}
\end{array}
\end{equation}
%%
(4) $Z = \Theta[\alpha/\beta \circ \Gamma]$. Then $\Delta[\alpha/\beta
\circ \Gamma] = \Delta'[\Theta[\alpha/\beta \circ \Gamma]]$ for
some $\Delta'$. Then by hypothesis there is an interpolant for
$\Delta'[\Theta[\alpha]] \bvdash \gamma$ with respect to
$\Theta[\alpha]$. We show that $\theta$ is the desired
interpolant.
%%
\begin{equation}
\begin{array}{c}
\Gamma \bvdash \beta \qquad \Theta[\alpha] \bvdash \theta \\\hline
\Theta[\alpha/\beta \circ \Gamma] \bvdash \theta
\end{array}
\qquad\qquad
\Delta'[\theta] \bvdash \gamma
\end{equation}
%%
In addition 
%%
\begin{align}
\sigma_c(\theta) & \leq  \min\{%
\sigma_c(\Delta' \circ \gamma), \sigma_c(\Theta[\alpha])\} \\\notag
	& \leq  \min\{\sigma_c(\Delta' \circ \gamma),
\sigma_c(\Theta[\alpha/\beta \circ \Gamma])\}
\end{align} 
%%
This ends the proof for the case ({\mtt{\tf}}--\textbf{I}). The case 
({\mtt{\tb}}--\textbf{I}) again is fully analogous.
%%
\begin{thm}
$\mathsf{AB}$ has interpolation.
\proofend
\end{thm}
%%
Now we move on to $\mathsf{L}$. Clearly, we only have to discuss the
new rules. Let us first consider the case where we add $\bullet$ 
together with its introduction rules. Assume that the last rule is 
($\bullet$--\textbf{I}).
%%
\begin{equation}
\begin{array}{c}
\Gamma[\alpha \circ \beta] \bvdash \gamma \\\hline
\Gamma[\alpha \bullet \beta] \bvdash \gamma
\end{array}
\end{equation}
%%
Choose a substructure $Z$ of $\Gamma[\alpha \circ \beta]$.
(1) $Z$ does not contain the marked occurrence of  $\alpha
\bullet \beta$. Then $\Gamma[\alpha \bullet \beta]
= \Gamma'[Z, \alpha \bullet \beta]$, and by induction hypothesis
we get an interpolant $\theta$ for $\Gamma'[Z, \alpha \circ \beta]
\bvdash \gamma$ with respect to $Z$. It is easily checked that
$\theta$ also is an interpolant for $\Gamma'[Z, \alpha \bullet \beta] %
\bvdash \gamma$ with respect to $Z$.
(2) Let $Z = \Theta[\alpha \bullet \beta]$. Then
$\Gamma[\alpha \bullet \beta] = \Gamma'[\Theta[\alpha \bullet
\beta]]$. By induction hypothesis there is an interpolant $\theta$
for $\Gamma'[\Theta[\alpha \circ \beta]] \bvdash \gamma$ with respect
to $\Theta[\alpha \circ \beta]$, and it also is an interpolant for
$\Gamma'[\Theta[\alpha \bullet \beta]] \bvdash \gamma$
with respect to $\Theta[\alpha \bullet \beta]$.
In both cases we have found an interpolant.

Now we turn to the case (\textbf{I}--$\bullet$).
%%
\begin{equation}
\begin{array}{c}
\Gamma \bvdash \alpha \qquad \Delta \bvdash \beta \\\hline
\Gamma \circ \Delta \bvdash \alpha \bullet \beta
\end{array}
\end{equation}
%%
There are now three cases for $Z$. (1) $\Gamma = \Gamma'[Z]$.
By  induction hypothesis there is an interpolant $\theta_{\ell}$
for $\Gamma'[Z] \bvdash \alpha$ with respect to $Z$. This is the
desired interpolant. (2) $\Delta = \Delta'[Z]$. Analogous to (1).
(3) $Z = \Gamma \circ \Delta$. By hypothesis there is an interpolant
$\theta_{\ell}$ for $\Gamma \bvdash \alpha$ with respect to
$\Gamma$ and an interpolant $\theta_r$ for $\Delta
\bvdash \beta$ with respect to $\Delta$. Put $\theta :=
\theta_{\ell} \bullet \theta_r$. This is the desired
interpolant. For
%%
\begin{equation}
\begin{array}{c}
\Gamma \bvdash \theta_{\ell} \qquad \Delta \bvdash \theta_r \\\hline
\Gamma \circ \Delta \bvdash \theta_{\ell} \bullet \theta_r
\end{array}
\qquad \qquad
\begin{array}{c}
\theta_{\ell} \bvdash \alpha \qquad \theta_r \bvdash \beta \\\hline
\begin{array}{c}
\theta_{\ell} \circ \theta_r \bvdash \alpha \bullet \beta \\\hline
\theta_{\ell} \bullet \theta_r \bvdash \alpha \bullet \beta
\end{array}
\end{array}
\end{equation}
%%
In addition it is calculated that
$\sigma_c(\theta) \leq \min\{\sigma_c(\alpha \bullet
\beta), \sigma_c(\Gamma \circ \Delta)\}$.

This concludes a proof of interpolation for $\mathsf{NL}$.
Finally we must study $\mathsf{L}$. The rules (\textbf{ass1}), 
(\textbf{ass2}) pose a technical problem since we cannot 
proceed by induction on the derivation. For the applications 
of these rules change the structure. Hence we change to another 
system of sequents and turn --- as discussed above --- to 
sequents of the form $\Gamma \bvdash \alpha$ where $\Gamma$ is 
a sequence of categories.
In this case the rules (\textbf{ass1}) and (\textbf{ass2}) must be
eliminated. However, in the proof we must make more distinctions
in cases. The rules (\textbf{I}--{\mtt{\tf}}) and (\textbf{I}--{\mtt{\tb}})
are still unproblematic. So we look at a more complicated case,
namely an application of the rule ({\mtt{\tf}}--\textbf{I}).
%%
\begin{equation}
\begin{array}{c}
\Gamma \bvdash \beta \qquad \Delta[\alpha] \bvdash \gamma
\\\hline
\Delta[\alpha/\beta, \Gamma] \bvdash \gamma
\end{array}
\end{equation}
%%
We can segment the structure $\Delta[\alpha/\beta, \Gamma]$
into $\Delta', \alpha/\beta, \Gamma, \Delta''$.
Let a subsequence $Z$ be distinguished in $\Delta', \alpha/\beta, \Gamma,
\Delta''$. The case where $Z$ is fully contained in $\Delta'$
is relatively easy; likewise the case where $Z$ is fully contained
in $\Delta''$. The following cases remain.
(1) $Z = \Delta_1, \alpha/\beta, \Gamma_1$,
where $\Delta' = \Delta_0, \Delta_1$ for some $\Delta_0$,
and $\Gamma = \Gamma_1, \Gamma_2$ for some $\Gamma_2$.
Even if $Z$ is not empty $\Delta_1$ as well as
$\Gamma_1$ may be empty. Assume  $\Gamma_1 \neq \varepsilon$.
In this case $\theta_{\ell}$ an interpolant
for $\Gamma \bvdash \beta$ with respect to $\Gamma_2$ and
$\theta_r$ an interpolant of $\Delta[\alpha] \bvdash \gamma$
with respect to $\Delta_1, \alpha$. (Here it becomes clear why we
need not assume $\Delta_1 \neq \varepsilon$.)
The following sequents are therefore derivable.
%%
\begin{equation}
\begin{array}{r@{}l@{\qquad}r@{}l}
\Gamma_2 & \bvdash \theta_{\ell} & \Gamma_1, \theta_{\ell} & \bvdash \beta \\
\Delta_1, \alpha & \bvdash \theta_r & \Delta_0, \theta_r, \Delta''
    & \bvdash \gamma
\end{array}
\end{equation}
%%
Now put $\theta := \theta_r/\theta_{\ell}$. Then we have
on the one hand
%%
\begin{equation}
\begin{array}{c}
\Delta_1, \alpha \bvdash \theta_r \qquad \Gamma_1, \theta_{\ell} \bvdash
    \beta \\\hline
\Delta_1, \alpha/\beta, \Gamma_1, \theta_{\ell} \bvdash \theta_r
    \\\hline
\Delta_1, \alpha/\beta, \Gamma_1 \bvdash \theta_r/\theta_{\ell}
\end{array}
\end{equation}
%%%%
and on the other 
%%%%
\begin{equation}
\begin{array}{c}
\Gamma_2 \bvdash \theta_{\ell} \qquad \Delta_0, \theta_r,
    \Delta ''\bvdash \gamma \\\hline
\Delta_0, \theta_r/\theta_{\ell}, \Gamma_2, \Delta'' \bvdash
\gamma
\end{array}
\end{equation}
%%
The conditions on the numbers of occurrences of symbols are
easy to check. (2) As Case (1), but $\Gamma_1$ is empty.
Let then $\theta_{\ell}$ be an interpolant for $\Gamma \bvdash
\beta$ with respect to $\Gamma$ and $\theta_r$ an interpolant
for $\Delta_0, \Delta_1, \alpha, \Delta'' \bvdash \gamma$
with respect to $\Delta_1, \alpha$. Then put $\theta := \theta_r/
\theta_{\ell}$. $\theta$ is an interpolant for the end sequent
with respect to $Z$.
%%
\begin{equation}
\begin{array}{c}
\theta_{\ell} \bvdash \beta \qquad \Delta_1, \alpha \bvdash \theta_r
    \\\hline
\Delta_1, \alpha/\beta, \theta_{\ell} \bvdash \theta_r \\\hline
\Delta_1, \alpha/\beta \bvdash \theta_r / \theta_{\ell}
\end{array}
\qquad
\begin{array}{c}
\Gamma \bvdash \theta_{\ell} \qquad \Delta_0, \theta_r, \Delta''
    \bvdash \gamma \\\hline
\Delta_0, \theta_r /\theta_{\ell}, \Gamma, \Delta'' \bvdash
    \gamma
\end{array}
\end{equation}
%%%
(3) $Z$ does not contain the marked occurrence of $\alpha/\beta$.
In this case $Z = \Gamma_2, \Delta_1$ for some final part $\Gamma_2$
of $\Gamma$ and an initial part $\Delta_1$ of $\Delta''$. $\Gamma_2$
as well as  $\Delta_1$ may be assumed to be nonempty, since otherwise
we have a case that has already been discussed. The situation is
therefore as follows with $Z = \Gamma_2, \Delta_1$.
%%
\begin{equation}
\begin{array}{c}
\Gamma_1, \Gamma_2 \bvdash \beta \qquad \Delta', \alpha,
\Delta_1, \Delta_2 \bvdash \gamma \\\hline
\Delta', \alpha/\beta, \Gamma_1, \Gamma_2, \Delta_1,
\Delta_2 \bvdash \gamma
\end{array}
\end{equation}
%%
Let $\theta_{\ell}$ be an interpolant for $\Gamma_1, \Gamma_2 %
\bvdash \beta$ with respect to $\Gamma_2$ and $\theta_r$ an
interpolant for $\Delta', \alpha, \Delta_1, \Delta_2 \bvdash \gamma$
with respect to $\Delta_1$. Then the following are derivable
%%
\begin{equation}
\begin{array}{r@{}l@{\qquad}r@{}l}
\Gamma_2 & \bvdash \theta_{\ell} & \Gamma_1, \theta_{\ell}
    & \bvdash \beta \\
\Delta_1 & \bvdash \theta_r & \Delta', \alpha, \theta_r, \Delta_2
    & \bvdash \gamma
\end{array}
\end{equation}
%%
Now we choose $\theta := \theta_{\ell} \bullet \theta_r$.
Then we have both
%%
\begin{equation}
\begin{array}{c}
\Gamma_2 \bvdash \theta_{\ell} \qquad \Delta_1 \bvdash \theta_r
    \\\hline
\Gamma_2, \Delta_1 \bvdash \theta_{\ell} \bullet \theta_r
\end{array}
\end{equation}
%%%
and
%%%
\begin{equation}
\begin{array}{c}
\Gamma_1, \theta_{\ell} \bvdash \beta \qquad \Delta', \alpha,
    \theta_r, \Delta_2 \bvdash \gamma \\\hline
\Delta', \alpha/\beta, \Gamma_1, \theta_{\ell}, \theta_r,
    \Delta_2 \bvdash \gamma \\\hline
\Delta', \alpha/\beta, \Gamma_1, \theta_{\ell} \bullet
    \theta_r, \Delta_2 \bvdash \gamma
    \end{array}
\end{equation}
%%
In this case as well the conditions on numbers of occurrences are
easily checked. This exhausts all cases. Notice that we have used
$\bullet$ to construct the interpolant.  In the case of the rules
(\textbf{I}--$\bullet$) and ($\bullet$--\textbf{I}) there are no
surprises with respect to $\mathsf{AB}$.
%%
\begin{thm}[Roorda]
%%%
\index{Roorda, Dirk}%%%
%%%
$\mathsf{L}$ has interpolation.
\proofend
\end{thm}
%%
Now we shall move on to show that $\mathsf{L}$ is context
free. To this end we introduce a series of weak calculi of which we
shall show that together they are not weaker than $\mathsf{L}$.
These calculi are called $\mathsf{L}_m$, $m < \omega$. 
%%%
\index{$L_m$}%%%
%%%
The axioms of $\mathsf{L}_m$ are sequents $\Gamma \bvdash \alpha$ such that
the following holds.
%%
\begin{dingautolist}{192}
\item $\Gamma = \beta_1, \beta_2$ or $\Gamma = \beta_1$
for certain categories $\beta_1$ and $\beta_2$.
\item $\Gamma \bvdash \alpha$ is derivable in $\mathsf{L}$.
\item $|\alpha|, |\beta_1|, |\beta_2| < m$.
\end{dingautolist}
%%
(cut) is the only rule of inference. The main work is in the
proof of the following theorem.
%%
\begin{thm}[Pentus]
\label{reduktion1}
\index{Pentus, Mati}%%%
Let $\Gamma = \beta_0,  \beta_1,  \dotsc, \beta_{n-1}$.
$\Gamma \bvdash \alpha$ is derivable in $\mathsf{L}_m$ iff
%%
\begin{dingautolist}{192}
\item
$|\beta_i| < m$ for all $i < m$,
\item
$|\alpha| < m$ and
\item
$\Gamma \bvdash \alpha$ is derivable in $\mathsf{L}$.
\end{dingautolist}
\end{thm}
%%
We shall show first how to get from this fact that $\mathsf{L}$--grammars
are context free. We weaken the calculi still further. The calculus
$\mathsf{L}_m^{\boxminus}$ 
%%%%
\index{$L_m^{\boxminus}$}%%%
%%%
has the axioms of $\mathsf{L}_m$ but
(cut) may be applied only if the left hand premiss is an axiom.
%%
\begin{lem}
\label{reduktion2}
For all sequents $\Gamma \bvdash \alpha$ the following holds:
$\Gamma \bvdash \alpha$ is derivable in $\mathsf{L}_m^{\boxminus}$
iff $\Gamma \bvdash \alpha$ is derivable in $\mathsf{L}_m$.
\end{lem}
%%
The proof is relatively easy and left as an exercise.
%%
\begin{thm}
The languages accepted by $\mathsf{L}$--grammars are context free.
\end{thm}
%%
\proofbeg
Let $\BL = \auf \mbox{\tt S}, C, \zeta, A, \mathsf{L}\zu$ be given. 
Let $m$ be larger than the maximum of all $|\alpha|$, $\alpha
\in \zeta(a)$, $a \in A$. Since $A$ as well as $\zeta(a)$
are finite, $m$ exists. For simplicity we shall assume that
$C = \bigcup \auf \pi(\alpha) : \alpha \in \zeta(a),
a \in A\zu$. Now we put $N := \{\alpha : |\alpha| < m\}$.
$G := \auf \mbox{\tt S}, N, A, R\zu$, where
%%
\begin{align}
\begin{split}
R := & \phantom{\mbox{}\cup\mbox{}}\{\alpha \pf a : a \in \zeta(a)\} \\
     & \cup \{\alpha \pf \beta : \alpha, \beta \in N,
     \stackrel{\mathsf{L}}{\leadsto} \beta \bvdash \alpha\} \\
     & \cup \{\alpha \pf \beta_0 \beta_1 : 
	\alpha, \beta_0, \beta_1 \in N, 
	\stackrel{\mathsf{L}}{\leadsto} 
	\beta_0, \beta_1 \bvdash \alpha\}
\end{split}
\end{align}
%%
Now let $\BL \vdash \vec{x}$, $\vec{x} = x_0 \conc x_1 \conc
\dotsb x_{n-1}$. Then for all $i < n$ there exist 
an $\alpha_i \in \zeta(x_i)$ such that 
$\Gamma \bvdash \mbox{\tt S}$ is derivable 
in $\mathsf{L}$, where $\Gamma := \alpha_0, \alpha_1, \dotsc, \alpha_{n-1}$. 
By Theorem~\ref{reduktion1} and Lemma~\ref{reduktion2} $\Gamma \bvdash
\mbox{\tt S}$ is also derivable in $\mathsf{L}_m^{\boxminus}$. Induction
over the length of the derivation yields that $\vdash_G \alpha_0
\conc \alpha_1 \conc \dotsb \conc \alpha_{n-1}$ and hence also
$\vdash_G \vec{x}$. Now let conversely $\vdash_G \vec{x}$. We
extend the category assignment $\zeta$ to $\zeta^+ \colon A \cup N \pf
\Cat_{\mbox{\smtt{\tb}}, \bullet, \mbox{\smtt{\tf}}}(C)$ by putting
$\zeta^+(\alpha) := \{\alpha\}$ while $\zeta^+ \restriction A =
\zeta$. By induction over the length of the derivation of
$\vec{\alpha}$ one shows that from $\vdash_G \vec{\alpha}$ we get
$\BL \vdash \vec{\alpha}$. 
\proofend

Now on to the proof of Theorem~\ref{reduktion1}.
%%
\begin{defn}
%%%
\index{category!thin}%%
\index{sequent!thin}%%
\index{thin category}%%
%%%
A category $\alpha$ is called \textbf{thin} if $\sigma_c(\alpha)
\leq 1$ for all $c \in C$. A sequent $\Gamma \bvdash \alpha$
is called \textbf{thin} if the following holds.
%%%
\begin{dingautolist}{192}
\item $\Gamma \bvdash \alpha$ is derivable in $\mathsf{L}$.
\item All categories occurring in $\Gamma$ as well as $\alpha$ are
thin.
\item $\sigma_c(\Gamma, \alpha) \leq 2$ for all $c \in C$.
\end{dingautolist}
%%%
\end{defn}
%%
For a thin category $\alpha$ we always have $|\alpha| =
|\pi(\alpha)|$. We remark that for a thin sequent only
$\sigma_c(\Gamma, \alpha) = 0$ or $= 2$ can occur since
$\sigma_c(\Gamma, \alpha)$ always is an even number in
a derivable sequent (see Exercise~\ref{ex:gerade}). 
Let us look at a thin sequent
$\Gamma[\Delta] \bvdash \alpha$ and an interpolant $\theta$
of it with respect to $\Delta$. Then $\sigma_c(\theta) \leq %
\sigma_c(\Delta) \leq 1$. For either $c \not\in \pi(\Delta)$,
and then $c \not\in \pi(\theta)$, whence $\sigma_c(\theta) = 0$.
Or $c \in \pi(\Delta)$; but then $c \in \pi(\Gamma, \alpha)$,
and so by assumption $\sigma_c(\Delta) = 1$.
%%
\begin{equation}
\sigma_c(\Delta, \theta) \leq
\sigma_c(\Delta) + \sigma_c(\theta) \leq \sigma_c(\Gamma[\Delta],
\alpha) + \sigma_c(\theta) \leq 2 + 1
\end{equation}
%%
Now $\sigma_c(\Delta) + \sigma_c(\theta)$ is an even number
hence either $0$ or $2$. Hence $\Delta \bvdash \theta$
also is thin. Likewise it is shown that $\Gamma[\theta] \bvdash \alpha$
is thin.
%%
\begin{lem}
\label{unduenn}
Let $\Gamma, \Theta, \Delta \bvdash \alpha$ be a sequent
and $c, d \in C$ two distinct elementary categories.
Further, let $c \in \pi(\Gamma) \cap \pi(\Delta)$ as well as
$d \in \pi(\Theta) \cap \pi(\alpha)$. Then $\Gamma, \Theta, \Delta \bvdash
\alpha$ is not thin.
\end{lem}
%%
\proofbeg
Let $\GF_G(C)$ be the free group generated by the elementary categories.
The elements of this group are finite products of the form
$c_0^{s_0} \cdot c_2^{s_2} \cdot \dotsb \cdot c_{n-1}^{s_{n-1}}$,
where $c_i \neq c_{i+1}$ for $i < n-1$ and $s_i \in \BZ - \{0\}$.
(If $n = 0$ then the empty product denotes the group unit, 1.)
For if $c_0 = c_1$ the term $c_0^{s_0} \cdot c_1^{s_1}$ can
be shortened to $c_0^{s_0 + s_1}$. Look at the group valued 
interpretation $\gamma$ sending every element of $C$ to itself. 
If the sequent was thin we would have $\gamma(\Gamma) \cdot 
\gamma(\Theta) \cdot \gamma(\Delta) = \gamma(\alpha)$. By 
hypothesis the left hand side is of the form $w \cdot c^{\pm 1} 
\cdot x \cdot d^{\pm 1} \cdot y \cdot c^{\pm 1} \cdot z$ for certain 
products $w, x, y, z$. The right hand side equals $t \cdot d^{\pm 1}
\cdot u$ for certain $t, u$. Furthermore, we know that terms
which stand for $w$, $x$, $y$, $z$ as well as $t$ and $u$ cannot
contain $c$ or $d$ if maximally reduced. But then equality cannot
hold.
\proofend
%%
\begin{lem}
Let $\alpha_0, \alpha_1, \dotsc, \alpha_{n} \bvdash \alpha_{n+1}$ be 
thin, $n > 0$. Then there is a $k$ with $0 < k < n + 1$ and
$\pi(\alpha_k) \subseteq \pi(\alpha_{k-1}) \cup \pi(\alpha_{k+1})$.
\end{lem}
%%
\proofbeg
The proof is by induction on $n$. We start with $n = 1$. Here
the sequent has the form $\alpha_0, \alpha_1 \bvdash \alpha_2$. Let
$c \in \pi(\alpha_1)$. Then $\sigma_c(\alpha_1) = 1$ since the
sequent is thin. And since $\sigma_c(\alpha_0, \alpha_1, \alpha_2) = 2$,
we have $\sigma_c(\alpha_0, \alpha_2) = 1$, whence
$c \in \pi(\alpha_0) \cup \pi(\alpha_2)$. This finishes the case
$n = 1$. Now let $n > 1$ and the claim proved for all $m < n$. 
\textbf{Case a.} $\pi(\alpha_0, \alpha_1, \dotsc, %
\alpha_{n-2}) \cap \pi(\alpha_{n}) = \varnothing$. Then we choose
$k := n$. For if $c \in \pi(\alpha_{n})$ then $\sigma_c(\alpha_0,
\dotsc, \alpha_{n-2}) = 0$, and so we have $\sigma_c(\alpha_{n-1})
+ \sigma_c(\alpha_{n+1}) = 1$. Hence we get $c \in \pi(\alpha_{n-1})
\cup \pi(\alpha_{n+1})$. \textbf{Case b.} $\pi(\alpha_0,
\alpha_1, \dotsc, \alpha_{n-2}) \cap \pi(\alpha_{n})
\neq \varnothing$. Then there exists an elementary category $c$ with
$c \in \pi(\alpha_0, \dotsc, \alpha_{n-2})$ and
$c \in \pi(\alpha_{n})$. Put $\Gamma := \alpha_0,
\alpha_1, \dotsc, \alpha_{n-1}$, $\Delta := \alpha_{n}$.
Let $\theta$ be an interpolant for $\Gamma, \Delta
\bvdash \alpha_{n+1}$ with respect to $\Gamma$. Then $\Gamma \bvdash
\theta$ and $\theta, \alpha_n \bvdash \alpha_{n+1}$ are thin.
By induction hypothesis there exists a $k$ such that
$\pi(\alpha_k) \subseteq \pi(\alpha_{k-1}) \cup \pi(\alpha_{k+1})$,
if $k < n - 1$, or $\pi(\alpha_k) \subseteq
\pi(\alpha_{k-1}) \cup \pi(\theta)$ in case $k = n-1$. If
$k < n-1$ then $k$ is the desired number for the main sequent.
Let now $k = n-1$. Then 
%%%
\begin{equation}
\pi(\alpha_{n-1}) \subseteq \pi(\alpha_{n-2}) \cup \pi(\theta) 
\subseteq \pi(\alpha_{n-2}) \cup \pi(\alpha_{n}) \cup \pi(\alpha_{n+1})
\end{equation}
%%%
We show that $k$ in this case too is the desired number for the main 
sequent. Let $\pi(\alpha_{n-1}) \cap \pi(\alpha_{n+1}) \neq \varnothing$,
say $d \in \pi(\alpha_{n-1}) \cap \pi(\alpha_{n+1})$. Then surely
$d \not\in \pi(\alpha_{n})$, so $d \neq c$. Therefore the sequent
is not thin, by Lemma~\ref{unduenn}. Hence we have
$\pi(\alpha_{n-1}) \cap \pi(\alpha_{n+1}) = \varnothing$, and so
$\pi(\alpha_{n-1}) \subseteq \pi(\alpha_{n-2}) \cup \pi(\alpha_{n})$.
\proofend
%%
\begin{lem}
\label{duennableitbar}
Let $\Gamma \bvdash \gamma$ be an $\mathsf{L}$--derivable thin sequent 
in which all categories have length $< m$. Then $\Gamma \bvdash \gamma$ 
is already derivable in $\mathsf{L}_m$.
\end{lem}
%%
\proofbeg
Let $\Gamma = \alpha_0, \alpha_1, \dotsc, \alpha_{n-1}$;
put $\alpha_n := \gamma$. If $n \leq 2$ then $\Gamma \bvdash \gamma$
already is an axiom of $\mathsf{L}_m$. So, let $n > 2$. By the
previous lemma there is a $k$ such that 
$\pi(\alpha_k) \subseteq \pi(\alpha_{k-1})
\cup \pi(\alpha_{k+1})$. \textbf{Case 1.} $k < n$. \textbf{Case 1a.}
$| \pi(\alpha_{k-1}) \cap \pi(\alpha_k)| \geq |\pi(\alpha_{k+1}) \cap
\pi(\alpha_k)|$. Put $\Xi := \alpha_0, \alpha_1, \dotsc,
\alpha_{k-2}$, $\Theta := \alpha_{k+1}, \dotsc, \alpha_{n-1}$, 
and $\Delta := \alpha_{k-1}, \alpha_k$. Let
$\theta$ be an interpolant for $\Xi, \Delta, \Theta
\bvdash \alpha_n$ with respect to $\Delta$. Then the sequent
%%
\begin{equation}
\alpha_0, \dotsc, \alpha_{k-2}, \theta,
\alpha_{k+1}, \dotsc, \alpha_{n-1} \bvdash \alpha_n
\end{equation}
%%
is thin. Furthermore
%%
\begin{align}
\pi(\theta) \subseteq & (\pi(\alpha_{k-1}) \cup \pi(\alpha_k))
\cap \pi(\Xi, \Theta, \alpha_n)  \\\notag
        = & 
(\pi(\alpha_{k-1}) \cap \pi(\Xi, \Theta, \alpha_n)) 
\cup (\pi(\alpha_k) \cap \pi(\Xi, \Theta, \alpha_n)).
\end{align}
%%
Let $c \in \pi(\alpha_{k-1})$. Then $\sigma_c(\alpha_{k-1}) = 1$
and $\sigma_c(\Xi, \alpha_{k-1}, \alpha_k, \Theta,
\alpha_n) = 2$, from which $\sigma_c(\Xi, \alpha_k, \Theta,
\alpha_n) = 1$. Hence either $\sigma_c(\alpha_k) = 1$ or
$\sigma_c(\Xi, \Theta, \alpha_n) = 1$. Since $c$ was arbitrary
we have 
%%%
\begin{equation}
\pi(\alpha_k) \cap \pi(\Xi, \Theta, \alpha_n) =
\pi(\alpha_{k-1}) - (\pi(\alpha_{k-1}) \cap \pi(\alpha_k))
\end{equation}
%%%
By choice of $k$, $\pi(\alpha_k) \cap \pi(\Xi, \Theta,
\alpha_n) = \pi(\alpha_k) \cap \pi(\alpha_{k+1})$. Hence 
%%
\begin{equation}
\begin{array}{l@{}l@{}l}
\pi(\theta) & = & 
(\pi(\alpha_{k-1}) \cap \pi(\Xi, \Theta, \alpha_n)) 
(\pi(\alpha_k) \cap \pi(\Xi, \Theta, \alpha_n)) \\
        & \subseteq &
(\pi(\alpha_k) - (\pi(\alpha_{k-1}) \cap \pi(\alpha_k)) 
   \cup (\pi(\alpha_k) \cap \pi(\alpha_{k+1})).
\end{array}
\end{equation}
%%
So
%%
\begin{align}
\begin{split}
|\pi(\theta)| & = |\pi(\alpha_{k-1})| + |\pi(\alpha_{k-1}) \cap
\pi(\alpha_k)| + |\pi(\alpha_k) \cap \pi(\alpha_{k+1})| \\
          & \leq 
|\pi(\alpha_{k-1})| \\
          & <  m
\end{split}
         \end{align} 
%%
(Note that $|\pi(\alpha_{k-1})| = |\alpha_k|$.)
Therefore also $|\theta| < m$ and so $\alpha_{k-1},
\alpha_k \bvdash \theta$ is an axiom of $\mathsf{L}_m$. Hence, 
by induction hypothesis $\Xi, \theta, \Theta \bvdash \alpha_n$
is derivable in $\mathsf{L}_m$. A single application from
both sequents yields the main sequent.  It is therefore derivable
in $\mathsf{L}_m$. \textbf{Case 1b.} $|\pi(\alpha_{k-1}) \cap %
\pi(\alpha_k)| < |\pi(\alpha_k) \cap \pi(\alpha_{l+1})|$. Here
one puts $\Xi := \alpha_0, \dotsc, \alpha_{k-1}$, $\Delta := %
\alpha_k, \alpha_{k+1}$, $\Theta := \alpha_{k+1}, \dotsc, %
\alpha_{n-1}$ and proceeds as in Case 1a. \textbf{Case 2.} $k = n-1$.
So, $\pi(\alpha_{n-1}) \subseteq \pi(\alpha_{n-2}) \cup
\pi(\gamma)$. Also here we distinguish to cases.
\textbf{Case~2a.} $|\pi(\alpha_{n-2}) \cap \pi(\alpha_{n-1})|
\geq |\pi(\alpha_{n-1}) \cap \pi(\alpha_n)|$. This case is similar
to Case 1a. \textbf{Case 2b.} $|\pi(\alpha_{n-2})
\cap \pi(\alpha_{n-1})| < |\pi(\alpha_{n-1}) \cap \pi(\alpha_n)|$.
Here put $\Delta := \alpha_0, \dotsc, \alpha_{n-2}$,
$\Theta := \alpha_{n-1}$. Let $\theta$ be an interpolant for
$\Delta, \Theta \bvdash \alpha_n$ with respect to $\Delta$.
Then $\Delta \bvdash \theta$ as well as $\theta, \alpha_{n-1}
\bvdash \alpha_n$ are thin. Further we have
%%
\begin{align}
\begin{split}
\pi(\theta) \subseteq & \pi(\Delta) \cap (\pi(\alpha_{n-1}) \cup
\pi(\alpha_n)) \\
	 = & (\pi(\Delta) \cap \pi(\alpha_{n-1})) \cup
(\pi(\Delta) \cap \pi(\alpha_n))  \\
        = & 
	(\pi(\alpha_{n-2}) \cap \pi(\alpha_{n-1})) 
	\cup
        (\pi(\alpha_n) - (\pi(\alpha_{n-1}) \cap \pi(\alpha_n))).
\end{split}
\end{align}
%%
As in Case 1a we conclude that
%%
\begin{align}
\begin{split}
|\pi(\theta)| & = 
    |\pi(\alpha_{n-2}) \cap \pi(\alpha_{n-1})| +
    |\pi(\alpha_n)| - |\pi(\alpha_{n-1}) \cap \pi(\alpha_n)| \\
          & < 
    |\pi(\alpha_n)| \\
          & < m
	\end{split}
\end{align}
%%
Hence $\theta, \alpha_{n-1} \bvdash \alpha_n$ is an axiom of
$\mathsf{L}_m$. By induction hypothesis, $\Delta \bvdash \theta$ 
is derivable in $\mathsf{L}_m$. A single application of (cut) yields 
the main sequent, which is therefore derivable in $\mathsf{L}_m$.
\proofend
%%

\noindent
Finally we proceed to the proof of Theorem~\ref{reduktion1}. Let
$|\gamma_i| < m$ for all $i < n$, and $|\alpha| < m$. Finally, let
$\gamma_0, \gamma_1, \dotsc, \gamma_{m-1} \bvdash \alpha$ be
derivable in $\mathsf{L}$. We choose a derivation of this sequent.
We may assume here that the axioms are only sequents of the
form $c \bvdash c$. For every occurrence of an axiom $c \bvdash c$
we choose a new elementary category $\wht{c}$ and replace this
occurrence of $c \bvdash c$ by $\wht{c} \bvdash \wht{c}$. We extend
this to the entire derivation and so we get a new derivation
of a sequent $\wht{\gamma}_0, \wht{\gamma}_1,
\dotsc, \wht{\gamma}_{n-1} \bvdash \wht{\alpha}$. We get
$\sigma_c(\wht{\alpha}) + \sum_{i < n} \sigma_c(\wht{\gamma_i}) = 2$,
if $c$ occurs at all in the sequent. Nevertheless, the sequent
need not be thin, since it may contain categories which are not
thin. However, if $\sigma_c(\delta) = 2$ for some $\delta$ and
some $c$, then $c$ is not contained in any other category.
We exploit this as follows. By successively applying interpolation
we get the following sequents, which are all derivable in $\mathsf{L}$.
%%
\begin{equation}
\begin{array}{r@{}l@{\qquad}r@{}l}
\wht{\gamma}_0 & \bvdash \theta_0 & \theta_0, \wht{\gamma}_1,
    \wht{\gamma}_2, \dotsc, \wht{\gamma}_{n-1}
    & \bvdash \wht{\alpha} \\
\wht{\gamma}_1 & \bvdash \theta_1 & \theta_0, \theta_1,
    \wht{\gamma}_2, \dotsc, \wht{\gamma}_{n-1}
    & \bvdash \wht{\alpha} \\
\vdots\qquad & \qquad \vdots \\
\wht{\gamma}_{n-1} & \bvdash \theta_{n-1} & \theta_0, \theta_1,
    \dotsc, \theta_{n-1} & \bvdash \wht{\alpha} \\
\theta_0, \theta_1, \dotsc, \theta_{n-1} & \bvdash \gamma &
        \gamma & \bvdash \wht{\alpha}
\end{array}
\end{equation}
%%
It is not hard to show that $\sigma_c(\theta_i) \leq 1$
for all $c$ and all $i < n$. So the sequent
$\theta_0, \theta_1, \dotsc, \theta_{n-1} \bvdash \gamma$
is thin. Certainly $|\gamma| \leq |\wht{\alpha}| = |\alpha| < m$
as well as $|\theta_i| \leq |\wht{\alpha}_i| = |\alpha_i| < m$
for all $i < n$. By Lemma~\ref{duennableitbar} the sequent
$\theta_0, \theta_1, \dotsc, \theta_{n-1} \bvdash \gamma$ is
derivable in $\mathsf{L}_m$. The sequents $\wht{\gamma}_i \bvdash %
\theta_i$, $i < n$, as well as $\gamma \bvdash \wht{\alpha}_n$ are 
axioms of $\mathsf{L}_m$. Hence $\wht{\gamma}_0, \wht{\gamma}_0, \dotsc,
\wht{\gamma}_{n-1} \bvdash \wht{\alpha}$ is derivable in $\mathsf{L}_m$.
We undo the replacement in the derivation. This can in fact
be done by applying a homomorphism (substitution) $t$ which
replaces $\wht{c}$ by $c$. So, we get a derivation of
$\gamma_0, \gamma_1, \dotsc, \gamma_{n-1} \bvdash \gamma_n$ in
$\mathsf{L}_m$. This concludes the proof of Theorem~\ref{reduktion1}.

We remark that Pentus has also shown in \cite{pentus:models} that 
%%%
\index{Pentus, Mati}%%%
%%%%
$\mathsf{L}$ is complete with respect to so--called L--frames. 
%%%
\begin{defn}
%%%
\index{L--frame}%%%
%%%
An \textbf{L--frame} is a free semigroup of the form $\auf A^+, \cdot\zu$. 
A \textbf{valuation} is a function $v : C \pf \wp(A^+)$. $v$ is 
extended to categories and sequents as follows:
%%%
\begin{equation}
\begin{split}
v(\mbox{\mtt ($\alpha\bullet\beta$)}) & := v(\alpha) \cdot v(\beta) \\
v(\mbox{\mtt ($\alpha${\tf}$\beta$)}) & := v(\alpha){/\!/} v(\beta) \\
v(\mbox{\mtt ($\beta${\tb}$\alpha$)}) & := 
	v(\beta){\backslash\!\backslash} v(\alpha) \\
v(\Gamma \circ \Delta) & := v(\Gamma) \cdot v(\Delta)
\end{split}
\end{equation} 
%%%
$\Gamma \bvdash \alpha$ is \textbf{true under} $v$ if $v(\Gamma) \subseteq 
v(\alpha)$. It is \textbf{valid in an L--frame} if it is true under all 
valuations.
\end{defn}
%%%
\begin{thm}[Pentus]
%%%
\index{Pentus, Mati}%%
%%%
$\stackrel{\mathsf{L}}{\leadsto} \Gamma \bvdash \alpha$ iff 
$\Gamma \bvdash \alpha$ is valid in all L--frames.
\end{thm}
%%
A survey of this subject area can be found in \cite{buszkowski:proof}.
%%
\vplatz
\exercise
Prove Theorem~\ref{gruppenwertig}.
%%
\vplatz
\exercise
\label{ex:gerade}
Let $\Gamma \bvdash \alpha$ be derivable in $\mathsf{L}$, $c \in C$. Show
that $\sigma_c(\Gamma) + \sigma_c(\alpha)$ is an even number.
%%
\vplatz
\exercise
Prove Lemma~\ref{reduktion2}.
%%%
\vplatz
\exercise
Show that if $\stackrel{\mathsf{L}}{\leadsto} \Gamma \bvdash \alpha$, 
$\Gamma \bvdash \alpha$ is valid in all L--frames. 

 \section{Montague Semantics I}
\index{Montague Semantics}%%%
\label{kap3-6}
%
%
%
Until the beginning of the 1970s semantics of natural languages was
considered a hopeless affair. Natural language was thought of as 
being completely illogical so that no formal theory 
of semantics for natural languages could ever be given. By contrast, 
Montague 
%%%
\index{Montague, Richard}%%%
%%%
believed that natural languages can be analysed in the same way as formal 
languages.  Even if this was too optimistic (and it is quite certain that
Montague did deliberately overstate his case) there is enough
evidence that natural languages are quite well--behaved. To prove
his claim, Montague considered a small fragment of English, for
whose semantics he produced a formal account. In this section we 
shall give a glimpse of the theory shaped by Montague. Before
we can start, we have to talk about predicate logics and its
models. For Montague has actually built his semantics somewhat
differently than we have done so far. In place of defining the
interpretation in a model directly, he defined a translation
into $\lambda$--calculus over predicate logic, whose interpretation
on the other hand is fixed by some general conventions.

%%%
\index{logic!first--order}%%
\index{first--order logic}%%
\index{FOL}%%
A language of first--order predicate logic with identity
has the following symbols:
%%
\begin{dingautolist}{192}
\item a set $R$ of relation symbols, a disjoint set $F$ of
    function symbols,
\item a countably infinite set $V := \{\mbox{\tt x}_i: i \in \omega\}$
    of variables,
\item the equality symbol {\tt =},
\item the booleans {\mtt\symbol{5}}, {\mtt\symbol{4}}, {\mtt\symbol{31}}, 
	{\mtt\symbol{25}}, 
\item the quantifiers {\mtt\symbol{20}}, {\mtt\symbol{21}}.
\end{dingautolist}
%%
As outlined in Section~\ref{kap1}.\ref{kap1-1}, the language is defined
by choosing a signature $\auf \Omega, \Xi\zu$. Then $r$ is a
$\Xi(r)$--ary relation symbol and $f$ a $\Omega(f)$--ary function
symbol. Equality is always a binary relation symbol (so,
$\Xi(\mbox{\tt =}) = 2$). We define the set of terms as usual.
Next we define formulae (see also Section~\ref{kap2}.\ref{kap2-6}).
%%
\begin{dingautolist}{192}
\item If $t_i$, $i < \Xi(r)$, are terms then
    {\tt $r$($t_0, \dotsc, t_{\Xi(r)-1}$)} is a formula.
\item If $t_0$ and $t_1$ are terms then {\mtt $t_0$\symbol{61}$t_1$}
    is a formula.
\item If $\varphi$ and $\psi$ are formulae, so are
    {\mtt (\symbol{5}$\varphi$)}, {\mtt ($\varphi$\symbol{4}$\psi$)},
    {\mtt ($\varphi$\symbol{31}$\psi$)} and 
	{\mtt ($\varphi$\symbol{25}$\psi$)}.
\item If $\varphi$ is a formula and $x \in V$, then
    {\mtt (\symbol{20}$x$)$\varphi$} and 
    {\mtt (\symbol{21}$x$)$\varphi$}
    are formulae.
\end{dingautolist}
%%
A $\auf \Omega, \Xi\zu$--{\bf structure} is a triple
$\auf M, \{f^{\GM} : f \in F\}, \{r^{\GM} : r \in R\}\zu$
such that $f^{\GM} \colon M^{\Omega(f)} \pf M$ for every $f \in F$
and $r^{\GM} \subseteq M^{\Xi(r)}$ for every $r \in R$. Now
let $\beta \colon V \pf M$. Then we define $\auf \GM, \beta\zu \vDash
\varphi$ for a formula by induction.  To begin, we associate with
every $t$ its value $[t]^{\beta}$ under $\beta$.
%%
\begin{align}
\begin{split}
\mbox{}[x]^{\beta} & := \beta(x) \\
\mbox{}[f(t_0, \dotsc, t_{\Omega(f)-1})]^{\beta} &
    := f^{\GM}([t_0]^{\beta}, \dotsc, [t_{\Omega(f)-1}]^{\beta})
\end{split}
\end{align}
%%
Now we move on to formulae. (In this definition, $\gamma \sim_x \beta$,
for $x \in V$, if $\beta(y) \neq \gamma(y)$ only if $y = x$.)
%%
\begin{align}
\begin{split}
\auf \GM, \beta\zu \vDash \mbox{\mtt ($s_0$\symbol{61}$s_1$)}
	 & :\Dpf [s_0]^{\beta} = [s_1]^{\beta} \\
\auf \GM, \beta\zu \vDash \mbox{\mtt $r$($\vec{s}$)} & :\Dpf
    \auf [s_i] : i < \Xi(r)\zu \in r^{\GM} \\
\auf \GM, \beta\zu \vDash \mbox{\mtt (\symbol{5}$\varphi$)} 
	& :\Dpf \auf \GM, \beta\zu \nvDash \varphi \\
\auf \GM, \beta\zu \vDash \mbox{\mtt ($\varphi$\symbol{4}$\psi$)}
	 & :\Dpf \auf \GM, \beta\zu \vDash \varphi \text{ and }
    \auf \GM, \beta\zu \vDash \psi \\
\auf \GM, \beta\zu \vDash \mbox{\mtt ($\varphi$\symbol{31}$\psi$)}
	 & :\Dpf \auf \GM, \beta\zu \vDash \varphi \text{ or }
    \auf \GM, \beta\zu \vDash \psi \\
\auf \GM, \beta\zu \vDash \mbox{\mtt ($\varphi$\symbol{25}$\psi$)} 
	& :\Dpf \auf \GM, \beta\zu \nvDash \varphi \text{ or }
    \auf \GM, \beta\zu \vDash \psi \\
\auf \GM, \beta\zu \vDash \mbox{\mtt (\symbol{21}$x$)$\varphi$} 
	& :\Dpf \text{ there is } \beta' \sim_x \beta:
    \auf \GM, \beta'\zu \vDash \varphi \\
\auf \GM, \beta\zu \vDash \mbox{\mtt (\symbol{20}$x$)$\varphi$} 
	& :\Dpf \text{ for all } \beta' \sim_x \beta:
    \auf \GM, \beta'\zu \vDash \varphi
\end{split}
\end{align}
%%
In this way formulae are interpreted in models.
%%
\begin{defn}
Let $\Delta$ be a set of formulae, and $\varphi$ a
formula. Then $\Delta \vDash \varphi$ if for all
models $\auf \GM, \beta\zu$: if $\auf \GM, \beta\zu
\vDash \delta$ for every $\delta \in \Delta$, then
also $\auf \GM, \beta\zu \vDash \varphi$.
\end{defn}
%%
For example, the arithmetical terms in {\tt +}, {\tt 0} and 
{\mtt\symbol{42}} with the relation {\mtt <} can be interpreted 
in the structure $\BN$ where $\mbox{\tt +}^{\BN} = +$ and 
$\mbox{\mtt\symbol{42}}^{\BN} = \cdot$ are the usual operations, 
$\mbox{\mtt 0}^{\BN} = 0$ and $\mbox{\tt <}^{\BN} = <$. Then for 
the valuation $\beta$ with $\beta(\mbox{\mtt x$_{\szwei}$}) = 7$ 
we have:
%%
\begin{equation}
\label{eq:38ast}
\auf \BN, \beta\zu \vDash
    \mbox{\mtt (\symbol{20}x$_{\snull}$)(\symbol{20}x$_{\seins}$)(%
(x$_{\snull}$\symbol{42}x$_{\seins}$)\symbol{61}x$_{\szwei}$\symbol{25}%
(x$_{\snull}$\symbol{61}1\symbol{31}x$_{\seins}$\symbol{61}1))}
\end{equation}
%%
This formula says that $\beta(\mbox{\mtt x$_{\szwei}$})$ is a 
prime number. For a number $w$ is a prime number iff for all numbers
$u$ and $v$: if $u \cdot v = w$ then $u = 1$ or $v = 1$.
We compare this with \eqref{eq:38ast}. \eqref{eq:38ast} holds if
for all $\beta'$ different only on \mbox{\mtt x$_{\snull}$} 
from $\beta$
%%
\begin{equation}
\auf \BN, \beta'\zu \vDash
    \mbox{\mtt (\symbol{20}x$_{\seins}$)((x$_{\snull}$\symbol{42}x%
$_{\seins}$)\symbol{61}x$_{\szwei}$\symbol{25}(x$_{\snull}$\symbol{61}%
1\symbol{31}x$_{\seins}$\symbol{61}1))} 
\end{equation}
%%
This in turn is the case if for all $\beta''$ different only on 
\mbox{\mtt x$_{\seins}$} from $\beta'$
%%
\begin{equation}
\auf \BN, \beta''\zu \vDash
    \mbox{\mtt ((x$_{\snull}$\symbol{42}x$_{\seins}$)\symbol{61}%
x$_{\szwei}$\symbol{25}(x$_{\snull}$\symbol{61}1\symbol{31}x%
$_{\seins}$\symbol{61}1))} 
\end{equation}
%%
This means: if $u := \beta''(\mbox{\mtt x$_{\snull}$})$, 
$v := \beta''(\mbox{\mtt x$_{\seins}$})$ and $w := \beta''(\mbox{\mtt %
x$_{\szwei}$})$ and if we have $w = u \cdot v$,
then $u = 1$ or $v = 1$. This holds for all $u$ and $v$.
Since on the other hand $w = \beta(\mbox{\mtt x$_{\szwei}$})$ 
we have \eqref{eq:38ast} iff $\beta(\mbox{\mtt x$_{\szwei}$})$, 
that is to say 7, is a prime number.

The reader may convince himself that for every $\beta$
%%
\begin{multline}
\auf \BN, \beta\zu \vDash
    \mbox{\mtt (\symbol{20}x$_{\snull}$)(\symbol{21}x$_{\seins}$)(%
\symbol{20}x$_{\szwei}$)(\symbol{20}x$_{\sdrei}$)} \\
\mbox{\mtt (x$_{\snull}$<x%
$_{\seins}$\symbol{4}((x$_{\szwei}$\symbol{42}x$_{\sdrei})$%
\symbol{61}x$_{\seins}$\symbol{25}(x$_{\szwei}$\symbol{61}1\symbol{31}x%
$_{\sdrei}$\symbol{61}1)))}
\end{multline}
%%
This says that for every number there exists a prime number
larger than it. 

For later use we introduce a type $e$.  This is the type of 
terms. $e$ is realized by $M$. Before we can start designing 
a semantics for natural language we shall have to eliminate 
the relations from predicate logic.  To this end we shall 
introduce a new basic type, $t$, which is the type of truth 
values. It is realized by the set $\{0,1\}$.  An $n$--place 
relation $r$ is now replaced by the characteristic function 
$r^{\spadesuit}$ from $n$--tuples of objects to truth values, 
which is defined as follows.
%%
\begin{equation}
r^{\spadesuit}(x_0, x_1, \dotsc, x_{\Xi(r)-1}) = 1
\text{ $:\Dpf$ } r(x_0, \dotsc, x_{\Xi(r)-1})
\end{equation}
%%
This allows us to use $\lambda$--calculus for handling
the argument places of $r$. For example, from the binary relation
$r$ we can define the following functions $r_1$ and $r_2$.
%%
\begin{align}
r_1 & := \lambda x_e. \lambda y_e. r^{\spadesuit}(x_e, y_e) \\
r_2 & := \lambda x_e. \lambda y_e. r^{\spadesuit}(y_e, x_e)
\end{align}
%%
So, we can define functions that either take the first argument
of $r^{\spadesuit}$ first, or one which takes the first argument
of $r^{\spadesuit}$ second.

Further, we shall also interpret {\mtt\symbol{5}}, {\mtt\symbol{4}}, 
{\mtt\symbol{31}} and {\mtt\symbol{25}} by the standard set--theoretic 
functions $-$, $\cap$, $\cup$ and $\supset$, respectively:
%%
\begin{equation}
\begin{array}{l|l}
  & - \\\hline
0 & 1 \\
1 & 0
\end{array}\qquad
\begin{array}{l|ll}
\cap & 0 & 1 \\\hline
0    & 0 & 0 \\
1    & 0 & 1
\end{array}\qquad \\
\begin{array}{l|ll}
\cup & 0 & 1 \\\hline
0     & 0 & 1 \\
1     & 1 & 1
\end{array}\qquad
\begin{array}{l|ll}
\supset & 0 & 1 \\\hline
0   & 1 & 1 \\
1   & 0 & 1
\end{array}
\end{equation}
%%
\newcommand{\GPi}{\mbox{\tt\textgreek{P}}}
\newcommand{\GSi}{\mbox{\tt\textgreek{S}}}
\newcommand{\Giota}{\mbox{\tt\textgreek{i}}}
%%
Syntactically speaking {\mtt\symbol{5}} has category $t/t$ and
{\mtt\symbol{4}}, {\mtt\symbol{31}} and {\mtt\symbol{25}} have 
category $(t\backslash t)/t$. Finally, also the quantifiers must 
be turned into functions.  To this end we introduce the function 
symbols $\GPi$ and $\GSi$ of type $((e \pf t) \pf t)$. Moreover, 
$\GPi(X)$ is true iff for all $x$ $X(x)$ is true, and $\GSi(X)$ 
is true iff for some $x$ $X(x)$ is true. 
{\mtt (\symbol{20}$x$)$\varphi$} is now replaced by 
{\mtt $\GPi$(\tlambda$x$.$\varphi$)}, and 
{\mtt (\symbol{21}$x$)$\varphi$} by 
{\mtt $\GSi$(\tlambda$x$.$\varphi$)}.
So, ignoring types for the moment, we have the equations
%%%
\begin{align} 
\mbox{\mtt \symbol{20}} & = \mbox{\mtt {\tlambda}x$_{\snull}$.%
{\tlambda}x$_{\seins}$.\GPi({\tlambda}x$_{\snull}$.x$_{\seins}$)}  \\
\mbox{\mtt \symbol{21}} & = \mbox{\mtt {\tlambda}x$_{\snull}$.%
{\tlambda}x$_{\seins}$.\GSi({\tlambda}x$_{\snull}$.x$_{\seins}$)} 
\end{align}
%%%
We shall however continue to write $\forall x.\varphi$ and 
$\exists x.\varphi$. This definition can in fact be used to 
define quantification for all functions. This is the core idea 
behind the language of 
%%%
\index{simple type theory (STT)}%%
%%%
simple type theory (STT) according to Church~\shortcite{church:simple}. 
%%%
\index{Church, Alonzo}%%%
%%%%
Church assumes that the set of basic categories contains at 
least $t$. The symbol {\mtt\symbol{5}} has the type $t \pf t$, 
while the symbols {\mtt\symbol{4}}, {\mtt\symbol{31}} and
{\mtt\symbol{25}} have type $t \pf (t \pf t)$. (Church actually works 
only with negation and conjunction as basic symbols, but this 
is just a matter of convenience.) To get the power of 
predicate logic we assume for each type $\alpha$ a symbol 
$\GPi^{\alpha}$ of type $(\alpha \pf t) \pf t$ and a symbol 
$\Giota^{\alpha}$ of type $\alpha \pf (\alpha \pf t)$.
Put $\CS := \Typ_{\pf}(B)$.
%%%
\begin{defn}
%%%
\index{Henkin frame}
%%%
A \textbf{Henkin frame} is a structure 
%%
\begin{equation}
\GH = \auf \{D_{\alpha} : \alpha \in \CS\}, \bullet, -, \cap,
\{\pi^{\alpha} : \alpha \in \CS\}, 
\{\iota^{\alpha} : \alpha \in \CS\}\zu
\end{equation}
%%
such that the following holds.
%%
\begin{dingautolist}{192}
\item
$\auf \{D_{\alpha} : \alpha \in \CS\},
\bullet\zu$ is a functionally complete typed applicative structure.
\item
$D_t = \{0,1\}$,
$- \colon D_t \pf D_t$ and $\cap \colon D_t \pf (D_t \pf D_t)$ are
complement and intersection, respectively.
\item
For every $a \in D_{\alpha \pf t}$
$\pi^{\alpha} \bullet a = 1$ iff
for every $b \in D_{\alpha}$: $b \bullet a = 1$.
\item
For every $a \in D_{\alpha \pf t}$, if there is a $b \in
D_{\alpha}$ such that $a \bullet b = 1$ then also 
$a \bullet (\iota^{\alpha} \bullet a) = 1$.
\end{dingautolist}
\end{defn}
%%%
A valuation into a Henkin frame is a function $\beta$ such that
for every variable $x$ of type $\alpha$ $\beta(x) \in D_{\alpha}$.
For every $N$ of type $t$, $\auf \GH, \beta\zu \vDash N$ iff 
$[N]^{\beta} = 1$. Further, for a set $\Gamma$ of expressions of 
type $t$ and every $N$ of type $t$, $\Gamma \vDash N$ if
for every Henkin frame and every valuation $\beta$: if $\auf \GH,
\beta\zu \vDash M$ for all $M \in \Gamma$ then $\auf \GH, \beta\zu
\vDash N$.

$\pi^{\alpha}$ is the interpretation of $\GPi^{\alpha}$ and 
$\iota^{\alpha}$ the interpretation of $\Giota^{\alpha}$. So,
$\pi^{\alpha}$ is the device discussed above that allows to define 
the universal quantifier for functions of type $\alpha \pf t$.
$\iota^{\alpha}$ on the other hand is a kind of choice or `witness' 
function. If $a$ is a function from objects of type $\alpha$ into 
truth values then $\iota^{\alpha} \bullet a$ is an object of type
$\alpha$, and, moreover, if $a$ is at all true on some $b$ of type
$\alpha$, then it is true on $\iota^{\alpha} \bullet a$. In 
Section~\ref{kap6}.\ref{kap6-4a} we shall deliver 
an axiomatization of STT and show that the 
axiomatization is complete with respect to these models. The reason 
for explaining about STT is that every semantics or calculus that
will be introduced in the sequel can easily be interpreted into
STT.

We now turn to Montague Semantics. To begin we choose a very small 
\index{Montague Semantics}%%%
base of words. 
%%
\begin{equation}
\{\mbox{\tt Paul}, \mbox{\tt Peter}, \mbox{\tt sleeps},
\mbox{\tt sees}\}
\end{equation}
%%
The type of (the meaning of) {\tt Paul} and {\tt Peter} is $e$, the 
type of {\tt sleeps} is $e \pf t$, the type of {\tt sees} 
$e \pf (e \pf t)$. This means: names are interpreted by individuals, 
intransitive verbs by unary relations, and transitive verbs by binary 
relations. The (finite) verb {\tt sleeps} is interpreted by the 
relation $\textsf{sleeps}'$ and {\tt sees} by the relation 
$\textsf{see}'$. Because of our convention a transitive verb 
denotes a function (!) of type $e \pf (e \pf t)$. So the semantics 
of these verbs is
%%
\begin{align}
\mbox{\tt sleeps} & \mapsto \lambda x_e . \textsf{sleep}'(x_e) \\
\mbox{\tt sees} & \mapsto \lambda x_e. \lambda y_e. \mbox{\sf %
    see}'(y_e, x_e)
\end{align}
%%
We already note here that the variables are unnecessary. After
we have seen how the predicate logical formulae can be massaged
into typed $\lambda$--expressions, we might as well forget this
history and write $\textsf{sleep}'$ in place of the function
$\lambda x_e.\textsf{sleep}'(x_e)$ and $\textsf{see}'$ in
place of $\lambda x_e.\lambda y_e.\textsf{see}'(y_e,x_e)$.
This has the additional advantage that we need not mention the
variables at all (which is a moot point, as we have seen above).
We continue in this section to use the somewhat more longwinded
notation, however. We agree further that the value of {\tt Paul} 
shall be the constant $\textsf{paul}'$ and the value of 
{\tt Peter} the constant $\textsf{peter}'$. Here are finally 
our 0--ary modes.
%%
\begin{equation}
\begin{array}{lll}
\auf \mbox{\tt Paul}, & e, & \textsf{paul}'\zu \\
\auf \mbox{\tt Peter}, & e, & \textsf{peter}'\zu \\
\auf \mbox{\tt sleeps}, & e\backslash t, & \lambda x_e.
    \textsf{sleep}'(x_e)\zu \\
\auf \mbox{\tt sees}, & (e\backslash t)/e, &
    \lambda x_e. \lambda y_e. \mbox{\sf see}'(y_e, x_e)\zu
\end{array}
\end{equation}
%%
The sentences {\tt Peter sleeps} or {\tt Peter sees Peter}
are grammatical, and their meaning is $\textsf{sleep}'(\textsf{paul}')$ and
$\textsf{see}'(\textsf{peter}', \textsf{peter}')$.

The syntactic categories possess an equivalent in syntactic
terminology. $e$ for example is the category of proper
names. The category $e \backslash t$ is the category of
intransitive verbs and the category $(e \backslash t)/e$ is
the category of transitive verbs.

This minilanguage can be extended.  For example, we can introduce 
the word {\tt not} by means of the following constant mode.
%%
\begin{equation}
\auf \mbox{\tt not}, (e\backslash t)\backslash (e \backslash t),
    \lambda e_{e \pf t}. \lambda x_e. \nicht
    x_{e \pf t}(x_e)\zu
\end{equation}
%%
The reader is asked to verify that now {\tt sleeps not} is
an intransitive verb, whose meaning is the complement of the
meaning  of {\tt sleeps}. So, {\tt Paul sleeps not}
is true iff {\tt Paul sleeps} is false.
This is perhaps not such a good example, since the negation
in English is formed using the auxiliary {\tt do}.  To give 
a better example, we may introduce {\tt and} by the following mode.
%%
\begin{multline}
\auf \mbox{\tt and},
    ((e \backslash t)\backslash (e \backslash t))/(e \backslash t),
\\
    \lambda x_{e \pf t}.\lambda y_{e \pf t}.\lambda z_{e}.
    x_{e \pf t}(z_e) \und y_{e \pf t}(z_e)\zu
\end{multline}
%%
In this way we have a small language which can generate infinitely
many grammatical sentences and which assigns them correct meanings.
Of course, English is by far more complex than this.

The real advance that Montague made was to show that one can
%%%
\index{Montague, Richard}%%%
%%%
treat quantification. Let us take a look at how this can be
done. (Actually, what we are going to outline right now is
not Montague's own solution, since it is not in line with 
Categorial Grammar. 
%%%
\index{Categorial Grammar}%%%
%%%
We will deal with Montague's approach to 
quantification in Chapter~\ref{kap6}.) Nouns like {\tt cat} and 
{\tt mouse} are not proper names but semantically speaking unary predicates.
For {\tt cat} does not denote a single individual but a class of
individuals. Hence, following our conventions, the semantic type
of {\tt cat} and {\tt mouse} is $e \pf t$. Syntactically speaking
this corresponds to either $t/e$ or $e \backslash t$. Here,
no decision is possible, for neither {\tt Cat Paul} nor 
{\tt Paul cat} is a grammatical sentence. Montague did not solve
this problem; he introduced a new category constructor $/\!/$,
which allows to distinguish a category $t /\!/ e$ from $t /e$
(the intransitive verb) even though they are not distinct in type.
Our approach is simpler. We introduce a category $n$ and stipulate
that $\sigma(n) := e \pf t$. This is an example where the basic
categories are different from the (basic) semantic types.
Now we say that the subject quantifier {\tt every} has the
sentactic category $(t/(e \backslash t))/n$. This means the
following. It forms a constituent together with a noun,
and that constituent has the category $t/(e \backslash t)$. 
This therefore is a constituent that needs an intransitive verb 
to form a sentence. So we have the following constant mode.
%%
\begin{equation}
\auf \mbox{\tt every}, (t/(e \backslash t))/n, 
    \lambda x_{e \pf t}. \lambda y_{e \pf t}.
    \forall x_e. (x_{e \pf t}(x_e) \pf y_{e \pf t}(x_e))\zu
\end{equation}
%%
Let us give an example.
%%
\begin{equation}
\mbox{\tt every cat sees Peter}
\end{equation}
%%
The syntactic analysis is as follows.
%%
\begin{equation}
\begin{array}{ccccc}
\mbox{\tt every} & \quad \mbox{\tt cat} & \quad & \mbox{\tt sees} &
    \quad \mbox{\tt Peter} \\
(t/(e \backslash t))/n & n & & (e \backslash t)/e & e
\\\cline{1-2}\cline{4-5}
\multicolumn{2}{c}{t/(e \backslash t)} & &
    \multicolumn{2}{c}{e \backslash t} \\\hline
    \multicolumn{5}{c}{t}
\end{array}
\end{equation}
%%
This induces the following constituent structure.
%%
\begin{equation}
((\mbox{\tt every}\; \mbox{\tt cat})\;
(\mbox{\tt sees}\; \mbox{\tt Peter}))
\end{equation}
%%
Now we shall have to insert the meanings in place of the words
and calculate. This means converting into normal form. For
by convention, a constituent has the meaning that arises from
applying the meaning of one immediate part to the meaning of the other.
That this is now well--defined is checked by the syntactic analysis.
We calculate in several steps.  {\tt sees Peter} is a constituent
and its meaning is
%%
\begin{equation}
(\lambda x_e.\lambda y_e.\textsf{see}'(y_e, x_e))(%
\textsf{peter}') = \lambda y_e. \textsf{see}'(y_e, \textsf{peter}') 
\end{equation}
%%
Further,  {\tt every cat} is a constituent with the following
meaning
%%
\begin{align}
\begin{split}
%\begin{array}{ll}
  & (\lambda x_{e \pf t}. \lambda y_{e \pf t}.
    (\forall x_e. x_{e \pf t}(x_e) \pf y_{e \pf t}(x_e)))
    (\lambda x_e.\textsf{cat}'(x_e)) \\
= &
    \lambda y_{e \pf t}.\forall x_e.
    ((\lambda x_e. \textsf{cat}'(x_e))(x_e) \pf
    y_{e \pf t}(x_e)) \\
= &
    \lambda y_{e \pf t}.\forall x_e.
    (\textsf{cat}'(x_e) \pf y_{e \pf t}(x_e))
    \end{split}
\end{align}
%%
Now we combine these two:
%%
\begin{align}
\begin{split}
 & (\lambda y_{e \pf t}.\forall x_e.
    \textsf{cat}'(x_e) \pf y_{e \pf t}(x_e))(%
    \lambda y_e. \textsf{see}'(y_e, \textsf{peter}')) \\
    = &
\forall x_e. (\textsf{cat}'(x_e) \pf
    (\lambda y_e. \textsf{see}'(y_e, \textsf{peter}'))(x_e)) \\
    = &
\forall x_e. (\textsf{cat}'(x_e) \pf
    \textsf{see}'(x_e, \textsf{peter}'))
\end{split}
\end{align}
%%
This is the desired result. Similarly to {\tt every} we define
{\tt some}:
%%
\begin{equation}
\auf \mbox{\tt some}, (t/(e \backslash t))/n, 
    \lambda x_{e \pf t}. \lambda y_{e \pf t}.
    \exists x_e. (x_{e \pf t}(x_e) \und y_{e \pf t}(x_e))\zu
\end{equation}
%%
If we also want to have quantifiers for direct objects we have
to introduce new modes.
%%
\begin{align}
\begin{split}
& \auf \mbox{\tt every}, ((e \backslash t)/((e \backslash t)/e))/n, \\
& \qquad \lambda x_{e \pf t}. \lambda y_{e \pf (e \pf t)}.
    \lambda y_e. \forall x_e. (x_{e \pf t}(x_e) \pf
    y_{e \pf (e \pf t)}(x_e)(y_e))\zu \\
\end{split} \\
\begin{split}
& \auf \mbox{\tt some}, ((e \backslash t)/((e \backslash t)/e))/n, \\
& \qquad \lambda x_{e \pf t}. \lambda y_{e \pf (e \pf t)}.
\lambda y_e. \exists x_e. (x_{e \pf t}(x_e) \pf
    y_{e \pf (e \pf t)}(x_e)(y_e))\zu
\end{split}
\end{align}
%%
For {\tt every cat} as a direct object is analyzed as a constituent
which turns a transitive verb into an intransitive verb. Hence it
must have the category $(e \backslash t)/((e \backslash t)/e)$.
From this follows immediately the category assignment for {\tt
every}.

Let us look at this using an example.
%%
\begin{equation}
\mbox{\tt some cat sees every mouse}
\end{equation}
%%
The constituent structure is as follows.
%%
\begin{equation}
((\mbox{\tt some}\; \mbox{\tt cat})\;
(\mbox{\tt sees}\; (\mbox{\tt every}\; \mbox{\tt mouse}))))
\end{equation}
%%
The meaning of {\tt every mouse} is, as is easily checked, the
following:
%%
\begin{equation}
\lambda y_{e \pf (e \pf t)}. \lambda y_e.
\forall x_e(\textsf{mouse}'(x_e) \pf y_{e \pf %
(e \pf t)}(x_e)(y_e))
\end{equation}
%%
From this we get for {\tt sees every mouse}
%%
\begin{align}
\label{eq:sem}
\begin{split}
 & \lambda y_e. \forall x_e(\textsf{mouse}'(x_e) \pf
(\lambda x_e. \lambda y_e.\textsf{see}'(y_e, x_e))%
(x_e)(y_e)) \\
= &
\lambda y_e. \forall x_e(\textsf{mouse}'(x_e) \pf
\textsf{see}'(y_e, x_e))
\end{split}
\end{align}
%%
{\tt some cat} is analogous to {\tt every cat}:
%%
\begin{equation}
\label{eq:sc}
\lambda y_{e \pf t}. \exists x_e. (\textsf{cat}'(x_e)
    \und y_{e \pf t}(x_e))
\end{equation}
%%
We combine \eqref{eq:sc} and \eqref{eq:sem}. 
%%
\begin{align}
\begin{split}
  & (\lambda y_{e \pf t}. \exists x_e. (\textsf{cat}'(x_e)
    \und y_{e \pf t}(x_e))) \\
  & \qquad (\lambda y_e. \forall x_e.(\textsf{mouse}'(x_e) \pf
	\textsf{see}'(y_e, x_e))) \\
= &
\exists x_e. (\textsf{cat}'(x_e) \und
(\lambda y_e. \forall x_e.(\textsf{mouse}'(x_e) 
\pf \textsf{see}'(y_e, x_e)))(x_e)) \\
= &
\exists x_e.(\textsf{cat}'(x_e) \und
\forall z_e.(\textsf{mouse}'(z_e) \pf
\textsf{see}'(x_e, z_e))) 
\end{split}
\end{align}
%%
One can see that the calculations require
some caution. Sometimes variables may clash and this calls
for the substitution of a variable. This is the case for
example when we insert a term and by doing so create a
bound occurrences of a variable. The $\lambda$--calculus
is employed to do this work for us. (On the other hand, 
if we used plain functions here, this would again be needless.)

Montague used the cancellation interpretation for his calculus,
hence the sequent formulation uses the calculus $\textsf{E}$. We 
have seen that this calculus can also be rendered into a sign grammar, 
which has two modes, forward application ($\mbox{\tt A}_{\sgr}$) and
backward application ($\mbox{\tt A}_{\skl}$). In syntactic theory,
however, the most popular version of grammar is the
Lambek--Calculus. However, the latter does not lend itself easily
to a compositional interpretation. The fault lies basically in the
method of hypothetical assumptions. Let us see why this is so. An
adjective like {\tt big} has category $n/n$, and its
type is $(e \pf t) \pf (e \pf t)$. (This is not quite
true, but good enough for illustration.) This means that it can
modify nouns such as {\tt car}, but not relational nouns
such as {\tt friend} or {\tt neighbour}. Let us assume that the
latter have category $n/g$ (where $g$ stands for the category 
of a genitive argument). Now, in Natural Deduction style
Lambek--Calculus we can derive a constituent {\tt big neighbour}
by first feeding it a hypothetical argument and then abstracting
over it.
%%
\begin{equation}
\begin{array}{ccccc}
\mbox{\tt big} & \quad & \mbox{\tt neighbour} & \quad & \\
n/n : \textsf{big}' & &
    n/g : \textsf{neighbour}' & & g : y \\\cline{3-5}
\vdots & & \multicolumn{3}{c}{n : \textsf{neighbour}'(y)} \\\cline{1-5}
\multicolumn{4}{c}{n : \textsf{big}'(\textsf{neighbour}'(y))}
    \\\cline{1-5}
\multicolumn{4}{c}{n/g : \lambda y.\textsf{big}'(\textsf{neighbour}'(y))}
\end{array}
\end{equation}
%%
This allows us, for example, to coordinate {\tt big neighbour} and
{\tt friend} and then compose with {\tt of mine}. Notice that this 
proof is not available in \textsf{E}. There also is a sign based 
analogue of this. Introduce binary modes $\mbox{\tt L}_{\sgr}$ and 
$\mbox{\tt L}_{\skl}$:
%%
\begin{equation}
\begin{split}
\mbox{\tt L}_{\sgr}(\auf \vec{x}, \alpha, M\zu,
    \auf \vec{y}, \gamma, x_{\gamma}\zu)
    & := \auf \vec{x}/\vec{y}, \alpha/\gamma,
    (\lambda x_{\gamma}.Mx_{\gamma})\zu \\
\mbox{\tt L}_{\skl}(\auf \vec{x}, \alpha, M\zu,
    \auf \vec{y}, \gamma, x_{\gamma}\zu)
    & := \auf \vec{y}\backslash\vec{x}, \gamma\backslash\alpha,
    (\lambda x_{\gamma}.Mx_{\gamma})\zu
\end{split}
\end{equation}
%%
A condition on the application of these modes is that the
variable $x_{\gamma}$ actually occurs free in the term.
Now introduce a new 0--ary mode with exponent $\copyright$,
which shall be a symbol not in the alphabet.
%%
\begin{equation}
\mbox{\mtt V}_{\alpha\mbox{\smtt :}i} := \auf \copyright, \alpha,
    x_{\alpha,i}\zu
\end{equation}
%%
Consider the structure term
%%
\begin{equation}
\mbox{\mtt L$_{\sgr}$A$_{\sgr}$BgA$_{\sgr}$NbV$_{\mbox{\smtt g:0}}$%
V$_{\mbox{\smtt g:0}}$}
\end{equation}
%%
Here, $\mbox{\tt Bg} := \auf \mbox{\tt big}, n/n, \textsf{big}'\zu$ 
and $\mbox{\tt Nb} := \mbox{\tt neighbour}, n/g, \textsf{neighbour}'\zu$.
On condition that it is definite, it has the following unfolding.
%%
\begin{equation}
\auf \mbox{\tt big neighbour}, n/g,
    \lambda x_{g,0}.\textsf{big}'\textsf{neighbour}'(x_{g,0})\zu
\end{equation}
%%
These modes play the role of hypothetical arguments in
Natural Deduction style derivations. However, the combined
effect of these modes is not exactly the same as in the
Lambek--Calculus. The reason is that abstraction can only be
over a variable that is introduced right or left peripherally
to the constituent. However, if we introduce two arguments in
succession, we can abstract over them in any order we please,
as the reader may check (see the exercises). The reason is
that $\copyright$ bears no indication of the name of the variable
that it introduces.  This can be remedied by introducing instead
the following 0--ary modes.
%%
\begin{equation}
\mbox{\tt T}_{\alpha\mbox{\smtt :}i} := \auf \copyright_{\alpha,i}, \alpha,
    x_{\alpha,i}\zu
\end{equation}
%%
Notice that these empty elements can be seen as the categorial
analogon of traces in Transformational Grammar (see
Section~\ref{kap5}.\ref{kap5-5}). Now the exponent reveals the exact
identity of the variable and the Lambek--Calculus is exactly
mirrorred by these modes. The price we pay is that there are
structure terms whose exponents are not pronounceable: they
contain elements that are strictly speaking not overtly visible.
The strings are therefore not surface strings. 

{\it Notes on this section.} Already in 
\cite{harris:structural} the idea is defended that one must 
%%%
\index{Harris, Zellig S.}%%%
%%%
sometimes pass through `nonexistent' strings, and TG has made 
much use of this. An alternative idea that suggests itself
is to use combinators. This route has been taken by
Steedman in \shortcite{steedman:gapping,steedman:surface}.
%%%
\index{Steedman, Mark}%%%
%%%
For example, the addition of the modes $\mbox{\tt B}_{\sgr}$ and
$\mbox{\tt B}_{\skl}$ assures us that we can derive the these
constituents as well. Steedman and Jacobson emphasize in their
work also that variables can be dispensed with in favour of
combinators. See \cite{jacobson:toward,jacobson:disorganisation} 
(and references therein) for a defense of variable free semantics. 
For a survey of approaches see \cite{boettnerthuemmel:free}.
%%
\vplatz
\exercise
Write an AB--grammar for predicate logic over
a given signature and a given structure. {\it Hint.} You need two
types of basic categories: $e$ and $t$, which now stand for
{\it terms\/} and {\it truth--values}. 
%%
\vplatz
\exercise
The solutions we have presented here fall short of taking certain
aspects of orthography into account. In particular, words are not 
separated by a blank, sentences do not end in a period and the first 
word of a sentence is written using lower case letters only. Can you 
think of a remedy for this situation?
%%
\vplatz
\exercise
Show that with the help of $\mbox{\tt L}_{\skl}$ and 
$\mbox{\tt L}_{\sgr}$ and the 0--ary modes 
$\mbox{\tt V}_{\alpha\mbox{\smtt :}i}$ it is possible
to derive the sign
%%
\begin{equation}
\auf \mbox{\tt give}, (e\backslash t)/e/e,
    \lambda x.\lambda y.\lambda z.\textsf{give}'(z)(x)(y)\zu
\end{equation}
%%
from the sign
%%
\begin{equation}
\auf \mbox{\tt give}, (e\backslash t)/e/e,
    \lambda x.\lambda y.\lambda z.\textsf{give}'(z)(y)(x)\zu
\end{equation}
%%
\vplatz
\exercise
\label{ex:boolesch}
We have noted earlier that {\tt and}, {\tt or} and {\tt not} are
polymorphic. The polymorphicity can be accommodated directly by
defining polyadic operations in the $\lambda$--calculus. Here is
how. Call a type $\alpha$ $t$--\textbf{final} if it has the following
form: (a) $\alpha = t$, or (b) $\alpha = \beta \pf \gamma$, where
$\gamma$ is $t$--final. Define $\mbox{\mtt\symbol{4}}_{\alpha}$, 
$\mbox{\mtt\symbol{31}}_{\alpha}$ and $\mbox{\mtt\symbol{5}}_{\alpha}$ 
by induction. Similarly, for every type $\alpha$ 
define functions $\GSi_{\alpha}$ and $\GPi_{\alpha}$ of type 
$\alpha \pf t$ that interpret the existential and universal
quantifier. 
%%
\vplatz
\exercise
%%%
\index{generalized quantifier}%%
%%%%
A (\textbf{unary}) \textbf{generalized quantifier} is a function
from properties to truth values (so, it is an object of type
$(e \pf t) \pf t$). Examples are {\tt some} and {\tt every},
but there are many more:
%%
\begin{align}
& \mbox{\tt more than three} \\
& \mbox{\tt an even number of} \\
& \mbox{\tt the director's}
\end{align}
%%
First, give the semantics of each of the generalized
quantifiers and define a sign for them. Now try to
define the semantics of {\tt more than}. (It takes
a number and forms a generalized quantifier.)
%%
\vplatz
\exercise
In $\CCG(\textsf{B})$, many (but not all) substrings are
constituents. We should therefore be able to coordinate them
with {\tt and}. As was noted for example by Eisenberg in
\shortcite{eisenberg:identity}, such a coordination is
constrained (the brackets enclose the critical constituents).
%%
\begin{align}
& ^{\ast}[\mbox{\tt John said that I}]\; \mbox{\tt and}\;
    [\mbox{\tt Mary said that she}] \\\notag
& \quad \mbox{\tt is the best swimmer.} \\
& [\mbox{\tt John said that I}] \;\mbox{\tt and}\;
    [\mbox{\tt Mary said that she}] \\\notag
    & \quad\mbox{\tt was the best swimmer.} 
\end{align}
%%
The constraint is as follows. $\vec{x}\oconc\mbox{\tt and}\oconc\vec{y}
\oconc\vec{z}$ is well--formed only if both $\vec{x}\oconc\vec{z}$ and
$\vec{y}\oconc\vec{z}$ are. The suggestion is therefore that first
the sentence $\vec{x}\oconc\vec{z}\oconc\mbox{\tt and}\oconc\vec{y}%
\oconc\vec{z}$ is formed and then the first occurrence of $\vec{z}\oconc$ 
is `deleted'. Can you suggest a different solution? {\it Note.} The 
construction
%%%
\index{forward deletion}%%
\index{backward deletion}%%
%%%
is known as {\bf forward deletion}. The more common {\bf backward
deletion} gives $\vec{x}\oconc\vec{z}\oconc\mbox{\tt and}\oconc\vec{y}$, 
and is far less constrained.
%%

% \newpage 
%	\thispagestyle{empty}
%	\mbox{}
 \chapter{Semantics}
\thispagestyle{empty}
\label{kap6}
%%
\section{The Nature of Semantical Representations}
\label{kap6-1}
\label{kap:feasibility}
%
%
This chapter lays the foundation of semantics. In contrast
to much of the current semantical theory we shall not use
a model--theoretic approach but rather an algebraic one.
As it turns out, the algebraic approach helps to circumvent
many of the difficulties that beset a model--theoretic analysis,
since it does not try to spell out the meanings in every detail,
only in as much detail as is needed for the purpose at hand.

In this section we shall be concerned with the question of
feasibility of interpretation. Much of semantical theory
simply defines mappings from strings to meanings without
assessing the question whether such mappings can actually be
computed. While on a theoretical level this gives satisfying
answers, one still has to address the question how it is
possible that a human being can actually understand a sentence.
The question is quite the same for computers. Mathematicians
`solve' the equation $x^{2} = 2$ by writing $x = \pm \sqrt{2}$.
However, this is just a piece of notation. If we want to know
whether or not $3^{\sqrt{2}} < 6$, this requires calculation.
This is the rule rather than the exception (think of trigonometric 
functions or the solutions of differential equations).
However, hope is not lost. There are algorithms by which the
number $\sqrt{2}$ can be approximated to any degree of precision
needed, using only elementary operations. Much of mathematical
theory has been inspired by the need to calculate difficult functions
(for example logarithms)  by means of elementary ones.
Evidently, even though we do not have to bother any more
with them thanks to computers, the computer still has to do
the job for us. Computer hardware actually implements sophisticated
algorithms for computing nonelementary functions. Furthermore,
computers do not compute with arbitrary degree of precision. Numbers
are stored in fixed size units (this is not necessary, but the
size is limited anyhow by the size of the memory of the
computer). Thus, they are only {\it close\/} to the actual input,
not necessarily equal. Calculations on the numbers propagate
these errors and in bad cases it can happen that small
errors in the input yield astronomic errors in the output
(problems that have this property independently of any algorithm
%%%%
\index{problem!ill--conditioned}%%
%%%%
that computes the solution are called \textbf{ill--conditioned}).
Now, what reason do we have to say that a particular machine
with a particular algorithm computes, say, $\sqrt{2}$? One
answer could be: that the program will yield {\it exactly\/}
$\sqrt{2}$ given exact input and enough time. Yet, for approximative
methods --- the ones we generally have to use --- the computation 
is never complete. However, then it computes a series of numbers $a_n$,
$n \in \omega$, which converges to $\sqrt{2}$. That is to say,
if $\varepsilon > 0$ is any real number (the error) we have to name
an $n_{\varepsilon}$ such that for all $n \geq n_{\varepsilon}$:
$|a_n - \sqrt{2}| \leq \varepsilon$, given exact computation.
That an algorithm computes such a series is typically shown using
pure calculus over the real numbers. This computation is actually
independent of the way in which the computation proceeds as long as it
can be shown to compute the approximating series. For example,
to compute $\sqrt{2}$ using Newton's method, all you have to
do is to write a program that calculates
%%
\begin{equation}
a_{n+1} := a_n - (a_n^2 -2)/2a_n
\end{equation}
%%
For the actual computation on a machine it matters very much
how this series is calculated. This is so because each operation
induces an error, and the more we compute the more we depart
from the correct value. Knowing the error propagation of the
basic operations it is possible to compute exactly, given
any algorithm, with what precision it computes. To sum up, in
addition to calculus, computation on real machines needs two
things:
%%
\begin{dinglist}{43}
\item
a theory of approximation, and
\item
a theory of error propagation.
\end{dinglist}
%%
Likewise, semantics is in need of two things: a theory of
approximation, showing us what is possible to compute and
what not, and how we can compute meanings, and second a
theory of error propagation, showing us how we can determine
the meanings in approximation given only limited resources
for computation. We shall concern ourselves with the first
of these. Moreover, we shall look only at a very limited
aspect, namely: what meanings can in principle be computed
and which ones cannot.

We have earlier characterized the computable functions as those
that can be computed by a Turing machine. To see that this is
by no means an innocent assumption, we shall look at
propositional logic. Standardly, the semantics of classical
propositional logic is given as follows. (This differs only
slightly from the setup of Section~\ref{kap3}.\ref{kap:prop}.)
The alphabet is $\{\mbox{\tt (}, \mbox{\tt )},
\mbox{\tt p}, \mbox{\tt 0}, \mbox{\tt 1}, \mbox{\mtt\symbol{5}}, 
\mbox{\mtt\symbol{4}}\}$ and the set of variables 
$V := \mbox{\tt p}(\mbox{\tt 0}\cup\mbox{\tt 1})^{\ast}$.
A function $\beta : V \pf 2$ is called a \textbf{valuation}.
%%%
\index{valuation}%%
%%%
We extend $\beta$ to a mapping $\oli{\beta}$ from
the entire language to $2$.
%%
\begin{align}
\notag
\oli{\beta}(p) & := \beta(p) & (p \in V) \\
\oli{\beta}(\mbox{\mtt (\symbol{5}$\varphi$)}) &
    := - \oli{\beta}(\varphi) & \\
\notag
\oli{\beta}(\mbox{\mtt ($\varphi$\symbol{4}$\chi$)}) &
    := \oli{\beta}(\varphi) \cap \oli{\beta}(\chi)
\end{align}
%%
To obtain from this a compositional interpretation for the
language we turn matters around and define the meaning of
a proposition to be a function from valuations to $2$. Let
$2^V$ be the set of functions from $V$ to $2$.
Then for every proposition $\varphi$, $[\varphi]$ denotes
the function from $2^V$ to $2$ that satisfies
%%
\begin{equation}
[\varphi](\beta) = \oli{\beta}(\varphi)
\end{equation}
%%
(The reader is made aware of the fact that what we have
performed here is akin to type raising, turning the argument
into a function over the function that applies to it.)
Also $[\varphi]$ can be defined inductively.
%%
\begin{align}
\notag
[p] & := \{\beta : \beta(p) = 1\} & (p \in V) \\
[\mbox{\mtt (\symbol{5}$\varphi$)}] & :=
    2^V - [\varphi] & \\
\notag
[\mbox{\mtt ($\varphi$\symbol{4}$\chi$)}] & :=
    [\varphi] \cap [\chi] &
\end{align}
%%
Now notice that $V$ is infinite. However, we have excluded that
the set of basic modes is infinite, and so we need to readjust
the syntax. Rather than working with only one type of expression,
we introduce a new type, that of a
%%%
\index{register}%%
%%%
\textbf{register}. Registers are elements of
$G := (\mbox{\mtt 0} \cup \mbox{\mtt 1})^{\ast}$. Then $V =
\mbox{\mtt p} \cdot G$. Valuations are now functions from
$G$ to $2$. The rest is as above. Here is now a sign grammar
for propositional logic. The modes are {\tt E} (0--ary),
$\mbox{\mtt P}_{\snull}$, $\mbox{\mtt P}_{\seins}$, $\mbox{\tt V}$,
$\mbox{\mtt J}_{\mbox{\smtt\symbol{5}}}$ (all unary), and 
$\mbox{\mtt J}_{\mbox{\smtt\symbol{4}}}$ (binary). The exponents 
are strings over the alphabets, categories are either $R$ or $P$, 
and meanings are either registers (for expressions of category $R$) 
or sets of functions from registers to $2$ (for expressions of 
category $P$).
%%
\begin{subequations}
\begin{align}
\mbox{\mtt E} & := \auf \varepsilon, R, \varepsilon\zu \\
\mbox{\mtt P}_{\snull}(\auf\vec{x}, R, \vec{y}\zu) & :=
    \auf \vec{x}\conc\mbox{\mtt 0}, R, \vec{y}\conc\mbox{\mtt 0}\zu \\
\mbox{\mtt P}_{\seins}(\auf\vec{x}, R, \vec{y}\zu) & :=
    \auf \vec{x}\conc\mbox{\mtt 1}, R, \vec{y}\conc\mbox{\mtt 1}\zu \\
\mbox{\mtt V}(\auf \vec{x}, R, \vec{x}\zu) & :=
    \auf \mbox{\mtt p$\vec{x}$}, P, [\mbox{\mtt p$\vec{x}$}]\zu \\
\mbox{\mtt J}_{\mbox{\smtt\symbol{5}}}(\auf \vec{x}, P, M\zu) & :=
    \auf \mbox{\mtt (\symbol{5}$\vec{x}$)}, P,
    2^V - M\zu \\
\mbox{\mtt J}_{\mbox{\smtt\symbol{4}}}(\auf \vec{x}, P, M\zu, 
	\auf\vec{y}, P, N\zu) &
    := \auf\mbox{\mtt ($\vec{x}$\symbol{4}$\vec{y}$)}, P,
    M \cap N\zu
\end{align}
\end{subequations}
%%
It is easily checked that this is well--defined. This defines
a sign grammar that meets all requirements for being compositional
except for one: the functions on meanings are not computable.
Notice that (a) valuations are infinite objects, and (b) there
are uncountably many of them.  However, this is not sufficient
as an argument because we have not actually said how we encode sets
of valuations as strings and how we compute with them. Notice
also that the notion of computability is defined only on strings.
Therefore, meanings too must be coded as strings. We may improve
the situation a little bit by assuming that valuations are functions
from finite subsets of $G$ to $2$. Then at least valuations can be
represented as strings (for example, by listing pairs consisting of
a register and its value). However, still the set of all valuations
that make a given proposition true is infinite. On the other hand,
there is an algorithm that can check for any given partial
function whether it assigns $1$ to a given register
(it simply scans the string for the pair whose first member
is the given register). Notice that if the function is not defined
on the register, we must still give an output. Let it be {\tt \#}.
We may then simply take the code of the Turing machine computing
that function as the meaning the variable (see
Section~\ref{kap1}.\ref{einsfuenf} for a definition). Then, inductively, we
can define for every proposition $\varphi$ a machine $T_{\varphi}$
that computes the value of $\varphi$ under any given partial
valuation that gives a value for the occurring variables, and
assigns {\tt \#} otherwise. Then we assign as the meaning of
$\varphi$ the code $T_{\varphi}^{\spadesuit}$ of that Turing
machine. However, this approach suffers from a number of deficiencies.

First, the idea of using partial valuations does not always help.
To see this let us now turn to predicate logic
(see Section~\ref{kap3}.\ref{kap3-6}). As in the case of propositional
logic we shall have to introduce binary strings for registers,
to form variables. The meaning of a formula $\varphi$ is by
definition a function from pairs $\auf \GM, \beta\zu$ to $\{0,1\}$,
where $\GM$ is a structure and $\beta$ a function from variables
to the domain of $\GM$. Again we have the problem to name finitary
or at least computable procedures. We shall give two ways of doing so
that yield quite different results. The first attempt is to
exclude infinite models. Then $\GM$, and in particular the domain
$M$ of $\GM$, are finite.  A valuation is a partial function from
$V$ to $M$ with a finite domain. The meaning of a term under such
a valuation is a member of $M$ or $= \star$. (For if $x_{\alpha}$
is in $t$, and if $\beta$ is not defined on $x_{\alpha}$ then
$t^{\beta}$ is undefined.) The meaning of a formula is either a
truth value or $\star$. The truth values can be inductively defined
as in Section~\ref{kap3}.\ref{kap3-6}. $M$ has to be finite, since we usually
cannot compute the value of $\forall x_{\alpha}.\varphi(x_{\alpha})$
without knowing all values of $x_{\alpha}$.

This definition has a severe drawback: it does not give the
correct results. For the logic of finite structures is stronger
than the logic of all structures. For example, the following set
of formulae is not satisfiable in finite structures while it
has an infinite model. (Here {\tt 0} is a 0--ary function symbol,
and {\tt s} a unary function symbol.)
%%
\begin{prop}
The theory $T$ is consistent but has no finite model.
%%
\begin{equation}
T := \{\mbox{\mtt (\symbol{20}x$_{\snull}$)(\symbol{5}sx$_{\snull}$=0)},
\mbox{\mtt (\symbol{20}x$_{\snull}$)(\symbol{20}x$_{\seins}$)(sx$_{\snull}$%
=sx$_{\seins}$\symbol{25}x$_{\snull}$=x$_{\seins}$)}\}
\end{equation}
\end{prop}
%%
\proofbeg
Let $\GM$ be a finite model for $T$. Then for some $n$ and some $k > 0$:
$s^{n+k}0 = s^n0$. From this it follows with the second formula that
$s^k 0 = 0$. Since $k > 0$, the first formula is false in $\GM$. There is,
however, an infinite model for these formulae, namely the set of
numbers together with 0 and the successor function.
\proofend

We remark here that the logic of finite structures is not
recursively enumerable if we have two unary relation symbols.
(This is a theorem from \cite{trakhtenbrodt:finite}.) 
%%%
\index{Trakht\'enbrodt, B.~A.}%%%
%%%
However,
the logic of all structures is clearly recursively enumerable,
showing that the sets are {\it very\/} different. This throws us 
into a dilemma: we can obviously not compute the meanings of
formulae in a structure directly, since quantification
requires search throughout the entire structure. (This
problem has once worried some logicians, see \cite{ferreiros:road}.
Nowadays it is felt that these are not problems of logic proper.)
So, once again we have to actually try out another {\it semantics}.

The first route is to let a formula denote the set of all formulae
that are equivalent to it. Alternatively, we may take the set of
all formulae that follow from it. (These are almost the same in
boolean logic. For example, $\varphi\dpf\chi$ can be defined
using $\pf$ and $\und$; and $\varphi \pf \chi$ can be defined by
$\varphi \dpf (\varphi\und\chi)$. So these approaches are not very 
different. However the second one is technically speaking more 
elegant.) This set is again infinite. Hence, we do something different. 
We shall take a formula to denote any formula that follows from it. 
(Notice that this makes formulae have infinitely many meanings.)
Before we start we seize the opportunity to introduce a more abstract 
%%%
\index{language!propositional}%%
%%%%
theory. A \textbf{propositional language} is a language of formulas 
generated by a set $V$ of variables and a signature. The identity of 
$V$ is the same as for boolean logic above. As usual, propositions are
considered here as certain strings. The language is denoted by the
letter $L$. A 
%%%%
\index{substitution}%%%
%%%%
\textbf{substitution} is given by a map $\sigma \colon V \pf
L$. $\sigma$ defines a map from $L$ to $L$ by replacement of
occurrences of variables by their $\sigma$--image. We denote by
$\varphi^{\sigma}$ the result of applying $\sigma$ to $\varphi$.
%%
\begin{defn}
A \textbf{consequence relation over} 
%%%%
\index{consequence relation}%%%
%%%%
$L$ is a relation $\vdash\;
\subseteq\; \wp(L) \times L$ such that the following holds.
(We write $\Delta \vdash \varphi$ for the more complicated
$\auf \Delta, \varphi\zu \in\; \vdash$.)
%%%
\begin{dingautolist}{192}
\item $\varphi \vdash \varphi$.
\item If $\Delta \vdash \varphi$
and $\Delta \subseteq \Delta'$ then $\Delta' \vdash \varphi$.
\item If $\Delta \vdash \chi$ and $\Sigma;\chi \vdash
    \varphi$, then $\Delta; \Sigma \vdash \varphi$.
\end{dingautolist}
%%%
$\vdash$ is called \textbf{structural} 
%%%
\index{consequence relation!structural}%%
%%%
if from $\Delta \vdash
\varphi$ follows $\Delta^{\sigma} \vdash \varphi^{\sigma}$ for
every substitution. $\vdash$ is \textbf{finitary} 
%%%
\index{consequence relation!finitary}%%%
%%%
if $\Delta \vdash \varphi$ implies that there is a finite subset 
$\Delta'$ of $\Delta$ such that $\Delta' \vdash \varphi$.
\end{defn}
%%
In the sequel consequence relations are always assumed to be
structural. A rule is an element of $\wp(L) \times L$,
that is, a pair $\rho = \auf \Delta, \varphi\zu$. $\rho$ is 
%%%
\index{rule!finitary}%%
%%%
\textbf{finitary} if $\Delta$ is finite; it is $n$--\textbf{ary} 
if $|\Delta| = n$. Given a set $R$ of rules, we call $\vdash^R$ the 
least structural consequence relation containing $R$. This relation
can be explicitly defined. Say that $\chi$ is a \textbf{1--step}
$R$--\textbf{consequence} of $\Sigma$ if there is a substitution
$\sigma$ and some rule $\auf \Delta, \varphi\zu \in R$ such that
$\Delta^{\sigma} \subseteq \Sigma$ and $\chi = \varphi^{\sigma}$.
%%%
\index{consequence!$n$--step \faul}%%%
%%%%
Then, an $n$--\textbf{step consequence} of $\Sigma$ is inductively
defined.
%%%
\begin{prop}
$\Delta \vdash^R \varphi$ iff there is a natural
number $n$ such that $\varphi$ is an $n$--step $R$--consequence
of $\Delta$.
\end{prop}
%%
The reader may also try to generalize the notion of a proof from
a Hilbert calculus and show that they define the same relation
on condition that the rules are all finitary. We shall also give
an abstract semantics and show its completeness. The notion of an
$\Omega$--algebra has been defined.
%%%
\begin{defn}
%%%
\index{matrix}%%
\index{truth value}%%
\index{truth value!designated}%%
%%%
Let $L$ be a propositional logic over the signature $\Omega$. A
\textbf{matrix} for $L$ and $\vdash$ is a pair $\GM = \auf \GA, D\zu$,
where $\GA$ is an $\Omega$--algebra (the algebra of \textbf{truth values})
and $D$ a subset of $A$, called the set of \textbf{designated
truth values}. Let $h$ be a homomorphism from $\goth{Tm}_{\Omega}(V)$
into $\GM$. We write $\auf \GM, h\zu \vDash \varphi$ if
$h(\varphi) \in D$ and say that $\varphi$ is \textbf{true under}
$h$ \textbf{in} $\GM$. Further, we write $\Delta \vDash_{\GM} \varphi$
if for all homomorphisms $h \colon \goth{Tm}_{\Omega}(V) \pf \GA$: if
$h[\Delta] \subseteq D$ then $h(\varphi) \in D$.
\end{defn}
%%
\begin{prop}
If $\GM$ is a matrix for $L$, $\vDash_{\GM}$ is a structural
consequence relation.
\end{prop}
%%%
Notice that in boolean logic $\GA$ is the 2--element boolean
algebra and $D = \{1\}$, but we shall encounter other cases
later on. Here is a general method for obtaining matrices.
%%
\begin{defn}
%%%
\index{set!deductively closed}%%
\index{deductively closed set}%%
\index{set!consistent}%%
\index{set!maximally consistent}%%
%%%
Let $L$ be a propositional language and $\vdash$ a consequence
relation. Put $\Delta^{\vdash} := \{\varphi : \Delta \vdash 
\varphi\}$. $\Delta$ is \textbf{deductively closed} if $\Delta 
= \Delta^{\vdash}$. It is \textbf{consistent} if $\Delta^{\vdash} 
\neq L$. It is \textbf{maximally consistent} if
it is consistent but no proper superset is.
\end{defn}
%%
\index{matrix!canonical}%%%
%%%%
A matrix $\GS$ is \textbf{canonical} for $\vdash$ if 
$\GS = \auf\goth{Tm}_{\Omega}(V), \Delta^{\vdash}\zu$ for some set
$\Delta$. (Here, $\goth{Tm}_{\Omega}(V)$ is the canonical algebra
with carrier set $L$ whose functions are just the associated string
functions.) It is straightforward to verify that
$\vdash\; \subseteq\; \vDash_{\GS}$. Now consider some set $\Delta$
and a formula such that $\Delta \nvdash \varphi$. Then put $\GS :=
\auf\goth{Tm}_{\Omega}(V), \Delta^{\vdash}\zu$ and let $h$
be the identity. Then $h[\Delta] = \Delta \subseteq \Delta^{\vdash}$,
but $h(\varphi) \not\in \Delta^{\vdash}$ by definition of
$\Delta^{\vdash}$. So, $\Delta \nvDash_{\GS} \varphi$. This
shows the following.
%%%
\begin{thm}[Completeness of Matrix Semantics]
%%
\label{thm:matrixcompleteness}
%%%
Let $\,\vdash\,$ be a structural consequence relation over $L$. Then
%%
\begin{equation}
\vdash\; = \;\bigcap \auf \vDash_{\GS} : \GS \mbox{ canonical for }
\vdash\zu
\end{equation}
%%
\end{thm}
%%
(The reader may verify that an arbitrary intersection of consequence
relations again is a consequence relation.) This theorem establishes
that for any consequence relation we can find enough matrices such
that they together characterize that relation. We shall notice also
the following. Given $\vdash$ and $\GM = \auf \GA, D\zu$, then
$\vDash_{\GM}\; \supseteq\; \vdash$ iff $D$ is closed under
the consequence. (This is pretty trivial: all it says is that if
$\Delta \vdash \varphi$ and $h$ is a homomorphism, then if
$h[\Delta] \subseteq D$ we must have $h(\varphi) \in D$.) Such sets are
%%%%
\index{filter}%%
%%%%
called \textbf{filters}. Now, let $\GM = \auf \GA, D\zu$ be a matrix,
and $\Theta$ a congruence on $\GA$. Suppose that for any $x$:
$[x]\Theta \subseteq D$ or $[x]\Theta \cap D = \varnothing$.
Then we call $\Theta$ 
%%%
\index{congruence relation!admissible}%%
%%%%
\textbf{admissible for} $\GM$ and put $\GM/\Theta := \auf \GA/\Theta, 
D/\Theta\zu$, where $D/\Theta := \{[x]\Theta
: x \in D\}$. The following is easy to show.
%%%
\begin{prop}
\label{prop:redmatrix}
Let $\GM$ be a matrix and $\Theta$ an admissible congruence on $\GM$.
Then $\vDash_{\GM/\Theta}\; = \;\vDash_{\GM}$.
\end{prop}
%%
\index{matrix!reduced}%%
%%%
Finally, call a matrix \textbf{reduced} if only the diagonal is an
admissible congruence. Then, by Proposition~\ref{prop:redmatrix}
and Theorem~\ref{thm:matrixcompleteness} we immediately derive that
every consequence relation is complete with respect to reduced
matrices. One also calls a class of matrices $\CK$ a (\textbf{matrix})
%%%
\index{matrix semantics}%%
\index{matrix semantics!adequate}%%
%%%
\textbf{semantics} and says $\CK$ is \textbf{adequate for} a consequence
relation $\vdash$ if $\vdash\; =\; \bigcap_{\GM \in \CK} \; \vDash_{\GM}$.

Now, given $L$ and $\vdash$, the system of signs for the consequence
relation is this.
%%
\begin{equation}
\Sigma_P := \{\auf \vec{x}, R, \vec{x}\zu :
    \vec{x} \in G\} \cup \{\auf \vec{x}, P, \vec{y}\zu :
    \vec{x} \vdash \vec{y}\}
\end{equation}
%%
How does this change the situation? Notice that we can axiomatize
the consequences by means of rules. The following is a set of
rules that fully axiomatizes the consequence. The proof of that
will be left to the reader (see the exercises), since it is only
peripheral to our interests.
%%
\begin{subequations}
\label{eq:41bool}
\begin{align}
\rho_d & := \auf \{\mbox{\mtt (\symbol{5}(\symbol{5}p))}\}, 
	\mbox{\mtt p}\zu \\
\rho_{dn} & := \auf \{\mbox{\mtt p}\}, 
	\mbox{\mtt (\symbol{5}(\symbol{5}p))}\zu \\
\rho_{u} & := \auf \{\mbox{\mtt p}, \mbox{\mtt (\symbol{5}p)}\}, 
	\mbox{\mtt p0}\zu \\
\rho_c  & := \auf \{\mbox{\mtt p}, \mbox{\tt p0}\}, 
	\mbox{\mtt (p\symbol{4}p0)}\zu \\
\rho_{p0} & := \auf \{\mbox{\mtt (p\symbol{4}p0)}\}, \mbox{\mtt p}\zu \\
\rho_{p1} & := \auf \{\mbox{\mtt (p\symbol{4}p0)}\}, \mbox{\mtt p0}\zu \\
\rho_{mp} & := \auf \{\mbox{\mtt p}, 
	\mbox{\mtt (\symbol{5}(p\symbol{4}(\symbol{5}p0)))}\}, 
	\mbox{\mtt p0}\zu
\end{align}
\end{subequations}
%%
With each rule we can actually associate a mode. We only give
examples, since the general scheme for defining modes is easily
extractable.
%%
\begin{align}
\mbox{\mtt F}_{\mbox{\smtt dn}}(\auf \vec{x}, P, 
	\mbox{\mtt (\symbol{5}(\symbol{5}$\vec{y}$))} \zu 
	& := \auf \vec{x}, P, \vec{y}\zu \\
\mbox{\mtt F}_{\mbox{\smtt c}}(\auf \vec{x}, P, \vec{y}\,\zu, 
	\auf\vec{x}, P, \vec{z}\, \zu) 
	& :=
\auf \vec{x}, P, \mbox{\mtt ($\vec{y}$\symbol{4}$\vec{z}$)}\zu
\end{align}
%%
If we have {\mtt\symbol{25}} as a primitive symbol then the following 
mode corresponds to the rule $\rho_{mp}$, Modus Ponens.
%%
\begin{equation}
\mbox{\tt F}_{\mbox{\smtt mp}}(\auf\vec{x}, P, 
	\mbox{\mtt ($\vec{y}$\symbol{25}$\vec{z}$)}\zu, 
	\auf \vec{x}, P, \vec{y}\zu) :=
    \auf \vec{x}, P, \vec{z}\zu
\end{equation}
%%
This is satisfactory in that it allows to derive all and only
the consequences of a given proposition. A drawback is that the
functions on the exponents are nonincreasing. They always
return $\vec{x}$. The structure term of the sign
$\auf \vec{x}, P, \vec{y}\zu$ on the other hand encodes a
derivation of $\vec{y}$ from $\vec{x}$.

Now, the reader may get worried by the proliferation of different
semantics. Aren't we always solving a different problem? Our
answer is indirect. The problem is that we do not know exactly
what meanings are. Given a natural language, what we can observe
more or less directly is the exponents. Although it is not easy
to write down rules that generate them, the entities are more or
less concrete. A little less concrete are the syntactic categories.
We have already seen in the previous chapter that the assignment
of categories to strings (or other exponents, see next chapter)
are also somewhat arbitrary. We shall return to this issue. Even
less clearly definable, however, are the meanings. What, for
example, is the meaning of \eqref{eq:410}?
%%
\begin{equation}
\label{eq:410} 
\mbox{\tt Caesar crossed the Rubicon.}
\end{equation}
%%
The first answer we have given was: a truth value. For this sentence
is either true or false. But even though it is true, it might have
been false, just in case Caesar did not cross the Rubicon. What makes
us know this? The second answer (for first--order theories) is: the
meaning is a set of models. Knowing what the model is and what the
variables are assigned to, we know whether that sentence is true. But
we simply cannot look at all models, and still it seems that we know
what \eqref{eq:410} means. Therefore the next answer is: its meaning
is an algorithm, which, given a model, tells us whether the sentence
is true. Then, finally, we do not have to know everything in order to
know whether \eqref{eq:410} is true. Most facts are irrelevant, for
example, whether Napoleon was French. On the other hand, suppose we
witness Caesar walk across the Rubicon, or suppose we know for 
sure that first he was north of the Rubicon and the next day to the 
south of it. This will make us believe that \eqref{eq:410} is true. Thus, the
algorithm that computes the truth value does not need all of a model;
a small part of it actually suffices. We can introduce partial
models and define algorithms on them, but all this is a variation
on the same theme. A different approach is provided by our last
answer: a sentence means whatever it implies.

We may cast this as follows. Start with the set $L$ of propositions
and a set (or class) $\CM$ of models. A \textbf{primary} (or 
%%%
\index{semantics!primary}%%
%%%%
\textbf{model theoretic}) \textbf{semantics} is given in terms of 
a relation $\vDash \subseteq L \times \CM$. Most approaches are 
variants of the primary semantics, since they more or less characterize
meanings in terms of facts. However, from this semantics we may
define a \textbf{secondary semantics}, 
%%%%
\index{semantics!secondary}%%
%%%%
which is the semantics of
consequence. $\Delta \vDash \varphi$ iff for all $M \in
\CM$: if $M \vDash \delta$ for all $\delta \in \Delta$ then $M
\vDash \varphi$. (We say in this case that $\Delta$ entails
$\varphi$.) Secondary semantics is concerned only with the
relationship between the objects of the language, there is no
model involved. It is clear that the secondary semantics is not
fully adequate. Notice namely that knowing the logical
relationship between sentences does not reveal anything about the
nature of the models. Second, even if we knew what the models were:
we could not say whether a given sentence is true in a given model
or not. It is perfectly conceivable that we know English to the
extent that we know which sentences entail which other sentences,
but still we are unable to say, for example, whether or not
\eqref{eq:410} is true even when we witnessed Caesar cross the
Rubicon. An example might make this clear. Imagine that all I know 
is which sentences of English imply which other sentences, but 
that I know nothing more about their actual meaning. Suppose now 
that the house is on fire. If I realize this I know that I am in 
danger and I act accordingly. However, suppose that someone shouts 
\eqref{eq:411} at me. Then I can infer that he thinks \eqref{eq:411}
is true. This will certainly make me believe that \eqref{eq:411} 
is true and even that \eqref{eq:412} is true as well. But still 
I do not know that the house is on fire, nor that I am in danger. 
%%
\begin{align}
\label{eq:411} & \mbox{\tt The house is on fire.} \\
\label{eq:412} & \mbox{\tt I am in danger.}
\end{align}
%%
Therefore, knowing how sentences hang together in a deductive
system has little to do with the actual world. The situation
is not simply remedied by knowing some of the meanings.
Suppose I additionally know that \eqref{eq:411} means
that the house is on fire. Then if I see that the house is on
fire then I know that I am in danger, and I also know that
\eqref{eq:412} is the case. But I still may fail to see that
\eqref{eq:412} means that I am in danger. It may just mean
something else that is being implied by \eqref{eq:411}.
This is reminiscent of Searle's thesis that language is about
the world: knowing what things mean is not constituted by an
ability to manipulate certain symbols. We may phrase this as
follows.
%%
\begin{quote}
{\sl Indeterminacy of secondary semantics.} No secondary
semantics can fix the truth conditions of propositions uniquely
for any given language.
\end{quote}
%%
Searle's claims go further than that, but this much is perhaps
quite uncontroversial. Despite the fact that secondary semantics
is underdetermined, we shall not deal with primary semantics at
all. We are not going to discuss what a word, say, {\tt life} {\it
really\/} means --- we are only interested in how its meaning
functions language internally. Formal semantics really cannot do
more than that.
%%
\nocite{zimmermann:meaning}
%%
In what is to follow we shall sketch an algebraic approach to
semantics. This contrasts with the far more widespread
model--theoretic approach. The latter may be more explicit
and intuitive, but on the other hand it is quite
inflexible.

%%%
\index{Leibniz' Principle}%%
\index{Leibniz, Gottfried W.}%%
%%%
We begin by examining a very influential principle in semantics,
called Leibniz' Principle.  We quote one of its original formulation
from \cite{leibniz:kalkuel} (from {\it Specimen Calculi Coincidentium}, 
(1), 1690). {\it Eadem vel Coincidentia sunt quae
sibi ubique substitui possunt salva veritate. Diversa quae non
possunt.} Translated it says: {\it The same or coincident are
those which can everywhere be substituted for each other not
affecting truth. Different are those that cannot.} Clearly,
substitution must be understood here in the context of sentences,
and we must assume that what we substitute is constituent
occurrences of the expressions. We therefore reformulate the
principle as follows.
%%
\begin{quote}
{\sl Leibniz' Principle.}
Two expressions $A$ and $B$ have the same meaning iff
in every sentence any occurrence of $A$ can be substituted by $B$
and any occurrence of $B$ by $A$ without changing the truth of
that sentence.
\end{quote}
%%
To some people this principle seems to assume bivalence. If
there are more than two truth values we might interpret Leibniz'
original definition as saying that substitution does not change
the truth value rather than just truth. (See also Lyons for
a discussion.) We shall not do that, however. First we give some 
unproblematic examples.  In second order logic ($\mathsf{SO}$, see 
Chapter~\ref{kap5-1}), the following is a theorem.
%%
\begin{equation}
\label{eq:leibniz}
(\forall x)(\forall y)(x \doteq y \dpf (\forall P)(P(x) \dpf P(y)))
\end{equation}
%%
Hence, Leibniz' Principle 
%%%
\index{Leibniz' Principle}%%%
%%%
holds of second order logic with respect
to terms. There is general no identity relation for predicates, but
if there is, it is defined according to Leibniz' Principle: two
predicates are equal iff they hold of the same individuals.
This requires full second order logic, for what we want to have is the
following for each $n \in \omega$ (with $P_n$ and $Q_n$ variables
for $n$--ary relations):
%%
\begin{equation}
(\forall P_n)(\forall Q_n)(P_n \doteq Q_n \dpf
(\forall \vec{x})(P_n(\vec{x}) \dpf Q_n(\vec{x})))
\end{equation}
%%
(Here, $\vec{x}$ abbreviates the $n$--tuple $x_0, \dotsc,
x_{n-1}$.) \eqref{eq:leibniz} is actually the basis for Montague's
%%%%
\index{Montague, Richard}%%%
%%%
type raising. Recall that Montague identified an individual
with the set of all of its properties. In virtue of \eqref{eq:leibniz}
this identification does not conflate distinct individuals. To turn
that around: by Leibniz' Principle, this identification is
one--to--one. We shall see in the next section that boolean algebras
of any kind can be embedded into powerset algebras. The background of
this proof is the result that if there are two elements $x$, $y$ in a
boolean algebra $\GB$ and for all homomorphisms $h \colon \GB  \pf 
\mathbf{2}$ we have $h(x) = h(y)$, then $x = y$. (More on that in 
the next section. We have to use homomorphisms here since properties are
functions that commute with the boolean operations, that is to say,
homomorphisms.) Thus, Leibniz' Principle also holds for boolean
%%%
\index{Leibniz' Principle}%%%
%%%
semantics, defined in Section~\ref{kap6}.\ref{kap6-2}. Notice that the proof
relies on the Axiom of Choice (in fact the somewhat weaker Prime
Ideal Axiom), so it is not altogether innocent.

We use Leibniz' Principle to detect whether two items have the same
meaning. One consequence of this principle is that semantics is
essentially unique. If $\mu \colon A^{\ast} \stackrel{p}{\epi} M$,
$\mu' \colon A^{\ast} \stackrel{p}{\epi} M'$ are surjective functions
assigning meanings to expressions, and if both satisfy Leibniz'
Principle, then there is a bijection $\pi \colon M \pf M'$  such that
$\mu' = \pi \circ \mu$ and $\mu = \pi^{-1} \circ \mu'$. Thus, as
far as formal semantics is concerned, any solution is as good any
other. 

As we have briefly mentioned in Section~\ref{kap3}.\ref{kap3-3}, we may use
the same idea to define types. This method goes back to Husserl,
and is a key ingredient to the theory of compositionality by
Wilfrid Hodges 
%%%
\index{Hodges, Wilfrid}%%%
%%%
(see his \shortcite{hodges:compositionality}). A
type is a class of expressions that can be substituted for each other
without changing meaningfulness. Hodges just uses pairs of
exponents and meanings. If we want to assimilate his setup to
ours, we may add a category $U$, and let for every mode $f$,
$f^{\tau}(U, \dotsc, U) := U$. However, the idea is to do without
categories.  If we further substract the meanings, we get what
Hodges calls a {\it grammar}. We prefer to call it an 
\textbf{H--grammar}. (The letter \textbf{H} honours Hodges here.)
%%%
\index{H--grammar}%%
%%%
Thus, an H--grammar is defined by some signature and corresponding
operations on the set $E$ of exponents, which may even be partial.
An
%%%
\index{H--semantics}%%
%%%
\textbf{H--semantics} is a partial map $\mu$ from the structure terms
(!) to a set $M$ of meanings. Structure terms $\Gs$ and $\Gt$ are
%%%%
\index{synonymy}%%%
%%%%
\textbf{synonymous} if $\mu$ is defined on both and $\mu(\Gs) =
\mu(\Gt)$. We write $\Gs \equiv_{\mu} \Gt$ to say that $\Gs$ and
$\Gt$ are synonymous. (Notice that $\Gs \equiv_{\mu} \Gs$ iff
$\mu$ is defined on $\Gs$.) An H--semantics $\nu$ is
%%%%
\index{equivalence}%%
%%%%
\textbf{equivalent} to $\mu$  if $\equiv_{\mu}\; =\; \equiv_{\nu}$.
%%%
\index{synonymy!H--\faul}%%
%%%%
An \textbf{H--synonymy} is an equivalence relation on a subset
of the set of structure terms. We call that subset the \textbf{field}
of the H--synonymy. Given an H--synonymy $\equiv$,
we may define $M$ to be the set of all equivalence classes
of $\equiv$, and set $\mu^{\equiv}(\Gs) := [\Gs]_{\equiv}$
iff $\Gs$ is in that subset, and undefined otherwise.
Thus, up to equivalence, H--synonymies and H--semantics
are in one--to--one correspondence. We say that $\equiv'$
\textbf{extends} $\equiv$ if the field of $\equiv'$ contains the
field of $\equiv$, and the two coincide on the field of
$\equiv$.
%%%
\begin{defn}
%%%
\index{category!$\mu$--\faul}%%
%%%
Let $G$ be an H--grammar and $\mu$ an H--semantics for it. We
write $\Gs \sim_{\mu} \Gs'$ iff for every structure
term $\Gt$ with a single free variable $x$, $[\Gs/x]\Gt$ is
$\mu$--meaningful iff $[\Gs'/x]\Gt$ is
$\mu$--meaningful. The equivalence classes of $\sim_{\mu}$
are called the $\mu$--\textbf{categories}.
\end{defn}
%%
This is the formal rendering of the `meaning categories' that
Husserl defines.
%%%
\begin{defn}
%%%%
\index{synonymy!Husserlian}%%
%%%%
$\nu$ and its associated synonymy is called $\mu$--\textbf{Husserlian}
if for all structure terms $\Gs$ and $\Gs'$: if $\Gs \equiv_{\nu}
\Gs'$ then $\Gs \sim_{\mu} \Gs'$. $\mu$ is called
\textbf{Husserlian} if it is $\mu$--Husserlian.
\end{defn}
%%
It is worthwhile to compare this definition with Leibniz' Principle.
%%%
\index{Leibniz' Principle}%%%
%%%
The latter defines identity in meaning via intersubstitutability
in all sentences; what must remain constant is truth. Husserl's meaning
categories are also defined by intersubstitutability in all sentences;
however, what must remain constant is the meaningfulness. We may connect
these principles as follows.
%%%
\begin{defn}
%%%
\index{structure term!sentential}%%%%
\index{synonymy!Leibnizian}%%%
%%%%
Let $\Sent$ be a set of structure terms and $\Delta 
\subseteq \Sent$. We call $\Gs$ \textbf{sentential}
if $\Gs \in \Sent$, and \textbf{true} if $\Gs \in \Delta$. $\mu$ is
\textbf{Leibnizian} if for all structure terms $\Gu$ and $\Gu'$:
$\Gu \equiv_{\mu} \Gu'$ iff for all structure terms $\Gs$
such that $[\Gu/x]\Gs \in \Delta$ also $[\Gu'/x]\Gs  \in \Delta$
and conversely.
\end{defn}
%%%
Under mild assumptions on $\mu$ it holds that Leibnizian
implies Husserlian. The following is from
\cite{hodges:compositionality}.
%%%
\begin{thm}[Hodges]
%%%
\index{Hodges, Wilfrid}%%%
\label{thm:hodges}%%
Let $\mu$ be an H--semantics for the H--grammar $G$. Suppose
further that every subterm of a $\mu$--meaningful structure term is
again $\mu$--meaningful. Then the following are equivalent.
%%
\begin{dingautolist}{192}
\item
For each mode $f$ there is an $\Omega(f)$--ary function $f^{\mu} \colon
M^{\Omega(f)} \pf M$ such that $\mu$ is a homomorphism of partial
algebras.
\item
If $\Gs$ is a structure term and $\Gu_i$, $\Gv_i$ ($i < n$) are
structure terms such that $[\Gu_i/x_i : i < n]\Gs$ and
$[\Gv_i/x_i : i < n]\Gs$ are both $\mu$--meaningful and if
for all $i < n$ $\Gu_i \equiv_{\mu} \Gv_i$ then
%%
$$[\Gu_i/x_i : i < n]\Gs \equiv_{\mu} [\Gv_i/x_i : i < n]\Gs\; .$$
%%
\end{dingautolist}
%%
Furthermore, if $\mu$ is Husserlian then the second already
holds if it holds for $n = 1$.
\end{thm}
%%%
It is illuminating to recast the approach by Hodges in algebraic
terms. This allows to compare it with the setup of
Section~\ref{kap3}.\ref{kap3-1}. Moreover, it will also give a proof of
Theorem~\ref{thm:hodges}. We start with a signature $\Omega$. The
set $\Tm_{\Omega}(X)$ forms an algebra which we have
denoted by $\goth{Tm}_{\Omega}(X)$. Now select a subset $D
\subseteq \Tm_{\Omega}(X)$ of {\it meaningful terms}. It
turns out that the embedding $i \colon D \mono \Tm_{\Omega}(X) 
\colon x \mapsto x$ is a strong homomorphism iff 
$D$ is closed under subterms. We denote the induced
algebra by $\GD$. It is a partial algebra. The map $\mu \colon D \pf M$
induces an equivalence relation $\equiv_{\mu}$. There are functions
$f^{\mu} \colon M^{\Omega(f)} \pf M$ that make $M$ into an
algebra $\GM$ and $\mu$ into a homomorphism iff
$\equiv_{\mu}$ is a weak congruence relation (see
Definition~\ref{defn:pcongruence} and the remark following it).
This is the first claim of Theorem~\ref{thm:hodges}. For the
second claim we need to investigate the structure of partial
algebras.
%%%
\begin{defn}
%%%
\index{$\asymp_{\GA}$, $\asymp$}%%%
%%%
Let $\GA$ be a partial $\Omega$--algebra. Put $x \asymp_{\GA} y$
(or simply $x \asymp y$) if for all $f \in \Pol_1(\GA)$:
$f(x)$ is defined iff $f(y)$ is defined.
\end{defn}
%%%
\begin{prop}
\label{prop:strongcong}
Let $\GA$ be a partial $\Omega$--algebra.
(a) $\asymp_{\GA}$ is a strong congruence relation on $\GA$. (b) A
weak congruence on $\GA$ is strong iff it is contained
in $\asymp_{\GA}$.
\end{prop}
%%%
\proofbeg%%
(a) Clearly, $\asymp$ is an equivalence relation. So, let $f \in
F$ and $a_i \asymp c_i$ for all $i < \Omega(f)$. We have to show
that $f(\vec{a}) \asymp f(\vec{c})$, that is, for all $g \in
\Pol_1(\GA)$: $g(f(\vec{a}))$ is defined iff
$g(f(\vec{c}))$ is. Assume that $g(f(\vec{a}))$ is defined. The
function $g(f(x_0, a_1, \dotsc, a_{\Omega(f)-1}))$ is a unary
polynomial $h_0$, and $h_0(a_0)$ is defined. By definition of
$\asymp$, $h_0(c_0) = g(f(c_0, a_1, \dotsc, a_{\Omega(f)-1}))$ 
is also defined. Next, 
%%%
\begin{equation}
h_1(x_1) := f(g(c_0, x_1, a_2, \dotsc, a_{\Omega(f)-1}))
\end{equation}
%%%%
is a unary polynomial and defined on $a_1$. So, it is defined on $c_1$ 
and we have $h_1(c_1) = f(g(c_0,
c_1, a_2, \dotsc, a_{\Omega(f)-1}))$. In this way we show that
$f(g(\vec{c}))$ is defined. (b) Let $\Theta$ be a weak congruence.
Suppose that it is not strong. Then there is a polynomial $f$ and
vectors $\vec{a}, \vec{c} \in A^{\Omega(f)}$ with $a_i\; \Theta\;
c_i$ ($i < \Omega(f)$) such that $f(\vec{a})$ is defined but
$f(\vec{c})$ is not. Now, for all $i < \Omega(f)$,
%%
\begin{multline}
\label{eq:theta}
f(a_0, \dotsc, a_{i-1}, a_i, c_{i+1}, \dotsc, c_{\Omega(f)-1})
\\ \qquad
\; \Theta\; f(a_0, \dotsc, a_{i-1}, c_i, c_{i+1}, \dotsc,
c_{\Omega(f)-1})
\end{multline}
%%
if both sides are defined. Now, $f(\vec{a})$ is not
$\Theta$--congruent to $f(\vec{c})$. Hence there is an $i <
\Omega(f)$ such that the left hand side of \eqref{eq:theta} 
is defined and the right hand side is not. Put 
%%
\begin{equation}
h(x) := f(a_0, \dotsc, a_{i-1}, x, c_{i+1}, \dotsc, 
c_{\Omega(f)-1})
\end{equation}
%%
Then $h(a_i)$ is defined, $h(c_i)$ is not, but $a_i\; \Theta \; c_i$. 
So, $\Theta\; \nsubseteq\; \asymp$. Conversely, if $\Theta$ is strong 
we can use \eqref{eq:theta} to show inductively that if $f(\vec{a})$ is 
defined, so are all members of the chain. Hence $f(\vec{c})$ is defined. 
And conversely.
\proofend
%%%
\begin{prop}
\label{prop:punary}%%
Let $\GA$ be a partial algebra and $\Theta$ an
equivalence relation on $\GA$. $\Theta$ is a strong congruence 
iff for all $g \in \Pol_1(\GA)$ and all $a, c
\in A$ such that $a\; \Theta\; c$: $g(a)$ is defined iff
$g(c)$ is, and then $g(a)\; \Theta\; g(c)$.
\end{prop}
%%
The proof of this claim is similar. To connect this with the
theory by Hodges, notice that $\sim_{\mu}$ is the same as
$\asymp_{\GD}$. $\equiv_{\mu}$ is Husserlian iff
$\equiv_{\mu} \subseteq \asymp_{\GD}$.
%%%
\begin{prop}
$\equiv_{\mu}$ is Husserlian iff it is contained in
$\asymp_{\GD}$ iff it is a strong congruence.
\end{prop}
%%%
Propositions~\ref{prop:strongcong} and \ref{prop:punary} together
show the second claim of Theorem~\ref{thm:hodges}.

If $\bullet$ is the only operation, we can actually use this
method to define the types (see Section~\ref{kap3}.\ref{kap3-3}). In
the following sections we shall develop an algebraic account
of semantics, starting first with boolean algebras and then
going over to intensionality, and finally carrying out the
full algebraization.

{\it Notes on this section.}
The idea that the logical interconnections between sentences 
constitute their meanings is also known as {\it holism}.
This view and its implications for semantics is discussed 
by Dresner \shortcite{dresner:holism}. 
%%%
\index{Dresner, Eli}%%
%%%
We shall briefly also mention the problem of reversibility
(see Section~\ref{kap6}.\ref{kap6-2}).
Most formalisms are designed only for assigning meanings 
to sentences, but it is generally hard or impossible to assign 
a sentence that expresses a given content. We shall briefly touch 
on that issue in  Section~\ref{kap6}.\ref{kap6-2}. 
%%
\vplatz
\exercise
Prove Proposition~\ref{prop:redmatrix}.
%%
\vplatz
\exercise
Let $\rho = \auf \Delta, \varphi\zu$ be a rule. Devise a mode
$\mbox{\tt M}_{\rho}$ that captures the effect of this rule
in the way discussed above. Translate the rules given above
into modes. What happens with 0--ary rules (that is, rules
with $\Delta = \varnothing$)?
%%
\vplatz \exercise There is a threefold characterization of a
consequence: as a consequence relation, as a closure operator, and
as a set of theories. Let $\vdash$ be a consequence relation. Show
that $\Delta \mapsto \Delta^{\vdash}$ is a closure operator. The
closed sets are the theories. If $\vdash$ is structural the set of
theories of $\vdash$ are inversely closed under substitutions. That
is to say, if $T$ is a theory and $\sigma$ a substitution, then
$\sigma^{-1}[T]$ is a theory as well. Conversely, show that every
closure operator on $\wp(\goth{Tm}_{\Omega}(V))$ gives rise to a
consequence relation and that the consequence relation is
structural if the set of theories is inversely closed under
substitutions.
%%
\vplatz
\exercise
Show that the rules \eqref{eq:41bool} are complete for boolean logic 
in {\mtt\symbol{4}} and {\mtt\symbol{5}}.
%%
\vplatz
\exercise
Show that for any given finite signature the set of predicate
logical formulae valid in all finite structures for that signature
is co--recursively enumerable. (The latter means that its
complement is recursively enumerable.)
%%%
\vplatz
\exercise
Let $L$ be a first--order language which contains at least the
symbol for equality ({\mtt =}). Show that a first--order theory 
$T$ in $L$ satisfies Leibniz' Principle if the following holds 
%%%
\index{Leibniz' Principle}%%%
%%%
for any relation symbol $r$
%%
\begin{equation}
T; \{\mbox{\mtt ($x_i$=$y_i$)} : i < \Xi(r)\} \vdash^{\mathsf{FOL}}
    \mbox{\mtt ($r$($\vec{x}$)\symbol{25}$r$($\vec{y}$))}
\end{equation}
%%
and the following for every function symbol $f$:
%%
\begin{equation}
T; \{\mbox{\mtt ($x_i$=$y_i$)} : i < \Omega(f)\} \vdash^{\mathsf{FOL}}
    \mbox{\mtt ($f$($\vec{x}$)=$f$($\vec{y}$))}
\end{equation}
%%
Use this to show that the first--order set theory $\mathsf{ZFC}$ 
satisfies Leibniz' Principle. Further, show that every equational theory
satisfies Leibniz' Principle.
%%%
\index{Leibniz' Principle}%%%
%%%
%%

 \section{Boolean Semantics}
%
%
%
Boolean algebras are needed in all areas of semantics, as is 
demonstrated in \cite{keenanfaltz:boolean}. Boolean algebras 
are the structures that correspond to propositional logic 
in the sense that the variety turns out to be generated from 
just one algebra: the algebra with two values $0$ and $1$, 
and the usual operations (Theorem~\ref{thm:var}). 
Moreover, the calculus of equations and the usual deductive 
calculus mutually interpret each other (Theorem~\ref{thm:eqtovdash}). 
This allows to show that the axiomatization is complete 
(Theorem~\ref{thm:2comp}). 
%%
\begin{defn}
\label{defn:ba}
%%%
\index{boolean algebra}%%
\index{algebra!boolean}%%
%%%
An algebra $\auf B, 0, 1, -,\cap, \cup\zu$, where $0, 1 \in B$, $-
\colon B \pf B$ and $\cap, \cup \colon B^2 \pf B$, is called a 
\textbf{boolean algebra} if it satisfies the following equations 
for all $x, y, z \in B$.
%%
$$\begin{array}{l@{\;}l@{\; = \;}l@{\quad\quad}l@{\;}l@{\; = \;}l}
\mbox{\rm (as$\cap$)} & x \cap (y \cap z) &  
(x \cap y) \cap z &
    \mbox{\rm (as$\cup$)} & x \cup (y \cup z) 
    & (x \cup y) \cup z \\
\mbox{\rm (co$\cap$)} & x \cap y & y \cap x &
    \mbox{\rm (co$\cup$)} & x \cup y & y \cup x \\
\mbox{\rm (id$\cap$)} & x \cap x & x &
    \mbox{\rm (id$\cup$)} & x \cup x & x \\
\mbox{\rm (ab$\cap$)} & x \cap (y \cup x) & x &
    \mbox{\rm (ab$\cup$)} & x \cup (y \cap x) & x \\
\mbox{\rm (di$\cap$)} & x \cap (y \cup z) & &
    \mbox{\rm (di$\cup$)} & x \cup (y \cap z) & \\
& \multicolumn{2}{r}{(x \cap y) \cup (x \cap z)\quad} &
& \multicolumn{2}{r}{(x \cup y) \cap (x \cup z)} \\
\mbox{\rm (li$-$)} & x \cap (-x) & 0 &
    \mbox{\rm (ui$\cup$)} & x \cup (-x) & 1 \\
\mbox{\rm (ne$\cap$)} & x \cap 1 & x &
    \mbox{\rm (ne$0$)} & x \cup 0 & x \\
\mbox{\rm (dm$\cap$)} & -(x \cap y) & (-x) \cup (-y) &
    \mbox{\rm (dm$\cup$)} & -(x \cup y) & (-x) \cap (-y) \\
\mbox{\rm (dn$-$)} & -(- x) & x & \multicolumn{3}{c}{}
\end{array}$$
%%
\end{defn}
%%
%%%
The operation $\cap$ is generally referred to as the \textbf{meet}
%%%
\index{meet}%%
%%%
(\textbf{operation}) and $\cup$ as the \textbf{join} 
%%%
\index{join}%%
%%%
(\textbf{operation}).  $- x$ is called the \textbf{complement} 
%%%
\index{complement}%%%
%%%
of $x$ and $0$ the \textbf{zero} 
%%%
\index{zero}%%
%%%
and 1 the \textbf{one} or \textbf{unit}. 
%%%
\index{one}%%
%%%
Obviously, the boolean algebras 
form an equationally definable class of algebras.

The laws (as$\cap$) and (as$\cup$) are called \textbf{associativity
laws}, 
%%%%
\index{associativity}%%
%%%%
the laws (co$\cap$) and (co$\cup$) \textbf{commutativity laws},
%%%
\index{commutativity}%%
%%%%
(id$\cap$) and (id$\cup$) the \textbf{laws of idempotence} 
%%%
\index{idempotence}%%
%%%
and (ab$\cap$) and (ab$\cup$) the \textbf{laws of absorption}.  
%%%
\index{absorption}%%
%%%
A structure $\auf L, \cap, \cup\zu$ satisfying these laws is called
a \textbf{lattice}. 
%%%
\index{lattice}%%
%%%
If only one operation is present and the
corresponding laws hold we speak of a \textbf{semilattice}. 
%%%
\index{semilattice}%%
%%%
(So, a semilattice is a semigroup that satisfies commutativity and
idempotence.) Since $\cap$ and $\cup$ are associative and
commutative, we follow the general practice and omit brackets
whenever possible. So, rather than $(x \cap (y \cap z))$
we simply write $x \cap y \cap z$. Also, $(x \cap (y \cap x))$
is simplified to $x \cap y$. Furthermore, given a
finite set $S \subseteq L$ the notation $\bigcup \auf x : x \in S\zu$
or simply $\bigcup S$ is used for the iterated join of the
elements of $S$. This is uniquely defined, since the join is
independent of the order and multiplicity in which the elements
appear.
%%
\begin{defn}
Let $\GL$ be a lattice. We write $x \leq y$ if
$x \cup y = y$.
\end{defn}
%%
Notice that $x \leq y$ iff $x \cap y = x$. This
can be shown using the equations above. We leave this as an
exercise to the reader. Notice also the following.
%%
\begin{lem}
\label{lem:order}
\begin{dingautolist}{192}
\item
$\leq$ is a partial ordering.
\item
$x \cup y \leq z$ iff $x \leq z$ and $y \leq z$.
\item
$z \leq x \cap y$ iff $z \leq x$ and $z \leq y$.
\end{dingautolist}
\end{lem}
%%
\proofbeg
\ding{192} (a) $x \cup x = x$, whence $x \leq x$. (b) Suppose
that $x \leq y$ and $y \leq x$. Then we get $x \cup y = x$  and
$y \cup x = y$, whence $y = x \cup y = x$. (c) Suppose
that $x \leq y$ and $y \leq z$. Then $x \cup y = y$ and
$y \cup z = z$ and so $x \cup z = x \cup (y \cup z) =
(x \cup y) \cup z = y \cup z = z$. \ding{193} Let $x \cup y
\leq z$. Then, since $x \leq x \cup y$, we have $x \leq z$ by
(\ding{192}c); for the same reason also $y \leq z$. Now assume
$x \leq z$ and $y \leq z$. Then $x \cup z = y \cup z = z$
and so $z = z \cup z = (x \cup z) \cup (y \cup z) =
(x \cup y) \cup z$, whence $x \cup y \leq z$. \ding{194}
Similarly, using $x \leq y$ iff $x \cap y = x$.
\proofend

In fact, it is customary to define a lattice by means of $\leq$.
This is done as follows.
%% Bild %%
%%
\begin{defn}
Let $\leq $ be a partial order on $L$.  Let $X \subseteq L$ be an
arbitrary set. The \textbf{greatest lower bound} (\textbf{glb}) of 
%%%
\index{greatest lower bound (glb)}%%%
%%%
$X$, also denoted $\bigcap X$, is that element $u$ such that for 
all $z$: if $x \geq z$ for all $x \in X$ then also $u \geq z$ (if it 
exists). Analogously, the \textbf{least upper bound} (\textbf{lub}) 
%%%
\index{least upper bound (lub)}%%%
%%%
of $X$, denoted by $\bigcup X$, is that element $v$ such that for all
$z$: if $x \leq  z$ for all $x \in X$ then also $v \leq z$ (if it 
exists).
\end{defn}
%%
Notice that there are partial orderings which have no lubs. For 
example, let $L = \auf \{0,1,2,3\}, \preccurlyeq\zu$, where 
%%%
\begin{equation}
\preccurlyeq := \{\auf 0,0\zu, \auf 0,2\zu, \auf 0,3\zu, 
	\auf 1,1\zu, \auf 1,2\zu, \auf 1,3\zu, \auf 2,2\zu, 
	\auf 3,3\zu\}
\end{equation}
%%%
Here, $\{0,1\}$ has no lub. This partial ordering does therefore 
not come from a lattice. For 
by the facts established above, the join of two elements
$x$ and $y$ is simply the lub of $\{x, y\}$, and the
meet is the glb of $\{x, y\}$. It is left to the reader to
verify that these operations satisfy all laws of lattices.
So, a partial order $\leq$ is the order determined by a lattice
structure iff all finite sets have a least upper
bound and a greatest lower bound.

The laws (di$\cap$) and (di$\cup$) are the \textbf{distributivity
laws}. A lattice is called \textbf{distributive} if they hold in it.
%%%
\index{lattice!distributive}%%
%%%
A nice example of a distributive lattice is the following. Take
a natural number, say $28$, and list all divisors of it:
1, 2, 4, 7, 14, 28. Write $x \leq y$ if $x$ is a
divisor of $y$. (So, $2 \leq 14$, $2 \leq 4$, but not $4 \leq 7$.)
Then $\cap$ turns out to be the greatest common divisor and
$\cup$ the least common multiple. Another example is the
linear lattice defined by the numbers $< n$ with $\leq$ the
usual ordering. $\cap$ is then the minimum and $\cup$ the
maximum.

%%%
\index{lattice!bounded}%%
%%%
A \textbf{bounded lattice} is a structure $\auf L, 0, 1, \cap, \cup\zu$
which is a lattice with respect to $\cap$ and $\cup$, and in which
satisfies (ne$\cap$) and (ne$\cup$). From the definition of $\leq$,
(ne$\cap$) means that $x \leq 1$ for all $x$ and (ne$\cup$) that
$0 \leq x$ for all $x$. Every finite lattice has a least and a
largest element and can thus be extended to a bounded lattice.
This extension is usually done without further notice.
%%
\begin{defn}
%%%
\index{join irreducibility}%%
\index{meet irreducibility}%%
%%%
Let $\GL = \auf L, \cap, \cup\zu$ be a lattice.  An element $x$ is
\textbf{join irreducible in} $\GL$ if for all $y$ and $z$ such that
$x = y \cup z$ either $x = y$ or $x = z$. $x$ is \textbf{meet
irreducible} if for all $y$ and $z$ such that $x = y \cap z$
either $x = y$ or $x = z$.
\end{defn}
%%
It turns out that in a distributive lattice irreducible
elements have a stronger property. Call $x$ \textbf{meet prime}
%%%
\index{meet prime}%%
%%%
if for all $y$ and $z$: from $x \geq y \cap z$ follows $x \geq y$
or $x \geq z$. Obviously, if $x$ is meet prime it is also meet
irreducible. The converse is generally false. Look at $M_3$ shown
in Figure~\ref{fig:nondist}. Here, $c \geq a \cap b (= 0)$, but neither
$c \geq a$ nor $c \geq b$ holds.
%%
\begin{figure}
\begin{center}
\begin{picture}(10,8)
\put(5,1){\makebox(0,0){$\bullet$}}
    \put(5,0){\makebox(0,0){$0$}}
    \put(5,1){\line(-1,1){3}}
    \put(5,1){\line(0,1){3}}
    \put(5,1){\line(1,1){3}}
\put(2,4){\makebox(0,0){$\bullet$}}
    \put(1,4){\makebox(0,0){$a$}}
    \put(2,4){\line(1,1){3}}
\put(5,4){\makebox(0,0){$\bullet$}}
    \put(4,4){\makebox(0,0){$b$}}
    \put(5,4){\line(0,1){3}}
\put(8,4){\makebox(0,0){$\bullet$}}
    \put(9,4){\makebox(0,0){$c$}}
    \put(8,4){\line(-1,1){3}}
\put(5,7){\makebox(0,0){$\bullet$}}
    \put(5,8){\makebox(0,0){$1$}}
\end{picture}
\end{center}
\caption{The Lattice $M_3$}
\label{fig:nondist}
\end{figure}
%%
\begin{lem}
\label{lem:prime}
Let $\GL$ be a distributive lattice. Then $x$ is meet (join)
prime iff $x$ is meet (join) irreducible.
\end{lem}
%%
Let us now move on to the complement. (li$\cap$) and (ui$\cup$)
have no special name. They basically ensure that $-x$ is the
unique element $y$ such that $x \cap y = 0$ and $x \cup y = 1$.
%%%
\index{de Morgan law}%%
\index{double negation}%%
%%%
The laws (dm$\cap$) and (dm$\cup$) are called \textbf{de Morgan
laws}. Finally, (dn$-$) is the law of \textbf{double negation}.
%%
\begin{lem}
\label{lem:complement}
The following holds in a boolean algebra.
%%
\begin{dingautolist}{192}
\item
$x \leq y$ iff $-y \leq -x$.
\item
$x \leq y$ iff $x \cap (-y) = 0$ iff $(-x) \cup y = 1$.
\end{dingautolist}
\end{lem}
%%
\proofbeg
\ding{192} $x \leq y$ means $x \cup y = y$, and so
$- y = -(x \cup y) = (-x) \cap (-y)$, whence $-y \leq -x$.
From $-y \leq -x$ we now get $x = --x \leq --y = y$.
\ding{193} If $x \leq y$ then $x \cap y = x$, and so
$x \cap (-y) = (x \cap y) \cap (-y) = x \cap 0 = 0$.
Conversely, suppose that $x \cap (-y) = 0$. Then
$x \cap y = (x \cap y) \cup (x \cap (-y)) =
x \cap (y \cup (-y)) = x \cap 1 = x$. So, $x \leq y$.
It is easily seen that $x \cap (-y) = 0$ iff
$(-x) \cup y = 1$.
\proofend

We can use the terminology of universal algebra (see
Section~\ref{kap1}.\ref{kap1-1}). So, the notions of homomorphisms and
subalgebras, congruences, of these structures should be clear.
We now give some examples of boolean algebras.  The first example
is the powerset of a given set. Let $X$ be a set.
Then $\wp(X)$ is a boolean algebra with $\varnothing$ in place of
0, $X$ in place of $1$, $- A = X - A$, $\cap$ and $\cup$ the
intersection and union. We write $\GP(X)$ for this algebra.
%%%
\index{field of sets}%%
\index{$\GP(X)$}%%%
%%%
A subalgebra of this algebra is called a \textbf{field of sets}.
Also, a subset of $\wp(X)$ closed under the boolean operations
is called a field of sets. The smallest examples are the algebra
$\mathbf{1} := \GP(\varnothing)$, consisting just of one element
($\varnothing$), and $\mathbf{2} := \GP(\{\varnothing\})$, the
algebra of subsets of $1 = \{\varnothing\}$. Now, let $X$ be a set
and $\GB = \auf B, 0, 1, \cap, \cup, -\zu$ be a boolean algebra.
Then for two functions $f, g \colon X \pf B$ we may define $-f$,
$f \cap g$ and $f \cup g$ as follows.
%%
\begin{align}
\notag
(-f)(x) & := - f(x) \\
(f \cup g)(x) & := f(x) \cup g(x) \\
\notag
(f \cap g)(x) & := f(x) \cap g(x)
\end{align}
%%
Further, let $\uli{0} \colon X \pf B \colon x \mapsto 0$ and $\uli{1} \colon
X \pf B \colon x \mapsto 1$. It is easily verified that the set of all
functions from $X$ to $B$ form a boolean algebra: %%
$\auf B^X, \uli{0}, \uli{1}, -, \cap, \cup\zu$. We denote this
algebra by $\GB^X$. 
%%%
\index{$\GB^X$}%%%
%%%%
The notation has been chosen on purpose: this
algebra is nothing but the direct product of $\GB$ indexed over
$X$. A particular case is $\GB = \mathbf{2}$. Here, we may actually
think of $f \colon X \pf 2$ as the characteristic function $\chi_M$ of
a set, namely the set $f^{-1}(1)$. It is then again verified that
$\chi_{-M} = - \chi_M$, $\chi_{M\cap N} = \chi_M \cap \chi_N$,
$\chi_{M \cup N} = \chi_M \cup \chi_N$. So we find the following.
%%
\begin{thm}
$\mathbf{2}^X$ is isomorphic to $\GP(X)$.
\end{thm}
%%
We provide some applications of these results. The intransitive verbs
of English have the category $e\backslash t$. Their semantic type
is therefore $e \pf t$. This in turn means that they are interpreted
as functions from objects to truth values. We assume that the truth
values are just $0$ and $1$ and that they form a boolean algebra
with respect to the operations $\cap$, $\cup$ and $-$. Then we
can turn the interpretation of intransitive verbs into a boolean
algebra in the way given above. Suppose that the interpretation
of {\tt and}, {\tt or} and {\tt not} is also canonically extended
in the given way. That is: suppose that they can now also be
used for intransitive verbs and have the meaning given above.
Then we can account for a number of inferences, such as the
inference from \eqref{eq:ex1} to \eqref{eq:ex2} and \eqref{eq:ex3},
and from \eqref{eq:ex2} and \eqref{eq:ex3} together to
\eqref{eq:ex1}. Or we can infer that \eqref{eq:ex1} implies
that \eqref{eq:ex4} is false; and so on.
%%
\begin{align}
\label{eq:ex1} & \mbox{\tt Claver walks and talks.} \\
\label{eq:ex2} & \mbox{\tt Claver walks.} \\
\label{eq:ex3} & \mbox{\tt Claver talks.} \\
\label{eq:ex4} & \mbox{\tt Claver does not walk.}
\end{align}
%%
With the help of that we can now also assign a boolean structure to the
transitive verb denotations. For their category is $(e \backslash t)/e$,
which corresponds to the type $e \pf (e \pf t)$. Now that the
set functions from objects to truth values carries a boolean
structure, we may apply the construction again. This allows
us then to deduce \eqref{eq:ex6} from \eqref{eq:ex5}.
%%
\begin{align}
\label{eq:ex5} & \mbox{\tt Claver sees or hears Patrick}. \\
\label{eq:ex6} & \mbox{\tt Claver sees Patrick or Claver hears Patrick.}
\end{align}
%%
Obviously, any category that finally ends in $t$ has a space of
denotations associated to it that can be endowed with the structure
of a boolean algebra. (See also Exercise~\ref{ex:boolesch}.) These
are, however, not all categories. However, for the remaining ones
we can use a trick used already by Montague. 
%%%
\index{Montague, Richard}%%%
%%%
Montague was concerned
with the fact that names such as {\tt Peter} and {\tt Susan} denote
objects, which means that their type is $e$. Yet, they fill a subject
NP position, and subject NP positions can also be filled by (nominative)
quantified NPs such as {\tt some philosopher}, which are of type
$(e \pf t) \pf t$. In order to have homogeneous type assignment,
Montague lifted the denotation of {\tt Peter} and {\tt Susan} to
$(e \pf t) \pf t$. In terms of syntactic categories we lift from $e$
to $t/(e \backslash t)$. We have met this earlier in
Section~\ref{kap3}.\ref{kap3-2} as raising. Cast in terms of boolean
algebras this is the following construction. From an arbitrary
set $X$ we first form the boolean algebra $\GP(X)$ and
then the algebra $\mathbf{2}^{\GP(X)}$.
%%
\begin{prop}
The map $x \mapsto x^{\dagger}$ given by
$x^{\dagger}(f) := f(x)$ is an embedding of $X$ into
$\mathbf{2}^{\GP(X)}$.
\end{prop}
%%
\proofbeg
Suppose that $x \neq y$. Then $x^{\dagger}(\chi_{\{x\}}) =
\chi_{\{x\}}(x) = 1$, while $y^{\dagger}(\chi_{\{x\}}) =
\chi_{\{x\}}(y) = 0$. Thus $x^{\dagger} \neq y^{\dagger}$.
\proofend

To see that this does the trick, consider the following sentence.
%%
\begin{equation}
\label{eq:ex7} \mbox{\tt Peter and Susan walk.}
\end{equation}
%%
We interpret {\tt Peter} now by $\mathsf{peter}'^{\dagger}$,
where $\mathsf{peter}'$ is the individual Peter. Similarly,
$\mathsf{susan}'^{\dagger}$ interprets {\tt Susan}.
Then \eqref{eq:ex7} means
%%
\begin{align}
\begin{split}
  & (\mathsf{peter}'^{\dagger} \cap \mathsf{susan}'^{\dagger})(
\mathsf{walk}') \\
 = & 
(\mathsf{peter}'^{\dagger}(\mathsf{walk}')) \cap
    (\mathsf{susan}'^{\dagger}(\mathsf{walk}')) \\
 =  & \mathsf{walk}'(\mathsf{peter}') \cap
    \mathsf{walk}'(\mathsf{susan}')
\end{split}
\end{align}
%%
So, this licenses the inference from \eqref{eq:ex7} to
\eqref{eq:ex8} and \eqref{eq:ex9}, as required. (We have
tacitly adjusted the morphology here.)
%%
\begin{align}
\label{eq:ex8} & \mbox{\tt Peter walks.} \\
\label{eq:ex9} & \mbox{\tt Susan walks.}
\end{align}
%%
It follows that we can make the denotations of any linguistic
category a boolean algebra.

The next theorem we shall prove is that boolean algebras are
(up to isomorphism) the same as fields of sets. Before we prove
the full theorem we shall prove a special case, which is very
%%%
\index{atom}%%
%%%
important in many applications. An \textbf{atom} is an element
$x \neq 0$ such that for all $y \leq x$: either $y = 0$ or $y = x$.
$\At(\GB)$ 
%%%
\index{$\At(\GB)$}%%%
%%%
denotes the set of all atoms of $\GB$.
%%
\begin{lem}
In a boolean algebra, an element is an atom iff it is join irreducible.
\end{lem}
%%
This is easy to see. An atom is clearly join irreducible.
Conversely, suppose that $x$ is join irreducible. Suppose
that $0 \leq y \leq x$. Then 
%%%
\begin{equation}
x = (x \cap y) \cup (x \cap (-y)) = y \cup (x \cap (-y))
\end{equation}
%%%
By irreducibility, either $y = x$ or 
$x \cap (-y) = x$. From the latter we get $x \leq -y$, or 
$y \leq -x$, using Lemma~\ref{lem:order}. Since also $y \leq x$, 
$y \leq x \cap (-x) = 0$. So, $y = 0$. Therefore, $x$ is an atom. 
Put 
%%%
\begin{equation}
\wht{x} := \{y \in \At(\GB) : y \leq x\}
\end{equation}
%%%%
The map $x \mapsto \wht{x}$ is a homomorphism: 
$\wht{x} = \At(\GA) - \wht{x}$. For let $u$ be an atom. 
For any $x$, $u = (u \cap x) \cup (u \cap (-x))$;
and since $u$ is irreducible, $u = u \cap x$ or $u = u \cap (-x)$, 
which gives $u \leq x$ or $u \leq -x$. But not both, since 
$u > 0$. Second, $\wht{x \cap y} = \wht{x} \cap \wht{y}$, as
is immediately verified. 

Now, if $\GB$ is finite, $\wht{x}$ is nonempty iff $x \neq 0$. 
%%
\begin{lem}
If $\GB$ is finite, $x = \bigcup \wht{x}$.
\end{lem}
%%
\proofbeg
Put $x' := \bigcup \wht{x}$. Clearly, $x' \leq x$. Now suppose 
$x' < x$. Then $(-x') \cap x \neq 0$. Hence there is an atom 
$u \leq (-x') \cap x$, whence $u \leq x$. But $u \nleq x'$, a 
contradiction.
\proofend

A boolean algebra is said to be \textbf{atomic} %%%
%%%
\index{boolean algebra!atomic}%%%
%%%
if $x$ is the lub $\wht{x}$ for all $x$.
%%
\begin{thm}
Let $\GB$ be a finite boolean algebra.  The map $x \mapsto \wht{x}$
is an isomorphism from $\GB$ onto $\GP(\At(\GB))$.
\end{thm}
%%
Now we proceed to the general case. First, notice that this theorem 
is false in general. A subset $N$ of $M$ is called \textbf{cofinite} 
%%%
\index{set!cofinite}%%
%%%
if its complement, $M - N$, is finite. Let
$\Omega$ be the set of all subsets of $\omega$ which are
either finite or cofinite. Now, as is easily checked, $\Omega$
contains $\varnothing$, $\omega$ and is closed under complement,
union and intersection. The singletons $\{x\}$ are the atoms.
However, not every set of atoms corresponds to an element of the
algebra. A case in point is $\{\{2k\} : k \in \omega\}$. Its
union in $\omega$ is the set of even numbers,  which is neither
finite nor cofinite. Moreover, there exist infinite boolean
algebras that have no atoms (see the exercises). Hence, we must
take a different route.
%%
\begin{defn}
%%%%
\index{point}%%%
\index{$\pt(\GA)$}%%%
%%%%%
Let $\GB$ be a boolean algebra. A \textbf{point} is a homomorphism
$h \colon \GB \pf \mathbf{2}$. The set of points of $\GB$ is denoted
by $\pt(\GB)$.
\end{defn}
%%
Notice that points are necessarily surjective. For we must
have $h(0^{\GB}) = 0$ and $h(1^{\GB}) = 1$. (As a warning
to the reader: we will usually not distinguish $1^{\GB}$ and
$1$.)
%%
\begin{defn}
%%%%
\index{filter}%%%
\index{ultrafilter}%%%
%%%%
A \textbf{filter} of $\GB$ is a subset that satisfies the following.
%%
\begin{dingautolist}{192}
\item $1 \in F$.
\item If $x, y \in F$ then $x \cap y \in F$.
\item If $x \in F$ and $x \leq y$ then $y \in F$.
\end{dingautolist}
%%
A filter $F$ is called an \textbf{ultrafilter} if $F \neq B$
and there is no filter $G$ such that $F \subsetneq G \subsetneq B$.
\end{defn}
%%
A filter $F$ is an ultrafilter iff for all $x$: either
$x \in F$ or $-x \in F$. For suppose neither is the case. Then
let $F^x$ be the set of elements $y$ such that there is a
$u \in F$ with $y \geq u \cap x$. This is a filter, as is
easily checked. It is a proper filter: it does not contain
$-x$. For suppose otherwise. Then $-x \geq u \cap x$ for
some $u \in F$. By Lemma~\ref{lem:complement} this means that
$0 = u \cap x$, from which we get $u \leq -x$. So, $-x \in F$,
since $u \in F$. Contradiction.
%%
\begin{prop}
Let $h \colon \GB \pf \GA$ be a homomorphism of boolean algebras. Then
$F_h := h^{-1}(1^{\GA})$ is a filter of $\GB$. Moreover, for any
filter $F$ of $\GB$, $\Theta_F$ defined by $x\; \Theta_F\; y$ iff
$x \dpf y \in F$ is a congruence. The factor algebra
$\GB/\Theta_F$ is also denoted by $\GB/F$ and the map
$x \mapsto [x]{\Theta_F}$ by $h_F$.
\end{prop}
%%
It follows that if $h \colon \GB \epi \GA$ then $\GA \cong \GB/F_h$.
Now we specialize $\GA$ to \textbf{2}. Then if $h \colon \GB \pf
\mathbf{2}$, we have a filter $h^{-1}(1)$. It is clear that
this must be an ultrafilter. Conversely, given an ultrafilter
$U$, $\GB/U \cong \mathbf{2}$. We state without proof the following
%%%
\index{finite intersection property}%%
%%%
theorem. A set $X \subseteq B$ has the \textbf{finite intersection
property} if for every finite $S \subseteq X$ we have 
$\bigcap S \neq 0$.
%%
\begin{thm}
\label{thm:fep}
For every subset of $B$ with the finite intersection property
there exists an ultrafilter containing it.
\end{thm}
%%
Now put $\wht{x} :=  \{h \in \pt(\GB) : h(x) = 1\}$. It is verified that
%%
\begin{align}
\notag
\wht{-x} & = - \wht{x} \\
\wht{x \cap y} & = \wht{x} \cap \wht{y} \\
\notag
\wht{x \cup y} & = \wht{x} \cup \wht{y}
\end{align}
%%
To see the first, assume $h \in \wht{-x}$. Then $h(-x) = 1$, from
which $h(x) = 0$, and so $h \not\in  \wht{x}$, that is to say
$h \in - \wht{x}$. Conversely, if $h \in - \wht{x}$ then
$h(x) \neq 1$, whence $h(-x) = 1$, showing $h \in \wht{-x}$.
Second, $h \in \wht{x \cap y}$ implies $h(x \cap y) = 1$,
so $h(x) = 1$ and $h(y) = 1$, giving $h \in \wht{x}$ as
well as $h \in \wht{y}$.  Conversely, if the latter holds then
$h(x \cap y) = 1$ and so $h \in \wht{x \cap y}$. Similarly with
$\cup$.
%%
\begin{thm}
\label{thm:var}
The map $x \mapsto \wht{x}$ is an injective homomorphism from $\GB$
into the algebra $\GP(\pt(\GB))$. Consequently,
every boolean algebra is isomorphic to a field of sets.
\end{thm}
%%
\proofbeg
It remains to see that the map is injective. To that end,
let $x$ and $y$ be two different elements. We claim that there
is an $h \colon \GB \pf \mathbf{2}$ such that $h(x) \neq h(y)$. For
we either have $x \nleq y$, in which case $x \cap (-y) > 0$;
or we have $y \nleq x$, in which case $y \cap -x > 0$. Assume
(without loss of generality) the first. There is an ultrafilter 
$U$ containing the set $\{x \cap (-y)\}$, by Theorem~\ref{thm:fep}.
Obviously, $x \in U$ but $y \not\in U$. Then $h_U$ is the desired
point.
\proofend

We point out that this means that every boolean algebra is a 
subalgebra of a direct product of \textbf{2}. The variety of 
boolean algebras is therefore generated by \textbf{2}.
The original representation theorem for finite boolean algebras
can be extended in the following way (this is the route that
Keenan and Faltz take). A boolean algebra $\GB$ is called 
\textbf{complete} 
%%%
\index{boolean algebra!complete}%%%
%%%
if any set has a least upper bound and a greatest lower bound.
%%%
\begin{thm}
Let $\GB$ be a complete atomic boolean algebra. Then $\GB \cong
\GP(\At(\GB))$.
\end{thm}
%%
It should be borne in mind that within boolean semantics (say,
in the spirit of Keenan and Faltz) 
%%%
\index{Keenan, Edward L.}%%%
\index{Faltz, Leonard L.}%%%
%%%
the meaning of a particular
linguistic item is a member of a boolean algebra, but it may at
the same time be a function from some boolean algebra to another.
For example, the denotations of adjectives form a boolean 
algebra, but they may also be seen as functions from the algebra 
of common noun denotations (type $e \pf t$) to itself. These maps 
are, however, in general not homomorphisms. The meaning of a 
particular adjective, say {\tt tall}, can in principle be any such 
function. However, some adjectives behave better than others.  
Various properties of such functions can be considered.
%%%
\begin{defn}
%%%%
\index{monotonicity}%%
\index{antitonicity}%%
\index{restrictiveness}%%
\index{intersectiveness}%%
%%%%
Let $\GB$ be a boolean algebra and $f \colon B \pf B$.
$f$ is called \textbf{monotone} iff for all $x, y \in B$:
if $x \leq y$ then $f(x) \leq f(y)$. $f$ is called \textbf{antitone}
if for all $x, y \in B$: if $x \leq y$ then $f(x) \geq f(y)$.
$f$ is called \textbf{restricting} iff for each
$x \in B$ $f(x) \leq x$. $f$ is called \textbf{intersecting} iff
for each $x \in B$: $f(x) = x \cap f(1)$.
\end{defn}
%%
Adjectives that denote intersecting functions are often also
called \textbf{intersective}. An example is {\tt white}. A white car
is something that is both white and a car. Hence we find that
$\mathsf{white}'$ is intersecting. Intersecting functions are
restricting but not necessarily conversely. The adjective {\tt
tall} denotes a restricting function (and is therefore also called
\textbf{restricting}). A tall student is certainly a student. Yet, a
tall student is not necessarily also tall. The problem is that
tallness varies with the property that is in question. (We may
analyze it, say, as: belongs to the 10 \% of the longest students.
Then it becomes clear that it has this property.) Suppose that
students of sports are particularly tall. Then a tall student of
sports will automatically qualify as a tall student, but a tall
student may not be a tall student of sports. On the other hand, if
students of sports are particularly short, then a tall student
will be a tall student of sports, but the converse need not hold.
There are also adjectives that have none of these properties (for 
example, {\tt supposed} or {\tt alleged}). We will return to 
sentential modifiers in the next section.

We conclude the section with a few remarks on the connection
with theories and filters. Let $\Omega$ be the signature of
boolean logic: the 0--ary symbols $\top$, $\bot$, the unary
$\nicht$ and the binary $\oder$ and $\und$. Then we can define
boolean algebras by means of equations, as we have done with
Definition~\ref{defn:ba}. For reference, we call the set of
equations $\mathsf{BEq}$. Or we may actually define a consequence
relation, for example by means of a Hilbert--calculus.
Table~\ref{tab:proposition} gives a complete set of axioms,
which together with the rule MP axiomatize boolean logic.
%%
\begin{table}
\caption{The Axioms of Propositional Logic}
\label{tab:proposition}
$$\begin{array}{l@{\quad}l}
\mbox{\rm (a0)} & p_0 \pf (p_1 \pf p_0) \\
\mbox{\rm (a1)} & (p_0 \pf (p_1 \pf p_2)) \pf ((p_0 \pf p_1) \pf
    (p_0 \pf p_2)) \\
\mbox{\rm (a2)} &
    ((p_0 \pf p_1) \pf p_0) \pf p_0 \\
\mbox{\rm (a3)} &
    \bot \pf p_0 \\
\mbox{\rm (a4)} &
    \nicht p_0 \pf (p_0 \pf \bot) \\
\mbox{\rm (a5)} &
    (p_0 \pf \bot) \pf \nicht p_0 \\
\mbox{\rm (a6)} & \top \\
\mbox{\rm (a7)} &
    p_0 \pf (p_1 \pf (p_0 \und p_1)) \\
\mbox{\rm (a8)} &
    (p_0 \und p_1) \pf p_0 \\
\mbox{\rm (a9)} &
    (p_0 \und p_1) \pf p_1 \\
\mbox{\rm (a10)} &
    p_0 \pf (p_0 \oder p_1) \\
\mbox{\rm (a11)} &
    p_1 \pf (p_0 \oder p_1) \\
\mbox{\rm (a12)} &
    ((p_0 \oder p_1) \pf p_2) \pf ((p_0 \pf p_2) \und
    (p_1 \pf p_2))
\end{array}$$
\end{table}
%%
Call this calculus $\mathsf{PC}$.  We have to bring the equational
calculus and the deductive calculus into correspondence. We have
a calculus of equations (see Section~\ref{kap1}.\ref{kap1-1}), which tells
us what equations follow from what other equations. Write
$\varphi \dpf \chi$ in place of $(\varphi \pf \chi) \und
(\chi \pf \varphi)$.
%%%
\index{$\mathsf{PC}$, $\vdash^{\mathsf{PC}}$}%%% 
%%
\begin{thm}
\label{thm:eqtovdash}
The following are equivalent.
\begin{dingautolist}{192}
\item
$\vdash^{\mathsf{PC}} \varphi \dpf \chi$.
\item
For every boolean algebra $\GA$:
$\GA \vDash \varphi = \chi$.
\item
$\mathsf{BEq} \vdash \varphi \doteq \chi$.
\end{dingautolist}
\end{thm}
%%
The proof is lengthy, but routine. \ding{193} and \ding{194} are 
equivalent by the fact that an algebra is a boolean algebra iff 
it satisfies $\mathsf{BEq}$. So, \ding{192} $\Dpf$ \ding{194} needs 
proof. 
%The interested reader is referred to \cite{kracht:tools} 
%for a proof of this equivalence. 
It rests on the following 
%%%
\begin{lem}
(a) $\vdash^{\mathsf{PC}} \varphi$ iff 
$\vdash^{\mathsf{PC}} \top \dpf \varphi$. 
\\
(b) $\mathsf{BEq} \vdash \varphi \doteq \chi$ iff $\mathsf{BEq} \vdash 
\top \doteq \varphi \dpf \chi$.
\end{lem}
%%%
\proofbeg
(a) Suppose that $\vdash^{\mathsf{PC}} \varphi$. Since 
$\vdash^{\mathsf{PC}} \varphi \pf (\top \pf \varphi)$ we get 
$\vdash^{\mathsf{PC}} \top \pf \varphi$. Similarly, from 
$\vdash^{\mathsf{PC}} \top$ we get $\vdash^{\mathsf{PC}} 
\varphi \pf \top$. Conversely, if $\vdash^{\mathsf{PC}} \top 
\dpf \varphi$, then with (a8) we get $\vdash^{\mathsf{PC}} 
\top \pf \varphi$ and with (a6) and MP, $\vdash^{\mathsf{PC}} 
\varphi$. (b) We can take advantage of our results on BAs here. 
Put $a \triangleup b := (-a \cup b) \cap (a \cup -b)$. The claim 
boils down to $a \triangleup b = 1$ iff $a = b$. Now, if $a \triangleup 
b = 1$, then $-a \cup b = 1$, from which $a \leq b$, and also 
$a \cup -b = 1$, from which $b \leq a$. Together this gives $a = b$. 
Conversely, if $a = b$ then $-a \cup b = -b \cup b = 1$ and 
$a \cup -b = a \cup -a = 1$, showing $a\triangleup  b = 1$. 
\proofend

The next thing to show is that if $\mathsf{BEq} \vdash 
\top \doteq \varphi \pf \chi; \top \doteq \varphi$ then also 
$\mathsf{BEq} \vdash \top \doteq \chi$. Finally, for all 
$\varphi$ of the form (a1) -- (a12), $\mathsf{BEq} \vdash 
\top \doteq \varphi$. This will show that $\vdash^{\mathsf{PC}} 
\varphi$ implies $\mathsf{BEq} \vdash \top \doteq \varphi$. 
\ding{192} is an immediate consequence. For the converse direction, 
first we establish that for all basic equations $\varphi \doteq 
\chi$ of $\mathsf{BEq}$ we have $\vdash^{\mathsf{PC}} \varphi 
\dpf \chi$. This is routine. Closure under substitution is 
guaranteed for theorems. So we need to show that this is 
preserved by the inference rules of Proposition~\ref{prop:eqcalc}, 
that is: 
%%%
\begin{subequations}
\begin{align}
& \vdash^{\mathsf{PC}} \varphi \dpf \varphi \\
\varphi \dpf \chi & \vdash^{\mathsf{PC}} \chi \dpf \varphi \\
\varphi \dpf \chi; \chi \dpf \psi & \vdash^{\mathsf{PC}} 
\varphi \dpf \psi \\
%\varphi \dpf \chi & \vdash^{\mathsf{PC}} \varphi^{\sigma} \dpf 
%	\chi^{\sigma} \\
\{\varphi_i \dpf \chi_i : i < \Omega(f)\} & 
	\vdash^{\mathsf{PC}} f(\vec{\varphi}) \dpf f(\vec{\chi}) 
\end{align}
\end{subequations}
%%% 
In the last line, $f$ is one of the basic functions. The verification 
is once again routine. We shall now show that the so--defined logic is 
indeed the logic of the two element matrix with designated element 1. 
By DT (which holds in $\mathsf{PC}$), $\varphi \dpf \chi$ iff $\varphi 
\vdash^{\mathsf{PC}} \chi$ and $\chi \vdash^{\mathsf{PC}} \varphi$. 
%%%
\index{$\Theta^{\dpf}$}%%
%%%%
\begin{equation}
\Theta^{\dpf} := \{\auf \varphi, \chi\zu \colon \vdash^{\mathsf{PC}}
\varphi \dpf \chi\}
\end{equation}
%%%
$\Theta^{\dpf}$ is a congruence on the term algebra. What is 
more, it is admissible for every deductively closed set. For if 
$\Sigma$ is deductively closed and $\varphi \in \Sigma$, then 
also $\chi \in \Sigma$ for every $\chi \, \Theta^{\dpf}\, \varphi$, 
by Modus Ponens.
%%%
\begin{lem}
$\goth{Tm}_{\Omega}(V)/\Theta^{\dpf}$ is a boolean algebra. Moreover, if
$\Sigma$ is a deductively closed set in $\goth{Tm}_{\Omega}(V)$
then $\Sigma/\Theta^{\dpf}$ is a filter on 
$\goth{Tm}_{\Omega}(V)/\Theta^{\dpf}$. If $\Sigma$ is maximally 
consistent, $\Sigma/\Theta^{\dpf}$ is an
ultrafilter. Conversely, if $F$ is a filter on 
$\goth{Tm}_{\Omega}(V)/\Theta^{\dpf}$, then $h^{-1}_{\Theta^{\dpf}}[F]$
is a deductively closed set. If $F$ is an ultrafilter, this set is a
maximally consistent set of formulae.
\end{lem}
%%%
Thus, $\vdash^{\mathsf{PC}}$ is the intersection of all
$\vDash_{\auf \GA, F\zu}$, where $\GA$ is a boolean algebra and
$F$ a filter. Now, instead of deductively closed sets we can also
take maximal (consistent) deductively closed sets. Their image
under the canonical map is an ultrafilter. However, the
equivalence $\Theta_U := \{\auf x, y\zu : x \dpf y \in U\}$ is a 
congruence, and it is admissible for $U$. Thus, we can once again 
factor it out and obtain the following completeness theorem.
%%
\begin{thm}
\label{thm:2comp} 
$\vdash^{\mathsf{PC}}\; = \; \vDash_{\auf \boldsymbol{2}, \{1\}\zu}$.
\end{thm}
%%
This says that we have indeed axiomatized the logic of the
2--valued algebra. What is more, equations can be seen as
statements of equivalence and conversely. We can draw from
this characterization a useful consequence. Call a propositional
logic \textbf{inconsistent} if every formula is a theorem.
%%%
\index{propositional logic!inconsistent}
%%%
\begin{cor}
$\mathsf{PC}$ is maximally complete. That is to say, if an
axiom or rule $\rho$ is not derivable in $\mathsf{PC}$,
$\mathsf{PC} + \rho$ is inconsistent.
\end{cor}
%%%
\proofbeg
Let $\rho = \auf \Delta, \varphi\zu$ be a rule that is not derivable
in $\mathsf{PC}$. Then by Theorem~\ref{thm:2comp} there is a valuation
$\beta$ which makes every formula of $\Delta$ true but $\varphi$ false.
Define the following substitution: $\sigma(p) := \top$ if
$\beta(p) = 1$, and $\sigma(p) := \bot$ otherwise. Then for
every $\chi \in \Delta$, $\sigma(\chi) \dpf \top$, while
$\sigma(\varphi) \dpf \bot$. Hence, as $\mathsf{PC}$ derives
$\sigma(\chi)$ for every $\chi \in \Delta$, it also derives
$\sigma(\varphi)$, and so $\bot$. On the other hand,
in $\mathsf{PC}$, everything follows from $\bot$. Thus,
$\mathsf{PC} + \rho$ is inconsistent.
\proofend

{\it Notes on this section.} The earliest sources of propositional
logic are the writing of the Stoa, notably by Chrysippos. 
%%%
\index{Chrysippos}%%
%%%
Stoic logic was couched in terms
of inference rules. The first to introduce equations and a calculus
of equations was Leibniz. The characterization of $\leq$ in terms of
union (or intersection) is explicitly mentioned by him. Leibniz
only left incomplete notes. Later, de Morgan, 
%%%
\index{de Morgan, Augustus}%%%
\index{Boole, George}%%%
\index{Frege, Gottlob}%%%
%%%
Boole and Frege have
completed the axiomatization of what is now known as Boolean logic.
%%
\vplatz
\exercise
Show that $x \leq y$ iff $x \cap y = x$.
%%
\vplatz
\exercise
For a lattice $\GL = \auf L, \cap, \cup\zu$ define $\GL^d
:= \auf L, \cup, \cap\zu$. Show that this is lattice as well. 
Obviously, $\GL^{dd} = \GL$. $\GL^d$ is called the 
%%%
\index{lattice!dual}%%
\index{$t^d$}%%
%%%%
\textbf{dual lattice} of $\GL$. The dual of a lattice 
term $t^d$ is defined as follows. $x^d := x$ if $x$ is a variable, 
$(t \cup t')^d := t^d \cap {t'}^d$, $(t \cap t')^d := t^d \cup {t'}^d$.
Evidently, $\GL \vDash s \doteq t$ iff $\GL^d \vDash s^d \doteq t^d$.
Deduce that $s \doteq t$ holds in every lattice iff 
$s^d \doteq t^d$ holds in every lattice. 
%%
\vplatz
\exercise
\label{ex:dual}
(Continuing the previous exercise.) For a boolean term define
additionally $0^d := 1$, $1^d := 0$, $(-t)^d := - t^d$ and
$\GB^d := \auf B, 1, 0, \cup, \cap, -\zu$ for
$\GB = \auf B, 0, 1, \cap, \cup, -\zu$. Show that $\GB^d \cong
\GB$. This implies that $\GB \vDash s \doteq t$ iff
$\GB \vDash s^d \doteq t^d$.
%%
\vplatz
\exercise
Prove Lemma~\ref{lem:prime}.
%%%
\vplatz
\exercise
Let $\leq$ be a partial ordering on $L$ with finite lubs and glbs. 
Define $x \cup y := \lub \{x,y\}$, and $x \cap y := \glb \{x,y\}$. 
Show that $\auf L, \cap, \cup\zu$ is a lattice.
%%
\vplatz
\exercise
Let $\BZ$ be the set of entire numbers. For $i,j \in \omega$ and
$j < 2^i$ let $R_{i,j} := \{m\cdot 2^i + j : m \in \BZ\}$. Let
$H$ be the set of subsets of $\BZ$ generated by all {\it finite\/}
unions of sets of the form $R_{i,j}$. Show that $H$ forms a
field of sets (hence a boolean algebra). Show that it has no
atoms.

 \section{Intensionality}
\label{kap6-3}
%
%
%
Leibniz' Principle 
%%%
\index{Leibniz' Principle}%%%
%%%
has given rise to a number of problems in
formal semantics. One such problem is its alleged failure
with respect to intensional contexts. This is what we shall
discuss now. The following context does not admit any substitution
of $A$ by a $B$ different from $A$ without changing the truth value
of the entire sentence.
%%
\begin{equation}
\label{eq:431} 
\mbox{\tt The expression `$A$' is the same expression as `$B$'.} 
\end{equation}
%%
Obviously, if such sentences were used to decide about synonymy, no
expression is synonymous with any other. However, the feeling with
these types of sentences is that the expressions do not enter with
their proper meaning here; one says, the expressions $A$ and $B$ are
%%%%
\index{use}\index{mention}%%
%%%
not \textbf{used} in \eqref{eq:431} they are only \textbf{mentioned}.
This need not cause problems for our sign based approach. We might
for example say that the occurrences of $A$ where $A$ is used
are occurrences with a different category than those where $A$ is
mentioned. If we do not assume this we must exclude those sentences
in which the occurrences of $A$ or $B$ are only mentioned, not
used. However, in that case we need a criterion for deciding when
an expression is used and when it is mentioned. The picture is as
follows. Let $S(x)$ be shorthand for a sentence $S$ missing a
constituent $x$. We call them \textbf{contexts}.
%%%
\index{context}%%
%%%
Leibniz' Principle says that $A$ and $B$ have identical
%%%
\index{Leibniz' Principle}%%%
%%%
%%%
\index{$A \equiv B$}%%
%%%
meaning, in symbols $A \equiv B$, iff $S(A) \dpf
S(B)$ is true for all $S(x)$. Now, let $\Sigma$ be the set of all
contexts, and $\Pi$ the set of all contexts where the missing
expression is used, not mentioned. Then we end up with two kinds
of identity:
%%
\begin{align}
A \equiv_{\Sigma} B & :\Dpf (\forall S(x) \in \Sigma)(S(A) \dpf S(B)) \\
A \equiv_{\Pi} B & :\Dpf (\forall S(x) \in \Pi)(S(A) \dpf S(B))
\end{align}
%%
Obviously, $\equiv_{\Sigma}\; \subseteq\; \equiv_{\Pi}$.
Generalizing this, we get a Galois correspondence here between
certain sets of contexts and equivalence relations on expressions.
%%%%
\index{context!hyperintensional}%%
%%%%
Contexts outside of $\Pi$ are called \textbf{hyperintensional}. In
our view, \eqref{eq:431} does not contain occurrences of the
language signs for $A$ and $B$ but only occurrences of strings.
Strings denote themselves. So, what we have inserted are not the
same signs as the signs of the language, and this means that
Leibniz' Principle is without force in example \eqref{eq:431}
%%%
\index{Leibniz' Principle}%%%
%%%
with respect to the signs. However, if put into the context {\tt
the meaning of `}$\uli{\mbox{\quad}}${\tt '}, we get the actual
meaning of $A$ that the language gives to it. Thus, the following
is once again transparent for the meanings of $A$ and $B$:
%%
\begin{align}
\label{eq:432} 
    & \mbox{\tt The expression `$A$' has the same meaning as the} 
\\\notag
    & \quad \mbox{\tt expression `$B$'.} 
\end{align}
%%
A hyperintensional context is
%%
\begin{align}
\label{eq:433} & \mbox{\tt John thinks that palimpsests are leaflets.} 
\end{align}
%%
What John thinks here is that the expression {\tt palimpsest}
denotes a special kind of leaflet, where in fact it denotes a
kind of manuscript. Although this is a less direct case of mentioning
an expression, it still is the case that the sign with exponent
{\tt palimpsest} is not an occurrence of the genuine English
language sign, because it is used with a different meaning.
The meaning of that sign is once again the exponent (string)
itself.

There are other problematic instances of Leibniz' Principle, for
%%%
\index{Leibniz' Principle}%%%
%%%
example the so--called {\it intensional\/} contexts. Consider the
following sentences.
%%
\begin{align}
\label{eq:434} & \mbox{\tt The morning star is the evening star.} \\
\label{eq:435} & \mbox{\tt John believes that the morning star is the}
	 \\\notag
    & \quad \mbox{\tt morning star.} \\
\label{eq:436}  & \mbox{\tt John believes that the morning star is the} 
	\\\notag
    & \quad \mbox{\tt evening star.} \\
\label{eq:437}  & \mbox{\tt The square root of 2 is less than 3/2.} \\
\label{eq:438}  & \mbox{\tt John believes that the square root of
    2 is} \\\notag
        & \quad \mbox{\tt less than 3/2.}
\end{align}
%%
It is known that \eqref{eq:434} is true. However, it is quite
conceivable that \eqref{eq:435} may be true and \eqref{eq:436}
false. By Leibniz' Principle, we must assume that {\tt the morning
%%%
\index{Leibniz' Principle}%%%
%%%
star} and {\tt the evening star} have different meaning. However,
as Frege points out, in {\it this\/} world they refer to the same
thing (the planet Venus), so they are not different. Frege therefore
%%%
\index{sense}\index{reference}%%
%%%
distinguishes \textbf{reference} ({\it Bedeutung\/}) from \textbf{sense}
({\it Sinn\/}). In \eqref{eq:434} the expressions enter with
their reference, and this is why the sentence is true. In
\eqref{eq:435} and \eqref{eq:436}, however, they do not enter with 
their reference,
otherwise John holds an inconsistent belief. Rather, they enter
with their senses, and the senses are different. Thus, we have
seen that expressions that are used (not mentioned) in a sentence
may either enter with their reference or with their sense. The
question is however the same as before: how do we know when an
expression enters with its sense rather than its reference?
The general feeling is that one need not be worried by that
question. Once the sense of an expression is given, we know what
its reference is. We may think of the sense as an algorithm that
gives us the reference on need. (This analogy has actually been
pushed by Yannis Moschovakis, who thinks that sense actually
{\it is\/} an algorithm (see \cite{moschovakis:sense}). However,
this requires great care in defining the notion of an algorithm,
otherwise it is too fine grained to be useful. Moschovakis shows 
that equality of meaning is decidable, while equality of denotation 
is not.) Contexts that do not vary with the sense only with the 
reference of their subexpression are called \textbf{extensional}.  
%%%%
\index{context!extensional}%%%
%%%
Nonextensional contexts are \textbf{intensional}.
%%%
\index{context!intensional}%%
%%%
Just how fine grained intensional contexts are is a difficult matter.
For example, it is not inconceivable that \eqref{eq:437} is true but 
\eqref{eq:438} is false. Since $\sqrt{2} < 1.5$ we expect that it 
cannot be otherwise, and that one cannot even believe otherwise. 
This holds, for example, under the modal analysis of belief by 
Hintikka~\shortcite{hintikka:knowledge}. Essentially, this is what 
we shall assume here, too. The problem of intensionality with 
respect to Leibniz' Principle disappears once we realize that it 
%%%
\index{Leibniz' Principle}%%%
%%%
speaks of identity in meaning, not just identity in
denotation. These are totally different things, as Frege rightly
observed. Of course, we still have to show how meaning and
denotation work together, but there is no problem with Leibniz'
Principle.

Intensionality has been a very important area of research in
formal semantics, partly because Montague 
%%%
\index{Montague, Richard}%%
\index{Carnap, Rudolf}%%%
%%%
already formulated
an intensional system. The influence of Carnap is clearly
visible here. It will turn out that equating intensionality
with normal modal operators is not always helpful. Nevertheless,
the study of intensionality has helped enormously in understanding
the process of algebraization.

%%%
\index{$\qu, \wD$}%%
%%%
Let $A := \{\mbox{\tt (}, \mbox{\tt )}, \mbox{\tt p}, \mbox{\tt 0},
\mbox{\tt 1}, \mbox{\mtt\symbol{4}}, \mbox{\mtt\symbol{5}}, 
\boldsymbol{\qu}\}$, where the boolean symbols are used as before 
and $\boldsymbol{\qu}$ is a unary symbol, which is written before 
its argument. We form expressions in the usual way, using brackets. 
The language we obtain shall be called $L_M$. The abbreviations 
$\varphi \pf \chi$ and $\varphi\dpf \chi$ as well as typical 
shorthands (omission of brackets) are used without warning.
Notice that we have a propositional language, so that the notions
of substitution, consequence relation and so on can be taken over
straightforwardly from Section~\ref{kap6}.\ref{kap:feasibility}.
%%%
\begin{defn}
%%%
\index{modal logic}%%
\index{modal logic!classical}%%
\index{modal logic!monotone}%%
\index{modal logic!normal}%%
%%%
A \textbf{modal logic} is a subset $L$ of $L_M$ which contains
all boolean tautologies and which is closed under substitution and
Modus Ponens. $L$ is called \textbf{classical} if from $\varphi
\dpf \chi \in L$ follows that $\qu \varphi \dpf\qu \chi \in
L$, \textbf{monotone} if from $\varphi \pf \chi\in L$
follows $\qu \varphi \pf \qu \chi \in L$.  $L$ is 
\textbf{normal} if for all $\varphi, \chi \in L_M$ (a) 
$\qu (\varphi \pf \chi) \pf (\qu \varphi \pf\qu\chi) \in L$, 
(b) if $\varphi \in L$ then $\qu \varphi \in L$.
\end{defn}
%%%
The smallest normal modal logic is denoted by $\mathsf{K}$, after Saul
Kripke. 
%%%%
\index{modal logic!quasi--normal}%%
%%%%
A \textbf{quasi--normal} modal logic is a modal logic that
contains $\mathsf{K}$. One also defines
%%%
\begin{equation}
\wD \varphi := \mbox{\mtt \symbol{5}($\boldsymbol{\qu}$(\symbol{5}%
$\varphi$))}
\end{equation}
%%%
\index{operator!necessity}%%
\index{operator!possibility}%%
\index{operator!dual}%%%
%%%%
and calls this the \textbf{dual operator} (see Exercise~\ref{ex:dual}). 
$\qu$ is usually called a \textbf{necessity operator}, $\wD$ a 
\textbf{possibility operator}.
%%%%
\index{$\oplus$}%%
%%%%
If $\varphi$ is an axiom and $L$ a (normal) modal logic,
then $L + \varphi$ ($L\oplus \varphi$) is the smallest
(normal) logic containing $L \cup \{\varphi\}$. Analogously
the notation $L + \Gamma$, $L \oplus \Gamma$ for a set
$\Gamma$ is defined.
%%%
\begin{defn}
%%%
\index{$\vdash_{L}, \Vdash_{L}$}%%
%%%%
Let $L$ be a modal logic. Then $\vdash_{L}$ is the
following consequence relation. $\Delta \vdash_{L} \varphi$
iff $\varphi$ can be deduced from $\Delta \cup L$
using (mp) only. $\Vdash_{L}$ is the consequence
relation generated by the axioms of $L$, the rule (mp) and
the rule (mn): $\auf \{\mbox{\tt p}\}, \mbox{\tt ($\boldsymbol{\qu}$p)}\zu$. 
$\vdash_{L}$ is called the \textbf{local consequence relation}, 
%%%%
\index{consequence relation!local}
%%%
$\Vdash_{L}$ the \textbf{global consequence relation} associated 
with $L$.
%%%
\index{consequence relation!global}
%%%
\end{defn}
%%%
It is left to the reader to verify that this indeed defines a
consequence relation. We remark here that for $\vdash_{\mathsf{K}}$
the rule (mn) is by definition admissible. However, it is not 
derivable (see the exercises) while in $\Vdash_{\mathsf{K}}$
it is, by definition. Before we develop the algebraic approach 
further, we shall restrict our attention to normal logics. For 
these logics, a geometric (or `model theoretic') semantics has 
been given.
%%%
\begin{defn}
%%%
\index{Kripke--frame}%%
\index{Kripke--frame!generalized}%%
\index{set!\faul\  of worlds}%%
\index{accessibility relation}%%
%%%
A \textbf{Kripke--frame} is a pair $\auf F, \lhd\zu$ where $F$ is a
set, the \textbf{set of worlds}, and $\lhd \subseteq F^2$, the
\textbf{accessibility relation}. A \textbf{generalized Kripke--frame}
is a triple $\auf F, \lhd, \BF\zu$ where $\auf F, \lhd\zu$ is a
Kripke--frame and $\BF \subseteq \wp(F)$ a field of sets closed
%%%
\index{$\sq$}%%
%%%
under the operation $\sq$ on $\wp(F)$ defined as follows:
%%
\begin{equation}
\sq A := \{x : \text{ for all }y: \text{ if }x \lhd y \text{ then
    }y \in A\}
\end{equation}
\end{defn}
%%%
%So we have found a representation for the algebra. A matrix however
%also consists of a deductively closed set. We have already said that
%we may take this set to be an ultrafilter. This ultrafilter is now
%nothing but a member of the set $U(\GM)$. Under this representation,
%a homorphism $h \colon \goth{Tm}_{\Omega}(V)$ to $\GM$ defines a
%homomorphism from $\goth{Tm}_{\Omega}(V)$ to the modal algebra
%defined over $U(\GM)$. However, the standard terminology is
%somewhat different.
%%%
\index{valuation}%%
%%%
Call a \textbf{valuation} into a  general Kripke--frame
$\GF = \auf F, \lhd, \BF\zu$ a function $\beta \colon V \pf \BF$. 
%%
\begin{equation}
\begin{split}
\auf \GF, \beta, x\zu \vDash p & \Dpf x \in \beta(p)\qquad (p \in V) \\
\auf \GF, \beta, x\zu \vDash \mbox{\mtt (\symbol{5}$\varphi$)}
    & \Dpf
\auf \GF, \beta, x\zu \nvDash \varphi \\
\auf \GF, \beta, x\zu \vDash \mbox{\mtt ($\varphi$\symbol{4}$\chi$)}
    & \Dpf
\auf \GF, \beta, x\zu \vDash \varphi;\chi \\
\auf \GF, \beta, x\zu \vDash \mbox{\mtt ($\boldsymbol{\qu}\varphi$)}
    & \Dpf
\mbox{for all}\;y\colon\mbox{if }x\lhd y\;\mbox{then}\;
\auf \GF, \beta, y\zu \vDash \varphi
\end{split}
\end{equation}
%%%
\index{frame consequence!local}%%
%%%
(One often writes $x \vDash \varphi$, suppressing $\GF$ and $\beta$.)
Furthermore, the \textbf{local frame consequence} is defined as follows.
$\Delta \vDash_{\GF} \varphi$ if for every $\beta$ and $x$:
if $\auf \GF, \beta, x\zu \vDash \delta$ for every $\delta \in
\Delta$ then $\auf \GF, \beta, x\zu \vDash \varphi$. This is
a consequence relation. Moreover, the axioms and rules of $\mathsf{PC}$
are valid. Furthermore,
%%
\begin{equation}
\vDash_{\GF} \qu(\varphi\pf\chi) \pf (\qu \varphi \pf \qu\chi)
\end{equation}
%%
For if $x \vDash \qu(\varphi\pf\chi); \qu\varphi$ and $x \lhd y$
then $y\vDash \varphi \pf\chi;\varphi$, from which $y \vDash \chi$. 
As $y$ was arbitrary, $x \vDash \qu\chi$. Finally, suppose that 
$\GF \vDash \varphi$. Then $\GF \vDash \qu\varphi$.
For choose $x$ and $\beta$. Then for all $y$ such that
$x \lhd y$: $\auf \GF, \beta, y\zu\vDash \varphi$, by assumption. 
Hence $\auf \GF, \beta, x\zu\vDash \qu\varphi$. Since $x$ and
$\beta$ were arbitrarily chosen, the conclusion follows.
Define $\Delta \vDash^g_{\GF} \varphi$ if for all $\beta$:
if $\oli{\beta}(\delta) = F$ for all $\delta \in \Delta$,
then also $\oli{\beta}(\varphi) = F$. This is the
%%%
\index{frame consequence!global}%%
%%%
\textbf{global frame consequence} determined by $\GF$.
%%%
For a class of frames $\CK$ we put
%%
\begin{equation}
\vDash_{\CK}\; := \bigcap \auf \vDash_{\GF}\; : \; \GF \in \CK\zu
\end{equation}
%%%
Analogously,
$\vDash_{\CK}^g$ is the intersection of all $\vDash^g_{\GF}$,
$\GF \in \CK$.
%%
\begin{thm}
For every class $\CK$ of frames there is a modal logic $L$
such that $\vDash_{\CK}\; = \;\vdash_{L}$. Moreover,
$\vDash_{\CK}^g \; = \; \Vdash_{L}$.
\end{thm}
%%
\proofbeg
We put $L := \{\varphi : \varnothing \vDash_{\CK} \varphi\}$.
We noticed that this is a normal modal logic if $\CK$ is one
membered. It is easy to see that this therefore holds for
all classes of frames. Clearly, since both $\vdash_{L}$
and $\vDash_{\CK}$ have a deduction theorem, they are equal if
they have the same tautologies. This we have just shown.
For the global consequence relation, notice first that
$L = \{\varphi : \varnothing \vDash^g_{\CK}\varphi\}$.
Moreover, $\Vdash_{L}$ is the smallest global consequence
relation containing $\vdash_{L}$, and similarly
$\vDash^g_{\CK}$ the smallest global consequence relation
containing $\vDash_{\CK}$.
\proofend

We shall give some applications of modal logic to the semantics
of natural language. The first is that of (meta)physical necessity.
In uttering \eqref{eq:439} we suggest that \eqref{eq:4310}
obtains whatever the circumstances. Likewise, in uttering
\eqref{eq:4311} we suggest that there are circumstances under which
\eqref{eq:4312} is true.
%%
\begin{align}
\label{eq:439} & \mbox{\tt 2+3 is necessarily greater than 4.} \\
\label{eq:4310} & \mbox{\tt 2+3 is greater than 4.} \\
\label{eq:4311} & \mbox{\tt Caesar might not have defeated Vercingetorix.} \\
\label{eq:4312} & \mbox{\tt Caesar has not defeated Vercingetorix.}
\end{align}
%%
The analysis is as follows. We consider {\tt necessarily} as an
operator on sentences. Although it appears here in postverbal
position, it may be rephrased by {\tt it is necessary that}, which
can be iterated any number of times. The same can be done with
{\tt might}, which can be rephrased as {\tt it is possible that}
and turns out to be the dual of the first. We disregard questions
of form here and represent sentential operators simply as $\qu$
and $\wD$, prefixed to the sentence in question. $\qu$ is a modal
operator, and it is normal. For example, if $A$ and $B$ are both
necessary, then so is $A \und B$, and conversely. Second, if $A$
is logically true, then $A$ is necessary. Necessity has been modelled
following to Carnap by frames of the form $\auf W, W\times W\zu$.
Metaphysically possible worlds should be possible no matter
what is the case (that is, no matter which world we are in). It 
turns out that the interpretation above yields a particular logic,
called $\mathsf{S5}$.
%%%
\index{$\mathsf{S5}$}%%
%%%
%%
\begin{equation}
\mathsf{S5} :=
\mathsf{K} \oplus \{p \pf \wD p, \wD p \pf \wD\wD p,
p \pf \qu\wD p\}
\end{equation}
%%
We defer a proof of the fact that this characterizes $\mathsf{S5}$.

%%%
\index{$[K_J]$, $[B_J]$}%%
%%%
Hintikka~\shortcite{hintikka:knowledge} has axiomatized the logic
of knowledge and belief. Write $[K_J]\varphi$ to represent
the proposition `John knows that $\varphi$' and $[B_J]\varphi$
to represent the proposition `John believes that $\varphi$'.
Then, according to Hintikka, both turn out to be normal modal
operators. In particular, we have the following axioms.
%%
\begin{align}
& [B_J](\varphi\pf\chi) && \mbox{\rm logical `omniscience' for belief} \\\notag
& \quad \pf ([B_J]\varphi \pf [B_J]\chi) &&  \\
& [B_J]\varphi \pf [B_J][B_J]\varphi && \mbox{\rm positive introspection 
	for belief} \\
& [K_J](\varphi\pf\chi) && \mbox{\rm logical omniscience} \\\notag
& \pf ([K_J]\varphi \pf [K_J]\chi) &&  \\
& [K_J]\varphi \pf \varphi && \mbox{\rm factuality of knowledge} \\
& [K_J]\varphi \pf [K_J][K_J]\varphi &&
    \mbox{\rm positive introspection} \\
& \nicht [K_J]\varphi \pf [K_J]\nicht [K_J]\varphi &&
    \mbox{\rm negative introspection}
\end{align}
%%
Further, if $\varphi$ is a theorem, so is $[B_J]\varphi$ and
$[K_J]\varphi$. Now, we may either study both operators in
isolation, or put them together in one language, which now has
two modal operators. We trust that the reader can make the
necessary amendments to the above definitions to take care
of any number of operators. We can can then also formulate
properties of the operators in combination. It turns out, namely,
that the following holds.
%%
\begin{equation}
[K_J]\varphi \pf [B_J]\varphi
\end{equation}
%%
\index{$\mathsf{K4}$}%%%
%%%%
The logic of $[B_J]$ is known as $\mathsf{K4} := \mathsf{K}
\oplus \wD\wD p \pf \wD p$ and it is the logic of all transitive
Kripke--frames; $[K_J]$ is once again $\mathsf{S5}$. The validity of 
this set of axioms for the given interpretation is of course open 
to question.

A different interpretation of modal logic is in the area of
{\it time}. Here there is no consensus on how the correct model
structures look like. If one believes in determinism, one may
for example think of time points as lying on the real line
$\auf \BR, <\zu$. Introduce an operator $\qrechts$ by
%%%
\index{$\qrechts$, $\rechts$, $\qlinks$, $\links$}%%
%%
\begin{equation}
\auf \BR, <, \beta, t\zu \vDash \qrechts\, \chi
    :\Dpf \mbox{ for all } t' > t:
    \auf \BR, <, \beta, t'\zu \vDash \chi
\end{equation}
%%
One may read $\qrechts\,\chi$ as {\it it will always be the case
that\/} $\chi$. Likewise, $\rechts\chi$ may be read as
{\it it will at least once be the case that\/} $\chi$.
The logic of $\qrechts$ is
%%
\begin{equation}
\mathsf{Th}\, \auf \BR, <\zu := \{\chi : \mbox{ for all
    }\beta, x:
    \auf \BR, <, \beta, x\zu \vDash \chi\}
\end{equation}
%%
Alternatively, we may define an operator $\qlinks$ by
%%
\begin{equation}
\auf \BR, <, \beta, t\zu \vDash \qlinks\, \chi
    :\Dpf \mbox{ for all } t' < t:
    \auf \BR, <, \beta, t'\zu \vDash \chi
\end{equation}
%%
to be read as {\it it has always been the case that\/} $\chi$.
Finally, $\links\,\chi$ reads {\it it has been the case that\/}
$\chi$. On $\auf \BR, <\zu$, $\qlinks$ has the same logic as
$\qrechts$. We may also study both operators in combination.
What we get is a bimodal logic (which is simply a logic over
a language with two operators, each defining a modal logic in
its own fragment).  Furthermore, $\auf \mathbb{R}, <\zu \vDash
p \pf \qrechts\,\links\, p; p \pf \qlinks\, \rechts \, p$.
The details need not be of much concern here.  Suffice it to say
that the modelling of time with the help of modal logic has received
great attention in philosophy and linguistics. Obviously,
to be able to give a model theory of tenses is an important
task. Already Montague integrated into his theory a treatment
%%%
\index{Montague, Richard}%%%
%%%
of time in combination with necessity (as discussed above).

We shall use the theory of matrices to define a semantics for
these logics. We have seen earlier that one can always choose
matrices of the form $\auf \goth{Tm}_{\Omega}(V), \Delta\zu$,
$\Delta$ deductively closed. Now assume that $L$ is
classical. Then put $\varphi\; \Theta_L^{\dpf}\; \chi$ if $\varphi \dpf
\chi \in L$. This is a congruence relation, and we can form
the factor algebra along that congruence. (Actually, classicality
is exactly the condition that $\dpf$ induces a congruence
relation.) It turns out that this algebra is a boolean algebra and
that $\qu$ is interpreted by a
%%%
\index{$\sq$, $\wD$}%%
%%%
function $\sq$ on that boolean algebra (and $\wD$ by a function
$\sD$).
%%%
\begin{defn}
%%%%
\index{boolean algebra!with operators (BAO)}%%%
%%%
A \textbf{boolean algebra with (unary) operators} (\textbf{BAO}) is an
algebra $\auf A, 0, 1, \cap, \cup, -, \auf \sq_i : i < \kappa\zu\zu$
such that $\sq_i \colon A \pf A$ for all $i < \kappa$.
\end{defn}
%%
If furthermore $L$ is a normal modal logic, $\sq$ turns
out to be a so--called hemimorphism.
%%%
\begin{defn}
%%%
\index{hemimorphism}%%%
\index{multimodal algebra}%%%
%%%%
Let $\GB$ be a boolean algebra and $h \colon B \pf B$ a map. $h$ is called
a \textbf{hemimorphism} if (i) $\sq 1 = 1$ and (2) for all $x, y \in B$:
$\sq (x \cap y) = \sq (x) \cap \sq (y)$. A \textbf{multimodal algebra} is
an algebra $\GM = \auf M, 0, 1, \cap, \cup, -, \auf \sq_i :
i < \kappa\zu\zu$, where $\auf M, 0, 1, \cap, \cup, -\zu$ is
a boolean algebra and $\sq_i$, $i < \kappa$, a hemimorphism on it.
\end{defn}
%%%
We shall remain with the case $\kappa = 1$ for reasons of
simplicity. A hemimorphism is thus not a homomorphism (since it
does not commute with $\cup$). The modal algebras form the
semantics of modal propositional logic. We also have to look at
the deductively closed sets. First, if 
$\varphi\; \Theta_L^{\dpf}\; \chi$ then $\varphi \in %
\Delta$ iff $\chi \in \Delta$. So, we can factor
$\Delta$ by $\Theta_L^{\dpf}$. It turns out that $\Delta$, being closed
under (mp), becomes a filter of the boolean quotient algebra.
Thus, normal modal logics are semantically complete with
respect to matrices $\auf \GM, F\zu$, where $\GM$ is a modal
algebra and $F$ a filter. We can refine this still further to $F$
being an ultrafilter. This is so since if $\Delta \nvdash_{L}
\varphi$ there actually is a maximally consistent set of formulae
that contains $\Delta$ but not $\varphi$, and reduced by $\Theta_L^{\dpf}$
this turns into an ultrafilter. Say that $\GM \vDash \chi$ if
$\auf \GM, U\zu \vDash \chi$ for all ultrafilters $U$ on $\GM$.
Since $x$ is in all ultrafilters iff $x = 1$, we have
$\GM \vDash \chi$ exactly if for all homomorphisms $h$ into $\GM$,
$h(\chi) = 1$. (Equivalently, $\GM \vDash \chi$ iff
$\auf \GM, \{1\}\zu \vDash \chi$.) Notice that $\GM \vDash \varphi
\pf \chi$ if for all $h$: $h(\varphi) \leq h(\chi)$.

Now we shall apply the representation theory of the previous section.
A boolean algebra can be represented by a field of sets, where
the base set is the set of all ultrafilters (alias points) over
the boolean algebra. Now take a modal algebra $\GM$. Underlying it
we find a boolean algebra, which we can represent by a field of
sets. The set of ultrafilters is denoted by $U(\GM)$. Now, for two
ultrafilters $U$, $V$ put $U \lhd V$ iff for all
$\sq x \in U$ we have $x \in V$. Equivalently, $U \lhd V$ iff 
$x \in V$ implies $\sD x \in U$. We end up with a structure
$\auf U(\GM), \lhd, \BS\zu$, where $\lhd$ is a binary relation
over $U(\GM)$ and $\BS \subseteq \wp(U(\GM))$ a field of sets
closed under the operation $A \mapsto \sq A$.

A modal algebra $\GM$ is an $\mathsf{S5}$--algebra if it satisfies
the axioms given above. Let $\GM$ be an $\mathsf{S5}$--algebra and $U, V, W$
ultrafilters. Then (a) $U \lhd U$. For let $x \in U$. Then
$\sD x \in U$ since $\GM \vDash p \pf \wD p$. Hence, $U \lhd U$.
(b) Assume $U \lhd V$ and $V \lhd W$. We show that $U \lhd W$.
Pick $x \in W$. Then $\sD x \in V$ and so $\sD\sD x \in U$.
Since $\GM \vDash \wD \wD p \pf \wD p$, we have $\sD x \in U$.
(c) Assume $U \lhd V$. We show $V \lhd U$. To this end, pick
$x \in U$. Then $\sq \sD x \in U$. Hence $\sD x \in V$, by
definition of $\lhd$. Hence we find that $\lhd$ is an equivalence
relation on $U(\GM)$. More exactly, we have shown the following.
%%%
\begin{prop}
Let $\GM$ be a modal algebra, and $\lhd \subseteq U(\GM)^2$ be
defined as above.
%%
\begin{dingautolist}{192}
\item $\GM \vDash p \pf \wD p$ iff $\lhd$ is reflexive.
\item $\GM \vDash \wD\wD p \pf \wD p$ iff $\lhd$ is
    transitive.
\item $\GM \vDash p \pf \wq\wD p$ iff $\lhd$ is symmetric.
\end{dingautolist}
\end{prop}
%%
The same holds for Kripke--frames. For example,
$\auf F,\lhd\zu \vDash p \pf \wD p$ iff $\lhd$ is
reflexive. Therefore, $\auf U(\GM), \lhd\zu$ already satisfies
all the axioms of $\mathsf{S5}$. Finally, let $\auf F, \lhd\zu$ be a
Kripke--frame, $G \subseteq F$ be a set such that $x \in G$ and
$x \lhd y$ implies $y \in G$.
%%%
\index{subframe!generated}%%
%%%
(Such sets are called \textbf{generated}.) Then the induced frame
$\auf G, \lhd \cap G^2\zu$ is called a \textbf{generated subframe}.
A special case of a generated subset is the set $F \uparrow x$
consisting of all points that can be reached in finitely many
steps from $x$. Write $\GF \uparrow x$ for the generated subframe
induced by $F \uparrow x$. Then a valuation $\beta$ on $\GF$
induces a valuation on $\GF\uparrow x$, which we denote also
by $\beta$.
%%%
\begin{lem}
\label{lem:gensub}
$\auf \GF, \beta, x\zu \vDash \varphi$ iff
$\auf \GF\uparrow x, \beta, x\zu \vDash \varphi$. It follows
that if $\GF \vDash L$ and $\GG$ is a generated subframe
of $\GF$ then also $\GG \vDash L$.
\end{lem}
%%%
A special consequence is the following. Let $\GF_0 :=
\auf F_0, \lhd_0\zu$ and $\GF_1 := \auf F_1, \lhd_1\zu$
be Kripke--frames. Assume that $F_0$ and $F_1$ are disjoint.
%%%
\begin{equation}
\GF_0 \oplus \GF_1 := \auf F_0 \cup F_1, \lhd_0 \cup \lhd_1\zu
\end{equation}
%%%
Then $\GF_0 \oplus \GF_1 \vDash \varphi$ iff $\GF_0 \vDash \varphi$ 
and $\GF_1 \vDash \varphi$. (For if $x \in F_0$ then 
$\auf \GF_0 \oplus \GF_1, \beta, x\zu\vDash \varphi$ iff $\auf \GF_0,
\beta, x\zu \vDash \varphi$, and analogously for $x \in F_1$.)
It follows that a modal logic which is determined by some class of
Kripke--frames is already determined by some class of Kripke--frames 
generated from a single point. This shows the following.
%%%
\begin{thm}
$\mathsf{S5}$ is the logic of all Kripke--frames of the form
$\auf M, M\times M\zu$.
\end{thm}
%%

Now that we have looked at intensionality we shall look at the question
of individuation of meanings. In algebraic logic a considerable amount
of work has been done concerning the semantics of propositional
languages. Notably in Blok and Pigozzi~\shortcite{blokpigozzi:algebraizable} 
%%%
\index{Blok, Wim}%%
\index{Pigozzi, Don}%%
%%%
Leibniz' Principle was made the starting point of a definition of
%%%
\index{Leibniz' Principle}%%%
%%%
algebraizability of logics. We shall exploit this work for our purposes
here.  We start with a propositional language of signature $\Omega$.
Recall the definition of logics, consequence relation and matrix from
Section~\ref{kap6}.\ref{kap6-1}. We distinguish between a {\it theory}, 
{\it (world) knowledge\/} and a {\it meaning postulate}.
%%%
\begin{defn}
%%%
\index{theory}%%%
\index{axiomatization}%%%
%%%%
Let $\vdash$ be a consequence relation. A $\vdash$--\textbf{theory} is
a set $\Delta$ such that $\Delta^{\vdash} = \Delta$. If $T$ is a
set such that $T^{\vdash} = \Delta$, $T$ is called an 
\textbf{axiomatization of} $\Delta$.
\end{defn}
%%
Theories are therefore sets of formulae, and they may contain variables.
For example, $\{\mbox{\mtt p0}, \mbox{\mtt (p1\symbol{25}(\symbol{5}p01))}\}$ 
is a theory. However, in virtue of the fact that
variables are placeholders, it is not appropriate to say that knowledge
is essentially a theory. Rather, for a theory to be knowledge it must
be closed under substitution. Sets of this form shall be called
{\it logics}.
%%%
\begin{defn}
%%%
\index{logic}%%
%%%
Let $\vdash$ be a structural consequence relation. A 
$\vdash$--\textbf{logic} is a $\vdash$--theory closed 
under substitution.
\end{defn}
%%%
Finally, we turn to meaning postulates. Here, it is appropriate not 
to use sets of formulae, but rather equations.
%%%
\begin{defn}
%%%
\index{meaning postulate}%%
%%%%
Let $L$ be a propositional language. A \textbf{meaning postulate}
for $L$ is an equation. Given a set $M$ of meaning postulates,
an equation $s \doteq t$ follows from $M$ if $s \doteq t$ holds
in all algebras satisfying $M$.
\end{defn}
%%%
Thus, the meaning postulates effectively axiomatize the variety of
meaning algebras, and the consequences of a set of equations can
be derived using the calculus of equations of Section~\ref{kap1}.\ref{kap1-1}.
In particular, if $f$ is an $n$--ary operation and $s_i \doteq t_i$
holds in the variety of meaning algebras, so does
$f(\vec{s}) \doteq f(\vec{t})$, and likewise, if $s \doteq t$
holds, then $\sigma(s) \doteq \sigma(t)$ holds for any substitution
$\sigma$. These are natural consequences if we assume that meaning
postulates characterize identity of meaning. We shall give an
instructive example.
%%
\begin{align}
\label{eq:4313} & \mbox{\tt Caesar crossed the Rubicon.} \\
\label{eq:4314} & \mbox{\tt John does not believe that Caesar crossed the}
\\\notag
 & \quad \mbox{\tt Rubicon.} \\
\label{eq:4315} & \mbox{\tt Bachelors are unmarried men.} \\
\label{eq:4316} & \mbox{\tt John does not believe that bachelors are} 
	\\\notag
    & \quad \mbox{\tt unmarried men.}
\end{align}
%%
It is not part of the meanings of the words that Caesar crossed the
Rubicon, so John may safely believe or disbelieve it. However, it is
part of the language that bachelors are unmarried men, so not believing
it means associating different meanings to the words. Thus, if
\eqref{eq:4315} is true and moreover a meaning postulate, 
\eqref{eq:4316} cannot be true.

It is unfortunate having to distinguish postulates that take the
form of a formula from those that take the form of an equation.
Therefore, one has sought to reduce the equational calculus to
the logical calculus and conversely. The notion of equivalential
logic has been studied among other by Janusz Czelakowski 
%%%
\index{Czelakowski, Janusz}%%%
\index{Suszko, Roman}%%%
%%%
and Roman Suszko. The following definition is due to Prucnal and
Wro\'{n}ski \shortcite{prucnalwronski}. 
%%%
\index{Prucnal, T.}%%%
\index{Wro\'nski, Andrzej}%%
%%%
(For a set $\Phi$
of formulae, we write $\Delta \vdash \Phi$ to say that $\Delta
\vdash \varphi$ for all $\varphi \in \Phi$.)
%%
\begin{defn}
%%%%
\index{equivalential term}%%
%%%%
\label{defn:equivalential}
Let $\vdash$ be a consequence relation. We call the set 
$\Delta(p,q) = \{\delta_i(p,q) : i \in I\}$ a \textbf{set of
equivalential terms for} $\vdash$ if the following holds
%%
\index{equivalential term!set of \faul s}%%
%%
\begin{subequations}
\label{eq:eqterm}
\begin{align}
& \vdash \Delta(p,p) \\
& \Delta(p,q) \vdash \Delta(q,p) \\
& \Delta(p,q); \Delta(q,r) \vdash \Delta(p,r) \\
& \bigcup_{i < \Omega(f)} \Delta(p_i,q_i) \vdash
        \Delta(f(\vec{p}), f(\vec{q})) \\
& p; \Delta(p,q) \vdash q
\end{align}
\end{subequations}
%%%%%
\index{consequence relation!equivalential}%%
\index{consequence relation!finitely equivalential}%%
%%%%%
$\vdash$ is called \textbf{equivalential} if it has a set of
equivalential terms, and \textbf{finitely equivalential} if it
has a finite set of equivalential terms. If $\Delta(p,q)
= \{\delta(p,q)\}$ is a set of equivalential terms for $\vdash$
then $\delta(p,q)$ is called an \textbf{equivalential term}
for $\vdash$.
%%%%%
\index{term!equivalential}%%
%%%%%
\end{defn}
%%
As the reader may check, $p \dpf q$ is an equivalential term for
$\vdash^{\mathsf{PC}}$. If there is no equivalential term then
synonymy is not definable language internally. \cite{zimmermann:meaning} 
discusses the nature of meaning postulates. He requires among other 
that meaning postulates should be expressible in the language itself. 
To that effect we can introduce a 0--ary symbol $\mathsf{1}$ and a binary 
symbol $\boldsymbol{\triangleup}$ such that $\boldsymbol{\triangleup}$ 
is an equivalential term for $\vdash$ in the expanded language. 
(So, we add \eqref{eq:eqterm} for $\Delta(p,q) := 
\{\mbox{\mtt ($p\!\boldsymbol{\triangleup}\!q$)}\}$.)
To secure that $\boldsymbol{\triangleup}$ and $\mathsf{1}$ do the job 
as intended, we shall stipulate that the logical and the equational 
calculus are intertranslatable in the following way.
%%
\begin{align}
\{\mbox{\mtt $s_i$=$t_i$} : i < n\} & \vDash \mbox{\mtt $u$=$v$}
    & \Dpf &&
\{\mbox{\mtt ($s_i\!\boldsymbol{\triangleup}\!t_i$)} : 
i < n\} & \vdash \mbox{\mtt ($u\!\boldsymbol{\triangleup}\!v$)} \\
\{\delta_i : i < \kappa\} & \vdash \varphi & \Dpf &&
    \{\mbox{\mtt $\delta_i$=1} :
    i < \kappa\} & \vDash \mbox{\mtt $\varphi$=1}
\end{align}
%%
Here, $\vDash$ denotes model theoretic consequence, or,
alternatively, derivability in the equational calculus
(see Section~\ref{kap1}.\ref{kap1-1}). In this way, equations are
translated into sets of formulae. In order for this 
translation to be faithful in both directions we must
require the following (see \cite{pigozzi:fregean}).
%%
\begin{equation}
x \vdash \mbox{\mtt ($x\!\boldsymbol{\triangleup}$1)}\qquad
\mbox{\rm (G--rule)}
\end{equation}
%%
An equivalent condition is $x; y \vdash 
\mbox{\mtt ($x\!\boldsymbol{\triangleup}\!y$)}$. 
Second, we must require that
%%
\begin{equation}
\mbox{\mtt ($x\!\boldsymbol{\triangleup}\!y$)=1}
\vDash \mbox{\mtt $x$=$y$}\qquad \mbox{\rm (R--rule)}
\end{equation}
%%
Then one can show that on any algebra $\GA$ and any two congruences
$\Theta$, $\Theta'$ on $\GA$, $\Theta = \Theta'$ iff
$[\mathsf{1}]\Theta = [\mathsf{1}]\Theta'$,
so every congruence is induced by a theory. (Varieties satisfying 
%%%
\index{variety!congruence regular}%%
%%%
this are called \textbf{congruence regular}.) 
%%%
\index{variety!congruence regular}%%%
%%%
Classical modal
logics admit the addition of $\triangleup$. $\mathsf{1}$ is simply
$\top$. The postulates can more or less directly be verified.
Notice however that for a modal logic $L$ there are two
choices for $\vdash$ in Definition~\ref{defn:equivalential}:
if we choose $\Vdash_{L}$ then $p \dpf q$ is an
equivalential term; if, however, we choose $\vdash_{L}$
then $\{\qu^n (p \dpf q) : n \in \omega\}$ is a set of
equivalential terms. In general no finite set can be
named in the local case.

In fact, this holds for any logic which is an extension of boolean
logic by any number of congruential operators.
There we may conflate meaning postulates with logics. However,
%%%%
\index{logic!Fregean}%%
%%%%
call a logic \textbf{Fregean} if it satisfies
$(p \dpf q) \pf (p  \triangleup q)$. A modal logic is Fregean
iff it contains $p \pf \qu p$. There are exactly
four Fregean modal logics the least of which is
$\mathsf{K} \oplus p \pf \qu p$. The other three are
$\mathsf{K} \oplus p \dpf \qu p$, $\mathsf{K} \oplus
\qu \bot$ and $\mathsf{K} \oplus \bot$, the inconsistent
logic. This follows from the following theorem.
%%%
\begin{prop}
$\mathsf{K} \oplus p \pf \qu p = \mathsf{K} \oplus
(\qu p \dpf (p \oder \qu \bot))$.
\end{prop}
%%%
\proofbeg
Put $L := \mathsf{K} \oplus p \pf \qu p$. $\qu p \dpf (p \oder \qu \bot) 
\vdash_{\mathsf{K}} p \pf \qu p$, so have to show that $\qu p \dpf (p %
\oder \qu \bot) \in L$. (1) $\vdash_{\mathsf{K}} \qu p \pf (\wD p \oder \qu \bot)$. 
Furthermore, $\vdash_{L} \wD p \pf p$. Hence also $\vdash_L 
\qu p \pf (p \oder \qu \bot)$. (2) $\qu \bot \pf \qu p$ is a theorem 
of $\mathsf{K}$, $p \pf \qu p$ holds by assumption. This shows the claim.
\proofend

Now, in a Fregean logic, any proposition $\varphi$ is equivalent
either to a nonmodal proposition or to a proposition
$(\chi \und \wD \top) \oder (\chi' \und \qu \bot)$,
where $\chi$ and $\chi'$ are nonmodal. It follows from this that
the least Fregean logic has only constant extensions: by the axiom
$\qu \bot$, by its negation $\wD \top$, or both (which yields the
inconsistent logic).

Now let us return to Leibniz' Principle.  Fix a theory $T$. Together
%%%
\index{Leibniz' Principle}%%%
%%%
with the algebra of formulae it forms a matrix. This matrix tells us
what is true and what is not. Notice that the members of the algebra
are called truth--values in the language of matrices. In the
present matrix, a sentence is true only if it is a member of $T$.
Otherwise it is false. Thus, although we can have as many truth values
as we like, sentences are simply true or false. Now apply Leibniz'
%%%
\index{Leibniz' Principle}%%%
%%%
Principle. It says: two propositions $\varphi$ and $\varphi'$ have
identical meaning iff for every proposition $\chi$ and
every variable $p$: $[\varphi/p]\psi \in T$
iff $[\varphi'/p]\chi \in T$. This leads to the following
definition.
%%%
\begin{defn}
Let $\Omega$ be a signature and $\GA$ an $\Omega$--algebra. Then for
any $F \subseteq A$ put
%%
\begin{equation}
\Delta_{\GA}F := \{\auf a,b\zu : \mbox{ for all }t \in
    \Pol_1(\GA): t(a) \in F \Dpf t(b) \in F\}
\end{equation}
%%
This is called the \textbf{Leibniz equivalence} 
%%%
\index{Leibniz equivalence}%%
%%%
and the map $\Delta_{\GA}$ the \textbf{Leibniz operator}.
%%%
\index{Leibniz operator}%%
%%%
\end{defn}
%%
Notice that the definition uses unary polynomials, not just terms.
This means in effect that we have enough constants to name all
expressible meanings, not an unreasonable assumption.
%%%
\begin{lem}
$\Delta_{\GA}F$ is an admissible congruence on $\auf \GA, F\zu$.
\end{lem}
%%%
\proofbeg%%
It is easy to see that this is an equivalence relation. By
Proposition~\ref{prop:punary} it is a congruence relation. We show
that it is compatible with $F$. Let $x \in F$ and $x\; \Delta_{\GA}F\;
y$. Take $t := p$ we get $[x/p]t = x$, $[y/p]p = y$. Since $x \in
F$ we have $y \in F$ as well.%%
\proofend

We know that the consequence relation of $\auf \GA, F\zu$ is the
same as the congruence relation of $\auf \GA/\Delta_{\GA}F,
F/\Delta_{\GA}F\zu$. So, from a semantical point of view we may
simply factor out the congruence $\Delta_{\GA}F$. Moreover, this
matrix satisfies Leibniz' Principle! So, in aiming to define
%%%
\index{Leibniz' Principle}%%%
%%%
meanings from the language and its logic, we must first choose a
theory and then factor out the induced Leibniz equivalence. We may
then take as the meaning of a proposition simply its equivalence
class with respect to that relation. Yet, the equivalence depends
on the theory chosen and so do therefore the meanings. If the
0--ary constant $\mbox{\tt c}_0$ is a particular sentence (say,
that Caesar crossed the Rubicon) then depending on whether this
sentence is true or not we get different meanings for our language
objects. However, we shall certainly not make the assumption that
meaning depends on accidental truth. Therefore, we shall say the
following.
%%%
\begin{defn}
%%%
\index{canonical Leibniz congruence}%%
\index{canonical Leibniz meaning}%%
\index{$\nabla_{\vdash}$}%%
%%%
Let $\vdash$ be a structural consequence relation over a language
of signature $\Omega$. Then the \textbf{canonical Leibniz congruence}
is defined to be
%%
\begin{equation}
\nabla_{\vdash} := \Delta_{\goth{Fm}_{\Omega}(\varnothing)}
\Taut(\vdash \cap\; \Tm_{\Omega}(\varnothing))
\end{equation}
%%
For a proposition $\varphi$ free of variables, the object
$[\varphi]_{\nabla_{\vdash}}$ is called the \textbf{canonical}
(\textbf{Leibniz}) \textbf{meaning} of $\varphi$.
%%
\end{defn}
%%%
The reader is asked to check that
$\Pol_1(\goth{Tm}_{\Omega}(\varnothing))
= \Clo_1(\goth{Tm}_{\Omega}(\varnothing))$, so that nothing
hinged on the assumption made earlier to admit all polynomials
for the definition of $\Delta_{\GA}$. We shall briefly comment on
the fact that we deal only with constant propositions. From the
standpoint of language, propositional variables have no meaning
except as placeholders. To ask for the meaning of
$\mbox{\mtt (p0\symbol{31}(\symbol{5}p11))}$ in the context of
language makes little sense since language is a means of communicating 
concrete meanings. A variable only stands in for the possible concrete
meanings. Thus we end up with a single algebra of meanings,
one that even satisfies Leibniz' Principle.
%%%
\index{Leibniz' Principle}%%%
%%%

In certain cases the Leibniz operator actually induces an
isomorphism between the lattice of deductively closed sets and the
lattice of congruences on $\GA$. This means that different theories
will actually generate different equalities and different equalities
will generate different theories. For example, in boolean logic, a
theory corresponds to a deductively closed set in the free algebra
of propositions.  Moreover, $\auf \varphi, \chi\zu \in \Delta_{\GB}T$ iff
$\varphi \dpf \chi \in T$. On the left hand side we find the
Leibniz congruence generated by $T$, on the right hand side we
find $T$ applied to a complex expression formed from $\varphi$ and
$\chi$. It means in words the following (taking $T$ to be the
theory generated by $\varnothing$): two propositions, $\varphi$ and
$\chi$, have the same meaning iff the proposition $\varphi \dpf \chi$ 
is a tautology. This does not hold for modal logic; for in modal logic 
a theory induces a nontrivial consequence only if it
is closed under the necessitation rule. (The exactness of the
correspondence is guaranteed for boolean logic by the fact
that it is Fregean.)

We shall now briefly address the general case of a language as a
system of signs. We assume for a start a grammar, with certain
modes. The grammar supplies a designated category  $t$ of
sentences. We may define notions of logic, theory and so on on the
level of definite structure terms of category $t$, since these are
unique by construction. This is how Church formulated the simple
theory of types, $\mathsf{STyp}$ (see next section). A 
%%%
\index{theory}%%
%%%
\textbf{theory} 
is now a set of definite structure terms of category $t$ which is 
closed under consequence. Given a theory $T$ we define the following 
relation on definite structure terms: $\Gt\; \Delta\; \Gt'$ iff
the two are intersubstitutable in any structure term preserving
definiteness, and for a structure term $\Gs$: $[\Gt/x]\Gs \in T$
iff $[\Gt'/x]\Gs \in T$. Again, this proves to be a
congruence on the partial algebra of definite structure terms, and
the congruence relation can be factored. What we get is the
algebra of natural meanings.

{\it Notes on this section.}
There have been accounts of propositional attitudes that propose 
representations of attitude reports that do not contain a 
representation of the embedded proposition. These accounts have 
difficulties with Leibniz' Principle. Against this argues
%%%
\index{Leibniz' Principle}%%%
%%%
\cite{recanati:oratio}. 
%%%
\index{Recanati, Fran\c{c}ois}%%
%%%
For him, the representation of the report 
contains a representation of the embedded proposition so that 
Leibniz' Principle does not need to be stipulated. Modal operators 
actually have that property. However, they do not provide enough 
analysis of the kinds of attitudes involved. On the other hand, 
modal operators are very flexible. In general, most attitudes will 
not give rise to a normal logic, though classicality 
must be assumed, in virtue of Leibniz' Principle. Also, there is a 
%%%
\index{Leibniz' Principle}%%%
%%%
consensus that a proposition is an expression modulo the laws
of $\mathsf{PC}$. However, notice that this means only that if two
expressions are interderivable in $\mathsf{PC}$, we must have the same
attitude towards them. It does not say, for example, that if
$\chi$ follows from $\varphi$ then if I believe $\varphi$ I
also believe that $\chi$. Classical logics need not be monotone
(see the exercises below).  For the general theory of modal logic
see \cite{kracht:tools}.
%%%
\vplatz
\exercise
Set up a Galois correspondence between contexts and equivalence
classes of expressions. You may do this for any category $\alpha$.
Can you characterize those context sets that generate the same
equivalence class?
%%%
\vplatz
\exercise
Show that $\vdash_{L}$ as defined above is a consequence
relation. Show that (mn) is not derivable in
$\vDash_{\mathsf{K}}$. {\it Hint.} You have to find a
formula $\varphi$ such that $\varphi \nvDash_{\mathsf{K}}
\qu \varphi$.
%%%
\vplatz 
\exercise 
Show that $\vDash_{\CK}^g$ is the global consequence relation
associated with $\mathsf{Th}\, \CK$.
%%%
\vplatz
\exercise
Show that the logic of knowledge axiomatized above is $\mathsf{S5}$.
%%%
\vplatz
\exercise
Let $\auf F, \lhd, \BF\zu$ be a generalized Kripke--frame and
$\beta$ a valuation into it. Put $\GM := \auf \BF, \varnothing, F, \cap,
\cup, \sq\zu$. Then $\beta$ has an obvious homomorphic extension
$\oli{\beta} \colon \goth{Tm}_{\Omega}(V) \pf \GM$. Show that
$\auf \GF, \beta, x\zu \vDash \varphi$ iff
$x \in \oli{\beta}(\varphi)$.
%%
\vplatz
\exercise
Show that there are classical modal logics which are not monotone.
{\it Hint.} There is a counterexample based on a two--element
algebra.
%%%
\vplatz
\exercise
Prove Lemma~\ref{lem:gensub}.
%%%
\vplatz
\exercise
Define $\boxtimes \varphi := \varphi \triangleup \top$. Let $L$ 
be the set of formulae in $\boxtimes$ and the boolean connectives 
that are derivable. Show that $L$ is a normal modal logic
containing $\mathsf{K4}$.

 \section{Binding and Quantification}
\label{kap6-4a}
%
%
%
Quantification and binding are one of the most intricate phenomena
of formal semantics. Examples of quantifiers we have seen already:
the English phrases {\tt every} and {\tt some}, and $\forall$ and 
$\exists$ of predicate logic. Examples of binding without quantification 
can be found easily in mathematics. The integral
%%
\begin{equation}
h(\vec{y}) := \int_0^1 f(x,\vec{y})dx
\end{equation}
%%
is a case in point. The integration operator takes a function (which
may have parameters) and returns its integral over the interval 
$[0,1]$. What this has in common with quantification is that the 
function $h(\vec{y})$ does not depend on $x$. Likewise, the limit 
$\lim_{n \pf \infty} a_n$ of a convergent series 
$a \colon \omega \pf \BR$, is independent of $n$.
(The fact that these operations are not everywhere defined shall
not concern us here.) So, as with quantifiers, integration and
limits take entities that depend on a variable $x$ and return an
entity that is independent of it. The easiest way to analyze this
phenomenon is as follows. Given a function $f$ that depends on $x$,
$\lambda x.f$ is a function which is independent of $x$. Moreover,
everything that lies encoded in $f$ is also encoded in $\lambda
x.f$. So, unlike quantification and integration,
$\lambda$--abstraction does not give rise to a loss of
information. This is ensured by the identity $(\lambda x.f)x = f$.
Moreover, extensionality ensures that abstraction also does not
add any information: the abstracted function is essentially
nothing more than the graph of the function. $\lambda$--abstraction
therefore is {\it the\/} mechanism of binding. Quantifiers,
integrals, limits and so on just take the
$\lambda$--abstract and return a value. This is exactly how we
have introduced the quantifiers: $\exists x.\varphi$
was just an abbreviation of {\tt \GSi(\tlambda$x$.$\varphi$)}. 
Likewise, the integral can be decomposed into two steps: first, 
abstraction of a variable and then the actual integration. Notice, 
how the choice of variable matters:
%%
\begin{equation}
y/3 = \int_0^1 x^2y dx \neq x/2 = \int_0^1 x^2y dy
\end{equation}
%%
The notation $dx$ actually does the same as $\lambda x$: it shows
us over which variable we integrate. We may define integration
as follows. First, we define an operation $I \colon {\BR}^{\BR} \pf {\BR}$,
which performs the integration over the interval $[0,1]$ of $f
\in {\BR}^{\BR}$. Then we define
%%
\begin{equation}
\int_0^1 f(x)dx := I(\lambda x.f)
\end{equation}
%%
This definition decouples the definition of the actual integration
from the binding process that is involved. In general, any operator
$O\auf x_i : i < n\zu.M$ which binds the variables $x_i$, $i < n$,
and returns a value, can be defined as
%%
\begin{equation}
\label{eq:integral}
O\auf x_i : i < n\zu.M := \wht{O}(\lambda x_0.\lambda x_1.\dotsb.%
\lambda x_{n-1}.M)
\end{equation}
%%
for a suitable $\wht{O}$. In fact, since $O\vec{x}.M$ does not
depend on $\vec{x}$, we can use \eqref{eq:integral} to define $\wht{O}$.
What this shows is that $\lambda$--calculus can be used as a
general tool for binding. It also shows that we can to some
extent get rid of explicit variables, something that is quite
useful for semantics. The elimination of variables 
removes a point of arbitrariness in the representation that
makes meanings nonunique. In this section, we shall introduce
two different algebraic calculi.  The first is the algebraic
approach to predicate logic using so called {\it cylindric
algebras}, the other an equational theory of $\lambda$--calculus,
which embraces the (marginally popular) variable free approach
to semantics for first order logic.

\index{logic!first--order}%%
\index{first--order logic}%%
\index{FOL}%%
We have already introduced the syntax and semantics of first--order
predicate logic. Now we are going to present an axiomatization. To
this end we expand the set of axioms for propositional logic by
the axioms (a13) -- (a21) in Table~\ref{tab:predlog}. 
%%
\begin{table}
\caption{The Axioms for Predicate Logic (FOL)}
\label{tab:predlog}
\mbox{\sc Axioms.} \\[2mm]
$\begin{array}{ll}
\multicolumn{2}{l}{\mbox{\rm (a0) --- (a12)} +} \\
\mbox{\rm (a13)} & (\forall x)(\varphi\pf\chi)
    \pf ((\forall x)\varphi \pf (\forall x)\chi) \\
\mbox{\rm (a14)} & (\forall x)\varphi \pf [t/x]\varphi \\
\mbox{\rm (a15)} & \varphi \pf (\forall x)\varphi \qquad 
    (x\not\in \fr(\varphi)) \\
\mbox{\rm (a16)} & (\forall x)\varphi \pf
    \nicht (\exists x)\nicht \varphi \\
\mbox{\rm (a17)} & \nicht (\exists x)\nicht \varphi
    \pf (\forall x)\varphi \\
\mbox{\rm (a18)} & (\forall x)(x \doteq x) \\
\mbox{\rm (a19)} & (\forall x)(\forall y)(x \doteq y \pf y \doteq x) \\
\mbox{\rm (a20)} & (\forall x)(\forall y)(\forall z)(x \doteq y 
	\und y \doteq z \pf x \doteq z) \\
\mbox{\rm (a21)} & (\forall x_0)\dotsb (\forall x_{\Xi(R)-1})(\forall 
	y)((\gund_{i < \Xi(R)} x_i
\doteq y_i) \\
    & \qquad \pf  (R(x_0, \dotsc, x_{\Xi(R)-1})
    \dpf [y/x_i]R(x_0, \dotsc, x_{\Xi(R)-1})))
\end{array}$
\\[2mm]
\mbox{\sc Rules.}\\[2mm]
\mbox{\rm (mp)}~~$\begin{array}{c}
    \varphi \quad \varphi\pf\chi \\\hline
    \chi
    \end{array}
    \qquad
  \mbox{\rm (gen)}~~\begin{array}{c}
    \varphi \\\hline
    (\forall x)\varphi
    \end{array}$
\end{table}
%%
\index{$\mathsf{FOL}$, $\vdash^{\mathsf{FOL}}$}%%%
%%%
The calculus (a0) -- (a21) with the rules (mp) and (gen) is called 
%%%
\index{first--order logic}%%
%%%%
$\mathsf{FOL}$. A \textbf{first--order theory} is a set of formulae 
containing (a0) -- (a21) and which is closed under (mp). 
We write $\Delta \vdash^{\mathsf{FOL}} \varphi$ if every theory 
containing $\Delta$ also contains $\varphi$.
In virtue of (a16) and (a17) we get that 
$(\forall x)\varphi \dpf \nicht(\exists x)\nicht\varphi$ as well as 
$(\exists x)\varphi \dpf \nicht(\forall x)\nicht\varphi$ which means 
that one of the quantifiers can be defined from the other. In (a21), 
we assume $i < \Xi(R)$. 

We shall prove its completeness using a more powerful result due to Leon 
Henkin that simple type theory ($\mathsf{STyp}$) is complete with respect 
to Henkin--frames. Notice that the status of (gen) is the same as that 
of (mn) in modal logic. (gen) is admissible with respect to the model 
theoretic consequence $\vDash$ defined in Section~\ref{kap3}.\ref{kap3-6}, 
but it is
not derivable in it. To see the first, suppose that $\varphi$ is a
theorem and let $x$ be a variable. Then $(\forall x)\varphi$ is a
theorem, too. However, $\varphi \vDash (\forall x)\varphi$ does
not follow. Simply take a unary predicate letter $P$ and a
structure consisting of two elements, 0, 1, such that $P$ is true
of 0 but not of 1. Then with $\beta(x) := 0$ we have $\auf \GM,
\beta\zu \vDash P(x)$ but $\auf \GM, \beta\zu \nvDash (\forall
x)P(x)$. Now let $\uli{P}$ be the set of all formulae that can be
obtained from (a0) -- (a21) by applying (gen). Then the following
holds.
%%
\begin{thm}
\label{thm:genelim}
$\vdash^{\mathsf{FOL}} \varphi$ iff $\varphi$ is derivable from 
$\uli{P}$ using only (mp).
\end{thm}
%%
\proofbeg
Let $\Pi$ be a proof of $\chi$ from $\uli{P}$ using only (mp). 
Transform the proof in the following way. First, prefix every 
occurring formula by $(\forall x)$. Further, for every $k$ such 
that $\varphi_k$ follows from $\varphi_i$ 
and $\varphi_i \pf \varphi_j$ for some $i,j < k$, insert in front of 
the formula $(\forall x)\varphi_j$ the sequence
%%%
\begin{equation}
(\forall x)(\varphi_i \pf \varphi_j) \pf ((\forall x)\varphi_i \pf 
	(\forall x)\varphi_j), 
(\forall x)\varphi_i \pf (\forall x)\varphi_j 
\end{equation}
%%%
The new proof is a proof of $(\forall x)\chi$ from $\uli{P}$, 
since the latter is closed under (gen). Hence, the set of formulae 
derivable from $\uli{P}$ using (mp) is closed under (gen). Therefore 
it contains all tautologies of $\mathsf{FOL}$.
\proofend

The next theorem asserts
that this axiomatization is complete.
%%%
\begin{defn}
Let $\Delta$ be a set of formulae of predicate logic over a
signature, $\varphi$ a formula over that same signature. Then
$\Delta \vdash \varphi$ iff $\varphi$ can be proved from 
$\Delta \cup \uli{P}$ using only (mp).
\end{defn}
%%%
\begin{thm}[G\"odel]
%%%
\index{G\"odel, Kurt}%%%
%%%
\label{thm:ent}
$\Delta \vdash^{\mathsf{FOL}} \varphi$ iff $\Delta \vDash \varphi$.
\end{thm}
%%%
Recall from Section~\ref{kap3}.\ref{kap3-6} the definition of the simple
theory of types. 
%%%
\index{simple type theory (STT)}%%%
%%%
There we have also defined the class of models,
the so called {\it Henkin--frames}. Recall further that this
theory has operators $\Pi^{\alpha}$, which allow to define the
universal quantifiers in the following way.
%%
\begin{equation}
(\forall x_{\alpha})N_t := (\Pi^{\alpha} (\lambda
x_{\alpha}.N_t))
\end{equation}
%%
The simple theory of types is axiomatized as follows. We define a
calculus exclusively on the terms of type $t$ (truth values).
However, it will also be possible to express that two terms are
equal. This is done as follows. Two terms $M_{\alpha}$ and
$N_{\alpha}$ of type $\alpha$ are equal if for every term
$O_{\alpha\pf t}$ the terms $O_{\alpha\pf t}M_{\alpha}$ and
$O_{\alpha\pf t}N_{\alpha}$ are equivalent.
%%
\newcommand{\neweq}{\triangleq}
%%
\begin{equation}
M_{\alpha} \neweq N_{\alpha}  :=
    (\forall z_{\alpha\pf t})(z_{\alpha\pf t}M_{\alpha}
    \dpf z_{\alpha\pf t}N_{\alpha})
\end{equation}
%%
For this definition we assume that $z_{\alpha\pf t}$ is free
neither in $M_{\alpha}$ nor in $N_{\alpha}$. If one dislikes 
the side conditions, one can prevent the accidental capture of 
$z_{\alpha\pf t}$ using the following more refined version:
%%
\begin{equation}
%%%
\index{$M_{\alpha} \neweq N_{\alpha}$}%%%
%%%
M_{\alpha} \neweq N_{\alpha}  :=
    (\lambda x_{\alpha}.\lambda y_{\alpha}.(\forall %
    z_{\alpha\pf t})(z_{\alpha\pf t}x_{\alpha} \dpf %
    z_{\alpha\pf t}y_{\alpha}))M_{\alpha}N_{\alpha}
\end{equation}
%%
However, if $z_{\alpha\pf t}$ is properly chosen, no problem will
ever arise. Now, let (s0) -- (s12) be the formulae (a0) -- (a12)
appropriately translated into the language of types.
%%
\begin{table}
\caption{The Simple Theory of Types}
\index{$\mathsf{STyp}$}%%%
\label{tab:styp}
\mbox{\sc Axioms.} \\[2mm]
\begin{tabular}{ll}
\multicolumn{2}{l}{(s0) --- (s12) +} \\
(s13) & $((\forall x_{\alpha})(y_t \pf M_{\alpha\pf t}x_{\alpha}))
    \pf (y_t \pf \Pi^{\alpha}M_{\alpha\pf t})$ \\
(s14) & $(\Pi^{\alpha} x_{\alpha \pf t}) \pf x_{\alpha\pf t}y_{\alpha}$ \\
(s15) & $(x_{t} \dpf y_{t}) \pf x_t \neweq y_t$ \\
(s16) & $((\forall z_{\alpha})(x_{\alpha\pf\beta}z_{\alpha}
    \neweq y_{\alpha\pf\beta}z_{\alpha})) \pf
    (x_{\alpha\pf\beta} \neweq y_{\alpha\pf\beta})$ \\
(s17) & $x_{\alpha\pf t}y_{\alpha} \pf x_{\alpha\pf t}
    (\iota^{\alpha}x_{\alpha \pf t})$
\end{tabular}
\\[2mm]
\mbox{\sc Rules.} \\
$\begin{array}{l@{\;}l@{\qquad}l@{\;}l}
\mbox{\rm (mp)} & \begin{array}{c}
              M_t \pf N_t \quad M_t \\\hline
          N_t
          \end{array}
        &
\mbox{\rm (ug)} & \begin{array}{c}
           M_{\alpha \pf t}x_{\alpha} \\\hline
           \Pi^{\alpha}M_{\alpha}
           \end{array}
           \quad x_{\alpha} \not\in \fr(M_{\alpha\pf t})
         \\
\mbox{\rm (conv)} & \begin{array}{c}
           M_t \quad M_t \equiv_{\alpha\beta} N_t \\\hline
           N_t
           \end{array}
          &
\mbox{\rm (sub)} & \begin{array}{c}
           M_{\alpha \pf t}x_{\alpha} \\\hline
           M_{\alpha\pf t}N_{\alpha}
           \end{array}
           \quad
           x_{\alpha} \not\in \fr(M_{\alpha\pf t})
\end{array}$
%%
\end{table}
%%
We call the Hilbert--style calculus consisting of (s0) -- (s17) and 
the rules given in Table~\ref{tab:styp} $\mathsf{STyp}$. 
%%%
\index{$\mathsf{STyp}$}%%%
%%%%
All instances of theorems of $\mathsf{PC}$ 
are theorems of $\mathsf{STyp}$. For predicate logic this will also
be true, but the proof of that requires work. The rule (gen) is a 
derived rule of this calculus. To see this, assume that $M_{\alpha \pf
t}z_{\alpha}$ is a theorem. Then, by (conv), $(\lambda
z_{\alpha}.M_{\alpha\pf t}z_{\alpha})z_{\alpha}$ also is a
theorem. Using (ug) we get $\Pi^{\alpha}(\lambda
z_{\alpha}.M_{\alpha\pf t}z_{\alpha})$, which by abbreviatory
convention is $(\forall z_{\alpha})M_{\alpha\pf t}$. We will also
show that (a14) and (a15) are theorems of $\mathsf{STyp}$.
%%
\begin{lem}
\label{lem:a14}%%
$\mathsf{STyp} \vdash (\forall x_{\alpha})y_t \pf
[N_{\alpha}/x_{\alpha}]y_t$.
\end{lem}
%%%
\proofbeg %
By convention, $(\forall x_{\alpha})y_t = \Pi^{\alpha}(\lambda
x_{\alpha}.y_t)$. Moreover, by (s14), $\mathsf{STyp} \vdash ((\forall
x_{\alpha})y_t) \pf (\lambda x_{\alpha}.y_t)x_{\alpha} = 
((\forall x_{\alpha})y_t) \pf y_t$. 
Using (sub) we get
%%
\begin{equation}
\vdash^{\mathsf{STyp}}
[N_{\alpha}/x_{\alpha}]((\forall x_{\alpha})y_t \pf y_t) =
(\forall x_{\alpha})y_t \pf [N_{\alpha}/x_{\alpha}]y_t
\end{equation}
%%
as required. \proofend
%%%
\begin{lem}
Assume that $x_{\alpha}$ is not free in $N_t$. Then 
%%
\begin{equation}
\mathsf{STyp} \vdash N_t \pf (\forall x_{\alpha})N_t\
\end{equation}
\end{lem}
%%
\proofbeg %
With $N_t \pf N_t \equiv_{\alpha\beta} N_t \pf (\lambda
x_{\alpha}.N_t)x_{\alpha}$ and the fact that $(\forall
x_{\alpha})(N_t \pf N_t)$ is derivable (using (gen)), we get with
(conv) $(\forall x_{\alpha})(N_t \pf ((\lambda%
x_{\alpha}.N_t)x_{\alpha}))$ and with (s13) and (mp) we get 
%%%
\begin{equation}
\vdash^{\mathsf{STyp}} N_t \pf \Pi^{\alpha}(\lambda x_{\alpha}.N_t) = 
N_t \pf (\forall x_{\alpha})N_t
\end{equation}
%%%
(The fact that $x_{\alpha}$ is 
not free in $N_t$ is required when using (s13). In order for the 
replacement of $N_t$ for $y_t$ in the scope of $(\forall x_{\alpha})$ 
to yield exactly $N_t$ again, we need that $x_{\alpha}$ is not free 
in $N_t$.) %%
\proofend
%%%
\begin{lem}
\label{lem:lambdaeta} 
If $\mbox{\sf\textgreek{lh}} \vdash
M_{\alpha} \boldsymbol{\doteq} N_{\alpha}$ then $\mathsf{STyp}\vdash
M_{\alpha} \neweq N_{\alpha}$.
\end{lem}
%%
\proofbeg 
{\sf\textgreek{lh}} = {\sf\textgreek{l}} + (ext), and
(ext) is the axiom (s16). Hence it remains to show that 
$\mbox{\sf\textgreek{l}} \vdash M_{\alpha} \boldsymbol{\doteq} 
N_{\alpha}$ implies $\vdash^{\mathsf{STyp}} M_{\alpha} 
\neweq N_{\alpha}$. So, assume the first. Then we have $M_{\alpha}
\equiv_{\alpha\beta} N_{\alpha}$. Hence
%%%
\begin{equation}
z_{\alpha\pf t}M_{\alpha} \pf z_{\alpha\pf t}M_{\alpha}
\equiv_{\alpha\beta} z_{\alpha\pf t}M_{\alpha} \pf z_{\alpha\pf
t}N_{\alpha}
\end{equation}
%%%
Hence, using (conv), and $\vdash^{\mathsf{STyp}} 
z_{\alpha \pf t}M_{\alpha} \pf z_{\alpha\pf t}M_{\alpha}$
we get 
%%
\begin{equation}
\vdash^{\mathsf{STyp}} z_{\alpha\pf t}M_{\alpha}
\pf z_{\alpha\pf t}N_{\alpha}
\end{equation}
%%
By symmetry, $\vdash^{\mathsf{STyp}} z_{\alpha\pf t}M_{\alpha}
\dpf z_{\alpha\pf t}N_{\alpha}$.
%%
Using (gen)  we get
%%%
\begin{equation}
\vdash^{\mathsf{STyp}} (\forall z_{\alpha \pf t})%
(z_{\alpha\pf t}M_{\alpha} \dpf z_{\alpha\pf t}N_{\alpha})
\end{equation}
%%%
By abbreviatory convention, $\vdash^{\mathsf{STyp}} 
M_{\alpha} \neweq N_{\alpha}$. 
\proofend

We shall now show that $\mathsf{STyp}$ is complete with respect to
Henkin--frames where $\neweq$ simply is interpreted as identity.
To do that, we first prove that $\neweq$ is a congruence relation.
%%%
\begin{lem}
\label{lem:neweq}%%
The following formulae are provable in $\mathsf{STyp}$.
%%
\begin{subequations}
\begin{align}
\label{eq:eq1}
& x_{\alpha} \neweq x_{\alpha} \\
\label{eq:eq2}
& x_{\alpha} \neweq y_{\alpha} \pf
    y_{\alpha} \neweq x_{\alpha} \\
\label{eq:eq3}
& x_{\alpha} \neweq y_{\alpha} \und y_{\alpha}
\neweq z_{\alpha}
    \pf x_{\alpha} \neweq z_{\alpha} \\
\label{eq:eq4}
& x_{\alpha\pf\beta} \neweq x'_{\alpha\pf\beta}
\und
    y_{\alpha} \neweq y'_{\alpha} .\pf.
    x_{\alpha\pf\beta}y_{\alpha} \neweq
    x'_{\alpha\pf\beta}y'_{\alpha}
\end{align}
\end{subequations}
\end{lem}
%%
\proofbeg%%
\eqref{eq:eq1} Let $z_{\alpha\pf t}$ be a variable of type $\alpha\pf t$.
Then $z_{\alpha\pf t}x_{\alpha} \dpf z_{\alpha\pf t}x_{\alpha}$ is
provable in $\mathsf{STyp}$ (as $p \dpf p$ is in $\mathsf{PC}$). Hence, 
$(\forall z_{\alpha\pf t})(z_{\alpha\pf t}x_{\alpha} \dpf z_{\alpha\pf
t}x_{\alpha})$ is provable, which is $x_{\alpha} \neweq x_{\alpha}$. 
\eqref{eq:eq2} and \eqref{eq:eq3} are shown using 
predicate logic. \eqref{eq:eq4} Assume $x_{\alpha\pf\beta} \neweq
x'_{\alpha\pf\beta}$  and $y_{\alpha} \neweq y'_{\alpha}$. 
Now,
%%%%
\begin{equation}
z_{\beta\pf t}(x_{\alpha\pf\beta}y_{\alpha}) \equiv_{\alpha\beta}
(\lambda u_{\alpha}.z_{\beta\pf t}(x_{\alpha\pf\beta}u_{\alpha}))
(y_{\alpha})
\end{equation}
%%%
Put $M_{\alpha\pf t} := (\lambda u_{\alpha}.z_{\beta\pf t}(%
x_{\alpha\pf\beta}u_{\alpha}))$. Using the rule (conv), we get
%%
\begin{equation}
\begin{split}
y_{\alpha} \neweq y'_{\alpha} & \vdash^{\mathsf{STyp}} M_{\alpha}y_{\alpha} \dpf
    M_{\alpha}y'_{\alpha} \\
    & \vdash^{\mathsf{STyp}} z_{\beta\pf t}(x_{\alpha\pf\beta}y_{\alpha}) \dpf
    z_{\beta\pf t}(x_{\alpha\pf\beta}y'_{\alpha})
\end{split}
\end{equation}
%%
Likewise, one can show that
%%
\begin{equation}
y_{\alpha} \neweq y'_{\alpha} \vdash^{\mathsf{STyp}}
    z_{\beta\pf t}(x'_{\alpha\pf\beta}y_{\alpha}) \dpf
    z_{\beta\pf t}(x'_{\alpha\pf\beta}y'_{\alpha})
\end{equation}
%%
Similarly, using $N_{(\alpha\pf\beta) \pf t} := (\lambda
u_{\alpha\pf\beta}.z_{\beta\pf t}(u_{\alpha\pf \beta}
    y_{\alpha}))$ one shows that
%%
\begin{equation}
x_{\alpha\pf\beta} \neweq x'_{\alpha\pf\beta} \vdash^{\mathsf{STyp}}
    z_{\beta\pf t}(x_{\alpha\pf\beta}y_{\alpha}) \dpf
    z_{\beta\pf t}(x'_{\alpha\pf\beta}y_{\alpha}) 
\end{equation}
%%
This allows to derive the desired conclusion. %%
\proofend

Now we get to the construction of the frame. Let $C_{\alpha}$ be
the set of closed formulae of type $\alpha$. Choose a maximally
consistent $\Delta \supseteq C_t$. Then, for each type
$\alpha$, define $\approx^{\alpha}_{\Delta}$ by $M_{\alpha}
\approx^{\alpha}_{\Delta} N_{\alpha}$ iff $M_{\alpha}
\neweq N_{\alpha} \in \Delta$. By Lemma~\ref{lem:neweq} this
is an equivalence relation for each $\alpha$, and, moreover, if
$M_{\alpha\pf\beta} \approx^{\alpha\pf\beta}_{\Delta}
M'_{\alpha\pf\beta}$ and $N_{\alpha} \approx^{\alpha}_{\Delta}
N'_{\alpha}$, then also $M_{\alpha\pf\beta}N_{\alpha}
\approx^{\beta}_{\Delta} M'_{\alpha\pf\beta}N'_{\alpha}$. For
$M_{\alpha} \in C_{\alpha}$ put
%%
\begin{equation}
[M_{\alpha}] := \{N_{\alpha} : M_{\alpha} \approx^{\alpha}_{\Delta}
N_{\alpha}\}
\end{equation}
%%
Finally, put $D_{\alpha} := \{[M_{\alpha}] : M_{\alpha} \in
C_{\alpha}\}$. Next, $\bullet$ is defined as usual, $- :=
[\mbox{\mtt\symbol{5}}]$, $\cap := [\mbox{\mtt\symbol{4}}]$, 
$\pi^{\alpha} := [\GPi^{\alpha}]$ and 
$\iota^{\alpha} := [\Giota^{\alpha}]$. This 
defines a structure (where $\CS := \Typ_{\pf}(B)$).
%%
\begin{equation}
\goth{Hfr}_{\Delta} := \auf \{D_{\alpha}:
\alpha \in \CS\}, \bullet, -, \cap, \auf
\pi^{\alpha} : \alpha \in \CS\}\zu, \auf
\iota^{\alpha} : \alpha \in \CS\}\zu\zu
\end{equation}
%%
\begin{lem}[Witnessing Lemma]
\begin{equation*}
\mathsf{STyp} \vdash^{\mathsf{STyp}} M_{\alpha\pf t}(\iota^{\alpha}(\lambda
x_{\alpha}.\nicht M_{\alpha})) \pf \Pi^{\alpha}M_{\alpha}
\end{equation*}
\end{lem}
%%%
\proofbeg
Write $\nicht N_{\alpha\pf t} := \lambda x_{\alpha}.\nicht %
(N_{\alpha\pf t}x_{\alpha})$. Now, by (s17)
%%
\begin{equation}
\vdash^{\mathsf{STyp}} (\nicht N_{\alpha\pf
t})y_{\alpha} \pf (\nicht N_{\alpha\pf t})(\iota^{\alpha}(\nicht %
N_{\alpha \pf t}))
\end{equation}
%%
Using classical logic we obtain
%%
\begin{equation}
\vdash^{\mathsf{STyp}} \nicht ((\nicht N_{\alpha\pf t})(\iota^{\alpha} %
(\nicht N_{\alpha\pf t}))) \pf \nicht ((\nicht N_{\alpha\pf
t})y_{\alpha})
\end{equation}
%%
Now, $\nicht ((\nicht N_{\alpha\pf t})y_t) \triangleright_{\beta}
\nicht \nicht (N_{\alpha\pf t}y_{\alpha})$, the latter being
equivalent to $N_{\alpha\pf t}y_{\alpha}$. Similarly,
$\nicht ((\nicht N_{\alpha\pf t})(\iota^{\alpha}(\nicht N_{\alpha\pf t})))$
is equivalent with $N_{\alpha\pf t}(\iota^{\alpha}(\nicht
N_{\alpha\pf t}))$. Hence
%%
\begin{equation}
\vdash^{\mathsf{STyp}} N_{\alpha\pf t}(\iota^{\alpha}(\nicht
N_{\alpha\pf t})) \pf N_{\alpha\pf t}y_{\alpha} 
\end{equation}
%%
Using (gen), (s13) and (mp) we get
%%
\begin{equation}
\vdash^{\mathsf{STyp}} (N_{\alpha\pf t}(\iota^{\alpha}(\nicht
N_{\alpha\pf t}))) \pf \Pi^{\alpha}N_{\alpha\pf t}
\end{equation}
%%
This we had to show.
\proofend
%%
\begin{lem}
$\goth{Hrf}_{\Delta}$ is a Henkin--frame.
\end{lem}
%%%
\proofbeg%%
By Lemma~\ref{lem:lambdaeta}, if $\mbox{\sf\textgreek{lh}} \vdash
M_{\alpha} \neweq N_{\alpha}$, then $[M_{\alpha}] = [N_{\alpha}]$.
So, the axioms of the theory {\sf\textgreek{lh}} are valid, and
$\auf \{D_{\alpha} : \alpha \in S\}, \bullet\zu$ is a functionally
complete (typed) applicative structure. Since $\Delta$ is
maximally consistent, $D_t$ consists of two elements, which we now
call $0$ and $1$. Furthermore, we may arrange it that $[M_t] = 1$
iff $M_t \in \Delta$. It is then easily checked that
the interpretation of {\mtt\symbol{5}} is complement, and the 
interpretation of {\mtt\symbol{4}} is intersection. Now we treat 
$\pi^{\alpha} := [\GPi^{\alpha}]$. We have to
show that for $a \in D_{\alpha\pf t}$
$\pi^{\alpha} \bullet a = 1$ iff
for every $b \in D_{\alpha}$: $a \bullet b = 1$. Or,
alternatively, $\Pi^{\alpha}M_{\alpha\pf t} \in \Delta$ iff 
$M_{\alpha\pf t}N_{\alpha} \in \Delta$ for every closed
term $N_{\alpha}$. Suppose that $\Pi^{\alpha}M_{\alpha\pf t} \in
\Delta$. Using Lemma~\ref{lem:a14} and the fact that $\Delta$ is
deductively closed, $M_{\alpha\pf t}N_{\alpha} \in \Delta$.
Conversely, assume $M_{\alpha\pf t}N_{\alpha} \in \Delta$ for
every constant term $N_{\alpha}$. Then
$M_{\alpha}(\iota^{\alpha}(\lambda x_{\alpha}.\nicht M_{\alpha\pf
t}x_{\alpha}))$ is a constant term, and it is in $\Delta$.
Moreover, by the Witnessing Lemma, $\Pi^{\alpha}M_{\alpha\pf t}
\in \Delta$. Finally, we have to show that for every $a \in
D_{\alpha\pf t}$: if there is a $b \in D_{\alpha}$ such that $a
\bullet b = 1$ then $a \bullet (\iota^{\alpha} \bullet a) = 1$. 
This means that for $M_{\alpha\pf t}$: if there
is a constant term $N_{\alpha}$ such that $M_{\alpha\pf
t}N_{\alpha} \in \Delta$ then $M_{\alpha\pf
t}(\iota^{\alpha}M_{\alpha\pf t}) \in \Delta$.
This is a consequence of (s17). %%
\newline\mbox{}
\proofend

Now, it follows that $\goth{Hfr}_{\Delta} \vDash N_t$ iff
$N_t \in \Delta$. More generally, let $\beta$ an assignment of
constant terms to variables. Let $M_{\alpha}$ be a term. Write
$M^{\beta}_{\alpha}$ for the result of replacing a free occurrence
of a variable $x_{\gamma}$ by $\beta(x_{\gamma})$. Then
%%%
\begin{equation}
\auf \goth{Hfr}_{\Delta}, \beta\zu \vDash M_{\alpha}
\quad\Dpf\quad M^{\beta}_{\alpha} \in \Delta
\end{equation}
%%
This is shown by induction.
%%
\begin{lem}
\label{lem:wohl}
Let $\Delta_0$ be a consistent set of constant terms. Then there
exists a maximally consistent set $\Delta$ of constant terms
containing $\Delta_0$.
\end{lem}
%%%
\proofbeg%%
Choose a well--ordering $\{N^{\delta} : \delta < \mu\}$ on the set
of constant terms. Define $\Delta_i$ by induction as follows.
$\Delta_{\kappa+1} := \Delta_{\kappa} \cup \{N^{\delta}\}$ if the
latter is consistent. Otherwise, $\Delta_{\kappa+1} :=
\Delta_{\kappa}$. If $\kappa < \mu$ is a limit ordinal,
$\Delta_{\kappa} := \bigcup_{\lambda < \kappa} \Delta_{\lambda}$.
We shall show that $\Delta := \Delta_{\mu}$ is maximally
consistent. Since it contains $\Delta_0$, this will complete the
proof. (a) It is consistent. This is shown inductively. By
assumption $\Delta_0$ is consistent, and if $\Delta_{\kappa}$ is
consistent, then so is $\Delta_{\kappa+1}$. Finally, let $\lambda$
be a limit ordinal and suppose that $\Delta_{\lambda}$ is
inconsistent. Then there is a finite subset $\Gamma$ which is
inconsistent. There exists an ordinal $\kappa < \lambda$ such that
$\Gamma \subseteq \Delta_{\kappa}$. $\Delta_{\kappa}$ is
consistent, contradiction. (b) There is no consistent superset.
Assume that there is a term $M \not\in \Delta$ such that $\Delta
\cup \{M\}$ is consistent. Then for some $\delta$, $M =
N^{\delta}$. Then $\Delta_{\delta} \cup \{N^{\delta}\}$ is
consistent, whence by definition
$N^{\delta} \in \Delta_{\delta+1}$. Contradiction. %%
\proofend
%%%
\begin{thm}[Henkin]
%%%
\index{Henkin, Leon}%%
%%%
(a) A term $N_t$ is a theorem of $\mathsf{STyp}$ iff it is
valid in all Henkin--frames. (b) An equation $M_{\alpha} \neweq
N_{\alpha}$ is a theorem of $\mathsf{STyp}$ iff it holds in
all Henkin--frames iff it is valid in the many sorted 
sense.
\end{thm}
%%%
We sketch how to prove Theorem~\ref{thm:ent}. Let $\auf \Omega, \Xi\zu$ 
be a signature for predicate logic. Define a translation into 
$\mathsf{STyp}$:
%%%
\begin{subequations}
\begin{align}
\label{eq:fdef}
f^{\heartsuit} & := (\lambda x_0)(\lambda x_1)\dotso
	(\lambda x_{\Omega(f)-1})f(x_0, \dotsc, x_{\Omega(f)-1}) \\
\label{eq:rdef}
r^{\heartsuit} & := (\lambda x_0)(\lambda x_1)\dotso
	(\lambda x_{\Xi(r)-1})r(x_0, \dotsc, x_{\Xi(r)-1}) 
\end{align}
\end{subequations}
%%%
This is extended to all formulae. Now we look at the signature for 
$\mathsf{STyp}$ with the constants $f^{\heartsuit}$, $r^{\heartsuit}$ 
of type 
%%
\begin{subequations}
\begin{align}
f^{\heartsuit} : & \ (e \pf (e \pf \dotso (e \pf e) \dotso )) \\
r^{\heartsuit} : & \ (e \pf (e \pf \dotso (e \pf t) \dotso )) 
\end{align}
\end{subequations}
%%
Now, given a first--order model $\GM$, we can construct a Henkin--frame 
for $\GM^{\heartsuit}$ with $D_e = M$ and $D_{\alpha\pf\beta} 
:= D_{\alpha} \pf D_{\beta}$, by interpreting $f^{\heartsuit}$ 
and $r^{\heartsuit}$ as given by \eqref{eq:fdef} and \eqref{eq:rdef}.
%%
\begin{lem}
\label{lem:predtostyp}
Let $\beta$ be a valuation on $\GM$. Extend $\beta$ to 
$\beta^{\heartsuit}$. Then
%%%
\begin{equation}
\auf \GM, \beta\zu \vDash \varphi \quad\Dpf\quad
\auf \GM^{\heartsuit}, \beta\zu \vDash \varphi^{\heartsuit}
\end{equation}
\end{lem}
%%
\begin{lem}
\label{lem:pred}
$\vdash^{\mathsf{STyp}} \varphi^{\heartsuit}$ iff 
$\vdash^{\mathsf{FOL}} \varphi$.
\end{lem}
%%%
Right to left is by induction. Now if $\nvdash^{\mathsf{FOL}}
\varphi$ then there is a model $\auf \GM, \beta\zu \vDash 
\nicht \varphi$, from which we get a Henkin--frame 
$\auf \GM^{\heartsuit}, \beta\zu \nvDash \nicht \varphi^{\heartsuit}$.
The proof of Theorem~\ref{thm:ent} is as follows. Clearly, if
$\varphi$ is derivable it is valid. Suppose that it is not 
derivable. Then $\varphi^{\heartsuit}$ is not derivable   
in $\mathsf{STyp}$. There is a Henkin--frame $\GM \nvDash 
\varphi^{\heartsuit}$. This allows to define a first--order model 
$\GM_{\heartsuit} \nvDash \varphi$.
%%
\vplatz
\exercise
The set of typed $\lambda$--terms is defined over a finite alphabet 
if the set $B$ of basic types is finite. Define from this a 
well--ordering on the set of terms. {\it Remark.} This shows 
that the proof of Lemma~\ref{lem:wohl} does not require the 
use of the Axiom of Choice for obtaining the well--ordering.
%%%
\vplatz
\exercise
Show Lemma~\ref{lem:predtostyp}.
%%%
\vplatz
\exercise
Complete the details of the proof of Theorem~\ref{thm:ent}. 
%%%
\vplatz
\exercise
Let $L$ be a normal modal logic. Show that $\Delta \Vdash_L \chi$ 
iff $\Delta^b \vdash_L \chi$, where $\Delta^b := \{\qu^n \delta : 
\delta \in \Delta, n \in \omega\}$. {\it Hint.} This is analogous 
to Theorem~\ref{thm:genelim}.

 \section{Algebraization}
\label{kap6-4b}
%%%
Now that we have shown completeness with respect to models and
frames, we shall proceed to investigate the possibility of
algebraization of predicate logic and simple type theory. Apart
from methodological reasons, there are also practical reasons for
preferring algebraic models over frames. If $\varphi$ is a
sentence and $\GM$ a model, then either $\GM \vDash \varphi$ or
$\GM \vDash \nicht\varphi$. Hence, the theory of a single model is
maximally consistent, that is, complete. One may argue that this
is as it should be; but notice that the base logic ($\mathsf{FOL}$,
$\mathsf{STyp}$) is not complete --- neither is the knowledge
ordinary people have. Since models are not enough for representing
incomplete theories, something else must step in their place.
These are {\it algebras\/} for some appropriate signature, for the
product of algebras is an algebra again, and the logic of the
product is the intersection of the logics of the factors. Hence,
for every logic there is an adequate algebra. However,
algebraization is not straightforward. The problem is that there
is no notion of binding in algebraic logic. Substitution always is
replacement of an occurrence of a variable by the named string,
there is never a preparatory replacement of variables being
performed. Hence, what creates in fact big problems is those
axioms and rules that employ the notion of a free or bound
variable. In predicate logic this is the axiom (a15). (Unlike 
$\mathsf{PC}$, $\mathsf{FOL}$ has no rule of substitution.)

It was once again Tarski who first noticed the analogy between
modal operators and quantifiers. Consider a language $L$ of first
order logic without quantifiers. We may interpret the atomic
formulae of this language as propositional atoms, and formulae
made from them using the boolean connectives. Then we have a
somewhat more articulate version of our propositional boolean
language. We can now introduce a quantifier $Q$ simply as a unary
operator. For example, {\mtt (\symbol{20}x0)} is a unary operator
on formulae. Given a formula $\varphi$, {\mtt (\symbol{20}x0)$\varphi$}
is again a formula. (Notice that the way we write the formulae is 
somewhat different, but this can easily be accounted for.) In this 
way we get an extended language: a language of formulae extended by 
a single quantifier. Moreover, the laws of {\mtt (\symbol{20}x0)} turn 
the logic exactly into a normal modal logic. The quantifier 
{\mtt (\symbol{21}x0)} then corresponds to $\wD$, the dual of $\qu$. 
Clearly, in order to reach full expressive power of predicate logic 
we need to add infinitely many such operators, one for each variable. 
The resulting algebras are called \textbf{cylindric algebras}. The
principal reference is to \cite{henkinmonktarski:cylindric1}.

We start with the intended models of cylindric algebras. A formula
may be seen as a function from models, that is, pairs $\auf \GM,
\beta\zu$, to $2$, where $\GM$ is a structure and $\beta$ an
assignment of values to the variables.  First of all, we shall
remove the dependency on the structure, which allows us to focus
on the assignments. There is a general first order model for any
complete (= maximal consistent) theory, in which exactly those
sentences are valid that belong to the theory. Moreover, this model
is countable. Suppose a theory $T$ is not complete. Then let $\Delta_i$, 
$i \in I$, be its completions. For each $i \in I$, let $\GM_i$ be the 
canonical structure associated with $\Delta_i$. If $\GA_i$ is the
cylindrical algebra associated with $\GM_i$ (to be defined below),
the algebra associated with $T$ will $\prod_{i \in I} \GA_i$. In this 
way, we may reduce the study to that of a cylindric algebra of a single 
structure.

Take a first order structure $\auf M, \GI\zu$, where $M$ is the
universe and $\GI$ the interpretation function. For simplicity, we
assume that there are no functions. (The reader shall see in the
exercises that there is no loss of expressivity in renouncing
functions.) Let $V := \{x_i : i \in \omega\}$ be the set of
variables. Let $\GV(V;M)$ be the boolean algebra of sets of
functions into $M$. Then for every formula $\varphi$ we
associate the following set of assignments:
%%
\begin{equation}
[\varphi] := \{\beta : \auf \GM, \beta\zu \vDash \varphi\}
\end{equation}
%%
Now, for each number $i$ we assume the following operation $\mathsf{A}_i$.
%%
\begin{equation}
\mathsf{A}_i(S) := \{\beta : \mbox{ for all }\gamma
\sim_{x_i} \beta: \gamma \in S\}
\end{equation}
%%
Then $\mathsf{E}_i(S) := - \mathsf{A}_i(- S)$. (The standard
notation for $\mathsf{E}_i$ is $\mathsf{c}_i$. The letter
$\mathsf{c}$ here is suggestive for `cylindrification'. We have decided
to stay with a more logical notation.) Furthermore, for every pair of
numbers $i, j \in \omega$ we assume the element $\mathsf{d}_{i,j}$.
%%
\begin{equation}
\mathsf{d}_{i,j} := \{\beta :  \beta(x_i) = \beta(x_j)\}
\end{equation}
%%
It is interesting to note that with the help of these elements
substitution can be defined. Namely, put
%%
\begin{equation}
\mathsf{s}^{i}_{j}(x) := 
    \begin{cases}
    x & \text{ if $i = j$,} \\
    \mathsf{E}_i(\mathsf{d}_{i,j} \cap x)
    & \text{ otherwise.}
    \end{cases}
\end{equation}
%%
\begin{lem}
\label{lem:subst}
Let $y$ be a variable distinct from $x$. Then $[y/x]\varphi$
is equivalent with {\mtt (\symbol{21}$x$).(($y$\symbol{61}$%
x$)\symbol{4}$\varphi$)}.
\end{lem}
%%
Thus, equality and quantification alone
can define substitution. The relevance of this observation for
semantics has been nicely explained in
\cite{dresner:tarski}. For example, in applications
it becomes necessary to introduce constants for the relational
symbols. Suppose, namely that {\tt taller} is a binary relation
symbol. Its interpretation is a binary relation on the domain.
If we want to replace the structure by its associated cylindric
algebra, the relation is replaced by an element of that algebra,
namely
%%
\begin{equation}
[\textsf{taller}'(x_0,x_1)] :=
    \{\beta : \auf \beta(x_0), \beta(x_1)\zu \in
    \GI(\mbox{\tt taller})\}
\end{equation}
%%
However, this allows us prima facie only to assess the meaning of
`$x_0$ is taller than $x_1$'. We do not know, for example, what
happens to `$x_2$ is taller than $x_7$'. For that we need the
substitution functions. Now that we have the unary substitution
functions, any finitary substitution becomes definable. In this
particular case,
%%
\begin{equation}
[\textsf{taller}'(x_2,x_7)] = \mathsf{s}^7_1
    \mathsf{s}^2_0 [\textsf{taller}'(x_0,x_1)]\
\end{equation}
%%
Thus, given the definability of substitutions, to define the 
interpretation of $R$ we only need to give the element
$[R(x_0, x_1, \ldots, x_{\Xi(R)-1})]$.

The advantage in using this formulation of predicate logic is
that it can be axiomatized using equations. It is directly
verified that the equations listed in the next definition
are valid in the intended structures.
%%%
\begin{defn}
A \textbf{cylindric algebra} 
%%%
\index{cylindric algebra}%%
%%%
of dimension $\kappa$, $\kappa$ a cardinal number, is a structure
%%
\begin{equation}
\GA = \auf A, 0, 1, -, \cap, \cup, \auf \mathsf{E}_{\lambda}:
    \lambda < \kappa\zu, \auf \mathsf{d}_{\lambda, \mu} :
    \lambda, \mu < \kappa\zu\zu
\end{equation}
%%
such that the following holds for all $x, y \in A$ and $\lambda,
\mu, \nu < \kappa$:
\begin{equation}
\begin{tabular}{ll}
\mbox{\rm (ca1)} &
    $\auf A, 0, 1, -, \cap, \cup\zu$ is a boolean algebra. \\
\mbox{\rm (ca2)} &
    $\mathsf{E}_{\lambda} 0 = 0$. \\
\mbox{\rm (ca3)} &
    $x \cup \mathsf{E}_{\lambda} x = \mathsf{E}_{\lambda} x$.
    \\
\mbox{\rm (ca4)} &
    $\mathsf{E}_{\lambda}(x \cap \mathsf{E}_{\lambda} y) =
    \mathsf{E}_{\lambda} x \cap \mathsf{E}_{\lambda} y$. \\
\mbox{\rm (ca5)} &
    $\mathsf{E}_{\lambda}\mathsf{E}_{\mu} x =
    \mathsf{E}_{\mu}\mathsf{E}_{\lambda} x$. \\
\mbox{\rm (ca6)} &
    $\mathsf{d}_{\lambda,\mu} = 1$. \\
\mbox{\rm (ca7)} &
    If $\lambda \neq \mu, \nu$ then
        $\mathsf{d}_{\mu,\nu} =
    \mathsf{E}_{\lambda}(\mathsf{d}_{\mu,\lambda}
        \cap \mathsf{d}_{\lambda, \nu})$. \\
\mbox{\rm (ca8)} &
    If $\lambda \neq \mu$ then
    $\mathsf{E}_{\lambda}(\mathsf{d}_{\lambda,\mu} \cap x)
    \cap \mathsf{E}_{\lambda}(\mathsf{d}_{\lambda,\mu}
    \cap (-x)) = 0$.
\end{tabular}
\end{equation}
\end{defn}
%%
We shall see that this definition allows to capture the effect of
the axioms above, with the exception of (a15). Notice first the
following. $\dpf$ is a congruence in $\mathsf{FOL}$ as well. For
if $\varphi \dpf \chi$ is a tautology then so is $(\exists
x_i)\varphi \dpf (\exists x_i)\chi$. Hence, we can encode the
axioms of $\mathsf{FOL}$ as equations of the form $\varphi \dpf
\top$ as long as no side condition concerning free or bound
occurrences is present. We shall not go into the details. For
example, in $\varphi = (\forall x)\chi$ $x$ occurs trivially bound.
It remains to treat the rule (gen). It corresponds to the rule
(mn) of modal logic. In equational logic, it is implicit anyway.
For if $x \doteq y$ then $O(x) \doteq O(y)$ for any unary operator
$O$.
%%%
\begin{defn}
Let $\GA$ be a cylindric algebra of dimension $\kappa$, and
$a \in A$. Then
%%
\begin{equation}
\Delta a := \{i : i < \kappa, \mathsf{E}_i a \neq a\}
\end{equation}
%%
is called the \textbf{dimension of} $a$. 
%%%
\index{dimension}%%%
\index{$\Delta a$}%%%
%%%
$\GA$ is said to be \textbf{locally finite dimensional} 
%%%
\index{cylindric algebra!locally finite dimensional}%%%
%%%
if $|\Delta a| < \aleph_0$ for all $a \in A$.
\end{defn}
%%
A particular example of a cylindric algebra is
$\CL_{\kappa}/\!\equiv$, $\CL_{\kappa}$ the formulae of pure
equality based on the variables $\mbox{\tt x}_i$, $i < \kappa$,
and $\varphi \equiv \chi$ iff $\varphi \dpf \chi$ is a
theorem. (If additional function or relation symbols are needed,
they can be added with little change to the theory.) This algebra
is locally finite dimensional and is freely $\kappa$--generated.

The second approach we are going to elaborate is one which takes
substitutions as basic functions. For predicate logic this has
been proposed by Halmos~\shortcite{halmos:polyadic}, 
%%%
\index{Halmos, Paul}%%%
%%%
but most people credit Quine~\shortcite{quine:variables} 
%%%
\index{Quine, Willard van Orman}%%%
%%%
for this idea. For an exposition see \cite{pigozzisalibra:vb}.  
Basically, Halmos takes substitution as primitive. This has
certain advantages that will become apparent soon. Let us agree
that the index set is $\kappa$, again called the \textbf{dimension}.
%%%
\index{dimension}%%%
%%%
Halmos defines operations $\mathsf{S}(\tau)$ for every function
$\tau \colon \kappa \pf \kappa$ such that there are only finitely many
$i$ such that $\tau(i) \neq i$. The theory of such functions is
axiomatized independently of quantification. Now, for every finite
set $I \subset \kappa$ Halmos admits an operator $\mathsf{E}(I)$, 
which represents quantification over each of variables $\mbox{\tt x}_i$, 
where $i \in I$. If $I = \varnothing$, $\mathsf{E}(I)$ is the 
identity, otherwise $\mathsf{E}(I)(\mathsf{E}(K)x) = 
\mathsf{E}(I \cup K)x$. Thus, it is immediately clear
that the ordinary quantifiers $\mathsf{E}(\{i\})$ suffice to
generate all the others. However, the axiomatization is somewhat
easier with the polyadic quantifiers. Another problem, noted in
\cite{sainthompson:scheme}, is the fact that the axioms for polyadic 
algebras cannot be schematized using letters for elements of the 
index set. However, Sain and Thompson \shortcite{sainthompson:scheme} 
%%%
\index{Sain, Ildik\'o}%%%
\index{Thompson, Richard S.}%%%
%%%
also note that the addition of transpositions is actually enough to 
generate the same functions. To see this, here are some definitions.
%%%
\begin{defn}
%%%
\index{support}%%
\index{$\supp(\pi)$}%%
\index{transformation}%%
\index{permutation}%%
\index{transposition}%%
%%
Let $I$ be a set and $\pi \colon I \pf I$. The \textbf{support} of 
$\pi$, $\supp(\pi)$, is the set $\{i : \pi(i) \neq i\}$. A function 
of finite support is called a \textbf{transformation}.
$\pi$ is called a \textbf{permutation of} $I$ if it is bijective. If
the support contains exactly two elements, $\pi$ is called a 
\textbf{transposition}.
\end{defn}
%%%
The functions whose support has at most two elements are of
special interest. Notice first the case when $\supp(\pi)$
has exactly one element. In that case, $\pi$ is called an 
\textbf{elementary substitution}. Then there are $i, j \in I$ such that
$\pi(i) = j$ and $\pi(k) = k$ if $k \neq i$. If $i$ and $j$ are in
$I$, then denote by $(i,j)$ the permutation that sends $i$ to $j$
and $j$ to $i$. Denote by $[i,j]$ the elementary substitution that
sends $i$ to $j$.
%%%
\begin{prop}
\label{prop:support}
Let $I$ be a set. The set $\Phi(I)$ of functions $\pi \colon I \pf I$
of finite support is closed under concatenation. Moreover,
$\Phi(I)$ is generated by the elementary substitutions and the
transpositions.
\end{prop}
%%%
The proof of this theorem is left to the reader. So,  it is enough 
if we take only functions corresponding to $[i,j]$ and $(i,j)$. The 
functions of the first kind are already known: these are 
the $\mathsf{s}^j_i$. For the functions of the second kind, 
write $\mathsf{p}_{i,j}$. Sain and Thompson effectively axiomatize
cylindric algebras that have these additional operations. They
call them \textbf{finitary polyadic algebras}. 
%%%
\index{polyadic algebra!finitary}%%%
%%%
Notice also the following useful fact, which we also leave as an exercise.
%%%
\begin{prop}
\label{prop:approx} Let $\pi \colon I \pf I$ be an arbitrary function,
and $M \subseteq I$ finite. Then there is a product $\gamma$ of
elementary substitutions such that $\gamma \restriction M = \pi
\restriction M$.
\end{prop}
%%%
This theorem is both stronger and weaker than the previous one. It
is stronger because it does not assume $\pi$ to have finite
support. On the other hand, $\gamma$ only approximates $\pi$ on a
given finite set. (The reader may take notice of the fact that
there is no sequence of elementary substitutions that equals the
transformation $(0\; 1)$ on $\omega$. However, we can approximate
it on any finite subset.)

Rather than developing this in detail for predicate logic we shall
do it for the typed $\lambda$--calculus, as the latter is more rich 
and allows to encode arbitrarily complex abstraction (for example, by
way of using $\mathsf{STyp}$). Before we embark on the project let us 
outline the problems that we have to deal with. Evidently, we wish 
to provide an algebraic axiomatization that is equivalent to the 
rules \eqref{eq:lea} -- \eqref{eq:leg} and \eqref{eq:lei}. First, 
the signature we shall choose has function application and abstraction 
as its primitives. However, we cannot have a single abstraction 
symbol corresponding to $\tlambda$, rather, for each variable 
(and each type) we must assume a different unary function symbol 
$\tlambda_i$, corresponding to $\tlambda \mbox{\tt x}_i$. Now, 
\eqref{eq:lea} -- \eqref{eq:lee} and \eqref{eq:lei} are already 
built into the Birkhoff--Calculus. Hence, our only concern are the 
rules of conversion. These are, however, quite 
tricky. Notice first that the equations make use of the substitution 
operation $[N/x]$. This operation is in turn defined with the 
definitions \eqref{eq:suba} -- \eqref{eq:subf}. Already \eqref{eq:suba}
for $N = \mbox{\tt x}_i$ can only be written down if we have an operation 
that performs an elementary substitution. So, we have to add the unary 
functions $\mathsf{s}_{i,j}$, to denote this substitution. Additionally, 
\eqref{eq:suba} needs to be broken down into an inductive definition.
To make this work, we need to add correlates of the variables. 
That is, we add zeroary function symbols $\mbox{\tt x}_i$ for 
every $i \in \omega$. Symbols for the functions $\mathsf{p}_{i,j}$ permuting 
$i$ and $j$ will also be added to be able to say that the variables 
all range over the same set. Unfortunately, this is not all. Notice 
that \eqref{eq:lef} is not simply an equation: it has a side condition, 
namely that $y$ is not free in $M$. In order to turn this into an 
equation we must introduce {\it sorts}, which will help us keep track 
of the free variables. Every term will end up having a unique sort, 
which will be the set of $i$ such that $\mbox{\tt x}_i$ is free in 
it. $B$ is the set of basic types. Call a member of 
$\CI := \Typ_{\pf}(B) \times \omega$ an \textbf{index}. 
%%%
\index{index}%%
\index{type}%%%
%%%%
If $\auf \alpha, i\zu$ is an index, $\alpha$ is its \textbf{type} 
and $i$ its numeral. Let $\CF$ be the set of pairs 
$\auf \alpha, \delta\zu$ where $\alpha$ is a type and $\delta$ 
a finite set of indices.

We now start with the signature. Let $\delta$ and $\delta'$ be finite 
sets of indices, $\alpha$, $\beta$ types, and $\iota = \auf \gamma, i\zu$, 
$\kappa = \auf \gamma', i'\zu$ indices. We list the symbols together with 
their type:
%%%
\begin{subequations}
\begin{align}
\mbox{\tt x}_{\iota} & :  \auf\auf\gamma, \{\iota\}\zu\zu  \\
\tlambda^{\auf\beta,\delta\zu}_{\iota} & : 
\auf \auf \beta, \delta\zu, \auf \gamma\pf\beta, \delta - \{\iota\}\zu\zu \\
\mbox{\tt p}_{\iota,\kappa}^{\auf \alpha, \delta\zu} & :
\auf \auf \alpha, \delta\zu, \auf \alpha, (\iota, \kappa)[\delta]\zu\zu\qquad 
\text{where $\gamma = \gamma'$} \\
\mbox{\tt s}_{\iota, \kappa}^{\auf \alpha, \delta\zu} & :  
\auf \auf \alpha, \delta\zu, \auf \alpha, [\iota, \kappa][\delta]\zu\zu \\
\bullet^{\auf \alpha\pf\beta, \delta\zu, \auf \alpha, \delta'\zu} 
   & : \auf \auf \alpha\pf\beta, \delta\zu, 
\auf \alpha, \delta'\zu, \auf \beta, \delta \cup \delta'\zu\zu
\end{align}
\end{subequations}
%%%
Here $(\iota, \kappa)[\delta]$ is the result of exchanging $\iota$ and 
$\kappa$ in $\delta$ and  $[\iota, \kappa][\delta]$ is the result of 
replacing $\kappa$ by $\iota$ in $\delta$. 
Notice that $\mbox{\tt x}_{\iota}$ is now a constant! 
We may also have additional functional symbols stemming from an 
underlying (sorted) algebraic signature. The reader is asked to 
verify that nothing is lost if we assume that additional function 
symbols only have arity 0, and signature $\auf \auf \alpha, 
\varnothing\zu\zu$ for a suitable $\alpha$. This greatly simplifies 
the presentation of the axioms. 

This defines the language. Notice that in addition to the constants 
$\mbox{\tt x}_{\iota}$ we also have variables $x_i^{\sigma}$ for each 
sort $\sigma$. The former represent the variable $\mbox{\tt x}_i$ (where 
$\iota = \auf \gamma, i\zu$) of the $\lambda$--calculus and the latter 
range over terms of sort $\sigma$. Now, in order to keep the notation 
perspicuous we shall drop the sorts whenever possible. That this is 
possible is assured by the following fact. If $t$ is a term without 
variables, and we hide all the sorts except for those of the variables, 
still we can recover the sort of the term uniquely. For the types this 
is clear, for the second component we observe the following. 
%%%%
\begin{lem}
If a term $t$ has sort $\auf \alpha, \delta\zu$ then
$\fr(t) = \{\mbox{\tt x}_{\iota} : \iota \in \delta\}$.
\end{lem}
%%
The proof of this fact is an easy induction.

\nocite{pigozzisalibra:vb}
For the presentation of the equations we therefore omit the sorts.
They have in fact only been introduced to ensure that we may talk 
about the set of free variables of a term. The equations (vb1) --- 
(vb6) of Table~\ref{tab:vb} characterize the behaviour of the 
substitution and permutation function with respect to the 
indices. We assume that $\iota$, $\mu$, $\nu$ all have the same 
type. (We are dropping the superscripts indicating the sort.)
The equations (vb7) -- (vb11) characterize the pure binding 
by the unary operators $\tlambda_{\iota}$. 
%%
\begin{table}
\caption{The Variable Binding Calculus $\mathsf{VB}$}
\label{tab:vb}
$$\begin{array}{ll@{\quad \doteq \quad}l}
\mbox{\rm (vb1)} & 
    \mbox{\tt s}_{\iota, \mu}\mbox{\tt x}_{\nu} & \left\{\begin{array}{ll}
    \mbox{\tt x}_{\iota} & \mbox{if } \nu \in \{\iota,\mu\}, \\
    \mbox{\tt x}_{\nu}    & \mbox{otherwise.}
    \end{array}\right. \\
    \multicolumn{3}{l}{}
    \\
\mbox{\rm (vb2)} & 
\mbox{\tt p}_{\iota,\mu}\mbox{\tt x}_{\nu} & \left\{\begin{array}{ll}
    \mbox{\tt x}_{\iota} & \mbox{if } \nu = \mu, \\
    \mbox{\tt x}_{\mu} & \mbox{if } \nu = \iota, \\
    \mbox{\tt x}_{\nu}    & \mbox{otherwise.}
    \end{array}\right. \\
    \multicolumn{3}{l}{}
    \\
\mbox{\rm (vb3)} & 
\mbox{\tt p}_{\iota,\iota}x & x \\
\mbox{\rm (vb4)} & 
\mbox{\tt p}_{\iota,\mu}x & \mbox{\tt p}_{\mu,\iota}x \\
\mbox{\rm (vb5)} & 
\mbox{\tt p}_{\iota,\mu}\mbox{\tt p}_{\mu,\nu}x & 
	\mbox{\tt p}_{\mu,\nu}
    \mbox{\tt p}_{\iota,\mu} x \qquad\quad\; \mbox{if } 
	|\{\iota, \mu, \nu\}| = 3
\\
\mbox{\rm (vb6)} & 
\mbox{\tt p}_{\iota, \mu}\mbox{\tt s}_{\mu, \iota} x & 
	\mbox{\tt s}_{\iota, \mu} x \\
\mbox{\rm (vb7)} & 
\mbox{\tt s}_{\iota,\mu}\tlambda_{\nu} x &
    \tlambda_{\nu} \mbox{\tt s}_{\iota,\mu} x
    \qquad\qquad\mbox{ if } |\{\iota,\mu,\nu\}| = 3 \\
\mbox{\rm (vb8)} & 
\mbox{\tt s}_{\mu,\iota} \tlambda_{\iota} x & \tlambda_{\iota} x \\
\mbox{\rm (vb9)} & 
\mbox{\tt p}_{\iota,\mu}\tlambda_{\nu} x &
    \tlambda_{\nu} \mbox{\tt p}_{\iota,\mu} x
    \qquad\qquad\mbox{ if } |\{\iota, \mu, \nu\}| = 3 \\
\mbox{\rm (vb10)} & 
\mbox{\tt p}_{\mu,\iota} \tlambda_{\mu} x & 
	\tlambda_{\iota} \mbox{\tt p}_{\mu,\iota} x \\
\mbox{\rm (vb11)} & 
\tlambda_{\iota}(y \bullet \mbox{\tt x}_{\iota}) & y \\
\mbox{\rm (vb12)} & 
(\tlambda_{\iota}(x \bullet y)) \bullet z &
    ((\tlambda_{\iota} x)\bullet z) \bullet ((\tlambda_{\iota}
    y) \bullet z) \\
\multicolumn{3}{l}{} \\
\mbox{\rm (vb13)} & 
(\tlambda_{\iota}\tlambda_{\mu} x) \bullet y & 
	\left\{\begin{array}{ll}
		\tlambda_{\mu} ((\tlambda_{\iota} x) \bullet y) & 
		\mbox{if }{\iota} \neq {\mu}, \mbox{\tt x}_{\mu} 
		\not\in \fr(y) \\
	\tlambda_{\iota} x & \mbox{if }{\iota} = {\mu} 
	\end{array}\right. \\
\multicolumn{3}{l}{} \\
\mbox{\rm (vb14)} & 
(\tlambda_{\iota} \mbox{\tt x}_{\mu}) \bullet y & 
	\left\{\begin{array}{ll}
		y & \mbox{if }{\iota} = {\mu} \\
		\mbox{\tt x}_{\mu} & \mbox{if }{\iota} \neq {\mu}
	\end{array}\right.
\end{array}$$
\end{table}
%%
The set of equations is invariant under permutation of the indices. 
Moreover, we can derive the invariance under replacement of bound 
variables, for example. Thus, effectively, once the interpretation 
of $\tlambda_{\auf \alpha,0\zu}$ is known, the interpretation of all 
$\tlambda_{\auf \alpha, i\zu}$, $i\in \omega$, is known as well. 
For using the equations we can derive that $\mbox{\tt p}_{\auf \alpha, 
i\zu, \auf \alpha, 0\zu}$ is the inverse of $\mbox{\tt p}_{\auf \alpha, 
0\zu, \auf \alpha, i\zu}$, 
and so 
%%
\begin{equation}
\tlambda_{\auf \alpha, i\zu} x \doteq \mbox{\tt p}_{\auf \alpha, 0\zu,
\auf\alpha, i\zu} \tlambda_{\auf \alpha, 0\zu} \mbox{\tt p}_{\auf \alpha, 
i\zu, \auf \alpha, 0\zu}
x
\end{equation}
%%
The equivalent of \eqref{eq:lef} now turns out to be derivable. 
However, we still need to take care of \eqref{eq:leg}. Since we 
do not dispose of the full substitution $[N/x]$, we need to break 
down \eqref{eq:leg} into an inductive definition (vb12) --- (vb14).
The condition $\mbox{\tt x}_{\mu} \not\in \fr(y)$ is 
just a shorthand; all it says is that we take only those 
equations where the term $y$ has sort $\auf \alpha, \delta\zu$ and 
${\mu} \not\in \delta$. Notice that the disjunction in (vb13) is 
not complete. From these equations we deduce that 
$\mbox{\tt x}_{\auf \alpha, k\zu} \doteq \mbox{\tt p}_{\auf \alpha, k\zu,
\auf \alpha, 0\zu} \mbox{\tt x}_{\auf \alpha, 0\zu}$, 
so we could in principle dispense with all but one variable symbol 
for each type. 

The theory of sorted algebras now provides us with a class of models 
which is characteristic for that theory. We shall not spell out a 
proof that these models are equivalent to models of the $\lambda$--calculus 
in a sense made to be precise. Rather, we shall outline a procedure 
that turns an $\Omega$--algebra into a model of the above equations.
Start with a signature $\auf F, \Omega\zu$, sorted or unsorted. For ease 
of presentation let it be sorted. Then the set $B$ of basic types is the 
set of sorts. Let $\mathsf{Eq}_{\Omega}$ be the equational theory 
of the functions from the signature alone. For complex types, put 
$A_{\alpha\pf\beta} := A_{\alpha} \pf A_{\beta}$. Now transform 
the original signature into a new signature $\Omega' = \auf F, 
\Omega'\zu$ where $\Omega'(f) = 0$ for all $f \in F$. Namely, for 
$f \colon \prod_{i < n} A_{\sigma_i} \pf A_{\tau}$ set 
%%
\index{$f^{\star}$}%%
%%%%
\begin{equation}
f^{\star} := \lambda x_{\auf \sigma_{n-1}, n-1\zu}.\dotsb .\lambda 
x_{\auf \sigma_0, 0\zu}. f^{\GA}(x_{\auf \sigma_{n-1}, n-1\zu},\dotsc, 
x_{\auf \sigma_0, 0\zu})
\end{equation}
%%
This is an element of $A_{\pi}$ where 
%%%
\begin{equation}
\pi := (\sigma_0 \pf (\sigma_1 \pf \dotsb (\sigma_{n -1} 
\pf \tau)\dotsb ))
\end{equation}
%%%
We blow up the types in the way described above. 
This describes the transition from the signature $\Omega$ to a new 
signature $\Omega^{\lambda}$. The original equations are turned into 
equations over $\Omega^{\lambda}$ as follows. 
%%%
\begin{subequations}
\begin{align}
\mbox{\tt x}_{\alpha, i}^{\lambda} & := \mbox{\tt x}_{\auf \alpha, i\zu} \\
(f(\vec{s}))^{\lambda} & := 
	(\dotsb ((f^{\ast} \bullet s_0^{\lambda}) \bullet 
s_1^{\lambda}) \bullet \dotsb \bullet s^{\lambda}_{n-1}) \\
(s \doteq t)^{\lambda} & := s^{\lambda} \doteq t^{\lambda} 
\end{align}
\end{subequations}
%%%
Next, given an $\Omega$--theory, $T$, let $T^{\lambda}$ be the 
translation of $T$, with the postulates (vb1) -- (vb14) added. 
It should be easy to see that if $T \vDash s \boldsymbol{\doteq} t$ 
then also $T^{\lambda} \vDash s^{\lambda} \boldsymbol{\doteq} t^{\lambda}$. 
For the converse we provide a general model construction that for each 
multisorted $\Omega$--structure for $T$ gives a multisorted 
$\Omega^{\lambda}$--structure for $T^{\lambda}$ in which that 
equation fails. 

An \textbf{environment} 
%%%
\index{environment}%%
%%%
is a function $\beta$ from $\CI = \CS \times \omega$ 
($\CS = \Typ_{\pf}(B)$) into $\bigcup \auf A_{\alpha} : \alpha \in \CS\zu$ 
such that for every index $\auf i, \alpha\zu$, 
$\beta(\auf \alpha, i\zu) \in A_{\alpha}$. We denote the set of 
environments by $\CE$. Now let $C_{\auf \alpha, \delta\zu}$
be the set of functions from $\CE$ to $A_{\alpha}$ which depend 
at most on $\delta$. That is to say, if $\beta$ and $\beta'$ are 
environments such that for all $\iota \in \delta$, $\beta(\iota) = 
\beta'(\iota)$, and if $f \in C_{\auf \alpha, \delta\zu}$ then 
$f(\beta) = f(\beta')$.

The constant $f$ is now interpreted by the function 
$[f^{\star}] \colon \beta \mapsto f^{\star}$. For the `variables' 
we put $[\mbox{\tt x}_{\iota}] \colon \beta \mapsto \beta(\iota)$. 
A transformation $\tau \colon \CI \pf \CI$ naturally induces a map 
$\wht{\tau} \colon \CE \pf \CE \colon \beta \mapsto \beta \circ \tau$. 
Further,
%%
\begin{equation}
\begin{split}
\wht{\tau \circ \sigma}(\beta) &
= \beta \circ (\tau \circ \sigma) \\
&
= (\beta \circ \tau) \circ \sigma \\
&
= \wht{\sigma}(\wht{\tau}(\beta)) \\
&
= (\wht{\sigma} \circ \wht{\tau})(\beta)
\end{split}
\end{equation}
%%
Let $\sigma = \auf \alpha, \delta\zu$, $\tau = \auf \alpha, 
[\iota, \mu][\delta]\zu$ and $\upsilon = \auf \alpha, 
(\iota, \mu)[\delta]\zu$.
%%
\begin{align}
\mbox{}[\mbox{\tt s}^{\sigma}_{\iota, \mu}] & \colon 
C_{\sigma} \pf C_{\tau} \colon f \mapsto f \circ \wht{[\iota, \mu]} \\
\mbox{}[\mbox{\tt p}^{\sigma}_{\iota, \mu}] & \colon 
C_{\sigma} \pf C_{\upsilon} \colon f \mapsto f \circ \wht{(\iota, \mu)}
\end{align}
%%
Next, $\bullet^{\auf \alpha\pf\beta, \delta\zu, \auf \alpha, \delta'\zu}$ 
is interpreted as follows.
%%
\begin{multline}
[\bullet^{\auf \alpha\pf\beta, \delta\zu, \auf \alpha, \delta'\zu}] \colon
    C_{\auf \alpha, \delta\zu} \times C_{\auf \beta, \delta'\zu} 
	\pf C_{\auf \alpha, \delta\cup\delta'\zu} \colon \\
	\qquad\qquad
	\auf f, g\zu \mapsto
    \{\auf \beta, f(\beta) \bullet g(\beta)\zu : \beta \in \CE\}
\end{multline}
%%
Finally, we define abstraction. Let $\iota = \auf \gamma, i\zu$.
%%
\begin{multline}
[\tlambda^{\auf \alpha, \delta\zu}_{\iota}] \colon 
	C_{\auf \alpha, \delta\zu} \pf C_{\auf \gamma\pf\beta, 
	\delta - \{\iota\}\zu} \colon \\
\qquad\qquad 
\Gx \mapsto
    \{\auf \beta, \{\auf y, f([y/\beta(\iota)]\beta)\zu 
	: y \in A_{\gamma}\}\zu : \beta \in \CE\}
\end{multline}
%%
It takes some time to digest this definition. Basically, given 
$f$, $g_{f}(\beta) := \{\auf y, f([y/\beta(\iota)]%
\beta)\zu : y \in A_{\gamma}\}$ is a function from $A_{\gamma}$ to 
$A_{\alpha}$ with parameter $\beta \in \CE$. Hence it is a member of 
$A_{\gamma\pf\alpha}$. It assigns to $y$ the value of $f$ on 
$\beta'$, which is identical to $\beta$ except that now 
$\beta(\iota)$ is replaced by $y$. This is the abstraction from $y$. 
Finally, for each $f \in C_{\auf \alpha, \delta\zu}$, 
$[\tlambda^{\auf \alpha, \delta\zu}_{\iota}](f)$ assigns to 
$\beta \in \CE$ the value $g_{f}(\beta)$. (Notice 
that the role of abstraction is now taken over by the set 
formation operator $\{x : \quad\}$.)
%%%
\begin{thm}
Let $\Omega$ a multisorted signature, and $T$ an equational theory over 
$\Omega$. Furthermore, let $\Omega^{\lambda}$ be the signature of 
the $\lambda$--calculus with 0--ary constants $f^{\star}$ for every 
$f \in F$. The theory $T^{\lambda}$ consisting of the translation of 
$T$ and the equations (vb1) --- (vb14) is conservative over $T$. This 
means that an equation $s \doteq t$ valid in the $\Omega$--algebras 
satisfying $T$ iff its translation is valid in all 
$\Omega^{\lambda}$--algebras satisfying $T^{\lambda}$. 
\end{thm}

{\it Notes on this section.} The theory of cylindric algebras has
given rise to a number of difficult problems. First of all, the
axioms shown above do not fully characterize the cylindric
algebras that are representable, that is to say, have as their
domain $U^{\kappa}$, $\kappa$ the dimension, and where relation
variables range over $n$--ary relations over $U$. Thus, although
this kind of cylindric algebra was the motivating example, the
equations do not fully characterize it. As Donald
Monk~\shortcite{monk:cylindric} 
%%%
\index{Monk, Donald}%%%
%%%
has shown, there is no finite set of
schemes (equations using variables for members of set of variable
indices) axiomatizing the class of representable cylindric
algebras of dimension $\kappa$ if $\kappa \geq \aleph_0$;
moreover, for finite $\kappa$, the class of representable algebras
is not finitely axiomatizable. J.~S.~Johnson has shown in
%%%
\index{Johnson, J.~S.}%%%
%%%
\cite{johnson:fpa} an analogue of the second result for polyadic
algebras, Sain and Thompson \shortcite{sainthompson:scheme} 
%%%
\index{Sain, Ildik\'o}%%%
\index{Thompson, Richard S.}%%%
%%%%
an analogue of the first.

The model construction for the model of the $\lambda$--calculus is
called a syntactical model in \cite{barendregt:lambda}. It is due 
to Hindley and Longo from \cite{hindleylongo:models}. 
%%%
\index{Hindley, J.~R.}%%%
\index{Longo, G.}%%
%%%
The approach of using functions from the set of variables into the algebra 
as the carrier set is called a {\it functional environment model}, 
and has been devised by Koymans (see \cite{koymans:lambda}). 
%%%
\index{Koymans, J.~P.~C.}%%%
%%%
A good overview over the different types of models is found in 
\cite{meyer:model} and \cite{koymans:lambda}.
%%
\vplatz
\exercise
For $f$ an $n$--ary function symbol let $R_f$ be an $n+1$--ary
relation symbol. Define a translation from terms to formulae as
follows. First, for a term $t$ let $x_t$ be a variable such
that $x_t \neq x_s$ whenever $s \neq t$.
%%
\begin{subequations}
\begin{align}
(\mbox{\mtt x$_i$})^{\dagger} & := \mbox{\mtt x$_{x_i}$\symbol{61}x$_{x_i}$} \\
f(t_0,\dotsc, t_{n-1})^{\dagger} & :=
    R_f(x_{t_0}, \dotsc, x_{t_{n-1}},
    x_{f(\vec{t})}) \und \gund_{i < n} t_i^{\dagger}
\end{align}
\end{subequations}
%%
Finally, extend this to formulae as follows.
%%
\begin{subequations}
\begin{align}
R(t_0, \dotsc, t_{n-1})^{\dagger} & := R(x_{t_0}, \dotsc, x_{t_{n-1}})
    \und\gund_{i < n} t_i^{\dagger} \\
\mbox{\mtt (\symbol{5}$\varphi$)}^{\dagger} & :=
    \mbox{\mtt (\symbol{5}$\varphi^{\dagger}$)} \\
\mbox{\mtt ($\varphi$\symbol{4}$\chi$)}^{\dagger} & :=
    \mbox{\mtt ($\varphi^{\dagger}$\symbol{4}$\chi^{\dagger}$)} \\
(\mbox{\mtt (\symbol{21}$x$)$\varphi$})^{\dagger} & :=
    \mbox{\mtt (\symbol{21}$x$)$\varphi^{\dagger}$}
\end{align}
\end{subequations}
%%
Now, let $\GM = \auf M, \Pi, \GI\zu$ be a signature. We replace
the function symbols by relation symbols, and let $\GI^+$ be the
extension of $\GI$ such that
%%
\begin{equation}
\GI^+(R_f) = \{\auf \vec{x},y\zu \in M^{\Xi(f)+1} :
\Pi(f)(\vec{x}) = y\}
\end{equation}
%%
Then put $\GM^{\dagger} := \auf M, \GI^+\zu$.
Show that $\auf \GM, \beta\zu \vDash \varphi$
iff $\auf \GM^+, \beta\zu \vDash \varphi^{\dagger}$.
%%%
%%
\vplatz 
\exercise 
Show that if $\GA$ is a cylindric algebra of
dimension $\kappa$, every $\mathsf{E}_{\lambda}$, $\lambda <
\kappa$, satisfies the axioms of $\mathsf{S5}$. Moreover, show that if
$\GA \vDash \varphi$ then $\GA \vDash \nicht\mathsf{E}_{\lambda}
\nicht\varphi$.
%%
\vplatz
\exercise
Prove Lemma~\ref{lem:subst}.
%%
\vplatz
\exercise
Show Proposition~\ref{prop:approx}.
%%
\vplatz
\exercise
Show that $\CL_{\kappa}/\!\equiv$ is a cylindric algebra of
dimension $\kappa$ and that it is locally finite dimensional.
%%%
\vplatz
\exercise
Prove Proposition~\ref{prop:support}.

 \section{Montague Semantics II}
\index{Montague Semantics}%%
\label{kap6-2}
%
%
%
This section deals with the problem of providing a language with a 
compositional semantics. The problem is to say, which languages that 
are weakly context free are also strongly context free. The principal 
result of this section is that if a language is strongly context free, 
it can be given a compositional interpretation based on an AB--grammar. 
Recall that there are three kinds of languages: languages as sets of 
strings, interpreted languages, and finally, systems of signs. A linear 
system of signs is a subset of $A^{\ast} \times C \times M$, where $C$ 
is a set of categories.
%%%
\begin{defn}
\label{defn:cfsg}
%%%
\index{sign grammar!context free}%%
\index{sign system!context free}%%
\index{sign grammar!quasi context free}%%
%%%
A linear sign grammar is \textbf{context free} if (a) $C$
is finite, (b) if $f$ is a mode of arity $n > 0$ then 
$f^{\varepsilon}(x_0,\dotsc,x_{n-1}) := \prod_{i< n} x_i$, 
(c) $f^{\mu}(m_0, \dotsc, m_{n-1})$ is defined if there exist
derivable signs $\sigma_i= \auf e_i, c_i, m_i\zu$, $i < n$, such
that $f^{\gamma}(\vec{c})$ is defined and (d) if 
$f \neq g$ then $f^{\gamma} \neq g^{\gamma}$. If only (a) --- 
(c) are satisfied, the grammar is \textbf{quasi context free}. 
$\Sigma$ is (\textbf{quasi}) \textbf{context free} if it is 
generated by a (quasi) context free linear sign grammar.
\end{defn}
%%%
This definition is somewhat involved. (a) says that if $f$
is an $n$--ary mode, $f^{\gamma}$ can be represented by a list of
$n$--ary immediate dominance rules. It conjunction with (b)
we get that we have a finite list of context free rules. Condition
(c) says that the semantics does not add any complexity to this
by introducing partiality. Finally, (d) ensures that the rules
of the CFG uniquely define the modes. (For we could in principle 
have two modes which reduce to the same phrase structure rule.) 
The reader may verify the following simple fact.
%%
\begin{prop}
Suppose that $\Sigma$ is a context free linear system of signs.
Then the string language of $\Sigma$ is context free.
\end{prop}
%%%
An interpreted string language is a subset $\CI$ of $A^{\ast} 
\times M$ where $M$ is the set of (possible) (sentence) meanings. 
The corresponding string language is $S(\CI)$. An interpreted 
language is \textbf{weakly context free} iff the string language 
is. 
%%%
\index{interpreted language!strongly context free}%%
\index{interpreted language!weakly context free}%%
%%%%
\begin{defn}
An interpreted language $\CI$ is \textbf{strongly context free}
if there is a context free linear system of signs $\Sigma$ and
a category {\tt S} such that $\mbox{\tt S}(\Sigma) = \CI$.
\end{defn}
%%%
For example, let $L$ be the set of declarative sentences of English.
$M$ is arbitrary. We take the meanings of declarative sentences to be
truth--values, here $0$ or $1$ (but see Section~\ref{kap6}.\ref{kap4x7}). 
A somewhat more refined 
approach is to let the meanings be functions from contexts to truth 
values. Next, we shall also specify what it means for a system of
signs to be context free.

Obviously, a linear context free system of signs defines a strongly
context free interpreted language. The converse does not hold,
however. A counterexample is provided by the following grammar,
which generates simple equality statements.
%%
\begin{equation}
\begin{split}
\mbox{\tt E} & \pf \mbox{\tt C=C} \\
\mbox{\tt C} & \pf \mbox{\tt D} \mid \mbox{\tt D+D} \\
\mbox{\tt D} & \pf \mbox{\tt 1} \mid \mbox{\tt 2}
\end{split}
\end{equation}
%%
Expressions of category {\tt E} are called equations, and they have
as their meaning either $\mathsf{true}$ or $\mathsf{false}$. Now, assign 
the following meanings to the strings.  {\tt 1} has as its {\tt D}-- 
and {\tt C}--meaning the number 1, {\tt 2} the number 2, and {\tt 1+1} 
as its {\tt C}--meaning the number 2. The {\tt E}--meanings are as follows.
%%
\newcommand{\sE}{\mbox{\smtt E}}
\begin{equation}
\begin{array}{ll@{\qquad}ll}
\mbox{}[\mbox{\tt 2=2}]^{\sE} & = \{\mathsf{true}\} 
	& [\mbox{\tt 1+1=2}]^{\sE} & = \{\mathsf{false}\} \\
\mbox{}[\mbox{\tt 1=2}]^{\sE} & = \{\mathsf{false}\} 
	& [\mbox{\tt 1=1+1}]^{\sE} & = \{\mathsf{true}\} \\
\mbox{}[\mbox{\tt 2=1}]^{\sE} & = \{\mathsf{false}\}
	& [\mbox{\tt 1+1=1}]^{\sE} & = \{\mathsf{true}\} \\
\mbox{}[\mbox{\tt 1=1}]^{\sE} & = \{\mathsf{true}\} 
	& [\mbox{\tt 2=1+1}]^{\sE} & = \{\mathsf{false}\} \\
                   &          & [\mbox{\tt 1+1=1+1}]^E & = \{\mathsf{true}\}
\end{array}
\end{equation}
%%
This grammar is unambiguous; and every string of category $X$ has 
exactly one $X$--meaning for $X \in \{\mbox{\tt C}, \mbox{\tt D}, 
\mbox{\tt E}\}$. Yet, there is no CFG of signs for this language. 
For the string {\tt 1+1} has the same {\tt T}--meaning as {\tt 2}, 
while substituting one for the other in an equation changes the 
truth value.

We shall show below that `weakly context free' and `strongly context 
free' coincide for interpreted languages. This means that the notion 
of an interpreted language is not a very useful one, since adding 
meanings to sentences does not help in establishing the structure of 
sentences. The idea of the proof is very simple. Consider an 
arbitrary linear sign grammar $\GA$ and a start symbol 
{\tt S}. We replace {\tt S} throughout by $\mbox{\tt S}^{\circ}$, 
where $\mbox{\tt S}^{\circ} \not\in C$. Now replace the algebra of 
meanings by the partial algebra of definite structure terms. This 
defines $\GB$.  Then for $c \in C - \{\mbox{\tt S}\}$, 
$\auf \vec{x}, c, \Gs\zu$ is a sign generated by $\GB$ iff 
$\Gs$ is a definite structure term such that $\Gs^{\varepsilon} = 
\vec{x}$ and $\Gs^{\gamma} = c$; 
and $\auf \vec{x}, \mbox{\tt S}^{\circ}, \Gs\zu$ is generated 
by $\GB$ iff $\Gs$ is a definite structure term such that 
$\Gs^{\varepsilon} = \vec{x}$, and $\Gs^{\gamma} = \mbox{\tt S}$. 
Finally, we introduce the following unary mode {\tt F}.
%%%
\begin{equation}
\mbox{\tt F}(\auf \vec{x}, \mbox{\tt S}^{\circ}, \Gs\zu) := 
	\auf \vec{x}, \mbox{\tt S}, \Gs^{\mu}\zu
\end{equation}
%%%
This new grammar, call it $\GA^{\circ}$, defines the same interpreted 
language with respect to {\tt S}. So, if an interpreted language is 
strongly context free, it has a context free sign grammar of this 
type. Now, suppose that the interpreted language $\CI$ is weakly 
context free. So there is a CFG $G$ generating $\mbox{\tt S}(\CI)$. 
At the first step we take the trivial semantics: everything is mapped 
to $0$. This is a strongly context free sign system, and 
we can perform the construction above. This yields a context free 
sign system where each $\vec{x}$ has as its $C$--denotations the 
set of structure terms that define a $C$--constituent with string 
$\vec{x}$. Finally, we have to deal with the semantics. Let $\vec{x}$ 
be an {\tt S}--string and let $|M_{\vec{x}}|$ be the set of meanings 
of $\vec{x}$ and $\mbox{\tt S}_{\vec{x}}$ the set of structure terms for 
$\vec{x}$ as an {\tt S}. If $|\mbox{\tt S}_{\vec{x}}| < |M_{\vec{x}}|$, 
there is no grammar for this language based on $G$. If, however, 
$|\mbox{\tt S}_{\vec{x}}| \geq |M_{\vec{x}}|$ there is a function 
$f_{\vec{x}} \colon \mbox{\tt S}_{\vec{x}} \pf M_{\vec{x}}$. Finally, 
put 
%%
\begin{equation}
\begin{split}
f^{\ast} & := \bigcup \auf f_{\vec{x}} : \vec{x} \in \mbox{\tt S}(\GI)\zu \\
\mbox{\tt F}^{\ast}(\auf \vec{x}, \mbox{\tt S}^{\circ}, \Gs\zu) & := 
	\auf \vec{x}, \mbox{\tt S}, f^{\ast}(\Gs)\zu
\end{split}
\end{equation}
%%%
This defines the sign grammar $\GA^{\ast}$. It is context free 
and its interpreted language with respect to the symbol {\tt S} is 
exactly $\CI$.
%%%
\begin{thm}
\label{thm:schwachstark}
Let $\CI$ be a countable interpreted language.
%%%
\begin{dingautolist}{192}
\item
If the string language of $\CI$ is context free then $\CI$ is
strongly context free.
\item
For every CFG $G$ for $\mbox{\tt S}(\CI)$ there exists a
context free system of signs $\Sigma$ and with a category 
{\tt S} such that
%%%%
\begin{enumerate}
\item $\mbox{\tt S}(\Sigma) = \CI$,
\item for every nonterminal symbol $A$ 
	$$\{\vec{x} : \mbox{ for some } 
    m \in M: \auf \vec{x}, A, m\zu \in \Sigma\} = \{\vec{x} :
    A \vdash_G \vec{x}\}$$
\end{enumerate}
\end{dingautolist}
\end{thm}
%%%
\proofbeg
\ding{193} has been established. For \ding{192} it suffices to 
observe that for every CFL there exists a CFG in which every 
sentence is infinitely ambigous. Just replace {\tt S} by 
$\mbox{\tt S}^{\bullet}$ and add the rules 
$\mbox{\tt S}^{\bullet} \pf \mbox{\tt S}^{\bullet} 
\mid \mbox{\tt S}$.
\proofend

Notice that the use of unary rules is essential. If there are no 
unary rules, a given string can have only exponentially many 
analyses. 
%%
\begin{lem}
\label{lem:cschaetz}
Let $L$ be a CFL and $d > 0$. Then there is a CFG $G$ and such 
that for all $\vec{x} \in L$ the set of nonisomorphic $G$--trees 
for $\vec{x}$ has at least $d^{|\vec{x}|}$ members.
\end{lem}
%%%
\proofbeg
Notice that it is sufficient that the result be proved for almost
all $\vec{x}$. For finitely many words we can provide as many
analyses as we wish. First of all, there is a grammar in Chomsky
normal form that generates $L$. Take two rules that can be used in
succession.
%%
\begin{equation}
A\quad \pf\quad BC, \qquad C\quad \pf\quad DE 
\end{equation}
%%
Add the rules
%%
\begin{equation}
A\quad \pf\quad  XE, \qquad X\quad\pf\quad AD
\end{equation}
%%
Then the string {\it ABC} has two analyses: $[A\; [B\; C]]$
and $[[A\; B]\; C]$. We proceed similarly if we have a pair of rules
%%
\begin{equation}
A\quad\pf\quad XE, \qquad X\quad\pf\quad AB
\end{equation}
%%
This grammar assigns exponentially many parses to a given
string. To see this notice that any given string $\vec{x}$
of length $n$ contains $n$ distinct constituents. For $n \leq 3$,
we use the `almost all' clause. Now, let $n > 3$. Then $\vec{x}$
has a decomposition $\vec{x} = \vec{y}_0\vec{y}_1\vec{y}_2$
into constituents. By inductive hypothesis, for $\vec{y}_i$
we have $d^{|\vec{y}_i|}$ many analyses. Thus $\vec{x}$ has at least
$2d^{|\vec{y}_0|}d^{|\vec{y}_1|}d^{|\vec{y}_2|} = 2d^{|\vec{x}|}$
analyses.
%%
\proofend

The previous proof actually assumes exponentially many 
different structures to a string. We can also give a simpler 
proof of this fact. Simply replace $N$ by $N \times d$ and 
replace in each rule every nonterminal $X$ by any one of the 
$\auf X, k\zu$, $k < d$.
%%%
\begin{thm}
\label{thm:bound}
Let $\CI$ be a countable interpreted language. Then $\CI$ is strongly
context free for a CFG without unary rules iff $\mbox{\tt S}(\CI)$
is context free and there is some constant $c > 0$ such that for almost 
all strings of $A^{\ast}$: the number of meanings of $\vec{x}$ is 
bounded by $c^{|\vec{x}|}$.
\end{thm}
%%
We give an example.  Let $A := \{\mbox{\tt Paul}, \mbox{\tt Marcus}, 
\mbox{\tt sees}\}$ and
%%
\begin{align}
\begin{split}
L := \{ & \mbox{\tt Paul sees Paul}, \mbox{\tt Paul sees Marcus}, \\
    & \mbox{\tt Marcus sees Paul}, \mbox{\tt Marcus sees Marcus}\}
\end{split}
\end{align}
%%
We associate the following truth values to the sentences.
%%
\begin{align}
\begin{split}
\CI = \{  & \auf \mbox{\tt Paul sees Paul}, 0\zu, \\
          & \auf \mbox{\tt Paul sees Marcus}, 1\zu, \\
          & \auf \mbox{\tt Marcus sees Paul}, 0\zu, \\
          & \auf \mbox{\tt Marcus sees Marcus},1\zu\}
\end{split}
\end{align}
%%
Furthermore, we fix a CFG that generates $L$:
%%
\begin{equation}
\begin{array}{lll@{\qquad}lll}
\rho_0 & := \mbox{\tt S} & \pf \mbox{\tt NP VP} 
	& \rho_1 & := \mbox{\tt VP} & \pf \mbox{\tt V NP} \\
\rho_2 & := \mbox{\tt NP} & \pf \mbox{\tt Paul} 
	& \rho_3 & := \mbox{\tt V} & \pf \mbox{\tt sees} \\
\rho_4 & := \mbox{\tt NP} & \pf \mbox{\tt Marcus} & &  
\end{array}
\end{equation}
%%
We construct a context free system of signs $\Sigma$ with
$\mbox{\tt S}(\Sigma) = \CI$. For every rule $\rho$ of arity 
$n > 0$ we introduce a symbol $\mbox{\tt N}_{\rho}$ of arity $n$. 
In the first step the interpretation is simply given by the 
structure term. For example, 
%%
\begin{equation}
\mbox{\tt N}_{\seins}(\auf \vec{x},\mbox{\tt V},\Gs\zu, 
\auf\vec{y}, \mbox{\tt NP},\Gt\zu) = 
\auf \vec{x}\oconc\vec{y}, \mbox{\tt VP}, \mbox{\tt N}_{\seins}\conc 
\Gs\conc\Gt\zu
\end{equation}
%%
(To be exact, on the left hand side we find the unfolding of 
$\mbox{\tt N}_{\seins}$ rather than the symbol itself.) Only 
the definition of 
$\mbox{\tt N}_{\snull}^{\mu}$ is somewhat different. Notice 
that the meaning of sentences is fixed. Hence the following 
must hold. 
%%
\begin{equation}
\label{eq:46a}
\begin{split}
(\mbox{\tt N}_{\snull}\mbox{\tt N}_{\svier}\mbox{\tt N}_{\seins}%
\mbox{\tt N}_{\sdrei}\mbox{\tt N}_{\svier})^{\mu} & = 1 \\
(\mbox{\tt N}_{\snull}\mbox{\tt N}_{\svier}\mbox{\tt N}_{\seins}%
\mbox{\tt N}_{\sdrei}\mbox{\tt N}_{\szwei})^{\mu} & = 0 \\
(\mbox{\tt N}_{\snull}\mbox{\tt N}_{\szwei}\mbox{\tt N}_{\seins}%
\mbox{\tt N}_{\sdrei}\mbox{\tt N}_{\svier})^{\mu} & = 1 \\
(\mbox{\tt N}_{\snull}\mbox{\tt N}_{\szwei}\mbox{\tt N}_{\seins}%
\mbox{\tt N}_{\sdrei}\mbox{\tt N}_{\szwei})^{\mu} & = 0
\end{split}
\end{equation}
%%
We can now do two things: we can redefine the action of the function 
$\mbox{\tt N}_{\snull}^{\mu}$. Or we can leave the action as given and 
factor out the congruence defined by \eqref{eq:46a} in the algebra of 
the structure terms enriched by the symbols $0$ and $1$. 
If we choose the latter option, we have
%%
\begin{align}
\begin{split}
M :=  \{ & 0, 1, \mbox{\tt N}_{\szwei}, \mbox{\tt N}_{\sdrei}, 
\mbox{\tt N}_{\svier}, \mbox{\tt N}_{\seins}\mbox{\tt N}_{\sdrei}%
\mbox{\tt N}_{\svier},  
\mbox{\tt N}_{\seins}\mbox{\tt N}_{\sdrei}\mbox{\tt N}_{\szwei},  
\mbox{\tt N}_{\snull}\mbox{\tt N}_{\svier}\mbox{\tt N}_{\seins}%
\mbox{\tt N}_{\sdrei}\mbox{\tt N}_{\svier}, 
\\ 
	&
\mbox{\tt N}_{\snull}\mbox{\tt N}_{\svier}\mbox{\tt N}_{\seins}%
\mbox{\tt N}_{\sdrei}\mbox{\tt N}_{\szwei}, 
\mbox{\tt N}_{\snull}\mbox{\tt N}_{\szwei}\mbox{\tt N}_{\seins}%
\mbox{\tt N}_{\sdrei}\mbox{\tt N}_{\svier},
\mbox{\tt N}_{\snull}\mbox{\tt N}_{\szwei}\mbox{\tt N}_{\seins}%
\mbox{\tt N}_{\sdrei}\mbox{\tt N}_{\szwei}\}
\end{split}
\end{align}
%%
Now let $\Theta$ be the congruence defined by \eqref{eq:46a}.
%%
\begin{align}
\begin{split}
M/\Theta :=  \{ & \{0, 
\mbox{\tt N}_{\snull}\mbox{\tt N}_{\svier}\mbox{\tt N}_{\seins}%
\mbox{\tt N}_{\sdrei}\mbox{\tt N}_{\szwei},  
\mbox{\tt N}_{\snull}\mbox{\tt N}_{\szwei}\mbox{\tt N}_{\seins}%
\mbox{\tt N}_{\sdrei}\mbox{\tt N}_{\szwei}\},  
\{\mbox{\tt N}_{\szwei}\}, \{\mbox{\tt N}_{\sdrei}\}, 
\\
&
\{\mbox{\tt N}_{\svier}\}, 
\{1, \mbox{\tt N}_{\snull}\mbox{\tt N}_{\svier}\mbox{\tt N}_{\seins}%
\mbox{\tt N}_{\sdrei}\mbox{\tt N}_{\svier}, 
\mbox{\tt N}_{\snull}\mbox{\tt N}_{\szwei}\mbox{\tt N}_{\seins}%
\mbox{\tt N}_{\sdrei}\mbox{\tt N}_{\svier}\}, 
\\ 
&
\{\mbox{\tt N}_{\seins}\mbox{\tt N}_{\sdrei}\mbox{\tt N}_{\svier}\},  
\{\mbox{\tt N}_{\seins}\mbox{\tt N}_{\sdrei}\mbox{\tt N}_{\szwei}\}
\}
\end{split}
\end{align}

Next we define the action on the categories. Let 
$\rho = B \pf A_0\dotsb A_{n-1}$. Then
%%
\begin{equation}
\mbox{\tt N}_{\rho}^{\gamma}(\gamma_0, \dotsc, \gamma_n)
    :=
    \begin{cases}
    B & \text{ if for all $i < n: \gamma_i = A_i$,} \\
        \star & \text{ otherwise.}
    \end{cases}
\end{equation}
%%
The functions on the exponents are fixed. 

Let us pause here and look at the problem of reversibility. Say 
that $f \colon S \pf \wp(T)$ is \textbf{finite} 
%%%%
\index{function!finite}%%
\index{function!bounded}%%
%%%%
if $|f(x)| < \omega$ for every $x \in S$. Likewise, $f$ is 
\textbf{bounded} if there is a number $k < \omega$ such that 
$|f(x)| < k$ for all $x$.
%%%
\begin{defn}
%%%
\label{defn:reverse}
\index{interpreted language!finitely reversible}%%
\index{interpreted language!boundedly reversible}%%
\index{$x^{\GI}$, $x_{\GI}$}%%
%%%
Let $\CI \subseteq E \times M$ be an interpreted language. 
$\CI$ is \textbf{finitely reversible} (\textbf{boundedly 
reversible}) if for every $x \in E$, $x^{\GI} := \{y : 
\auf x,y\zu \in \CI\}$ is finite (bounded), and for every 
$y \in M$, $y_{\GI} := \{x : \auf x,y\zu \in \CI\}$ is finite 
(bounded), and moreover, the functions $x \mapsto x^{\GI}$ and 
$y \mapsto y_{\GI}$ are computable.
\end{defn}
%%
The conditions on $x^{\GI}$ are independent 
from the conditions on $x_{\GI}$ 
(see \cite{dymetman:reversible,dymetman:thesis}).
%%
\begin{thm}[Dymetman]
There are interpreted languages $\CI$ and $\GK$ such that 
(a) $x \mapsto x^{\GI}$ is finite and computable but 
$y \mapsto y_{\GI}$ is not, 
(b) $y \mapsto y_{\GK}$ is finite and computable but 
$x \mapsto x^{\GI}$ is not.
\end{thm}
%%%
In the present context, the problem of enumerating the analyses 
is trivial. So, given a context free sign grammar, the function 
$\vec{x} \mapsto \vec{x}^{\GI}$ is always computable, although 
not always finite. If we insist on branching, $\vec{x}^{\GI}$ 
grows at most exponentially in the length of $\vec{x}$. We have 
established nothing about the maps $m \mapsto m_{\GI}$.

Let us now be given a sign system $\Sigma$ whose projection to 
$E \times C$ is context free. What conditions must be impose 
so that there exists a context free sign grammar for $\Sigma$?
Let $\vec{x} \in A^{\ast}$
%%%
\index{$[\vec{x}]_{\Sigma}^A$}%%
\index{$A$--meaning}%%
\index{meaning!$A$--\faul}%%
%%%
and $A \in C$. We write $[\vec{x}]_{\Sigma}^A :=
\{m : \auf \vec{x}, A, m\zu \in \Sigma\}$ and call this set
the set of $A$--\textbf{meanings of} $\vec{x}$ \textbf{in} $\Sigma$.
If $\Sigma$ is given by the context, we omit it. If a system of
has a CFG $\GA$ then it satisfies the following equations for 
every $A$ and every $\vec{x}$.
%%
\begin{equation}
\label{eq:numbmean}
\bigcup_{Z \in F} Z^{\mu}[[\vec{x}_0]^{B_0}\times \dotsb\times
    [\vec{x}_{\Omega(Z)-1}]^{B_{\Omega(Z)-1}}] =
    [\vec{x}_0\conc \dotsb \conc \vec{x}_{\Omega(Z)-1}]^A
\end{equation}
%%
where $Z^{\tau}(B_0, \dotsb, B_{n-1}) = A$. This means simply
put that the $A$--meanings of $\vec{x}$ can be computed
directly from the meanings of the immediate subconstituents.
From \eqref{eq:numbmean} we can derive the following estimate.
%%
\begin{equation}
\label{eq:nn2}
\sum_{Z \in F} \left(\prod_{i < \Omega(Z)} |[\vec{x}_i]^{B_i}|\right)
    \geq
    |[\vec{x}_0\conc \dotsb \conc \vec{x}_{\Omega(Z)-1}]^A|
\end{equation}
%%
This means that a string cannot have more meanings than it has 
readings (= structure terms). We call the condition \eqref{eq:nn2}
the \textbf{count condition}. In particular, it implies that 
there is a constant $c$ such that for almost all $\vec{x}$ and all 
$A$:
%%
\begin{equation}
|[\vec{x}]^A| \leq c^{|\vec{x}|}
\end{equation}
%%
Even if the count condition is satisfied it need not be 
possible to construct a context free system of signs. Here is an 
example. 
%%%
\begin{align}
\Sigma := & 
	 \{\auf \mbox{\tt 0}^n, \mbox{\tt T}, 0\zu : n \in \omega\} 
	\cup \{\auf \mbox{\tt a}\mbox{\tt 0}^n\mbox{\tt a}, \mbox{\tt S}, 
	n\zu : n \in \omega\}
\end{align}
%%%
A CFG for $\Sigma$ is 
%%%
\begin{align}
\mbox{\tt S} \pf & \mbox{\tt aTa} &
\mbox{\tt T} \pf & \varepsilon \mid \mbox{\tt T0} 
\end{align}
%%%
The condition \eqref{eq:numbmean} is satisfied. But there is no 
context free sign grammar for $\Sigma$. The reason is that no matter 
how the functions are defined, they must produce an infinite 
set of numbers from just one input, 0. Notice that we can even 
define a boundedly reversible system for which no CFG exists.
It consists of the signs $\auf \mbox{\tt 0}^n, \mbox{\tt T}, n\zu$, 
$n \in \omega$, the signs $\auf \mbox{\tt a0$^n$a}, \mbox{\tt S}, 2^n\zu$ 
and the signs $\auf \mbox{\tt a0$^n$a}, \mbox{\tt S}, 3^n\zu$ 
where $n$ is prime. We have $\mbox{\tt a0$^n$a}^{\GI} \subseteq 
\{2^n, 3^n\}$, whence $|\vec{x}^{\GI}| \leq 2$, and $|k_{\GI}| \leq 2$.
However, suppose we allow the functions $f^{\varepsilon}$ 
to be partial, but if defined $f^{\varepsilon}(\vec{x}_0, 
\dotsc, \vec{x}_{\Omega(f)-1}) = \prod_{i < \Omega(f)} \vec{x}_i$.
Then a partial context free grammar exists if the interpreted 
language defined by $\Sigma$ is boundedly reversible. (Basically, 
the partiality allows any degree of sensitivity to the string of 
which the expression is composed. Each meaning is represented by 
a bounded number of expressions.)

Let $\rho = A \pf B_0 \dotsb B_{n-1}$. Notice that  the function
$\mbox{\tt F}_{\rho}$ makes $A$--meanings from certain
$B_i$--meanings of the constituents of the string. However,
often linguists use their intuition to say what an $A$--string
means under a certain analysis, that is to say, structure term.
This --- as is easy to see --- is tantamount to knowing the
functions themselves, not only their domains and ranges. For
let us assume we have a function which assigns to every structure
term $\Gt$ of an $A$--string $\vec{x}$ an $A$--meaning. Then
the functions $\mbox{\tt F}_{\rho}$ are uniquely determined.
For let a derivation of $\vec{x}$ be given. This derivation
determines derivations of its immediate constituents, which
are now unique by assumption. For the tuple of meanings of
the subconstituents we know what the function does. Hence,
it is clear that for any given tuple of meanings we can
say what the function does. (Well, not quite. We do not know
what it does on meanings that are not expressed by a $B_i$--string,
$i < \Omega(f)$. However, on any account we have as much knowledge
as we need.)

Let us return to Montague Grammar. Let $\Sigma
%%%
\index{Montague Semantics}%%
%%%
\subseteq A^{\ast} \times C \times M$
be strongly context free, with $F$ the set of modes. We want to
show that there is an AB--grammar which generates
$\Sigma$. We have to precisify in what sense we want to understand
this. We cannot expect that $\Sigma$ is any context free system,
since AB--grammars are always binary branching. This, however,
means that we have to postulate other constituents than those of
$\Sigma$. Therefore we shall only aim to have the same sentence
meanings. In what way we can get more, we shall see afterwards.
To start, there is a trivial solution of our problem.
If $\rho = A \pf B_0 B_1$ is a rule we add a 0--ary mode
%%
\begin{equation}
\mbox{\tt N}_{\rho} = \auf \varepsilon, A/B_0/B_1,
\lambda x_{B_0}.\lambda x_{B_1}.F_{\rho}(x_{B_0}, x_{B_1})\zu
\end{equation}
%%
This allows us to keep our constituents. However, postulating
empty elements does have its drawbacks. It increases the costs 
of parsing, for example. We shall therefore ask whether one
can do without empty categories. This is possible. For, as we
have seen, with the help of combinators one can liberate oneself
from the straightjacket of syntactic structure. Recall from 
Section~\ref{kap2}.\ref{kap2-2} the transformation of a CFG into Greibach 
Normal Form. This uses essentially the tool of skipping a rule 
and of eliminating left recursion. We leave it to the reader to 
formulate (and prove) an analogon of the skipping of rules for 
context free sign grammars.  This allows us to concentrate on the 
elimination of left recursion. We look again at the construction of
Lemma~\ref{lem:linksrek}. Choose a nonterminal $X$.  Assume
that we have the following $X$--productions, where
$\vec{\alpha}_j$, $j < m$, and $\vec{\beta}_i$, $i < n$,
do not contain $X$.
%%
\begin{equation}
\rho_j := X \pf X\conc \vec{\alpha}_i, \quad i < m,
\qquad \sigma_i := X \pf \beta_i, \quad i < n
\end{equation}
%%
Further let $\mbox{\tt F}^{\mu}_{\rho_j}$, $j < m$, and
$\mbox{\tt F}^{\mu}_{\sigma_i}$, $i < n$, be given. To keep
the proof legible we assume that $\vec{\beta}_j = Y_j$,
$\vec{\alpha}_i = U_i$, are nonterminal symbols. (Evidently,
this can be achieved by introducing some more nonterminal
symbols.) We we have now these rules.
%%
\begin{equation}
\rho_j = X \pf X\conc U_i, \quad i < m,
\qquad \sigma_i = X \pf Y_i, \quad i < n
\end{equation}
%%
So, we generate the following structures.
%%
\begin{equation}
[Y' \; [U_{i_0}\; [U_{i_1}\dotsb [U_{i_{n-2}}\; U_{i_{n-1}}]\dotsb]]]
\end{equation}
%%
We want to replace them by these structures instead:
%%
\begin{equation}
[[\dotsb[[Y'\; U_{i_0}] \; U_{i_1}]\dotsb U_{i_{n-2}}]\;
U_{i_{n-1}}]
\end{equation}
%%
Proceed as in the proof of Lemma~\ref{lem:linksrek}. Choose a new
symbol $Z$ and replace the rules by the following ones.
%%
\begin{subequations}
\begin{align}
\lambda_j & := X \pf U_j, & j < m, &&
\nu_i & := Z \pf Y_i, & i < n, \\
\mu_j & := X \pf Y_j \conc Z, & j < m, &&
\xi   & := Z \pf Y_i \conc Z, & i < n.
\end{align}
\end{subequations}
%%
Now define the following functions.
%%
\begin{align}
\begin{split}
\mbox{\tt G}^{\mu}_{\lambda_i} & := \mbox{\tt F}^{\mu}_{\sigma_i} \\
\mbox{\tt G}^{\mu}_{\mu_i}(x_0,x_1) & := x_1(x_0) \\
\mbox{\tt G}^{\mu}_{\nu_i}(x_0,x_1) & :=
    \lambda x_2.\mbox{\tt F}^{\mu}_{\rho_i}(x_1(x_2), x_0)
    \\
\mbox{\tt G}^{\mu}_{\xi_i}(x_0) & :=
    \lambda x_0.\mbox{\tt F}^{\mu}_{\rho_i}(x_0, x_1)
\end{split}
\end{align}
%%
Now we have eliminated all left recursion on $X$. We only have to 
show that we have not changed the set of $X$--meanings for any 
string. To this end, let $\vec{x}$ be an $X$--string, say 
$\vec{x} = \vec{y} \conc \prod_{i < k} \vec{z}_i$, where $\vec{y}$ 
is a $Y_j$--string and $\vec{z}_i$ a $U_{j_i}$--string. Then in the 
transformed grammar we have the $Z$--strings
%%
\begin{equation}
\vec{u}_p := \prod_{p \leq i < k} \vec{z}_i
\end{equation}
%%
and $\vec{x}$ is an $X$--string. Now we still have to determine
the meanings. Let $\Gm$ be a meaning of $\vec{y}$ as a
$Y_i$--string and $\Gn_i$, $i < k$, a meaning of $\vec{z}_i$ as
a $U_{j_i}$--string. The meaning of $\vec{x}$ as an $X$--string
under this analysis is then
%%
\begin{equation}
\mbox{\tt F}^{\mu}_{\rho_{j_{n-1}}}(%
    \mbox{\tt F}^{\mu}_{\rho_{j_{n-2}}}(\dotsb (%
    \mbox{\tt F}^{\mu}_{\rho_{j_0}}(\Gm,\Gn_0),\Gn_1),
    \dotsc,\Gn_{n-2}),\Gn_{n-1})
\end{equation}
%%
As a $Z$--string $\vec{u}_{n-1}$ has the meaning
%%
\begin{equation}
\Gu_{n-1} :=
    \mbox{\tt G}^{\mu}_{\nu_{i_{n-1}}}(\Gn_{n-1}) =
    \lambda x_0.\mbox{\tt F}^{\mu}_{\rho_i}(x_0, \Gn_{n-1}) 
\end{equation}
%%
Then $\vec{u}_{n-2}$ has the meaning
%%
\begin{align}
\begin{split}
\Gu_{n-2} & := \mbox{\tt G}^{\mu}_{\nu_{i_{n-2}}}(\Gn_{n-2},\Gu_{n-1}) \\
    & = \lambda x_2.\mbox{\tt F}^{\mu}_{\rho_{n-2}}(\Gu_{n-1}(x_2),
        \Gn_{n-2}) \\
    & = \lambda x_2.\mbox{\tt F}^{\mu}_{\rho_{n-2}}(%
    \mbox{\tt F}^{\mu}_{\rho_i}(x_2, \Gn_{n-1}),\Gn_{n-2}).
\end{split}
\end{align}
%%
Inductively we get
%%
\begin{equation}
\Gu_{n-j} =
    \lambda x_0.\mbox{\tt F}^{\mu}_{\rho_{n-1}}(%
    \mbox{\tt F}^{\mu}_{\rho_{n-2}}(\dotsb
    \mbox{\tt F}^{\mu}_{\rho_{n-j}}(
    x_0, \Gn_{n-j}), \dotsc, \Gn_{n-2}),\Gn_{n-1})
\end{equation}
%%
If we put $j = n$, and if we apply at last the function
$\mbox{\tt F}^{\mu}_{\mu_j}$ on $\Gm$ and the result
we finally get that $\vec{x}$ has the same $X$--meaning
under this analysis. The converse shows likewise that
every $X$--analysis of $\vec{x}$ in the transformed
grammar can be transformed back into an $X$--analysis
in the old grammar, and the $X$--meanings of the two
are the same.

The reader may actually notice the analogy with the semantics
of the Geach rule. There we needed to get new constituent
structures by bracketing $[A [B\; C]]$ into $[[A\; B]\; C]$.
Supposing that $A$ and $B$ are heads, the semantics of the
rule forming $[A\; B]$ must be function composition. This is
what the definitions achieve here. Notice, however, that we
have no categorial grammar to start with, so the proof given
here is not fully analogous. Part of the semantics of the
construction is still in the modes themselves, while categorial
grammar requires that it be in the meaning of the lexical
items.

After some more steps, consisting in more recursion
elimination and skipping of rules we are finally done.
Then the grammar is in Greibach normal form. The latter
can be transformed into an AB--grammar, as we have already 
seen.
%%
\begin{thm}
Let $\Sigma$ be a context free linear system of signs.
Then there exists an AB--grammar that
generates $\Sigma$.
\end{thm}
%%
The moral to be drawn is that Montague grammmar is actually
%%%
\index{Montague Semantics}%%%
%%%
quite powerful from the point of view of semantics. If the string
languages are already context
free, then if any context free analysis succeeds, so does
an analysis in terms of Montague grammar (supposing here
that nothing except linear concatenation is allowed in the
exponents). We shall extend this result later to
\textbf{PTIME}--languages.

{\it Notes on this section.} With suitable conditions (such as 
nondeletion) the set $x^{\GI}$ becomes enumerable for every 
given $x$, simply because the number of parses of a string is 
finite (and has an a priori bound based on the length of 
$x$). Yet, $x^{\GI}$ is usually infinite (as we discussed in 
Section~\ref{kap4}.\ref{kap4-1}), and the sets $y_{\GI}$ need not be 
recursively enumerable for certain $y$. \cite{pogodalla:reseaux} 
%%%
\index{Pogodalla, Sylvain}%%%
%%%
studies how this changes for categorial grammars if semantic 
representations are not formulae but linear formulae. In that 
case, the interpreted grammar becomes reversible, and generation 
is polynomial time computable.
%%
\vplatz
\exercise
Let $G$ be a quasi context free sign grammar. Construct a 
context free sign grammar which generates the same interpreted 
language.
%%
\vplatz
\exercise
Let $G$ be determined by the two rules
$\mbox{\tt S} \pf \mbox{\tt SS} \mid \mbox{\tt a}$.
Show that the set of constituent structures of $G$ cannot be
generated by an AB--grammar. {\it Hint.} Let $d(\alpha)$  
be the number of occurrences of slashes ({\mtt{\tb}} or 
{\mtt{\tf}}) in $\alpha$. If $\alpha$ is the mother of $\beta$ 
and $\gamma$ then either $d(\beta) > d(\alpha)$ or $d(\gamma) > 
d(\alpha)$.
%%
\vplatz
\exercise
Let $\Sigma$ be strongly context free with respect to a
2--standard CFG $G$ with the following property:
there exists a $k \in \omega$ such that for every $G$--tree
$\auf T, <, \sqsubset, \ell\zu$ and every node $x \in T$ there 
is a terminal node $y \in T$ with $[y,x] \leq k$. Then there 
exists an AB--grammar for $\Sigma$ which generates the 
same constituent structures as $G$.
%%
\vplatz
\exercise
As we have seen above, left recursion can be eliminated from
a CFG $G$. Show that there exists a $\CCG(\mathsf{B})$ 
grammar which generates for every nonterminal $X$ the same set 
of $X$--strings. Derive from this that we can write an AB--grammar 
which for every $X$ generates the same $X$--strings as $G$. Why 
does it not follow that $L_B(G)$ can be generated by some AB--grammar?
{\it Hint.} For the first part of the exercise consider
Exercise~\ref{ueb:ab}.
%%
\vplatz
\exercise
%%%%
\index{$\oli{X}$--syntax}%%
%%%%
Let $\auf \mbox{\tt S}, C, A, \zeta\zu$ be an AB--grammar. Put
$\CC^0 := \bigcup_{a \in A} \zeta(a)$. These are the 0th projections.
Inductively we put for $\beta^i = \alpha/\gamma \in \CC^i$,
$\gamma \neq \alpha$, $\beta^{i+1} := \alpha$. In this way we
define the projections of the symbols from $\CC^0$. Show that
by these definitions we get a grammar which satisfies the
principles of $\oli{X}$--syntax.  {\it Remark.} The maximal
projections are not necessarily in $\CC^2$.

 \section{Partiality and Discourse Dynamics}
\label{kap4x7}
%
%
%
After having outlined the basic setup of Montague Semantics, we
%%%
\index{Montague Semantics}%%%
%%%
shall deal with an issue that we have so far tacitly ignored, namely
{\it partiality}. The name `partial logic' covers a wide variety of 
logics that deal with radically different problems. We shall look at 
two of them. The first is that of partiality as undefinedness. The
second is that of partiality as ignorance. We start with
partiality as undefinedness.

Consider the assignment $y := (x+1)/(u^2 - 9)$ to $y$ in a program.
This clause is potentially dangerous, since $u$ may equal 3, in
which case no value can be assigned to $y$. Similarly, for a sequence
$\Ga = (a_n)_{n \in \omega}$, $\lim \Ga := \lim_{n \pf \infty} a_n$
is defined only if the series is convergent. If not, no value can be
given. Or in type theory, a function $f$ may only be applied to $x$ if
$f$ has type $\alpha\pf\beta$ for certain $\alpha$ and $\beta$ $x$ has
type $\alpha$. In the linguistic and philosophical literature, this
phenomenon is known as \textbf{presupposition}. It is defined as a 
relation between propositions (see \cite{vandersandt:presupposition}).
%%
\begin{defn}
%%%
\index{presupposition}%%
\index{$\gg_{\vdash}$}%%
%%%
A proposition $\varphi$ \textbf{presupposes} $\chi$ if both
$\varphi \vdash \chi$ and $\nicht\varphi \vdash \chi$.
We write $\varphi \gg_{\vdash} \chi$ (or simply $\varphi \gg
\chi$) to say that $\varphi$ presupposes $\chi$.
\end{defn}
%%
The definition needs only the notion of a negation in order to be
well--defined. Clearly, in boolean logic this definition gives
pretty uninteresting results. $\varphi$ presupposes $\chi$ in 
$\mathsf{PC}$ iff $\chi$ is a tautology. However, if we have more
than two truth--values, interesting results appear. First, notice
that we have earlier remedied partiality by assuming a `dummy'
element $\star$ that a function assumes as soon as it is not
defined on its regular input. Here, we shall remedy the situation
by giving the expression itself the truth--value $\star$. That is
to say, rather than making functions themselves total, we make the
assignment of truth--values a total function. This has different
consequences, as will be seen. Suppose that we totalize the
operator $\lim_{n \pf \omega}$ so that it can be applied to all
sequences. Then if $(a_n)_{n \in \omega}$ is not a convergent
series, say $a_n = (-1)^n$, $3 = \lim_{n \pf \infty} a_n$ is not
true, since $\lim_{n \pf \omega} a_n = \star$ and $3 \neq \star$.
The negation of the statement will then be true. This is
effectively what Russel~\shortcite{russell:denoting} and
%%%
\index{Russell, Bertrand}%%
%%%
Kempson~\shortcite{kempson:presupposition} 
%%%
\index{Kempson, Ruth}%%
%%%
claim. Now suppose we say
that $3 = \lim_{n \pf \infty} a_n$ has no truth--value; then $3
\neq \lim_{n \pf \infty} a_n$ also has no truth--value. To
nevertheless be able to deal with such sentences rather than
simply excluding them from discourse, we introduce a third 
truth--value, $\star$. The question is now: how do we define the
3--valued counterparts of $-$, $\cap$ and $\cup$? In order to keep
confusion at a minimum, we agree on the following conventions.
%%%
\index{$\vdash_3$}%%%
%%%
$\vdash_3$ denotes the 3--valued consequence relation determined
by a matrix $\GM = \auf \{0,1,\star\}, \Pi, \{1\}\zu$, where $\Pi$
is an interpretation of the connectives. We shall assume that $F$
consists of a subset of the set of the 9 unary and 27 binary
symbols, which represent the unary and binary functions on the
three element set. This defines $\Omega$. Then $\vdash_3$ is
uniquely fixed by $F$, and the logical connectives will receive a
distinct name every time we choose a different function on
$\{0,1,\star\}$. What remains to be solved, then, is not what
logical language to use but rather by what connective to translate
the ordinary language connectors {\tt not}, {\tt and}, {\tt or},
and {\tt if$\dotsb$then}. Here, we assume that whatever interprets
them is a function on $\{0,1,\star\}$ (or $\{0,1,\star\}^{2}$),
whose restriction to $\{0,1\}$ is its boolean counterpart, which
is already given. For those functions, the 2--valued consequence
is also defined and denoted by $\vdash_2$.
%%%%
\index{$\vdash_2$}%%%
%%%

Now, if $\star$ is the truth--value reserved for the otherwise truth
valueless statements, we get the following three valued logic, 
due to Bochvar~\shortcite{bochvar:three}. Its characteristics is the
fact that undefinedness is strictly hereditary.
%%
\begin{equation}
\begin{array}{l|l}
        & - \\\hline
0       & 1 \\
1       & 0 \\
\star   &  \star
\end{array}
\qquad
\begin{array}{l|lll}
\cap    & 0     & 1     & \star \\\hline
0       & 0     & 0     & \star \\
1       & 0     & 1     & \star \\
\star   & \star & \star & \star
\end{array}
\qquad
\begin{array}{l|lll}
\cup    & 0     & 1     & \star \\\hline
0       & 0     & 1     & \star \\
1       & 1     & 1     & \star \\
\star   & \star & \star & \star
\end{array}
\end{equation}
%%%
The basic connectives are $\nicht$, $\und$ and $\oder$, which
are interpreted by $-$, $\cap$ and $\cup$. Here is a characterization
of presupposition in Bochvar's Logic. Call a connective
%%%
\index{connective!Bochvar}%%
%%%%
\ding{67} a \textbf{Bochvar}--\textbf{connective} if 
$\Pi(\mbox{\ding{67}})(\vec{x}) = \star$ iff $x_i = \star$ 
for some $i < \Omega(\mbox{\ding{67}})$.
%%
\begin{prop}
Let $\Delta$ and $\chi$ be composed using only Boch\-var--con\-nec\-tives.
Then $\Delta \vdash_3 \chi$ iff (i) $\Delta$ is not classically satisfiable 
or (ii) $\Delta \vdash_2 \chi$ and $\var(\chi) \subseteq \var[\Delta]$.
\end{prop}
%%
\proofbeg
Suppose that $\Delta \vdash_3 \chi$ and that
$\Delta$ is satisfiable. Let $\beta$ be a valuation such that
$\oli{\beta}(\delta) = 1$ for all $\delta \in \Delta$. Put
$\beta^+(p) := \beta(p)$ for all $p \in \var(\Delta)$
and $\beta^+(p) := \star$ otherwise. Suppose that
$\var(\chi) - \var(\Delta) \neq \varnothing$.
Then $\oli{\beta^+}(\chi) = \star$, contradicting our assumption.
Hence, $\var(\chi) \subseteq \var(\Delta)$.
It follows that every valuation that satisfies $\Delta$ also
satisfies $\chi$, since the valuation does not assume $\star$
on its variables (and can therefore be assumed to be a classical
valuation). Now suppose that $\Delta \nvdash_3 \chi$.
Then clearly $\Delta$ must be satisfiable. Furthermore, by the
argument above either $\var(\chi) - \var(\Delta)
\neq \varnothing$ or else $\Delta \vdash_2 \chi$.
\proofend
%%%

This characterization can be used to derive the following corollary.
%%%
\begin{cor}
Let $\varphi$ and $\chi$ be composed by Bochvar--connectives.
Then $\varphi \gg \chi$ iff $\var(\chi) \subseteq \var(\varphi)$ 
and $\vdash_2 \chi$.
\end{cor}
%%%
Hence, although Bochvar's logic makes room for undefinedness, the
notion of presupposition is again trivial. Bochvar's Logic seems
nevertheless adequate as a treatment of the $\iota$--operator. It
is formally defined as follows.
%%
\begin{defn}
$\iota$ is a partial function from predicates to objects such that
$\iota x.\chi(x)$ is defined iff there is exactly one
$b$ such that $\chi(b)$, and in that case $\iota x. \chi(x) := b$.
\end{defn}
%%
Most mathematical statements which involve presuppositions are instances
of a (hidden) use the $\iota$--operator. Examples are the derivative, 
the integral and the limit. In ordinary language, $\iota$ corresponds 
to the definite
determiner {\tt the}. Using the $\iota$--operator, we can bring out
the difference between the bivalent interpretation and the three
valued one. Define the predicate $\textsf{cauchy}'$ on infinite
sequences of real numbers as follows:
%%%
\begin{equation}
\textsf{cauchy}'(\Ga) :=
    (\forall \varepsilon > 0)(\exists n)(\forall m \geq n)
    |\Ga(m) - \Ga(n)| < \varepsilon
\end{equation}
%%%
This is in formal terms the definition of a Cauchy sequence.
Further, define a predicate $\textsf{cum}'$ as follows.
%%%
\begin{equation}
\textsf{cum}'(\Ga)(x) := (\forall \varepsilon > 0)(\exists n)%
    (\forall m \geq n)|\Ga(m) - x| < \varepsilon
\end{equation}
%%%
This predicate says that $x$ is a cumulation point of $\Ga$.
Now, we may set $\lim \Ga := \iota x.\textsf{cum}'(\Ga)(x)$.
Notice that $\textsf{cauchy}'(\Ga)$ is equivalent to
%%%
\begin{equation}
(\exists x)(\textsf{cum}'(\Ga)(x) \und (\forall y)(\textsf{cum}'(\Ga)(y) 
\pf y \doteq x)) 
\end{equation}
%%%
This is exactly what must be true for $\lim \Ga$ to be defined.
%%%
\begin{align}
\label{eq:471} & \mbox{\tt The limit of $\Ga$ equals three.} \\
\label{eq:472} & \iota x.\textsf{cum}'(\Ga)(x) \doteq 3 \\
\label{eq:473} & (\exists x)(\textsf{cum}'(\Ga)(x) \und
    (\forall y)(\textsf{cum}'(\Ga)(y) \pf y \doteq x)
    \und x \doteq 3).
\end{align}
%%%
Under the analysis \eqref{eq:472} the sentence \eqref{eq:471}
presupposes that $\Ga$ is a Cauchy--sequence. \eqref{eq:473} does
not presuppose that. However, the dilemma for the translation
\eqref{eq:473} is that the negation of \eqref{eq:471} is also
false (at least in ordinary judgement). What this means is that
the truth--conditions of \eqref{eq:474} are not expressed by 
\eqref{eq:476}, but by \eqref{eq:477} which in three valued 
logic is \eqref{eq:475}.
%%%
\begin{align}
\label{eq:474} & \mbox{\tt The limit of $\Ga$ does not equal three.} \\
\label{eq:475} & \nicht (\iota x.\textsf{cum}'(\Ga)(x) \doteq 3) \\
\label{eq:476} & \nicht (\exists x)(\textsf{cum}'(\Ga)(x) \und
    (\forall y)(\textsf{cum}'(\Ga)(y) \pf y \doteq x)
    \und x \doteq 3) \\
\label{eq:477} & (\exists x)(\textsf{cum}'(\Ga)(x) \und
    (\forall y)(\textsf{cum}'(\Ga)(y) \pf y \doteq x)
    \und \nicht(x \doteq 3))
\end{align}
%%%
It is difficult to imagine how to get the translation \eqref{eq:477} 
in a bivalent approach, although a proposal is made below. The problem 
with a bivalent analysis is that it can be shown to be inadequate, 
because it rests on the assumption that the primitive predicates are
bivalent. However, this is problematic. The most clear--cut case is that 
of the truth--predicate. Suppose we define the semantics of $\mathsf{T}$ 
on the set of natural numbers as follows.
%%
\begin{equation}
\label{eq:47dagger}
\mathsf{T}(\ulcorner \varphi \urcorner) \dpf \varphi
\end{equation}
%%
Here, $\ulcorner \varphi \urcorner$ is, say, the G\"odel code of
$\varphi$. It can be shown that there is a $\chi$ such
that $\mathsf{T}(\ulcorner \chi \urcorner) \dpf \nicht\chi$ is
true in the natural numbers. This contradicts 
\eqref{eq:47dagger}. The sentence $\chi$ corresponds to the 
following liar paradox.
%%
\begin{align}
\label{eq:478} & \mbox{\tt This sentence is false.}
\end{align}
%%
Thus, as Tarski has observed, a truth--predicate that is consistent
with the facts in a sufficiently rich theory must be partial. As
sentence \eqref{eq:478} shows, natural languages are sufficiently
rich to produce the same effect. Since we do not want to give up
the correctness of the truth--predicate (or the falsity predicate),
the only alternative is to assume that it is partial. If so, however,
there is no escape from the use of three valued logic, since
bivalence must fail.

Let us assume therefore that we three truth--values. What Bochvar's
logic gives us is called the logic of hereditary undefinedness.
For many reasons it is problematic, however. Consider the
following two examples.
%%%
\begin{align}
\label{eq:479} & \mbox{\tt If $\Ga$ and $\Gb$ are convergent sequences,
    $\lim (\Ga + \Gb)$} \\\notag
        & \quad \mbox{\tt $= \lim \Ga + \lim \Gb$.} \\
\label{eq:4710} & \textsf{if $u \neq 3$ then $y := (x+1)/(u^2 - 9)$
    else $y := 0$ fi}
\end{align}
%%%
By Bochvar's Logic, \eqref{eq:479} presupposes that
$\Ga$, $\Gb$ and $\Ga + \Gb$ are convergent series. \eqref{eq:4710}
presupposes that $u \neq 3$. However, none of
the two sentences have nontrivial presuppositions. Let us illustrate
this with \eqref{eq:479}. Intuitively, the if--clause preceding the
equality statement excludes all sequences from consideration
where $\Ga$ and $\Gb$ are nonconvergent sequences. One can show
that $\Ga + \Gb$, the pointwise sum of $\Ga$ and $\Gb$, is then
also convergent. Hence, the if--clause covers all cases of
partiality. The statement $\lim (\Ga + \Gb) = \lim \Ga + \lim \Gb$
never fails. Similarly, {\tt and} has the power to eliminate
presuppositions.
%%%
\begin{align}
\label{eq:4711} & \mbox{\tt $\Ga$ and $\Gb$ are convergent series
    and $\lim (\Ga + \Gb)$} \\\notag
        & \quad \mbox{\tt $= \lim \Ga + \lim \Gb$.} \\
\label{eq:4712} & u:= 4; y := (x+1)/(u^2 - 9)
\end{align}
%%%
As it turns out, there is an easy fix for that. Simply associate
the following connectives with {\tt if$\dotsb$then} and {\tt and}.
%%
\index{$\und'$, $\oder'$, $\pf'$}%%%
\begin{equation}
\begin{array}{l|lll}
\pf' &  0    &  1    & \star  \\\hline
0     &  1    &  1    &  1    \\
1     &  0    &  1    & \star \\
\star & \star & \star & \star
\end{array}
\qquad
\begin{array}{l|lll}
\und' &  0    &  1    & \star \\\hline
0     &  0    &  0    &  1    \\
1     &  0    &  1    & \star \\
\star & \star & \star & \star
\end{array}
\qquad
\begin{array}{l|lll}
\oder'  &  0    &  1    & \star    \\\hline
0       &  0    &  1    & \star    \\
1       &  1    &  1    & 1        \\
\star   & \star & \star & \star
\end{array}
\end{equation}
%%
The reader may take notice of the fact that while $\und'$ and
$\pf'$ are reasonable candidates for {\tt and} and 
{\tt if$\dotsb$then}, $\oder'$ is not as good for {\tt or}.

In the linguistic literature, various attempts have been made to
explain these facts. First, we distinguish the {\it presupposition\/} 
of a sentence from its {\it assertion}. The definition of
these terms is somewhat cumbersome. The general idea is that the
presupposition of a sentence is a characterization of those
circumstances under which it is either true or false, and the
assertion is what the sentence says when it is either true of
false (that is to say, the assertion tells us when the sentence is
true given that it is either true or false). Let us attempt to define
this.  Let $\varphi$ be a proposition. Call $\chi$ a 
%%%
\index{presupposition!generic}%%
%%%%
\textbf{generic presupposition of} $\varphi$ if the following holds. 
(a) $\varphi \gg \chi$, (b) if $\varphi \gg \psi$ then 
$\chi \vdash_3 \psi$.  If $\chi$ is a generic presupposition of 
$\varphi$, $\chi \pf \varphi$ is called an 
%%%
\index{assertion}%%
%%%
\textbf{assertion} of $\varphi$. First, notice
that presuppositions are only defined up to interderivability.
This is not a congruence. We may have $\varphi {\dashv\vdash}_3
\chi$ without $\varphi$ and $\chi$ receiving the same truth--value
under all assignments. Namely, $\varphi {\dashv\vdash}_3 \chi$ 
iff $\varphi$ and $\chi$ are truth--equivalent, that is,
$\oli{\beta}(\varphi) = 1$ exactly when $\oli{\beta}(\chi) = 1$.
In order to have full equivalence, we must also require
$\nicht \varphi {\dashv\vdash}_3 \nicht \chi$. Second, notice 
that $\varphi \oder \nicht \varphi$ satisfies (a) and (b).
However, $\varphi\oder\nicht\varphi$ presupposes itself, 
something that we wish to avoid. 
%%%
\index{bivalence}%%%
%%%%
So, we additionally require the generic presupposition to be 
\textbf{bivalent}. Here, $\varphi$ is \textbf{bivalent} if for 
every valuation $\beta$ into $\{0,1,\star\}$: $\oli{\beta}(\varphi) 
\in \{0,1\}$. Define the following connective.
%%
\begin{equation}
%%%
\index{$\downarrow$}%%%
%%%%
\begin{array}{l|lll}
\downarrow & 0     & 1     & \star \\\hline
0          & \star & 0     & \star \\
1          & \star & 1     & \star \\
\star      & \star & \star & \star
\end{array}
\end{equation}
%%
\begin{defn}
%%%
\index{presupposition!generic}%%
\index{assertion}%%
%%%
Let $\varphi$ be a proposition. $\chi$ is a \textbf{generic
presupposition of} $\varphi$ \textbf{with respect to} $\vdash_3$ if
(a) $\varphi \gg \chi$, (b) $\chi$ is bivalent and (c) if $\varphi
\gg \psi$ then $\chi \vdash_3 \psi$. $\chi$ is the \textbf{assertion}
of $\varphi$ if (a) $\chi$ is bivalent and (b) $\chi \dashv\vdash_3
\varphi$. Write $P(\varphi)$ for the generic presupposition (if it
exists), and $A(\varphi)$ for the assertion.
\end{defn}
%%%
It is not a priori clear that a proposition has a generic
presupposition. A case in point is the truth--predicate.
%%%
\index{$\equiv_3$}%%%
%%%%
Write $\varphi \equiv_3 \chi$ if $\beta(\varphi) = \beta(\chi)$ 
for all $\beta$.
%%%
\begin{prop}
$\varphi \equiv_3 \; {A(\varphi)\downarrow P(\varphi)}$.
\end{prop}
%%%
\index{projection algorithm}%%%
%%%%
The \textbf{projection algorithm} is a procedure that assigns generic
presuppositions to complex propositions by induction over their
structure. Table~\ref{tab:projection} shows a projection algorithm
for the connectives defined so far.
%%%
\begin{table}
\caption{The Projection Algorithm}%%
\label{tab:projection}
$$\begin{array}{l@{\; = \;}l@{\qquad}l@{\; = \;}l}
A(\nicht \varphi) & \nicht A(\varphi) & P(\nicht \varphi) &
        P(\varphi) \\
A(\varphi\und\chi) & A(\varphi)\und A(\chi) &
    P(\varphi\und\chi) & P(\varphi) \und P(\chi) \\
A(\varphi\oder\chi) & A(\varphi)\oder A(\chi) &
    P(\varphi\oder\chi) & P(\varphi) \und P(\chi) \\
A(\varphi\pf\chi) & A(\varphi) \pf A(\chi) &
    P(\varphi\pf\chi) & P(\varphi) \und P(\chi) \\
A(\varphi\und'\chi) & A(\varphi)\und A(\chi) &
    P(\varphi\und'\chi) & P(\varphi) \\
    \multicolumn{2}{r}{} &
    \multicolumn{2}{r}{\und (A(\varphi) \pf P(\chi))} \\
A(\varphi\oder'\chi) & A(\varphi) \oder A(\chi) &
    P(\varphi\oder'\chi) & P(\varphi) \\
    \multicolumn{2}{r}{} &
    \multicolumn{2}{r}{\und (\nicht A(\varphi) \pf P(\chi))} \\
A(\varphi\pf'\chi) & (A(\varphi)\und P(\varphi)) &
    P(\varphi\pf'\chi) & P(\varphi) \\
    \multicolumn{2}{r}{\pf A(\chi)} & \multicolumn{2}{r}{\und
    (A(\varphi) \pf P(\chi))}
\end{array}$$
\end{table}
%%%%
It is an easy matter to define projection algorithms for all
connectives. The prevailing intuition is that the three valued
character of {\tt and}, {\tt or} and {\tt if$\dotsb$then}
%%%%
\index{context change}%%%
%%%
is best explained in terms of \textbf{context change}. A
%%%
\index{text}%%
%%%
\textbf{text} is a sequence of propositions, say $\Delta =
\auf \delta_i : i < n\zu$. A text is \textbf{coherent}
%%%
\index{text!coherent}%%
%%%
if for every $i < n$: $\auf \delta_j : j < i\zu \vdash_3
\delta_i \oder \nicht\delta_i$. In other words, every member
is either true or false given that the previous propositions are
considered true. (Notice that the order is important now.)
In order to extend this to parts of the $\delta_j$
we define the \textbf{local context} as follows.
%%%
\begin{defn}
%%%
\index{context!local}%%
%%%%
Let $\Delta = \auf \delta_i : i < n\zu$. The \textbf{local context} of
$\delta_j$ is $\auf \delta_i : i < j\zu$. For a subformula occurrence
of $\delta_j$, the \textbf{local context} of that occurrence is defined
as follows.
%%%
\begin{dingautolist}{192}
\item If $\Sigma$ is the local context of $\varphi\und'\chi$ then
    (a) $\Sigma$ is the local context of $\varphi$ and (b)
    $\Sigma;\varphi$ is the local context of $\chi$.
\item If $\Sigma$ is the local context of $\varphi\pf'\chi$ then
    (a) $\Sigma$ is the local context of $\varphi$ and (b)
    $\Sigma;\varphi$ is the local context of $\chi$.
\item If $\Sigma$ is the local context of $\varphi\oder'\chi$ then
    (a) $\Sigma$ is the local context of $\varphi$ and (b)
    $\Sigma;\nicht\varphi$ is the local context of $\chi$.
\end{dingautolist}
%%%
\index{bivalence}%%%%%
%%%%
$\delta_j$ is \textbf{bivalent in its local context} if for all
valuations that make all formulae in the local context true,
$\delta_j$ is true or false.
\end{defn}
%%%
The presupposition of $\varphi$ is the formula $\chi$ such that
$\varphi$ is bivalent in the context $\chi$, and which implies all
other formulae that make $\varphi$ bivalent. It so turns out that
the context dynamics define a three valued extension of a
2--valued connective, and conversely. The above rules are an exact
match. Such formulations have been given by Irene
Heim~\shortcite{heim:context}, %%
%%%
\index{Heim, Irene}%%%
%%%
Lauri Karttunen~\shortcite{karttunen:context} 
%%%
\index{Kartttunen, Lauri}%%%
%%%
and also Jan van Eijck~\shortcite{vaneijck:presupposition}. 
%%%
\index{van Eijck, Jan}%%%
%%%
It is easy to understand 
this in computer programs. A computer program may contain clauses 
carrying presuppositions (for example clauses involving divisions), 
but it need not fail. For if whenever a clause carrying a 
presupposition is evaluated, that presupposition 
is satisfied, no error ever occurs at runtime. In
other words, the local context of that clause satisfies the
presuppositions. What the local context is, is defined by the
evaluation procedure for the connectives. In computer languages, 
the local context is always to the left. But this is not necessary.
The computer evaluates $\alpha\; \mathsf{and}\; \beta$ by first
evaluating $\alpha$ and then $\beta$ only if $\alpha$ is true --- 
but it could also evaluate $\beta$ first and then evaluate $\alpha$ whenever 
$\beta$ is true. In \cite{kracht:control} it is shown
that in the definition of the local context all that needs to be
specified is the directionality of evaluation. The rest follows 
from general principles. Otherwise one gets connectives that extend 
the boolean connectives in an improper way (see below on that notion).

The behaviour of presuppositions in quantification and propositional 
attitude reports is less straightforward. We shall only give a sketch.
%%%
\begin{align}
\label{eq:4713} & \mbox{\tt Every bachelor of the region 
	got a letter from} \\\notag
   & \quad \mbox{\tt that marriage agency.} \\
\label{eq:4714} & \mbox{\tt Every person in that region is a bachelor.} \\
\label{eq:4715} & \mbox{\tt John believes that his neighbour is a
    bachelor.}
\end{align}
%%%
We have translated {\tt every} using the quantifier $\forall$ in 
predicate logic. We wish to extend it to a three--valued quantifier, 
which we also call $\forall$. \eqref{eq:4713} is true even if not 
everybody in the region is a bachelor; in fact, it is true exactly 
if there is no non--bachelor. Therefore we say that $(\forall x)\varphi$ 
is true iff there is no $x$ for which $\varphi(x)$ is false. 
$(\forall x)\varphi$ is false if there is an $x$ for which 
$\varphi(x)$ is false. Thus, the presupposition effectively 
restricts the range of the quantifier.  $(\forall x)\varphi$ 
is bivalent. This predicts that \eqref{eq:4714}
has no presuppositions. On \eqref{eq:4715} the intuitions vary.
One might say that it does not have any presuppositions, or else
that it presupposes that the neighbour is a man (or perhaps: that
John believes that his neighbour is a man). This is deep water
(see \cite{geurts:presupposition}).

Now we come to the second interpretation of partiality, namely
{\it ignorance}. Let $\star$ now stand for the fact that the
truth--value is not known. Also here the resulting logic is not
unique. Let us take the example of a valuation $\beta$ that is
only defined on some of the variables. Now let us be given
a formula $\varphi$. Strictly speaking $\oli{\beta}(\varphi)$
is not defined on $\varphi$ if the latter contains a variable
that is not in the domain of $\beta$. On the other hand, there
are clear examples of propositions that receive a definite
truth--value no matter how we extend $\beta$ to a total function.
For example, even if $\beta$ is not defined on $p$, every extension
of it must make $p \oder \nicht p$ true. Hence, we might say
that $\beta$ also makes $p \oder \nicht p$ true. This is the idea
%%%
\index{supervaluation}%%
%%%%
of \textbf{supervaluations} by Bas van Fraassen. 
%%%
\index{van Fraassen, Bas}%%%
%%%
Say that
$\varphi$ is \textbf{sv--true} (\textbf{sv--false}) under $\beta$ if
$\varphi$ is true under every total $\gamma \supseteq \beta$.
If $\varphi$ is neither sv--true nor sv---false, call it
\textbf{sv--indeterminate}. Unfortunately, there is no logic
to go with this approach. Look at the interpretation of {\tt or}.
Clearly, if either $\varphi$ or $\chi$ is sv--true, so is
their disjunction. However, what if both $\varphi$ and $\chi$
are sv--indeterminate?
%%
\begin{equation}
\begin{array}{l|lll}
\cup  & 0     & 1 & \star \\\hline
0     & 0     & 1 & \star \\
1     & 1     & 1 & 1     \\
\star & \star & 1 & ?
\end{array}
\end{equation}
%%
The formula $p \oder q$ is sv--indeterminate under the empty valuation.
It has no definite truth--value, because both $p$ and $q$ could turn out
to be either true or false. On the other hand, $p \oder \nicht p$ is
sv--true under the empty valuation, even though both $p$ and $\nicht p$
are sv--indeterminate. So, the supervaluation approach is not so
well--suited. Stephen Kleene actually had the idea of doing a worst
case interpretation: if you can't always say what the value is, fill
%%%
\index{Kleene, Stephen C.}%%%
\index{connective!strong Kleene}%%%
%%%
in $\star$. This gives the so--called \textbf{Strong Kleene Connectives}
(the weak ones are like Bochvar's).
%%
\index{$\cap^{\diamond}$, $\cup^{\diamond}$}%%%
%%
\begin{equation}
\begin{array}{l|l}
        & - \\\hline
0       & 1 \\
1       & 0 \\
\star   &  \star
\end{array}
\qquad
\begin{array}{l|lll}
\cap^{\diamond}    & 0     & 1     & \star \\\hline
0         & 0     & 0     & 0     \\
1         & 0     & 1     & \star \\
\star     & 0     & \star & \star
\end{array}
\qquad
\begin{array}{l|lll}
\cup^{\diamond}  & 0     & 1     & \star \\\hline
0       & 0     & 1     & \star \\
1       & 1     & 1     & 1     \\
\star   & \star & 1     & \star
\end{array}
%%\qquad
%\begin{array}{l|lll}
%\supset^{\diamond}  & 0     & 1     & \star \\\hline
%0          & 1     & 1     & 1     \\
%1          & 0     & 1     & \star \\
%\star      & \star & 1     & \star
%\end{array}
\end{equation}
%%%
These connectives can be defined in the following way.
Put $1^{\diamond} := \{1\}$, $0^{\diamond} := \{0\}$
and $\star := \{0,1\}$. Now put
%%
\begin{equation}
f^{\diamond}(x_0^{\diamond},x_1^{\diamond}) :=
f[x_0^{\diamond} \times x_1^{\diamond}]
\end{equation}
%%
For example, $\cup^{\diamond}(\auf 0^{\diamond},\star^{\diamond}\zu)
= \cup[\{0\} \times \{0,1\}] = \{0 \cup 0, 0 \cup 1\} = \{0,1\} =
\star^{\diamond}$. So, we simply take sets of truth--values and
calculate with them.

A more radical account of ignorance is presented by constructivism 
and intuitionism. A constructivist denies that the truth or falsity 
of a statement can always be assessed directly. In particular, an 
existential statement is true only if we produce an instance that 
satisfies it. A universal statement can be considered true only if we
possess a proof that any given element satisfies it. For example,
Goldbach's conjecture is that every even number greater than 2 is
the sum of two primes. According to a constructivist, at present
it is neither true nor false. For on the one hand we have no proof
that it holds, on the other hand we know of no even number greater
than 2 which is not the sum of two primes. Both
constructivists and intuitionists unanimously reject axiom (a2).
(Put $\bot$ for $p_1$. Then in conjunction with the other rules 
this gives $(\nicht p_0 \pf p_0) \pf p_0$. This corresponds to 
the \textbf{Rule of Clavius}: from $\nicht p_0 \pf p_0$ conclude 
$p_0$.) They also reject $p \oder \nicht p$, the so--called 
%%%
\index{law of the excluded middle}%%
%%%
\textbf{Law of the Exluded Middle}. The difference between a 
constructivist and an intuitionist is the treatment of negative 
evidence. While a
constructivist accepts basic negative evidence, for example, that
this lemon is not green, for an intuitionist there is no such
thing as direct evidence to the contrary. We only witness the
absence of the fact that the lemon is green. Both, however, are
reformist in the sense that they argue that the mathematical
connectives {\tt and}, {\tt or}, {\tt not}, and {\tt if$\dotsb$then} 
have a different meaning. However, one can actually give a
reconstruction of both inside classical mathematics. We shall deal
first with intuitionism. Here is a new set of connectives, 
defined with the help of $\qu$, which satisfies $\mathsf{S4}$.
%%
\begin{equation}
\begin{split}
\nicht^i \varphi & := \qu (\nicht \varphi) \\
\varphi \oder^i \chi & := \varphi \oder \chi \\
\varphi \und^i \chi & := \varphi \und \chi \\
\varphi \pf^i \chi & := \qu (\varphi \pf \chi)
\end{split}
\end{equation}
%%
Call an I--proposition a proposition formed from variables and
$\bot$ using only the connectives just defined.
%%
\begin{defn}
%%%
\index{I--model}%%
%%%
An \textbf{I--model} is a pair $\auf P, \leq, \beta\zu$, where
$\auf P, \leq\zu$ is a partially ordered set and $\beta(p) =
\, \uparrow\!\beta(p)$ for all variables $p$.
\end{defn}
%%
Intuitively, the nodes of $P$ represent stages in the development
of knowledge. Knowledge develops in time along $\leq$. We say that
$x$ \textbf{accepts} $\varphi$ if $\auf P, \leq, x, \beta\zu \vDash
\varphi$, and that $x$ \textbf{knows} $\varphi$ if $\auf P, \leq,
x, \beta\zu \vDash \qu \varphi$. By definition of $\beta$, once a
proposition $p$ is accepted, it is accepted for good and therefore
considered known. Therefore G\"odel simply translated variables $p$ 
by the formula $\qu p$. Thus, intuitionistically the statement that 
$p$ may therefore be understood as `$p$ is known' rather than `$p$ 
is accepted'. The systematic conflation of knowledge and simple 
temporary acceptance as true is the main feature of intuitionistic 
logic.
%%
\begin{prop}
Let $\auf P, \leq, \beta\zu$ be an I--model and $\varphi$ an
I--pro\-po\-si\-tion. Then $\oli{\beta}(\varphi) =\; {\uparrow
\oli{\beta}(\varphi)}$ for all $\varphi$.
\end{prop}
%%
Constructivism in the definition by Nelson adds to intuitionism
a second valuation for those variables that are definitely
rejected, and allows for the possibility that neither is the
case. (However, nothing can be both accepted and rejected.)
This is reformulated as follows.
%%
\begin{defn}
%%%
\index{C--model}%%
%%%
A \textbf{C--model} is a pair $\auf P, \leq, \beta\zu$, where
$\auf P, \leq\zu$ is a partially ordered set and $\beta \colon V \times P
\pf \{0,1,\star\}$ such that if $\beta(p,v) = 1$ and $v \leq w$
then also $\beta(p,w) = 1$, and if $\beta(p,v) = 0$ and $v  \leq w$
then $\beta(p,w) = 0$. We write $\auf P, \leq, x, \beta\zu \vDash^+
p$ if $\beta(p,x) = 1$ and $\auf P, \leq, x, \beta\zu \vDash^- p$
if $\beta(p,x) = 0$.
\end{defn}
%%
We can interpret any propositional formula over 3--valued logic that
we have defined so far. We have to interpret $\qu$ and $\wD$,
however.
%%
\begin{equation}
\begin{split}
x \vDash^+ \qu \varphi & :\Dpf \text{ for no }
    y \geq x: y\vDash^- \varphi \\
x \vDash^- \qu \varphi & :\Dpf \text{ there is }
    y \geq x: y \vDash^- \varphi \\
x \vDash^+ \wD \varphi & :\Dpf \text{ there is }
    y \geq x: y\vDash^+ \varphi \\
x \vDash^- \wD \varphi & :\Dpf \text{ for no }
    y \geq x: y\vDash^+ \varphi
\end{split}
\end{equation}
%%
Now define the following new connectives.
%%
\begin{equation}
\begin{split}
\nicht^c \varphi & := \nicht \varphi \\
\varphi \oder^c \chi & := \varphi \oder^{\diamond} \chi \\
\varphi \und^c \chi & := \varphi \und^{\diamond} \chi \\
\varphi \pf^c \chi & := \qu (\varphi \pf^{\diamond} \chi)
\end{split}
\end{equation}
%%
In his data semantics (see \cite{veltman:conditionals}), Frank
Veltman 
%%%
\index{Veltman, Frank}%%%
%%%
uses constructive logic and proposes to interpret {\tt
must} and {\tt may} as $\qu$ and $\wD$, respectively. What is
interesting is that the set of points accepting $\wD \varphi$ is
lower closed but {\it not\/} necessarily upper closed, while the
set of points rejecting it is upper but not necessarily lower
closed. The converse holds with respect to $\qu$. This is natural,
since if our knowledge grows there are less things that {\it
may\/} be true but more that {\it must\/} be.

The interpretation of the arrow carries the germ of the relational
interpretation discussed here. A different strand of thought is
the theory of conditionals (see again \cite{veltman:conditionals}
and also \cite{gaerdenfors:flux}). The conditional $\varphi >
\chi$ is accepted as true under Ramsey's interpretation 
%%%
\index{Ramsey, Frank}%%%
%%%
if, on taking $\varphi$ as a hypothetical assumption (doing as if
$\varphi$ is the case) and performing the standard reasoning, we
find that $\chi$ is true as well. Notice that after this routine
of hypothetical reasoning we retract the assumption that
$\varphi$. In G\"ardenfors models there are no assignments in the
ordinary sense. A proposition $\varphi$ is mapped directly onto a
function, the update function $U_{\varphi}$. The states in the
G\"ardenfors model carry no structure.
%%%
\begin{defn}
%%%
\index{G\"ardenfors model}%%
%%%%
\index{G\"ardenfors, Peter}%%%
%%%
A \textbf{G\"ardenfors model} is a pair $\auf G, U\zu$, where
$G$ is a set, and $U \colon \Tm_{\Omega} \pf G^G$ subject to the
following constraints.
%%%
\begin{dingautolist}{192}
\item For all $\chi$: $U_{\chi} \circ U_{\chi} = U_{\chi}$.
\item For all $\varphi$ and $\chi$:
    $U_{\varphi} \circ U_{\chi} = U_{\chi} \circ U_{\varphi}$.
\end{dingautolist}
%%%
We say that $x \in G$ \textbf{accepts} $\varphi$ if $U_{\varphi}(x) = x$.
%%%
\end{defn}
%%%
Put $x \leq y$ iff
there is a finite set $\{\chi_i : i < n\}$ such that 
%%%
\begin{equation}
y = U_{\chi_0} \circ U_{\chi_1} \circ \dotsb \circ U_{\chi_{n-1}}(x)
\end{equation}
%%%
The reader may verify that this relation is reflexive and transitive.
If we require that $x = y$ iff $x$ and $y$ accept the
same propositions, then this ordering is also a partial ordering.
We can define as follows. If $U_{\chi} = U_{\varphi} \circ
U_{\psi}$ then we write $\chi = \varphi\und\psi$. Hence, if our
language actually has a conjunction, the latter is a condition on
the interpretation of it. To define $\pf$, G\"ardenfors does the
following. Suppose that for all $\varphi$ and $\chi$ there exists
a $\delta$ such that $U_{\varphi} \circ U_{\delta} = U_{\chi}
\circ U_{\delta}$. Then we simply put $U_{\varphi \dpf \psi}
:= U_{\delta}$. Finally, since $\varphi\pf\psi$ is equivalent to
$\varphi\dpf \varphi\und \psi$, once we have $\und$ and $\dpf$, we
can also define $\pf$. For negation we need to assume the
existence of an inconsistent state. The details need not concern
us here. Obviously, G\"ardenfors models are still more general
than data semantics. In fact, any kind of logic can be modelled by
a G\"ardenfors model (see the exercises).

{\it Notes on this section.} It is an often discussed problem whether
or not a statement of the form $(\forall x)\varphi$ is true if there
are no $x$ at all. Equivalently, in three valued logic, it might be
said that $(\forall x)\varphi$ is undefined if there is no $x$ such
that $\varphi(x)$ is defined.
%%%
\vplatz
\exercise
A three valued binary connective \ding{67} satisfies the
%%%
\index{Deduction Theorem}%%%
%%%%
Deduction Theorem if for all $\Delta$, $\varphi$ and
$\chi$: $\Delta; \varphi \vdash_3 \chi$ iff
$\Delta \vdash_3 \varphi \mbox{\ding{67}} \chi$. Establish all 
truth--tables for binary connectives that satisfy the Deduction 
Theorem. Does any of the implications defined above have this 
property?
%%%
\vplatz \exercise Let $\CL$ be a language and $\vdash$ a
structural consequence relation over $\CL$. Let $G_{\vdash}$ be
the set of theories of $\vdash$. For $\varphi \in \CL$, let
$U_{\varphi} \colon T \mapsto (T \cup \{\varphi\})^{\vdash}$. Show that
this is a G\"ardenfors model. Show that the set of formulae
accepted by all $T \in G_{\vdash}$ is exactly the set of
tautologies of $\vdash$.

 \chapter{PTIME Languages}
\thispagestyle{empty}
\label{kap4}
%
%
%
\section{Mildly--Context Sensitive Lan\-gua\-ges}
\label{kap4-1}
%
%
%
The introduction of the Chomsky hierarchy has sparked off a lot of
research into the complexity of formal and natural languages.
Chomsky's own 
%%%
\index{Chomsky, Noam}%%%
%%%
position was that language was not even of Type 1.
In transformational grammar, heavy use of context sensitivity and
deletion has been made. However, Chomsky insisted that these
grammars were not actually models of performance, neither of
sentence production nor of analysis; they were just models of
competence. They were theories of language or of languages,
couched in algorithmic terms. In the next chapter we shall study a
different type of theory, based on axiomatic descriptions of
structures. Here we shall remain with the algorithmic approach. If
Chomsky is right, the complexity of the generated languages is
only of peripheral interest and, moreover, cannot even be
established by looking at the strings of the language. Thus, 
if observable language data can be brought to bear on the 
question of the `language faculty', we actually need to have 
%%
\begin{dinglist}{43}
\item
a theory of the human language(s),
\item
a theory of human sentence production, and
\item
a theory of human sentence analysis (and understanding).
\end{dinglist}
%%
Namely, the reason that a language may fail to show its complexity 
in speech or writing is that humans simply are unable to produce
the more complex sentences, even though given enough further 
means they would be able to produce any of them. The same
goes for analysis. Certain sentences might be avoided not
because they are illegitimate but because they are misleading
or too difficult to understand. An analogy that might help is
the notion of a programming language. A computer is thought to be 
able to understand every program of a given computer language if 
it has been endowed with an understanding of the syntactic primitives 
and knows how to translate them into executable routines. Yet, some 
programs may simply be too large for the computer to be translated 
let alone executed. This may be remedied by giving it more memory
(to store the program) or a bigger processing unit (to be able
to execute it). None of the upgrading operations, however, seem
to touch on the basic ability of the computer to understand the
language: the translation or compilation program usually remains
the same. Some people have advanced the thesis that certain
monkeys possess the symbolic skills of humans but since they
cannot handle recursion, so that their ability to use language is
restricted to single clauses consisting of single word 
phrases (see for example \cite{haider:exaptiv} and 
\cite{haider:generativ}, Pages 8 -- 12).

One should be aware of the fact that the average complexity of 
spoken language is linear (= $O(n)$) for humans. We understand 
sentences as they are uttered, and typically we seem to be able 
to follow the structure and message word by word. To conclude that 
therefore human languages must be regular is premature. For one 
thing, we might just get to hear the easy sentences because they 
are also easy to generate: it is humans who talk to humans. Additionally, 
it is not known what processing device the human brain is. Suppose 
that it is a finite state automaton. Then the conclusion is 
certainly true. However, if it is a pushdown automaton, the language 
can be deterministically context free. More complex devices can be 
imagined giving rise to even larger classes of languages that can 
be parsed in linear time. This is so since it is not clear that what 
is one step for the human brain also is one step for, say, a Turing 
machine. It is known that the human brain works with massive use of 
parallelism, for example.

Therefore, the problem with the line of approach advocated by
Chomsky is that we do not possess a reliable theory of human
sentence processing let alone of sentence production (see
\cite{levelt:speaking} for an overview of the latter).
Without them, however, it is impossible to assess the
correctness of any proposed theory of grammar. Many people have
therefore ignored this division of labour into three faculties
(however reasonable that may appear) and tried to assess the
complexity of the language as we see it. Thus let us ask once
more:
%%
\begin{quote}
How complex is human language (are human languages)?
\end{quote}
%%
While the Chomsky hierarchy has suggested measuring complexity in
terms of properties of rules, it is not without interest to try to
capture its complexity in terms of resources (time and space
complexity). The best approximation that we can so far give is this. 
%%
\begin{quote}
Human languages are in \textbf{PTIME}.
\end{quote}
%%
In computer science, \textbf{PTIME} problems are also called `tractable',
since the time consumption grows slowly. On the other hand,
\textbf{EXPTIME} problems are called `intractable'. Their time
consumption grows too fast. In between the two lie the classes
\textbf{NPTIME} and \textbf{PSPACE}. Still today it is not known whether
or not \textbf{NPTIME} is contained in (and hence equal to) \textbf{PTIME}.
Problems which are \textbf{NPTIME}--complete usually do possess algorithms 
that run (deterministically) in polynomial time --- on the average.

Specifically, Aravind Joshi 
%%%
\index{Joshi, Aravind}%%%
%%%
has advanced the claim that languages 
are what he calls `mildly context sensitive' (see \cite{joshi-1985}). 
Mildly context sensitive languages are characterized as follows.
%%%
\begin{defn}
%%%
\index{constant growth property}%%
%%%
$L \subseteq A^{\ast}$ has the \textbf{constant growth property} 
if it is finite or there is a number $c_L$ such that for every 
$\vec{x}\in L$ there is a $\vec{y}\in L$ such that 
$|\vec{x}| < |\vec{y}| \leq |\vec{x}| + c_L$. 
\end{defn}
%%
\begin{dingautolist}{192}
\item Every context free language is mildly context sensitive.
    There are mildly context sensitive languages which are
    not context free.
\item Mildly context sensitive languages can be recognized in
    deterministic polynomial time.
\item There is only a finite number of crossed dependency types.
\item Mildly context sensitive languages have the constant
    growth property. 
\end{dingautolist}
%%
These conditions are not very strong except for the second.
It implies that the mildly context sensitive languages
form a proper subset of the context sensitive languages.
\ding{192} needs no comment. \ding{195} is quite weak.
Moreover, it seems that for every natural language there is a 
number $d_L$ such that for every $n \geq d_L$ there is a string 
of length $n$ in $L$. Rambow~\shortcite{rambow:formal} 
%%%
\index{Rambow, Owen}%%%
%%%
proposes to replace it with the requirement of semilinearity, but 
that seems to be too strong (see Michaelis and 
Kracht~\shortcite{michaeliskracht:semilinearity}).
%%%
\index{Michaelis, Jens}%%
\index{Kracht, Marcus}%%%
%%%
Also \ding{194} is problematic. What exactly is a crossed dependency 
type? In this chapter we shall study grammars in which the notion 
of structure can be defined as with context free languages. 
Constituents are certain subsets of disjoint (occurrences of) 
subwords. If this definition is accepted, \ding{194} can be 
interpreted as follows: there is a number $n$ such that a given 
constituent has no more than $n$ parts. This is certainly not what Joshi 
%%%
\index{Joshi, Aravind}%%%
%%%
had in mind when formulating his conditions, but it is certainly not 
easy to come up with a definition that is better than this one and
as clear.

So, the conditions are problematic with the exception of \ding{193}.
Notice that \ding{193} implies \ding{192}, and, as we shall see, also 
\ding{194} (if only weak equivalence counts here). \ding{195}
shall be dropped. In order not to create confusion we shall call a 
language a \textbf{PTIME language} 
%%%%
\index{language!PTIME}%%
%%%%
if it has a deterministic polynomial time recognition algorithm (see 
Definition~\ref{defn:complang}).
In general, we shall also say that a function
$f \colon A^{\ast} \pf B^{\ast}$ is in \textbf{PTIME} if there is a
deterministic Turing machine which computes that function.
Almost all languages that we have considered so far are
\textbf{PTIME} languages. This shall emerge from the theorems
that we shall prove further below.
%%%
\begin{prop}
Every context free language is in \textbf{PTIME}.
\end{prop}
%%
This is a direct consequence of Theorem~\ref{thm:cky}.
However, we get more than this.
%%
\begin{prop}
\label{prop:schnitt}
Let $A$ be a finite alphabet and $L_1$, $L_2$ languages
over $A$. If $L_1, L_2 \in$ \textbf{PTIME} then so is
$A^{\ast} - L_1$, $L_1 \cap L_2$ and $L_1 \cup L_2$.
\end{prop}
%%
The proof of this theorem is very simple and left as an
exercise. So we get that the intersection of CFLs,
for example $\{\mbox{\tt a}^n \mbox{\tt b}^n \mbox{\tt c}^n :
n \in \omega\}$, are \textbf{PTIME} languages. Condition \ding{193} 
for mildly context sensitive languages is satisfied by the class
of {\bf PTIME} languages. Further, we shall show that the
full preimage of a \textbf{PTIME} language under the Parikh--map
is again a \textbf{PTIME} language. To this end we shall
identify $M(A)$ with the set of all strings of the form
$\prod_{i < n} \mbox{\tt a}_i^{p_i}$.  The Parikh--map is
identified with the function $\pi \colon A^{\ast} \pf A^{\ast}$,
which assigns to a string $\vec{x}$ the string
$\mbox{\tt a}_0^{p_0} \conc \mbox{\tt a}_1^{p_1} \conc
\dotsb \conc \mbox{\tt a}_{n-1}^{p_{n-1}}$, where
$p_j$ is the number of occurrences of $\mbox{\tt a}_j$ in
$\vec{x}$. Now take an arbitrary polynomial time computable
function $g \colon A^{\ast} \pf 2$. Clearly, $g \restriction 
M(A)$ is also in {\bf PTIME}. The preimage of $1$ under this 
function is contained in the image of $\pi$. $g^{-1}(1) 
\cap M(A)$ can be thought of in a natural way as a subset of 
$M(A)$.
%%
\begin{thm}
\label{thm:semi}
Let $L \subseteq M(A)$ be in \textbf{PTIME}. Then $\pi^{-1}[L]$, 
the full preimage of $L$ under $\pi$, also is in \textbf{PTIME}. If
$L$ is semilinear, $\pi^{-1}[L]$ is in \textbf{PTIME}.
\end{thm}
%%
The reader is warned that there nevertheless are semilinear languages
which are not in {\bf PTIME}. For {\bf PTIME} is countable, but
there are uncountably many semilinear languages (see
Exercise~\ref{ex:semilincont}). Theorem~\ref{thm:semi} follows 
directly from
%%
\begin{thm}
Let $f \colon B^{\ast} \pf A^{\ast}$ be in \textbf{PTIME} and
$L \subseteq A^{\ast}$ in \textbf{PTIME}. Then
$M := f^{-1}[L]$ also is in \textbf{PTIME}.
\end{thm}
%%
\proofbeg
By definition $\chi_L \in \mbox{\bf PTIME}$. Then
$\chi_M = \chi_L \circ f \in \mbox{\bf PTIME}$. This is
the characteristic function of $M$.
\proofend

Another requirement for mildly context sensitive languages
was the constant growth property. We leave it to the reader 
to show that every semilinear language has the constant growth 
property but that there are languages which have the constant 
growth property without being semilinear.

We have introduced the Polish Notation for terms in
Section~\ref{kap1}.\ref{einseins}. Here we shall introduce a somewhat
exotic method for writing down terms, which has been motivated
by the study of certain Australian languages (see \cite{ebertkracht}). 
Let $\auf F, \Omega\zu$ be a finite signature. Further, let 
%%%
\index{$\Omega_{\intercal}$}%%
%%%
\begin{equation}
\Omega_{\intercal} := \max \{\Omega(f) : f \in F\}
\end{equation}
%%%
Inductively, we assign to every term $t$ a set 
$M(t) \subseteq (F \cup \{\mbox{\tt 0},
\mbox{\tt 1},\dotsc,\Omega_{\intercal} - 1\})^{\ast}$:
%%
\begin{dingautolist}{192}
\item If $t = f$ with $\Omega(f) = 0$ then put $M(f) := \{f\}$.
\item If $t = f(s_0, \dotsc, s_{\Omega(f)-1})$ then put
%%
    \begin{equation*}
     M(t) := \{f\} \cup \bigcup_{i < \Omega(f)}
        \{\vec{x}\conc i : \vec{x} \in M(s_i)\}
    \end{equation*}
\end{dingautolist}
%%
An element of $M(t)$ is a product $f \conc \vec{y}$,
where $f \in F$ and $\vec{y}\in \Omega_{\intercal}^{\ast}$. We call
$f$ the {\bf main symbol} and $\vec{y}$ its {\bf key}.
%%%
\index{key}%%
\index{A--form}%%
%%%
We choose a new symbol, {\tt \#}. 
Now we say that $\vec{y}$ is an {\bf A--form} of $t$ if $\vec{y}$
is the product of the elements of $M(t)$ in an arbitrarily chosen
(nonrepeating) sequence, separated by {\tt \#}. For example, 
let $t := \mbox{\tt (((x+a)-y)+(z-c))}$. Then
%%
\begin{equation}
M(t) = \{\mbox{\tt +}, \mbox{\tt -0}, \mbox{\tt -1},
\mbox{\tt +00}, \mbox{\tt x000}, \mbox{\tt a001},
\mbox{\tt y01}, \mbox{\tt z10}, \mbox{\tt c11}\}
\end{equation}
%%
Hence the following string is an A--form of $t$:
%%
\begin{equation}
\mbox{\tt c11\#z10\#+00\#-0\#-1\#y01\#x000\#a001\#+}
\end{equation}
%%
\begin{thm}
Let $\auf F, \Omega\zu$ be a finite signature and
$L$ the language of A--forms of terms of this signature.
Then $L$ is in \textbf{PTIME}.
\end{thm}
%%
\proofbeg 
For each A--form $\vec{x}$ there is a unique term $t$ such that 
$\vec{x}$ is the A--form of $t$, and there is a method to calculate 
$M(t)$ on the basis of $\vec{x}$. One simply has to segment $\vec{x}$ into 
correctly formed parts. These parts are maximal sequences consisting 
of a main symbol and a key, which we shall now simply call {\bf stalks}. 
The segmentation into stalks is unique. We store $\vec{x}$ on a read and 
write tape $\tau_i$. Now we begin the construction of $t$. $t$ will be 
given in Polish Notation. $t$ will be constructed on Tape $\tau_o$ in 
left--to--right order. We will keep track of the unfinished function 
symbols Tape on $\tau_s$. We search through the keys (members of 
$\Omega_{\intercal}^{\ast}$) in lexicographic order. On a separate tape, 
$\tau_k$, we keep note of the current key. 
%%%
\begin{dingautolist}{192}
\item $\tau_s := \varepsilon$, $\tau_k := \varepsilon$, $\tau_o := 
	\varepsilon$.
\item Match $\tau_i = \vec{y}\mbox{\tt \#}f\conc \tau_k\mbox{\tt 
\#}\vec{z}$. If match succeeds, put $\tau_i := 
	\vec{y}\mbox{\tt \#}\vec{z}$, $\tau_s := \tau_s \conc f$, 
	$\tau_o := \tau_o\conc f$. Else exit: `String is not an 
	A--form.'
\item Let $g$ be the last symbol of $\tau_s$, $n$ the last symbol of 
	$\tau_k$. 
\begin{itemize}
	\item
	If $n = \Omega(g) - 1$, $\tau_s := \tau_s/g$, $\tau_k := 
	(\tau_k/n)$. 
\item Else
	$\tau_s := \tau_s$, $\tau_k := (\tau_k/n)\conc (n+1)$. 
\end{itemize}
\item If $\tau_k = \tau_s = \varepsilon$, go to \ding{196} else 
	go to \ding{193}.
\item If $\tau_i = \varepsilon$ exit: `String is an A--form.' 
	Else exit: `String is not an A--form.'
\end{dingautolist}
%%%
It is left to the reader to check that this algorithm does what it 
is supposed to do. Polynomial runtime is obvious. 
\proofend

Ebert and Kracht~\shortcite{ebertkracht} 
%%%
\index{Ebert, Christian}%%%
\index{Kracht, Marcus}%%%
%%%
show that this algorithm requires $O(n^{3/2} \log n)$ time to 
compute.  Now we shall start the proof of an important 
theorem on the characterization of \textbf{PTIME} languages. An 
important step is a theorem by Chandra, Kozen and Stockmeyer
%%%
\index{Chandra, A.~K.}%%
\index{Stockmeyer, L.~J.}%%%
\index{Kozen, Dexter C.}%%%
%%%%
\shortcite{chandraetal:alternation}, which characterizes the class
\textbf{PTIME} in terms of space requirement. It uses 
special machines, which look almost like Turing machines
but have a special way of handling parallelism. Before we can
%%%
\index{Turing machine!logarithmically space bounded}%%
%%
do that, we introduce yet another class of functions. We say that a
function $f \colon A^{\ast} \pf B^{\ast}$ is in \textbf{LOGSPACE} if
%%%
\index{\textbf{LOGSPACE}}%%%
%%%
it can be computed by a so called deterministic logarithmically
bounded Turing machine. Here, a Turing machine is called
\textbf{logarithmically space bounded} if it has $k + 2$ tapes (where
$k$ may be $0$) such that the length of the tapes
number 1 through $k$ is bounded by $\log_2 |\vec{x}|$.
Tape 0 serves as the input tape, Tape $k+1$ as the output tape.
Tape 0 is read only, Tape $k+1$ is write only. At the end of the
computation, $T$ has to have written $f(\vec{x})$ onto that tape.
(Actually, a moment's reflection shows that we may assume that the
length of the intermediate tapes is bounded by $c \log_2 |\vec{x}|$, 
for some $c > 0$, cf. also Theorem~\ref{thm:speedup}.) This means 
that if $\vec{x}$ has length 12 the tapes $2$ to $k + 1$ have length 
$3$ since $3 < \log_2 12 < 4$. It need not concern us further why 
this restriction makes sense.  We shall see in 
Section~\ref{kap4}.\ref{kap4-2} that it is well--motivated. We 
emphasize that $f(\vec{x})$ can be arbitrarily large. It is
not restricted at all in its length, although we shall see later
that the machine cannot compute outputs that are too long
anyway. The reader may reflect on the fact that we may require
the machine to use the last tape only in this way: it moves
strictly to the right without ever looking at the previous cells
again. Further, we can see to it that the intermediate tapes
only contain single binary numbers.
%%
\begin{defn}
Let $A^{\ast}$ be a finite alphabet and $L \subseteq
A^{\ast}$. We say that $L$ is in \textbf{LOGSPACE} if
$\chi_L$ is deterministically \textbf{LOGSPACE}--computable.
\end{defn}
%%
\begin{thm}
Let $f \colon A^{\ast} \pf B^{\ast}$ be \textbf{LOGSPACE}--computable.
Then $f$ is in \textbf{PTIME}.
\end{thm}
%%%
\proofbeg
We look at the configurations of the machine (see 
Definition~\ref{defn:configuration}). A configuration
is defined with the exception of the output tape. It consists
of the positions of the read head of the first tape and the
content of the intermediate tapes plus the position of the
read/write heads of the intermediate tapes. Thus the configurations
are $k$--tuples of binary numbers of length $\leq c \log_2 |\vec{x}|$, 
for some $c$. A position on a string likewise corresponds to a binary 
number. So we have $k+1$ binary numbers and there are at most
%%
\begin{equation}
2^{(k+1)c \log_2 |\vec{x}|} = |\vec{x}|^{c(k+1)}
\end{equation}
%%
of them. So the machine can calculate at most $|\vec{x}|^{c(k+1)}$
steps. For if there are more the machine is caught in a loop, and
the computation does not terminate. Since this was excluded,
there can be at most polynomially many steps.
\proofend

Since $f$ is polynomially computable we immediately get that
$|f(\vec{x})|$ is likewise polynomially bounded.  This shows
that a space bound implies a time bound.  (In general, if $f(n)$
is the space bound then $c^{f(n)}$ is the corresponding time bound 
for a certain $c$.)

We have found a subclass of \textbf{PTIME} which is defined by its space 
consumption. Unfortunately, these classes cannot be shown to be equal. 
(It has not been disproved but is deeemed unlikely that they are equal.) 
We have to do much more work. For now, however, we remark the following.
%%
\begin{thm}
Suppose that $f \colon A^{\ast} \pf B^{\ast}$ and
$g \colon B^{\ast} \pf C^{\ast}$ are \textbf{LOGSPACE} computable.
Then so is $g \circ f$.
\end{thm}
%%
\proofbeg
By assumption there is a logarithmically space bounded
deterministic $k+2$--tape machine $T$ which computes $f$ and a
logarithmically space bounded deterministic $\ell + 2$--tape
machine $U$ which computes $g$. We cascade these machines in
the following way. We use $k + \ell + 3$ tapes, of which the
first $k+2$ are the tapes of $T$ and the last $\ell +2$ the
tapes of $U$. We use Tape $k + 1$ both as the output tape of 
$T$ and as the input tape of $U$. The resulting machine is 
deterministic but not necessarily logarithmically space bounded. 
The problem is Tape $k+1$. However, we shall now demonstrate that 
this tape is not needed at all. For notice that $T$ cannot but move 
forward on this tape and write on it, while $U$ on the other hand 
can only progress to read the input. Now rather than having Tape $k$ 
ready, it would be enough for $U$ if it can access the symbol number $i$ on 
the output tape of $T$ on request. Clearly, as $T$ can compute that 
symbol, $U$ only needs to communicate the request to $T$ by issuing 
$i$ in binary. (This takes only logarithmic space. For we have 
$|f(\vec{x})| \leq p(|\vec{x}|)$ for some polynomial $p$
for the length of the output computed by $T$, so we have
$\log_2 |f(\vec{x})| \leq \lambda \log_2 |\vec{x}|$ for some
natural number $\lambda$.) The proof follows once we make this 
observation: there is a machine $T'$ that computes the $i$th 
symbol of the output tape of $T$, given the input for $T$ input 
and $i$, using only logarithmic space. The rest is simple: everytime 
$U$ needs a symbol, it calls $T'$ issuing $i$ in binary. The global 
input $T'$ reads from $U$'s input tape.
\proofend

This proof is the key to all following proofs. We shall now show
that there is a certain class of problems which are, as one says,
{\it complete\/} with respect to the class {\bf PTIME} modulo {\bf
LOGSPACE}--reductions. 

An $n$--\textbf{ary boolean function} 
%%%
\index{boolean function}%%%
%%%
is an arbitrary function
$f \colon 2^n \pf 2$. Every such function is contained in the
polynomial clone of functions generated by the functions $\cup$,
$\cap$ and $-$ (see Exercise~\ref{exercise:basis}). We shall now 
assume that $f$ is composed from projections using the functions 
$\cap$, $\cup$, and $-$. For example, let 
$f(x_0, x_1, x_2) := - (- x_2 \cap (x_1 \cup x_0))$.
Now for the variables $x_0$, $x_1$ and $x_2$ we insert concrete
values (either 0 or 1). Which value does $f$ have? This problem 
can clearly be solved in {\bf PTIME}. However, the formulation of 
the problem is a delicate affair. Namely, we want to think of 
$f$ not as a string, but as a network. (The difference is that 
in a network every subterm needs to be represented only once.)
To write down networks, we shall have to develop a more elaborate 
coding. Networks are strings over the alphabet 
$W := \{\mbox{\mtt (}, \mbox{\mtt )}, \mbox{\mtt ,},
\mbox{\mtt 0}, \mbox{\mtt 1}, \mbox{\mtt a}, \mbox{\mtt b},
\mbox{\mtt\symbol{4}}, \mbox{\mtt\symbol{31}}, \mbox{\mtt\symbol{5}}\}$. 
%%%
\index{cell}%%
%%%
A {\bf cell} is a string of the form
{\mtt ($\vec{\alpha}$,$\vec{\varepsilon}$,$\vec{\eta}$,\symbol{31})},
{\mtt ($\vec{\alpha}$,$\vec{\varepsilon}$,$\vec{\eta}$,\symbol{4})},
or of the form {\mtt ($\vec{\alpha}$,$\vec{\varepsilon}$,\symbol{5})},
where $\vec{\alpha}$ is a binary sequence --- written in the alphabet
$\{\mbox{\tt a}, \mbox{\tt b}\}$ ---, and
$\vec{\varepsilon}, \vec{\eta}$ either are binary strings
(written down using the letters {\tt a} and {\tt b} in place
of {\tt 0} and {\tt 1}) or a single symbol of the form {\tt 0}
or {\tt 1}. $\vec{\alpha}$ is called the \textbf{number} 
%%%
\index{number}
%%%
of the cell and $\vec{\varepsilon}$ and $\vec{\eta}$ the 
\textbf{argument key},
%%%
\index{argument key}%%%
%%%
unless it is of the form {\tt 0} or {\tt 1}.  Further, we assume
that the number represented by $\vec{\varepsilon}$ and $\vec{\eta}$
is smaller than the number represented by $\vec{\alpha}$. (This
makes sure that there are no cycles.) A sequence of cells is called 
a \textbf{network} 
%%%
\index{network}%%%
%%%
if (a) there are no two cells with identical number,
(b) the numbers of cells are the numbers from 1 to a
certain number $\nu$, and (c) for every cell with
argument key $\vec{\varepsilon}$ (or $\vec{\eta}$) there
is a cell with number $\vec{\varepsilon}$ ($\vec{\eta}$).
The cell with the highest number is called the \textbf{goal}
%%%
\index{network!goal}%%
%%%
of the network. Intuitively, a network defines a boolean
function into which some constant values are inserted for
the variables. This function shall be evaluated.
With the cell number $\vec{\alpha}$ we associate 
a value $w_{\vec{x}}(\vec{\alpha})$ as follows.
%%
\begin{equation}
w_{\vec{x}}(\vec{\alpha}) := 
%%
\begin{cases}
w_{\vec{x}}(\vec{\varepsilon}) \cup w_{\vec{x}}(\vec{\eta}) &
\text{ if $\vec{x}$ contains 
{\mtt ($\vec{\alpha}$,$\vec{\varepsilon}$,$\vec{\eta}$,\symbol{31})},} \\
%%
w_{\vec{x}}(\vec{\varepsilon}) \cap w_{\vec{x}}(\vec{\eta}) &
\text{ if $\vec{x}$ contains 
{\mtt ($\vec{\alpha}$,$\vec{\varepsilon}$,$\vec{\eta}$,\symbol{4})},} \\
%%
- w_{\vec{x}}(\vec{\varepsilon}) &
\text{ if $\vec{x}$ contains 
{\mtt ($\vec{\alpha}$,$\vec{\varepsilon}$,\symbol{5})}.}
\end{cases}
%%
\end{equation}
%%
We write $w(\vec{\alpha})$ in place of $w_{\vec{x}}(\vec{\alpha})$.
%%%
\index{$w(\vec{x})$}%%
%%%
The value of the network is the value of its goal (incidentally
the cell with number $\nu$). Let $\xi \colon W^{\ast} \pf \{0,1,\star\}$
%%%
\index{$\xi(\vec{x})$}%%%
%%%
be the following function. $\xi(\vec{x}) := \star$ if $\vec{x}$ is
not a network. Otherwise, $\xi(\vec{x})$ is the value of $\vec{x}$.
We wish to define a machine calculating $\xi$. 
We give an example. We want to evaluate 
$f(x_0, x_1, x_2) = - (- x_2 \cap (x_1 \cup x_0))$
for $x_0 := 0$, $x_1 := 1$ and $x_2 := 0$. Then we write down
the following network:
%%%
\begin{equation}
\mbox{\mtt (a,0,\symbol{5})(b,1,0,\symbol{31})(ba,a,b,\symbol{4}%
)(bb,ba,\symbol{5})}
\end{equation}
%%%
Now $w(\mbox{\mtt a}) = - 0 = 1$, $w(\mbox{\mtt b}) =
1 \cup 0 = 1$, $w(\mbox{\mtt ba}) = w(\mbox{\mtt a}) \cap
w(\mbox{\mtt b}) = 1 \cap 1 = 1$, and
$v(\vec{x}) = w(\mbox{\mtt bb}) = - w(\mbox{\mtt ba}) =
- 1 = 0$.
%%
\begin{lem}
The set of all networks is in \textbf{LOGSPACE}.
\end{lem}
%%
\proofbeg
The verification is a somewhat longwinded matter but not difficult
to do. To this end we shall have to run over the string several times
in order to check the different criteria. The first condition is that 
no two cells have the same number. To check that we need the following: 
for every two positions $i$ and $j$ that begin a number, if $i \neq j$, 
then the numbers that start there are different. To compare the numbers 
means to compare the strings starting at these positions.  (To do that 
requires to memorize only one symbol at a time, running back and forth
between the strings.) This requires memorizing two further positions. 
However, a position takes only logarithmic space. 
\proofend
%%
\begin{thm}
$\xi$ is in \textbf{PTIME}.
\end{thm}
%%
\proofbeg
Let $\vec{x}$ be given. First we compute whether $\vec{x}$ is a
network. This computation is in \textbf{PTIME}. If $\vec{x}$ is not a network,
output $\star$. If it is, we do the following. Moving up with the
number $k$ we compute the value of the cell number $k$. For each
cell we have to memorize its value on a separate tape, storing
pairs $(\vec{\alpha}, w(\vec{\alpha}))$ consisting of the name
and the value of that cell. This can also be done in polynomial time.
Once we have reached the cell with the highest number we are done.
\proofend

It is not known whether the value of a network can be calculated 
in \textbf{LOG\-SPACE}. The problem is that we may not be able
to bound the number of intermediate values. Now the following holds.
%%
\begin{thm}
Let $f \colon A^{\ast} \pf \{\mbox{\mtt 0},\mbox{\mtt 1}\}$ be in
\textbf{PTIME}. Then there exists a function $N \colon A^{\ast} \pf W^{\ast}$
in \textbf{LOGSPACE} such that for every $\vec{x} \in
A^{\ast}$ $N(\vec{x})$ is a network and $f(\vec{x}) =
(\xi \circ N)(\vec{x})$.
\end{thm}
%%
\proofbeg
First we construct a network and then show that it is in
\textbf{LOGSPACE}. By assumption there exist numbers $k$ and $c$
such that $f(\vec{x})$ is computable in $\rho := c \cdot
|\vec{x}|^k$ time using a deterministic Turing machine $T$.
We define a construction algorithm for a sequence $\CC(\vec{x})
:= (C(i,j))_{i,j}$, where $0 \leq i, j \leq c \cdot |\vec{x}|^k$. 
$\CC(\vec{x})$ is ordered in the following way:
%%
\begin{align}\notag
& C(0,0), C(0,1), C(0,2), \dotsc, \\
& C(1,0), C(1,1), C(1,2), \dotsc, \\\notag
& C(2,0), C(2,1), C(2,2), \dotsc
\end{align}
%%
$C(i,j)$ contains the following
information: (a) the content of the $j$th cell of the tape
of $T$ at time point $i$, (b) information, whether the read
head is on that cell at time point $i$, (c) if the read head
is on this cell at $i$ also the state of the automaton. This
information needs bounded length. Call the bound $\lambda$.
We denote by $C(i,j,k)$, $k < \lambda$, the $k$th binary digit 
of $C(i,j)$. $C(i+1,j)$ depends only on $C(i,j-1)$, $C(i,j)$ 
and $C(i,j+1)$. ($T$ is deterministic. Moreover, we assumed that 
$T$ works on a tape that is bounded to the left. See 
Exercise~\ref{ex:oneside} that this is no loss of generality.)
$C(0,j)$ are determined by $\vec{x}$ alone.  (A) We have 
$C(i+1,j) = C(i,j)$ if either (A1) at $i$ the head is not at
$j-1$ or else did not move right, or (A2) at $i$ the head is not at
$j+1$ or else did not move left; (B) $C(i+1,j)$ can be computed
from (B1) $C(i,j-1)$ if the head was at $i$ positioned at $j$
and moved right, (B2) $C(i,j)$ if the head was at $i$ positioned
at $j$ and did not move, (B3) $C(i,j+1)$ if the head at $i$ was
positioned at $j+1$ and moved left. Hence, for every $k < \lambda$
there exist boolean functions $f^k_L$, $f^k_M$ and $f^k_R$ such that
$f^k_L, f^k_R \colon \{0,1\}^{2\lambda} \pf \{0,1\}^{\lambda}$,
$f^k_M \colon \{0,1\}^{3\lambda} \pf \{0,1\}^{\lambda}$, and
%%
\begin{align}
\notag
C(i+1,0,k) & = f_L(C(i,0), C(i,1)) \\
C(i+1,j,k) & = f_M(C(i,j-1), C(i,j), C(i,j+1)) \\
\notag
C(i+1,\rho,k) & = f_R(C(i,\rho-1), C(i,\rho))
\end{align}
%%
These functions can be computed from $T$ in
time independent of $\vec{x}$. Moreover, we can compute
sequences of cells that represent these functions. Basically,
the network we have to construct results in replacing for every
$i > 0$ and appropriate $j$ the cell $C(i,j)$ by a sequence of
cells calling on appropriate other cells to give the value $C(i,j)$.
This sequence is obtained by adding a fixed number to each argument
key of the cells of the sequence computing the boolean functions.

Now let $\vec{x}$ be given. We compute a sequence of cells
$\gamma(i,j)$ corresponding to $C(i,j)$. The row $\gamma(0,j)$
is empty. Then ascending in $i$, the rows $\gamma(i,j)$ are
computed and written on the output tape. If row $i$ is
computed, the following numbers are computed and remembered:
the length of the $i$th row, the position of the read head at
$i+1$ and the number of the first cell of $\gamma(i,j')$,
where $j'$ is the position of the read head at $i$. Now the
machine writes down $C(i+1,j)$ with ascending $j$. This is done
as follows. If $j$ is not the position of the read head at $i+1$,
the sequence is a sequence of cells that repeats the value of the
cells of $\gamma(i,j)$. So, $\gamma(j+1,i,k)
= \mbox{\mtt ($\vec{\alpha}$,$\vec{\varepsilon}$,1,\symbol{4})}$
for $\vec{\alpha}$ the number of the actual cell and
$\vec{\varepsilon}$ is $\vec{\alpha}$ minus some appropriate
number, which is computed from the length of the $i$th row
the length of the sequence $\gamma(i,j')$ and the length of
the sequence $\gamma(i+1,j)$. If $j$ is the position of the
read head, we have to insert more material, but basically
it is a sequence shifted by some number, as discussed above. The 
number by which we shift can be computed in \textbf{LOGSPACE} 
from the numbers which we have remembered. Obviously, it can be 
decided on the basis of this computation when the machine $T$ 
terminates on $\vec{x}$ and therefore when to stop the sequence. 
The last entry is the goal of the network.
\proofend

One also says that the problem of calculating the value of a 
network is complete with respect to the class \textbf{PTIME}.
A network is \textbf{monotone} 
%%%
\index{network!monotone}%%
%%%
if it does not contain the symbol {\mtt\symbol{5}}.
%%
\begin{thm}
There exists a \textbf{LOGSPACE}--computable function $M$
which transforms an arbitrary network into a monotone
network with identical value.
\end{thm}
%%
The proof is longwinded but rather straightforward, so we shall
only sketch it. Call $g : 2^n \pf 2$ \textbf{monotone} 
%%%
\index{boolean function!monotone}%%%
%%%
if for every $\vec{x}$ and $\vec{y}$ such that $x_i \leq y_i$ for all 
$i < n$, $g(\vec{x}) \leq g(\vec{y})$. (We write 
$\vec{x} \leq \vec{y}$ if $x_i \leq y_i$ for all $i < n$.)
%%%
\begin{lem}
\label{lem:monotone}
Let $f$ be an $n$--ary boolean function. Then there exists a 
$2n$--ary boolean function $g$ which is monotone such that 
%%%
\begin{equation}
f(\vec{x}) = g(x_0, - x_0, x_1, - x_1, \dotsc, x_{n-1}, -x_{n-1})
\end{equation}
%%%
\end{lem}
%%%
\begin{thm}
Every monotone boolean function is a polynomial function over 
$\cap$ and $\cup$.
\end{thm}
%%%
\proofbeg
One direction is easy. $\cap$ and $\cup$ is monotone, and 
every composition of monotone functions is again monotone. 
For the other direction, let $f$ be monotone. Let $M$ be 
the set of minimal vectors $\vec{x}$ such that $f(\vec{x}) = 1$.
For every vector $\vec{x}$, put $p(\vec{x}) := 
\bigcap_{x_i = 1} x_i$. (If $\vec{x} = 0\dotsc 0$, then 
$p(\vec{x}) := \mbox{\mtt 1}$.) Finally, put 
%%%
\begin{equation}
\rho_M := \bigcup_{\vec{x} \in M} p(\vec{x})
\end{equation}
%%%
If $M = \varnothing$, put $\rho_f := \mbox{\mtt 0}$. It is easily 
seen that $f(\vec{x}) = \rho_f(\vec{x})$ for all $\vec{x} \in 2^n$.
\proofend

What is essential (and left for the reader to show) is that 
the map $f \mapsto g$ translates into a \textbf{LOGSPACE} 
computable map on networks. So, if $g$ is an arbitrary 
\textbf{PTIME}--computable function from $A^{\ast}$ to $\{0,1\}$, 
there exists a \textbf{LOGSPACE}--computable function $N^+$
constructing monotone networks such that $g(\vec{x}) = 
(\xi \circ N^+)(\vec{x})$ for all $\vec{x} \in A^{\ast}$.

Now we shall turn to the promised new type of machines.
%%
\begin{defn}
An \textbf{alternating Turing machine}
%%%
\index{Turing machine!alternating}%%
%%%
is a sextuple
%%
\begin{equation}
\auf A, \mbox{\tt L}, Q, q_0, f, \gamma\zu
\end{equation}
%%
where $\auf A, \mbox{\tt L}, Q, q_0, f\zu$ is a Turing machine
and $\gamma \colon Q \pf \{\und, \oder\}$ an arbitrary function.
A state $q$ is called \textbf{universal} if $\gamma(q) = \und$, and
otherwise \textbf{existential}.
%%
\end{defn}
%%
We tacitly generalize the concepts of Turing machines to the
alternating Turing machines (for example an {\it alternating
$k$--tape Turing machine}, and a {\it logarithmically space bounded
alternative Turing machine\/}). To this end one has to add the
function $\gamma$ in the definitions. Now we have to define
when a Turing machine accepts an input $\vec{x}$. This is done
via configurations. A configuration is said to be
\textbf{accepted} by $T$  if one of the following is the case:
%%
\begin{dinglist}{43}
\item
$T$ is in an existential state and one of the
immediately subsequent configurations is accepted by $T$.
\item
$T$ is in a universal state and all immediately subsequent
configurations are accepted by $T$.
\end{dinglist}
%%
Notice that the machine accepts a configuration that has no
immediately subsequent configurations if (and only if) it is 
in a universal state. The difference between universal and 
existential states is effective if the machine is not deterministic. 
Then there can be several subsequent configurations. Acceptance
by a Turing machine is defined as for an existential state if 
there is a successor state, otherwise like a universal state. 
If in a universal state, the machine must split itself into 
several copies that compute the various subsequent alternatives. 
Now we define \textbf{ALOGSPACE} 
%%%
\index{\textbf{ALOGSPACE}}%%%
%%%
to be the set of functions 
computable by a logarithmically space bounded alternating 
multitape Turing machine.
%%
\begin{thm}[Chandra \& Kozen \& Stockmeyer]
%%%
\index{Chandra, A.~K.}%%
\index{Stockmeyer, L.~J.}%%%
\index{Kozen, Dexter C.}%%%
%%%%
$$\mbox{\textbf{ALOGSPACE}} = \mbox{\textbf{PTIME}}.$$
\end{thm}
%%
The theorem is almost proved. First, notice 
%%
\begin{lem}
\label{lem:enth}
\textbf{LOGSPACE} $\subseteq$ \textbf{ALOGSPACE}.
\end{lem}
%%
For every deterministic logarithmically space bounded
Turing machine also is an alternating machine
by simply letting every state be universal. Likewise the
following claim is easy to show, if we remind ourselves of
the facts concerning \textbf{LOGSPACE}--computable functions.
%%
\begin{lem}
\label{lem:verkn}
Let $f \colon A^{\ast} \pf B^{\ast}$ and $g\colon B^{\ast} \pf C^{\ast}$ 
be functions. If $f$ and $g$ are in \textbf{ALOG\-SPACE}, so is 
$g \circ f$.
\end{lem}
%%
\begin{lem}
\textbf{ALOGSPACE} $\subseteq$ \textbf{PTIME}.
\end{lem}
%%
Also this proof is not hard. We already know that there are
at most polynomially many configurations. The dependency
between these configurations can also be checked in polynomial
time. (Every configuration has a bounded number of successors.
The bound only depends on $T$.) This yields a computation
tree which can be determined in polynomial time. Now we must
determine in the last step whether the machine accepts the
initial configuration. To this end we must determine by induction
on the depth in a computation tree whether the respective
configuration is accepted. This can be done as well in
polynomial time. This completes the proof.

Now the converse inclusion remains to be shown. For this we use
the following idea. Let $f$ be in \textbf{PTIME}. We can write $f$ as
$\xi \circ N^+$ where $N^+$ is a monotone network computing $f$. As
remarked above we can construct $N^+$ in \textbf{LOGSPACE} and in
particular because of Lemma~\ref{lem:enth} in \textbf{ALOGSPACE}.
It suffices to show that $\xi$ is in \textbf{ALOGSPACE}. For then
Lemma~\ref{lem:verkn} gives us that $f = \xi \circ N^+ \in 
\textbf{ALOGSPACE}$.
%%
\begin{lem}
$\xi \in$ \textbf{ALOGSPACE}.
\end{lem}
%%
\proofbeg
We construct a logarithmically space bounded alternating
machine which for an arbitrary given monotone network
$\vec{x}$ calculates its value $w(\vec{x})$. Let a network 
be given. First move to the goal. Descending from it
compute as follows.
%%
\begin{dingautolist}{192}
\item
If the cell contains {\mtt\symbol{4}} change into the universal 
state $q_1$. Else change into the existential state $q_2$. 
Goto \ding{194}.
\item
Choose an argument key $\vec{\alpha}$ of the current
cell and go to the cell number $\vec{\alpha}$.
\item
If $\vec{\alpha}$ is not an argument key go into state
$q_f$ if $\vec{\alpha} = \mbox{\mtt 1}$ and into $q_g$ if
$\vec{\alpha} = \mbox{\mtt 0}$.
Here $q_f$ is universal and $q_g$ existential
and there are no transitions defined from $q_f$ and $q_g$.
\end{dingautolist}
%%
All other states are universal; however, the machine works
nondeterministically only in one case, namely if it gets the
values of the arguments. Then it makes a nondeterministic choice.
If the cell is an $\oder$--cell then it will accept that
configuration if one argument has value 1, since the state is
existential. If the cell is a $\und$--cell then it shall accept
the configuration if both arguments have value $1$ for now the
state is universal. The last condition is the termination
condition. If the string is not an argument key then it is either
{\mtt 0} or {\mtt 1} and its value can be computed without recourse
to other cells. If it is {\mtt 1} the automaton changes into a
final state which is universal and so the configuration is
accepted. If the value is {\mtt 0} the automaton changes into a
final state which is existential and the configuration is
rejected. 
\proofend

{\it Notes on this section.} The gap between \textbf{PTIME} and 
\textbf{NPTIME} is believed to be 
a very big one, but it is not known whether the two really are 
distinct. The fact that virtually all languages are in 
\textbf{PTIME} is good news, telling us that natural languages 
are tractable, at least syntactically. Concerning the tractability 
of languages as sign systems not very much is known, however. 
%%
\vplatz
\exercise
Show Proposition~\ref{prop:schnitt}:
{\it With $L_1$ and $L_2$ also $L_1 \cup L_2$, $L_1 \cap L_2$
as well as $A^{\ast} - L_1$ are in \textbf{PTIME}.}
%%
\vplatz
\exercise
Show the following. {\it If $L_1$ and $L_2$ are in \textbf{PTIME}
then so is $L_1 \cdot L_2$.}
%%
\vplatz
\exercise
Show that $L$ has the constant growth property if $L$ is semilinear.
Give an example of a language which has the constant growth property
but is not semilinear.
%%
\vplatz
\exercise
\label{exercise:basis}
Let $f \colon 2^n \pf 2$ a boolean function. Show that it can be obtained
from the projections and the functions $-$, $\cup$, and $\cap$.
{\it Hint.} Start with the functions $g_{\vec{x}} \colon 2^n \pf 2$ such that
$g_{\vec{x}}(\vec{y}) := 1$ iff $\vec{x} = \vec{y}$.
Show that they can be generated from $-$ and $\cap$. Proceed
to show that every boolean function is either the constant $0$
or can be obtained from functions of type $g_{\vec{x}}$ using
$\cup$.
%%
\vplatz
\exercise
Prove Lemma~\ref{lem:monotone}.
%%
\vplatz
\exercise
%%%
\index{language!weakly semilinear}%%%
%%%
Call a language $L \subseteq A^{\ast}$ \textbf{weakly semilinear} if
every intersection with a semilinear language $\subseteq A^{\ast}$
has the constant growth property. Show that every semilinear language
is also weakly semilinear. Let $M := \{\mbox{\tt a}^m
\mbox{\tt b}^n : n \geq 2^m\}$. Show that $M \subseteq \{\mbox{\tt a},
\mbox{\tt b}\}^{\ast}$ is weakly semilinear but not semilinear.
%%
\vplatz
\exercise
Show that every function $f \colon A^{\ast} \pf B^{\ast}$ which is
computable using an alternating Turing machine can also be computed
using a Turing machine. (It is not a priori clear that
the class of alternating Turing machines is not more powerful
than the class of Turing machines. This has to be shown.)

 \section{Literal Movement Grammars}
\label{kap4-2}
%
%
%
The concept of a literal movement grammar --- \textbf{LMG} for short 
--- has been introduced by Annius Groenink in \shortcite{groenink:surface} 
%%%
\index{Groenink, Annius}%%%
%%%
(see also \cite{groenink:lmgs}). With the help of these grammars one 
can characterize the \textbf{PTIME} languages by means of a generating 
device. The idea to this characterization goes back to a result by
William Rounds \shortcite{rounds:ptime}. 
%%%
\index{Rounds, William}%%
%%%
Many grammar types
turn out to be special subtypes of LMGs. The central feature of 
LMGs is that the rules contain a context free skeleton which describes 
the abstract structure of the string and in addition to this
a description of the way in which the constituent is formed
from the basic parts. The notation is different from that of 
CFGs. In an LMG, nonterminals denote properties 
of strings and therefore one writes `$Q(\vec{x})$' in place 
of just `$Q$'. The reason for this will soon become obvious. 
If $Q(\vec{x})$ obtains for a given string $\vec{x}$ we say 
that $\vec{x}$ has the property $Q$ or that $\vec{x}$ \textbf{is a}
$Q$--\textbf{string}. The properties play the role of the 
nonterminals in CFGs, but technically speaking they are 
handled differently. Since $\vec{x}$ is metavariable for strings, 
we now need another set of (official) variables for strings in the 
formulation of the LMGs. To this end we use the plain symbols $x$, 
$y$, $z$ and so on (possibly with subscripts) for these variables. 
In addition to these variables there are also constants {\tt a}, 
{\tt b}, for the symbols of our alphabet $A$. We give a simple 
example of an LMG. It has two rules.
%%
\begin{equation}
S(xx) \hrn S(x).; \qquad S(\mbox{\tt a}) \hrn.
\end{equation}
%%
These rules are written in Horn--clause format, as in Prolog, 
and they are exactly interpreted in the same way: the left hand side 
obtains with the variables instantiated to some term if the right 
hand obtains with the variables instantiated in the same way. 
So, the rules correspond to more familar looking formulae:
%%%
\begin{equation}
S(\mbox{\tt a}), \quad (\forall x)(S(x) \pf S(x \conc x))
\end{equation}
%%%%
(Just reverse the arrow and interpret the comma as conjunction.)
%%%
\begin{defn}
%%%%
\index{Horn--formula}%%
%%%%
A formulae $\varphi$ of predicate logic is called a 
\textbf{Horn--formula} iff it has the form 
%%%
\begin{equation}
(\forall x_0)\dotso (\forall x_{q-1})(\bigwedge_{i < n} \chi_i %
\pf \varphi)
\end{equation}
%%%%
where the $\chi_i$, $i < n$, and $\varphi$ are atomic formulae. 
\end{defn}
%%%
Here, it is assumed that only the variables $x_i$, $i < q$, 
occur in the $\chi_i$ and in $\varphi$. We abbreviate 
$(\forall x_0) \dotso (\forall x_{p-1})$ by $(\forall \vec{x})$.
Now, consider the case where the language has the following 
functional signature: for every letter from $A$ a zeroary 
function symbol (denoted by the same letter), $\varepsilon$ 
(zeroary) and $^{\smallfrown}$ (binary). Further, assume the 
following set of axioms:
%%%
\begin{multline}
S_G := \{(\forall xyz)(x^{\smallfrown}(y^{\smallfrown}z) \doteq 
(x^{\smallfrown}y)^{\smallfrown}z),  \\
(\forall x)(\varepsilon^{\smallfrown}x \doteq x), 
(\forall x)(x^{\smallfrown}\varepsilon \doteq x)\}
\end{multline}
%%%
Then a Horn--clause is of the form
%%%
\begin{equation}
(\forall \vec{x})(U_0(s_0) \und U_1(s_1) \und \dotso %
\und U_{n-1}(s_{n-1}) \pf T(t))
\end{equation}
%%%
where $t$ and the $s_i$ ($i < n$) are string polynomials. 
This we shall write as
%%
\begin{equation}
\label{rule:generic}
T(t) \hrn U_0(s_0),U_1(s_1),\dotsc, U_{n-1}(s_{n-1}).
\end{equation}
%%%
\begin{defn}
%%%
\index{literal movement grammar (LMG)}%%
\index{LMG (see literal movement grammar)}%%
%%%
A \textbf{literal movement grammar}, or \textbf{LMG} for short,
is a quintuple $G =  \auf A, R, \Xi, S, H\zu$, where $A$ is the 
alphabet of terminal symbols, $R$ a set of so--called 
%%%
\index{predicate}%%%
%%%%
\textbf{predicates}, $\Xi : R \pf \omega$ a signature, 
$S \in R$ a distinguished symbol such that $\Xi(S) = 1$, and
$H$ a set of Horn--formulae in the language consisting of 
constants for every letter of $A$, the empty string, concatenation, 
and the relation symbols of $R$. $\vec{x}$ is a $G$--\textbf{sentence}
iff $S(\vec{x})$ is derivable from $H$ and $S_G$:
%%%%
\begin{equation}
L(G) := \{\vec{x} : \; S_G;H \vdash S(\vec{x})\}
\end{equation}
\end{defn}
%%%
We call $G$ a $k$--\textbf{LMG} if $\max \{\Xi(Q) : Q \in R\} \leq k$. 
The grammar above is a 1--LMG.
%%
\begin{prop}
$L(G) = \{\mbox{\tt a}^{2^n} : 0 \leq n\}$.
\end{prop}
%%
\proofbeg
Surely $\mbox{\tt a} \in L(G)$. This settles the case
$n = 0$. By induction one shows $\mbox{\tt a}^{2^n} \in L(G)$
for every $n > 0$. For if $\mbox{\tt a}^{2^n}$ is a string
of category $S$ so is $\mbox{\tt a}^{2^{n+1}} =
\mbox{\tt a}^{2^n} \conc \mbox{\tt a}^{2^n}$.
This shows that $L(G) \supseteq \{\mbox{\tt a}^{2^n} :
n \geq 0\}$. On the other hand this set satisfies the formula
$\varphi$. For we have $\mbox{\tt a} \in L(G)$ and
with $\vec{x} \in L(G)$ we also have $\vec{x}\, \vec{x} \in L(G)$.
For if $\vec{x} = \mbox{\tt a}^{2^n}$ for a certain
$n \geq 0$ then $\vec{x} \, \vec{x} = \mbox{\tt a}^{2^n \cdot 2} =
\mbox{\tt a}^{2^{n+1}} \in L(G)$.
\proofend

There is an inductive definition of $L(G)$ by means of generation. 
We write $\vdash_G S(\vec{x})$ (vector arrow!), if either
$S(\vec{x}) \hrn.$ is a rule or $\vec{x} = \vec{y}\, \vec{y}$
and $\vdash_G S(\vec{y})$. Both definitions define the same set 
of strings. Let us elaborate the notion of a 1--LMG in more detail.
The maximum of all $n$ such that $G$ has an $n$--ary rule is 
called
%%
\index{branching number}%%
%%%
the \textbf{branching number of} $G$. In the rule
%%
\begin{equation}
S(xx) \hrn S(x).
\end{equation}
%%
we have $n = 1$ and $T = U_0 = S$, $t = xx$ and $s_0 = x$. 
In the rule
%%
\begin{equation}
S(\mbox{\tt a}) \hrn .
\end{equation}
%%
we have $n = 0$, $T = S$ and $t = \mbox{\tt a}$.
%%%
\begin{defn}
%%%
\index{$\vdash_G$}%%%
%%%
Let $G = \auf A, R, \Xi, S, H\zu$ be an LMG. Then we write 
$\Gamma \vdash^n_G \gamma$ iff $\Gamma;H;S_G \vdash^n \gamma$ in 
predicate logic and $\Gamma \vdash_G \gamma$ iff $\Gamma; 
H; S_G \vdash \gamma$ in predicate logic.
\end{defn}
%%%
We shall explain in some detail how we determine whether or not 
$\vdash_G^n Q(\vec{x})$ ($Q$ unary). Call a \textbf{substitution} a 
function $\alpha$ which associates a term to each variable; 
and a \textbf{valuation} a function that associates a string 
in $A^{\ast}$ to each string variable.
%%%
\index{substitution}%%%
\index{valuation}%%
%%%
Given $\alpha$ we define $s^{\alpha}$ for a polynomial
by homomorphic extension. For example, if $s = \mbox{\tt a} \,
x^2 \, \mbox{\tt b} \, y$ and $\alpha(x) = \mbox{\tt ac}$, 
$\alpha(y) = \mbox{\tt bba}$ then $s^{\alpha} = \mbox{\tt aacacbbba}$, 
as is easily computed. Notice that strings can be seen as constant 
terms modulo equivalence, a fact that we shall exploit here by 
confusing valuations with substitutions that assign constant 
terms to the string variable. (The so--called \textbf{Herbrand--universe} 
%%%
\index{Herbrand--universe}%%%
%%%
is the set of constant terms. It is known that any Horn--formula 
that is not valid can be falsified in the Herbrand--universe, 
in this case $\GZ(A)$.)
%%
\begin{equation}
T(\vec{x}) \hrn U_0(\vec{y}_0), U_1(\vec{y}_1),\dotsc,
U_{m-1}(\vec{y}_{m-1}).
\end{equation}
%%
\index{rule!instance}%%
%%%%
is an \textbf{instance} of the rule
%%
\begin{equation}
T(t) \hrn U_0(s_0),U_1(s_1),\dotsc, U_{m-1}(s_{m-1}).
\end{equation}
%%
if there is a valuation $\beta$ such that
$\vec{x} = t^{\beta}$ and $\vec{y}_i = s^{\beta}_i$
for all $i < m$. Similarly, 
%%
\begin{equation}
T(t') \hrn U_0(s'_0),U_1(s'_1),\dotsc, U_{m-1}(s'_{m-1}).
\end{equation}
%%
is an instance of 
%%
\begin{equation}
T(t) \hrn U_0(s_0),U_1(s_1),\dotsc, U_{m-1}(s_{m-1}).
\end{equation}
%%
if there is a substitution $\alpha$ such that $s'_i = (s_i)^{\alpha}$ 
for all $i < m$ and $t' = t^{\alpha}$. The notion of generation by an 
LMG can be made somewhat more explicit.
%%%
\begin{prop}
(a) $\vdash_G^0 Q(\vec{x})$ iff $Q(\vec{x}) \hrn .$ is a ground 
instance of a rule of $G$. 
(b) $\vdash_G^{n+1} Q(\vec{x})$ iff $\vdash_G^n Q(\vec{x})$ or
there is a number $m$, predicates $R_i$, $i < m$, and strings
$\vec{y}_i$, $i < m$, such that
%%
\begin{dingautolist}{192}
\item $\vdash_G^n R_i(\vec{y}_i)$, and
\item $Q(\vec{x}) \hrn R_0(\vec{y}_0),\dotsc, R_{m-1}(\vec{y}_{m-1})$
is a ground instance of a rule of $G$.
\end{dingautolist}
\end{prop}
%%
We shall give an example to illustrate these definitions.
Let $K$ be the following grammar.
%%
\begin{equation}
S(vxyz) \hrn S(vyxz).;
    \qquad S(\mbox{\tt abc}\conc x) \hrn S(x).;
    \quad S(\varepsilon) \hrn.
\end{equation}
%%
Then $L(K)$ is that language which contains all strings
that contain an identical number of {\tt a}, {\tt b} and
{\tt c}. To this end one first shows that $(\mbox{\tt abc})^{\ast}
\subseteq L(K)$ and in virtue of the first rule $L(K)$ is closed
under permutations. Here $\vec{y}$ is a \textbf{permutation of} 
%%%
\index{permutation}%%%
%%%
$\vec{x}$ if $\vec{y}$ and $\vec{x}$ have identical
image under the Parikh map. Here is an example (the general case is
left to the reader as an exercise). We can derive
$S(\mbox{\tt abc})$ in one step from
$S(\varepsilon)$ using the second rule, and
$S(\mbox{\tt abcabc})$ in two steps, using
again the second rule. In a third step we can derive
$S(\mbox{\tt aabbcc})$ from this, using the
first rule this time. Put $\alpha(v) := \mbox{\tt a}$,
$\alpha(x) := \mbox{\tt ab}$, $\alpha(y) := \mbox{\tt bc}$
and $\alpha(z) := \mbox{\tt c}$. Then
%%
\begin{equation}
(vxyz)^{\alpha} = \mbox{\tt aabbcc}, \quad (vyxz)^{\alpha} =
\mbox{\tt abcabc}
\end{equation}


Let $H$ be a CFG. We define a 1--LMG $H^{\spadesuit}$ as follows.  
(For the presentation we shall assume that $H$ is already in Chomsky 
normal form.) For every nonterminal $A$ we introduce a unary 
predicate $\uli{A}$. The start symbol is $\uli{\mbox{\tt S}}$.
If $A \pf BC$ is a rule from $H$ then $H^{\spadesuit}$ contains the rule
%%
\begin{equation}
\uli{A}(xy) \hrn \uli{B}(x),\uli{C}(y).
\end{equation}
%%
If $A \pf a$ is a terminal rule then we introduce the following 
rule into $H^{\spadesuit}$:
%%
\begin{equation}
\uli{A}(a) \hrn .
\end{equation}
%%
One can show relatively easily that $L(H) = L(H^{\spadesuit})$.

The 1--LMGs can therefore generate all CFLs.
Additionally, they can generate languages without constant
growth, as we have already seen. Let us note the following facts.
%%
\begin{thm}
\label{thm:lbgschnitt}
Let $L_1$ and $L_2$ be languages over $A$ which can be generated
by 1--LMGs. Then there exist 1--LMGs generating the languages
$L_1 \cap L_2$ and $L_1 \cup L_2$.
\end{thm}
%%
\proofbeg
Let $G_1$ and $G_2$ be 1--LMGs which generate $L_1$ and $L_2$,
respectively. We assume that the set of nonterminals of
$G_1$ and $G_2$ are disjoint. Let $S_i$
be the start predicate of $G_i$, $i \in \{1,2\}$.
Let $H_{\cup}$ be constructed as follows. We form the
union of the nonterminals and rules of $G_1$ and $G_2$.
Further, let $S^{\heartsuit}$ be a new
predicate, which will be the start predicate of $G_{\cup}$.
At the end we add the following rules:
%%
$S^{\heartsuit}(x) \hrn S_1(x).$;
$S^{\heartsuit}(x) \hrn S_2(x)$.
%%
This defines $G_{\cup}$. $G_{\cap}$ is defined similarly,
only that in place of the last two rules we have a single
rule, $S^{\heartsuit}(x) \hrn S_1(x), S_2(x)$. It is easily checked that
$L(G_{\cup}) = L_1 \cup L_2$ and $L(G_{\cap}) = L_1 \cap L_2$.
We show this for $G_{\cap}$. We have $\vec{x} \in L(G_{\cap})$
if there is an $n$ with $\vdash_{G_{\cap}}^n S^{\heartsuit}(\vec{x})$.  
This in turn is the case exactly if $n > 0$ and 
$\vdash_{G_{\cap}}^{n-1} S_1(\vec{x})$ as well as 
$\vdash_{G_{\cap}}^{n-1} S_2(\vec{x})$.  This is nothing but 
$\vdash_{G_1}^{n-1} S_1(\vec{x})$ and 
$\vdash_{G_2}^{n-1} S_1(\vec{x})$. Since $n$
was arbitrary, we have $\vec{x} \in L(G_{\cap})$ iff
$\vec{x} \in L(G_1) = L_1$ and $\vec{x} \in L(G_2) = L_2$,
as promised.
\proofend

The 1--LMGs are quite powerful, as the following theorem shows.
%%
\begin{thm}
\label{thm:lbgra}
Let $A$ be a finite alphabet and $L \subseteq A^{\ast}$.
$L = L(G)$ for a 1--LMG iff $L$ is recursively
enumerable.
\end{thm}
%%
The proof is left to the reader as an exercise. Since the set of
recursively enumerable languages is closed under union and
intersection, Theorem~\ref{thm:lbgschnitt} already follows from
Theorem~\ref{thm:lbgra}. It also follows that the complement
of a language that can be generated by a 1--LMG does not have
to be such a language again. For the complement of a recursively
enumerable language does not have to be recursively enumerable
again. (Otherwise every recursively enumerable set would also
be decidable, which is not the case.)

In order to arrive at interesting classes of languages we shall
restrict the format of the rules. Let $\rho$ be the following rule.
%%
\begin{equation}
\rho := T(t) \hrn U_0(s_0),U_1(s_1),\dotsc, U_{n-1}(s_{n-1}).
\end{equation}
%%
%%
\index{literal movement grammar!nondeleting}%%
%%
\begin{dinglist}{43}
\item $\rho$ is called \textbf{upward nondeleting} if
	\index{rule!upward nondeleting}
    every variable which occurs in one of the $s_i$, $i < n$,
    also occurs in $t$.
\item $\rho$ is called \textbf{upward linear} if no variable
	\index{rule!upward linear}
    occurs more than once in $t$.
\item $\rho$ is called \textbf{downward nondeleting} if every
	\index{rule!downward nondeleting}
    variable which occurs in $t$ also occurs in one of
    the $s_i$.
\item $\rho$ is called \textbf{downward linear} if none of the
	\index{rule!downward linear}
    variables occurs twice in the $s_i$. (This means:
    the $s_i$ are pairwise disjoint in their variables
    and no variable occurs twice in any of the $s_i$.)
\item $\rho$ is called \textbf{noncombinatorial} if the
%%%
	\index{rule!noncombinatorial}
\index{literal movement grammar!noncombinatorial}%%
%%%
    $s_i$ are variables.
\item $\rho$ is called \textbf{simple} if it is noncombinatorial,
%%%%
	\index{rule!simple}
\index{literal movement grammar!simple}%%
%%%%
    upward nondeleting and upward linear.
\end{dinglist}
%%
$G$ has the property $\CP$ if all rules of $G$ possess
$\CP$. In particular the type of simple grammars shall
be of concern for us. The definitions are not always
what one would intuitively expect. For example, the following
rule is called upward nondeleting even though applying this
rule means deleting a symbol:
$\mbox{\tt U}(x) \hrn \mbox{\tt U}(x \conc \mbox{\tt a})$.
This is so since the definition focusses on the variables
and ignores the constants. Further, downward linearity could
alternatively be formulated as follows. One requires any
symbol to occur in $t$ as often as it occurs in the
$s_i$ taken together. This, however, is too strong a 
requirement. One would like to allow a variable to occur 
twice to the right even though on the left it occurs only once.
%%
\begin{lem}
\label{lem:einf}
Let $\rho$ be simple. Further, let 
%%
\begin{equation}
Q(\vec{y}) \hrn R_0(\vec{x}_0),R_1(\vec{x}_1),
\dotsc, R_{n-1}(\vec{x}_{n-1}).
\end{equation}
%%
be an instance of $\rho$. Then $|\vec{y}| \geq \sum_{i < n} 
|\vec{x}_i| \geq \max \{|\vec{x}_i| : i < n\}$. Further, $\vec{x}_i$ 
is a subword of $\vec{y}$ for every $i < n$.
\end{lem}
%%
\begin{thm}
\label{thm:1pol}
Let $L \subseteq A^{\ast}$ be generated by some simple 1--LMG.
Then $L$ is in \textbf{PTIME}.
\end{thm}
%%
\proofbeg
Let $\vec{x}$ be an arbitrary string and $n := \sharp
N \cdot |\vec{x}|$. Because of Lemma~\ref{lem:einf}  for every
predicate $Q$: $\vdash_G Q(\vec{x})$ iff
$\vdash_G^n Q(\vec{x})$.  From this follows that every
derivation of $S(\vec{x})$ has length at most
$n$. Further, in a derivation there are only predicates of the
form $Q(\vec{y})$ where $\vec{y}$ is a subword of $\vec{x}$.
The following chart--algorithm (which is a modification of the
standard chart--algorithm) only takes polynomial time:
%%
\begin{dinglist}{43}
\item For $i = 0, \dotsc, n$:
    For every substring $\vec{y}$ of length $i$ and every
    predicate $Q$ check if there are subwords
    $\vec{z}_j$, $j < p$, of length $< i$ and
    predicates $R_j$, $j < p$, such that $Q(\vec{y})
    \hrn R_0(\vec{z}_0),R_1(\vec{z}_1),\dotsc, R_{p-1}(\vec{z}_{p-1}).$
    is an instance of a rule of $G$.
\end{dinglist}
%%
The number of subwords of length $i$ is proportional to $n$. For 
given $p$, a string of length $n$ can be decomposed in $O(n^{p-1})$ 
ways as product of $p$ (sub)strings. Thus for every $i$, $O(n^p)$ 
many steps are required, in total $O(n^{p+1})$ on a deterministic 
multitape Turing machine.
\proofend

The converse of Theorem~\ref{thm:1pol} is in all likelihood
false. Notice that in an LMG, predicates need not be unary.
Instead, we have allowed predicates of any arity. There
sometimes occurs the situation that one wishes to have uniform
arity for all predicates. This can be arranged as follows. For an
$i$--ary predicate $A$ (where $i < k$) we introduce
a $k$--ary predicate $A^{\ast}$ which satisfies
%%
\begin{equation}
A^{\ast}(x_0, \dotsc, x_{k-1}) \dpf
A(x_0, \dotsc, x_{i-1}) \und \gund_{j = i}^{k-1}
    x_j \doteq \varepsilon.
\end{equation}
%%
There is a small difficulty in that the start predicate is required
to be unary. So we lift also this restriction and allow the
start predicate to have any arity. Then we put
%%
\begin{equation}
L(G) := \{\prod_{i < \Omega(S)} \vec{x}_i :\; \vdash_G S(\vec{x}_0,
    \dotsc, \vec{x}_{\Omega(S)-1})\} 
\end{equation}
%%
This does not change the generative power. An important class of LMGs, 
which we shall study in the sequel, is the class of simple LMGs.
Notice that in virtue of the definition of a simple rule a variable
is allowed to occur on the right hand side several times, while on
the left it may not occur more than once. This restriction however
turns out not to have any effect. Consider the following grammar $H$.
%%
\begin{equation}
\label{egalite}
\begin{array}{l@{\quad \hrn \quad}l@{\qquad}l}
E(\varepsilon, \varepsilon)  & . & \\
E(a,a) & . & (a \in A)  \\
E(xa, ya) & E(x,y). & (a \in A)
\end{array}
\end{equation}
%%
It is easy to see that $\vdash_H E(\vec{x}, \vec{y})$ iff 
$\vec{x} = \vec{y}$. $H$ is simple. Now take a rule
in which a variable occurs several times on the left hand side.
%%
\begin{equation}
S(xx) \leftarrow S(x).
\end{equation}
%%
We replace this rule by the following one and add \eqref{egalite}.
%%
\begin{equation}
S(xy) \leftarrow S(x), \quad E(x,y).
\end{equation}
%%
This grammar is simple and generates the same strings.
Furthermore, we can see to it that no variable occurs more than
three times on the right hand side, and that $s^j_i \neq s^j_k$
for $i \neq k$.  Namely, replace $s^j_i$ by distinct variables,
say $x^j_i$, and add the clauses $E(x^j_i, x^{j'}_{i'})$, if $s^j_i =
s^{j'}_{i'}$. We do not need to introduce all of these clauses.
For each variable we only need two. (If we
want to have $A_i = A_j$ for all $i < n$ we simply have to require
$A_i = A_j$ for all $j \equiv i+1 \pmod{n}$.)

With some effort we can generalize Theorem~\ref{thm:1pol}.
%%
\begin{thm}
\label{thm:npol}
Let $L \subseteq A^{\ast}$ be generated by a simple LMG.
Then $L$ is in \textbf{PTIME}.
\end{thm}
%%
The main theorem of this section will be to show that the
converse also holds. We shall make some preparatory remarks.
We have already seen that {\bf PTIME} = {\bf ALOGSPACE}. Now 
we shall provide another characterization of this class. Let $T$ 
be a Turing machine. We call $T$ {\bf read only} if none of 
its heads can write. If 
$T$ has several tapes then it will get the input on all of its 
tapes. (A read only tape is otherwise useless.) Alternatively, 
we may think of the machine as having only one tape but several read
heads that can be independently operated.
%%
\begin{defn}
Let $L \subseteq A^{\ast}$. We say that $L$ is in \textbf{ARO}
%%%
\index{\textbf{ARO}}%%%
%%%%
if there is an alternating read only Turing machine
which accepts $L$.
\end{defn}
%%
\begin{thm}
$\mbox{\textbf{ARO}} = \mbox{\textbf{ALOGSPACE}}$.
\end{thm}
%%
\proofbeg
Let $L \in \mbox{\bf ARO}$. Then there exists an alternating read
only Turing machine $T$ which accepts $L$. We have to find a
logarithmically space bounded alternating Turing machine that
recognizes $L$. The input and output tape remain, the other
tapes are replaced by read and write tapes, which are initially
empty. Now, let $\tau$ be a read only tape. The actions that
can be performed on it are: moving the read head to the left
or to the right (and reading the symbol). We code the position
of the head using binary coding. Evidently, this coding needs only
$\log_2 |\vec{x}| + 1$ space. Calculating the successor and
predecessor (if it exists) of a binary number is {\bf LOGSPACE}
computable (given some extra tapes). Accessing the $i$th symbol
of the input, where $i$ is given in binary code, is as well.
This shows that we can replace the read only tapes by 
logarithmically space bounded tapes. Hence $L \in 
\mbox{\bf ALOGSPACE}$. Suppose now that $L \in \mbox{\bf ALOGSPACE}$. 
Then $L = L(U)$ for an alternating, logarithmically space bounded 
Turing machine $U$. We shall construct a read only alternating 
Turing machine which accepts the same language. To this end we 
shall replace every intermediate Tape $\tau$ by several read 
only tapes which together perform the same actions. Thus, all we
need to show is that the following operations are computable on
read only tapes (using enough auxiliary tapes). (For simplicity,
we may assume that the alphabet on the intermediate tapes is just
{\tt 0} and {\tt 1}.) (a) Moving the head to the right, (b) moving
the head to the left, (c) writing {\tt 0} onto the tape, (d)
writing {\tt 1} onto the tape. Now, we must use at least two
read only tapes; one, call it $\tau_a$, contains the content of
Tape $\tau$, $\tau_b$ contains the position of the head of
$\tau$. The position $i$, being bounded by $\log_2 |\vec{x}|$,
can be coded by placing the head on the cell number $i$. Call 
$i_a$ the position of the head of $\tau_a$, $i_b$ the position 
of the head of $\tau_b$. Arithmetically, these steps correspond 
to the following functions: (a) $i_b \mapsto i_b+1$, 
(b) $i_b+1 \mapsto i_b$, (c) replacing the $i_b$th symbol
in the binary code of $i_a$ by {\tt 0}, (d) replacing the
$i_b$th symbol in the binary code of $i_a$ by {\tt 1}. We must
show that we can compute (c) and (d). (It is easy to see that
if we can compute this number, we can reset the head of $\tau_b$
onto the position corresponding to that number.) (A) The
$i_b$th symbol in the binary code of $i_a$ is accessed as follows.
We successively divide $i_a$ by $2$, exactly $i_b$ times,
throwing away the remainder. If the number is even, the result
is {\tt 0}, otherwise it is {\tt 1}. (B) $2^{i_b}$ is computed
by doubling $1$ $i_b$ times. So, (c) is performed as follows.
First, check the $i_b$th digit in the representation.
If it is {\tt 0}, leave $i_a$ unchanged. Otherwise, substract
$2^{i_b}$. Similarly for (d). This shows that we can find an
alternating read only Turing machine that recognizes $L$.
\proofend

Now for the announced proof. Assume that $L$ is in
\textbf{PTIME}. Then we know that there is an alternating
read only Turing machine which accepts $L$. This machine
works with $k$ tapes. For simplicity we shall assume that
the machine can move only one head in a single step.
We shall construct a $2k+2$--LMG $G$ such that $L(G) = L$.
Assume for each $a \in A$ two binary predicates, $L^a$ and $R^a$, 
with the following rules.
%%
\begin{align}
L^a(\varepsilon,a) & \hrn . & L^a(xc,yc) & \hrn L^a(x,y). \\
R^a(\varepsilon,a) & \hrn . & R^a(cx,cy) & \hrn R^a(x,y).
\end{align}
%%
It is easy to see that $L^a(\vec{x},\vec{y})$ is derivable iff
$\vec{y} = a \vec{x}$ and $R^a(\vec{x},\vec{y})$ is derivable iff
$\vec{y} = \vec{x}\, a$. 

If $\vec{w}$ is the input we can code the position of
a read head by a pair $\auf \vec{x}, \vec{y}\zu$ for which
$\vec{x}\,\vec{y} = \vec{w}$. A configuration
is simply determined by naming the state of the machine and
$k$ pairs $\auf \vec{x}_i, \vec{y}_i\zu$ with
$\vec{x}_i \, \vec{y}_i = \vec{w}$. Our grammar will
monitor the actions of the machine step by step.
To every state $q$ we associate a predicate $q^{\star}$.
If $q$ is existential the predicate is $2k+2$--ary.
If $q$ changes to $r$ when reading the letter
$a$ and if the machine moves to the left on Tape
$j$ then the following rule is added to $G$.
%%
\begin{multline}
\label{rule:left}
q^{\star}(w, x_j y_j, x_0, y_0, \dotsc, x_{j-1}, y_{j-1},
    x'_j, y'_j, x_{j+1}, y_{j+1}, 
\dotsc, x_{k-1}, y_{k-1})
    \\
\; \hrn 
    r^{\star}(w, w, x_0, y_0, \dotsc, x_{k-1}, y_{k-1}),
    L^a(y_j, y'_j), R^a(x'_j,x_j).
\end{multline}
%%
If the machine moves the head to the right we instead add the
following rule.
%%
\begin{multline}
\label{rule:right}
q^{\star}(w, x_j y_j, x_0, y_0, \dotsc, x_{j-1}, y_{j-1},
    x'_j, y'_j, x_{j+1}, y_{j+1}, \dotsc, x_{k-1}, y_{k-1})
    \\ 
\; \hrn
    r^{\star}(w, w, x_0, y_0, \dotsc, x_{k-1}, y_{k-1}),
    R^a(x_j, x'_j), L^a(y'_j, y_j).
\end{multline}
%%
If the machine does not move the head, then the following rule is
added.
%%
\begin{equation}
\label{rule:stay}
q^{\star}(w, w', x_0, y_0, \dotsc, x_{k-1}, y_{k-1}) 
\hrn 
   r^{\star}(w, w', x_0, y_0, \dotsc, x_{k-1}, y_{k-1}).
\end{equation}
%%
Notice that the first two argument places of the predicate are used
to get rid of `superfluous' variables. If the state $q$ is
universal and if there are exactly $p$ transitions with
successor states $r_i$, $i < p$, which do not need to be
different, then $q^{\star}$ becomes $2k+2$--ary and we
introduce symbols $q(i)^{\star}$, $i < p$, which are
$2k+2$--ary. Now, first the following rule is introduced.
%%
\begin{align}
& q^{\star}(w, w', x_0, y_0, \dotsc, x_{k-1}, y_{k-1}) \hrn \\\notag
& \qquad q(0)^{\star}(w, w', x_0, y_0, \dotsc, x_{k-1}, y_{k-1}), \\\notag
& \qquad q(1)^{\star}(w, w', x_0, y_0, \dotsc, x_{k-1}, y_{k-1}), \\\notag
& \qquad \dotsc, \\\notag
& \qquad q(p-1)^{\star}(w, w', x_0, y_0, \dotsc, x_{k-1}, y_{k-1}).
\end{align}
%%
Second, if the transition $i$ consists in the state $q$ changing 
to $r_i$ when reading the symbol $a$ and if the machine moves to 
the left on Tape $j$, $G$ gets \eqref{rule:left} with $q^{\star}(i)$ 
in place $q^{\star}$ and $r^{\star}_j$ in place of $r^{\star}$.
If movement is to the right, instead we use \eqref{rule:right}.
If the machine does not move the head, then \eqref{rule:stay} 
is added.

All of these rules are simple. If $q$ is an accepting state,
then we also take the following rule on board.
%%
\begin{equation}
q^{\star}(w,w',x_0, y_0, \dotsc, x_{k-1}, y_{k-1}) \hrn .
\end{equation}
%%
The last rule we need is
%%
\begin{equation}
S(w) \hrn q_0^{\star}(w, w, \varepsilon,
    w, \varepsilon, w, \dotsc, \varepsilon, w).
\end{equation}
%%
This is a simple rule. For the variable $w$ occurs to the
left only once. With this definition made we have to show
that $L(G) = L$. Since $L = L(T)$ it suffices to show
that $L(G) = L(T)$. We have $\vec{w} \in L(T)$ if
there is an $n \in \omega$ such that $T$ moves into an
accepting state from the initial configuration for
$\vec{w}$. Here the initial configuration is as follows.
On all tapes we have $\vec{w}$ and the read heads are
to the left of the input. An end configuration is a
configuration from which no further moves are possible.
It is accepted if the machine is in a universal state.

We say that $\zeta := q^{\star}(\vec{w}, \vec{w}, %
\vec{x}_0, \vec{y}_0, \dotsc, \vec{x}_{k-1}, \vec{y}_{k-1})$
codes the configuration $\zeta^K$ where $T$ is in state $q$, 
and for each $i < k$ (a) Tape $i$ is filled  by $x_iy_i$, 
and (b) the read head Tape $i$ is on the symbol immediately 
following $\vec{x}_i$. Now we have:
%%
\begin{dingautolist}{192}
\item $\vdash_G^0 \zeta$ iff
    $\zeta^K$ is an accepting end configuration.
\item If $q$ is existential then $\zeta \hrn \eta$ is an
    instance of a rule of $G$ iff
    $T$ computes $\eta^K$ from $\zeta^K$ in one step.
\item If $q$ is universal then $\zeta$ is derivable from
     $\eta_i$, $i < p$, in two rule steps iff
    $T$ computes the transitions
    $\zeta^K \pf \eta_i^K$ ($i < p$).
\end{dingautolist}
%%
Let $\vec{w} \in L(G)$. This means that
$\vdash_G^n S(\vec{w})$ and so
%%
\begin{equation}
\vdash_G^{n-1} \zeta := q_0^{\star}(\vec{w}, \vec{w}, \varepsilon,
    \vec{w}, \varepsilon, \vec{w}, \dotsc, \varepsilon,
    \vec{w}).
\end{equation}
%%
This corresponds to the initial configuration of
$T$ for the input $\vec{w}$. We conclude from what we have said
above that if $\vdash_G^{n-1} \zeta$ there exists a
$k \leq n$ such that $T$ accepts $\zeta^K$ in $k$ steps.
Furthermore: if $T$ accepts $\zeta^K$ in $k$ steps, then
$\vdash_G^{2k} \zeta$. Hence we have $L(G) = L(T)$.
%%
\begin{thm}[Groenink]
%%%
\index{Groenink, Annius}%%%
%%%
$L$ is accepted by a simple LMG iff
$L \in \mbox{\textbf{PTIME}}$.
\end{thm}
%%
{\it Notes on this section.} 
LMGs are identical to \textbf{Elementary Formal Systems} (\textbf{EFS}s) %%%
%%%
\index{elementary formal system}%%
%%% 
defined in \cite{smullyan:formal}, Page 4. Smullyan used them to 
define recursion without the help of a machine. \cite{post:formal} 
did the same, however his rules are more akin to actions of a Turing 
machine than to rules of an EFS (or an LMG). There is an alternative 
characterization of \textbf{PTIME}--languages. Let 
$\mathsf{FOL}(\mathsf{LFP})$ be 
the expansion of first--order predicate logic (with constants for 
each letter and a single binary symbol $<$ in addition to equality)
by the least--fixed point operator. Then the \textbf{PTIME}--languages 
are exactly those that can be defined in $\mathsf{FOL}(\mathsf{LFP})$. 
A proof can be found in \cite{ebbinghausflum:finite}. 
%%
\vplatz
\exercise
Prove Theorem~\ref{thm:lbgra}. {\it Hint.} You have to simulate
the actions of a Turing machine by the grammar. Here we code the
configuration by means of the string, the states by means of
the predicates.
%%
\vplatz
\exercise
Prove Theorem~\ref{thm:npol}.
%%
\vplatz
\exercise
Construct a simple 1--LMG $G$ such that 
$\{\mbox{\tt a}^n\mbox{\tt b}^n\mbox{\tt c}^n :
n \in \omega\} = L(G)$.
%%
\vplatz
\exercise
Let $G = \auf A, R, \Xi, S, H\zu$ be an LMG which generates
$L$. Furthermore, let $U$ be the language of all
$\vec{x}$ whose Parikh image is that of some $\vec{y} \in L$.
(In other words: $U$ is the permutation closure of $L$.)
Let 
%%%
\begin{equation}
G^p := \auf A, R \cup \{S^{\heartsuit}\}, \Xi^{\heartsuit},
S^{\heartsuit}, H^p\zu
\end{equation}
%%%
where $\Xi(S^{\heartsuit}) = 1$, and let
%%
\begin{equation}
H^p := R \cup \{S^{\heartsuit}(x) \hrn
    S(x).; S^{\heartsuit}(vyxz) \hrn S^{\heartsuit}(vxyz).\}
\end{equation}
%%
Show that $L(G^p) = U$.
%%
\vplatz
\exercise
Let $L$ be the set of all theorems of intuitionistic logic. Write
a 1--LMG that generates this set. {\it Hint.} You may use the
Hilbert--style calculus here.

 \newcommand{\sotimes}{\mbox{\small$\otimes$}}
\section{Interpreted LMGs}
\label{kap4-3}
%
%
%
In this section we shall concern ourselves with interpreted LMGs.
The basic idea behind interpreted LMGs is quite simple.
Every rule is connected with a function which tells us how
the meanings of the elements on the right hand side are
used to construct the meaning of the item on the left. We
shall give an example. The following grammar generates ---
as we have shown above  --- the language $\{\mbox{\tt a}^{2^n} :
n \geq 0\}$.
%%
\begin{equation}
\label{eq:531}
S(xx) \hrn S(x).\qquad S(\mbox{\tt a}) \hrn .
\end{equation}
%%
We write a grammar which generates all pairs
$\auf \mbox{\tt a}^{2^n}, n\zu$. So, we take the number
$n$ to be the meaning of the string $\mbox{\tt a}^{2^n}$.
For the first rule we choose the function $\lambda n.n+1$
as the meaning function and for the second the constant
$0$. We shall adapt the notation to the one used previously
and write as follows.
%%
\begin{equation}
\label{eq:53ast}
\mbox{\tt aaaa} : S : 2 \qquad \mbox{ or } \qquad
\auf \mbox{\tt aaaa}, S, 2\zu
\end{equation}
%%
Both notations will be used concurrently. \eqref{eq:53ast} names 
a sign with exponent {\tt aaaa} with category (or predicate) $S$ and
with meaning 2. The rules of the above grammar are written
as follows:
%%
\begin{equation}
\auf xx, S, n+1\zu \hrn
\auf x, S, n\zu. \qquad
\auf \mbox{\tt a}, S, 0\zu\hrn .
\end{equation}
%%
This grammar is easily transformed into a sign grammar.
We define a 0--ary mode {\mtt A$_{\snull}$} and a unary
mode {\tt A$_{\seins}$}.
%%
\begin{equation}
\begin{split}
\mbox{\mtt A$_{\snull}$} & := \auf \mbox{\tt a}, S, 0\zu, \\
\mbox{\mtt A$_{\seins}$}(\auf x, S, n\zu) &
    := \auf xx, S, n+1\zu.
\end{split}
\end{equation}
%%
The structure term 
{\mtt A$_{\seins}$A$_{\seins}$A$_{\seins}$A$_{\snull}$} 
for example defines the sign $\auf \mbox{\tt a}^8, S, 3\zu$.

It seems that one can always define a sign grammar from a LMGs
in this way. However, this is not so. Consider
adding the following rule to \eqref{eq:531}.
%%
\begin{equation}
\auf x\mbox{\tt ab}y, S, 3n\zu
\hrn \auf x\mbox{\tt aa}y, S, n\zu.
\end{equation}
%%
The problem with this rule is that the left hand side is not
uniquely determined by the right hand side. For example, from
$\auf \mbox{\tt aaaa}, S, 2\zu$ we can derive in one step
$\auf \mbox{\tt abaa}, S,6\zu$ as well as
$\auf \mbox{\tt aaba}, S,6\zu$ and $\auf \mbox{\tt aaab},
S, 6\zu$. We shall therefore agree on the following.
%%
\begin{defn}
Let
%%
\begin{align}
& \rho = T(t^0, t^1, \dotsc, t^{p-1}) \hrn
U_0(s^0_0, s^1_0, \dotsc, s^{q_0-1}_0), \\\notag
& \qquad\; 
U_1(s^0_1, s^1_1, \dotsc, s^{q_1-1}_1), \dotsc, 
U_{n-1}(s^0_{n-1}, s^1_{n-1}, \dotsc, s^{q_n -1}_{n-1})
\end{align}
%%
be a rule. $\rho$ is called \textbf{definite} 
%%%
\index{rule!definite}%%
%%%
if for all instances of the rule the following holds: For all
$\alpha$, if the $(s^j_i)^{\alpha}$ are given, the
$(t^j)^{\alpha}$ are uniquely determined. An
%%%
\index{literal movement grammar!definite}
%%%
LMG is called \textbf{definite} if each of its rules is definite.
\end{defn}
%%
Clearly, to be able to transform an LMG into a sign grammar we 
need that it is definite. However, this is still a very general 
concept. Hence we shall restrict our attention to simple LMGs. 
They are definite, as is easily seen. These grammars have the 
advantage that the $s^j_i$ are variables over strings and
the $t^j$ polynomials. We can therefore write them
in $\lambda$--notation. Our grammar can therefore be
specified as follows.
%%
\begin{equation}
\begin{split}
\mbox{\mtt A$_{\snull}$} & := \auf \mbox{\tt a}, S, 0\zu \\
\mbox{\mtt A$_{\seins}$} & := \auf \lambda x.x \conc x, S, \lambda n.n+1\zu
\end{split}
\end{equation}
%%
In certain cases the situation is not so simple. For this
specification only works if a variable of the right hand side
occurs there only once. If it occurs several times, we
cannot regard the $t^j$ as polynomials using concatenation.
Namely, they are partial, as is easily seen. An easy example
is provided by the following rule.
%%
\begin{equation}
C(x) \hrn A(x), B(x).
\end{equation}
%%
Intuitively, one would choose $\lambda x.x$ for the string
function; however, how does one ensure that the two strings
on the right hand side are equal? For suppose we were
to introduce a binary mode {\tt C}.
%%
\begin{equation}
\mbox{\tt C}(\auf \vec{x},\alpha,X\zu, \auf \vec{y},\beta,Y\zu)
:= \auf \mbox{\tt C}^{\varepsilon}(\vec{x},\vec{y}),
    \mbox{\tt C}^{\tau}(\alpha,\beta),
    \mbox{\tt C}^{\mu}(X,Y)\zu
\end{equation}
%%
Then we must ensure that
$\mbox{\tt C}^{\varepsilon}(\vec{x},\vec{y})$ is only defined
if $\vec{x} = \vec{y}$. So in addition to concatenation on $A^{\ast}$ 
we also have to have a binary operation $\iota$, which is defined 
as follows.
%%
\begin{equation}
\iota(\vec{x}, \vec{y}) := 
	\begin{cases}
    \vec{x} & \text{ if $\vec{x} = \vec{y}$,} \\
    \star   & \text{ otherwise.}
    	\end{cases}
\end{equation}
%%
With the help of this operation we can transform the rule
into a binary mode. Then we simply put $\mbox{\tt %
C}^{\varepsilon} := \lambda x.\lambda y.\iota(x,y)$.

We shall try out our concepts by giving a few examples.
Let $\vec{x} \in \{\mbox{\tt L}, \mbox{\tt O}\}^{\ast}$ be
a binary sequence. This is the binary code $n^{\flat}$ of a
natural number $n$. This binary sequence we shall take as the
meaning of the same number in Turing code. For the number $n$ 
it is the sequence $n^{\sharp} := \mbox{\tt a}^{n+1}$. Here
is a grammar for the language $\{\auf n^{\sharp}, S, n^{\flat}\zu
: n \in \omega\}$.
%%
\begin{equation}
\begin{split} 
\auf x \, \mbox{\tt a}, S, n\zu & \hrn
    \auf x, T, n\zu. \\
\auf xx \, \mbox{\tt a}, T, n \conc \mbox{\tt L}\zu  & \hrn
    \auf x, T, n\zu. \\
\auf xx, T, n \conc \mbox{\tt O}\zu  & \hrn
    \auf x, T, n\zu. \\
\auf \varepsilon, T, \varepsilon\zu & \hrn.
\end{split}
\end{equation}
%%
Notice that the meanings are likewise computed using concatenation.
In place of $\lambda n.2n$ or $\lambda n.2n+1$ we therefore have
$\lambda x.x \conc \mbox{\tt O}$ and
$\lambda x.x \conc \mbox{\tt L}$.

We can also write a grammar which transforms binary codes into
Turing codes, by simply exchanging exponent and meaning. 

A somewhat more complex example is a grammar which derives
triples $\auf \vec{x}\sotimes \vec{y}, S, \vec{z}\zu$ of binary 
numbers where $\vec{z}$ is the binary code of the sum of 
the numbers represented by $\vec{x}$ and $\vec{y}$. (The symbol 
$\sotimes$ 
%%%%
\index{$\sotimes$}%%
%%%%
serves to separate $\vec{x}$ from $\vec{y}$.)
%%
\begin{subequations}
\begin{align}
\auf x\sotimes y, S, z \zu & \hrn \auf x\sotimes y, A, z\zu. \\\notag
\auf \mbox{\tt O}x\sotimes y, S, z\zu & \hrn \auf x\sotimes y, S, z\zu.
	\\\notag
\auf x\sotimes \mbox{\tt O}y, S, z\zu & \hrn \auf x\sotimes y, S, z\zu.
	 \\\notag
\auf x\sotimes y, S, \mbox{\tt O}z\zu & \hrn \auf x\sotimes y, S, z\zu.
	 \\\notag
\auf x\sotimes y, S, \mbox{\mtt L}z\zu & \hrn \auf x\sotimes y, U, z\zu. 
	\\
\auf \mbox{\tt O}x\sotimes \mbox{\tt O}y, A, \mbox{\tt O}z\zu
    & \hrn \auf x\sotimes y, A, z\zu. \\\notag
\auf \mbox{\tt O}x\sotimes \mbox{\tt O}y, A, \mbox{\tt L}z\zu
    & \hrn \auf x\sotimes y, U, z\zu. \\\notag
\auf \mbox{\tt O}x\sotimes \mbox{\tt L}y, U, \mbox{\tt O}z\zu
    & \hrn \auf x\sotimes y, U, z\zu. \\\notag
\auf \mbox{\tt O}x\sotimes \mbox{\tt L}y, A, \mbox{\tt L}z\zu
    & \hrn \auf x\sotimes y, A, z\zu. \\\notag
\auf \mbox{\tt L}x\sotimes \mbox{\tt O}y, U, \mbox{\tt O}z\zu
    & \hrn \auf x\sotimes y, U, z\zu. \\\notag
\auf \mbox{\tt L}x\sotimes \mbox{\tt O}y, A, \mbox{\tt L}z\zu
    & \hrn \auf x\sotimes y, A, z\zu. \\\notag
\auf \mbox{\tt L}x\sotimes \mbox{\tt L}y, U, \mbox{\tt O}z\zu
    & \hrn \auf x\sotimes y, A, z\zu. \\\notag
\auf \mbox{\tt L}x\sotimes \mbox{\tt L}y, U, \mbox{\tt L}z\zu
    & \hrn \auf x\sotimes y, U, z\zu. \\
\auf \mbox{\tt O}\sotimes \mbox{\tt O}, A, \mbox{\tt O}\zu & \hrn . 
    & 
\auf \mbox{\tt O}\sotimes \mbox{\tt L}, A, \mbox{\tt L}\zu \hrn . \\\notag
\auf \mbox{\tt L}\sotimes \mbox{\tt O}, A, \mbox{\tt L}\zu & \hrn .
    &
\auf \mbox{\tt L}\sotimes \mbox{\tt L}, U, \mbox{\tt O}\zu \hrn .
\end{align}
\end{subequations}
%%
Now let us return to the specification of interpreted LMGs.
First of all we shall ask how LMGs can be interpreted to
become sign grammars. To this end we have to reconsider our
notion of an exponent. Up to now we have assumed that exponents
are strings. Now we have to assume that they are sequences of
strings (we say rather `vectors of strings', since strings are
themselves sequences). This motivates the following definition.
%%
\begin{defn}
Let $A$ be a finite set. We denote by $V(A) := \bigcup_{k < \omega} 
(A^{\ast})^k$ the set of vectors of strings over $A$. Furthermore, 
let $F^{V} := \{\varepsilon, 0, \conc, \sotimes, \triangleright, 
\triangleleft, \zeta, \iota\}$, 
%%%
\index{$\sotimes$, $\triangleleft$, $\triangleright$, $\zeta$, $\iota$}%%%
%%%
$\Omega(\conc) = \Omega(\sotimes) =
\Omega(\iota) = 2$, $\Omega(\triangleright) = \Omega(\triangleleft) =
\Omega(\zeta) = 1$; $\Omega(\varepsilon) = \Omega(0) = 0$. Here, the
following is assumed to hold. (Strings are denoted 
by vector arrows, while $\Gx$, $\Gy$ and $\Gz$ range over $V(A)$.)
%%
\begin{dingautolist}{192}
\item
$\Gz \sotimes (\Gy \sotimes \Gz) = (\Gz \sotimes \Gy) \sotimes \Gz$
\item
$0 = \auf\,\zu$ is the empty sequence.
\item
$\conc$ is the usual concatenation of strings, so it is not defined
on vectors of length $\neq 1$.
\item
$\varepsilon$ is the empty string.
\item
$\triangleright \bigotimes_{i< m} \vec{x} =
\bigotimes_{i < m-1} \vec{x}_i$, and $\triangleleft \bigotimes_{i < m}
\vec{x}_i = \bigotimes_{0 < i < m} \vec{x}_i$.
\item
$\zeta(\Gx) = \star$ if $\Gx$ is not a string;
$\zeta(\vec{x}) = \vec{x}$ otherwise.
\item
$\iota(\Gx,\Gy) = \Gx$ if $\Gx = \Gy$ and
$\iota(\Gx, \Gy) = \star$ otherwise.
\end{dingautolist}
%%
The resulting (partial)
%%%
\index{string vector algebra}%%
%%%
algebra is called the \textbf{algebra of string vectors over} $A$ 
and is denoted by $\GV(A)$.
%%%
\index{$\GV(\GA)$}%%
%%%
\end{defn}
%%
In this algebra the following laws hold among other.
%%
\begin{equation}
\begin{split}
\Gx \sotimes 0 & = \Gx \\
0 \sotimes \Gx & = \Gx \\
\Gx \sotimes (\Gy \sotimes \Gz) & = (\Gx \sotimes \Gy) \sotimes \Gz \\
\triangleright (\Gx \sotimes \zeta (\Gy)) & = \Gx \\
\triangleleft (\zeta (\Gy) \sotimes \Gx) & = \Gx
\end{split}
\end{equation}
%%
The fourth and fifth equation hold under the condition
that $\zeta (\Gy)$ is defined. A vector $\Gx$ has length
$m$ if $\zeta (\triangleright^{m-1}\Gx)$ is defined but 
$\zeta(\triangleright^m\Gx)$ is not. In this case
$\zeta (\triangleright^{m-(i+1)}\triangleleft^i \Gx)$
is defined for all $i < m$ and they are the
projection functions. Now we have:
%%
\begin{equation}
\Gx = \triangleright^{m-1}\Gx \sotimes \triangleright^{m-2}
\triangleleft \Gx \sotimes \dotsb \sotimes
\triangleright \triangleleft^{m-2}\Gx \sotimes
\triangleleft^{m-1}\Gx.
\end{equation}
%%
All polynomial functions that appear in the sequel can be
defined in this algebra. The basis is the following theorem.
%%
\begin{thm}
Let $p \colon (A^{\ast})^m \pf A^{\ast}$ be a function which is
a polynomial in $\conc$ and $\iota$. Then there exists a
vector polynomial $\Gq \colon V(A) \pf V(A)$ such that
%%
\begin{dingautolist}{192}
\item $\Gq(\Gx)$ is defined only if $\Gx \in (A^{\ast})^m$.
\item If $\Gx \in (A^{\ast})^m$ and $\Gx = \auf \vec{x}_i :
    i < m\zu$ then
    $\Gq(\Gx) = p(\vec{x}_0, \dotsc, \vec{x}_{m-1})$.
\end{dingautolist}
\end{thm}
%%
\proofbeg
Let $p$ be given. We assume that one of the variables appears
at least once. (Otherwise $p = \varepsilon$ and then we put
$q := \varepsilon$.) Let $q$ arise from $p$ by replacement
of $x_i$ by $\zeta(\triangleright^{m-(i+1)} \triangleleft^i \Gx)$,
for all $i < m$. This defines $q$. (It is well defined, for
the symbols $\varepsilon$, $\conc$, $\iota$ are in the signature
$F^V$.) Let now $\Gx$ be given. As remarked above, $q$ is
defined on $\Gx$ only if $\Gx$ has length $m$. In this case
$\Gx = \auf \vec{x}_i : i < n\zu$ for certain $\vec{x}_i$, and
we have $\vec{x}_i = \zeta(\triangleright^{m-(i+1)} \triangleleft^i \Gx)$.
Since the symbols $\varepsilon$, $\conc$ and $\iota$ coincide on
the strings in both algebras (that of the strings and that
of the vectors) we have $q(\Gx) = q(\vec{x}_0, \dotsc, \vec{x}_{m-1})$.
\proofend

That $p \colon (A^{\ast})^m \pf (A^{\ast})^n$ is a polynomial function
means that there exist polynomials $p_i$, $i < n$, such that
%%
\begin{equation}
p(\vec{x}_0, \dotsc, \vec{x}_{m-1}) =
\auf p_i(\vec{x}_0, \dotsc, \vec{x}_{m-1}) : i < n\zu.
\end{equation}
%%
We can therefore replace the polynomials on strings by
polynomials over vectors of strings. Thise simplifies the
presentation of LMGs considerably. We can now write down a
rule as follows.
%%
\begin{align}
& \auf q(\Gx_0, \dotsc, \Gx_{m-1}), A, f(X_0,\dotsc,X_{m-1})\zu \\\notag
& \qquad \hrn 
\auf \Gx_0, B_0, X_0\zu, \dotsc,
\auf \Gx_{m-1}, B_{m-1}, X_{m-1}\zu.
\end{align}
%%
We shall make a further step and consider LMGs as categorial
grammars. To this end we shall first go over to Chomsky
Normal Form. This actually brings up a surprise. For there are
$k$--LMGs for which no $k$--LMG in Chomsky Normal Form can be
produced (see the exercises). However, there exists a
$k'$--LMG in Chomsky Normal Form for a certain effectively
determinable $k' \leq \pi k$, where $\pi$ is the maximal
productivity of a rule. Namely, look at a rule. We introduce new
symbols $Z_i$, $i < m-2$, and replace this rule by the
following rules.
%%
\begin{equation}
\begin{split}
& \auf \Gx_0 \sotimes \Gx_1, Z_0, X_0 \times X_1) 
    \quad \hrn \quad \auf \Gx_0, B_0, X_0\zu, \quad
    \auf \Gx_1, B_1, X_1\zu. \\
& \auf \Gy_0 \sotimes \Gx_2, Z_1, Y_0 \times X_2) 
    \quad \hrn \quad
    \auf \Gy_0, Z_0, Y_0\zu, \quad
    \auf \Gx_2, B_2, X_2\zu. \\
& \dotsb \\
& \auf \Gy_{m-4} \sotimes \Gx_{m-2}, Z_{m-3}, %
	Y_{m-4} \times X_{m-2}\zu \\
& \qquad \hrn \quad \auf \Gy_{m-4}, Z_{m-4}, Y_{m-4}\zu
    \quad \auf \Gx_{m-2}, B_{m-2}, X_{m-2}\zu. \\
& \auf q^{\ast}(\Gy_{m-3}\sotimes\Gx_{m-1}), A,
    f^{\ast}(Y_{m-3} \times X_{m-1})\zu \\
& \qquad \hrn\quad
    \auf \Gy_{m-3}, Z_{m-3}, Y_{m-3}\zu, \quad
    \auf \Gx_{m-1}, B_{m-1}, X_{m-1}\zu.
\end{split}
\end{equation}
%%
Here $q^{\ast}$ and $f^{\ast}$ are chosen in such a way that
%%
\begin{equation}
\begin{split}
q^{\ast}(\Gx_0 \sotimes \dotsm \sotimes \Gx_{m-1}) & = 
    q(\Gx_0, \dotsc, \Gx_{m-1}) \\
f^{\ast}(X_0 \times \dotsb \times X_{m-1}) & = 
    f(X_0, \dotsc, X_{m-1})
\end{split}
\end{equation}
%%
It is not hard to see how to define the functions by polynomials.
Hence, in the sequel we may assume that we have at most binary
branching rules. 0--ary rules are the terminal rules.
A unary rule has the following form.
%%
\begin{equation}
\auf q(\Gx), C, f(X)\zu \hrn \auf \Gx, A, X\zu.
\end{equation}
%%
We keep the sign $\auf \Gx, A, X\zu$ and introduce a new
sign $\mbox{\tt Z}_{\rho}$ which has the following form.
%%
\begin{equation}
\mbox{\tt Z}_{\rho} := \auf \lambda \Gx.q(\Gx), C/A,
\lambda x.f(x)\zu
\end{equation}
%%
There is only one binary mode, {\tt C}, which is defined thus:
%%
\begin{equation}
\mbox{\tt C}(\auf p, A, X\zu, \auf q, B, Y\zu) :=
\auf p(q), A \cdot B, (XY)\zu 
\end{equation}
%%
This is exactly the scheme of application in categorial grammar.
One difference remains. The polynomial $p$ is not necessarily
concatenation. Furthermore, we do not have to distinguish
between two modes, since we in the string case we have the 
possibility of putting $p$ to be either $\lambda x.x \conc \vec{y}$ 
or $\lambda x.\vec{y} \conc x$.
Application has in this way become independent of the accidental
order. Many more operations can be put here, for example
reduplication. The grammar that
we have mentioned at the beginning of the section is defined
by the following two modes.
%%
\begin{equation}
\begin{split}
\mbox{\mtt D}_{\snull} & := \auf \mbox{\tt a}, S, 0\zu \\
\mbox{\mtt D}_{\seins} & := \auf \lambda x.x \conc x, S/S, \lambda n.n+1\zu
\end{split}
\end{equation}
%%
To the previous structure term
{\mtt A$_{\seins}$A$_{\seins}$A$_{\seins}$A$_{\snull}$} 
now corresponds the structure term
%%
\begin{equation}
\mbox{\mtt CD$_{\seins}$CD$_{\seins}$CD$_{\seins}$D$_{\snull}$}
\end{equation}
%%
In this way the grammar has become an AB--grammar, with
one exception: the treatment of strings must be explicitly defined.

The binary rules remain. A binary rule has the following form.
%%
\begin{equation}
\auf q(\Gx, \Gy), C, f(X,Y)\zu \hrn
    \auf \Gx, A, X\zu, \auf \Gy, B, Y\zu.
\end{equation}
%%
We keep the sign on the right hand side and introduce a
new sign.
%%
\begin{equation}
\mbox{\tt Z}_{\rho} :=
    \auf \lambda \Gy.\lambda \Gx.q(\Gx,\Gy),
    (C/A)/B, \lambda y.\lambda x.f(x,y)\zu
\end{equation}
%%
%% Order of arguments (Code in the category)
\vplatz
\exercise
Show that for any $k > 1$ there are simple $k$--LMGs $G$
with branching number $3$ such that for no simple $k$--LMG
$H$ with branching number 2, $L(G) = L(H)$.
%%
\vplatz
\exercise
\label{ex:arabic}
Here are some facts from Arabic.
%%%
\index{Arabic}%%
%%%
In Arabic a root typically consists of three consonants. 
Examples are {\tt ktb} `to write', {\tt drs} `to study'. There 
are also roots with four letters, such as {\tt drhm} (from Greek
{\it Drachme\/}), which names a numismatic unit. From a
root one forms so--called  {\it binyanim}, roughly translated
as `word classes', by inserting vowels or changing the
consonantism of the root. In Table~\ref{tab:531} we give some 
examples of verbs derived from the root {\tt ktb}.
%%
\begin{table}
\caption{Arabic Binyanim: {\tt ktb}}
\label{tab:531} 
\begin{center}
\begin{tabular}{llll}
 & \mbox{\rm I} & \mbox{\tt katab} & \mbox{\rm to write} \\
 & \mbox{\rm II} & \mbox{\tt kattab} & \mbox{\rm to make write}\\
 & \mbox{\rm III} & \mbox{\tt kaatab} & \mbox{\rm to correspond} \\
 & \mbox{\rm VI} & \mbox{\tt takaatab} & \mbox{\rm to write to each other} \\
 & \mbox{\rm VIII} & \mbox{\tt ktatab} & \mbox{\rm to write, to be inscribed}
\end{tabular} \\[3mm]
\begin{tabular}{lllll}
      & Perf. Act. & Perf. Pass. & Impf. Act. & Impf. Pass. \\\hline
 \mbox{\rm I}    & \mbox{\tt katab} & \mbox{\tt kutib} & 
	\mbox{\tt aktub} & \mbox{\tt uktab} \\
 \mbox{\rm II}   & \mbox{\tt kattab} & \mbox{\tt kuttib} & 
	\mbox{\tt ukattib} & \mbox{\tt ukattab} \\
 \mbox{\rm III}  & \mbox{\tt kaatab} & \mbox{\tt kuutib} & 
	\mbox{\tt ukaatib} & \mbox{\tt ukaatab} \\
 \mbox{\rm VI}   & \mbox{\tt takaatab} & \mbox{\tt tukuutib} & 
	\mbox{\tt atakattab} & \mbox{\tt utakattab} \\
 \mbox{\rm VIII} & \mbox{\tt ktatab} & \mbox{\tt ktutib} & 
	\mbox{\tt aktatib} & \mbox{\tt uktatab}
\end{tabular}
\end{center}
\end{table}
%%
Of these forms we can in turn form verbal forms in different
tenses and voices.
%%

%%
We have only shown the transparent cases, there are other classes
whose forms are not so regular. Write an interpreted LMG that
generates these forms. For the meanings, simply assume unary
operators, for example $\mbox{\sf caus}'$ for II, $\mbox{\sf pass}'$
for passive, and so on.
%%
\vplatz
\exercise
\label{ex:chinese}
\index{Mandarin}\index{Chinese}%%%
In Mandarin (a Chinese language) a yes--no--question is formed as follows. 
A simple assertive sentence has the form \eqref{ex:533} and
the corresponding negative sentence the form \eqref{ex:534}.
Mandarin is an SVO--language, and so the verb phrase follows the
subject. The verb phrase is negated by prefixing {\tt bu}.
(We do not write tones.)
%%%
\begin{align}
\label{ex:533} & \mbox{\tt Ta zai jia.} \\\notag 
	& \mbox{\rm He/She/It (is) at home} \\
\label{ex:534} & \mbox{\tt Ta bu zai jia.} \\\notag
                & \mbox{\rm He/She/It (is) not at home}
\end{align}
%%
The yes--no--question is formed by concatenating the subject
phrase with the positive verb phrase and then the negated verb 
phrase.
%%%
\begin{align}
\label{ex:535} & \mbox{\tt Ta zai jia bu zai jia?} \\\notag
                & \mbox{\it Is he/she/it at home?}
\end{align}
%%
As Radzinski~\shortcite{radzinski:copying} 
%%%
\index{Radzinski, Daniel}%%%
%%%%
argues, the verb phrases 
have to be completely identical (with the exception of {\tt bu}). For 
example, \eqref{ex:536} is grammatical, \eqref{ex:537} is ungrammatical. 
However, \eqref{ex:538} is again grammatical and means roughly what 
\eqref{ex:536} means.
%%%
\begin{align}
\label{ex:536} & \mbox{\tt Ni xihuan ta-de chenshan bu xihuan ta-de} \\\notag
 & \mbox{\rm You like his shirt not like his} \\\notag
    & \mbox{\tt chenshan?} \\\notag
    & \mbox{\rm shirt?} \\\notag
    & \mbox{\it Do you like his shirt?} \\
\label{ex:537} & 
	^{\ast}\mbox{\tt Ni xihuan ta-de bu xihuan ta-de chenshan?} \\\notag
    & \mbox{\rm You like his not like his shirt?} \\
\label{ex:538} & \mbox{\tt Ni xihuan bu xihuan ta-de chenshan?} \\\notag 
    & \mbox{\rm You like not like his shirt?}
\end{align}
%%
Write an interpreted LMG generating these examples. Use $?:\varphi$
to denote the question, whether or not $\varphi$ is the case.

 \section{Discontinuity}
%%
%%
\label{kap4-4}
%
%
%
In this section we shall study a very important type of grammars,
the so--called \textbf{Linear Context--Free Rewrite Systems} --- {\bf
LCFRS} for short  (see \cite{vijay-weir-joshi:LCFRS}). These 
grammars are weakly equivalent to what we call linear LMGs.  
%%
\begin{defn}
%%%
\index{literal movement grammar!linear}%%
%%%
A $k$--LMG is called \textbf{linear} if it is a simple
$k$--LMG and every rule which is not 0--ary is downward
nondeleting and downward linear, while 0--ary rules have
the form $X(\vec{x}_0, \dotsc, \vec{x}_{\Xi(X)-1}) 
\hrn .$ with $\vec{x}_i \in A$ for all $i < \Xi(X)$.
\end{defn}
%%
In other words, if we have a rule of this form
%%
\begin{equation}
A(t_0, \dotsc, t_{k-1}) \hrn
B_0(s_0^0, \dotsc, s^0_{k-1}), 
\dotsc, B_{n-1}(s_0^{n-1}, \dotsc, s_{k-1}^{n-1}).
\end{equation}
%%
then for every $i < k$ and every $j < n$: $s_i^j = x_i^j$ 
and $x^i_j = x^{i'}_{j'}$ implies $i = i'$ and $j = j'$. 
Finally, $\prod_{i < k} t_i$ is a term containing each of these 
variables exactly once, in addition to occurrences of constants.

It is easy to see that the generative capacity is not diminished 
if we disallowed constants. In case $k = 1$ we get exactly the 
CFGs, though in somewhat disguised form: for now 
--- if there are no constants --- a rule is of the form
%%
\begin{equation}
A(\prod_{i < n} x_{\pi(i)}) \hrn
B_0(x_0), B_1(x_1), \dotsc, B(x_{n-1}).
\end{equation}
%%
where $\pi$ is a permutation of the numbers $< n$. If $\rho
:= \pi^{-1}$ is the permutation inverse to $\pi$ we can
write the rule in this way.
%%
\begin{equation}
A(\prod_{i < n} x_i) \hrn
B_{\rho(0)}(x_0), B_{\rho(1)}(x_1), \dotsc, B_{\rho(n-1)}(x_{n-1})
.
\end{equation}
%%
(To this end we replace the variable $x_i$ by the variable
$x_{\rho(i)}$ for every $i$. After that we permute the
$B_i$. The order of the conjuncts is anyway insignificant 
in an LMG.) This is as one can easily see exactly the form of
a context free rule. For we have
%%
\begin{equation}
\prod_{i< n} x_i = x_0x_1 \dotsb x_{n-1} 
\end{equation}
%%
This rule says therefore that if we have constituents
$\vec{u}_i$ of type $B_{\rho(i)}$ for $i < n$, then
$\prod_{i < n} \vec{u}_i$ is a constituent of type $A$.

The next case is $k = 2$. This defines a class of
grammars which have been introduced before, using a somewhat
different notation and which have been shown to be powerful
enough to generate non CFLs such as Swiss German. In linear 2--LMGs 
we may have rules of this kind.
%%
\begin{equation}
\begin{split}
A(x_1x_2, y_1y_2) & \hrn B(x_1, y_1), C(x_2, y_2). \\
A(x_2x_1, y_1y_2) & \hrn B(x_1, y_1), C(x_2, y_2). \\
A(y_1x_1y_2, x_2) & \hrn B(x_1, y_1), C(x_2, y_2). \\
A(x_1, y_1x_2y_2) & \hrn B(x_1, y_1), C(x_2, y_2).
\end{split}
\end{equation}
%%
The following rules, however, are excluded.
%%
\begin{align}
A(x_1, y_1y_2) & \hrn B(x_1, y_1), C(x_2, y_2). \\
A(x_2x_2x_1, y_1y_2) & \hrn B(x_1, y_1), C(x_2, y_2).
\end{align}
%%
The first is upward deleting, the second not linear. We shall see
that the language $\{\mbox{\tt a}^n\mbox{\tt b}^n\mbox{\tt c}^n
\mbox{\tt d}^n : n \in \omega\}$ can be generated by a linear
2--LMG, the language $\{\mbox{\tt a}^n \mbox{\tt b}^n %
\mbox{\tt c}^n\mbox{\tt d}^n\mbox{\tt e}^n : n \in \omega\}$
however cannot. The second fact follows from Theorem~\ref{thm:2kpump}. 
For the first language we give the following grammar.
%%
\begin{equation}
\begin{split}
\mbox{\tt S}(y_0x_0y_1,z_0x_1z_1) & \hrn
\mbox{\tt S}(x_0,x_1), \mbox{\tt A}(y_0,y_1),
\mbox{\tt B}(z_0, z_1). \\
\mbox{\tt S}(\varepsilon, \varepsilon) & \hrn . \\
\mbox{\tt A}(\mbox{\tt a},\mbox{\tt b}) & \hrn . \\
\mbox{\tt B}(\mbox{\tt c},\mbox{\tt d}) & \hrn .
\end{split}
\end{equation}
%%
This shows that $2$--linear LMGs are strictly stronger than
CFGs. As a further example we shall look again at Swiss German 
(see Section~\ref{kap2}.\ref{kap2-6}). We define the following 
grammar. (Recall that $x \oconc y := x^{\smallfrown}%
\Box^{\smallfrown}y$.) 
%%
\begin{equation}
$$\begin{array}{l@{\, \hrn \,}l@{\qquad\qquad}l@{\,\hrn\,}l}
\mbox{\tt NPa}(\mbox{\tt d'chind}) & . &
\mbox{\tt NPd}(\mbox{\tt em Hans}) & . \\
\mbox{\tt NPsm}(\mbox{\tt Jan}) & . &
\mbox{\tt NPa}(\mbox{\tt es huus}) & . \\
\mbox{\tt Vdr}(\mbox{\tt h\"alfe}) & . &
\mbox{\tt Vf}(\mbox{\tt l\"ond}) & . \\
\mbox{\tt Var}(\mbox{\tt laa}) & . \\
\mbox{\tt Van}(\mbox{\tt aastriche}) & . &
\mbox{\tt C}(\mbox{\tt das}) & . \\
\mbox{\tt NPss}(\mbox{\tt mer}) & . &
\mbox{\tt Vc}(\mbox{\tt s\"ait}) & . \\
\mbox{\tt S}(x\oconc y\conc\mbox{\tt .}) & 
	\multicolumn{3}{l}{\mbox{\tt NPsm}(x), \mbox{\tt VP}(y).} \\
\mbox{\tt VP}(x\conc\mbox{\tt ,}\oconc y) & 
	\multicolumn{3}{l}{\mbox{\tt Vc}(x), \mbox{\tt CP}(y).} \\
\mbox{\tt CP}(x\oconc y\oconc z_0\oconc z_1\oconc u) & 
	\multicolumn{3}{l}{\mbox{\tt C}(x), \mbox{\tt NPss}(y),
    \mbox{\tt VI}(z_0, z_1), \mbox{\tt Vf}(u).} \\
\mbox{\tt VI}(x\oconc z_0,y\oconc z_1) & 
	\multicolumn{3}{l}{\mbox{\tt NPa}(x), \mbox{\tt Var}(y), 
	\mbox{\tt VI}(z_0, z_1).} \\
\mbox{\tt VI}(x\oconc z_0,y\oconc z_1) & \multicolumn{3}{l}{\mbox{\tt NPd}(x),
    \mbox{\tt Vdr}(y), \mbox{\tt VI}(z_0, z_1).} \\
\mbox{\tt VI}(x,y) & \multicolumn{3}{l}{\mbox{\tt NPa}(x), \mbox{\tt Van}(y).}
\end{array}$$
\end{equation}
%%
This grammar is pretty realistic also with respect to the
constituent structure, about which more below. For simplicity
we have varied the arities of the predicates. Notice in
particular the last two rules. They are the real motor of
the Swiss German infinitive constructions. For we can
derive the following.
%%
\begin{align}
& \mbox{\tt VI}(\mbox{\tt d'chind } z_0, \mbox{\tt laa } z_1) \hrn
    \mbox{\tt VI}(z_0, z_1). \\
& \mbox{\tt VI}(\mbox{\tt em Hans } z_0, \mbox{\tt h\"alfe } z_1) \hrn
    \mbox{\tt VI}(z_0, z_1). \\
& \mbox{\tt VI}(\mbox{\tt d'chind}, \mbox{\tt aastriche}) \hrn . \\
& \mbox{\tt VI}(\mbox{\tt es huus}, \mbox{\tt aastriche}) \hrn . \\
& \mbox{\tt VI}(\mbox{\tt d'chind em Hans }z_0,
    \mbox{\tt laa h\"alfe }z_1) \hrn
    \mbox{\tt VI}(z_0, z_1). \\
& \mbox{\tt VI}(\mbox{\tt em Hans es huus }z_0,
    \mbox{\tt h\"alfe laa }z_1) \hrn
    \mbox{\tt VI}(z_0, z_1).
\end{align}
%%
However, we do not have
%%
\begin{equation}
\mbox{\tt VI}(\mbox{\tt em Hans}, \mbox{\tt laa}) \hrn .
\end{equation}
%%
The sentences of Swiss German as reported in Section~\ref{kap2}.\ref{kap2-6}
are derivable and some further sentences, which are all
grammatical.

Linear LMGs can also be characterized by the vector polynomials
which occur in the rules. We shall illustrate this by way of
example with linear 2--LMGs and here only for the at most
binary rules. We shall begin with the unary rules.
They can make use of these vector polynomials.
%%
\begin{equation}
\begin{split}
\Gi(x_0,x_1) & := \auf x_0,x_1\zu \\
\Gp_X(x_0,x_1) & := \auf x_1, x_0\zu \\
\Gp_F(x_0,x_1) & := \auf x_0x_1, \varepsilon \zu \\
\Gp_G(x_0, x_1) & := \auf x_1x_0, \varepsilon \zu \\
\Gp_H(x_0,x_1) & := \auf \varepsilon, x_0x_1\zu \\
\Gp_K(x_0,x_1) & := \auf \varepsilon, x_1x_0\zu
\end{split}
\end{equation}
%%
Then the following holds.
%%
\begin{align}
\notag
\Gp_X(\Gp_X(x_0,x_1)) & = \Gi(x_0, x_1) \\
\Gp_G(x_0,x_1) & = \Gp_F(\Gp_X(x_0,x_1)) \\
\notag
\Gp_K(x_0,x_1) & = \Gp_H(\Gp_X(x_0,x_1))
\end{align}
%%
This means that one has $\Gp_G$ at one's disposal if one also has
$\Gp_X$ and $\Gp_F$, and that one has $\Gp_F$ if one also has $\Gp_X$
and $\Gp_G$ and so on. With binary rules already the situation
gets quite complicated. Therefore we shall assume that we have all
unary polynomials. A binary vector polynomial is of the form $\auf
p_0(x_0, x_1, y_0, y_1), p_1(x_0, x_1, y_0, y_1)\zu$ such that $q
:= p_0 \conc p_1$ is linear. Given $q$ there exist exactly 5 
choices for $p_0$ and $p_1$, determined exactly by the cut--off 
point. So we only need to list $q$. Here we can assume that in
$q(x_0,x_1,y_0,y_1)$ $x_0$ always appears to the left of $x_1$ and
$y_0$ to the left of $y_1$. Further, one may also assume that
$x_0$ is to the left of $y_0$ (otherwise exchange the $x_i$ with
the $y_i$). After simplification this gives the following polynomials.
%%
\begin{equation}
\begin{split}
q_C(x_0,x_1, y_0,y_1) & := x_0x_1y_0y_1 \\
q_W(x_0,x_1, y_0,y_1) & := x_0y_0x_1y_1 \\
q_Z(x_0,x_1, y_0,y_1) & := x_0y_0y_1x_1
\end{split}
\end{equation}
%%
Let us take a look at $q_W$. From this polynomial we get the
following vector polynomials.
%%
\begin{equation}
\begin{split}
\Gq_{W0}(\auf x_0,x_1\zu,\auf y_0, y_1\zu) & :=
	\auf \varepsilon, x_0y_0x_1y_1\zu \\
\Gq_{W1}(\auf x_0,x_1\zu,\auf y_0, y_1\zu) & :=
	\auf x_0, y_0x_1y_1\zu \\
\Gq_{W2}(\auf x_0,x_1\zu,\auf y_0, y_1\zu) & :=
	\auf x_0y_0, x_1 y_1\zu \\
\Gq_{W3}(\auf x_0,x_1\zu,\auf y_0, y_1\zu) & :=
	\auf x_0y_0x_1, y_1\zu \\
\Gq_{W4}(\auf x_0,x_1\zu,\auf y_0, y_1\zu) & :=
	\auf x_0y_0x_1y_1, \varepsilon \zu
\end{split}
\end{equation}
%%
We say that a linear LMG has polynomial 
basis $Q$ if in the rules of this grammar only vector polynomials 
from $Q$ have been used. It is easy to see that if $\Gq$ is a 
polynomial that can be presented by means of polynomials from $Q$, 
then one may add $\Gq$ to $Q$ without changing the generative 
capacity. Notice also that it does not matter if the polynomial 
contains constants.  If we have, for example,
%%
\begin{equation}
\mbox{\tt X}(\mbox{\tt a}x\mbox{\tt bc}y)
    \hrn \mbox{\tt X}(x), \mbox{\tt Z}(y).
\end{equation}
%%
we can replace this by the following rules.
%%
\begin{align}
\begin{split}
\mbox{\tt X}(uxvwy) & \hrn
    \mbox{\tt A}(u), \mbox{\tt X}(x),
    \mbox{\tt B}(v), \mbox{\tt C}(w),
    \mbox{\tt Z}(y). \\
\mbox{\tt A}(\mbox{\tt a}) & \hrn . \\
\mbox{\tt B}(\mbox{\tt b}) & \hrn . \\
\mbox{\tt C}(\mbox{\tt c}) & \hrn .
\end{split}
\end{align}
%%
This is advantageous in proofs. We bring to the attention of the
reader some properties of languages that can be generated by
linear LMGs. The following is established in 
\shortcite{vijay-weir-joshi:LCFRS}, see also \cite{weir:phd}.
%%
%%%
%%
\begin{prop}[Vijay--Shanker \& Weir \& Joshi]
%%%
\index{Vijay--Shanker, K.}%%%
\index{Weir, David}%%%
\index{Joshi, Aravind}%%%
%%%
\label{prop:linear-semilinear}
Let $G$ be a linear $k$--LMG. Then $L(G)$ is semilinear.
\end{prop}
%%
A special type of linear LMGs are the so--called head grammars.
These grammars have been introduced by Carl Pollard
%%%%
\index{Pollard, Carl}%%
%%%%
in \shortcite{pollard:head}. The strings that are manipulated are
of the form $\vec{x}a\vec{y}$ where $\vec{x}$ and $\vec{y}$
are strings and $a \in A$. One speaks in this connection of
$a$ in the string as the distinguished \textbf{head}.
This head is marked here by underlining it. Strings containing
an underlined occurrence of a letter are called \textbf{marked}.
The following rules for manipulating marked strings are now
admissible. 
%%
\begin{equation}
\begin{split}
h_{C1}(\vec{v}\uli{a}\vec{w}, \vec{y}\uli{b}\vec{z}) &
    := \vec{v}\uli{a}\vec{w}\vec{y}b\vec{z} \\
h_{C2}(\vec{v}\uli{a}\vec{w}, \vec{y}\uli{b}\vec{z}) &
    := \vec{v}a\vec{w}\vec{y}\uli{b}\vec{z} \\
h_{L1}(\vec{v}\uli{a}\vec{w}, \vec{y}\uli{b}\vec{z}) &
    := \vec{v}\uli{a}\vec{y}b\vec{z}\vec{w} \\
h_{L2}(\vec{v}\uli{a}\vec{w}, \vec{y}\uli{b}\vec{z}) &
    := \vec{v}a\vec{y}\uli{b}\vec{z}\vec{w} \\
h_{R1}(\vec{v}\uli{a}\vec{w}, \vec{y}\uli{b}\vec{z}) &
    := \vec{v}\vec{y}b\vec{z}\uli{a}\vec{w} \\
h_{R2}(\vec{v}\uli{a}\vec{w}, \vec{y}\uli{b}\vec{z}) &
    := \vec{v}\vec{y}\uli{b}\vec{z}a\vec{w}
\end{split}
\end{equation}
%%
(Actually, this definition is due to \cite{roach:87}, 
%%%
\index{Roach, Kelly}%%%
%%%
who showed that Pollard's definition is weakly equivalent to this 
one.) Notice that the head is not allowed to be empty. In 
\cite{pollard:head} the functions are partial: for example, 
$h_{C1}(\varepsilon, \vec{w})$ is undefined. Subsequently, the 
definition has been changed slightly, basically to allow for empty 
heads. In place of marked strings one takes 2--vectors of strings. 
The marked head is the comma. This leads to the following definition. 
(This is due to \cite{shanker-weir-joshi:coling86}. 
See also \cite{seki}.)
%%
\begin{defn}
%%%
\index{head grammar}%%
%%%
A \textbf{head grammar} is a linear 2--LMG with the following
polynomial basis.
%%
\begin{equation}
\begin{split}
\Gp_{C1}(\auf x_0,x_1\zu,\auf y_0,y_1\zu) &
    := \auf x_0, x_1y_0y_1\zu \\
\Gp_{C2}(\auf x_0,x_1\zu,\auf y_0,y_1\zu) &
    := \auf x_0x_1y_0, y_1\zu \\
\Gp_{L1}(\auf x_0,x_1\zu,\auf y_0,y_1\zu) &
    := \auf x_0, y_0y_1x_1\zu \\
\Gp_{L2}(\auf x_0,x_1\zu,\auf y_0,y_1\zu) &
    := \auf x_0y_0, y_1x_1\zu \\
\Gp_{R1}(\auf x_0,x_1\zu,\auf y_0,y_1\zu) &
    := \auf x_0y_0y_1, x_1\zu 
%\\
%\Gp_{R2}(\auf x_0,x_1\zu,\auf y_0,y_1\zu) &
%    := \auf x_0y_0, y_1 x_1\zu
\end{split}
\end{equation}
%%
\end{defn}
%%
It is not difficult to show that the following basis of polynomials 
is sufficient: $\Gp_{C1}$, $\Gp_{C2}$ and 
%%%
\begin{equation}
\Gp_{W}(\auf x_0, x_1\zu, \auf y_0, y_1\zu) := \auf x_0y_0, y_1x_1\zu
\end{equation}
%%%
Notice that in this case there are {\it no\/} extra unary
polynomials. However, some of them can be produced by feeding
empty material. These are exactly the polynomials $\Gi$, $\Gp_F$ 
and $\Gp_H$. The others cannot be produced, since
the order of the component strings must always be respected.
For example, one has
%%
\begin{equation}
\Gp_{C2}(\auf x_0,x_1\zu,\auf \varepsilon, \varepsilon\zu) = \auf x_0x_1,
    \varepsilon\zu = \Gp_F(x_0, x_1)
\end{equation}
%%

We shall now turn to the description of the structures that 
correspond to the trees for CFGs. Recall the definitions of 
Section~\ref{kap1}.\ref{kap1-4}. 
%%
\begin{defn}
\label{defn:context}
%%%
\index{context!$n$--\faul}%%
%%%
A sequence $C = \auf \vec{w}_i : i < n+1\zu$ of strings is called
an $n$--\textbf{context}. A sequence $\Gv = \auf \vec{v}_i : i < n\zu$
\textbf{occurs in} $\vec{x}$ \textbf{in the context} $C$ if
%%
\begin{equation}
\vec{x} = \vec{w}_0\conc\prod_{i < n} \vec{v}_i\vec{w}_{i+1}
\end{equation}
%%
We write $C(\Gv)$ in place of $\vec{x}$.
%%
\end{defn}
%%
Notice that an $n$--sequence of strings can alternatively be
regarded as an $n-1$--context. Let $G$ be a $k$--linear
LMG. If $\vdash_G A(\vec{x}_0, \dotsc, \vec{x}_{k-1})$
then this means that the $k$--tuple $\auf \vec{x}_i : i < k\zu$
is a constituent of category $A$. If $\vdash_G S(\vec{x})$, then 
the derivation will consist in deriving statements of the form 
$A(\auf \vec{y}_i : i < \Xi(A)\zu)$ such that there is an 
$n+1$--context for $\auf \vec{y}_i : i < \Xi(A)\zu$ in 
$\vec{x}$.

The easiest kinds of structures are trees where each nonterminal 
node is assigned a tuple of subwords of the terminal string. Yet, 
we will not follow this approach as it generates stuctures that are 
too artificial. Ideally, we would like to have something analogous to 
constituent structures, where constituents are just appropriate 
subsets of the terminal nodes. 
%%%
\begin{defn}
An \textbf{labelled ordered tree of discontinuity degree} $k$ is a 
%%%
\index{discontinuity degree}%%%
%%%
labelled, ordered tree such that $[x]$ has at most $k$ discontinuous 
pieces.
\end{defn}
%%%
If $G$ is given, the labels are taken from $A$ and $R$, and $A$ is 
the set of labels of leaves, while $R$ is the set of labels for 
nonleaves. Also, if $Q$ is a $k$--ary predicate, it must somehow 
be assigned a unique $k$--tuple of subwords of the terminal string.
To this end, we segment $[x]$ into $k$ continuous parts. The way 
this is done exactly shall be apparent later. Now, if $x$ is 
assigned $B$ and if $[x]$ is the disjoint union of the continuous 
substrings $\vec{y}_i$, $i < k$, and if $\vec{y}_i$ precedes in 
$\vec{y}_j$ in $\vec{x}$ iff $i < j$ then $B(\auf \vec{y}_i : i < k\zu)$.

However, notice that the predicates apply to sequences of substrings. 
This is to say that the linear order is projected from the 
terminal string. Additionally, the division into substrings can be 
read off from the tree (though not from $[x]$ alone). There is, 
however, one snag. Suppose $G$ contains a rule of the form 
%%%
\begin{equation}
\label{rule:nonmon}
A(x_1x_0y_0, y_1) \hrn B(x_0, x_1), C(y_0, y_1)
\end{equation}
%%
Then assuming that we can derive $B(\vec{u}_0, \vec{u}_1)$ and 
$C(\vec{v}_0, \vec{v}_1)$, we can also derive 
$A(\vec{u}_1\, \vec{u}_0\, \vec{v}_0, \vec{v}_1)$. 
However, this means that $\vec{u}_1$ must precede $\vec{u}_0$ 
in the terminal string, which we have excluded. We can prevent 
this by introducing a nonterminal $B^{\ast}$ such that 
$B^{\ast}(x_0, x_1) \dpf B(x_1, x_0)$, and then rewrite the rule as
%%%
\begin{equation}
A(x_0x_1y_0, y_1) \hrn B^{\ast}(x_0, x_1), C(y_0, y_1).
\end{equation}
%%
The problem with the rule \eqref{rule:nonmon} is that it switches 
the order of the $x_i$'s. Rules that do this (or switch the order of 
the $y_i$'s) are called nonmonotone in the sense of the following 
definition.
%%
\begin{defn}
%%%
\index{rule!monotone}%%
\index{literal movement grammar!monotone}%%
%%%
Let $\rho = L \hrn M_0\dotsb M_{n-1}$ be a linear rule,
$L = B(\auf t^j : j < k\zu)$ and
$M_i = A_i(\auf x^j_i : j < k_i\zu)$, $i < n$. $\rho$ is called
\textbf{monotone} of for every $i < n$ and every pair
$j < j' < k_i$ the following holds: if $x_i^j$ occurs in 
$t^q$ and $x_i^{j'}$ in $t^{q'}$ then either $q < q'$ or 
$q = q'$ and $x_i^j$ occurs before $x_i^{j'}$ in the polynomial $t_q$.
An LMG is monotone if all of its rules are.
\end{defn}
%%
Now, for every LCFRS there exists a monotone LCFRS that generates 
the same strings (and modulo lexical rules also the same structures). 
For every predicate $A$ and every permutation $\pi : k \pf k$, 
$k := \Xi(A)$ assume a distinct predicate $A^{\pi}$ with the 
intended interpretation that $A(x_0, \dotsc, x_{k-1})$ iff 
$A^{\pi}(x_{\pi^{-1}(0)}, \dotsc, x_{\pi^{-1}(k-1)})$. Every 
rule $A(\vec{s}) \hrn B_0\dotsc B_{p-1}$ is replaced by all possible 
rules $A^{\pi}(\pi^{-1}(\vec{s})) \hrn B_0 \dotsc B_{p-1}$. 

Let $A \hrn B_0 \dotso B_{p-1}$ be a rule. 
Now put $\rho_i(j) := k$ iff $x^k_i$ is the $j$th variable 
from the variables $x_i^q$, $q < \Xi(B_i)$, which occurs in 
$\prod_{i < \Xi(A)} t_i$, counting from the left. Then 
$A' \hrn B_0^{\rho_0} \dotso B^{\rho_{p-1}}_{p-1}$ will replace 
the rule $A \hrn B_0 \dotso B_{p-1}$, where $A'$ results from 
$A$ by applying the substitution $x_i^j \mapsto x_i^{\rho^{-1}_i(j)}$ 
(while the variables of the $B_i^{\rho}$ remain in the original 
order). This rule is monotone. For example, assume that we have 
the rule
%%%
\begin{equation}
A(x_2y_1x_1, x_0y_2y_0y_3) \hrn B_0(x_0, x_1, x_2), 
C(y_0, y_1, y_2, y_3).
\end{equation}
%%%
Then we put $\rho_0 : 0 \mapsto 2, 1 \mapsto 1, 2 \mapsto 0$, 
and $\rho_1 : 0 \mapsto 2, 1 \mapsto 0, 2 \mapsto 1, 3 \mapsto 3$.
So we get 
%%%
\begin{equation}
A(x_0y_0x_1, x_2y_1y_2y_3) \hrn B^{\rho_0}(x_0, x_1, x_2), 
C^{\rho_1}(y_0, y_1, y_2, y_3).
\end{equation}
%%%
Every terminal rule is monotone. Thus, we can essentially throw 
out all nonmonotone rules. Thus, for nonmonotone LCFRS there is 
a monotone LFCRS generating the same strings. 

We can a derive useful theorem on LCFRSs.
%%%
\begin{defn}
%%%
\index{language!$k$--pumpable}%%
%%%
A language $L$ is called $k$--\textbf{pumpable} if there is a 
constant $p_L$ such that for all $\vec{x}$ of length $\geq p_L$ 
there is a decomposition 
$\vec{x} = \vec{u}_0\prod_{i < k}({\vec{v}_i}{\vec{u}_{i+1}})$
such that 
%%
\begin{equation}
\{\vec{u}_0\prod_{i < k}({\vec{v}_i\,}^n{\vec{u}_i}) : n \in \omega\} 
\subseteq L
\end{equation}
\end{defn}
%%%
\begin{thm}[Groenink]
%%%
\label{thm:2kpump}
\index{Groenink, Annius}%%%
%%%
Suppose that $L$ is a $k$--LCFRL. Then $L$ is $2k$--pumpable.
\end{thm}
%%%
We provide a proof sketch based on derivations. Transform the language 
into a language of signs, by adding the trivial semantics. Then 
let $\GA$ be a sign grammar based on a monotone $k$--LCFRS for it.
Observe that if $\Gt$ is a structure term containing $x$ free 
exactly once, and if $\Gt(\Gu)$ and $\Gu$ are definite 
and unfold to signs of identical category, then with $\Gs(\Gt(\Gu))$ 
also $\Gs(\Gt^n(\Gu))$ is definite for every $n$ (the rule skeleton 
is context free). Now, if $\vec{x}$ is large enough, its structure 
term will be of the form $\Gs(\Gt(\Gu))$ such that $\Gt(\Gu)$ 
and $\Gu$ have the same category. Finally, suppose that the 
grammar is monotone. Then $\Gt^{\varepsilon}$ is a monotone, 
linear polynomial function on $k$--tuples; hence there are 
$\vec{v}_i$, $i < 2k$, such that 
%%%
\begin{equation}
p(x_0, \dotsc, x_{k-1}) = \auf \vec{v}_{2i}x_i\vec{v}_{2i+1} : 
i < k\zu
\end{equation}
%%%

Thus let us now focus on monotone LCFRSs. We shall define the 
structures that are correspond to derivations in monotone LCFRSs, 
and then show how they can be generated using graph grammars.
%%%
\begin{defn}
Suppose that $\GT = \auf T, <, \sqsubset, \ell\zu$ is an ordered labelled 
tree with degree of discontinuity $k$ and $G = \auf A, R, \Xi, S, H\zu$ 
a monotone LCFRS. We say that $\GT$ is a $G$--\textbf{tree} if 
%%%
\begin{dingautolist}{192}
\item $\ell(v) \in A$ iff $v$ is a leaf, 
\item for every nonleaf $v$: there is a set $\{H(v)_i : i < k\}$ 
	of leaves such that $k = \Xi(\ell(v))$, and 
	$H(v)_i \subseteq [x]$ is continuous, 
\item if $v$ has daughters $w_i$, $i < n$, then there is a rule
	$$A(t_0, \ldots, t_{\Xi(A)}) \hrn 
	B_0(x_0^0, \dotsc, x_0^{\Xi(B_0)}), \\ \dotsc, 
	B_{n-1}(x_{n-1}^0, \dotsc, x_{n-1}^{\Xi(B_{n-1})})$$
	such that 
	\begin{itemize}
	\item
	$v$ has label $A$, $w_i$ has label $B_i$ ($i < n$), 
	\item	
	if $t_j = \prod_{i < \nu} x_{g(i)}^{h(i)}$ then $H(v)_j = 
	\bigcup_{i < \nu} H(w_{g(i)})_{h(i)}$.
	\end{itemize}
\end{dingautolist}
\end{defn}
%%
The last clause is somewhat convoluted. It says that whatever 
composition we assume of the leaves associated with $v$, it must 
be compatible with the composition of the $w_i$, and the way the 
polynomials are defined. Since the preterminals are unary, it can 
be shown that $H(v)_i$ is unique for all $v$ and $i$.

The following grammar is called $G^{\heartsuit}$.
%%
\begin{equation}
\begin{array}{llll}
\multicolumn{4}{l}{\mbox{\tt S}(y_0x_0y_1,z_0x_1z_1) \hrn
    \mbox{\tt S}(x_0,x_1),\mbox{\tt X}(y_0,y_1),
    \mbox{\tt Y}(z_0, z_1).} \\
\mbox{\tt S}(\varepsilon, \varepsilon) \hrn . & & & \\
\multicolumn{2}{l}{\mbox{\tt X}(x,y) \hrn \mbox{\tt A}(x), \mbox{\tt B}(y).}
&  
\multicolumn{2}{l}{\mbox{\tt Y}(x,y) \hrn \mbox{\tt C}(x), \mbox{\tt D}(y).} \\
\mbox{\tt A}(\mbox{\tt a}) \hrn  . & & 
\mbox{\tt C}(\mbox{\tt c}) \hrn  . &  \\
\mbox{\tt B}(\mbox{\tt b}) \hrn  . & & 
\mbox{\tt D}(\mbox{\tt d}) \hrn  . & 
\end{array}
\end{equation}
%%
$G^{\heartsuit}$ derives {\mtt aabbccdd} with the structure shown 
in Figure~\ref{fig:disconti}.
%%
\begin{figure}
\begin{center}
\begin{picture}(23,20)
\put(1,1){\makebox(0,0){\tt a}}
\put(4,1){\makebox(0,0){\tt a}}
\put(7,1){\makebox(0,0){\tt b}}
\put(10,1){\makebox(0,0){\tt b}}
\put(13,1){\makebox(0,0){\tt c}}
\put(16,1){\makebox(0,0){\tt c}}
\put(19,1){\makebox(0,0){\tt d}}
\put(22,1){\makebox(0,0){\tt d}}
%%
\multiput(1,1.5)(3,0){8}{\line(0,1){3}}
%%
\put(1,5){\makebox(0,0){\tt A}}
\put(4,5){\makebox(0,0){\tt A}}
\put(7,5){\makebox(0,0){\tt B}}
\put(10,5){\makebox(0,0){\tt B}}
\put(13,5){\makebox(0,0){\tt C}}
\put(16,5){\makebox(0,0){\tt C}}
\put(19,5){\makebox(0,0){\tt D}}
\put(22,5){\makebox(0,0){\tt D}}
%%
\put(1,5.5){\line(0,1){3}}
\put(4,5.5){\line(1,2){1.5}}
\put(7,5.5){\line(-1,2){1.5}}
\put(10,5.5){\line(-3,1){9}}
\put(13,5.5){\line(3,1){9}}
\put(16,5.5){\line(1,2){1.5}}
\put(19,5.5){\line(-1,2){1.5}}
\put(22,5.5){\line(0,1){3}}
%%
\put(1,9){\makebox(0,0){\tt X}}
\put(5.5,9){\makebox(0,0){\tt X}}
\put(17.5,9){\makebox(0,0){\tt Y}}
\put(22,9){\makebox(0,0){\tt Y}}
%%
\put(5.5,9.5){\line(2,1){6}}
\put(17.5,9.5){\line(-2,1){6}}
\put(11.5,13){\makebox(0,0){\tt S}}
%%
\put(1,9.5){\line(3,2){10.5}}
\put(22,9.5){\line(-3,2){10.5}}
\put(11.5,13.5){\line(0,1){3}}
\put(11.5,17){\makebox(0,0){\tt S}}
\end{picture}
\end{center}
\caption{A Structure Tree for $G^{\heartsuit}$}
\label{fig:disconti}
\end{figure}
%%
This tree is not exhaustively ordered. This is the main difference
with CFGs. Notice that the tree does not reflect
the position of the empty constituent. Its segments are found between
the second and the third as well as between the sixth and the
seventh letter. One can define the structure tree also in this way
that it explicitly contains the empty strings. To this end one has
to replace also occurrences of $\varepsilon$ by variables. The rest
is then analogous.

We shall now define a context free graph grammar that generates the 
same structures as a given monotone LCFRS. For the sake of 
simplicity we assume that all predicates are $k$--ary, and that all 
terminal rules are of the form $Y(a, \varepsilon, \dotsc, \varepsilon)$, 
$a \in A$. Monotonicity is not necessary to assume, but makes life 
easier. The only problem 
that discontinuity poses is that constituents cannot be ordered 
with respect to each other in a simple way. The way they are related 
to each other shall be described by means of special matrices.  

Assume that $\vec{u}$ is a string, $C = \auf \vec{v}_i : i < k+1\zu$ 
a $k+1$--context for $\vec{x}_i$, $i < k$, in $\vec{u}$ and 
$D = \auf \vec{w}_i : i < k+1\zu$ a $k+1$--context for $\vec{y}_j$, 
$j < k$, in $\vec{u}$. Now, write $M(C,D) := (\mu_{ij})_{ij}$ for the 
following matrix: 
%%
\begin{equation}
\mu_{pq} = 1 \text{ iff } \prod_{i < p} \vec{v}_i\vec{x}_i 
\text{ is a prefix of } \vec{w}_0\conc \prod_{i < q} \vec{y}_i\vec{w}_{i+1} 
\end{equation}
%%%
We call $M(C,D)$ the \textbf{order scheme of} $C$ and $D$. 
Intuitively, the order scheme tells us which of the $\vec{x}_p$ 
precede which of the $\vec{y}_q$ in $\vec{u}$.  We say that $C$ and 
$D$ \textbf{overlap} if there are $i, j < k$ such that the (occurrences 
of) $\vec{x}_p$ and $\vec{y}_q$ overlap. This is the case iff 
$\mu_{pq} = \mu_{qp} = 0$.
%%%
\index{overlap}%%
%%%
\begin{lem}
Assume that $\GB$ is a labelled ordered tree for a monotone LCFRS 
with yield $\vec{u}$. Further, let $x$ and $y$ be nodes. Then $\ell(x)$ 
and $\ell(y)$ overlap iff $x$ and $y$ are comparable. 
\end{lem}
%%%
Notice that in general $\mu_{qp} = 1 - \mu_{pq}$ in the nonoverlapping 
case, which is what we shall assume from now on. We illustrate this with 
an example. Figure~\ref{fig:order} shows some 2--schemes for monotone 
rules together with the orders which define them. (We omit the vector 
arrows; the ordering is defined by `is to the left of in the string'.) 
%%
\begin{figure}
$$\begin{array}{ccc}
\left(\begin{array}{ll}
1 & 1 \\
1 & 1
\end{array}\right) &
\left(\begin{array}{ll}
1 & 1 \\
0 & 1
\end{array}\right) &
\left(\begin{array}{ll}
1 & 1 \\
0 & 0
\end{array}\right) \\
x_0x_1y_0y_1 &
x_0y_0x_1y_1 &
x_0y_0y_1x_1 \\
\\
%%
\left(\begin{array}{ll}
0 & 1 \\
0 & 1
\end{array}\right) &
\left(\begin{array}{ll}
0 & 1 \\
0 & 0
\end{array}\right) &
\left(\begin{array}{ll}
0 & 0 \\
0 & 0
\end{array}\right) \\
y_0x_0x_1y_1 &
y_0x_0y_1x_1 &
y_0y_1x_0x_1
\end{array}$$
\caption{Order Schemes}
\label{fig:order}
\end{figure}
%%
For every $k$--scheme $M$ let $\xi_M$ be a relation.
Now we define our graph grammar. $N^{\circ} := \{A^{\circ} 
: A \in N\}$ is a set of fresh nonterminals. The set of vertex 
colours $F_V := N \cup N^{\circ} \cup A$, the set of terminal vertex
colours is $F_V^T := N  \cup A$, the set of edge
colours is $\{<\} \cup \{\xi_M : M \mbox{ a $k$--scheme}\}$.
(As we will see immediately, $\sqsubset$ is the relation which
fits to the relation $\xi_{\BI}$ where $\BI = (1)_{ij}$ is
the matrix which consists only of 1s.) The start graph is the
one--element graph $\GS$ which has one edge. The vertex has 
colour $S$, the edge has only one colour, $\xi_{K}$, where 
$K = (\kappa_{pq})_{pq}$ with $\kappa_{pq} = 1$ iff $p < q$. 
For every rule $\rho$ we add a graph replacement rule. Let
%%
\begin{equation}
\rho = B(\auf t^j : j < k\zu) \hrn A_0(\auf x_0^j : j < k\zu) 
\dotsb A_{n-1}(\auf x_{n-1}^j : j < k \zu)
\end{equation}
%%
be given. Let $p := \prod_{i < k} t^i$ be the characteristic
polynomial of the rule. Then the graph replacement $\rho^{\gamma}$
replaces the node $B^{\circ}$ by the following graph.
%%
\begin{equation}
\begin{array}{l}
\begin{picture}(12,10)
\put(2,3){\makebox(0,0){$\bullet$}}
\put(2,2){\makebox(0,0){$v_0$}}
\put(2,.5){\makebox(0,0){$A_0^{\circ}$}}
\put(2,3){\line(2,1){6}}
\put(5,3){\makebox(0,0){$\bullet$}}
\put(5,2){\makebox(0,0){$v_1$}}
\put(5,.5){\makebox(0,0){$A_1^{\circ}$}}
\put(5,3){\line(1,1){3}}
\put(8,3){\makebox(0,0){$\dotsb$}}
\put(11,3){\makebox(0,0){$\bullet$}}
\put(11,2){\makebox(0,0){$x_{n-1}$}}
\put(11,.5){\makebox(0,0){$A_{n-1}^{\circ}$}}
\put(11,3){\line(-1,1){3}}
\put(8,6){\makebox(0,0){$\bullet$}}
\put(8,7){\makebox(0,0){$w$}}
\put(8,8.5){\makebox(0,0){$B$}}
\end{picture}
\end{array}
\end{equation}
%%
Furthermore, between the nodes $v_i$ and $w_j$, $i \neq j$, the 
following relations hold (which are not shown in the picture).
Put $\mu(i,j)_{i'j'} := 1$ if $x_i^{i'}$ is to the left of
$x_j^{j'}$ in $p$. Otherwise, put $\mu(i,j)_{i'j'} := 0$. 
Then put $H_{ij} := (\mu(i,j)_{i'j'})_{i'j'}$. The relation 
from $v_i$ to $v_j$ is $\xi_{H_{ij}}$. Notice that by definition 
always either $x_i^{i'}$ is to the left of $x_j^{j'}$ or to the 
right of it. Hence the relation between $v_j$ and $v_i$ is exactly 
$\xi_{1 - H}$. This is relevant insofar as it allows us to 
concentrate on one colour functional only: $\GI\GI$. Now supppose 
that $w$ is in relation $\xi_M$ to a node $u$. (So, there is an 
edge of colour $\xi_M$ from $v$ to $u$.) We need to determine the 
relation (there is only one) from $v_i$ to $u$. This is $\xi_N$, 
where $N = (\nu_{pq})_{pq}$ and $\nu_{pq} = 1$ iff 
$\mu_{p'q} = 1$, where $x_i^p$ occurs in $t^{p'}$. 
The map that sends $M$ to $N$ and $<$ to $<$ is the desired 
colour functional $\GI\GI$. The other functionals can be 
straightforwardly defined.

There is a possibility of defining structure in some restricted cases,
namely always when the right hand sides do not contain a variable
twice. This differs from linear grammars in that variables are
still allowed to occur several times on the left, but only once
on the right. An example is the grammar
%%
\begin{equation}
\mbox{\tt S}(xx) \hrn \mbox{\tt S}(x).; \quad
    \mbox{\tt S}(\mbox{\tt a}) \hrn .
\end{equation}
%%
The notion of structure that has been defined above can be transferred 
to this grammar. We simply do as if the first rule was of this form
%%
\begin{equation}
\mbox{\tt S}(xy) \hrn \mbox{\tt S}(x), \mbox{\tt S}(y).
\end{equation}
%%
where it is clear that $x$ and $y$ always represent the same string.
In this way we get the structure tree for {\tt aaaaaaaa} shown 
in Figure~\ref{fig:exp}.
%%
\begin{figure}
\begin{center}
\begin{picture}(23,20)
\put(1,1){\makebox(0,0){\tt a}}
\put(4,1){\makebox(0,0){\tt a}}
\put(7,1){\makebox(0,0){\tt a}}
\put(10,1){\makebox(0,0){\tt a}}
\put(13,1){\makebox(0,0){\tt a}}
\put(16,1){\makebox(0,0){\tt a}}
\put(19,1){\makebox(0,0){\tt a}}
\put(22,1){\makebox(0,0){\tt a}}
%%
\multiput(1,1.5)(3,0){8}{\line(0,1){3}}
%%
\put(1,5){\makebox(0,0){\tt S}}
\put(4,5){\makebox(0,0){\tt S}}
\put(7,5){\makebox(0,0){\tt S}}
\put(10,5){\makebox(0,0){\tt S}}
\put(13,5){\makebox(0,0){\tt S}}
\put(16,5){\makebox(0,0){\tt S}}
\put(19,5){\makebox(0,0){\tt S}}
\put(22,5){\makebox(0,0){\tt S}}
%%
\put(1,5.5){\line(1,2){1.5}}
\put(4,5.5){\line(-1,2){1.5}}
\put(7,5.5){\line(1,2){1.5}}
\put(10,5.5){\line(-1,2){1.5}}
\put(13,5.5){\line(1,2){1.5}}
\put(16,5.5){\line(-1,2){1.5}}
\put(19,5.5){\line(1,2){1.5}}
\put(22,5.5){\line(-1,2){1.5}}
%%
\put(2.5,9){\makebox(0,0){\tt S}}
\put(8.5,9){\makebox(0,0){\tt S}}
\put(14.5,9){\makebox(0,0){\tt S}}
\put(20.5,9){\makebox(0,0){\tt S}}
%%
\put(2.5,9.5){\line(1,1){3}}
\put(8.5,9.5){\line(-1,1){3}}
\put(14.5,9.5){\line(1,1){3}}
\put(20.5,9.5){\line(-1,1){3}}
%%
\put(5.5,13){\makebox(0,0){\tt S}}
\put(17.5,13){\makebox(0,0){\tt S}}
%%
\put(5.5,13.5){\line(2,1){6}}
\put(17.5,13.5){\line(-2,1){6}}
\put(11.5,17){\makebox(0,0){\tt S}}
%%
\end{picture}
\end{center}
\caption{An Exponentially Growing Tree}
\label{fig:exp}
\end{figure}
%%

{\it Notes on this section.} In \cite{seki}, a slightly more 
general type of grammars than the LCFRSs is considered, which are 
called {\it Multiple Context Free Grammars} (MCFGs). In our
terminology, MCFGs maybe additionally upward deleting. In \cite{seki} 
weak equivalence between MCFGs and LCFRs is shown. See also 
%%%
\index{Seki, Hiroyuki}%%%
\index{Matsumura, Takashi}%%%
\index{Fujii, Mamoru}%%%
\index{Kasami, Tadao}%%
%%%%
\cite{kasami:mcfg}. \cite{michaelis:minimalism} 
%%%
\index{Michaelis, Jens}%%%
%%%
also defines monotone rules for MCFGs and shows that any MCFL can be  
generated by a monotone MCFG. For the relevance of these 
grammars in parsing see \cite{villemonte:parsing,villemonte:mcfg}. 
%%%
\index{Villemonte de la Clergerie, Eric}%%%
%%%
In his paper \shortcite{stabler:minimalism},
%%%
\index{Stabler, Edward P.}%%
%%%
Edward Stabler describes a formalisation of minimalist grammars akin
to Noam Chomsky's 
%%%
\index{Chomsky, Noam}%%%
%%%
Minimalist Program (outlined in \cite{chomsky:minimalist}). Subsequently, 
in \cite{michaelis:minimalism,michaelis:lacl98,michaelis:lacl01} and
\cite{harkema:minimalism} 
%%%
\index{Harkema, Henk}%%%
%%%
it is shown that the languages
generated by this formalism are exactly those that can be generated
by simple LMGs, or, for that matter, by LCFRSs.
%%
\vplatz
\exercise
Show that the derivation $\Gamma'$ is determined by $\Gamma$ up
to renaming of variables.
%%
\vplatz
\exercise
Prove Proposition~\ref{prop:linear-semilinear}.
%%%
\vplatz
\exercise
Let $A_k := \{\mbox{\mtt a}_i : i < k\}$, and let 
$W_k := \{\prod_{i < k}\mbox{\mtt a}_i^n : n \in \omega\}$. 
Show that $W_k$ is a $m$--LCFRL iff $k \leq 2m$.
%%
\vplatz
\exercise
Determine the graph grammar $\gamma G^{\heartsuit}$.
%%
\vplatz
\exercise
Show the following. {\it Let $N = \{x_i : i < k\} \cup \{y_i : i < k\}$
and $<$ a linear ordering on $N$ with $x_i  < x_j$ as well as $y_i < y_j$
for all $i < j < k$. Then if $m_{ij} = 1$ iff
$x_i < y_j$ then $M = (m_{ij})_{ij}$ is a $k$--scheme.
Conversely: let $M$ be a $k$--scheme and $<$ defined by
(1) $x_i < x_j$ iff $i < j$, (2) $y_i < y_j$ iff
$i< j$, (3) $x_i < y_j$ iff $m_{ij} = 1$. Then
$<$ is a linear ordering. The correspondence between orderings
and schemes is biunique.}
%%
\vplatz
\exercise
Show the following: {\it If in a linear $k$--LMG all structure trees
are exhaustively ordered, the generated tree set is context free.}
%%

 \section{Adjunction Grammars}
\label{kap4-5}
%
%
%
In this and the next section we shall concern ourselves with some
alternative types of grammars which are all (more or less) equivalent
to head grammars. These are the tree adjoining grammars (TAGs), CCGs
(which are some refined version of the adjunction grammars of
Section~\ref{kap1}.\ref{kap1-4} and the grammars $\CCG(\mathbf{Q})$
of Section~\ref{kap3}.\ref{kap3-2}, respectively) and the so--called
linear index grammars.

Let us return to the concept of tree adjoining grammars.
These are pairs $G = \auf \BC, N, A, \BA\zu$,
where $\BC$ is the set of centre trees and
$\BA$ a set of adjunction trees. In an adjunction
tree a node is called \textbf{central}
%%%
\index{node!central}%%
%%%
if it is above the distinguished leaf or identical to it.

It is advantageous to define a naming scheme for nodes in 
an adjunction tree. Let $\GT = \auf T, <_T, \sqsubset_T, \ell_T\zu$ 
be a centre tree. Then put $N(T) := T$. If adjoining 
$\GA = \auf A, <_A, \sqsubset_A, \ell_A\zu$ at $x$ to $\GT$
yields $\GU = \auf U, <_U, \sqsubset_U, \ell_U\zu$ 
then
%%%
\begin{equation}
\label{eq:55rec}
N(\GU) := (N(T) - \{x\}) \cup \{x \conc \GA \conc v : v \in A\}
\end{equation}
%%%
Let $H$ be the set of all nodes of centre or adjunction trees. 
Then $N(\GT) \subseteq  H \cdot (\GA \cdot H)^{\ast}$. Furthermore, 
as is inductively verified, $N(\GT)$ is prefix free. Thus, as $x$ 
gets replaced by strings of which $x$ is a suffix, no name that 
is added equals any name in $N(\GU)$. So, the naming scheme is 
unique. Moreover, it turns out that the structure of $\GT$ is 
uniquely determined by $N(\GT)$. The map that identifies $x \in T$ 
with its name is called $\nu$. Put $N := N(\GT)$ and
%%
\begin{equation}
%%%
\index{$\GN(\GT)$}%%%
%%%
\GN(\GT) := \auf N, <^N, \sqsubset^N, \ell^N\zu 
\end{equation}
%%
If $\vec{\sigma} = \vec{\gamma}\conc\GA \conc j$, let 
$\ell^N(\vec{\sigma}) := X$ for the unique $X$ which is the label 
of the node $j$ in $\GA$. Second, put $\vec{\sigma} \sqsubset_N 
\vec{\tau}$ if $\vec{\sigma} = \vec{\gamma}\conc j \conc \GA\conc %
\vec{\eta}$ and $\vec{\tau} = \vec{\gamma}\conc j'\conc \GA \conc %
\vec{\theta}$ for certain $\vec{\gamma}$, $\vec{\eta}$, $\vec{\theta}$, 
$\GA$ and $j \neq j'$ such that $j \sqsubset j'$. Third, 
$\vec{\sigma} < \vec{\tau}$ if $\vec{\sigma} = \vec{\gamma}\conc j %
\conc\GA\conc\vec{\eta}$ and 
$\vec{\tau} = \vec{\gamma}\conc j' \conc \GA\conc \vec{\theta}$ 
for certain $\vec{\gamma}$, $\vec{\eta}$, $\vec{\theta}$, $\GA$ 
and $j \neq j'$, such that (a) $j < j'$, (b) $j'$ and every node 
of $\vec{\theta}$ is central in its corresponding adjunction tree.
%%%
\begin{prop}
$\nu$ is an isomorphism from $\GT$ onto $\GN(\GT)$.
\end{prop}
%%%
Thus, we basically only need to define $N(\GT)$. It turns out 
that the sets $N(\GT)$ are the sets of leaves of a CFG.
For the sake of simplicity we assume that the set of nodes of
$\GB$ is the set of numbers $j(\GB) = \{0, 1, \dotsc, j(\GB)-1\}$.
$j_{\ast} := \max \{j(\GB) : \GB \in \BA \cup \BC\}$. The terminals 
are the numbers $< j_{\ast}$. The nonterminals are pairs $(i,\GB)$ 
where $\GB$ is a tree and $i < j_{\ast}$. The start symbols are 
$(0,\GC)$, $\GC \in \BC$. First, we shall define the 
grammar$^{\ast}$ $D(G)$. 
%%%
\index{$D(G)$}%%
%%%
The rules are of this form.
%%
\begin{align}
\label{eq:rules} (j,\GA) & \pf X_0 \quad X_1 \dotsb X_{j(\GA)-1} 
\end{align}
%%
where \eqref{eq:rules} is to be seen as rule a scheme: for every $\GA$
and every admissible $j$ we may choose whether $X_i$ 
is $i$ or $(i,\GB_i)$ ($i < j(\GA)$) for some tree 
$\GB_i$ which can be adjoined at $i$ in $\GA$. This grammar$^{\ast}$
%%%
\index{derivation grammar}%%
\index{grammar!derivation}%%
%%%
we denote by $D(G)$ and call it the \textbf{derivation grammar}.
%%%
\index{derivation}%%
%%%
A \textbf{derivation} for $G$ is simply a tree generated by 
$D(G)$. The following is clear: trees from $D(G)$ are in one--to--one
correspondence with their sets of leaves, which in turn 
define tree of the adjunction grammar. It should be said that 
the correspondence is not always biunique. (This is so since 
any given tree may have different derivations, and these 
get matched with nonisomorphic trees in $D(G)$. However, 
each derivation tree maps to exactly one tree of $G$ modulo 
isomorphism.)

TAGs differ from the unregulated tree adjunction grammars in 
that they allow to specify
%%
\begin{dingautolist}{192}
\item whether adjunction at a certain node is licit,
\item which trees may be adjoined at which node, and
\item whether adjunction is obligatory at certain nodes.
\end{dingautolist}
%%
We shall show that \ding{192} increases the generative power
but \ding{193} and \ding{194} do not in presence of \ding{192}.
To establish control over derivations, we shall have to change 
our definitions a little bit. We begin with \ding{192}.
To control for the possiblity of adjunction, we assume that 
the category symbols are now of the form $a$, $a \in A$, or $X$ 
and $X^{\triangledown}$ 
%%%
\index{$X^{\triangledown}$}%%%
%%%
respectively, where $X \in N$. Centre and adjunction trees are 
defined as before. Adjunction is also
defined as before. There is a leaf $i$ which has the same 
label as the root (all other leaves carry terminal labels). However, 
no adjunction is licit at nodes with label $X^{\triangledown}$. 
Notice that root and distinguished leaf must carry the same nonterminal, 
it is not admitted that one carries $X$ while the other has 
$X^{\triangledown}$. Even if we admitted that, this would not increase 
the generative capacity. Using such grammars one can generate 
the language $\{\mbox{\tt a}^n \mbox{\tt b}^n\mbox{\tt c}^n\mbox{\tt d}^n :
n \in \omega\}$. Figure~\ref{fig:abcdn} shows such a grammar. The 
centre tree is shown to the left, the adjunction tree to the right.
%%
\begin{figure}
\begin{center}
\begin{picture}(10,17)
\put(5,6){\makebox(0,0){$\varepsilon$}}
\put(5,7){\makebox(0,0){$\bullet$}}
\put(5,10){\makebox(0,0){$\bullet$}}
\put(5,7){\line(0,1){3}}
\put(5,11){\makebox(0,0){\tt S}}
\end{picture}
%%%
\qquad
%%%
\begin{picture}(10,17)
\put(5,2){\line(0,1){12}}
\put(5,5){\line(-1,-1){3}}
\put(5,5){\line(1,-1){3}}
\put(5,11){\line(-1,-1){3}}
\put(5,11){\line(1,-1){3}}
\put(5,2){\makebox(0,0){$\bullet$}}
\put(5,5){\makebox(0,0){$\bullet$}}
\put(5,8){\makebox(0,0){$\bullet$}}
\put(5,11){\makebox(0,0){$\bullet$}}
\put(5,14){\makebox(0,0){$\bullet$}}
\put(2,2){\makebox(0,0){$\bullet$}}
\put(2,8){\makebox(0,0){$\bullet$}}
\put(8,2){\makebox(0,0){$\bullet$}}
\put(8,8){\makebox(0,0){$\bullet$}}
\put(5,1){\makebox(0,0){$\mbox{\tt S}^{\triangledown}$}}
\put(5,15){\makebox(0,0){$\mbox{\tt S}^{\triangledown}$}}
\put(2,1){\makebox(0,0){\tt b}}
\put(2,7){\makebox(0,0){\tt a}}
\put(8,1){\makebox(0,0){\tt c}}
\put(8,7){\makebox(0,0){\tt d}}
\put(4,11){\makebox(0,0){\tt T}}
\put(4,8){\makebox(0,0){\tt S}}
\put(4,5){\makebox(0,0){\tt T}}
\end{picture}
\end{center}
\caption{A Tree Adjoining Grammar for $\{\mbox{\tt a}^n\mbox{\tt b}^n%
\mbox{\tt c}^n\mbox{\tt d}^n : n \in \omega\}$}
\label{fig:abcdn}
\end{figure}
%%
It is not hard to show that one can generally reduce such grammars 
to those where both the root and the leaf carry labels of the form
$X^{\triangledown}$. Namely, if the root does not carry an adjunction 
prohibition, but, say, a label $X \in N$, then add a new root which has 
label $X^{\triangledown}$, and similarly for the distinguished leaf.
Also, notice that adjunction prohibition for interior nodes of 
adjunction trees can be implemented by changing their label to a 
newly added nonterminal. Trivially, no tree can be adjoined there.
%%
\begin{defn}
%%%
\index{tree adjoining grammar!standard}%%
\index{TAG}%%
%%%
A \textbf{standard tree adjoining grammar} (or simply a \textbf{TAG}) 
is an adjunction grammar in which the adjunction trees carry an
adjunction prohibition at the root and the distinguished leaf.
\end{defn}
%%
Now let us turn to \ding{193} and \ding{194}. It is possible to 
specify in standard TAGs whether adjunction is obligatory and 
which trees may be adjoined. So, we also have a function $f$ which 
maps all nodes with nonterminal labels to sets of adjunction trees. 
(If for some $i$ $f(i) = \varnothing$ then that node effectively 
has an adjunction prohibition.) We can simulate this as follows.
Let $\BA$ be the set of adjunction trees. We think of the
nonterminals as labels of the form $\auf X, \GT\zu$
and $\auf X, \GT\zu^{\triangledown}$, respectively, where $X \in N$
and $\GT \in \BA$. A (centre or adjunction) tree $\GT$ is replaced 
by all trees $\GT'$ on the same set of nodes, where $i$ carries
the label $\auf X,\GU\zu$ if $i$ had label $X$ in $\GT$ if
$\GU \in f(i)$, and $\auf X,\GU\zu^{\triangledown}$ if
$i$ has the label $X^{\triangledown}$ in $\GT$. However, if $i$ 
is the root, it will only get the label $\auf i, \GT\zu^{\triangledown}$. 
The second element says nothing but which tree is going to be 
adjoined next. This eliminates the second point from the list, 
as we can reduce the grammars by keeping the tree structure.

Now let us turn to the last point, the obligation for adjunction.
We can implement this by introducing labels of the form
$X^{\bullet}$.  (Since obligation and prohibition to adjoin
are exclusive, $\bullet$ occurs only when $\triangledown$ does not.)
A tree is complete only if there are no nodes with label
$X^{\bullet}$ for any $X$. Now we shall show that for every adjunction
grammar of this kind there exists a grammar generating the same
set of trees where there is no obligation for adjunction.
We adjoin to a centre tree as often as necessary to
eliminate the obligation. The same we do for adjunction trees.
The resulting trees shall be our new centre and adjunction
trees. Obviously, such trees exist (otherwise we may choose
the set of centre trees to be empty). Now we have to show that
there exists a finite set of minimal trees. Look at a tree
without adjunction obligation and take a node. This node
has a history. It has been obtained by successive adjunction.
If this sequence contains an adjunction tree twice, we may
cut the cycle. (The details of this operation are left to the
reader.) This grammar still generates the same trees. So,
we may remain with the standard form of TAGs.

Now we shall first prove that adjunction grammars cannot generate
more languages as linear 2--LMGs. From this it immediately follows
that they can be parsed in polynomial time. The following is from 
\shortcite{shanker-weir-joshi:coling86}, who incidentally show the 
converse of that theorem as well: head grammars are weakly 
equivalent to TAGs.
%%
\begin{thm}[Vijay--Shanker \& Weir \& Joshi]
%%%
\index{Vijay--Shanker, K.}%%%
\index{Weir, David}%%%
\index{Joshi, Aravind}%%%
%%%
For every TAG $G$ there exists a head grammar $K$ such that $L(K) = L(G)$.
\end{thm}
%%
\proofbeg 
Let $G$ be given. We assume that the trees have pairwise
disjoint sets of nodes. We may also assume that the trees are at
most binary branching. (We only need to show weak equivalence.)
Furthermore, we can assume that the nodes are strictly branching
if not preterminal. The set of all nodes is denoted by $M$. The
alphabet of nonterminals is $N' := \{i^a : i \in M\} \cup \{i^n :
i \in M\}$. The start symbol is the set of all $i^a$ and $i^n$
where $i$ is the root of a centre tree. By massaging the grammar
somewhat one can achieve that the grammar contains only one start
symbol. Now we shall define the rules. For a local tree we put
%%
\begin{equation}
\label{eq:rulet}
i(a,\varepsilon) \hrn  .
\end{equation}
%%
if $i$ is a leaf with terminal symbol $a$. If $i$ is a
distinguished leaf of an adjunction tree we also take
the rule
%%
\begin{equation}
\label{eq:rulee}
i^n(\varepsilon,\varepsilon) \hrn .
\end{equation}
%%
Now let $i \pf j\quad k$ be a branching local tree.
Then we add the following rules.
%%
\begin{equation}
\label{eq:rulep}
i^a(x_0x_1, y_0y_1) \hrn j^n(x_0,x_1), k^n(y_0,y_1).
\end{equation}
%%
Further, if $i$  is a node to which a tree with root $j$ can 
be adjoined, then also this is a rule.
%%
\begin{equation}
\label{eq:rulef}
i^n(x_0y_0, y_1x_1) \hrn j^n(x_0,x_1)\quad i^a(y_0,y_1). 
\end{equation}
%%
If adjunction is not necessary or prohibited at $i$, then
finally the following rule is added.
%%
\begin{equation}
\label{eq:rulen}
i^n(x_0, x_1) \hrn i^a(x_0,x_1).
\end{equation}
%%
This ends the definition of $K$. In view of the rules
\eqref{eq:rulep} it is not entirely clear that we are dealing with a
head grammar. So, replace the rules \eqref{eq:rulep} by the following
rules:
%%
\begin{align}
i^a(x_0, x_1y_0y_1) & \hrn j^{n\bullet}(x_0,x_1), k^{n\bullet}(y_0,y_1).
\\
j^{n\bullet}(x_0x_1y_0,y_1) & \hrn j^n(x_0,x_1), \mbox{\tt L}(y_0,y_1).
\\
k^{n\bullet}(x_0, x_1y_0y_1) & \hrn \mbox{\tt L}(x_0, x_1),
    k^n(y_0,y_1). \\
\mbox{\tt L}(\varepsilon, \varepsilon) & \hrn .
\end{align}
%%
These are rules of a head grammar; \eqref{eq:rulep} can be derived 
from them. For this reason we remain with the rules \eqref{eq:rulep}.

It remains to show that $L(K) = L(G)$. First the inclusion
$L(G) \subseteq L(K)$. We show the following. Let $\GT$ be a local
tree which contains exactly one distinguished leaf and
nonterminal leaves $x_i$, $i < n$, with labels $k_i$. Let therefore
$j < i$ be distinguished. We associate with $\GT$ a vector
polynomial $\Gp(\GT)$ which returns
%%
\begin{equation}
\auf \prod_{i < j} \vec{y}_i\vec{z}_i,
\prod_{j < i < n} \vec{y}_i\vec{z}_i\zu
\end{equation}
%%
for given pairs of strings $\auf \vec{y}_i, \vec{z}_i\zu$.
It is possible to show by induction over $\GT$ that there is a
$K$--derivation
%%
\begin{multline}
i^n(\Gp(\GT)(\auf \auf \vec{y}_i, \vec{z}_i\zu : i < n\zu))
\hrn^{\ast} k_0^n(\auf \vec{y}_0, \vec{z}_0\zu), \\
\dotsc, 
k_{n-1}^n(\auf \vec{y}_{n-1}, \vec{z}_{n-1}\zu).
\end{multline}
%%
If no leaf is distinguished in $\GT$ the value of $p(\GT)$ is
exactly
%%
\begin{equation}
\auf \vec{y}_0\vec{z}_0,
\prod_{0 < i < n} \vec{y}_i\vec{z}_i\zu 
\end{equation}
%%
This claim can be proved inductively over the derivation of
$\GT$ in $G$. From this it follows immediately that $\vec{x}
\in L(K)$ if $\vec{x} \in L(G)$. For the converse inclusion one
has to choose a different proof. Let $\vec{x} \in L(K)$.
We choose a $K$--derivation of $\vec{x}$. Assume that no rule of
type \eqref{eq:rulef} has been used. Then $\vec{x}$ is the string of a
centre tree as is easily seen. Now we assume that the claim
has been shown for derivations with fewer than $n$ applications
of \eqref{eq:rulef} and that the proof has exactly $n$ applications.
We look at the last application. This is followed only by
applications of \eqref{eq:rulep}, \eqref{eq:rulet} and 
\eqref{eq:rulee}. These commute if they belong to different 
subtrees. We can therefore rearrange the order such that our 
application of \eqref{eq:rulef} is followed exactly by those 
applications of \eqref{eq:rulep}, \eqref{eq:rulet} and 
\eqref{eq:rulee} which belong to that subtree. They derive
%%
\begin{equation}
i^a(\vec{x}_0, \vec{x}_1).
\end{equation}
%%
where $i$ is the left hand side of the application of \eqref{eq:rulef}, 
and $\auf \vec{x}_0, \vec{x}_1\zu$ is the pair of the adjunction
tree whose root is $i$. ($\vec{x}_0$ is to the left of the
distinguished leaf, $\vec{x}_1$ to the right.)
Before that we have the application of our rule \eqref{eq:rulef}:
%%
\begin{equation}
j^a(\vec{x}_0\vec{y}_0,\vec{y}_1\vec{x}_1) \hrn
i^a(\vec{x}_0, \vec{x}_1), j^n(\vec{y}_0,\vec{y}_1).
\end{equation}
%%
Now we eliminate this part of the derivation. This means that in place
of $j^a(\vec{x}_0\vec{y}_0, \vec{y}_1\vec{x}_1)$ we only have
$j^n(\vec{y}_0, \vec{y}_1)$. This however is derivable (we
already have the derivation). But on the side of the adjunction
this corresponds exactly to the disembedding of the corresponding
adjunction tree.
%%
\proofend

The converse also holds. However, the head grammars do not exhaust 
the 2--LMGs. For example look at the following grammar $G$.
%%
\begin{align}
\notag
\mbox{\tt S}(y_0x_0y_1,x_1) & \hrn \mbox{\tt T}(x_0,x_1),
    \mbox{\tt H}(y_0,y_1). \\
\notag
\mbox{\tt T}(x_0, \mbox{\tt c}x_1\mbox{\tt d}) & \hrn
    \mbox{\tt U}(x_0,y_1). \\
\notag
\mbox{\tt U}(\mbox{\tt a}x_0\mbox{\tt b}, x_1) & \hrn
    \mbox{\tt S}(x_0,x_1). \\
\mbox{\tt S}(\mbox{\tt ab}, \mbox{\tt cd}) & \hrn . \\
\notag
\mbox{\tt H}(\mbox{\tt t}x_0\mbox{\tt u}, x_1) & \hrn
    \mbox{\tt K}(x_0, x_1). \\
\notag
\mbox{\tt K}(x_0, \mbox{\tt v}x_1\mbox{\tt w}) & \hrn
    \mbox{\tt H}(x_0,x_1). \\
\notag
\mbox{\tt H}(\varepsilon, \varepsilon) & \hrn.
\end{align}
%%
To analyze the generated language we remark the following facts.
%%
\begin{lem}
$\mbox{\tt H}(\vec{x}, \vec{y})$ iff
$\auf \vec{x}, \vec{y}\zu = \auf \mbox{\tt t}^n\mbox{\tt u}^n,
\mbox{\tt v}^n\mbox{\tt w}^n\zu$ for some $n \in \omega$.
\end{lem}
%%
As a proof one may reflect that first of all $\vdash_G
\mbox{\tt H}(\varepsilon, \varepsilon)$ and secondly
%%
\begin{equation}
\vdash_G \mbox{\tt H}(\mbox{\tt t}x_0\mbox{\tt u},
    \mbox{\tt v}x_1\mbox{\tt w}) \mbox{ iff }
    \vdash_G \mbox{\tt H}(x_0, x_1)
\end{equation}
%%
From this the following characterization can be derived.
%%
\begin{lem}
Let $\vec{x}_n := \mbox{\tt t}^n\mbox{\tt u}^n$ and
$\vec{y}_n := \mbox{\tt v}^n\mbox{\tt w}^n$. Then
%%
\begin{multline}
L(G) = \{\mbox{\tt a}\vec{x}_{n_0}\mbox{\tt a}\vec{x}_{n_1}
    \mbox{\tt a}\dotsb\vec{x}_{n_{k-1}}\mbox{\tt ab}
    \vec{y}_{n_{k-1}}\mbox{\tt b}\dotsb\mbox{\tt b}
    \vec{y}_{n_1}\mbox{\tt b}\vec{y}_{n_0}\mbox{\tt bc}^k
    \mbox{\tt d}^k : \\
k \in \omega, n_i \in \omega \text{ for all }i < k\}
\end{multline}
%%
In particular, for every $\vec{x} \in L(G)$
%%
\begin{equation}
\mu(\vec{x}) = m(\mbox{\tt a} + \mbox{\tt b} + \mbox{\tt c}
      + \mbox{\tt d}) + n(\mbox{\tt t} + \mbox{\tt u} +
        \mbox{\tt v} + \mbox{\tt w})
\end{equation}
%%
for certain natural numbers $m$ and $n$.
\end{lem}
%%
For example
%%
\begin{center}
{\tt aabbccdd}, {\tt atuabvwbccdd}, {\tt attuuatuabbvwbvvwwbcccddd},
$\dotsc$
\end{center}
%%
are in $L(G)$ but not
%%
\begin{center}
{\tt atuabbcd}, {\tt attuuatuabvwbvvwwbccdd}
\end{center}
%%
Now for the promised proof that there is no TAG which can
generate this language. 
%%
\begin{lem}
Let $H$ be a TAG with $L(H) = L(G)$ and $\GB$ a centre or
adjunction tree. Then
%%
\begin{equation}
\mu(\GB) = m_{\GB}(\mbox{\tt a} + \mbox{\tt b} + \mbox{\tt c}
    + \mbox{\tt d}) + n_{\GB}(\mbox{\tt t} + \mbox{\tt u} +
        \mbox{\tt v} + \mbox{\tt w})
\end{equation}
%%
for certain natural numbers $m_{\GB}$ and $n_{\GB}$.
%%
\end{lem}
%%
We put $\rho_{\GB} := n_{\GB}/m_{\GB}$. (This is $\infty$,
if $m = 0$.) Certainly, there exists the minimum of all
$\rho_{\GB}$ for all adjunction trees. It is easy to show that
it must be $0$. So there exists an adjunction tree
which consists only of {\tt t}, {\tt u}, {\tt v} and {\tt w},
in equal number. Further  there exists an adjunction tree
which contains {\tt a}.

Let $\vec{x}$ be a string from $L(G)$ such that
%%
\begin{equation}
\mu(\vec{x}) = m(\mbox{\tt a} + \mbox{\tt b} + \mbox{\tt c}
    + \mbox{\tt d}) + n(\mbox{\tt t} + \mbox{\tt u} +
        \mbox{\tt v} + \mbox{\tt w})
\end{equation}
%%
for certain natural numbers $m$ and $n$ such that
(a) $m$ is larger than any $m_{\GB}$, and
(b) $n/m$ is smaller than any $\rho_{\GB}$ that is not equal to 0.
It is to be noticed that such a $\vec{x}$ exists.
If $m$ and $n$ are chosen, the following string does the
job.
%%
\begin{equation}
\label{eq:55insert}
\mbox{\tt a}\mbox{\tt t}^n\mbox{\tt u}^n\mbox{\tt a}^{m-1}
\mbox{\tt b}^{n-1}\mbox{\tt v}^n\mbox{\tt w}^n\mbox{\tt bc}^m
\mbox{\tt d}^m
\end{equation}
%%
This string results from a centre tree by adjoining  (a$'$) an
$\GA$ in which {\tt a} occurs, by adjoining (b$'$) a $\GB$
in which {\tt a} does not occur. Now we look at points in which
$\GB$ has been inserted in \eqref{eq:55insert}. These can only 
be as follows.
%%
\begin{equation}
\mbox{\tt a}\mbox{\tt t}^n\bullet\mbox{\tt u}^n\mbox{\tt a}^{m-1}
\mbox{\tt b}^{n-1}\mbox{\tt v}^n\bullet\mbox{\tt w}^n\mbox{\tt bc}^m
\mbox{\tt d}^m
\end{equation}
%%
However, let us look where the adjunction tree $\GA$ has been
inserted.
%%
\begin{equation}
\mbox{\tt a}\mbox{\tt t}^n\mbox{\tt u}^n\mbox{\tt a}^{m-1}
\circ \mbox{\tt b}^{n-1}\mbox{\tt v}^n\mbox{\tt w}^n\mbox{\tt bc}^m
\circ\mbox{\tt d}^m
\end{equation}
%%
If we put this on top of each other, we get 
%%
\begin{equation}
\mbox{\tt a}\mbox{\tt t}^n\bullet\mbox{\tt u}^n\mbox{\tt a}^{m-1}
\circ \mbox{\tt b}^{n-1}\mbox{\tt v}^n\mbox{\tt w}^n\bullet
\mbox{\tt bc}^m \circ\mbox{\tt d}^m
\end{equation}
%%
Now we have a contradiction. The points of adjunction may not
cross! For the subword between the two $\bullet$ must be a
constituent, likewise the part between the two $\circ$. However,
these constituents are not contained in each other. (In order
for this to become a real proof one has to reflect over the fact
that the constituent structure is not changed by adjunction.
This is Exercise~\ref{ex:tagstruct}.)

So we have a 2--LMG which generates a language that cannot be
generated by a TAG. This grammar is 2--branching. In turn, 
2--branching 2--LMGs are weaker than full linear 2--LMGs. 
Some parts of the argumentation shall be transferred
to the exercises, since they are not of central concern.
%%
\begin{defn}
%%%
\index{literal movement grammar!$n$--branching}%%%
%%%
A linear LMG is called $n$--\textbf{branching} if the polynomial
base consists of at most $k$--ary vector polynomials.
\end{defn}
%%
The reason for this definition is the following fact.
%%
\begin{prop}
\label{prop:nverzweig}
Let $L = L(G)$ for some $n$--branching, $k$--linear
LMG $G$. Then there exists a $k$--linear LMG $H$ with $L(H) = L$
in which every rule is at most $n$--branching.
\end{prop}
%%
To this end one has to see that a rule with more than $n$ daughters
can be replaced by a canonical sequence of rules with at most
$n$ daughters, if the corresponding vector polynomial is generated
by at most $n$--ary polynomials. On the other hand it is not
guaranteed that there is no $n$--branching grammar if higher
polynomials have been used. Additionally, it is possible to construct
languages such that essentially $n+1$--ary polynomials have been
used and they cannot be reduced to at most $n$--ary polynomials.
Define as before
%%
\begin{align}
\vec{x}_n & := \mbox{\tt t}^n\mbox{\tt u}^n &
\vec{y}_n & := \mbox{\tt v}^n\mbox{\tt w}^n
\end{align}
%%
The following polynomial is not generable using polynomials that
are at most ternary.
%%
\begin{equation}
\Gq(\auf w_0,w_1\zu,\auf x_0,x_1 \zu, \auf y_0,y_1\zu,
\auf z_0z_1\zu) 
:= \auf w_0x_0y_0z_0, y_1w_1z_1x_1\zu
\end{equation}
%%
From this we can produce a proof that the following language
cannot be generated by a 2--branching LMG.
%%
\begin{equation}
L = \{\vec{x}_{n_0}\vec{x}_{n_1}\vec{x}_{n_2}\vec{x}_{n_3}%
\vec{y}_{n_2}\vec{y}_{n_0}\vec{y}_{n_3}\vec{y}_{n_1} :
n_0, n_1, n_2, n_3 \in \omega\}
\end{equation}

We close this section with a few remarks on the semantics. Adjunction 
is an operation that takes complete trees as input and returns a 
complete tree. This concept is not easily coupled with a semantics 
that assembles the meanings of sentences from their parts. It is 
--- at least in Montague semantics --- impossible to recover the 
meaning components of a sentence after completion, which would be 
necessary for a compositional account.  \cite{harris:structures} 
%%%
\index{Harris, Zellig S.}%%%
%%%
only gives a modest sketch of how adjunction 
is done in semantics. Principally, for this to work one needs a full 
record of which items are correlated to which parts of meaning 
(which is assumed, for example, in many syntactic theories, for 
example LFG and HPSG).
%%
\vplatz
\exercise
\label{ex:tagstruct}
Let $\GB$ be a tree and $\GA$ an adjunction tree. Let $\GC$
be the result of adjoining $\GA$ to $x$ in $\GB$. We view
$\GB$ in a natural way as a subtree of $\GC$ with $x$
the lower node of $\GA$ in $\GC$. Show the following: the
constituents of $\GB$ are exactly the intersection of constituents
of $\GC$ with the set of nodes of $\GB$.
%
\vplatz
\exercise
Show that the language $L := \{\mbox{\tt a}^n\mbox{\tt b}^n%
\mbox{\tt c}^n\mbox{\tt d}^n : n \in \omega\}$  cannot be
generated by an unregulated TAG. {\it Hint.} Proceed as in the
proof above.  Take a string which is large enough so that a tree
has been adjoined and analyze the places where it has been
adjoined.
%%
\vplatz
\exercise
Show that in the example above $\min \{\rho_{\GB} : \GB \in \BA\} = 0$. 
{\it Hint.} Compare the discussion in Section~\ref{kap2}.\ref{kap2-6}.
%%
\vplatz
\exercise
Show the following: {\it For every TAG $G$ there is a
TAG $G^{\diamondsuit}$ in standard form such that
$G^{\diamondsuit}$ and $G$ have the same constituent structures.}
What can you say about the labelling function?
%%
\vplatz
\exercise
Prove Proposition~\ref{prop:nverzweig}.
%%

 \section{Index Grammars}
\label{kap4-6}
%
%
%
Index grammars broaden the concept of CFGs in a very
special way. They allow to use in addition of the nonterminals a
sequence of indices; the manipulation of the sequences is however 
very limited. Therefore, we may consider these grammars alternatively 
as grammars that contain rule schemata rather than individual rules.
Let as usual $A$ be our alphabet, $N$ the set of nonterminals (disjoint 
with $A$). Now add a set $I$ of \textbf{indices}, disjoint to both $A$ 
and $N$. Furthermore, {\tt \#} shall be a symbol that does not occur in
$A \cup N \cup I$. An \textbf{index scheme} $\sigma$ has the
%%%
\index{index scheme}%%
%%%
form
%%
\begin{equation}
A \conc \vec{\alpha} \pf
    B_0 \conc \vec{\beta}_0  \quad
    \dotsb \quad
    B_{n-1} \conc \vec{\beta}_{n-1}
\end{equation}
%%
or alternatively the form
%%
\begin{equation}
A \conc \vec{\alpha} \pf a
\end{equation}
%%
where $\vec{\alpha}, \vec{\beta}_i \in I^{\ast} \cup \{\mbox{\tt\#}\}$ 
for $i < n$, and $a \in A$. The schemata of the second kind are called
%%%
\index{index scheme!terminal}%%
\index{instantiation}%%
%%%
\textbf{terminal schemata}. An \textbf{instantiation of} $\sigma$ is
a rule
%%
\begin{equation}
A \conc \vec{x}\vec{\alpha} \pf
    B_0 \conc \vec{y}_0\vec{\beta}_0  \quad
    \dotsb \quad
    B_{n-1} \conc \vec{y}_{n-1}\vec{\beta}_{n-1}
\end{equation}
%%
where the following holds.
%%
\begin{dingautolist}{192}
\item If $\vec{\alpha} = \mbox{\tt\#}$ then $\vec{x} = \varepsilon$
    and $\vec{y}_i = \varepsilon$ for all $i < n$.
\item If $\vec{\alpha} \neq \mbox{\tt\#}$ then for all
    $i < n$: $\vec{y}_i = \varepsilon$ or $\vec{y}_i = \vec{x}$.
\item For all $i < n$: if $\vec{\beta}_i = \mbox{\tt\#}$ then $\vec{y}_i =
    \varepsilon$.
\end{dingautolist}
%%
For a terminal scheme the following condition holds:
if $\vec{\alpha} = \mbox{\tt\#}$ then $\vec{x} = \varepsilon$.
An index scheme simply codes the set of all of its instantiations.
So we may also call it a \textbf{rule scheme}.
%%%
\index{rule scheme}%%
%%%
If in a rule scheme $\sigma$ we have $\vec{\alpha} = \mbox{\tt\#}$
as well as $\vec{\beta}_i = \mbox{\tt\#}$ for all $i < n$ then
we have the classical case of a context free rule. We therefore 
call an index scheme \textbf{context free} 
%%%
\index{index scheme!context free}%%
\index{index scheme!linear}%%
%%%
if it has this form. We call it \textbf{linear} if 
$\vec{\beta}_i \neq \mbox{\tt\#}$ for at 
most one $i < n$. Context free schemata are therefore also
linear but the converse need not hold. One uses the following
suggestive notation. $A[\;]$ denotes an $A$ with an arbitrary
stack; on the other hand, $A$ is short for $A\mbox{\tt\#}$. Notice
for example the following rule.
%%
\begin{equation}
\mbox{\tt A}[\mbox{\tt i}] \pf \mbox{\tt B}[\;] \quad
\mbox{\tt A} \quad \mbox{\tt C}[\mbox{\tt ij}]
\end{equation}
%%
This is another form for the scheme
%%
\begin{equation}
\mbox{\tt Ai} \pf \mbox{\tt BA\#Cij}
\end{equation}
%%
which in turn comprises all rules of the following form
%%
\begin{equation}
\mbox{\tt A}\vec{x}\mbox{\tt i} \pf \mbox{\tt B}\vec{x}
\quad \mbox{\tt A}\mbox{\tt\#}\quad \mbox{\tt C}\vec{x}\mbox{\tt ij}
\end{equation}
%%
\begin{defn}
%%%
\index{index grammar}%%
\index{index grammar!linear}%%
\index{LIG}%%
%%%
We call an \textbf{index grammar} a sextuple
$G = \auf S, A, N, I, \mbox{\tt\#}, R\zu$ where $A$, $N$, and $I$ are
pairwise disjoint finite sets not containing $\mbox{\tt\#}$, $S \in N$
the \textbf{start symbol} and $R$ a finite set of index schemata 
over $A$, $N$, $I$ and $\mbox{\tt\#}$. $G$ is called \textbf{linear} or a 
\textbf{LIG} if all its index schemata are linear.
\end{defn}
%%
The notion of a derivation can be formulated over strings as
well as trees. (To this end one needs $A$, $N$ and $I$ to be
disjoint. Otherwise the category symbols cannot be uniquely
reconstructed from the strings.) The easiest is to picture
an index grammar as a grammar $\auf S, N, A, R\zu$, where in
contrast to a context free rule set we have put an infinite
set of rules which is specified by means of schemata, which
may allow infinitely many instantiations. This allows us to
transfer many notions to the new type of grammars. For example,
it is easily seen that for an index grammar there is a
2--standard form which generates the same language.

The following is an example of an index grammar.
Let $A := \{ \mbox{\tt a}\}$, $N := \{\mbox{\tt S}, \mbox{\tt T},
\mbox{\tt U}\}$, $I := \{\mbox{\tt i}, \mbox{\tt j}\}$, and
%%
\begin{equation}
\begin{split}
\mbox{\tt S}[\;] & \pf \mbox{\tt T}[\mbox{\tt j}] & 
\qquad\qquad \mbox{\tt T}[\;] & \pf \mbox{\tt T}[\mbox{\tt i}] \\
\mbox{\tt T}[\mbox{\tt i}] & \pf \mbox{\tt U}[\;] &
\mbox{\tt U}[\mbox{\tt i}] & \pf \mbox{\tt U}[\; ] \quad
    \mbox{\tt U}[\;] \\
\mbox{\tt U}[\mbox{\tt j}] & \pf \mbox{\tt a}
\end{split}
\end{equation}
%%
This defines the grammar $G$. We have $L(G) = \{\mbox{\tt a}^{2^n}
: n \in \omega\}$. As an example, look at the following derivation.
%%
\begin{equation}
\begin{array}{l@{\quad}l@{\quad}l}
\auf \mbox{\tt S}, & \mbox{\tt Tj}     & \mbox{\tt Tji}, \\
\mbox{\tt Tjii},   & \mbox{\tt Tjiii}, & \mbox{\tt Ujii}, \\
\mbox{\tt UjiUji}, & \mbox{\tt UjiUjUj}, & \mbox{\tt UjUjUjUj}, \\
\mbox{\tt aUjUjUj},& \mbox{\tt aaUjUj}, & \mbox{\tt aaaUj}, \\
\mbox{\tt aaaa}\zu
\end{array}
\end{equation}
%%
Index grammars are therefore quite strong. Nevertheless, one can
show that they too can only generate \textbf{PTIME}--languages.
(For index grammars one can define a variant of the chart--algorithm
This variant also needs only polynomial time.)
Of particular interest are the linear index grammars.

Now we turn to the equality between LIGs and TAGs. Let $G$ be an
LIG; we shall construct a TAG which generates the same constituent
structures. We shall aim for roughly the same proof as with CFGs. 
The idea is again to look for nodes $x$ and $y$ with identical 
label $X\vec{x}$. This however can fail. For on the
one hand we can expect to find two nodes with identical label from
$N$, but they may have different index stack. It may happen
that no such pair of nodes exists. Therefore we shall introduce
the first simplification. We only allow rules of the following
form.
%%
\begin{subequations}
\begin{align}
\label{eq:56a}
X[i]  & \pf Y_0 \dotsb Y_{j-1}\; Y_j[\;] \; Y_{j+1} \dotsb Y_{n-1} \\
\label{eq:56b}
X[\;] & \pf Y_0 \dotsb Y_{j-1}\; Y_j[i] \; Y_{j+1} \dotsb Y_{n-1} \\
\label{eq:56c}
X     & \pf Y_0 \dotsb Y_{n-1} \\
\label{eq:56d}
X     & \pf a
\end{align}
\end{subequations}
%%
In other words, we only admit rules that stack or unstack
a single letter, or which are context free.  Such a grammar
%%%
\index{index grammar!simple}%%
%%%
we shall call \textbf{simple}. It is clear that we can turn $G$ into
simple form while keeping the same constituent structures. Then
we always have the following property. If $x$ is a node with
label $X\vec{x}$ and if $x$ immediately dominates the node
$x'$ with label $Y\vec{x}i$ then there exists a node $y' \leq x'$
with label $V\vec{x}i$ which immediately dominates a node with
label $W\vec{x}$. At least the stacks are now identical, but
we need not have $Y = V$. To get this we must do a second step.
We put $N' := N^2 \times \{o,e,a\}$ (but write 
$\auf A, B\zu^x$ in place of $\auf A, B, x\zu$). The superscript 
keeps score of the fact whether at this point we stack an index 
($a$), we unstack a letter ($e$) or we do nothing ($o$). The index
alphabet is $I' := N^2 \times I$. The rules above are now reformed
as follows. (For the sake of perspicuity we assume
that $n = 3$ and $j = 1$.) For a rule of the form \eqref{eq:56b}
we add all rules of the form
%%
\begin{equation}
\auf X, X'\zu^a \pf \auf Y_0, Y_0'\zu^{a/o} \;
    \auf Y_1, Y_1'\zu^{a/o}[\auf X,X',i\zu]
    \; \auf Y_2,Y_2'\zu^{a/o} 
\end{equation}
%%
So we stack in addition to the index $i$ also the information about
the label with which we have started. The superscript $a$ is
obligatory for $\auf X, X'\zu$! From the rules of the form 
\eqref{eq:56a} we make rules of the following form.
%%
\begin{equation}
\auf X,X'\zu^{a/o}[\auf W,Y_1',i\zu] \pf
    \auf Y_0, Y_0'\zu^{a/o} \quad
    \auf W,Y_1'\zu^e \quad
    \auf Y_2, Y_2'\zu^{a/o} 
\end{equation}
    %%
However, we shall also add these rules:
%%
\begin{equation}
\label{eq:s}
\auf Y_1, Y_1'\zu^e \pf \auf Y_1',Z\zu^{a/o}
\end{equation}
%%
for all $Y_1, Y_1', Z \in N$.
%%
The rules of the form \eqref{eq:56c} are replaced thus.
%%
\begin{equation}
\auf X, X'\zu^{o} \pf \auf Y_0, Y_0'\zu^{a/o} \quad
    \auf Y_1, Y_1'\zu^{a/o}\quad \auf Y_2,Y_2'\zu^{a/o}
\end{equation}
%%
Finally, the rules of the form \eqref{eq:56d} are replaced by these rules.
%%
\begin{equation}
\auf X,X'\zu^o \pf a 
\end{equation}
%%
We call this grammar $G^{\spadesuit}$. We shall at first see
why $G$ and $G^{\spadesuit}$ generate the same constituent structures.
To this end, let us be given a $G^{\spadesuit}$--derivation.
We then get a $G$--derivation as follows.  Every symbol of the form
$\auf X,X'\zu^{a/e/o}$ is replaced by $X$, every stack symbol
$\auf X,X',i\zu$ by $i$. Subsequently, the rules of type \eqref{eq:s} 
are skipped. This yields a $G$--derivation, as is easily checked.
It gives the same constituent structure. Conversely, let a
$G$--derivation be given with associated ordered labelled tree
$\GB$. Then going from bottom to top we do the following. Suppose
a rule of the form \eqref{eq:56b} has been applied to a node $x$ and
that $i$ has been stacked. Then look for the highest node
$y < x$ where the index $i$ has been unstacked. Let $y$ have
the label $B$, $x$ the label $A$. Then replace $A$ by
$\auf A, B\zu^a$ and the index $i$ on all nodes up to
$y$ by $\auf A, B, i\zu$. In between  $x$ and $y$ we insert a
node $y^{\ast}$ with label $\auf A, B\zu^e$. $y^{\ast}$ has
$y$ as its only daughter. $y$ keeps at first the label $B$. If
however no symbol has been stacked at $x$ then exchange the
label $A$ by $\auf A, A'\zu^o$, where $A'$ is arbitrary. If one
is at the bottom of the tree, one has a $G^{\spadesuit}$--tree.
Again the constituent structures have been kept, since only
unary rules have been inserted.

Now the following holds.  If at $x$ the index $\auf A, B, i\zu$
has been stacked then $x$ has the label $\auf A, B\zu^a$ and there
is a node $y$ below $x$ at which this index is again removed.
It has the label $\auf A, B\zu^e$. We say that $y$ is
\textbf{associated to} $x$. Now define as in the case of CFLs
centre trees as trees whose associated string
is a terminal string and in which no pair of associated
nodes exist. It is easy to see that in such trees no symbol
is ever put on the stack. No node carries a stack symbol and
therefore there are only finitely many such trees. Now we
define the adjunction trees. These are trees in which the
root has label $\auf A, B\zu^a$ exactly one leaf has a nonterminal
label and this is $\auf A, B\zu^e$. Further, in the interior
of the tree no pair of associated nodes shall exist. Again it
is clear that there are only finitely many such trees. They
form the basic set of our adjunction trees. However, we do
the following.  The labels $\auf X, X'\zu^o$ we replace by
$\auf X,X'\zu$, the labels $\auf X,X'\zu^a$ and $\auf X,X'\zu^e$
by $\auf X,X'\zu^{\triangledown}$. (Root and associated node
get an adjunction prohibition.) Now the proof is as in the
context free case.

Now let conversely a TAG $G = \auf\BC, N, A, \BA\zu$ be given. 
We shall construct a LIG which generates the same
constituent structures. To this end we shall assume that all
trees from $\BC$ and $\BA$ are based on pairwise disjoint
sets of nodes. Let $K$ be the union of all sets of nodes. This is
our set of nonterminals. The set $\BA$ is our set of
indices. Now we formulate the rules.  Let $i \pf j_0 \quad j_1 %
\dotsb j_{n-1}$ be a local subtree of a tree. 
\\
(A) $i$ is not central. Then add
%%
\begin{equation}
i \pf j_0\quad j_1 \dotsb j_{n-1}
\end{equation}
%%
(B) Let $i$ be root of $\GT$ and $j_k$ central
(and therefore not a distinguished leaf). Then add
%%
\begin{equation}
i[\;] \pf j_0 \quad j_{k-1}\quad  j_k[\GT] \quad j_{k+1}
\dotsb j_{n-1}
\end{equation}
%%
(C) Let $j_k$ be a distinguished leaf of $\GT$.
Then add
%%
\begin{equation}
i[\GT] \pf j_0 \quad j_{k-1}\quad j_k[\;]\quad j_{k+1}%
\dotsb j_{n-1}
\end{equation}
%%
(D) Let $i$ be central in $\GT$, but not a root
and $j_k$ central but not a distinguished leaf.
Then let
%%
\begin{equation}
i[\;] \pf j_0 \dotsb j_{k-1} \quad j_k[\;] \quad j_{k+1}\dotsb
    j_{n-1}
\end{equation}
%%
be a rule.  Nothing else shall be a rule. This defines the grammar
$G^I$. (This grammar may have start trees over distinct start
symbols. This can be remedied.)
Now we claim that this grammar generates the same constituent
structures over $A$. This is done by induction over the length
of the derivation. Let $\GT$ be a centre tree, say
$\GT = \auf B, <, \sqsubset, \ell\zu$.  Then let
$\GT^I := \auf B, <, \sqsubset, \ell^I\zu$, where
$\ell^I(i) := i$ if $i$ is nonterminal and $\ell^I(i) := \ell(i)$
otherwise. One establishes easily that this tree is derivable.
Now let $\GT = \auf B, <, \sqsubset, \ell\zu$ and
$\GT^I = \auf B, <, \sqsubset, \ell^I\zu$ already be
constructed; let $\GU = \auf C, <', \sqsubset', \ell'\zu$
result from $\GT$ by adjoining a tree $\GB$ to a node
$x$. By making $x$ into the root of an adjoined tree
we get $B \subseteq C$, $<' \cap B^2 = \; <$, $\sqsubset' \cap B^2 =
\; \sqsubset$ and $\ell' \restriction B = \ell$. Now
$\GU^I = \auf C, <', \sqsubset', \ell'^I\zu$.
Further, there is an isomorphism between the  adjunction tree
$\GB$ and the local subtree induced on $C \cup \{x\}$. Let
$\pi \colon C \cup \{x\} \pf B$ be this isomorphism.
Put $\ell'^I(y) := \ell^I(y)$ if $y \in B - C$.
Put $\ell'^I(y) := \pi(y)$ if $\pi(y)$ is not central;
and put $\ell'^I(y) := \ell'^I(x) := \ell^I(x)$ if
$y$ is a distinguished leaf. Finally, assume
$\ell^I(x) = X\vec{x}$, where $X$ is a nonterminal symbol
and $\vec{x} \in I^{\ast}$. If $y$ is central but not
root or leaf then put
%%
\begin{equation}
\ell'^I(y) := \pi(y)\vec{x}\GB
\end{equation}
%%
Now it is easily checked that the so--defined tree
is derivable in $G^I$. We have to show likewise
that if $\GU$ is derivable in $G^I$ there exists a tree
$\GU^A$ with $(\GU^A)^I \cong \GU$ which is derivable in $G$.
To this end we use the method of disembedding. One looks
for nodes $x$ and $y$ such that they have the same stack,
$x > y$, there is no element between the two that has
the same stack. Further, there shall be no such pair in
$\low{x} - (\low{y} \cup \{x\})$. It is easily seen that 
this tree is isomorphic to an adjunction tree. We disembed 
this tree and gets a tree which is strictly smaller. (Of 
course, the existence of such a tree must still be shown. 
This is done as in the context free case. Choose
$x$ of minimal height such that such there exists a $y < x$
with identical stack. Subsequently, choose $y$ maximal with
this property. In $\low{x} - (\low{y} \cup \{x\})$ there can
then be no pair $x'$, $y'$ of nodes with identical stack such
that $y' < x'$. Otherwise, $x$ would not be minimal.) We
summarize.
%%
\begin{thm}
A set of constituent structures is generated by a linear index
grammar iff it is generated by a TAG.
\end{thm}
%%
We also say that these types of grammars are equivalent in
constituent analysis.

A rule is called \textbf{right linear} if the index is only passed
on to the right hand daughter. So, the right hand rule is
right linear, the left hand rule is not:
%%
\begin{equation}
\mbox{\tt A}[\;] \pf \mbox{\tt B}\quad \mbox{\tt C}[\mbox{\tt i}]
\quad \mbox{\tt B}, \qquad \mbox{\tt A}[\; ] \pf \mbox{\tt B}\quad
\mbox{\tt C}\quad \mbox{\tt B}[\mbox{\tt i}]
\end{equation}
%%
\index{index grammar!right linear}%%
%%%
An index grammar is called \textbf{right linear} if all of its
rules are right linear. Hence it is automatically linear.
The following is from 
\cite{michaelis_wartena:fg97,michaelis_wartena:csli-ligs}. 
%%
\begin{thm}[Michaelis \& Wartena]
%%%
\index{Michaelis, Jens}%%
\index{Wartena, Christian}%%%
%%%
A language is generated by a right linear index grammar iff
it is context free.
\end{thm}
%%
\proofbeg
Let $G$ be right linear, $X \in N$. Define $H_X$ as follows. The
alphabet of nonterminals has the form $T := \{X^{\diamond} :
X \in N\}$. The alphabet of terminals is the one of $G$, likewise
the alphabet of indices. The start symbol is $X$. Now for every
rule
%%
\begin{equation}
A[\;] \pf B_0 \dotsb B_{n-1}\quad B_n[i]
\end{equation}
%%
we add the rule
%%
\begin{equation}
A^{\diamond}[\;] \pf A \quad B^{\diamond}[i]
\end{equation}
%%
This grammar is right regular and generates a CFL (see the 
exercises). So there exists a CFG
$L_X := \auf S_X^L, N_X^L, N, R_X^L\zu$ which generates $L(H_X)$.
(Here $N$ is the alphabet of nonterminals of $G$ but the terminal
alphabet of $L_X$.) We assume that $N^X_L$ is disjoint to our
previous alphabets. We put $N' := \bigcup N_X^L \cup N$ as well as
$R' := \bigcup R_X^L \cup R \cup R^-$ where $R$ is the set of
context free rules of $G$ and $R^-$ the set of rules
$A[\;] \pf B_0 \dotsb B_{n-1}$ such that
$A[\;] \pf B_0 \dotsb B_{n-1}\quad B_n[i] \in R$.
Finally, let $G' := \auf S_L, N', A, R'\zu$.
$G'$ is certainly context free. It remains to show that
$L(G') = L(G)$. To this end let $\vec{x} \in L(G)$. There exists
a tree $\GB$ with associated string $\vec{x}$ which is derived
from $G$. By induction over the height of this tree one shows that
$\vec{x} \in L(G')$. The inductive hypothesis is this:
{\it For every $G$--tree $\GB$ with associated string $\vec{x}$
there exists a $G'$--tree $\GB'$ with associated string
$\vec{x}$; and if the root of $\GB$ carries the label
$X\vec{x}$ then the root of $\GB'$ carries the label $X$.} If
$\GB$ contains no stack symbols, this claim is certainly true.
Simply take $\GB' := \GB$. Further, the claim is easy to see if
the root has been expanded with a context free rule.
Now let this not be the case; let the tree have a root with label
$U$. Let $P$ be the set of right hand nodes of $\GB$. For
every $x \in P$ let $B(x)$ be that tree which contains all nodes
which are below $x$ but not below any $y \in P$ with $y < x$. It
is easy to show that these sets form a partition of $\GB$.
Let $u \prec x$, $u \not\in P$. By induction hypothesis,
the tree dominated by $u$ can be restructured into a tree
$\GT_u$ which has the same associated string and the same
root label and which is generated by $G'$. The local tree of
$x$ in $B(x)$ is therefore an instance of a rule of $R^-$. We
denote the tree obtained from $x$ in such a way by $\GB'_x$.
$\GB'_x$ is a $G'$--tree. Furthermore:  if $y < x$, $y \in P$,
and if $u < x$ then $u \sqsubset y$. Therefore we have
that $P = \{x_i : i < n\}$ is an enumeration with
$x_i > x_{i+1}$ for all $i < n-1$. Let $A_i$ be the root label
of $x_i$ in $\GB'_{x_i}$. The string $\prod_{i < n} A_i$ is a
string of $H_U$. Therefore it is generated by $L_U$. Hence it
is also generated by $G'$. So, there exists a tree $\GC$ associated
to this string. Let the leaves of this tree be exactly the $x_i$ and
let $x_i$ have the label $A_i$. Then we insert $\GB'_{x_i}$ at
the place of $x_i$ for all $i < n$. This defines $\GD$.
$\GD$ is a $G'$--tree with associated string $\vec{x}$. The
converse inclusion is left to the reader.
\proofend

We have already introduced Combinatory Categorial Grammars
(CCGs) in Section~\ref{kap3}.\ref{kap3-2}. The concept of these grammars
was very general. In the literature, the term CCG is usually
fixed --- following Mark Steedman --- to a particular variant
where only those combinators may be added that perform
function application and generalized function composition.
In order to harmonize the notation, we revise it as follows.
%%
\begin{equation}
%%%
\mbox{\mtt $\alpha$\fslash$\beta$} 
	\text{ replaces }\mbox{\mtt $\alpha${\tf}$\beta$}, 
\mbox{\mtt $\alpha$\bslash$\beta$} 
	\text{ replaces }\mbox{\mtt $\beta$\tb$\alpha$}
\end{equation}
%%
We take $p_i$ as a variable for elements from $\{\mbox{\mtt\symbol{43}}, 
\mbox{\mtt\symbol{45}}\}$. A \textbf{category} is a well formed 
string over $\{B, \mbox{\tt (}, \mbox{\tt )}, \mbox{\mtt\fslash}, 
\mbox{\mtt\bslash}\}$. We agree on {\it obligatory\/} left associative 
bracketing. That means that the brackets that we do not need to 
write assuming left associativity actually are {\it not\/} present 
in the string. Hence {\mtt a{\fslash}b{\bslash}c} is a category, as 
is {\mtt a{\fslash}(b{\bslash}c)}. However, 
{\mtt ((a{\fslash}b){\bslash}c)} and {\mtt (a{\fslash}(b{\bslash}c))}
%%%
\index{block}%%
%%%
are not. A \textbf{block} is a sequence of the form 
{\mtt {\fslash}$a$} or {\mtt {\bslash}$a$}, $a$ basic, or of 
the form {\mtt {\fslash}($\beta$)} or {\mtt {\bslash}($\beta$)},
where $\beta$ is a complex category symbol. (Often we ignore 
the details of the enclosing brackets.)
%%%
\index{p--category}%%
%%
A \textbf{p--category} is a sequence of blocks, seen as a string.
With this a category is simply a string of the form
$\alpha \conc \Delta$ where $\Delta$ is a  p--category.
If $\Delta$ and $\Delta'$ are p--categories, so is
$\Delta \conc \Delta'$. For a category $\alpha$ we define
by induction the head, $\alpha$, $K(\alpha)$, as follows.
%%
\begin{dingautolist}{192}
%%%
\index{$K(\alpha)$}%%%
%%%
\item $K(b) := b$.
\item $K(\mbox{\mtt $\alpha${\fslash}$\beta$}) := 
K(\mbox{\mtt $\alpha${\bslash}$\beta$}) := K(\alpha)$.
\end{dingautolist}
%%
\begin{lem}
Every category $\alpha$ can be uniquely segmented as
$\alpha = K(\alpha)\conc\Delta$ where $\Delta$ is a
p--category.
\end{lem}
%%
If we regard the sequence simply as a string we can use 
$\conc$ as the concatenation symbol of blocks
as well as of sequences of blocks. We admit the following
operations. (If $\beta$ is basic, omit the additional 
enclosing brackets.)
%%
\begin{align}
\mbox{\mtt $\alpha${\fslash}($\beta$)} \circ_1 \beta & := \alpha \\
\beta        \circ_2 \mbox{\mtt $\alpha${\bslash}($\beta$)} & := \alpha \\
\mbox{\mtt $\alpha${\fslash}($\beta$)} \circ_3^n \beta \conc \Delta^n &
    := \alpha \conc \Delta^n \\
\beta \conc \Delta^n \circ_4^n \mbox{\mtt $\alpha${\bslash}($\beta$)} &
    := \alpha \conc \Delta^n
\end{align}
%%
Here $\Delta^n$ is a variable for p--categories consisting of
$n$ blocks. In addition it is possible to restrict the choice
of heads for $\alpha$ and $\beta$. This means that we define
operations $\circ^{F,A,n}_i$ in such a way that
%%
\begin{equation}
\alpha \circ^{L,R,n}_i \beta :=
\begin{cases}
    \alpha \circ^n \beta & \text{ if $K(\alpha) \in L,
        K(\beta) \in R$,} \\
    \star & \text{ otherwise.}
    \end{cases}
\end{equation}
%%
This means that we have to step back from our ideal to let
the categories be solely determined by the combinators.
%%
\begin{defn}
A \textbf{combinatory categorial grammar} (or \textbf{CCG}) 
%%%
\index{combinatory categorial grammar}%%
\index{CCG (see combinatory categorial grammar)}%%
%%%
is a categorial grammar which uses finitely many operations
from 
%%%
\begin{equation}
\{\circ_1^{L,R}, \circ_2^{L,R} : L, R \subseteq B\}
\cup \{\circ_3^{L,R,n}, \circ^{L,R,n}_4 : n \in \omega,
L, R \subseteq B\}
\end{equation}
\end{defn}
%%
Notice by the way that $\circ_1 = \circ_3^0$ and
$\circ_2 = \circ_4^n$. This simplifies the calculations.
%%
\begin{lem}
\label{lem:ccg}
Let $G$ be a CCG over $A$ and $M$ the set of categories which are
subcategories of some $\alpha \in \zeta(a)$, $a \in A$. Then the
following holds. If $\vec{x}$ is a string of category $\alpha$ in
$G$ then $\alpha = \beta \conc \Delta$ where $\alpha \in M$ and
$\Delta$ is a  p--category over $M$.
\end{lem}
%%
The proof is by induction over the length of $\vec{x}$ and is left
as an exercise.
%%
\begin{thm}
For every CCG $G$ there exists a linear index grammar $H$
which generates the same trees.
\end{thm}
%%
\proofbeg
Let $G$ be given. In particular, $G$ associates with every letter
$a \in A$ a finite set $\zeta(a)$ of categories. We consider the
set $M$ of subterms of categories from $\bigcup \auf \zeta(a) :
a \in A\zu$. This is a finite set. We put $N := M$ and $I := M$.
By Lemma~\ref{lem:ccg}, categories can be written as pairs
$\alpha[\Delta]$ where $\alpha \in N$ and $\Delta$ is a
p--category over $I$. Further, there exist finitely many
operations which we write as rules. Let for example $\circ_1^{L,R}$
be an operation. This means that we have rules of the form
%%
\begin{subequations}
%\label{eq:klammern}
\begin{align}
\alpha & \pf \mbox{\mtt $\alpha${\fslash}$a$} \quad a \\
\alpha & \pf \mbox{\mtt $\alpha${\fslash}($\beta$)} \quad \beta 
\end{align}
\end{subequations}
%%
where $K(\beta) \in L$ and $K(\alpha) \in R$, and $\beta$ not 
basic. We write this into linear index rules. Notice that in 
any case $\beta \in M$ because of Lemma~\ref{lem:ccg}. Furthermore, 
we must have $\alpha \in M^+$. So we write down all the rules of 
the form
%%
\begin{equation}
\label{eq:56dagger}
\alpha \pf \delta[\Delta] \quad \beta
\end{equation}
%%
where $\delta[\Delta] = \mbox{\mtt $\alpha${\fslash}$\beta$}$ 
for certain $\alpha, \Delta \in M^{\ast}$ and $\delta, \beta \in M$.
We can group these into finitely many rule schemata.
Simply fix $\beta$ where $K(\beta) \in R$. Let $\BB$ be
the set of all sequences $\auf \gamma_i : i < p\zu \in M^{\ast}$
whose concatenation is \mbox{\mtt {\fslash}$\beta$}. $\BB$ is 
finite.  Now put for \eqref{eq:56dagger} all rules of the form
%%
\begin{equation}
\label{eq:56ddagger}
\alpha'[]  \pf \alpha'[\Delta] \quad \beta
\end{equation}
%%
where $\alpha' \in M$ is arbitrary with $K(\alpha) \in L$ and
$\Delta \in \BB$. Now one can see easily that every instance of
\eqref{eq:56dagger} is an instance of \eqref{eq:56ddagger} and 
conversely.

Analogously for the rules of the following form.
%%
\begin{equation}
\alpha \pf \beta \quad \mbox{\mtt $\alpha${\bslash}$\beta$} 
\end{equation}
%%
In a similar way we obtain from the operations
$\circ_3^{L,R,n}$ rules of the form
%%
\begin{equation}
\label{eq:56star}
    \alpha \conc \Delta^n \pf \mbox{\mtt $\alpha${\fslash}$\beta$} \quad
    \beta \conc \Delta^n
\end{equation}
%%
where $K(\alpha) \in L$ and $K(\beta) \in R$. Now it turns out
that, because of Lemma~\ref{lem:ccg} $\Delta^n \in M^n$ and $\beta
\in M$. Only $\alpha$ may again be arbitrarily large. Nevertheless
we have $\alpha \in M^+$, because of Lemma~\ref{lem:ccg}.
Therefore, \eqref{eq:56star} only corresponds to finitely many index
schemata. \proofend

The converse does not hold: for the trees which are generated by an
LIG  need not be 3--branching. However, the two grammar types are
weakly equivalent.

{\it Notes on this section.} There is a descriptive characterization
of indexed languages akin to the results of Chapter~\ref{kap5}
which is presented in \cite{langholm:indexed}. 
%%%
\index{Langholm, Tore}%%%
%%%
The idea there is to replace the index by a so--called contingency 
function, which is a function on the nodes of the constituent structure 
that codes the adjunction history.
%%
%\vplatz
%\exercise
%%%%
%\index{index grammar!right regular}%%
%%%%
%A linear index grammar is called {\it right regular} if
%rules either have the form $X \pf a Y[i]$ the form
%$X[i] \pf a Y$ or the form $x \pf aY$, where $X$ and $Y$ are
%nonterminals. Show that $L$ is context free iff
%it is the language of a right regular linear index grammar.
%{\it Hint.} Model the actions of a pushdown automaton
%with the help of right linear rules.
%%
\vplatz
\exercise
Show the following claim. {\it For every index
grammar $G$ there is an index grammar $H$ in 2--standard form such
that $L(G) = L(H)$.  If $G$ is linear (context free) $H$ can be
chosen linear (context free) as well.}
%%
\vplatz
\exercise
Prove the Lemma~\ref{lem:ccg}.
%%
\vplatz
\exercise
Write an index grammar that generates the sentences of predicate
logic. (See Section~\ref{kap2}.\ref{kap2-6} for a definition.)
%%
\vplatz \exercise Let {\it NB\/} be the set of formulae of predicate 
logic with {\mtt\symbol{61}} and {\mtt\symbol{4}} in which every 
quantifier binds at least one occurrence of a variable. Show that
there is no index grammar that generates {\it NB}. {\it Hint.} It
is useful to concentrate on formulae of the form $QM$, where $Q$
is a sequence of quantifiers and $M$ a formula without quantifiers
(but containing any number of conjuncts). Show that in order to
generate these formulae from {\it NB}, a branching rule is needed.
Essentially, looking top down, the index stack has to memorize
which variables have been abstracted over, and the moment that
there is a branching rule, the stack is passed on to both
daughters. However, it is not required that the left and right branch 
contain the same variables.

 \section{Compositionality and Constituent Structure}
\label{kap4-7}
%
%
%
In this section we return to the discussion of compositionality,
which we started in Chapter~\ref{kap3}. Our concern is how
constituent structure and compositionality constrain each other.
In good cases this shows that the semantics of a language does 
not allow a certain syntactic analysis. This will allow to give 
substance to the distinction between weak and strong generative 
capacity of grammar types.

Recall once again Leibniz' Principle. 
%%%
\index{Leibniz' Principle}%%%
%%%
It is defined on the basis 
of constituent substitution and truth equivalence. However, 
constituent substitution is highly problematic in itself, for it 
hardly ever is string substitution. If we do string substitution 
of {\tt fast} by {\tt good} in \eqref{ex:561}, we get \eqref{ex:562}
and not the correct \eqref{ex:563}.
%%%
\begin{align}
\label{ex:561} & \mbox{\tt Simon is faster than Paul.} \\
\label{ex:562} & ^{\ast}\mbox{\tt Simon is gooder than Paul.} \\
\label{ex:563} & \mbox{\tt Simon is better than Paul.} 
\end{align}
%%
Notice that substitution in $\lambda$--calculus and predicate logic 
also is not string substitution but something more complex. (This  
is true even when variables are considered simple entities.)
What makes matters even more difficult in natural languages is the fact 
that there seems to be no uniform algorithm to perform such substitution. 
Another, related problem is that of determining occurrences. For 
example, does {\tt cater} occur as a subexpression in the word 
{\tt caterpillar}, 
{\tt berry} as a subexpression of {\tt cranberry}? What about {\tt kick} 
in {\tt kick the bucket}? Worse still, does {\tt tack} occur as a
subexpression of {\tt stack}, {\tt rye} as a subexpression of {\tt
rice}? Obviously, no one would say that {\tt tack} occurs in {\tt
stack}, and that by Leibniz' Principle its meaning is distinct from 
%%%
\index{Leibniz' Principle}%%%
%%%
{\tt needle} since there is no word {\tt sneedle}. Such an argument 
is absurd. Likewise, in the formula $\mbox{\tt (p0}\und\mbox{\tt p01)}$, 
the variable {\tt p} does not occur, even though the string {\tt p} 
is a substring of the formula as a string.

To be able to make progress on these questions we have to resort to 
the distinction between language and grammar. As the reader will 
see in the exercises, there is a tight connection between the choice 
of constituents and the meanings these constituents can have. If we 
fix the possible constituents and their meanings this eliminates some 
but not all choices.  However it does settle the question of identity 
in meaning and can then lead to a detection of subconstituents. For if 
two given expressions have the same meaning we can conclude that they 
can be substituted for each other without change in the truth value in 
any given sentence on condition that they also have 
the same category. (Just an aside: sometimes substitution can be 
blocked by the exponents so that substitution is impossible even when 
the meanings and the category are the same. These facts are however 
generally ignored. See below for further remarks.) So, Leibniz' 
%%%
\index{Leibniz' Principle}%%%
%%%
Principle does tell us something about which substitutions are 
legitimate, and which occurrences of substrings 
are actually occurrences of a given expression. If {\tt sneedle} 
does not exist we can safely conclude that {\tt tack} has no proper 
occurrence in {\tt stack} if substitution is simply string substitution. 
Moreover, Leibniz' Principle also says that if two expressions are 
%%%
\index{Leibniz' Principle}%%%
%%%
intersubstitutable everywhere without changing the truth value, 
then they have the same meaning.  
%(This actually means that there 
%might exist languages that do not satisfy it.)
%%%
\begin{defn}
%%%
\index{occurrence}%%
\index{replacement}%%
%%%
Let $\Gs$ be a structure term with a single free variable, $x$, and $\Gu$ 
a structure term unfolding to $\tau$. If $[\Gu/x]\Gs$ unfolds to 
$\sigma$ we say that the sign $\tau$ \textbf{occurs in $\sigma$ 
under the analyses $\Gs$ and $\Gu$}. Suppose now that $\Gu'$ is a structure 
term unfolding to $\tau'$, and that $[\Gu'/x]\Gs$ is definite and unfolds to 
$\sigma'$. Then we say that $\sigma'$ \textbf{results from} $\sigma$ 
\textbf{by replacing the occurrence of $\tau$ by $\tau'$ under the 
analyses $\Gs$, $\Gu$ and $\Gu'$}.
\end{defn}
%%
This definition is complicated since a given sign may have different
structure terms, and before we can define the substitution operation
on a sign we must fix a structure term for it. This is particularly 
apparent when we want to define simultaneous substitution. Now, in 
ordinary parlance one does not usually mention the structure 
term. And substitution is typically defined not on signs but on exponents
(which are called \textbf{expressions}). This, however, is dangerous
and the reason for much confusion. For example, we have proved
in Section~\ref{kap3}.\ref{kap3-1} that almost every recursively 
enumerable sign system has a compositional grammar. The proof used 
rather tricky functions on the exponents. Consequently, there is no 
guarantee that if $\vec{y}$ is the exponent of $\tau$ and $\vec{x}$ the
exponent of $\sigma$ there is anything in $\vec{x}$ that resembles
$\vec{y}$. Contrast this with CFGs, where a
subexpression is actually also a substring. To see the dangers
of this we discuss the theory of compositionality of
\cite{hodges:compositionality}. Hodges 
%%%
\index{Hodges, Wilfrid}%%
%%%
discusses in passim the following principle, which he attributes to 
Tarski 
%%%
\index{Tarski, Alfred}%%%
%%%
(from \cite{tarski:truth}). The original formulation in 
\shortcite{hodges:compositionality} was flawed. The correct 
version according to Hodges (p.c.) is this.
%%%
\begin{quote}
%%%
\index{Tarski's Principle}%%%
%%%
{\sl Tarski's Principle.} If there is a $\mu$--meaningful structure
    term $[\Gs/x]\Gu$ unequal to $\Gu$ and $[\Gs'/x]\Gu$ also is
    a $\mu$--meaningful structure term with $[\Gs'/x]\Gu \sim_{\mu}
    [\Gs/x]\Gu$ then $\Gs \sim_{\mu} \Gs'$.
\end{quote}
%%
Notice that the typed $\lambda$--calculus satisfies this condition.
Hodges dismisses Tarski's Principle on the following grounds.
%%
\begin{align}
\label{ex:569} & \mbox{\tt The beast ate the meat.} \\
\label{ex:5610} & \mbox{\tt The beast devoured the meat.} \\
\label{ex:5611} & \mbox{\tt The beast ate.} \\
\label{ex:5612} & ^{\ast}\mbox{\tt The beast devoured.} 
\end{align}
%%
Substituting {\tt the beast ate} by {\tt the beast devoured} in
\eqref{ex:569} yields a meaningful sentence, but it does not
with \eqref{ex:5611}. (As an aside: we consider the appearance 
of the upper case letter as well as the period as the result of 
adding a sign that turns the proposition into an assertion. Hence 
the substitution is performed on the string beginning with a lower 
case {\tt t}.) Thus, substitutability in one sentence does not 
imply substitutability in another, so the argument goes. The 
problem with this argument is that it assumes that we can 
substitute {\tt the beast ate} for {\tt the beast devoured}. 
Moreover, it assumes that this is the effect of replacing a 
structure term $\Gu$ by $\Gu'$ in some structure term for 
\eqref{ex:569}.  Thirdly, it assumes that if we perform the same 
substitution in a structure term for \eqref{ex:5611} we get 
\eqref{ex:5612}. Unfortunately, none of these assumptions is justified. 
(The pathological examples of Section~\ref{kap3}.\ref{kap3-1} should suffice 
to destroy this illusion.) What we need is a strengthening of the 
conditions concerning admissible operations on exponents. In the 
example sentence, the substituted strings are actually nonconstituents, 
so even under standard assumptions they do not constitute 
counterexamples. We can try a different substitution, for example 
replacing {\tt the meat} by $\varepsilon$. This is a constituent 
substitution under the ordinary analysis. But this does not save 
the argument, for we do not know which grammar underlies the examples. 
It is not clear that Tarski's Principle is a good principle. But the 
argument against it is fallacious. 

Obviously, what is needed is a restriction on the syntactic
operations. In this book, we basically present two approaches. 
One is based on polynomials (noncombinatorial LMGs), the other 
on $\lambda$--terms for strings. In both cases the idea is that 
the functions should not destroy any material (ideally, each rule 
should {\it add\/} something).  In this way the notion of composition 
does justice to the original meaning of the word. (Compositionality 
derives from Latin \textsf{compositi\={o}}`the putting together'.) 
Thus, every derived string is the result of applying some polynomial 
applied to certain vectors, and this polynomial determines the structure
as well as --- indirectly  --- the meaning and the category. 
Both approaches have a few abstract features in common. For example, 
that the application of a rule is progressive with respect to some 
progress measurse. (For LMGs this measure is the combined length 
of the parts of the string vector.)
%%
\begin{defn}
%%%
\index{progress measure}%%%
\index{sign grammar!progressive}%%%
%%%
A \textbf{progress measure} is a  function $\mu : E \pf \omega$.
A function $f : E^n \pf E$ is \textbf{progressive with respect to} 
$\mu$ if 
\begin{equation}
\mu(f(\vec{e})) > \max \{\mu(e_i) : i < n\}
\end{equation}
%%%
$f$ is \textbf{strictly progressive} if 
%%%
\index{sign grammar!strictly progressive}%%%
%%%
\begin{equation}
\mu(f(\vec{e})) > \sum_{i < n} \mu(e_i)
\end{equation}
%%% 
A sign grammar is (\textbf{strictly}) \textbf{progressive with respect 
to} $\mu$ if for all modes $f$, $f^{\varepsilon}$ is (strictly) 
progressive.
\end{defn}
%%
For example, a CFG is progressive with respect to length if it has 
no unary rules, and no empty productions (and then it is also strictly 
progressive). Let $\mu$ be a progress measure, $\GA$ a progressive 
grammar generating $\Sigma$. Then a given exponent $e$ can be derived 
with a term that has depth at most $\mu(e)$. This means that its 
length is $\leq \Omega_{\intercal}^{\mu(e)}$, where $\Omega_{\intercal} := 
\max \{\Omega(f) : f \in F\}$. The number of such terms is 
$\leq |F|^{\Omega_{\intercal}^{\mu(e)}}$, so it is doubly exponential in $\mu(e)$! 
If $\GA$ is strictly progressive, the length of the structure term 
is $\leq \mu(e)$, so we have at most $|F|^{\mu(e)}$ many. Now, 
finally, suppose that the unfolding of a term is at most 
exponential in its length, then we can compute for strictly 
progressive grammars in time $O(2^{c\mu(e)})$ whether $e$ is 
in $\pi_0[\Sigma]$.
%%
\begin{thm}
Suppose that $\GA$ is strictly progressive with respect to the 
pro\-gress measure $\mu$. Assume that computing the unfolding of a 
term can be done in time exponential in the length. Then for every $e$, 
`$e \in \pi_0[\Sigma]$' can be solved in time $O(c^{\mu(e)})$ 
for some $c > 0$. If $\GA$ is only progressive, `$e \in \pi_0[\Sigma]$' 
can be solved in time $O(c^{c^{\mu(e)}})$ for some $c > 0$.
\end{thm}
%%
Notice that this one half of the finite reversibility for grammars 
(see Definition~\ref{defn:reverse}). The other half requires a similar 
notion of progress in the semantics. This would correspond to the 
idea that the more complex a sentence is the more complex its meaning. 
(Unlike in classical logic, where for example 
{\mtt (p\symbol{31}(\symbol{5}p))} is simpler than {\mtt p}.)

We still have not defined the notion of intersubstitutability 
that enters Leibniz' Principle. We shall give a definition based 
%%%
\index{Leibniz' Principle}%%%
%%%
on a grammar. Recall that for Leibniz' Principle we need a category 
of sentences, and a notion of truth. So, let $\GA$ be a sign grammar 
and {\tt S} a category. Put 
%%%
\index{$\Sent_{\GA, \mbox{\smtt S}}$}%%%
%%%
\begin{equation}
\Sent_{\GA,\mbox{\smtt S}} := \{\Gs : \Gs^{\mu} = \mbox{\tt S}\}
\end{equation}
%%%
This defines the set of sentential terms (see also 
Section~\ref{kap5}.\ref{kap5-1}). Let $\Gt$ be a structure term with a 
single occurrence of a free variable, say $x$. Then given $\Gs$, 
$[\Gs/x]\Gt$ if definite is the result of putting $\Gs$ in place 
of $x$. Thus $\Gt^{\varepsilon} \in \Pol_1(\GE)$. We define the 
%%%
\index{context set}%%
%%%%
\textbf{context set} of $e$ as follows. 
%%%
\begin{multline}
\label{eq:contset}
\Cont_{\GA, \mbox{\smtt S}}(e) := 
	\{\Gt^{\varepsilon} : \text{for some }\Gs \text{ such that 
	}\Gs^{\varepsilon} = e \colon \\
	[\Gs/x]\Gt \in \Sent_{\GA, \mbox{\smtt S}}\}
\end{multline}
%%%
We shall spell this out for a CFG. In a CFG, $\Cont_{\GA, \mbox{\smtt S}} 
\in \Pol_1(\GZ(A))$. Moreover, if $x$ occurs only once, 
the polynomials we get are quite simple: they are of the form 
$p(x) = \vec{u}\conc x\conc\vec{v}$ for certain strings 
$\vec{u}$ and $\vec{v}$. Putting $C := \auf \vec{u}, \vec{v}\zu$, 
$p(\vec{x}) = C(\vec{x})$. Thus, the context set for $\vec{x}$ 
defined in \eqref{eq:contset} is the set of all $C$ such that 
$C(\vec{x})$ is a sentence, and $C$ is a constituent occurrence 
of $\vec{x}$ in it. Thus, \eqref{eq:contset} defines the substitution 
classes of the exponents. We shall also define what it means 
to be syntactically indistinguishable in a sign system.
%%%
\begin{defn}
%%%
\index{$\simeq_{\Sigma}$}%%%
\index{exponents!syntactically indistinguishable}%%
%%%%
$e$ and $e'$ are \textbf{syntactically indistinguishable} --- we 
write $e \simeq_{\Sigma} e'$ --- iff 
%%
\begin{dingautolist}{192}
	\item
	for all $c \in C$ and all $m \in M$: if  
	$\auf e,c,m\zu\in \Sigma$ then there is
         an $m' \in M$ such that $\auf e',c,m'\zu \in \Sigma$ and
	\item
	for all $c \in C$ and all $m' \in M$: if  $\auf e',c,m'\zu\in \Sigma$
	 then there is an $m \in M$ such that  $\auf e,c,m\zu \in \Sigma$. 
	\end{dingautolist}
\end{defn}
%%%
This criterion defines which syntactic objects should belong 
to the same substitution category. Obviously, we can also use Husserl's 
criterion here. However, there is an intuition that certain sentences 
are semantically but not syntactically well formed. Although the distinction 
between syntactic and semantic well--formedness breaks the close 
connection between syntactic categories and context sets, it seems 
intuitively justified. Structural linguistics, following Zellig Harris
and others, typically defines categories in this way, using context sets.
We shall only assume here that categories may not distinguish syntactic
objects finer than the context sets. 
%%
\begin{defn}
%%%
\index{grammar!natural}%%
%%%
Let $\Sigma$ be a system of signs. A sign grammar $\GA$ that generates 
$\Sigma$ is \textbf{natural with respect to} {\tt S} if 
$\Cont_{\GA,\mbox{\smtt S}}(e) = \Cont_{\GA,\mbox{\smtt S}}(e')$ 
implies $e \simeq_{\Sigma} e'$.
\end{defn}
%%%
A context free sign grammar is natural iff the underlying CFG is 
reduced. Here is an example. Let 
%%
\begin{equation}
\Sigma := \{\auf \mbox{\mtt a}, \mbox{\mtt A}, 0\zu, 
\auf \mbox{\mtt b}, \mbox{\mtt B}, 0\zu, 
\auf \mbox{\mtt c}, \mbox{\mtt C}, 3\zu, 
\auf \mbox{\mtt ac}, \mbox{\mtt S}, 5\zu, 
\auf \mbox{\mtt bc}, \mbox{\mtt S}, 7\zu\}
\end{equation}
%%
Let $\GA$ be the sign grammar based on the following rules. 
%%
\begin{align}
\mbox{\mtt S} & \pf \mbox{\mtt AC} \mid \mbox{\mtt BC} &
\mbox{\mtt A} & \pf \mbox{\mtt a} \\\notag
\mbox{\mtt B} & \pf \mbox{\mtt b} &
\mbox{\mtt C} & \pf \mbox{\mtt c}  
\end{align}
%%
This corresponds to choosing five modes, $\mbox{\mtt F}_{\snull}$, 
$\mbox{\mtt F}_{\seins}$, $\mbox{\mtt F}_{\szwei}$ all unary, and
$\mbox{\mtt F}_{\sdrei}$, $\mbox{\mtt F}_{\svier}$ both binary.
%%%
\begin{align}
\notag
\mbox{\mtt F}_{\snull} & = \auf \mbox{\tt a}, \mbox{\tt A}, 0\zu \\ 
\mbox{\mtt F}_{\seins} & = \auf \mbox{\tt b}, \mbox{\tt B}, 0\zu \\ 
\notag
\mbox{\mtt F}_{\szwei} & = \auf \mbox{\tt c}, \mbox{\tt C}, 3\zu 
\end{align}
%%%
Further, $\mbox{\mtt F}_{\sdrei}^{\gamma}$ is a two place function 
defined only on $\auf \mbox{\tt A}, \mbox{\tt C}\zu$ with result 
{\mtt S}, $\mbox{\mtt F}_{\sdrei}^{\mu}$ a two place function 
defined only on $\auf 0, 3\zu$ with value $5$. Similarly, 
$\mbox{\mtt F}_{\svier}^{\gamma}(\mbox{\tt A}, \mbox{\tt C}) 
= \mbox{\mtt S}$, and is undefined elsewhere, and 
$\mbox{\mtt F}_{\svier}^{\mu}(0, 3) = 7$, and is undefined 
elsewhere. Then the only definite structure terms are 
{\mtt F$_{\snull}$}, {\mtt F$_{\seins}$}, {\mtt F$_{\szwei}$}, 
{\mtt F$_{\sdrei}$F$_{\snull}$F$_{\szwei}$}, 
and {\mtt F$_{\svier}$F$_{\seins}$F$_{\szwei}$}. Together they unfold 
to exactly $\Sigma$.

This grammar is, however, not natural. We have 
%%%
\begin{equation}
\Cont_{\GA, \mbox{\smtt S}}(\mbox{\mtt a}) = 
	\Cont_{\GA, \mbox{\smtt S}}(\mbox{\mtt b}) = 
\{\auf \varepsilon, \mbox{\mtt c}\zu\} 
\end{equation}
%%%
However, {\mtt a} and {\mtt b} do not have the same category.

Now look at the grammar $\GB$ based on the following rules: 
%%
\begin{align}
\mbox{\mtt S} & \pf \mbox{\mtt ac} \mid \mbox{\mtt BC} &
\mbox{\mtt A} & \pf \mbox{\mtt a} \\\notag
\mbox{\mtt B} & \pf \mbox{\mtt b} &
\mbox{\mtt C} & \pf \mbox{\mtt c}
\end{align}
%%
Here we compute that 
%%%
\begin{equation}
\Cont_{\GB, \mbox{\smtt S}}(\mbox{\mtt a}) = \varnothing \neq 
\{\auf \varepsilon, \mbox{\mtt c}\zu\} = 
\Cont_{\GB, \mbox{\smtt S}}(\mbox{\mtt b}) 
\end{equation}
%%%
Notice that in this grammar, {\mtt a} has no constituent occurrence 
in a sentence. Only {\mtt b} has. So, {\tt ac} is treated as an idiom.  
%%%
\begin{defn}
%%%
\index{sign system!strictly compositional}%%%
%%%
A vectorial system of signs is \textbf{strictly compositional} 
if there is a natural sign grammar for $\Sigma$ in which for 
every $f \in F$ the function $f^{\GZ}$ is a vector term which 
is stricly progressive with respect to the combined lengths of 
the strings.
\end{defn}
%%%
The definition of compositionality is approximately the one that
is used in the literature (modulo adaptation to systems of signs)
while the notion of strict compositionality is the one which we
think is the genuine notion reflecting the intuitions concerning 
compositionality. 

A particularly well--known case of noncompositionality in the strict
sense is the analysis of quantification by Montague.
%%%
\index{Montague, Richard}%%
%%%
%%%
\begin{align}
\label{ex:5613} & \mbox{\tt Nobody has seen Paul.} \\
\label{ex:5614} & \mbox{\tt No man has seen Paul.}
\end{align}
%%%
In the traditional, pre--Fregean understanding the subject of this
sentence is {\tt nobody} and the remainder of the sentence is the
predicate;  further, the predicate is predicated of the subject. 
Hence, it is said of nobody that he has seen Paul. Now, who is this 
`nobody'? Russell, 
%%%
\index{Russell, Bertrand}%%%
\index{Frege, Gottlob}%%%
%%%
following Frege's analysis has claimed that the 
syntactic structure is deceptive: the subject of this sentence is 
contrary to all expectations {\it not\/} the argument of its predicate. 
Many, including Montague, have endorsed that view. For them, the 
subject denotes a so--called {\it generalized quantifier}. 
Type theoretically the generalized quantifier of a subject has the 
type $e \pf (e \pf t)$. This is a set of properties, in this case the 
set of properties that are disjoint to the set of all humans. Now, 
{\tt has seen Paul} denotes a property, and \eqref{ex:5614} is true 
iff this property is in the set denoted by {\tt no human}, that
is to say, if it is disjoint with the set of all humans.

The development initiated by Montague 
%%%
\index{Montague, Richard}%%%
%%%
has given rise to a rich
literature. Generalised quantifiers have been a big issue for
semantics for quite some time (see \cite{keenanwesterstahl:generalized}). 
Similarly for the treatment of intensionality that he proposed. 
Montague systematically assigned intensional types as meanings, 
which allowed to treat world or situation dependencies. The 
general ideas were laid out in the
semiotic program and left room for numerous alternatives. This is
what we shall discuss here. However, first we scrutinize
Montague's analysis of quantifiers. The problem that he chose to
deal with was the ambiguity of sentences that were unambiguous
with respect to the type assignment. Instructive examples are the
following.
%%
\begin{align}
\label{ex:5615} & \mbox{\tt Some man loves every woman.} \\
\label{ex:5616} & \mbox{\tt Jan is looking for a unicorn.}
\end{align}
%%
Both sentences are ambiguous. \eqref{ex:5615} may say that there
is a man such that he loves all women. Or it may say that for
every woman there is a man who loves her. In the first reading the
universal quantifier is in the scope of the existential, in the
second reading the existential quantifier is in the scope of the
universal quantifier. Likewise, \eqref{ex:5616} may mean two
things. That there is a real unicorn and Jan is looking for it, or
Jan is looking for something that is in his opinion a unicorn.
Here we are dealing with scope relations between an existential quantifier
and a modal operator. We shall concentrate on example \eqref{ex:5615}. 
The problem with this sentence is that the universal quantifier {\tt
every woman} may not take scope over {\tt some man loves}. This is
so since the latter does not form a constituent. (Of course, we
may allow it to be a constituent, but then this option creates
problems of its own. In particular, this does not fit in with the
tight connections between the categories and the typing regime.)
So, if we insist on our analysis there is only one reading: the
universal quantifier is in the scope of the existential. Montague
%%%
\index{Montague, Richard}%%%
%%%
solved the problem by making natural language look more like 
predicate logic. He assumed an infinite set of pronouns called 
$\mbox{\tt he}_n$. These pronouns exist in inflected forms and 
in other genders as well, so we also have $\mbox{\tt him}_n$, 
$\mbox{\tt her}_n$ and so on. (We shall ignore gender and case 
inflection at this point.) The other reading is created as 
follows. We feed to the verb the pronoun $\mbox{\tt he}_{\snull}$ and 
get the constituent {\tt loves he$_{\snull}$}. This is an intransitive 
verb. Then we feed another pronoun, say $\mbox{\tt he}_{\seins}$ 
and get {\mtt he$_{\seins}$ loves he$_{\snull}$}. Next we combine this 
with {\tt every man} and then with {\tt a woman}. These operations 
substitute genuine phrases for these pronouns in the following 
way.  Assume that we have two signs:
%%
\begin{equation}
\sigma = \auf \vec{x}, e \backslash t, m\zu, \quad
    \sigma' = \auf \vec{y}, t, m'\zu 
\end{equation}
%%
Further, let $Q_n$ be the following function on signs:
%%
\begin{equation}
Q_n(\sigma, \sigma') := \auf \subst_n(\vec{x},\vec{y}),
    t, Q(m,\lambda x_n.m')\zu
\end{equation}
%%
Here $\subst_n(\vec{x},\vec{y})$ is defined as follows.
%%%
\begin{dingautolist}{192}
\item
For some $k$: $\vec{x} = \mbox{\tt he}_k$. Then $\subst_n(\vec{x},\vec{y})$
    is the result of replacing all occurrences of
    $\mbox{\tt he}_n$ by $\mbox{\tt he}_k$.
\item
For all $k$: $\vec{x} \neq \mbox{\tt he}_k$. Then $\subst_n(\vec{x}, \vec{y})$
    is the result of replacing the first occurrence of
    $\mbox{\tt he}_n$ by $\vec{x}$ and deleting the index
    $n$ on all other occurrences of $\mbox{\tt he}_n$.
\end{dingautolist}
%%
At last we have to give the signs for the pronouns. These are
%%
\begin{equation}
\mbox{\tt P}_n := \auf \mbox{\tt he}_n, \mbox{\tt NP}, \lambda
    x_{\CP}.x_{\CP}(x_n)\zu
\end{equation}
%%
Depending on case, $x_{\CP}$ is a variable of type $e \pf t$
(for nominative pronouns) or of type $e \pf (e \pf t)$ (for
accusative pronouns). Starting with the sign
%%
\begin{equation}
\mbox{\tt L} := \auf \mbox{\tt loves}, (e\backslash t)/e,
    \lambda x_0. \lambda x_1.
    \mbox{\sf love}'(x_1, x_0)\zu 
\end{equation}
%%
we get the sign
%%
\begin{align}
& \mbox{\mtt A}_{\sgr}\mbox{\mtt P}_{\seins}\mbox{\mtt A}_{\sgr}%
\mbox{\tt P}_{\snull}\mbox{\mtt L}  \\\notag
= & 
\auf \mbox{\mtt he$_{\seins}$ loves he$_{\snull}$}, t,
    \mbox{\sf loves}'(x_1,x_0))\zu
\end{align}
%%
Now we need the following additional signs.
%%
\begin{align}
\begin{split}
\mbox{\tt E}_n & := \auf \lambda x.\lambda y.%
\subst_n(\mbox{\tt every}\oconc x, y),
    (t/t)/(e\backslash t), \\
	& \qquad\qquad \lambda x. \lambda y.\forall x_n.
    (x(x_n) \pf y)\zu 
\end{split} \\
\begin{split}
\mbox{\tt S}_n & := \auf \lambda x.\lambda y.%
\subst_n(\mbox{\tt some}\oconc x, y), (t/t)/(e\backslash t), \\
	& \qquad\qquad \lambda x. \lambda y.\exists x_n.
    (x(x_n) \und y)\zu 
\end{split} \\
\mbox{\tt M} & := \auf \mbox{\tt man}, e\backslash t,
    \lambda x.\mbox{\sf man}'(x)\zu \\
\mbox{\tt W} & := \auf \mbox{\tt W}, e\backslash t,
    \lambda x.\mbox{\sf woman}'(x)\zu
\end{align}
%%
If we feed the existential quantifier first we get the reading
$\forall\exists$, and if we feed the universal quantifier first
we get the reading $\exists\forall$. The structure terms are as 
follows.
%%
\begin{align}
& \mbox{\mtt A$_{\sgr}$A$_{\sgr}$E$_{\snull}$WA$_{\sgr}$A$_{\sgr}$S$%
_{\seins}$MA$_{\sgr}$P$_{\snull}$A$_{\sgr}$P$_{\seins}$L} \\
& \mbox{\tt A$_{\sgr}$A$_{\sgr}$S$_{\seins}$MA$_{\sgr}$A$_{\sgr}$%
E$_{\snull}$WA$_{\sgr}$P$_{\snull}$A$_{\sgr}$P$_{\seins}$L} 
\end{align}
%%
We have not looked at the morphological realization of the phrase
$\vec{x}$. Number and gender must be inserted with the
substitution. So, the case is determined by the local context, the
other features are not. We shall not go into this here. (Montague
%%%
\index{Montague, Richard}%%%
%%%
had nothing to say about morphology as English has very little. We
can only speculate what would have been the case if Montague had
spoken, say, an inflecting language.) Notice that the present analysis 
makes quantifiers into sentence adjuncts. 

Recall from Section~\ref{kap2}.\ref{kap2-6} that the grammar of sentences 
is very complex. Hence, since Montague defined the meaning of
%%%
\index{Montague, Richard}%%%
%%%
sentences to be closed formulae, it is almost unavoidable that
something had to be sacrificed. In fact, the given analysis violates
several of our basic principles. First, there are infinitely
many lexical elements. Second, the syntactic structure is not
respected by the translation algorithm, and this yields the wrong
results. Rather than taking an example from a different language,
we shall exemplify the problems with the genitive pronouns. We
consider {\tt his} as the surface realization of $\mbox{\tt
him's}$, where {\tt 's} is the so--called Anglo Saxon genitive.
Look for example at \eqref{ex:5717}, which resulted from
\eqref{ex:5718}. Application of the above rules gives
\eqref{ex:5719}, however.
%%
\begin{align}
\label{ex:5717} & \mbox{\tt with a hat on his head every man looks better} \\
\label{ex:5718} & \subst_{\snull}(\mbox{\tt every man}, 
\mbox{\mtt with a hat on him$_{\snull}$'s head %
 he$_{\snull}$} \\\notag
& \quad \mbox{\mtt looks better}) \\
\label{ex:5719} & \mbox{\tt with a hat on every man's head
    he looks better}
\end{align}
%%
This happens not because the possessive pronouns are also part of the
rules: of course, they have to be part of the semantic algorithm.
It is because the wrong occurrence of the pronoun is being
replaced by the quantifier phrase. This is due to the fact that
the algorithm is ignorant about the syntactic structure (which
the string reveals only partly) and second because the algorithm
is order sensitive at places where it should better not be. 
See Section~\ref{kap5}.\ref{kap5-5} on GB, a theory that has 
concerned itself extensively with the question which NP may be a 
pronoun, or a reflexive pronoun or empty. Fiengo and May 
%%%
\index{May, Robert}%%%
\index{Fiengo, Robert}%%%
%%%
\shortcite{fiengomay:indices} speak quite plastically of 
{\it vehicle change}, to name the phenomenon that a variable appears 
sometimes as a pronoun, sometimes as {\rm pro} (an empty pronoun, see 
Section~\ref{kap5}.\ref{kap5-5}),
sometimes as a lexical NP and so on. The synchronicity between
surface structure and derivational history which has been
required in the subsequent categorial grammar, is not found
with Montague. 
%%%
\index{Montague, Richard}%%%
%%%
He uses instead a distinction proposed by Church 
%%%
\index{Church, Alonzo}%%%
%%%
between \textbf{tectogrammar} (the inner structure,
%%%
\index{tectogrammar}%%%
%%%
as von Humboldt would have called it) and 
%%%
\index{von Humboldt, Alexander}%%%
%%%
\index{phenogrammar}%%%
%%%
\textbf{phenogrammar} (the outer structure, which is simply what we 
see). Montague admits quite powerful phenogrammatical operations, 
and it seems as if only the label distinguishes him from GB theory. 
For in principle his maps could be interpreted as transformations. 

We shall briefly discuss the problems of blocking and other apparent
failures of compositionality. In principle, we have allowed
the exponent functions to be partial. They can refuse to operate
on certain items. This may be used in the analysis
of defective words, for example {\tt courage}. This word exists
only in the singular (though it arguably also has a plural meaning). 
There is no form {\tt courages}. In morphology, one says that each 
word has a root; in this case the root may simply be {\tt courage}.
The singular is formed by adding $\varepsilon$, the plural by adding
{\tt s}. The word {\tt courage} does not let the plural be formed. 
It is defective. If that is so, we are in trouble with Leibniz' 
%%%
\index{Leibniz' Principle}%%%
%%%
Principle.  Suppose we have a word $X$ that is synonymous with 
{\tt courage} but exists in the singular and the plural (or only 
in the plural like {\tt guts}). Then, by Leibniz' Principle, the two roots 
can never have the same meaning, since it is not possible to exchange 
them for each other in all contexts (the context where $X$ appears in 
the plural is a case in point). To avoid this, we must actually assume 
that there is no root form of {\tt courage}.
%%%
\index{singulare tantum}%%
%%%
The classical grammar calls it a \textbf{singulare tantum}, a 
`singular only'. This is actually more appropriate. If namely this
word has no root and exists only as a singular form, one simply
cannot exchange the root by another. We remark here that English
has \textbf{pluralia tanta} (`plural only' nouns), for example
{\tt troops}. In Latin, {\tt tenebrae} `darkness', {\tt
indutiae} `cease fire' are examples. Additionally, there are 
words which are only formwise derived from the singular
counterpart (or the root, for that matter). One such example is
{\tt forces} in its meaning `troops', in Latin {\tt fortunae} 
`assets', whose singular {\tt fortuna} means `luck'. Again, if
both forms are assumed to be derived from the root, we have
problems with the meaning of the plural. Hence, some of these forms
(typically --- but not always --- the plural form) will have to 
be part of the lexicon (that is, it constitutes a 0--ary mode).

Once we have restricted the admissible functions on exponents, 
we can show that weak and strong generative capacity do not 
necessarily coincide. Recall the facts from Exercise~\ref{ex:chinese}, 
taken from \cite{radzinski:copying}. In Mandarin yes--no--questions are 
formed by iterating the statement with the negation word in between. 
Although it is conceivable that
%%%
\index{Mandarin}%%
%%%
Mandarin is context free as a string language, Radzinski 
%%%
\index{Radzinski, Daniel}%%%
%%%
argues that it is not strongly context free. Now, suppose we understand
by strongly context free that there is a context free sign grammar.
Then we shall show that under mild conditions Mandarin is not
strongly context free. To simplify the dicussion, we shall define
a somewhat artificial counterpart of Mandarin. Start with a context 
free language $G$ and a meaning function $\mu$ defined on $G$. Then 
put $M := G \cup G \oconc \mbox{\tt bu}\oconc G$.
Further, put
%%
\begin{equation}
\nu(\vec{x}\oconc\mbox{\tt bu}\oconc\vec{y}) :=
    \begin{cases}
    \mu(\vec{x}) \oder \nicht \mu(\vec{y}) &
        \text{ if $\vec{x} \neq \vec{y}$,} \\
    \mu(\vec{x})? & \text{ if $\vec{x} = \vec{y}$.}
    \end{cases}
\end{equation}
%%%
Here, $?$ forms questions. We only need to assume that it is 
injective on the set $\mu[G]$ and that $?[\mu[G]]$ is disjoint 
from $\{\mu(\vec{x}) \oder \mu(\vec{y}) : \vec{x}, \vec{y} \in 
L(G)\}$. (This is the case in Mandarin.) Assume that
there are two distinct expressions $\vec{u}$ and $\vec{v}$ of 
equal category in $G$ such that $\mu(\vec{u}) = \mu(\vec{v})$. 
Then they can be substituted for each other. Now suppose that 
$G$ has a sublanguage of the form $\{\vec{r}\, {\vec{z}\,}^i\, \vec{s} :
i \in \omega\}$ such that $\mu(\vec{r}\, {\vec{z}\,}^i\, \vec{s}) =
\mu(\vec{r}\, {\vec{z}\,}^j\, \vec{s})$ for all $i, j$. We claim 
that $M$ together with $\nu$ is not context free. Suppose otherwise. 
Then we have a context free grammar $H$ together with a meaning 
function that generates it. By the Pumping Lemma, there is a $k$ 
such that ${\vec{z}\,}^k$ can be adjoined into some 
$\vec{r}\, {\vec{z}\,}^i\, \vec{s}$ any number of times. (This is 
left as an exercise.) Now look at the expressions
%%e
\begin{equation}
\vec{r}\conc {\vec{z}\,}^i\conc\vec{s}\oconc\mbox{\tt bu}\oconc
\vec{r}\conc{\vec{z}\,}^{j} \conc\vec{s}
\end{equation}
%%%
Adjunction is the result of substitution. However, the $\nu$--meaning 
of these expressions is $\top$ if $i \neq j$ and a yes--no question
if $i = j$. Now put $j = i+k$. If we adjoin ${\vec{z}\,}^k$ on the
left side, we get a yes--no question, if we substitute it to the
right, we do not change the meaning, so we do not get a yes--no question.
It follows that one and the same syntactic substitution operation
defines two different semantic functions, depending on where it is
performed. Contradiction. Hence this language is not strongly context 
free. It is likely that Mandarin satisfies the additional assumptions. 
For example, colour words are extensional. So, {\tt blue shirt} 
means the same as {\tt blue blue shirt}, {\tt blue blue blue shirt}, 
and so on.

%%%
\index{Bahasa Indonesia}%%
%%%
Next we look at Bahasa Indonesia. Recall that it forms the plural by
reduplication. If the lexicon is finite, we can still generate the
set of plural expressions. However, we must assume a distinct
syntactic category for each noun. This is clearly unsatisfactory.
For every time the lexicon grows by another noun, we must add a
few rules to the grammar (see \cite{manasterramer:copying}). However, 
let us grant this point. Suppose, we have two nouns, $\vec{m}$ and
$\vec{n}$, which have identical meaning. If there is no syntactic 
or mophological blocking, by Leibniz' Principle any constituent 
%%%
\index{Leibniz' Principle}%%%
%%%
occurrence of the first can be substituted by the second and vice 
versa. Therefore, if $\vec{m}$ has two constituent
occurrences in $\vec{m}\mbox{\tt -}\vec{m}$, we must have a word
$\vec{m}\mbox{\tt -}\vec{n}$ and a word $\vec{n}\mbox{\tt
-}\vec{m}$, and both mean the same as the first. This is precisely
what is not the case. Hence, no such pair of words can exist if
Bahasa Indonesia is strongly context free. This argument relies 
on a stronger version of Leibniz' Principle: that semantic identity 
%%%
\index{Leibniz' Principle}%%%
%%%
enforces substitutability {\it tout court}. Notice that our
previous discussion of context sets does not help here. The noun
$\vec{m}$ has a different context set as the noun $\vec{n}$, since
it occurs in a plural noun $\vec{m}\mbox{\tt -}\vec{m}$, where
$\vec{n}$ does not occur. However, notice that the context set of
$\vec{m}$ contains occurrences of $\vec{m}$ itself. If that
circularity is removed, $\vec{m}$ and $\vec{n}$ become
indistinguishable.

These example might suffice to demonstrate that the relationship 
between syntactic structure and semantics is loose but not entirely 
free. One should be extremely careful, though, of hidden assumptions. 
Many arguments in the literature showing that this or that 
language is not strongly context free rest on particular assumptions 
that are not made explicit. 

{\it Notes on this section.} The idea that syntactic operations should 
more or less be restricted to concatenation give or take some minor 
manipulations is advocated for in \cite{hausser:surface}, who calls 
this {\it surface compositionality}. Hausser also noted that Montague 
%%%
\index{Hausser, Roland}%%%
\index{Montague, Richard}%%%
%%%
did not actually define a surface compositional grammar. Most present 
day categorial grammars are, however, surface compositional.  
%%%
\vplatz
\exercise
Suppose that $A = \{\mbox{\tt 0}, \mbox{\tt 1}, \dotsc, \mbox{\tt 9}\}$,
with the following modes.
%%
\begin{align}
\mbox{\tt O} & := \auf \mbox{\tt 0}, \mbox{\tt Z}, 0\zu \\
\mbox{\tt S}(\auf \vec{x}, \mbox{\tt Z}, n\zu) &
    := \auf \Suc(\vec{x}), \mbox{\tt Z}, n+1\zu
\end{align}
%%
Here, $\Suc(\vec{x})$ denotes the successor of $\vec{x}$ in the decimal
notation, for example, $\Suc(\mbox{\tt 19}) = \mbox{\tt 20}$. Let a
string be given. What does a derivation of that string look like?
When does a sign $\sigma$ occur in another sign $\tau$? Describe
the exponent of $[\Gs'/\Gs]\Gt$, for given structure terms
$\Gs$, $\Gs'$, $\Gt$. Define a progress measure for which this 
grammar is progressive.
%%
\vplatz
\exercise
Let $A := \{\mbox{\mtt 0}, \dotsc, \mbox{\mtt 9}, \mbox{\mtt +},
\mbox{\mtt (}, \mbox{\mtt )}, \mbox{\mtt\symbol{42}}\}$. We shall 
present two ways for generating ordinary arithmetical terms. Recall that
there is a convention to drop brackets in the following
circumstances. (a) When the same operation symbol is used in
succession ({\mtt 5+7+4} in place of {\mtt (5+(7+4))}),
(b) when the enclosed term is multiplicative ({\mtt 3\symbol{42}4+3}
in place of {\mtt (3\symbol{42}4)+3}). Moreover, (c) the outermost
brackets are dropped. Write a sign grammar that generates triples
$\auf \vec{x}, T, n\zu$, where $\vec{x}$ is a term and $n$ its
value, where the conventions (a), (b) and (c) are optionally used.
(So you should generate $\auf \mbox{\mtt 5+7+4}, T, 16\zu$ as well as
$\auf \mbox{\mtt (5+(7+4))}, T, 16\zu$). Now apply Leibniz' Principle
%%%
\index{Leibniz' Principle}%%%
%%%
to the pairs {\mtt (5+7)} and {\mtt 5+7}, {\tt 5+(7+4)} and
{\mtt 5+7+4}. What problems arise? Can you suggest a solution?
%%%
\vplatz
\exercise
(Continuing the previous exercise.) Write a grammar that treats
every accidental occurrence of a term as a constituent occurrence
in some different parse. For example, the occurrence of {\tt 3+4}
in {\tt 3+4\symbol{42}7} is in the grammar of the
previous exercise a nonconstituent occurrence, now however it shall
be a constituent occurrence under some parse. Apply Leibniz'
%%%
\index{Leibniz' Principle}%%%
%%%
Principle. Show that {\tt 5+7} is not identical to {\tt 7+5},
and {\tt 2+5+7} is not identical to {\tt 2+7+5} and so on. 
Which additive terms without brackets are identical 
in meaning, by Leibniz Principle?
%%%
\vplatz
\exercise
%%%
\index{Latin}%%
%%
The Latin verbs {\tt aio} and  {\tt inquam} (`I say') are highly
defective. They exist only in the present. Apart from one or two more
forms (which we shall ignore for simplicity), Figure~\ref{fig:aio}
gives a synopsis of what forms exist of these verbs and contrast 
them with the forms of {\tt dico}.
%%
\begin{figure}
\begin{center}
\begin{tabular}{|l|l|l||l|}
\hline
{\tt aio} & {\tt inquam} & {\tt dico} & `I say' \\
{\tt ais} & {\tt inquis} & {\tt dicis} & `you(sg) say' \\
{\tt ait} & {\tt inquit} & {\tt dicit} & `he says' \\
          &              & {\tt dicimus} & `we say' \\
          &              & {\tt dicitis} & `you(pl) say' \\
{\tt aiunt} & {\tt inquiunt} & {\tt dicunt} & `they say' \\\hline
\end{tabular}
\end{center}
\caption{Latin Verbs of Saying}
\label{fig:aio}
\end{figure}
%%
The morphology of {\tt inquam} is irregular in that form (we 
expect {\tt inquo}); also the syntax of {\tt inquit} is
somewhat peculiar (it is used parenthetically).  Discuss whether
{\tt inquit} and {\tt dicit} can be identical in meaning by
Leibniz' Principle or not. Further, the verb {\tt memini} is
%%%
\index{Leibniz' Principle}%%%
%%%
formwise in the perfect, but it means `I remember'; similarly {\tt
odi} `I hate'.

 \section{de Saussure Grammars}
\label{kap4-8}
%
%
%
In his famous Cours de Linguistique G\'en\'erale, de Saussure 
%%%
\index{de Saussure, Ferdinand}%%%
%%%
speaks about linguistic signs and the nature of language as a system of
signs. In his view, a sign is constituted by two elements: 
its \textbf{signifier}
%%%
\index{signifier}%%
\index{signified}%%
%%%
and its \textbf{signified}. In our terms, these are the exponent and the
meaning, respectively. Moreover, de Saussure says that signifiers
are {\it linear}, without further specifying what he means by that.
To a modern linguist all this seems obviously false: there are
categories, and linguistic objects are structured, they are not
linear. Notably Chomsky 
%%%
\index{Chomsky, Noam}%%%
%%%
has repeatedly offered arguments to support
this view. He believed that structuralism was fundamentally mistaken.
In this section we shall show that the rejection of de Saussure's ideas 
is ill--founded. To make the point, we shall look at a few recalcitrant 
syntactic phenomena and show how they can be dealt with using totally 
string based notions. 

Let us return to the idea mentioned earlier, that of
$\lambda$--terms on strings. We call a \textbf{string term}
%%%
\index{string term}%%
%%%%
a $\lambda$--term over the algebra of strings (consisting of constants 
for every $a \in A$, $\varepsilon$, and $\conc$). We assume here that
strings are typed, and that we have strings of different type. Assume
for the moment that there is only one type, that of a string, denoted
by $s$. Then $\lambda x.\lambda y. y \conc x$ is the function of
reverse concatenation, and it is of type $s \pf (s \pf s)$. Now we
wish to implement restrictions on these terms that make sure we
do not lose any material. Call a $\lambda$--term 
%%%
\index{$\lambda$--term!relevant}%%
%%%
\textbf{relevant} if for all subterms $\lambda x.N$, $x$ occurs at 
least once free in $N$. $\lambda x.\lambda y. y \oconc x$ is relevant, 
$\lambda x.  \lambda y. x$ is not. Clearly, relevance is a necessary 
restriction. However, it is not sufficient. Let $\CP$ and $\CQ$ be variables 
of type $s \pf s$, $x$ a variable of type $x$. Then function composition,
$\lambda \CP.\lambda \CQ. \lambda x.\CP(\CQ(x))$, is a  relevant
$\lambda$--term. But this is problematic. Applying this term leaves
no visible trace on the string, it just changes the analysis.
Thus, we shall also exclude {\it combinators}. This means, an
admissible $\lambda$--term is a relevant term that contains
$\conc$ or an occurrence of a constant at least once.
%%%
\begin{defn}
%%%
\index{string term!weakly progressive}%%%
\index{string term!progressive}%%%
%%%
A string term $\tau$ is \textbf{weakly progressive} if it is relevant 
and not a combinator. $\tau$ is \textbf{progressive} if it is weakly 
progressive and does not contain $\varepsilon$.
\end{defn}
%%%
\begin{defn}
%%%
\index{sign!de Saussure}%%%
\index{grammar!de Saussure}%%%
\index{de Saussure sign}%%%
\index{de Saussure grammar}%%%
%%%
A \textbf{de Saussure sign} or simply \textbf{dS--sign} is a pair
$\delta = \auf e,m\zu$, where $e$ is a progressive string term
and $m$ a $\lambda$--term over meanings. The \textbf{type} of
$\delta$ is the pair $\auf \sigma, \tau\zu$, where $\sigma$ is the
type of $e$ and $\tau$ the type of $m$. If $\delta' =
\auf e', m'\zu$ is another de Saussure sign then $\delta(\delta')$
is defined iff $ee'$ is defined and $mm'$ is defined,
and then
%%
\begin{equation}
\delta(\delta') := \auf ee', mm'\zu
\end{equation}
%%
\index{functor sign}%%
\index{argument sign}%%
%%%
In this situation we call $\delta$ the \textbf{functor sign} and
$\delta'$ the \textbf{argument sign}. A \textbf{de Saussure grammar}
is a finite set of dS--signs.
\end{defn}
%%
So, the typing regime of the strings and the typing regime of the
meanings do all the work here.
%%%
\begin{prop}
Let $\delta$ and $\delta'$ be dS--signs of type $\auf \sigma, \tau\zu$
and $\auf \sigma',\tau'\zu$, respectively. Then $\delta(\delta')$
is defined iff there are $\mu$, $\nu$ such that
$\sigma = \sigma' \pf \mu$, $\tau = \tau' \pf \nu$, and then
$\delta(\delta')$ has type $\auf \mu, \nu\zu$.
\end{prop}
%%%
The rest is actually the same as in AB--grammars. Before we 
shall prove any results, we shall
comment on the definition itself. In Montague Grammar and much
of Categorial Grammar there is a conflation of information
that belongs to the realm of meaning and information that belongs
to the realm of exponents. The category $\beta/\alpha$, for example,
tells us that the meaning must be a function of type $\sigma(\alpha) 
\pf \sigma(\beta)$, and that the exponent giving us the argument 
must be found to the right. $\alpha \backslash \beta$, is different
only in that the exponent is to be found to the left. While this
seems to be reasonable at first sight, it is already apparent
that the syntactic categories simply elaborate the semantic types.
(This is why $\sigma$ is a homomorphism.) The information
concerning the semantic types is however not necessary, since
the merger would fail anyhow if we did not supply signs with
the correct types. So, we could leave it to syntax to specify only
the directionality. However, syntax is not well equipped for that.
There are discontinuous constituents and this is not easily
accommodated in categorial grammar. Much of the research can be
seen as an attempt to upgrade the string handling potential in
this direction. Notice further that the original categorial
apparatus created distinctions that are nowhere attested.
For example, adjectives in English are of category $n/n$. In 
order to modify a relational noun, however, they must be
lifted to the category of a relational noun. The lifting will have
to specify whether the noun is looking for its complement on its
right or on its left. Generally, however, modifiers and functors
do not care very much about the makeup of their arguments. 
However, in \textsf{AB} and \textsf{L}, categories must be 
explicit about these details.

De Saussure grammars do away with some of the problems that beset 
CGs. They do not require to iterate the semantic types in the category, 
and the string handling has more power than in standard categorial 
grammar. We shall discuss a few applications of de Saussure grammars. 
These will illustrate both the strength as well as certain 
deficiencies.

A striking fact about de Saussure grammars is that they allow
for word order variation in the most direct way. Let us take a
transitive verb, {\tt see}, with meaning 
$\mbox{\sf see}' = \lambda x. \lambda y.
\mbox{\sf see}'(y)(x)$. Its first argument is the direct object
and the second its subject. We assume no case marking, so that
the following nouns will be either subject or object.
%%
\begin{align}
\mbox{\sc john} & := \auf \mbox{\tt John}, \mbox{\sf john}'\zu &
\mbox{\sc mary} & := \auf \mbox{\tt Mary}, \mbox{\sf mary}'\zu 
\end{align}
%%
Now we can give to the verb one of the following six signs.
of which each corresponds to a different word order pattern.
Recall that $x \oconc y = x \conc \square \conc y$.
%%
\begin{align}
\mbox{\sc sees}_0 & := \auf \lambda x.\lambda y.
    y \oconc x \oconc \mbox{\tt sees}, \mbox{\sf see}'\zu
        && \mbox{\rm SOV} \\
\mbox{\sc sees}_1 & := \auf \lambda x.\lambda y.
    y \oconc \mbox{\tt sees} \oconc x, \mbox{\sf see}'\zu
        && \mbox{\rm SVO} \\
\mbox{\sc sees}_2 & := \auf \lambda x.\lambda y.
    \mbox{\tt sees} \oconc y \oconc x, \mbox{\sf see}'\zu
        && \mbox{\rm VSO} \\
\mbox{\sc sees}_3 & := \auf \lambda x.\lambda y.
    x \oconc y \oconc \mbox{\tt sees}, \mbox{\sf see}'\zu
        && \mbox{\rm OSV} \\
\mbox{\sc sees}_4 & := \auf \lambda x.\lambda y.
    x \oconc \mbox{\tt sees} \oconc y, \mbox{\sf see}'\zu
        && \mbox{\rm OVS} \\
\mbox{\sc sees}_5 & := \auf \lambda x.\lambda y.
    \mbox{\tt sees}\oconc x \oconc y, \mbox{\sf see}'\zu
        && \mbox{\rm VOS}
\end{align}
%%
The structure term for a basic sentence expressing that John sees
Mary is in all cases the same. (Structure terms will be written 
using brackets, to avoid confusion. The convention is that bracketing 
is left--associative.) It is $\mbox{\sc sees}_i(\mbox{\sc mary})%
(\mbox{\sc john})$, $i < 6$. Only that the order of the words
is different in each case. For example,
%%%
\begin{align}
 & \mbox{\sc sees}_0(\mbox{\sc mary})(\mbox{\sc john}) \\\notag
 = & \auf \lambda x.\lambda y.
    y \oconc x \oconc \mbox{\tt sees}, \mbox{\sf see}'\zu
(\auf \mbox{\tt Mary}, \mbox{\sf mary}'\zu)
(\auf \mbox{\tt John}, \mbox{\sf john}'\zu) \\\notag
 = & \auf \lambda y. y \oconc \mbox{\tt Mary sees},
    \mbox{\sf see}'(\mbox{\sf mary}')\zu
    (\auf \mbox{\tt John}, \mbox{\sf john}'\zu) \\\notag
 = & \auf \mbox{\tt John Mary sees}, \mbox{\sf see}'(\mbox{\sf %
    mary}')(\mbox{\sf john}')\zu \\
 & \mbox{\sc sees}_4(\mbox{\sc mary})(\mbox{\sc john}) \\\notag
 = & \auf \lambda x.\lambda y.
    x \oconc \mbox{\tt sees} \oconc y, \mbox{\sf see}'\zu
(\auf \mbox{\tt Mary}, \mbox{\sf mary}'\zu)
(\auf \mbox{\tt John}, \mbox{\sf john}'\zu) \\\notag
 = & \auf \lambda y. \mbox{\tt Mary sees} \oconc
    y, \mbox{\sf see}'(\mbox{\sf mary}')\zu
    (\auf \mbox{\tt John}, \mbox{\sf john}'\zu) \\\notag
 = & \auf \mbox{\tt Mary sees John}, \mbox{\sf see}'(\mbox{\sf %
    mary}')(\mbox{\sf john}')\zu
\end{align}
%%
Notice that this construction can be applied to heads in general,
and to heads with any number of arguments. Thus, de Saussure
grammars are more at ease with word order variation than categorial
grammars. Moreover, in the case of OSV word order we see that the
dependencies are actually crossing, since the verb does not form
a constituent together with its subject.

We have seen in Section~\ref{kap4}.\ref{kap4-3} how interpreted LMGs 
can be transformed into AB--grammars using vector
polynomials. Evidently, if we avail ourselves of vector polynomials
(for example by introducing pair formation and projections and
redefining the notion of progressivity accordingly) this result
can be reproduced here for de Saussure grammars. Thus, de Saussure
grammars suitably generalized are as strong as interpreted LMGs.
However, we shall actually not follow this path. We shall not use
pair formation; instead, we shall stay with the more basic apparatus.
The examples that we shall provide below will give evidence that
this much power is actually sufficient for natural languages, though
some modifications will have to be made.

Next we shall look at plural in Bahasa Indonesia (or Malay).
%%%
\index{Bahasa Indonesia}%%
\index{Malay}%%
%%%
The plural is formed by reduplicating the noun. For example,
the plural of {\tt orang} `man' is {\tt orang-orang},
the plural of {\tt anak} `child' is {\tt anak-anak}. To
model this, we assume one type of strings, $n$.
%%
\begin{equation}
\mbox{\sc plu} := \auf \lambda x. x\conc\mbox{\tt -}\conc x,
    \lambda \CP. \{x : \CP(x)\}\zu
\end{equation}
%%
The term $\lambda x.x\conc\mbox{\tt -}\conc x$ is progressive.
The plural operation can in principle be iterated; we shall see 
below how this can be handled. (We see no obvious semantical 
reason why it cannot, so it must be blocked morphologically.) 
Now let us turn to English. In English, the
%%%%
\index{English}%%
%%%%
plural is formed by adding an {\tt s}. However, some
morphophonological processes apply, and some nouns form their
plural irregularly. Table~\ref{tab:engplu} gives an (incomplete) 
list of plural formation. Above the line we find regular plurals, 
below irregular plurals.
%%
\begin{table}
\caption{Plural in English}
\label{tab:engplu}
\begin{center}
\begin{tabular}{|l|l|l|}
\hline
Singular & Plural & \\\hline
{\tt tree} & {\tt trees} & plain suffix \\
{\tt bush} & {\tt bushes} & e-insertion \\\hline
{\tt ox}   & {\tt oxen} & en--suffix \\
{\tt fish} & {\tt fish} & no change \\
{\tt man}  & {\tt men}  & vowel change \\\hline
\end{tabular}
\end{center}
\end{table}
%%
As we have outlined in Section~\ref{kap1}.\ref{kap1-3}, these differences are
explained by postulating different plural morphs, one for each noun
class. We can account for that by introducing noun class distinctions
in the semantic types. For example, we may introduce a semantic type
for nouns endings in a nonsibilant, another for nouns ending in a
sibilant, and so on. However, apart from introducing the distinction
where it obviously does not belong, this proposal has another drawback.
Recall, namely, that linguists speak of a plural {\it morpheme}, which
abstracts away from the particular realizations of plural formation.
Mel'\v{c}uk 
%%%
\index{Mel'\v{c}uk, Igor}%%%
%%%
defines a morpheme as a set of signs that have identical
category and identical meaning. So, for him the plural morpheme is
simply the set of plural morphs. Now, suppose that we want the
morpheme to be a (de Saussure) {\it sign}. Then its meaning is that 
of any of its morphs, but the string function cannot be a 
$\lambda$--term. For it may act differently on identical strings of 
different noun class. A good example is German {\tt Bank}. 
%%%
\index{German}%%%
%%%
Like its English counterpart it can denote (i) a money institute, (ii) 
something to sit on, (iii) the bank of a river. However, in the first case 
its plural is {\tt Banken} and in the other two it is {\tt B\"anke}. Now, 
since the function forming the plural cannot access the meaning we 
must distinguish two different string classes, one for nouns that 
form the plural by umlaut plus added {\tt e}, and the other for 
nouns that form the plural by adding {\tt en}. Further, we shall 
assume that German {\tt Bank} is in both, but with different meanings.
Thus, we have two signs with exponent {\tt Bank}, one to mean money
institute and the other to mean something to sit on or the bank of
a river. This is the common practice. The classes are morphological,
that is, they do not pertain to meaning, just to form.

Thus we are led to the introduction of string types. We assume that
types are ordered by some partial ordering $\leq$, so that if
$\alpha$ and $\beta$ are string types and $\alpha \leq \beta$ then
any string of type $\alpha$ is a string of type $\beta$. Moreover, 
we put $\alpha \pf \beta \leq \alpha' \pf \beta'$ iff 
$\alpha \leq \alpha'$ and $\beta \leq \beta'$. No other relations 
hold between nonbasic types. The basic type $s$ is the largest basic 
type. Returning now to English, we shall split the type $n$ into 
various subtypes. In particular, we need the types {\it ni}, {\it nr}, 
of irregular and regular nouns.  We shall first treat the regular 
nouns. The rule is that if a noun ends in a sibilant, the vowel 
{\tt e} is inserted, otherwise not.  Since this is a completely 
regular phenomenon, we can only define the string function if we 
have a predicate {\sf sib} that is true of a string iff 
it ends in a sibilant. Further, we need to be able to define a 
function by cases. 
%%
\begin{equation}
\mbox{\sf rplu} := \lambda x.\mbox{\sf if}\; \mbox{\sf sib}(x)\; 
	\mbox{\sf then} \; x\conc\mbox{\tt es}\; 
	\mbox{\sf else} \; x \conc \mbox{\tt s}\; \mbox{\sf fi;}
\end{equation}
%%
Thus, we must have a basic type of booleans plus some functions.
We shall not spell out the details here. Notice that definitions 
by cases are not necessarily unique, so they have to be used with 
care. Notice a further problem. The minute that we admit different 
types we have to be specific about the type of the resulting string. 
This is not an innocent matter. The operation $\conc$ is defined 
on all strings. Suppose now that {\tt tree} is a string of type 
{\it nr}, which type does {\tt trees} have? Obviously, we do not 
want it to be just a string, and we may not want it to be of type 
{\it nr\/} again. (The difference between regular and irregular is 
needed only for plural formation.) Also, as we shall see below, there 
are operations that simply change the type of a string without changing
the string itself. Hence we shall move from a system of implicit
typing to one of explicit typing (see \cite{mitchell:type} for an 
overview). Rather than using variables for each type, we use a single 
set of variables. $\lambda$--abstraction is now written 
$\lambda x:\sigma.M : \tau$ in place of $\lambda x.M$. Here $x$  
must be a variable of type $\sigma$, and the result will be of 
type $\tau$. Thus, $\lambda x:\mbox{\it nr}.x \conc \mbox{\tt s} %
: \mbox{\it n}$ denotes the function that turns and {\it nr\/}--string 
into an {\it n\/}--string by appending {\tt s}. The reader may recall 
from Section~\ref{kap6}.\ref{kap6-1} the idea that strings can be taken 
to mean different things depending on what type they are paired with. 
Internally, a typed string term is represented by $\auf N, \sigma\zu$, 
where $N$ is the string term and $\sigma$ its type. The operation 
$M : \tau$ does the following: it evaluates $M$ on $N$, and gives it 
the type $\tau$.  Now, the function is also defined for all 
$\sigma' \leq \sigma$, so we finally have
%%
\begin{equation}
(\lambda x:\sigma.M:\tau)(N : \sigma')
    = \begin{cases}
    [N/x]M : \tau & \text{ if $\sigma' \leq
        \sigma$,} \\
    \ast & \text{ otherwise.}
    \end{cases}
\end{equation}
%%
Now we turn to the irregular plural. Here we face two choices.
We may simply take all singular and plural nouns as being in the
lexicon; or we devise rules for all occurring subcases. The
first is not a good idea since it does not allow us to say that
{\tt ox} and the plural morpheme actually occur in {\tt oxen}.
The sign is namely an unanalyzable unit. So we discard the first
alternative and turn to the second. In order to implement the
plural we again need a predicate of strings that tells us whether
a string equals some given string. The minimum we have to do is
to introduce an equality predicate on strings. This allows to define
the plural by cases. However, suppose we add a binary predicate
$\mbox{\sf suf}(x,y)$ which is true of $x$ and $y$ iff
$x$ is a suffix of $y$. Then the regular plural can be defined
also as follows:
%%
\begin{align}
\mbox{\sf rplu} := 
	& \lambda x : \mbox{\it nr}.\; \mbox{\sf if } %
	(\mbox{\sf suf}(\mbox{\tt s}, x)\;  
	        \mbox{\sf or}\; \mbox{\sf suf}(\mbox{\tt sh}, x)) \\\notag
	& \quad
     		\mbox{\sf then} \;
    x\conc\mbox{\tt es} \; 
	\mbox{\sf else}\; x \conc \mbox{\tt s}\;
	\mbox{\sf fi} : n;
\end{align}
%%
Moreover, equality is definable from {\sf suf}. Evidently, since 
we allow a function to be defined by cases, the irregular plural forms 
can be incorporated here as well, as long as they are additive 
(as is {\tt oxen} but not {\tt men}). For nonadditive plural 
formations see the remarks on umlaut in Section~\ref{kap5}.\ref{kap5-3}.

Now take another case, causatives. Many English verbs have causative 
forms. Examples are {\tt laugh}, {\tt drink}, {\tt clap}.
%%
\begin{align}
\label{ex:581} & \mbox{\tt The audience laughed the conductor off the 
	stage.} \\
\label{ex:582} & \mbox{\tt The manager drank his friends under the table.}
\\
\label{ex:583} & \mbox{\tt The audience was clapping the musician back onto} 
\\\notag
  & \quad \mbox{\tt the stage.}
\end{align}
%%
In all these cases the meaning of the causative is regularly formed
so that we may actually assume that there is a sign that performs
the change. But it leaves no visible trace. Thus, we must at least 
allow operators that perform type conversion even when they change 
nothing in the semantics. In the type system we have advocated above 
they can be succinctly represented by
%%
\begin{equation}
\lambda x : \sigma. x : \tau
\end{equation}
%%
Now, the conversion of a string of one type into another is often
accompanied by morphological marking. For example, the gerund in
English turns a verb into a noun ({\tt singing}). It is formed
regularly by suffixing {\tt ing}. So, it has the following sign:
%%
\begin{equation}
\mbox{\sc ger} := \auf \lambda x : v.x \conc \mbox{\tt ing} : n,
    \lambda x.x\zu
\end{equation}
%%
The semantics of these nominalizations is rather complex (see
\cite{hammlambalgen:nominalization}), so we have put the identity 
for simplicity here. Signs that consist of nothing more than a 
%%%%
\index{conversioneme}%%%
%%%%
type conversion are called \textbf{conversionemes}
in \cite{melcuk:morphologie}. Obviously, they are not
progressive in the intuitive sense. For we can in principle
change a string from $\sigma$ to $\tau$ and back; and we could
do this as often as we like. However, there is little harm in
admitting such signs. The additional complexity can be handled
in much the same way as unproductive context free rules.

%%%
\index{Swiss German}%%
%%%
Another challenge is Swiss German. Since we do not want to make
use of products, it is not obvious how we can instrumentalize
the $\lambda$--terms to get the word order right. Here is how
this can be done. We distinguish the main (inflected) verb from
its subordinate verbs, and raising from nonraising verbs.
(See Table~\ref{tab:swissgerman}. We use `$\pm$i' short for 
`$\pm$inflected', `$\pm$r' for `$\pm$raising', and `$\pm$t' for 
`$\pm$transitive'. We have suppressed the type information as 
it is of marginal relevance here.)
%%%
\begin{table}
\caption{Swiss German Verbs}
\label{tab:swissgerman}
\begin{center}
\begin{tabular}{lll}
--i--r+t & {\sc aaste} & := $\auf\lambda x.\lambda y. \lambda z.
    y \oconc x \oconc z \oconc \mbox{\tt aastriche},
        \mbox{\sf paint}'\zu$ \\
+i--r+t & {\sc aast} & := $\auf\lambda x.\lambda y.y \oconc
    x \oconc \mbox{\tt aastricht}, \mbox{\sf paint}'\zu$ \\
--i--r--t & {\sc schwe} & := $\auf\lambda y. \lambda z.
    y \oconc z \oconc \mbox{\tt schwimme},
        \mbox{\sf sim}'\zu$ \\
+i--r--t & {\sc schw} & := $\auf\lambda x.x \oconc \mbox{\tt schwimmt},
    \mbox{\sf swim}'\zu$ \\
--i+r+t & {\sc laa} & := $\auf\lambda x. \lambda \CP. \lambda v.\lambda w.
    \CP(v \oconc x)(w \oconc \mbox{\tt laa}), \mbox{\sf let}'\zu$ \\
+i+r+t & {\sc laat} & :=  
	$\auf\lambda \CP. \lambda x.\CP(x)(\mbox{\tt laa}),
    \mbox{\sf let}'\zu$
\end{tabular}
\end{center}
\end{table}
%%
Here, $v$, $x$, $z$ are variables over NP--cluster strings, $w$, 
$y$ variables over verb--cluster strings, and $\CP$ a variable for 
functions from NP--cluster strings to functions from verb--cluster 
strings to strings. NPs are by contrast very basic:
%%
\begin{align}
\mbox{\sc mer} & := \auf \mbox{\tt mer}, \mathsf{we}'\zu &
\mbox{\sc huus} & := \auf \mbox{\tt es huus}, \mathsf{house}'\zu
\end{align}
%%
We ignore case for the moment.
The lowest clause is translated as follows.
%%
\begin{align}
& \mbox{\sc aaste}(\mbox{\sc huus}) \\\notag
= & \auf \lambda y.\lambda z. y \oconc
    \mbox{\tt es huus} \oconc z \oconc \mbox{\tt aastriche}, 
   \mbox{\sf paint}'(\mbox{\sf house}')\zu
\end{align}
%%
The recursive part, raising verb plus object, is translated as follows:
%%
\begin{equation}
\begin{split}
& \mbox{\sc h\"alfe}(\mbox{\sc chind}) \\
= & \auf (\lambda x.\lambda \CP.\lambda v.\lambda w.\CP(v \oconc x)%
(w \oconc \mbox{\tt h\"alfe}))(\mbox{\tt em chind}), \\
  & \quad
	\mbox{\sf help}'(\mbox{\sf children}')\zu \\
= & \auf (\lambda \CP.\lambda v.\lambda w.\CP(v \oconc \mbox{\tt em chind})%
(w \oconc \mbox{\tt h\"alfe})), \\
  & \quad 
	\mbox{\sf help}'(\mbox{\sf children}')\zu 
\end{split}
\end{equation}
%%
If we combine the two we get something that is of the same kind 
as the lower infinitive, showing that the recursion is adequately 
captured:
%%
\begin{equation}
\begin{split}
& \mbox{\sc h\"alfe}(\mbox{\sc chind})(\mbox{\sc aaste}(\mbox{\sc huus}))
\\
= &  \auf \lambda v.\lambda w. v \oconc \mbox{\tt em$\;$chind es$\;$huus}%
\oconc w \oconc \mbox{\tt h\"alfe aastriche}, \\
& \quad 
\mbox{\sf help}'(\mbox{\sf children}')(\mbox{\sf paint}'(\mbox{\sf house}')))
\end{split}
\end{equation}
%%
Had we inserted a finite verb, the second `hole' would have been 
closed. There would have been just a place for the subject. Once 
that is inserted, there are no more holes left. The recursion is 
finished. Notice that the structure term has the form of the 
corresponding English structure. The $\lambda$--terms simply 
transliterate it into Swiss German. Let us briefly speak about 
%%%
\index{case}%%
%%%
case. We insert only the bare nouns and let the verb attach the 
appropriate case marker. For example, if $\mbox{\sc dat}$ is the 
function that turns a DP into a dative marked DP, the sign 
{\sc h\"alfe} will be 
%%%
\begin{equation}
\mbox{\sc h\"alfe} := \auf\lambda x. \lambda \CP. \lambda v.\lambda w.
    \CP(v \oconc \mbox{\sc dat}(x))(w \oconc \mbox{\tt laa}), 
	\mbox{\sf help}'\zu
\end{equation}
%%%

Next, we shall deal with case agreement inside a noun phrase. In 
many languages, adjectives agree in case with the noun they modify. 
We take our example from Finnish. 
%%%
\index{Finnish}%%
%%%
The phrase {\tt iso juna} `a/the big train' inflects in the 
singular as follows. (We show only a fraction of the case system.)
%%
\begin{equation}
\begin{array}{lll}
\mbox{\rm nominative} & \mbox{\tt iso juna} \\
\mbox{\rm genitive}   & \mbox{\tt ison junan} \\
\mbox{\rm allative}   & \mbox{\tt isolle junalle} \\
\mbox{\rm inessive}   & \mbox{\tt isossa junassa}
\end{array}
\end{equation}
%%
In the present case, it is the same suffix that is added to the
adjective as well as the noun. Now, suppose we analyze the allative
as a suffix that turns a caseless noun phrase into a case marked
noun phrase. Then we want to avoid analyzing the allative
{\tt isolle junalle} as consisting of occurrences of the allative
case. We want to say that it occurs once, but is spelled out twice.
To achieve this, we introduce two types: $\nu$, the type of 
case marked nouns, and $\kappa$, the type of case markers. 
Noun roots will be of type $\kappa \pf \nu$, adjectives of type 
$(\kappa \pf \nu) \pf (\kappa \pf \nu)$. 
%%
\begin{align}
\mbox{\sc juna} & := \auf \lambda x : \kappa.\mbox{\tt juna}\conc x : \nu, 
	\mbox{\sf train}'\zu \\
\mbox{\sc iso} & := \auf \lambda \CP : \kappa\pf\nu. \lambda x : \kappa. 
	\mbox{\tt iso} \conc x \oconc \CP(x) : \nu, \mbox{\sf big}'\zu
\end{align}
%%
So, $x$ has the type $\kappa$, $\CP$ the type $\kappa \pf \nu$. 
These signs combine to 
%%
\begin{equation}
\mbox{\sc iso}(\mbox{\sc juna}) =
\auf \lambda x : \kappa.\mbox{\tt iso}\conc x \oconc \mbox{\tt juna} \conc x 
	: \nu, \mbox{\sf big}'(\mbox{\sf train}')\zu
\end{equation}
%%
Finally, assume the following sign for the allative.
%%
\begin{equation}
\mbox{\sc all} := \auf \mbox{\tt lle} : \kappa, \mbox{\sf move-to}'\zu
\end{equation}
%%
Then the last two signs combine to
%%
\begin{align}
 & \mbox{\sc all}(\mbox{\sc iso}(\mbox{\sc juna})) \\\notag 
= & \auf \mbox{\tt isolle junalle}: \nu, \mbox{\sf move-to}'%
(\mbox{\sf big}'(\mbox{\sf train}'))\zu
\end{align}
%%
This has the advantage that the tectogrammatical structure of signs 
is much like their semantic structure, and that we can stack as many 
adjectives as we like: the case ending will automatically be distributed 
to all constituents. Notice that LMGs put a limit on the number of 
occurrences that can be controlled at the same time, and so they 
cannot provide the same analysis for agreeing adjectives. Thus, 
de Saussure grammars sometimes provide more adequate analyses than 
do LMGs. We remark here that the present analysis conforms to the idea 
proposed in \cite{harris:structural}, 
%%%
\index{Harris, Zellig S.}%%%
%%%
who considers agreement simply
as a multiple manifestation of a single morpheme. Case assignment
can also be handled in a rather direct way. Standardly, a verb that
takes a case marked noun phrase is assumed to select the noun phrase
as a noun phrase of that case. Instead, however, we may assume that
the sign for a case marking verb actually carries the case marker
and attaches it to the NP. The Finnish verb {\tt tuntua} `to
resemble' selects ablative case. Assume that it has an
ablative marked argument that it takes directly to its right.
Then its sign may be assumed to be like this (taking the 3rd person
singular present form).
%%
\begin{equation}
\mbox{\sc tuntuu} :=
    \auf \lambda x.\mbox{\tt tuntuu}\oconc x(\mbox{\tt lta}), 
	\mbox{\sf resemble}'\zu
\end{equation}
%%
The reason is that if we simply insist that the noun phrase comes 
equipped with the correct case, then it enters with its ablative 
case meaning rather than with its typical NP meaning. Notice namely 
that the ablative has an unmotivated appearance here given the 
semantics of the ablative case in Finnish. (Moreover, it is the 
only case possible with this verb.) So, semantically the situation
is the same as if the verb was transitive. Notice that the fact
that {\tt tuntua} selects an ablative NP is a coincidence in this 
setup. The ablative form is directly added to the complement 
selected. This is not the best way of arranging
things, and in \cite{kracht:against} a proposal has been
made to remedy the situation.

There is a list of potential problems for de Saussure grammars. We
%%%
\index{German}%%
%%%
mention a few of them. The plural in German is formed with some
stems by umlauting them (see Section~\ref{kap1}.\ref{kap1-3}). This is 
(at least on the surface) an operation that is not additive. 
As mentioned earlier, we shall discuss this phenomenon in 
Section~\ref{kap5}.\ref{kap5-3}. Another problem is what is known 
as \textbf{suppletion}. 
%%%%
\index{suppletion}%%%
%%%%
We exemplify this phenomenon with the gradation of Latin adjectives.
%%%
\index{Latin}%%
%%%
Recall that adjectives in many languages possess three forms: a
positive ({\tt happy}) a comparative ({\tt happier}) and a
superlative ({\tt happiest}). This is so in Latin. 
Table~\ref{tab:gradation} gives some examples. Adjectives above 
the line are regularly formed, the ones below are irregular.
%%
\begin{table}
\caption{Gradation of Latin Adjectives}
\label{tab:gradation}
\begin{center}
\begin{tabular}{lll}
\mbox{\rm Positive} & \mbox{\rm Comparative} & \mbox{\rm Superlative} \\\hline
\mbox{\tt parvus} & \mbox{\tt parvior} & \mbox{\tt parvissimus} \\
\mbox{\tt beatus} & \mbox{\tt beatior} & \mbox{\tt beatissimus} \\\hline
\mbox{\tt bonus}  & \mbox{\tt melior}  & \mbox{\tt optimus} \\
\mbox{\tt malus}  & \mbox{\tt peior}   & \mbox{\tt pessimus} 
\end{tabular}
\end{center}
\end{table}
%%
Interesting is the fact that it is not the comparative or 
superlative suffix that is irregular: it is the root form itself.
The expected form $^{\ast}${\tt bonior} is replaced by {\tt melior}: 
the root changes from {\tt bon} to {\tt mel}. (The English adjective
{\tt good} is also an example.) A different phenomenon is exhibited 
by English {\tt worse}. The comparative is formed either by 
adding {\tt er} ({\tt better}, {\tt faster}) or by adding 
{\tt more}. The form {\tt worse} resists a decomposition into 
a stem and a suffix. In the case of {\tt worse} we speak of a 
%%%
\index{portmanteau morph}
%%%
\textbf{portmanteau morph}. Portmanteau morphs can be treated in 
de Saussure grammars only as lexical items (since we only allow 
additive phonological processes).  

{\it Notes on this section.} A proposal similar to de Saussure grammars 
one has been made by Philippe de Groote~\shortcite{degroote:abstract}.
%%%
\index{de Groote, Philippe}%%
%%
\vplatz
\exercise
Recall from Section~\ref{kap3}.\ref{kap3-3} the notion of a combinatory
extension of categorial grammar. We may attempt the same for
de Saussure grammars. Define a new mode of combination, {\tt B},
as follows.
%%
\begin{equation}
\mbox{\tt B}(\auf e, m\zu)(\auf e', m'\zu) :=
    \auf \lambda x.e(e'(x)), \lambda y.m(m'(y))\zu
\end{equation}
%%
Here, $e$ is of type $\mu \pf \nu$, $e'$ of type $\lambda \pf \mu$
and $x$ of type $\lambda$, so that the string term $\lambda x.
e(e'(x))$ is of type $\lambda \pf \nu$. Likewise for the semantics.
Show that this extension does not generate different signs, it
just increases the set of structure terms. Contrast this with
the analogous extension of AB--grammars. Look especially
at mixed composition rules.
%%
\vplatz
\exercise
%%%
\index{Arabic}%%
%%%
Review the facts from Exercise~\ref{ex:arabic} on Arabic. Write
a de Saussure grammar that correctly accounts for them.
{\it Hint.} This is not so simple. First, define {\it schemes},
which are functions of type 
%%%
\begin{equation}
s \pf (s \pf (s \pf (s \pf (s \pf (s \pf s)))))
\end{equation}
%%%
They provide a way of combining consonantism
(root) and vocalism. The first three arguments form the
consonantism, the remaining three the vocalism. The change
in consonantism or vocalism can be defined on schemes before
inserting the actual consonantism and vocalism.
%%
\vplatz
\exercise
%%%
\index{Mandarin}%%
%%%
Write a de Saussure grammar that generates the facts of Mandarin
shown in Exercise~\ref{ex:chinese}.
%%
\vplatz
\exercise
In European languages, certain words inside a NP do not inflect for
case (these are adverbs, relative clauses and other) and
moreover, no word can more than two cases. Define case marking
functions that take care of this. (If you need concrete examples,
you may elaborate the Finnish example using English substitute
words.)
%%
\vplatz
\exercise
We have talked briefly in Section~\ref{kap4}.\ref{kap4-1} about Australian
case marking systems. We shall simplify the facts (in particular
the word order) as follows. We define a recursive translation from
$\PN_{\Omega}$ ($\Omega$--terms $t$ in Polish notation) 
inductively as follows. We assume case markers 
$\mbox{\tt c}_i$, $i < \Omega$. For a constant term $c$, put 
$c^{\diamond} := c$. If $F$ is an $n$--ary function symbol and 
$t_i$, $i < n$, terms then put
%%
\begin{equation}
(Ft_0\dotsb t_{n-1})^{\diamond} := F\oconc
    t_0^{\diamond}\conc \mbox{\tt c}_0 \oconc
    t_1^{\diamond}\conc \mbox{\tt c}_1 \oconc
    \dotsb \oconc
    t_{n-1}^{\diamond}\conc \mbox{\tt c}_{n-1}
\end{equation}
%%
Write a de Saussure grammar that generates the set 
$\{\auf t^{\diamond}, t\zu : t \in \PN_{\Omega}\}$.
%%
\vplatz
\exercise
%%%
\index{functional head}%%%
%%%
In many modern theories of grammar, so--called \textbf{functional
heads} play a fundamental role. Functional elements are
elements that are responsible for the correct shape of the
structures, but have typically very little --- if any --- content.
A particularly useful idea is to separate the content of an
element from its syntax. For example, we may introduce
the morphological type of a transitive verb ({\it tv\/})
without specifying any selectional behaviour.
%%
\begin{equation}
\mbox{\sc see} := \auf \mbox{\tt see} : \mbox{\it tv},
    \mbox{\sf see}'\zu
\end{equation}
%%
Then we assume one or two functional elements that turn this sign
into the signs $\mbox{\sc see}_i$, $i < 6$. Show how this can
be done for the particular case of the signs $\mbox{\sc see}_i$.
Can you suggest a general recipe for words of arbitrary category?
Do you see a solution of the problem of ablative case selection 
in Finnish?

% \newpage 
%	\thispagestyle{empty}
%	\mbox{}
 \chapter{The Model Theory of Linguistic Structures}
\thispagestyle{empty}
%%
\label{kap5}
%
%
%
\section{Categories}
\label{kap5-1}
%
%
%
Up to now we have used plain nonterminal symbols in our description
of syntactic categories --- symbols with no internal structure. For 
many purposes this is not a serious restriction. But it does not allow 
to capture important regularities of language. We give an example 
from German. 
%%%%
\index{German}%%
%%%%
The sentences \eqref{ex:611} -- \eqref{ex:616} are grammatical.
%%
\begin{align}
\label{ex:611} & \mbox{\tt Ich sehe.} \\\notag
                & \mbox{\rm I see-{\sc 1.Sg}} \\
\label{ex:612} & \mbox{\tt Du siehst.} \\\notag
                & \mbox{\rm You.{\sc Sg} see-{\sc 2.Sg}} \\
\label{ex:613} & \mbox{\tt Er/Sie/Es sieht.} \\\notag
                & \mbox{\rm He/She/It see-{\sc 3.Sg}} \\
\label{ex:614} & \mbox{\tt Wir sehen.} \\\notag
                & \mbox{\rm We see-{\sc 1.Pl}} \\
\label{ex:615} & \mbox{\tt Ihr seht.} \\\notag
               & \mbox{\rm You.{\sc Pl} see-{\sc 2.Pl}} \\
\label{ex:616} & \mbox{\tt Sie sehen.} \\\notag
                & \mbox{\rm They see-{\sc 3.Pl}}
\end{align}
\\[2mm]
%%%
By contrast, the following sentences are ungrammatical.
%%
\begin{align}
\label{ex:617} & ^{\ast}\mbox{\tt Ich siehst}/\mbox{\tt sieht}/
	\mbox{\tt sehen}/\mbox{\tt seht.} \\\notag
                & \mbox{\rm I see-{\sc 2.Sg}/see-{\sc 3.Sg}/see-{\sc
    1/3.Pl}/see-{\sc 2.Pl}} \\
\label{ex:618} & ^{\ast}\mbox{\tt Du sehe}/\mbox{\tt sieht}/
	\mbox{\tt sehen}/\mbox{\tt seht.} \\\notag
                & \mbox{\rm You.{\sc Sg} see-{\sc 1.Sg}/see-{\sc 3.Sg}/%
see-{\sc 1/3.Pl}/see-{\sc 2.Pl}}
\end{align}
%%%
One says that the finite verb of German agrees with the subject in 
person and number. This means that the verb has different forms 
depending on whether the subject is in the 1st, 2nd or 3rd person, 
and whether it is singular or plural. 

How can we account for this? On the one hand, we may simply assume 
that there are six different kinds of subjects (1st, 2nd or 3rd person, 
singular or plural) as well as five different kinds of verb forms (since 
two are homophonous, namely 1st and 3rd person plural). And the 
subjects of one kind can only cooccur with a matching verb form.
But the grammars we looked at so far do not allow to express this 
fact at this level of generality; all one can do is provide lists 
of rules. A different way has been proposed among other in 
%%%
\index{GPSG (see Generalized Phrase Structure Grammar)}%%
\index{Generalized Phrase Structure Grammar}%%%
%%%
\textbf{Generalized Phrase Structure Grammar} (\textbf{GPSG},
see \cite{gazdarpullumsag:gpsg}). Let us start with the following
basic rule.
%%
\begin{equation}
\label{eq:61ast} \mathsf{S} \pf \mathsf{NP}\quad \mathsf{VP}
\end{equation}
%%
Here the symbols $\mathsf{S}$, $\mathsf{NP}$ and $\mathsf{VP}$ are symbols 
not for a single category but for a whole set of them. (This is why 
we have not used typewriter font.) In fact,
the labels are taken to be {\it descriptions of categories}.
They are not string anymore. 
This means that these `labels' can be combined using boolean
connectives such as negation, conjunction and disjunction.
For example, if we introduce the properties $\mathsf{1}$, $\mathsf{2}$
and $\mathsf{3}$ as well as $\mathsf{Sg}$ and $\mathsf{Pl}$ then our rule
\eqref{eq:61ast} can be refined as follows:
%%
\begin{equation}
\mathsf{S} \pf \mathsf{NP}\und
\mathsf{1}\und \mathsf{Sg} \quad \mathsf{VP} \und
\mathsf{1} \und \mathsf{Sg}
\end{equation}
%%
Furthermore, we have the following terminal rules.
%%
\begin{equation}
\mathsf{NP}\und\mathsf{1}\und\mathsf{Sg} \pf
\mbox{\tt ich}, \quad
\mathsf{VP} \und \mathsf{1}\und\mathsf{Sg} \pf
\mbox{\tt sehe}
\end{equation}
%%
Here $\mathsf{NP}\und \mathsf{1} \und \mathsf{Sg}$ is 
the description of a category which is a noun phrase ($\mathsf{NP}$) 
in the first person ($\mathsf{1}$) singular ($\mathsf{Sg}$). This means 
that we can derive the sentence \eqref{ex:611}. In order
for the sentences \eqref{ex:617} and \eqref{ex:618} not
to be derivable we now have to eliminate the rule \eqref{eq:61ast}.
But this excludes the sentences \eqref{ex:612} -- \eqref{ex:616}.
To get them back again we still have to introduce five more rules.
These can however be fused into a single schematic rule.
In place of $\mathsf{NP}$ we now write $[\mbox{\sc cat} : %
\mbox{\it np\/}]$, in place of $\mathsf{1}$ we write
$[\mbox{\sc pers} : \mbox{\it 1\/}]$, and in place of
$\mathsf{Pl}$ we write $[\mbox{\sc num} : \mbox{\it pl\/}]$.
Here, we call {\sc cat}, {\sc per} and {\sc num}
%%%
\index{attribute}%%
%%%
\textbf{attributes}, and {\it np}, {\it vp}, {\it 1}, and so on
\textbf{values}.
%%%
\index{value}%%
%%%
In the pair $[\mbox{\sc cat} : \mbox{\it np\/}]$ we say that
the attribute {\sc cat} has the value {\it np}. A set of pairs
$[A : v]$, where $A$ is an attribute and $v$ a value
%%%
\index{attribute--value structure (AVS)}%%
\index{AVS (see attribute value structure)}%%%
%%%
is called an \textbf{attribute--value structure} or simply an 
\textbf{AVS}.

The rule \eqref{eq:61ast} is now replaced by the schematic rule 
\eqref{eq:61ddagger}.
%%
\begin{equation}
\label{eq:61ddagger}
\left[\begin{array}{l@{\quad : \quad}l}
\mbox{\sc cat} & \mbox{\it s}
\end{array}\right]
\pf
\left[\begin{array}{l@{\quad : \quad}l}
\mbox{\sc cat} & \mbox{\it np} \\
\mbox{\sc per} & \alpha \\
\mbox{\sc num} & \beta
\end{array}\right]
\quad
\left[\begin{array}{l@{\quad : \quad}l}
\mbox{\sc cat} & \mbox{\it vp} \\
\mbox{\sc per} & \alpha \\
\mbox{\sc num} & \beta
\end{array}\right]
\end{equation}
%%
Here, $\alpha$ and $\beta$ are variables. However, they have
different value range; $\alpha$ may assume values from the set
$\{\mbox{\it 1}, \mbox{\it 2}, \mbox{\it 3\/}\}$ $\beta$ values
from the set $\{\mbox{\it sg}, \mbox{\it pl\/}\}$. This fact shall be
dealt with further below. One has to see to it that the properties
inducing agreement are passed on. This means that the following
rule also has to be refined in a similar way.
%%
\begin{equation}
\label{eq:61dagger}
\mathsf{VP} \pf \mathsf{V}\quad \mathsf{NP}
\end{equation}
%%
This rule says that a VP may be a constituent comprising a
(transitive) verb and an NP. The agreement features have to be
passed on to the verb.
%%
\begin{equation}
\left[\begin{array}{l@{\quad : \quad}l}
\mbox{\sc cat} & \mbox{\it vp} \\
\mbox{\sc per} & \alpha \\
\mbox{\sc num} & \beta
\end{array}\right]
\pf
\left[\begin{array}{l@{\quad : \quad}l}
\mbox{\sc cat} & \mbox{\it v} \\
\mbox{\sc per} & \alpha \\
\mbox{\sc num} & \beta
\end{array}\right]
\quad
\left[\begin{array}{l@{\quad : \quad}l}
\mbox{\sc cat} & \mbox{\it np}
\end{array}\right]
\end{equation}
%%
Now, there are languages in which the verb not only agrees with
the subject but also with the object in the same categories. This
means that it does not suffice to simply write $[\mbox{\sc per} : \alpha]$;
we also have to say whether $\alpha$ concerns the subject or
the object. Hence the structure relating to agreement has
to be further embedded into the structure. 
%%
\begin{equation}
\left[
\begin{array}{l@{\, : \,}l}
    \mbox{\sc cat} & \mbox{\it vp} \\
    \mbox{\sc per} & \alpha \\
    \mbox{\sc num} & \beta
\end{array}\right]
\pf
\left[
\begin{array}{l@{\, : \,}l}
    \mbox{\sc cat} & \mbox{\it v} \\
    \mbox{\sc agrs} & \left[
        \begin{array}{l@{\, : \,}l}
            \mbox{\sc per} & \alpha \\
            \mbox{\sc num} & \beta
        \end{array}\right] \\
\multicolumn{2}{l}{}    \\
    \mbox{\sc agro} & \left[
        \begin{array}{l@{\, : \,}l}
            \mbox{\sc per} & \alpha' \\
            \mbox{\sc num} & \beta'
        \end{array}\right]
\end{array}\right]
\quad
\left[\begin{array}{l@{\, : \,}l}
\mbox{\sc cat} & \mbox{\it np} \\
\mbox{\sc per} & \alpha' \\
\mbox{\sc num} & \beta'
\end{array}\right]
\end{equation}
%%
It is clear that this rule does the job as intended. One can
make it look even nicer by assuming also for the NP an embedded
structure for the agreement complex. This is what we shall do
below. Notice that the value of an attribute is now not only
a single value but may in turn be an entire AVS. Thus, two kinds 
of attributes are distinguished.  {\it 1}, {\it sg\/} are called 
\textbf{atomic values}. In the present context, all basic expressions 
are either (atomic) values or attributes.
%%%
\index{value!atomic}%%
\index{attribute!Type 0}%%
\index{attribute!Type 1}%%
%%%
Attributes which have only atomic values are called \textbf{Type 0 
attributes}, all others are \textbf{Type 1 attributes}. This is the 
basic setup of \cite{gazdar:cstructures}. In the so--called
\textbf{Head Driven Phrase--Structure Grammar} by Carl Pollard
and Ivan Sag
%%%
\index{HPSG (see Head Driven Phrase Structure Grammar)}%%
\index{Head Driven Phrase Structure Grammar}%%
%%%
(\textbf{HPSG}, see \cite{pollardsag:hpsg}) this has been pushed much
further. In HPSG, the entire structure is encoded using AVSs of the 
kind just shown.  Not only the bare linguistic features but 
also the syntactic structure itself is coded into AVSs. We shall 
study these structures from a theoretical point of view in 
Section~\ref{kap5}.\ref{kap5-6}. Before we enter this investigation
we shall move one step further. The rules that we have introduced
above use variables for values of attributes. This certainly
is a viable option. However, HPSG has gone into a different
direction here. It introduces what are in fact structure variables,
%%%
\index{variable!structure \faul}%%
%%%
whose role it is to share entire AVSs between certain members of an 
AVS. To see how this works we continue with our example. Let us now 
write an NP not as a flat AVS, but let us instead embed
the agreement related attribute value pairs as the value
of an attribute {\sc agr}. A 3rd person NP in the plural
is now represented as follows.
%%
\begin{equation}
\left[\begin{array}{l@{\quad : \quad}l}
\mbox{\sc cat} & \mbox{\it np\/} \\
\mbox{\sc agr} & \left[
    \begin{array}{l@{\quad : \quad}l}
    \mbox{\sc num} & \mbox{\it pl\/} \\
    \mbox{\sc per} & \mbox{\it 3\/}
    \end{array}\right]
\end{array}\right]
\end{equation}
%%
The value of {\sc agr} is now structured in the same way as the 
values of {\sc agrs} and {\sc agro}. Now we can rewrite our rules 
with the help of structure variables as follows. The rule 
\eqref{eq:61ddagger} now assumes the form
%%
\begin{equation}
\label{eq:61ddd}
\left[\begin{array}{l@{\quad : \quad}l}
\mbox{\sc cat} & \mbox{\it s}
\end{array}\right]
\pf
\left[\begin{array}{l@{\quad : \quad}l}
\mbox{\sc cat} & \mbox{\it np} \\
\mbox{\sc agr} & \framebox{1}
\end{array}\right]
\quad
\left[\begin{array}{l@{\quad : \quad}l}
\mbox{\sc cat} & \mbox{\it vp} \\
\mbox{\sc agrs} & \framebox{1} \\
\end{array}\right]
\end{equation}
%%
The rule that introduces the object now has this shape.
%%
\begin{equation}\left[
\begin{array}{l@{\; : \;}l}
    \mbox{\sc cat} & \mbox{\it vp} \\
    \mbox{\sc agrs} & \framebox{1}
\end{array}\right]
\pf
\left[
\begin{array}{l@{\; : \;}l}
    \mbox{\sc cat} & \mbox{\it v} \\
    \mbox{\sc agrs} & \framebox{1} \\
    \mbox{\sc agro} & \framebox{2}
\end{array}\right]
\quad
\left[\begin{array}{l@{\; : \;}l}
    \mbox{\sc cat} & \mbox{\it np\/} \\
    \mbox{\sc agr} & \framebox{2}
\end{array}\right]
\end{equation}
%%
The labels $\framebox{1}$ and $\framebox{2}$ are variables for
AVSs. If some variable occurs several times in a rule then every
occurence stands for the same AVS. This is precisely what is needed 
to formulate agreement. AVS variables help to avoid that agreement 
blows up the rule apparatus beyond recognition. The rules have 
become once again small and perspicuous. (However, the agreement 
facts of languages are full of tiny details and exceptions, which 
make the introduction of more rules unavoidable.)

Now if AVSs are only the description, then what are categories? 
In a nutshell, it is thought that categories are {\it Kripke--frames}. 
One assumes a set of vertices and associates with each attribute a 
binary relation on this set. So, attributes are edge colours, atomic 
values turn into vertex colours. And a syntactic tree is no longer
an exhaustively ordered tree with simple labels but an exhaustively 
ordered tree with labels having complex structure. Or, as it is more 
convenient, we shall assume that the tree structure itself also is 
coded by means of AVSs. The Figure~\ref{fig:sstruk} shows an example 
of a structure which --- as one says --- is \textbf{licensed} by 
the rule
%%%
\index{licensing}%%
%%%
\eqref{eq:61ddd}.
%%
\setlength{\unitlength}{1.2em}
\begin{figure}
\begin{center}
\begin{picture}(20,10)
\put(4,3){\makebox(0,0){$\bullet$}}
    \put(3,3){\makebox(0,0)[r]{\it sg}}
    \put(5.5,3){\makebox(0,0){\sc num}}
\put(7,5){\vector(-3,-2){2.8}}
\put(4,7){\makebox(0,0){$\bullet$}}
    \put(3,7){\makebox(0,0)[r]{\it 1}}
    \put(7,5){\vector(0,-1){2.8}}
\put(7,5){\makebox(0,0){$\bullet$}}
    \put(7,5){\line(1,1){3}}
\put(7,5){\vector(-3,2){2.8}}
\put(7,2){\makebox(0,0){$\bullet$}}
    \put(7.5,3.5){\makebox(0,0)[l]{\sc cat}}
    \put(7,1){\makebox(0,0){\it np}}
\put(5.5,7){\makebox(0,0){\sc per}}
%%
\put(10,8){\vector(1,0){2.8}}
    \put(11.5,9){\makebox(0,0){\sc cat}}
\put(13,8){\makebox(0,0){$\bullet$}}
    \put(14,8){\makebox(0,0)[l]{\it s}}
\put(10,8){\makebox(0,0){$\bullet$}}
\put(13,5){\line(-1,1){3}}
\put(13,5){\makebox(0,0){$\bullet$}}
\put(13,2){\makebox(0,0){$\bullet$}}
    \put(13,5){\vector(0,-1){2.8}}
    \put(12.5,3.5){\makebox(0,0)[r]{\sc cat}}
    \put(13,1){\makebox(0,0){\it vp}}
\put(13,5){\vector(3,-2){2.8}}
    \put(14.5,3){\makebox(0,0){\sc num}}
\put(16,3){\makebox(0,0){$\bullet$}}
    \put(17,3){\makebox(0,0)[l]{\it sg}}
\put(13,5){\vector(3,2){2.8}}
    \put(14.5,7){\makebox(0,0){\sc per}}
\put(16,7){\makebox(0,0){$\bullet$}}
    \put(17,7){\makebox(0,0)[l]{\it 1}}
\end{picture}
\end{center}
\caption{The Kripke--frame of an AVS}
\label{fig:sstruk}
\end{figure}
%%
The literature on AVSs is rich (see the books \cite{johnson:avlogic} 
and \cite{carpenter:logic}). In its basic form, however, it is quite 
simple. Notice that it is a mistake to view attributes as objects. 
In fact, AVSs are not objects, they are descriptions of objects. 
Moreover, they can be the values of attributes. Therefore we treat 
values like {\it np}, {\it 1\/} as properties which can be combined 
with the usual boolean operations, for example $\nicht$, $\und$, $\oder$ 
or $\pf$. This has the advantage that we are now able to represent 
the category of the German verb form {\tt sehen} in either of 
the following ways.
%%
\begin{equation}
\left[\begin{array}{l@{\quad : \quad}l}
\mbox{\sc cat} & \mbox{\it v} \\
\mbox{\sc per} & \mbox{\it 1} \oder \mbox{\it 3} \\
\mbox{\sc num} & \mbox{\it pl}
\end{array}\right]
\quad
\left[\begin{array}{l@{\quad : \quad}l}
\mbox{\sc cat} & \mbox{\it v} \\
\mbox{\sc per} & \nicht \mbox{\it 2} \\
\mbox{\sc num} & \mbox{\it pl}
\end{array}\right]
\end{equation}
%%
The equivalence between these two follows only if we assume
that the values of {\sc per} can be only {\it 1}, {\it 2} or 
{\it 3}. This fact, however, is a fact of German, and will be 
part of the grammar of German. (In fact, it seems to hold pretty 
universally across languages.) Notice that the collocation of 
attribute--value pairs into an attribute--value structure is 
nothing but the logical conjunction. So the left hand AVS can 
also be written down as follows.
%%
\begin{equation}
[\mbox{\sc cat} : \mbox{\it v\/}]
\und [\mbox{\sc per} : \mbox{\it 1\/} \oder \mbox{\it 3\/}]
\und [\mbox{\sc num} : \mbox{\it pl\/}]
\end{equation}
%%
One calls \textbf{underspecification}
%%%
\index{underspecification}%%
%%%
the fact that a representation does not fix an object in all
detail but that it leaves certain properties unspecified.
Disjunctive specifications are a case in point. However, they
do not in fact provide the most welcome case. The most ideal 
case is when certain attributes are not contained in the AVS
so that their actual value can be anything. For example, the
category of the English verb form {\tt saw} may be (partially!) 
represented thus.
%%
\begin{equation}
\left[\begin{array}{l@{\quad : \quad}l}
\mbox{\sc cat} & \mbox{\it v} \\
\mbox{\sc temp} & \mbox{\it past\/}
\end{array}\right]
\end{equation}
%%
This means that we have a verb in the past tense. The number and
person are simply not mentioned.  We can --- but need not --- write
them down explicitly.
%%
\begin{equation}
\left[\begin{array}{l@{\quad : \quad}l}
\mbox{\sc cat} & \mbox{\it v} \\
\mbox{\sc temp} & \mbox{\it past\/} \\
\mbox{\sc num} & \top \\
\mbox{\sc per} & \top
\end{array}\right]
\end{equation}
%%
Here $\top$ is the maximally unspecified value. We have ---
this is a linguistical, that is to say, an empirical, fact ---:
%%
\begin{equation}
[\mbox{\sc per} : \mbox{\it 1} \oder \mbox{\it 2} \oder \mbox{\it 3\/}]
\end{equation}
%%
From this we can deduce that the category of {\tt saw} also has 
the following representation.
%%
\begin{equation}
\left[\begin{array}{l@{\quad : \quad}l}
\mbox{\sc cat} & \mbox{\it v} \\
\mbox{\sc temp} & \mbox{\it past\/} \\
\mbox{\sc num} & \top \\
\mbox{\sc per} & \mbox{\it 1} \oder \mbox{\it 2} \oder
    \mbox{\it 3\/}
\end{array}\right]
\end{equation}
%%
Facts of language are captured by means of axioms. More on that later.

Since attribute--value pairs are propositions, we can combine
them in the same way. The category of the English verb form 
{\tt see} has among other the following grammatical representation.
%%
\begin{equation}
\nicht \left[\begin{array}{l@{\quad : \quad}l}
\mbox{\sc cat} & \mbox{\it v} \\
\mbox{\sc per} & \mbox{\it 3} \\
\mbox{\sc num} & \mbox{\it sg}
\end{array}\right]
\oder
\left[\begin{array}{l@{\quad : \quad}l}
\mbox{\sc cat} & \mbox{\it v} \\
\mbox{\sc num} & \mbox{\it pl}
\end{array}\right]
\end{equation}
%%%
This can alternatively be written as follows.
%%
\begin{equation}
[\mbox{\sc cat} : \mbox{\it v\/} ] \und
(\nicht ([\mbox{\sc per} : \mbox{\it 3\/}] \und
    [\mbox{\sc num} : \mbox{\it sg\/}]) \oder
[\mbox{\sc num} : \mbox{\it pl\/}])
\end{equation}
%%
In turn, this can be simplified.
%%
\begin{equation}
[\mbox{\sc cat} : \mbox{\it v\/} ] \und
(\nicht [\mbox{\sc per} : \mbox{\it 3\/}]
\oder [\mbox{\sc num} : \mbox{\it pl\/}])
\end{equation}
%%
This follows on the basis of the given interpretation. Since AVSs 
are not the objects themselves but descriptions thereof, we may
exchange one description of an object or class of objects by any
other description of that same object or class of objects. We call
an AVS \textbf{universally true} if it is always true, that is, if it
holds of every object.
%%
\begin{dingautolist}{192}
\item
If $\varphi$ is a tautology of propositional logic then
    $\varphi$ holds for all replacements of AVSs for the
    propositional variables.
\item
If $\varphi$ is universally true, then so is
$[\mbox{\sc X} : \varphi]$.
\item 
$[\mbox{\sc X} : \varphi \pf \chi]. \pf .
    [\mbox{\sc X} : \varphi] \pf [\mbox{\sc X} : \chi]$.
\item
If $\varphi$ and $\varphi \pf \chi$ are universally true then 
so is $\chi$.
\end{dingautolist}
%%
%%%
\index{attribute!definite}%%
%%%
In order to avoid having to use $\pf$, we shall write 
$\varphi \leq \chi$ if $\varphi \pf \chi$ is universally true.
Most attributes are \textbf{definite}, that is, they can have at most
one value in any object. For such attributes we also have
%%
\begin{equation}
[\mbox{\sc X} : \varphi] \und
    [\mbox{\sc X} : \chi]. \pf .
    [\mbox{\sc X} : \varphi \und \chi]
\end{equation}
%%
Definite attributes are the norm. Sometimes, however, one
needs nondefinite attributes; they are called
%%%
\index{attribute!set valued}%%
%%%
\textbf{set valued} to distinguish them from the definite ones.

\index{second order logic}%%
\index{second order logic!monadic}%%
\index{MSO (see monadic second order logic)}%%%
\index{SO (see second order logic)}
%%%
The AVSs are nothing but an alternative notation for formulae 
of some logical language. In the literature, two different kinds 
of logical languages have been proposed. Both serve the purpose 
equally well. The first is the so--called \textbf{monadic second order 
predicate logic} ($\mathsf{MSO}$), which is a fragment of \textbf{second 
order logic} ($\mathsf{SO}$). Second order logic extends standard first 
order predicate logic as follows. There additionally are variables 
and quantifiers for predicates of any given arity $n \in \omega$. 
The quantifiers are also written $\forall$ and $\exists$ and the 
variables are $P^n_i$, $n, i \in \omega$. Here, $n$ tells us that 
the variable is a variable for $n$--ary relations. So, 
$\mbox{\rm PdV} := \{P^n_i : n, i \in \omega\}$ is the set of 
predicate variables 
for unary predicates and $V := \{x_i : i \in \omega\}$ the set of 
object variables. We write $P^n_i(\vec{x})$ to say that $P^n_i$ applies 
to (the $n$--tuple) $\vec{x}$. If $\varphi$ is a formula so are 
$(\forall P^n_i)\varphi$ and $(\exists P^n_i)\varphi$. The set of 
(M)SO--formulae defined over a given signature $\Omega$ is 
denoted by $\mathsf{SO}(\Omega)$ and $\mathsf{MSO}(\Omega)$, respectively. 
%%%%
\index{$\mathsf{MSO}(\Omega)$}%%
%%%%
The structures are the same as
%%%
\index{structure}%%
%%%
those of predicate logic (see Section~\ref{kap3}.\ref{kap3-6}): 
triples $\GM = \auf M, \{f^{\GM} : f \in F\}, %
\{r^{\GM} : r \in R\}\zu$, where $M$ is a nonempty set,
$f^{\GM}$ the interpretation of the function $f$ in $\GM$ and
$r^{\Gm}$ the interpretation of the relation $r$. A \textbf{model}
%%%
\index{model}%%
%%%
is a triple $\auf \GM, \gamma, \beta\zu$ where $\GM$ is a structure
$\beta \colon V \pf M$ a function assigning to each variable an
element from $M$ and $\gamma \colon P \pf \wp(M)$ a function assigning
to each $n$--ary predicate variable an $n$--ary relation on
$M$. The relation $\auf \GM, \gamma, \beta\zu \vDash \varphi$ is
defined inductively.
%%
\begin{equation}
\auf \GM, \gamma, \beta\zu \vDash P^n_i(\vec{x})
\quad:\Dpf\quad
\beta(\vec{x}) \in \gamma(P^n_i)
\end{equation}
%%
We define $\gamma \sim_P \gamma'$ if $\gamma'(Q) = \gamma(Q)$
for all $Q \neq P$.
%%
\begin{align}
%$$\begin{array}{l@{\quad\Dpf\quad}l@{\; : \;}l}
\auf \GM, \gamma, \beta\zu \vDash (\forall P)\varphi &
    :\Dpf \text{for all }\gamma' \sim_P \gamma : &&
    \auf \GM, \gamma', \beta\zu \vDash \varphi \\
\auf \GM, \gamma, \beta\zu \vDash (\exists P)\varphi &
    :\Dpf \text{for some }\gamma' \sim_P \gamma : &&
    \auf \GM, \gamma', \beta\zu \vDash \varphi
%\end{array}$$
\end{align}
%%
We write $\GM \vDash \varphi$ iff for all $\gamma$ and $\beta$
$\auf \GM, \gamma, \beta\zu \vDash \varphi$. $\mathsf{MSO}$ is that
fragment of $\mathsf{SO}$ which has only predicate variables for 
unary relations ($n = 1$). When using MSO we drop the superscript 
`$1$' in the variables $P^1_i$.

Another type of languages that have been proposed are modal
languages (see \cite{blackburn:avstructures} and
\cite{kracht:av}). We shall pick out one specific language that 
is actually an extension of the ones proposed in the quoted 
literature, namely \textbf{quantified modal logic} (QML). 
%%%
\index{quantified modal logic}%%%
\index{QML (see quantified modal logic)}%%
%%%
This language possesses a denumerably infinite set $\mbox{\it PV} 
:= \{p_i : i \in \omega\}$ of proposition variables, a set {\rm Md} of 
so--called \textbf{modalities}, and a set {\rm Cd} of propositional constants. 
And finally, there are the symbols $\nicht$, $\und$, $\oder$, $\pf$, 
$[-]$, $\auf -\zu$, $\forall$ and $\exists$. Formulas (called 
propositions) are defined inductively in the usual way. Moreover, 
if $\varphi$ is a proposition, so is $(\forall p_i)\varphi$ and
$(\exists p_i)\varphi$.
%%%
\index{Kripke--frame}%%
\index{Kripke--model}%%
%%%
The notions of Kripke--frame and Kripe--model remain the same.
A \textbf{Kripke--frame} is a triple $\auf F, R, C\zu$, where 
$R : \mbox{\rm Md} \pf \wp(F^2)$ and $C : \mbox{\rm Cd} \pf \wp(F)$.
If $m$ is a modality, $R(m)$ is the accessibility relation 
associated with $m$. In particular, we have 
%%%
\begin{equation}
\auf \GF, \beta, x\zu \vDash \auf m\zu \varphi \quad:\Dpf\quad
\text{there is }y: x\; R(m)\; y \text{ and } \auf \GF, \beta, y\zu 
	\varphi
\end{equation}
%%%
For the constants we put
%%%
\begin{equation}
\auf \GF, \beta, x\zu \vDash c \quad:\Dpf\quad x \in C(c)
\end{equation}
%%%
%%%
We define
%%
\begin{equation}
\begin{array}{rl@{\quad :\Dpf \quad}rl}
\auf \GF, \beta, x\zu & \vDash (\forall p)\varphi   &
    \text{for all }\beta' \sim_p \beta:
    & \auf \GF, \beta', x\zu \vDash \varphi \\
\auf \GF, \beta, x\zu & \vDash (\exists p)\varphi &
    \text{for some }\beta' \sim_p \beta:
    & \auf \GF, \beta', x\zu \vDash \varphi
\end{array}$$
\end{equation}
%%
We write $\auf \GF, \beta\zu \vDash \varphi$ if for all $x \in F$
$\auf \GF, \beta,x\zu \vDash \varphi$; we write $\GF \vDash \varphi$,
if for all $\beta$ we have $\auf \GF, \beta\zu \vDash \varphi$.

We define an embedding of $\mathsf{QML}(\Omega)$ into 
%%%
\index{$\Omega^m$}%%
%%%%
$\mathsf{MSO}(\Omega^m)$, where $\Omega^m$ is defined as follows. 
Let $R := \{r^m : m \in \mbox{\rm Md}\}$ 
and $\mbox{\rm Cd} := \{Q^c : c \in K\}$. $\Omega^m(r^m) := 2$, 
$\Omega^m(Q^c) := 1$. Then define $\varphi^{\dagger}$ as in 
Table~\ref{tab:qml2mso}.
%%
\begin{table}
\caption{Translating $\mathsf{QML}$ into $\mathsf{MSO}$}
\label{tab:qml2mso}
$$\begin{array}{rl@{\qquad}rl}
p_i^{\dagger} & := P_i(x_0) &
c^{\dagger} & := Q^c(x_0) \\
(\nicht \varphi)^{\dagger} & := \nicht \varphi^{\dagger} &
(\varphi_1 \und \varphi_2)^{\dagger} &
    := \varphi_1^{\dagger} \und \varphi_2^{\dagger} \\
(\varphi_1 \oder \varphi_2)^{\dagger} &
    := \varphi_1^{\dagger} \oder \varphi_2^{\dagger} &
(\varphi_1 \pf \varphi_2)^{\dagger} & := 
	\varphi_1^{\dagger} \pf \varphi_2^{\dagger} \\
((\forall p_i)\varphi)^{\dagger} & := (\forall P_i)\varphi^{\dagger} &
((\exists p_i)\varphi)^{\dagger} & := (\exists P_i)\varphi^{\dagger} \\
([m]\varphi)^{\dagger} &
\multicolumn{3}{l}{:= (\forall x_0)(r^m(x_0, x_i) \pf %%
    [x_i/x_0]\varphi^{\dagger})} \\
(\auf m\zu\varphi)^{\dagger} &
\multicolumn{3}{l}{:= (\exists x_0)(r^m(x_0, x_i) \und %%
	[x_i/x_0]\varphi^{\dagger})}
\end{array}$$
\end{table}
%%
Here in the last two clauses $x_i$ is a variable that does not
already occur in $\varphi^{\dagger}$. Finally, if $\GF$ is a 
Kripke--frame, we define an MSO--structure $\GF^m$ as follows. 
The underlying set is $F$, $(r^m)^{\GF^m} := R(m)$ for every 
$m \in \mbox{\rm Md}$, and $(Q^c)^{\GF^m} := C(c)$.
Now we have
%%
\begin{thm}
Let $\varphi \in \mathsf{QML}(\Omega)$. Then $\varphi^{\dagger} \in
\mathsf{MSO}(\Omega^m)$. And for every Kripke--frame $\GF$: 
$\GF \vDash \varphi$ iff $\GF^m \vDash \varphi^{\dagger}$.
\end{thm}
%%
\proofbeg
We shall show the following. Assume $\beta \colon \mbox{\it PV}
\pf \wp(F)$ is a valuation in $\GF$ and that $x \in F$ and
$\gamma \colon \mbox{\rm PdV} \pf \wp(F)$ and $\delta \colon V 
\pf F$ valuations for the predicate and the object variables. Then if
$\gamma(P_i) = \beta(p_i)$ for all $i \in \omega$ and
$\delta(x_0) = x$ we have
%%
\begin{equation}
\label{eq:61dddd}
\auf \GF, \beta, x\zu \vDash \varphi
    \quad\Dpf\quad
    \auf \GF^m, \gamma, \delta\zu \vDash \varphi^{\dagger}
\end{equation}
%%
It is not hard to see that \eqref{eq:61dddd} allows to derive the 
claim. We prove \eqref{eq:61dddd} by induction. If $\varphi = p_i$ 
then $\varphi^{\dagger} = P_i(x_0)$ and the claim holds in virtue 
of the fact that $\beta(p_i) = \gamma(P_i)$ and $\gamma(x_0) = x$. 
Likewise for $\varphi = c \in C$. The steps for $\nicht$, $\und$, 
$\oder$ and $\pf$ are routine. Let us therefore consider
$\varphi = (\exists p_i)\eta$. Let $\auf \GF, \beta, x\zu \vDash
\varphi$. Then for some $\beta'$ which differs from $\beta$
at most in $p_i$: $\auf \GF, \beta', x\zu \vDash \eta$. Put
$\gamma'$ as follows: $\gamma'(P_i) := \beta'(p_i)$ for all
$i \in \omega$. By induction hypothesis
$\auf \GF^m, \gamma', \delta\zu \vDash \eta^{\dagger}$ and
$\gamma'$ differs from $\gamma$ at most in $P_i$.
Therefore we have $\auf \GF^m, \gamma, \delta\zu \vDash
(\exists P_i)\eta^{\dagger} = \varphi^{\dagger}$, as desired.
The argument can be reversed, and the case is therefore settled.
Analogously for $\varphi = (\forall P_i)\eta$. Now for
$\varphi = \auf m\zu\eta$. Let $\auf \GF, \beta, x\zu \vDash %
\varphi$. Then there exists a $y$ with
$x\; r^m\, y$ and $\auf \GF, \beta, y\zu \vDash \eta$. Choose
$\delta'$ such that $\delta'(x_0) = y$ and
$\delta'(x_i) = \delta(x_i)$ for every $i > 0$. Then by
induction hypothesis $\auf \GF^m, \gamma, \delta'\zu\vDash
\eta^{\dagger}$. If $x_i$ is a variable that does not occur in
 $\eta^{\dagger}$ then let $\delta''(x_i) := \delta'(x_i)$,
$\delta''(x_0) := x$ and $\delta''(x_j) := \delta'(x_j)$ for all
$j \not\in \{0,i\}$. Then $\auf \GF^m, \gamma, \delta''\zu\vDash
r^m(x_0,x_i); [x_i/x_0]\eta^{\dagger} = \varphi^{\dagger}$. Hence
we have $\auf \GF^m, \gamma, \delta''\zu \vDash  \varphi^{\dagger}$.
Now it holds that $\delta''(x_0) = x = \delta(x_0)$ and $x_i$ is
bound. Therefore also $\auf \GF^m, \gamma, \delta\zu \vDash
\varphi^{\dagger}$.  Again the argument is reversible,
and the case is proved. Likewise for $\varphi = [m]\eta$.
\proofend
%%
%\vplatz
%\exercise
%Prove Proposition~\ref{prop:disjunkt}.
%%
\vplatz
\exercise
Let $1$, $2$ and $3$ be modalities. Show that a Kripke--frame 
satisfies the following formula iff $R(3) = R(1) \cup R(2)$.
%%
\begin{equation}
\auf 3\zu p \dpf \auf 1\zu p \oder \auf 2\zu p
\end{equation}
%%
\vplatz
\exercise
Let $1$, $2$ and $3$ be modalities. Show that a Kripke--frame 
satisfies the following formula iff $R(3) = R(1) \circ R(2)$.
%%
\begin{equation}
\auf 3\zu p \dpf \auf 1\zu \auf 2\zu p
\end{equation}
%%%
\vplatz
\exercise
In HPSG one writes $[\mbox{\sc cat} : \alpha \oplus \beta]$ if 
{\sc cat} has at least two values: $\alpha$ and $\beta$. (If 
$\alpha\und\beta$ is consistent, {\sc cat} can take also one 
value, $\alpha\und\beta$, but not necessarily.) Devise a 
translation into $\mathsf{QML}$ for $\oplus$. What if $[\mbox{\rm cat} 
: \alpha\oplus\beta]$ also means that {\sc cat} can have no other 
value than $\alpha$ and $\beta$?
%%
\vplatz
\exercise
\label{ex:transclose}
Let $r$ be a binary relation symbol. Show that in a model of
$\mathsf{MSO}$ the following holds:
$\auf \GM, \gamma, \beta\zu \vDash Q(x,y)$ iff
$x \; (r^{\GM})^{\ast}\; y$ (this means that $y$ can be reached from
$x$ in finitely many $r$--steps).
%%
\begin{equation}
Q(x,y) := (\forall P)(P(x) \und (\forall yz)(P(y) \und
    y\; r\; z .\pf .P(z)). \pf . P(y))
\end{equation}
%%

 \section{Axiomatic Classes I: Strings}
\label{kap5-2}
%
%
%
For the purposes of this chapter we shall code strings in a new
way. This will result in a somewhat different formalization than 
the one discussed in Section~\ref{kap1}.\ref{kap1-4}. The differences are, 
however, marginal.
%%%%
\nocite{ebbinghausflum:finite}
%%%
\begin{defn}
%%%
\index{Z--structure}%%
%%%
A \textbf{Z--structure over the alphabet} $A$ is a triple of the
form $\GL = \auf L, \prec, \{Q_a : a \in A\}\zu$, where $L$ is 
an arbitrary finite set, $\{Q_a : a \in A\}$ a partition of $L$ 
and $\prec$ a binary relation on $L$ such that both $\prec^+$ 
and its inverse are linear, irreflexive, total orderings of 
$L$.
\end{defn}
%%
Z--structures are not strings. However, it is not difficult to 
define a map which assigns a string to each Z--structure. However,
if $L \neq \varnothing$ there are infinitely many Z--structures 
which have the same string associated with them and they form 
a proper class.

Fix $A$ and denote by $\mathsf{MSO}$ the MSO--language of the binary 
relation symbol $\uli{\prec}$ as well as a unary predicate constant
$\uli{a}$ for every $a \in A$.
%%%
\begin{defn}
%%%
\index{theory!MSO--}%%
\index{model class}%%
%%%
Let $\CK$ be a set or class of Z--structures over an alphabet
$A$. Then $\mathsf{Th}\, \CK$ denotes the set
$\{\varphi \in \mathsf{MSO} : \mbox{ for all }\GL  \in
\CK: \GL \vDash \varphi\}$, called the \textbf{MSO--theory
of} $\CK$. If $\Phi$ is a set of sentences from $\mathsf{MSO}$ then 
let $\mathsf{Mod}\, \Phi$ be the set of all $\GL$ which satisfy 
every sentence from $\Phi$. $\mathsf{Mod}\, \Phi$ is called the 
\textbf{model class of} $\Phi$.
\end{defn}
%%
Recall from Section~\ref{kap1}.\ref{kap1-1} the notion of a context. It is
easy to see that $\vDash$ together with the class of Z--structures
and the MSO--formulae form a context. From this we directly get the
following
%%
\begin{thm}
The map $\rho \colon \CK \mapsto \mathsf{Mod}\, \mathsf{Th}\, \CK$
is a closure operator on the class of of classes Z--structures over
$A$. Likewise, $\lambda \colon \Phi \mapsto \mathsf{Th}\, \mathsf{Mod}\,
\Phi$ is a closure operator on the set of all subsets of
$\mathsf{MSO}$.
\end{thm}
%%
(We hope that the reader does not get irritated by the difference
between classes and sets. In the usual set theory one has to
distinguish  between sets and classes. Model classes are except for
trivial exception always classes  while classes of formulae are
always sets, because they are subclasses of the set of formulae.
This difference can be neglected in what is to follow.)
%%%%
\index{logic}%%
\index{class!axiomatic}%%
\index{class!finitely MSO--axiomatisable}%%
%%%
We now call the sets of the form $\lambda(\CK)$ \textbf{logics} and
the classes of the form $\rho(\Phi)$ \textbf{axiomatic classes}. A
class is called \textbf{finitely MSO--axiomatizable} if it has the 
form $\mathsf{Mod}(\Phi)$ for a finite $\Phi$, while a logic is 
\textbf{finitely MSO--axiomatizable} if it is the logic of a finitely
axiomatizable class. We call a class of Z--structures over $A$
\textbf{regular} if it is the class of all Z--structures of a regular
language. Formulae are called \textbf{valid} if they hold in all 
structures. The following result from \cite{buechi:weak} is the 
central theorem of this section.
%%
\begin{thm}[B\"uchi]
\index{B\"uchi, J.}%%
\label{thm:buechi} 
A class of Z--structures is finitely
MSO--axiomatizable iff it corresponds to a regular language 
which does not contain $\varepsilon$.
\end{thm}
%%
This sentence says that with the help of $\mathsf{MSO}$ we can only 
define regular classes of Z--structures. If one wants to describe 
nonregular classes, one has to use stronger logical languages (for 
example $\mathsf{SO}$). The proof of this theorem requires a lot of 
work. Before we begin, we have to say something about the formulation
of the theorem. By definition, models are only defined on
nonempty sets. This is why a model class always defines a
language not containing $\varepsilon$. It is possible to change
this but then the Z--structure of $\varepsilon$ (which is
actually unique) is a model of every formula, and then
$\varnothing$ is regular but not MSO--axiomatizable.
So, complete correspondence cannot be expected. But this is the 
only exception.

Let us begin with the simple direction. This is the claim that
every regular class is finitely MSO--axiomatizable. Let $\CK$ be 
a regular class and $L$ the corresponding regular language. Then 
there exists a finite state automaton
$\GA = \auf A, Q, i_0, F, \delta\zu$ with $L(\GA) = L$. We
may choose $Q := n$ for a natural number $n$ and $i_0 = 0$.
Look at the sentence $\delta(\GA)$ defined in Table~\ref{tab:diagram}.
%%
\begin{table}
\caption{The Formula $\delta(\GA)$}
%%%
\index{$\delta(\GA)$}%%%
%%%
\label{tab:diagram}
$$\begin{array}{lll@{\quad}l}
\delta(\GA) :=
    & & (\forall xyz)(x \uli{\prec} y \und
    x \uli{\prec} z. \pf .y \doteq z)  & \mbox{\rm (a)} \\
    & \und & (\forall xyz)(y \uli{\prec} x \und
    z \uli{\prec} x. \pf .y \doteq z) & \mbox{\rm (b)} \\
    & \und & (\forall P)\{(\forall xy)(x \uli{\prec} y. \pf
        .P(x) \pf P(y)) \und (\exists x)P(x). \\ 
	& & \quad 
	\pf .  (\exists x)(P(x) \und (\forall y)(x \uli{\prec} y 
        \pf \nicht P(y)))\} & \mbox{\rm (c)} \\
    & \und & (\forall P)\{(\forall xy)(x \uli{\prec} y. \pf
        .P(y) \pf P(x)) \und (\exists x)P(x).  \\
	& & \quad 
	\pf .(\exists x)(P(x) \und (\forall y)(y \uli{\prec} x 
        \pf \nicht P(y)))\} & \mbox{\rm (d)} \\
    & \und & (\forall P)\{(\forall xy)(x \uli{\prec} y \pf
        (P(x) \dpf P(y))) \\
	& & \quad \und (\exists x)P(x). \pf. 
	(\forall x)P(x)\} &
            \mbox{\rm (e)} \\
    & \und & (\forall x)\goder_{a \in A} \uli{a}(x) &
        \mbox{\rm (f)} \\
    & \und & (\forall x)\gund_{a \neq b} \uli{a}(x) \pf \nicht
        \uli{b}(x) & \mbox{\rm (g)} \\
    & \und & (\exists P_0P_1\dotsb P_{n-1})\{(\forall x)((\forall y)\nicht(y
        \uli{\prec} x).  \pf . P_0(x)) \\
        &      & \qquad \und (\forall x)((\forall y)\nicht (x
        \uli{\prec} y).  \pf .
        \goder_{i \in F} P_i(x)) \\
        &      & \qquad \und (\forall xy)(x \uli{\prec} y \pf
        \gund_{a \in A}  [\uli{a}(y) \und P_i(x). \\
        & & \qquad\qquad \pf . \goder_{j \in \delta(i,a)} P_j(y)])\} & 
\mbox{\rm (h)}
\end{array}$$
\end{table}
%%
\begin{lem}
Let $\GL$ be an MSO--structure. Then $\GL \vDash \delta(\GA)$ iff 
$\GL$ is a Z--structure and its associated string is in $L(\GA)$.
\end{lem}
%%
\proofbeg
Let $\GL \vDash \delta(\GA)$ and let $\GL$ be an
MSO--structure. Then there exists a binary relation
$\prec$ (the interpretation of $\uli{\prec}$) and for every
$a \in A$ a subset $Q_a \subseteq L$. By (a) and (b)
an element $x$ has at most one $\prec$--successor and at most 
one $\prec$--predecessor. By (c), every nonempty subset which 
is closed under $\prec$--successors contains a last element,  and
by (d) every nonempty subset which is closed under $\prec$--predecessors
contains a first element. Since $L$ is not empty, it has a least 
element, $x_0$. Let $H := \{x_i : i < \kappa\}$ be a maximal set 
such that $x_{i+1}$ is the (unique) $\prec$--successor of $x_i$.  
$H$ cannot be infinite, for otherwise $\{x_i : i < \omega\}$ 
would be a successor closed set without last element. So, $H$ 
is finite. $H$ is also closed under predecessors. So, $H$ is a 
maximal connected subset of $L$. By (e), every maximal connected 
nonempty subset of $L$ is identical to $L$. So, $H = L$, and hence
$L$ is finite, connected, and linear in both directions.

Further, by (f) and (g) every $x \in L$ is contained in exactly one 
set $Q_a$. Therefore $\GL$ is a Z--structure. We have to show that its 
string is in $L(\GA)$. (h) says that we can find sets 
$H_i \subseteq L$ for $i < n$ such that if $x$ is the first element 
with respect to $\prec$ then $x \in H_0$, if $x$ is the last element 
with respect to $\prec$ then $x \in H_j$ for some $j \in F$ and if 
$x \in  H_i$, $y \in Q_a$, $x \prec y$ then $y \in H_j$ for some 
$j \in \delta(i,a)$. This means that the string is in $L(\GA)$. (There 
is namely a biunique correspondence between accepting runs of the 
automaton and partitions into $H_i$. Under that partition $x \in H_i$ 
means exactly that the automaton is in state $i$ at $x$ in that run.) 
Now let $\GL \nvDash \delta(\GA)$. Then either $\GL$ is not a 
Z--structure or there exists no accepting run of the automaton $\GA$. 
Hence the string is not in $L(\GA)$.
This concludes the proof.
\proofend

Notice that we can define $<$, the transitive closure of $\prec$, and
$>$, the transitive closure of $\succ$ (and converse of $<$) by an 
MSO--formula (see Exercise~\ref{ex:transclose}). 
Further, we will write $x \leq y$ for $x < y \oder x \doteq y$, 
and $x \prec y$ in place of $x \uli{\prec} y$. Now, given a 
Z--structure $\GL = \auf L, \prec, \{Q_a : a \in A\}\zu$ put
%%
\begin{equation}
M(\GL) := \auf L, \prec, \succ, <, >, \{Q_a : a \in A\}\zu 
\end{equation}
%%
\index{MZ--structure}%%
%%%%
A structure of the form $M(\GL)$ we call an \textbf{MZ--structure}.

Now we shall prove the converse implication of 
Theorem~\ref{thm:buechi}. To this end we shall make a detour. We put 
$M := \{+,-,\prec, \succ\}$ and $C := \{c_a : a \in A\}$.
Then we call $\mathsf{QML}$ the language of quantified modal logic 
with basic modalities from $M$ and propositional constants from 
$C$. Now we put
%%
\begin{equation}
\begin{array}{ll@{}l@{\und}l}
\sigma := &  & \auf\prec\zu p \pf \auf +\zu p &
    \auf\succ\zu p \pf \auf-\zu p \\
  & \und & p \pf [\succ] \auf\prec\zu p &
    p \pf [\prec]\auf \succ\zu p \\
  & \und & \auf \prec \zu p \pf [\prec]p &
    \auf\succ\zu p \pf [\succ]p \\
  & \und & \auf+\zu\auf+\zu p \pf \auf+\zu p
    & \auf-\zu\auf-\zu p \pf \auf-\zu p \\
  & \und & [+]([+] p \pf p) \pf [+] p &
    [-]([-]p \pf p) \pf [-]p  \\
  & \und & \gund_{a \neq b} c_a \pf \nicht c_b 
   & \goder_{a \in A} c_a
\end{array}
\end{equation}
%%%
\index{structures!connected}%%
%%%
We call a structure \textbf{connected} if it is not of the form
$\GF \oplus \GG$ with nonempty $\GF$ and $\GG$. As already 
mentioned, QML--formulae cannot distinguish between 
connected and nonconnected structures.
%%
\begin{thm}
Let $\GF$ be a connected Kripke--frame for $\mathsf{QML}$.
Put $Z(\GF) := \auf F, R(\prec), R(\succ), R(+), R(-),
\{K(c^a) : a \in A\}\zu$. $\GF \vDash \sigma$ iff
$Z(\GF)$ is an MZ--struc\-ture over $A$.
\end{thm}
%%
\proofbeg
The proof is not hard but somewhat lengthy. It consists of the 
following facts (the others are dual to these).
%%%
\begin{dingautolist}{192}
\item 
{\it $\GF \vDash \auf\prec\zu p \pf \auf +\zu p$ 
iff $R(\prec) \subseteq R(+)$}. 
\item {\it $\GF \vDash p \pf [\succ] \auf\prec\zu p$ iff
$R(\succ) \subseteq R(\prec)^{\smallsmile}$.} 
\item
{\it $\GF \vDash \auf \prec\zu p \pf [\prec] p$ iff
every point has at most one $R(\prec)$--successor.}, 
\item 
{\it $\GF \vDash
\auf+\zu\auf+\zu p \pf \auf+\zu p$ iff $R(+)$ is
transitive.} 
\item
{\it $\GF \vDash [+](p \pf [+] p) \pf [+] p$ iff 
$R(+)$ is transitive and free of infinite ascending chains 
	(or cycles).} 
\end{dingautolist}
%%%
We show only \ding{192} and \ding{196}.
\ding{192} For this let $\GF \nvDash \auf\prec\zu p \pf \auf+\zu p$. Then 
there exists $\beta$ and $x$ such that
$\auf \GF, \beta, x\zu \vDash \auf \prec\zu p; \nicht \auf+\zu p$.
This means that there is a $y \in F$ with $y \in \beta(p)$
and $x\; R(\prec)\; y$. If $x \; R(+)\; y$ then $x \vDash \auf+\zu p$, 
contradiction. This shows $R(\prec) \nsubseteq R(+)$. Assume conversely
$R(\prec) \nsubseteq R(+)$.  Then there exist $x$ and $y$ such
that $x\; R(\prec)\; y$ but not $x\; R(+)\; y$. Now set
$\beta(p) := \{y\}$. Then we have $\auf \GF, \beta, x\zu \vDash
\auf \prec\zu p; \nicht\auf+\zu p$. 
\ding{196} Because of \ding{195} we restrict ourselves to 
proving this for transitive $R(+)$. Assume $\GF \nvDash 
[+]([+]p \pf p) \pf [+]p$. Then there exists a $\beta$ and 
a $x_0$ with $\auf \GF, \beta, x\zu \vDash
[+](p \pf [+]p); \auf +\zu \nicht p$. So there exists a
$x_1$ with $x_0\; R(+)\; x_1$ and $x_1 \not\in \beta(p)$.
Then $x_1 \vDash [+]p \pf p$ and therefore $x_1 \vDash %
\auf +\zu \nicht p$. Because of the transitivity of $R(+)$
we also have $x_1 \vDash [+]([+]p \pf p)$. Repeating this argument 
we find an infinite chain $\auf x_i : i \in \omega\zu$ such that
$x_i\; R(+)\; x_{i+1}$. Therefore, $R(+)$ contains an infinite 
ascending chain. Conversely, assume that $R(+)$ has an infinite 
ascending chain. Then there exists a set $\auf x_i : i \in \omega\zu$
with $x_i\; R(+)\; x_{i+1}$ for all $i \in \omega$. Put $\beta(p) :=
\{y : \mbox{ there is } i \in \omega: y \; R(+)\; x_i\}$.
Then it holds that $\auf \GF, \beta, x_0\zu
\vDash [+]([+]p \pf p); \auf+\zu \nicht p$.
For let $x\; R(+)\; y$ and suppose that $y \vDash [+]p$. 
Then there exists an $i$ with $y \; R(+)\; x_i$ (for $R(+)$ 
is transitive). Hence $y \vDash p$, whence $y \vDash [+]p \pf p$. 
Since $y$ was arbitrary, we have $x \vDash [+]([+]p \pf p)$. 
Also $x_1 \nvDash p$ and $x_0 \; R(+)\; x_1$. Hence
$x_0 \vDash \auf+\zu \nicht p$, as required. 
\proofend

Notice that a finite frame has an infinite ascending chain iff 
it has a cycle. Now we define an embedding of $\mathsf{MSO}$ into 
$\mathsf{QML}$. To this end we need some preparations. As one convinces 
oneself easily the following laws hold.
%%
%%% Verschiebe das nach Kap 4
\begin{dingautolist}{192}
\item $(\forall x)\varphi \dpf \nicht (\exists x)(\nicht
        \varphi)$, $\nicht\nicht \varphi \dpf \varphi$.
\item $(\forall x)(\varphi_1 \und \varphi_2) \dpf
	(\forall x)\varphi_1 \und (\forall x)\varphi_2$, \\
        $(\exists x)(\varphi_1 \oder \varphi_2)
	\dpf (\exists x)\varphi_1 \oder (\exists x)\varphi_2$.
\item $(\forall x)(\varphi_1 \oder \varphi_2) \dpf
	(\varphi_1 \oder (\forall x)\varphi_2)$,
	$(\exists x)(\varphi_1 \und \varphi_2) \dpf
	(\varphi_1 \und (\exists x)\varphi_2)$,
        if $x$ does not occur freely in $\varphi_1$.
%\item $(\forall x)(\varphi_1 \oder \varphi_2) \dpf
%(\varphi_2 \oder (\forall x)\varphi_1)$,
%$(\exists x)(\varphi_1 \und \varphi_2) \dpf
%(\varphi_2 \und (\exists x)\varphi_1)$,
%    if $x$ does not occur freely in $\varphi_2$.
\end{dingautolist}
%%
Finally, for every variable $y \neq x$:
%%
\begin{align}
(\forall x)\varphi(x) \dpf & \phantom{\mbox{}\und\mbox{}}
    (\forall x)(x < y \pf \varphi(x)) 
\\\notag
   & \und (\forall x)(y < x \pf \varphi(x)) \und \varphi(y)
\end{align}
%%
We now define following quantifiers.
%%
\begin{equation}
\begin{split}
(\forall x < y)\varphi & := (\forall x)(x < y \pf \varphi) \\
(\forall x > y)\varphi & := (\forall x)(x > y \pf \varphi) \\
(\exists x < y)\varphi & := (\exists x)(x < y \pf \varphi) \\
(\exists x > y)\varphi & := (\exists x)(x > y \pf \varphi)
\end{split}
\end{equation}
%%%
\index{quantifier!restricted}%%
%%%
We call these quantifiers \textbf{restricted}. Evidently, we can
replace an unrestricted quantifier $(\forall x)\varphi$
by the conjunction of restricted quantifiers.  
%%
\begin{lem}
For every MSO--formula $\varphi$ with at least one free
object variable there is an MSO--formula
$\varphi^g$ with restricted quantifiers such that
in all Z--struc\-tu\-res $\GL$: $\GL \vDash \varphi \dpf \varphi^g$.
\end{lem}
%%
We define the following functions $f$, $f^+$, $g$ and $g^+$ on 
unary predicates.
%%
\begin{equation}
\begin{split}
(f(\varphi))(x) & := (\exists y \prec x)\varphi(y) \\
(g(\varphi))(x) & := (\exists y \succ x)\varphi(y) \\
(f^+(\varphi))(x) & := (\exists y < x)\varphi(y) \\
(g^+(\varphi))(x) & := (\exists y > x)\varphi(y)
\end{split}
\end{equation}
%%
A somewhat more abstract approach is provided by the notion of
a {\it universal modality}.
%%
\begin{defn}
\index{modality!universal}%%
%%%
Let $M$ be a set of modalities and $\omega \in M$. Further, let $L$
be a $\mathsf{QML}(M)$--modal logic. $\omega$ is called a
\textbf{universal modality of} $L$ if the following
formulae are contained in $L$:
%%%
\begin{dingautolist}{192}
\item $[\omega]p \pf p$, $[\omega]p \pf [\omega][\omega]p$, 
	$p \pf [\omega]\auf \omega\zu p$.
\item $[\omega]p \pf [m]p$, for all $m \in M$.
\end{dingautolist}
\end{defn}
%%%
\begin{prop}
Let $L$ be a $\mathsf{QML}(M)$--logic, $\omega \in M$ a
universal modality and $\GF = \auf F, R\zu$ a
connected Kripke--frame with $\GF \vDash L$.
Then $R(\omega) = F \times F$.
\end{prop}
%%%
The proof is again an exercise. The logic of Z--structures allows 
to define a universal modality. Namely, set
%%
\begin{equation}
[\omega]\varphi := \varphi \und [+]\varphi \und [-]\varphi
\end{equation}
%%
This satisfies the requirements above.

The obvious mismatch between $\mathsf{MSO}$ and $\mathsf{QML}$ is that the
former allows for several object variables to occur freely,
while the latter only contains one free object variable (the world
variable), which is left implicit. However, given that $\varphi$
contains only one free object variable, we can actually massage it
into a form suitable for $\mathsf{QML}$. Let $P_x$ be a predicate
variable which does not occur in $\varphi$. Define
$\{P_x/x\}\varphi$ inductively as in Table~\ref{tab:62sub}. 
(Here, assume that $x \not\in \{v,w\}$ and $x \neq y$.)
%%
\begin{table}
\caption{Mimicking the Variables in $\mathsf{QML}$}
\index{$\{P_x/x\}\varphi$}%%%
\label{tab:62sub}
$$\begin{array}{l@{\; := \;}l@{\quad}l@{\; := \;}l}
\{P_x/x\}(x \doteq y) & P_x(y) &
\{P_x/x\}(y \doteq x)  & P_x(y) \\
%%%
\{P_x/x\}(v \doteq w) & v \doteq w & 
\{P_x/x\}(x \prec y) & (g(P_x))(y) \\
%%%
\{P_x/x\}(y \prec x)  & (f(P_x))(y) &
\{P_x/x\}(v \prec w) & v \prec w \\
%%%
\{P_x/x\}(\uli{a}(x)) & \uli{a}(x) &
\{P_x/x\}(\nicht \varphi) & \nicht \{P_x/x\}\varphi \\
%%%%
\{P_x/x\}(\varphi_1 \und \varphi_2) & \multicolumn{2}{l}{%
\{P_x/x\}\varphi_1 \und \{P_x/x\}\varphi_2} \\
%%%
\{P_x/x\}(\varphi_1 \oder \varphi_2) & \multicolumn{2}{l}{%
\{P_x/x\}\varphi_1 \oder \{P_x/x\}\varphi_2} \\
%%%%
\{P_x/x\}((\exists y)\varphi) & \multicolumn{2}{l}{%
(\exists y)\{P_x/x\}\varphi} \\
%%%
\{P_x/x\}((\exists x)\varphi) & \multicolumn{2}{l}{(\exists x)\varphi} \\
%%%%
\{P_x/x\}((\forall P)\varphi) & \multicolumn{2}{l}{%
(\forall P)\{P_x/x\}\varphi} \\
%%%
\{P_x/x\}((\exists P)\varphi) & \multicolumn{2}{l}{%
(\exists P)\{P_x/x\}\varphi}
%%%%
\end{array}$$
\end{table}
%%
Let $\gamma(P_x) = \{\beta(x)\}$. Then
%%
\begin{equation}
\auf \GM, \gamma, \beta\zu \vDash \varphi
\quad\Dpf\quad
\auf \GM, \gamma, \beta\zu \vDash \{P_x/x\}\varphi
\end{equation}
%%
\begin{lem}
Let
%%
\begin{equation}
\begin{split}
\nu(P) := (\exists x)(P(x)) & \und 
    (\forall x)(P(x) \pf \nicht (f^+(P))(x)) \\
    & \und (\forall x)(P(x) \pf \nicht (g^+(P))(x))
\end{split}
\end{equation}
%%
Then $\auf \GM, \gamma, \beta\zu \vDash \nu(P)$ iff
$\gamma(P) = \{x\}$ for some $x \in M$.
\end{lem}
%%
This is easy to show. Notice that $\nu(P)$ contains no free
occurrences of $x$. This is crucial, since it allows to
directly translate $\nu(P)$ into a $\mathsf{QML}$--formula.

Now we define an embedding of $\mathsf{MSO}$ into $\mathsf{QML}$.
Let $h \colon P \pf \mbox{\it PV\/}$ be a bijection from the
set of predicate variables of $\mathsf{MSO}$ onto the set of 
propositional variables of $\mathsf{QML}$.
%%
\begin{equation}
\begin{split}
(\uli{a}(x))^{\diamond} & := c_a &
(\nicht \varphi)^{\diamond} & := \nicht \varphi^{\diamond} \\
(P(y))^{\diamond} & := h(P) &
(\varphi_1 \und \varphi_2)^{\diamond} &
   :=  \varphi_1^{\diamond} \und \varphi_2^{\diamond} \\
(f(\varphi))^{\diamond} & := \auf\succ\zu \varphi^{\diamond} &
(\varphi_1 \oder \varphi_2)^{\diamond} &
   :=  \varphi_1^{\diamond} \oder \varphi_2^{\diamond} \\
(g(\varphi))^{\diamond} & := \auf\prec\zu \varphi^{\diamond} &
((\exists P)\varphi)^{\diamond} & := (\exists h(P))\varphi^{\diamond} \\
(f^+(\varphi))^{\diamond} & := \auf-\zu \varphi^{\diamond} &
((\forall P)\varphi)^{\diamond} & := (\forall h(P))\varphi^{\diamond} \\
(g^+(\varphi))^{\diamond} & := \auf+\zu \varphi^{\diamond} & & 
\end{split}
\end{equation}
%%
For the first--order quantifiers we put
%%%
\begin{multline}
((\forall x)\varphi(x))^{\diamond} := 
(\forall p_x)((\auf \omega\zu p_x \und
        (\forall p'_x)([\omega](p'_x \pf p_x) \\
	\pf ([\omega]\nicht p_x. \oder. [\omega](p'_x \dpf p_x)))) 
	    \pf \{p_x/p\}\varphi^{\diamond}) 
\end{multline}
and 
\begin{multline}
((\exists x)\varphi(x))^{\diamond} :=  
(\forall p_x)((\auf \omega\zu p_x \und 
	(\forall p'_x)([\omega](p'_x \pf p_x) \\
	\pf ([\omega]\nicht p_x.  \oder .[\omega](p'_x \dpf p_x)))) 
	    \und \{p_x/x\}\varphi^{\diamond})
\end{multline}
%%
The correctness of this translation follows from the fact that
%%
\begin{multline}
\auf \GF, \beta, x\zu \vDash 
\auf \omega\zu p_x \und (\forall p'_x)([\omega](p'_x \pf p_x) \\
\pf (([\omega]\nicht p_x) \oder [\omega](p'_x \dpf p_x)))
\end{multline}
%%
exactly if $\beta(p_x) = \{v\}$ for some $v \in F$.
%%
\begin{thm}
Let $\varphi \in \mathsf{MSO}$ contain at most one free 
variable, the object variable $x_0$. Then there exists a
QML--formula $\varphi^M$ such that for all Z--struc\-tu\-res
$\CL$:
%%
\begin{equation}
\auf \GL, \beta\zu \vDash \varphi(x_0)
\mbox{ iff }\auf M(\GL), \beta(x_0)\zu \vDash
\varphi(x_0)^M
\end{equation}
%%
\end{thm}
%%
\begin{cor}
\label{cor:mql}
Modulo the identification $\GL \mapsto M(\GL)$, $\mathsf{MSO}$
and $\mathsf{QML}$ define the same classes of connected
nonempty and finite Z--structures. Further: $\CK$ is a finitely
MSO--axiomatizable class of Z--structures iff $M(\CK)$ is a 
finitely QML--axiomatizable class of MZ--structures.
\end{cor}
%%
This shows that it is sufficient to prove that finitely 
QML--axiomatizable classes of MZ--structures define regular 
languages. This we shall do now. Notice that for the proof we 
only have to look at grammars with rules of the form 
$X \pf a \mid aY$ and no rules of the form $X \pf \varepsilon$. 
Furthermore, instead of regular grammars we can work with 
regular grammars$^{\ast}$, where we have a set of start symbols. 

Let $G = \auf \Sigma, N, A, R\zu$ be a regular grammar$^{\ast}$
and $\vec{x}$ a string. For a derivation of $\vec{x}$
we define a Z--structure over $A \times N$ (!) as follows.
We consider the grammar$^{\ast}$ $G^{\times} := \auf \Sigma,
N, A \times N, R^{\times}\zu$ which consists of the following
rules.
%%
\begin{align}
R^{\times} := &  \phantom{\mbox{}\cup\mbox{}}
\{X \pf \auf a,X\zu \; Y : X \pf aY \in R\} \\\notag
              & \cup \{X \pf \auf a, X\zu : X \pf a \in R\}
\end{align}
%%
The map $h \colon A \times N \pf A \colon \auf a, X\zu \mapsto a$
defines a homomorphism from $(A \times N)^{\ast}$ to
$A^{\ast}$, which we likewise denote by $h$. It also gives
us a map from Z--structures over $A \times N$ to Z--structures
over $A$. Every $G$--derivation of $\vec{x}$ uniquely defines
a $G^{\times}$--derivation of a string $\vec{x}^{\times}$
with $h(\vec{x}^{\times}) = \vec{x}$ and this in turn
defines a Z--structure
%%
\begin{equation}
\GM = \auf L, \prec, \{Q_{\auf a, X\zu}^{\GM} :
\auf a, X\zu \in A \times N\}\zu
\end{equation}
%%
From $\GM$ we define a model over the alphabet $A \cup N$, 
also denoted by $\vec{x}^{\times}$.
%%
\begin{equation}
\vec{x}^{\times} := \auf L, \prec, \{Q^{\GL}_a : a \in A\},
\{Q^{\GL}_X : X \in N\}\zu
\end{equation}
%%
Here $w \in Q^{\GL}_a$ iff $w \in Q_{\auf a,X\zu}^{\GM}$
for some $X$ and $w \in Q^{\GL}_X$ iff $w \in
Q_{\auf a,X\zu}^{\GM}$ for some $a \in A$. 
%%
\begin{defn}
\index{grammar$^{\ast}$!faithful}%%
\index{code}%%
\index{formula!codable}%%%
%%%%
Let $G = \auf \Sigma, N,A,R\zu$ be a regular grammar$^{\ast}$
and $\varphi \in \mathsf{QML}$ a constant formula (with constants 
for elements of $A$). We say, $G$ is \textbf{faithful to} $\varphi$
if there is a subset $H \subseteq N$ such that for
every string $\vec{x}$, every $\vec{x}^{\times}$ and every
$w$: $\auf \vec{x}^{\times}, w\zu \vDash \varphi$ iff
there exists $X \in H$ with $w \in Q_X$. We say, $H$ 
\textbf{codes} $\varphi$ \textbf{with respect to} $G$.
\end{defn}
%%
The idea behind this definition is as follows. Given a set
$H$ and a formula $\varphi$, $H$ codes $\varphi$
with respect to $G$ if in every derivation of a string
$\vec{x}$ $\varphi$ is true in $\vec{x}^{\times}$ at exactly 
those nodes where the nonterminal $H$ occurs. The reader may 
convince himself of the following facts.
%%
\begin{prop}
\label{prop:boolcode}
Let $G$ be a regular grammar$^{\ast}$ and let $H$
code $\varphi$ and $K$ code $\chi$ with respect to $G$.
Then the following holds.
%%
\begin{dingautolist}{192}
\item $N - H$ codes $\nicht \varphi$ with respect to $G$.
\item $H \cap K$ codes $\varphi \und \chi$ with respect to $G$.
\item $H \cup K$ codes $\varphi \oder \chi$ with respect to $G$.
\end{dingautolist}
\end{prop}
%%
We shall inductively show that every QML--formula
can be coded in a regular grammar$^{\ast}$ on condition that
one suitably extends the original grammar.
%%
\begin{defn}
\index{product of grammars$^{\ast}$}%%
%%%
Suppose that $G_1 = \auf \Sigma_1, N_1, A, R_1\zu$ and
$G_2 = \auf \Sigma_2, N_2, A, R_2\zu$ are regular 
grammars$^{\ast}$. Then put
%%
\begin{equation}
\index{$G_1\times G_2$}%%%
%%%
G_1 \times G_2 := \auf \Sigma_1\times\Sigma_2, N_1 \times N_2,
A, R_1 \times R_2\zu
\end{equation}
%%
where
%%
\begin{align}
R_1 \times R_2 := & \phantom{\mbox{}\cup\mbox{}}
\{\auf X_1,X_2\zu \pf a \; \auf Y_1, Y_2\zu :
    X_i \pf aY_i \in R_i\} \\\notag
    & \cup  \{\auf X_1, X_2\zu \pf a : X_i \pf a \in R_i\}
\end{align}
%%
\index{product}%%
%%%%
We call $G_1 \times G_2$ the \textbf{product} of the 
grammars$^{\ast}$ $G_1$ and $G_2$.
\end{defn}
%%
We have
%%
\begin{prop}
Let $G_1$ and $G_2$ be grammars$^{\ast}$ over $A$. Then
$L(G_1 \times G_2) = L(G_1) \cap L(G_2)$.
\end{prop}
%%
The following theorem is not hard to show and therefore left
as an exercise.
%%
\begin{lem}
\label{lem:prodcode}
Let $\varphi$ be coded in $G_2$ by $H$. Then
$\varphi$ is coded in $G_1 \times G_2$ by $N_1 \times H$ and
in $G_2 \times G_1$ by $H \times N_1$.
\end{lem}
%%
\begin{defn}
%%%
\index{code}%%
%%%
Let $\varphi \in \mathsf{QML}$. A \textbf{code}
for $\varphi$ is a pair $\auf G, H\zu$ where $L(G) = A^{\ast}$
and $H$ codes $\varphi$ with respect to $G$. $\varphi$
is called \textbf{codable} if it has a code.
\end{defn}
%%
Assume $\auf G, H\zu$ is a code for $\varphi$ and let
$G'$ be given. Then we have $L(G' \times G) = L(G')$ and
$\varphi$ is coded in $G' \times G$ by $N' \times H$.
Therefore, it suffices to name just one code for every
formula. Moreover, the following fact makes life simpler 
for us.
%%
\begin{lem}
Let $\Delta$ be a finite set of codable formulae.
Then there exists a grammar$^{\ast}$ $G$ and sets $H_{\varphi}$,
$\varphi \in \Delta$, such that $\auf G, H_{\varphi}\zu$
is a code of $\varphi$.
\end{lem}
%%
\proofbeg
%%
\newcommand{\bigtimes}{\mathsf{X}}%%
%%%
Let $\Delta := \{\delta_i : i < n\}$
and let $\auf G_i, M_i\zu$ be a code of $\delta_i$,
$i < n$. Put $G := \bigtimes_{i < n} G_i$
and $H_i := \bigtimes_{j < i} N_i \times
H_i \times \bigtimes_{i < j < n} N_j$.
Iterated application of Lemma~\ref{lem:prodcode}
yields the claim.
\proofend
%%%
\begin{thm}[Coding Theorem]
\index{Coding Theorem}%%%
%%%
\label{thm:code}
Every constant QML--formula is codable.
\end{thm}
%%
\proofbeg
The proof is by induction over the structure of the formula.
We begin with the code of $\uli{a}$, $a \in A$. Define
the following grammar$^{\ast}$ $G_{a}$. Put $N := \{X, Y\}$ and
$\Sigma := \{X, Y\}$. The rules are
%%
\begin{equation}
X \pf a \mid aX \mid aY, \quad Y \pf b \mid bX \mid bY 
\end{equation}
%%
where $b$ ranges over all elements from $A - \{a\}$. The
code is $\auf G_a, \{X\}\zu$ as one easily checks.
The inductive steps for $\nicht$, $\und$, $\oder$
and $\pf$ are covered by Proposition~\ref{prop:boolcode}.
Now for the case $\varphi = \auf \prec\zu \eta$. We assume that
$\eta$ is codable and that $C_{\eta} = \auf G_{\eta}, H_{\eta}\zu$
is a code. Now we define $G_{\varphi}$. Let
$N_{\varphi} := N_{\eta}\times \{0,1\}$ and $\Sigma_{\varphi} :=
\Sigma_{\eta} \times \{0,1\}$. Finally, let the rules be of the 
form
%%
\begin{equation}
\auf X,1\zu \pf a\; \auf Y,0\zu, \quad
\auf X,1\zu \pf a\; \auf Y,1\zu,
\end{equation}
%%
where $X \pf aY \in R_{\eta}$ and $Y \in H_{\eta}$ and
of the form
%%
\begin{equation}
\auf X,0\zu \pf a\; \auf Y,0\zu, \quad
\auf X,0\zu \pf a\; \auf Y,1\zu,
\end{equation}
%%
where $X \pf aY \in R_{\eta}$ and $Y \not\in H_{\eta}$. Further,
we take all rules of the form
%%
\begin{equation}
\auf X,0\zu \pf a
\end{equation}
%%
for $X \pf a \in R_{\eta}$. One easily checks that for every
rule $X \pf aY$ or $X \pf a$ there is a rule $G_{\varphi}$.
The code of $\varphi$ is now $\auf G_{\varphi}, N_{\eta} \times \{1\}\zu$.
Now to the case $\varphi = \auf \succ\zu \eta$. Again we
assume that $\eta$ is codable and that the code is
$C_{\eta} = \auf G_{\eta}, H_{\eta}\zu$. Now we define
$G_{\varphi}$. Let $N_{\varphi} := N_{\eta}\times \{0,1\}$,
$\Sigma_{\varphi} := \Sigma_{\eta} \times \{0\}$. Finally, let
$R_{\varphi}$ be the set of rules of the form
%%
\begin{equation}
\auf X,0\zu \pf a\; \auf Y,1\zu, \quad
\auf X,1\zu \pf a\; \auf Y,1\zu,
\end{equation}
%%
where $X \pf aY \in R_{\eta}$ and $X \in H_{\eta}$ and
of the form
%%
\begin{equation}
\auf X,0\zu \pf a\; \auf Y,0\zu, \quad
\auf X,1\zu \pf a\; \auf Y,0\zu,
\end{equation}
%%
where $X \pf aY \in R_{\eta}$ and $X \not\in H_{\eta}$; and
finally for every rule $X \pf a$ we take the rule
%%
\begin{equation}
\auf X,0\zu \pf a, \quad \auf X, 1\zu \pf a
\end{equation}
%%
on board. The code of $\varphi$ is now $\auf G_{\varphi}, N_{\eta}
\times \{1\}\zu$. Now we look at $\varphi = \auf +\zu \eta$.
Again we put $N_{\varphi} := N_{\eta} \times \{0,1\}$ as well
as $\Sigma_{\varphi} := \Sigma_{\eta} \times \{0,1\}$. We take
all rules of the form
%%
\begin{equation}
\auf X,0\zu \pf a\;\auf Y,0\zu
\end{equation}
%%
where $X \pf aY \in R_{\eta}$. Further, the rules have the form
%%
\begin{equation}
\auf X,1\zu \pf a\;\auf Y,1\zu
\end{equation}
%%
where $X \pf aY \in R_{\eta}$ and $Y \not\in H_{\eta}$.
Moreover, we take the rules
%%
\begin{equation}
\auf X,1\zu \pf a\;\auf Y,0\zu
\end{equation}
%%
for $X \pf aY \in R_{\eta}$ and $Y \in H_{\eta}$,
as well as all rules
%%
\begin{equation}
\auf X,0\zu \pf a
\end{equation}
%%
where $X \pf a \in R_{\eta}$. The code of $\varphi$ is then 
$\auf G_{\varphi}, N_{\eta} \times \{1\}\zu$. Now we look at 
$\varphi = \auf -\zu \eta$. Let $N_{\varphi} :=
N_{\eta} \times \{0,1\}$, and $\Sigma_ {\varphi} :=
\Sigma_{\eta} \times \{0\}$. The rules are of the form
%%
\begin{equation}
\auf X,1\zu \pf a\;\auf Y,1\zu
\end{equation}
%%
for $X \pf aY \in R_{\eta}$. Further there are rules of the form
%%
\begin{equation}
\auf X,0\zu \pf a\;\auf Y,1\zu
\end{equation}
%%
for $X \pf aY \in R_{\eta}$ and $X \in H_{\eta}$.
Moreover, we take the rules
%%
\begin{equation}
\auf X,0\zu \pf a\;\auf Y,0\zu
\end{equation}
%%
where $X \pf aY \in R_{\eta}$ and $X \not\in H_{\eta}$,
and, finally, all rules of the form
%%
\begin{equation}
\auf X,0\zu \pf a, 
\qquad
\auf X,1\zu \pf a, 
\end{equation}
%%
where $X \pf a \in R_{\eta}$. The code of $\varphi$ is then 
$\auf G_{\varphi}, N_{\eta} \times \{1\}\zu$.
The biggest effort goes into the last case, $\varphi = (\forall p_i)\eta$.
To start, we introduce a new alphabet, namely  $A \times \{0,1\}$,
and a new constant, $\mathsf{c}$. Let
$\uli{a} := \uli{\auf a,0\zu} \oder \uli{\auf a,1\zu}$.
Further, assume that $\mathsf{c} \dpf \goder_{a \in A} \uli{\auf a,1\zu}$ 
holds. Then $\uli{\auf a,1\zu} \dpf \uli{a} \und \mathsf{c}$
and $\uli{\auf a,0\zu} \dpf  \uli{a} \und \nicht \mathsf{c}$.
Then let $\eta' := \eta[\mathsf{c}/p_i]$. We can apply the inductive
hypothesis to this formula. Let $\Delta$ be the set of subformulae
of $\eta'$. For  an arbitrary subset $\Sigma  \subseteq \Delta$
let
%%
\begin{equation}
L_{\Sigma} := \gund_{\delta \in \Sigma} \delta \und
    \gund_{\delta \not\in \Sigma} \nicht\delta 
\end{equation}
%%
We can find a grammar$^{\ast}$ $G$ and for each $\delta \in \Delta$
sets $H_{\delta}$ such that $\auf G, H_{\delta}\zu$ codes $\delta$. 
Hence for every $\Sigma \subseteq \Delta$ there exist $H_{\Sigma}$ 
such that $\auf G, H_{\Sigma}\zu$ codes $L_{\Sigma}$. Now we first 
form the grammar$^{\ast}$ $G^1$ with the nonterminals 
$N \times \wp(\Delta)$ and the
alphabet $A \times \{0,1\}$. The set of rules is the set
of all
%%
\begin{equation}
\auf X,\Sigma\zu \pf \auf a,i\zu \quad\auf X',\Sigma'\zu
\end{equation}
%%
where $X \pf aX' \in R$, $X \in H_{\Sigma}$ and $X' \in
H_{\Sigma'}$; further all rules of the form
%%
\begin{equation}
\auf X, \Sigma\zu \pf \auf a,i\zu
\end{equation}
%%
where $X \pf a \in R$ and $X \in H_{\Sigma}$. Put
$H^1_{\Sigma} := H_{\Sigma} \times \{\Sigma\}$.
Again one easily sees that $\auf G^1, H^1_{\Sigma}\zu$
is a code for $\Sigma$ for every $\Sigma \subseteq
\Delta$. We now step over to the grammar$^{\ast}$ $G^2$ with
$N^2 := N \times \{0,1\} \times \wp(\Delta)$ and
$A^2 := A$ as well as all rules
%%
\begin{equation}
\auf X,i,\Sigma\zu \pf a \quad\auf X',i',\Sigma'\zu
\end{equation}
%%
where $\auf X,\Sigma\zu \pf \auf a,i\zu \quad\auf X',\Sigma'\zu
\in R^1$
%%
and
%%
\begin{equation}
\auf X, i, \Sigma\zu \pf a
\end{equation}
%%
where $\auf X, \Sigma\zu \pf \auf a,i\zu \in R^1$.  Finally, we
define the following grammar$^{\ast}$. $N^3 := 
N \times \wp(\wp(\Delta))$, $A^3 := A$, and let
%%
\begin{equation}
\auf X, \BA\zu \pf a \quad \auf Y, \BB\zu
\end{equation}
%%
be a rule iff $\BB$ is the set of all $\Sigma'$ for which
$\Sigma \in \BA$ and there are $i, i' \in \{0,1\}$ such that
%%
\begin{equation}
\auf X, i, \Sigma\zu \pf a \quad\auf Y, i', \Sigma'\zu \in R^2
\end{equation}
%%
Likewise
%%
\begin{equation}
\auf X, \BA\zu \pf a \in R^3
\end{equation}
%%
iff there is a $\Sigma \in \BA$ and some $i \in \{0,1\}$ with
%%
\begin{equation}
\auf X,i,\Sigma\zu \pf a \in R^2
\end{equation}
%%
Put $H_{\varphi} := \{\Sigma : \eta' \in \Sigma\}$. We claim:
$\auf G^3, H_{\varphi}\zu$ is a code for $\varphi$. For a proof
let $\vec{x}$ be a string and let a $G^3$--derivation of $\vec{x}$
be given. We construct a $G^1$--derivation. Let
$\vec{x} = \prod_{i < n} x_i$. By assumption we have a
derivation
%%
\begin{equation}
\auf X_i, \BA_i\zu \pf x_i \quad \auf X_{i+1}, \BA_{i+1}\zu
\end{equation}
%%
for $i < n-1$ and
%%
\begin{equation}
\auf X_{n-1}, \BA_{n-1}\zu \pf x_{n-1}
\end{equation}
%%
By construction there exists a $j_{n-1}$ and a $\Sigma_{n-1}
\in \BA_{n-1}$ such that
%%
\begin{equation}
\auf X_{n-1}, j_{n-1}, \Sigma_{n-1}\zu \pf a \in R^2
\end{equation}
%%
Descending we get for every $i < n-1$ a $j_i$ and a $\Sigma_i$
with
%%
\begin{equation}
\auf X_i, j_i, \Sigma_i\zu \pf a\quad
\auf X_{i+1}, j_{i+1}, \Sigma_{i+1}\zu \in R^2
\end{equation}
%%
We therefore have a $G^2$--derivation of $\vec{x}$. From this we
immediately get a $G^1$--derivation. It is over the alphabet
$A \times \{0,1\}$. By assumption $\eta'$ is coded in $G^1$
by $H_{\eta'}$. Then $\eta'$ holds in all nodes $i$ with
$X_i \in H_{\eta'}$. This is the set of all $i$ with
$X_i \in H_{\Sigma}$ for some $\Sigma \subseteq \Delta$ with
$\eta' \subseteq \Delta$. This is exactly the set of all
$i$ with $\auf X_i, \BA_i\zu \in H_{\varphi}$. Hence we have
$\auf X_i, \BA_i\zu \in H_{\varphi}$ iff the
Z--structure of $\vec{x}$ satisfies $\varphi$ in the given
$G^3$--derivation at $i$. This however had to be shown.
Likewise from a $G^1$--derivation of a string a $G^3$--derivation
can be constructed, as is easily seen.
\proofend

Now we are almost done. As our last task we have to show that
from the fact that a formula is codable we also get a grammar
which only generates strings that satisfy this formula. So let
$\Phi \subseteq \mathsf{MSO}$ be finite. We may assume that 
all members are sentences (if not, we quantify over the free
variables with a universal quantifier). By Corollary~\ref{cor:mql}
we can assume that in place of MSO--sentences we are
dealing with QML--formulae. Further, a finite conjunction
of QML--formulae is again a QML--formula so that we
are down to the case where $\Phi$ consists of a single
QML--formula $\varphi$. By Theorem~\ref{thm:code}, 
$\varphi$ has a code $\auf G,H\zu$, with $G = \auf \Sigma, N, A, R\zu$.
Put $G^{\varphi} := \auf \Sigma \cap H, N \cap H, A, R_H\zu$, where
%%
\begin{align}
R_H := & \phantom{\mbox{}\cup\mbox{}}
\{X \pf aY \in R : X, Y \in H\} \\\notag
  & \cup \{X \pf a \in R : X \in H\}
\end{align}
%%
Now there exists a $G^{\varphi}$--derivation of $\vec{x}$
iff $\vec{x}^{\times} \vDash  \varphi$.

{\it Notes on this section.} Some remarks are in order about FOL--definable
classes of Z--structures over the signature containing {\mtt <} (!) 
and $\uli{a}$, $a \in A$. A regular term is \textbf{$\ast$--free} if it 
does not contain $^{\ast}$, but may contain occurrences of 
$-$, which is a unary operator forming the complement of a language. 
Then the following are equivalent.
%%%
\begin{dingautolist}{192}
\item
The class of Z--structures for $L$ are finitely FO--axiomatizable.
\item
$L = L(t)$ for a $\ast$--free regular term $t$. 
\item
There is a $k \geq 1$ such that for every $\vec{y} \in A^+$ and 
$\vec{x}, \vec{z} \in A^{\ast}$: $\vec{x}\,{\vec{y}\,}^k\vec{z} \in L$ 
iff $\vec{x}\, {\vec{y}\,}^{k+1}\vec{z} \in L$.
\end{dingautolist}
%%%
See \cite{ebbinghausflum:finite} and references therein.
%%
\vplatz
\exercise
Show that every (!) language is an intersection of regular languages.
(This means that we cannot omit the condition of finite axiomatizability 
in Theorem~\ref{thm:buechi}.)
%%
\vplatz
\exercise
Let $\Phi$ be a finite MSO--theory, $L$ the regular language which
belongs to $\Phi$. $L$ is recognizable in $O(n)$--time
using a finite state automaton. Give upper bounds for the
number of states  of a minimal automaton recognizing $L$.
Use the proof of codability. Are the derived bounds optimal?
%%%
%\vplatz
%\exercise
%(Continuing the previous exercise.) Give an explicit constant
%$c_{\Phi}$ such that a single tape Turing machine  recognizes
%$L$ in $\leq c_{\Phi} \cdot n$ time. How does $c_{\Phi}$
%depend on $\Phi$?
%%%%
\vplatz
\exercise
An MSO--sentence is said to be in $\Sigma^1_1$ if it has the form
%%
\begin{equation}
(\exists P_0)(\exists P_1)
\dotsb (\exists P_{n-1})\varphi(P_0, \dotsc, P_{n-1})
\end{equation}
%%%
where $\varphi$ does not contain second order quantifiers.
$\varphi$ is said to be in $\Pi^1_1$ if it has the form
$(\forall P_0)(\forall P_1) \dotsb (\forall P_{n-1})\varphi(P_0, %
\dotsc, P_{n-1})$ where $\varphi$ does not contain second order
quantifiers. Show the following: {\it Every MSO--axiomatizable 
class $\CK$ of Z--structures is axiomatizable by a set of
$\Sigma^1_1$--sentences. If $\CK$ is finitely MSO--axiomatizable 
then it is axiomatizable by finitely many $\Sigma^1_1$--sen\-ten\-ces.}

 \section{Categorization and Pho\-no\-logy}
\label{kap5-3}
%
%
%
\nocite{birdellison:onelevel}
In this section we shall deal with syllable structure and phonological
rules. We shall look at the way in which discrete entities, known as 
{\it phonemes}, arise from a continuum of sounds or phones, and how the 
mapping between the sound continuum and the discrete space of language 
particular phonemes is to be construed. The issue is far from 
resolved; moreover, it seems that it depends on the way we look 
at language as a whole. Recall that we have assumed sign grammars
to be completely additive: there is no possibility to remove
something from an exponent that has been put there before. This 
has a number of repercussions. Linguists often try to define 
representations such that combinatory processes are additive. If 
this is taken to be a definition of linguistic processes (as in our 
definition of compositionality) the organisation of phonology and 
the phonetics--to--phonology mapping have to have a particular form. 
We shall discuss a few examples, notably 
%%%
\index{umlaut}%%%
\index{devoicing}%%%
%%%
umlaut and final devoicing. 

For example, the plural of the German noun {\tt Vater} is {\tt V\"ater}. 
How can this be realized in an additive way? First, notice that the 
plural is not formed from the singular; rather, both forms are derived 
from an underlying form, the root. Notice right away that the root 
cannot be a string, it must be a string where at most one vowel is marked 
for umlaut. (Not all roots will undergo umlaut and if so only 
one vowel is umlauted!) Technically, we can implement this by writing 
the root as a string vector: 
$\mbox{\tt V}\sotimes\mbox{\tt a}\sotimes\mbox{\tt ter}$. 
This allows us to restrict our attention to the representation 
of the vowel alone. 

Typically, in grammar books the root is assumed to be just like the 
singular: $\mbox{\tt V}\sotimes\mbox{\tt a}\sotimes \mbox{\tt ter}$. 
Early phonological theory on the other hand would have posited an 
abstract phoneme in place of {\tt a} or {\tt \"a}, a so--called 
%%%
\index{archiphoneme}
%%%
\textbf{archiphoneme}. Write {\tt A} for the archiphoneme that is 
underspecified between {\tt a} 
and {\tt \"a}. Then the root is $\mbox{\tt V}\sotimes \mbox{\tt A}
\sotimes \mbox{\tt ter}$, and the singular adds the specification, 
say an element {\tt x}, that makes {\tt A} be like {\tt a}, while the 
plural adds something, say {\tt y} that makes {\tt A} be like {\tt \"a}.
In other words, in place of {\tt a} and {\tt \"a} we have
{\tt Ax} and {\tt Ay}.
%%%
\begin{align}
\mbox{\rm sing} & \colon x\sotimes y \sotimes z \mapsto 
z\cdot y \cdot \mbox{\tt x} \cdot z \colon 
\mbox{\tt V}\sotimes \mbox{\tt A} \sotimes 
\mbox{\tt ter} \mapsto \mbox{\tt VAxter} \\
\mbox{\rm plur} & \colon x\sotimes y \sotimes z \mapsto 
z\cdot y \cdot \mbox{\tt y} \cdot z \colon 
\mbox{\tt V}\sotimes \mbox{\tt A} \sotimes 
\mbox{\tt ter} \mapsto \mbox{\tt VAyter}
\end{align}
%%%
This solution is additive. Notice, however, that {\tt A} cannot be 
pronounced, and so the root remains an abstract element. In certain 
representations, however, {\tt \"a} is derived from {\tt a}. Rather 
than treating the opposition between {\tt a} and {\tt \"a} as equipollent, 
we may treat it as privative: {\tt \"a} is {\tt a} plus something else. 
One specific proposal is  that {\tt \"a} differs from {\tt a} in 
having the symbol \textbf{i} in the i--tier (see 
\cite{ewenvanderhulst:phonology} and references therein). So, 
rather than writing 
the vowels using the Latin alphabet, we should write them as sequences
indicating the decomposition into primitive elements, and the process 
becomes literally additive. Notice that the alphabet that we use 
actually {\it is\/} additive. {\tt \"a} differs from {\tt a} by 
just two dots --- and this is the same with {\tt \"u} and 
{\tt \"o}. Historically, the dots derive from an `e' that was 
written above the vowel to indicate umlaut. (This does not always 
work; in Finnish {\tt \"u} is written {\tt y}, blocking for us this 
cheap way out for Finnish vowel harmony.) 
%%%
\index{devoicing}%%%
%%%
Final devoicing could be solved similarly by positing a decomposition 
of voiced consonants into voiceless consonant plus an abstract voice 
element (or rather: being voiceless is being voiced plus having a 
devoicing--feature). All these solutions, however, posit two levels of 
phonology: a surface phonology and a deep phonology. At the deep level, 
signs are again additive. This allows us to say that languages are 
compositional from the deep phonological level onwards.

The most influential model of phonology, by Chomsky and 
%%%
\index{Chomsky, Noam}%%
\index{Halle, Morris}%%
%%%
Halle~\shortcite{chomskyhalle:spe}, is however {\it not\/} 
additive. The model of phonology they favoured --- referred to simply 
%%%
\index{SPE--model}%%
%%%
as the \textbf{SPE--model} --- transforms deep structures into surface 
structures using context sensitive rewrite rules. We may illustrate 
these rules with German final 
%%%
\index{devoicing}%%%
%%%
devoicing. The rule says, roughly, that syllable final consonants 
(those following the vowel) are voiceless in German. However, as we 
have noted earlier (in Section~\ref{kap1}.\ref{kap1-3}), there is evidence to 
assume that some consonants are voiced and only become voiceless 
exactly when they end up in syllable final position. 
Hence, instead of viewing this as a constraint on the structure 
of the syllable we may see this as the effect of a rule that 
devoices consonants. Write {\tt +} for the syllable boundary. 
Sidestepping a few difficulties, we may write the rule of 
final devoicing as follows.
%%%
\begin{equation}
\mbox{\tt C}[+ \mbox{\it voiced\/}]
\Longrightarrow \mbox{\tt C}[- \mbox{\it voiced\/}] /
\underline{\quad}[-\mbox{\it voiced\/}]\; \mbox{\it or}
\; \underline{\quad}\mbox{\tt +} 
\end{equation}
%%
(Phonologists write +{\it voiced} what in attribute--value notation 
is $[\mbox{\sc voiced} : +]$.) This says that a consonant preceding 
a voiceless sound or a syllable boundary becomes voiceless. Using 
such rules, Chomsky 
%%%
\index{Chomsky, Noam}%%%
\index{Halle, Morris}%%%
%%%
and Halle have formulated a theory of the sound structure of English. 
This is a Type 1 grammar for English. It has 
been observed, however, by Ron Kaplan and Martin 
Kay~\shortcite{kaplankay:regular} 
%%%
\index{Kaplan, Ron}%%%
\index{Kay, Martin}%%%
%%%
and Kimmo Koskenniemi~\shortcite{koskenniemi:twolevel} that 
%%%
\index{Koskenniemi, Kimmo}%%%
%%%
for all that language really needs the relation between deep level 
and surface level is a regular relation and can be effected by a 
finite state transducer. Before we go into the details, we shall 
explain something about the general abstraction process in structural 
linguistics, exemplified here with phonemes, and on syllable structure.

Phonetics is the study of phones (= linguistic sounds) whereas 
phonology is the study of the phonemes of the languages. We may simply 
define a phoneme as a set of phones. Different languages group different
phones into different phonemes, so that the phonemes of languages
are typically not comparable. The grouping into phonemes is far
from trivial. A good exposition of the method can be found in
\cite{harris:structural}. We shall look at the process of
pho\-ne\-mi\-ci\-za\-tion in some detail. Let us assume for simplicity 
that words or texts are realized as sequences of discrete entities
called phones. This is not an innocent assumption: it is
for example often not clear whether the sequence [t] plus [\textesh],
resulting in an affricate [t\textesh], is to be seen as one or as 
two phones. (One can imagine that this varies from language to 
language.) Now, denote the set of phones by $\Sigma$. A word is not
a single sequence of phones, but rather a set of such sequences.
%%%
\begin{defn}
%%%
\index{language$^{\ast}$}%%
\index{word}%%
\index{realization}%%
%%%
$L$ is a \textbf{language}${}^{\ast}$ \textbf{over} $\Sigma$ if $L$ is a
subset of $\wp(\Sigma^{\ast})$ such that $\varnothing \not\in \Sigma$
and if $W, W' \in L$ and $W \cap W' \neq \varnothing$ then $W = W'$.
We call the members of $L$ \textbf{words}. $\vec{x} \in W$ is called a
\textbf{realization} of $W$. For two sequences $\vec{x}$ and $\vec{y}$
we write $\vec{x} \sim_L \vec{y}$ if they belong to (or realize) the
same word.
\end{defn}
%%%
One of the aims of phonology is to simplify the alphabet in such a
way that words are realized by as few as possible sequences. (That 
there is only one sequence for each word in the written system is an 
illusion created by orthographical convention. English orthography 
often has little connection with actual pronunciation.) We proceed 
by choosing a new alphabet, 
$P$, and a mapping $\pi \colon \Sigma \pf P$.  The map $\pi$ induces a 
partition on $\Sigma$. If $\pi(s) = \pi(s')$ we say that $s$ and $s'$ 
are \textbf{allophones}%%%
\index{allophone}%%%
%%%
. $\pi$ induces a mapping of $L$ onto a subset 
of $\wp(P^{\ast})$ in the following way. For a word $W$ we write
$\oli{\pi}[W] := \{\oli{\pi}(\vec{x}) : \vec{x} \in W\}$. Finally,
$\pi^{\ast}(L) := \{\oli{\pi}[W] : W \in L\}$. 
%%%
\begin{defn}
%%%
\index{map!discriminating}%%
%%%
Let $\pi \colon P \pf \Sigma$ be a map and $L \subseteq \wp(\Sigma^{\ast})$
be a language${}^{\ast}$. $\pi$ is called \textbf{discriminating for}
$L$ if whenever $W, W' \in L$ are distinct then $\oli{\pi}[W] \cap
\oli{\pi}[W'] = \varnothing$.
\end{defn}
%%
\begin{lem}
Let $L \subseteq \wp(\Sigma^{\ast})$ be a language$^{\ast}$ and 
$\pi \colon \Sigma \pf P$. If $\pi$ is discriminating for $L$, 
$\pi^{\ast}(L)$ is a language$^{\ast}$ over $P$.
\end{lem}
%%%
\begin{defn}
\label{defn:preph}
%%%
\index{phonemicization}%%
\index{phoneme}%%
%%%
A \textbf{phonemicization} of $L$ is a discriminating map 
$v \colon A \pf B$ such that for every discriminating 
$w \colon A \pf C$ we have $|C| \geq |B|$.
We call the members of $B$ \textbf{phonemes}.
\end{defn}
%%
If phonemes are sets of phones, they are clearly infinite sets.
To account for the fact that speakers can manipulate them, we must 
assume that they are finitely specified. Typically, phonemes are 
defined by means of articulatory gestures, which tell us (in an 
effective way) what basic motor program of the vocal organs is 
associated with what phoneme. For example, English [p] is 
voiceless. This says that the chords do not vibrate while 
it is being pronounced. It is further classified as an obstruent. 
This means that it obstructs the air flow. And thirdly 
it is classified as a bilabial: it is pronounced by putting the lips 
together. In English, there is exactly one voiceless bilabial obstruent, 
so these three features characterize English [p]. In Hindi, however, 
there are two phonemes with these features, an aspirated and an 
unaspirated one. (In fact, the actual pronunciation of English [p] 
for a Hindi speaker oscillates between two different sounds, see the 
discussion below.) As sounds have to be perceived and classified accordingly, 
each articulatory gesture is identifiable by an auditory feature that 
can be read off its spectrum. 

The analysis of this sort ends in the establishment of an alphabet $P$
of abstract sounds classes, defined by means of some features, which 
may either be called articulatory or auditory. (It is not universally 
agreed that features must be auditory or articulatory. We shall get to 
that point below.) These can be modeled in the logical language by means 
of {\it constants}. For example, the feature +{\it voiced\/} corresponds
to the constant \textsf{voiced}. Then $\nicht \textsf{voiced}$ is the 
same as being unvoiced.

The features are often interdependent. For example, vowels are always
%%%
\index{English}%%
\index{German}%%
%%%
voiced and continuants. In English and German voiceless plosives
are typically aspirated, while in French this is not the case; so 
[t] is pronounced with a subsequent [h]. (In older German books
one often finds {\tt Theil} (`part') in place of the modern
{\tt Teil}.) The aspiration is lacking when [t] is preceded 
within the syllable by a sibilant, which in standard German 
always is [\textesh], for example in {\tt stumpf}
[\textprimstress \textesh t\textupsilon mpf]. In German, vowels 
are not simply long or short. Also the vowel quality changes with 
the length. Long vowels are tense, short vowels are not. The 
letter {\tt i} is pronounced [\i] when it is short and [i:] 
when it is long (the colon indicates a long sound). (For example, 
{\tt Sinn} (`sense') [\textprimstress z\i n]  as opposed to {\tt Tief} 
(`deep') [\textprimstress t\textsuperscript{h}i:f].) Likewise for 
the other vowels.  Table~\ref{tab:langkurz} shows the pronunciation 
of the long and short vowels, drawn from \cite{ipahandbook}, Page 87 
(written by Klaus Kohler).
%%
\begin{table}
\caption{Long and short vowels of German}
\index{German}%%%
\label{tab:langkurz}
\begin{center}
\begin{tabular}{|l|l||l|l|}
\hline
long & short & long & short \\\hline\hline
i: & \i & y: & \textscy \\
a: & a & e: & \textschwa \\
o:  & \textopeno   & \o: & \oe \\
u:  & \textupsilon  & \textepsilon : & \textepsilon 
\\\hline
\end{tabular}
\end{center}
\end{table}
%%
Only the pairs [a:]/[a] and [\textepsilon :]/[\textepsilon]
are pronounced in the same way, differing only in length. It is 
therefore not easy to say which feature is distinctive: 
is length distinctive in German for vowels, or is it rather the tension?
This is interesting in particular when speakers learn a new language, 
because they might be forced to keep distinct two parameters that are 
cogradient in their own. For example, in Finnish vowels are almost purely 
distinct in length, there is no cooccurring distinction in tension. 
If so, tension cannot be used to differentiate a long vowel from a 
short one. This is a potential source of difficulty for Germans if 
%%%
\index{Finnish}%%
%%%
they want to learn Finnish.

If $L$ is a language$^{\ast}$ in which every word has exactly
one member, $L$ is uniquely defined by the language
$L^{\diamond} := \{\vec{x} : \{\vec{x}\} \in L\}$. Let us assume
after suitable reductions that we have such a language$^{\ast}$;
then we may return to studying languages in the customary sense.
It might be thought that languages do not possess nontrivial
phonemicization maps. This is, however, not so. For example, English
has two different sounds, [p] and [p\textsuperscript{h}]. The first 
occurs after [s], while the second appears for example word initially 
before a vowel. It turns out that in English [p] and 
[p\textsuperscript{h}] are not two but one phoneme. To see why, 
we offer first a combinatorial and then a logical analysis. Recall 
the definition of a context set. For regular languages it is 
simply
%%%
\index{context set}%%
\index{$\Cont_L(a)$}%%%
%%%
\begin{equation}
\Cont_L(a) := \{\auf \vec{x}, \vec{y} \zu :
        \vec{x}\conc a\conc\vec{y} \in L\}
\end{equation}
%%%
\index{complementary distribution}%%
%%%
If $\Cont_L(a) \cap \Cont_L(a') = \varnothing$,  $a$ and $a'$ are said 
to be in \textbf{complementary distribution}. An example is the
abovementioned [p] and [p\textsuperscript{h}]. Another example is [x] 
versus [$\chi$] in German. Both are written {\tt ch}. However, {\tt ch}
is pronounced [x] if occurring after [a], [o] and [u], while it is
pronounced [\c{c}] if occurring after other vowels and [r], [n]
or [l]. Examples are {\tt Licht} [\textprimstress l\i\c{c}t], 
{\tt Nacht} [\textprimstress naxt], {\tt echt} [\textprimstress e\c{c}t] 
and {\tt Furcht} [\textprimstress fu\textinvscr\c{c}t]. (If you do not 
know German, here is a short description of the sounds. [x] is
pronounced at the same place as [k] in English, but it is a
fricative. [\c{c}] is pronounced at the same place as {\tt y} in
English {\tt yacht}, however the tongue is a little higher, that
is, closer to the palatum and also the air pressure is somewhat
higher, making it sound harder.) Now, from
Definition~\ref{defn:preph} we extract the following.
%%%
\begin{defn}
Let $A$ be an alphabet and $L$ a language over $A$. 
$\pi \colon A \pf B$
is a \textbf{pre--phonemicization} if $\oli{\pi}$ is injective on
$L$. $\pi \colon A \epi B$ is a \textbf{phonemicization} if for all
pre--phonemicizations $\pi' \colon A \epi C$, $|C| \geq |B|$.
\end{defn}
%%
The map sending [x] and [\c{c}] to the same sound is a
pre--pho\-ne\-mi\-ci\-za\-tion in German. However, notice the following.
In the language $L_0 := \{\mbox{\tt aa}, \mbox{\tt bb}\}$, {\tt a}
and {\tt b} are in complementary distribution. Nevertheless, the map
sending both to the same element is not injective. So, complementary
distribution is not enough to make two sounds belong to the same
phoneme. We shall see below what is. Second, let $L_1 :=
\{\mbox{\tt ac}, \mbox{\tt bd}\}$. We may either send
{\tt a} and {\tt b} to {\tt e} and obtain the language
$M_0 := \{\mbox{\tt ec}, \mbox{\tt ed}\}$, or we may send
{\tt c} and {\tt d} to {\tt f} and obtain the language
$M_1 := \{\mbox{\tt af}, \mbox{\tt bf}\}$. Both maps are
phonemicizations, as is easily checked. So, phonemicizations are
not necessarily unique. In order to analyse the situation we have to
present a few definitions. The general idea is this. Suppose
that $A$ is not minimal for $L$ in the sense that it possesses
a noninjective phonemicization.  Then there is a pre--phonemicization
that conflates exactly two symbols into one. The image $M$ of this
map is a regular language again. Now, given the latter we can
actually recover for each member of $M$ its preimage under this
conflation. What we shall show now
is that moreover if $L$ is regular {\it there is an explicit procedure
telling us what the preimage is}. This will be cast in rather abstract
terms. We shall define here a modal language that is somewhat different
from $\mathsf{QML}$, namely $\mathsf{PDL}$ with converse. 

\index{propositional dynamic logic}%%
\index{PDL (see propositional dynamic logic)}%%%
The syntax of \textbf{propositional dynamic logic} (henceforth PDL) 
has the usual boolean connectives, the \textbf{program connectives} 
;, $\cup$, $^{\ast}$, further $?$ and the `brackets' $[-]$ and 
$\auf -\zu$. Further, there is a set $\Pi_0$ of \textbf{elementary 
programs}. 
%%%
\index{program!elementary}%%%
%%%%
\begin{dingautolist}{192}
\item Every propositional variable is a proposition. 
\item if $\varphi$ and $\chi$ is a proposition, so are 
	$\nicht\varphi$, $\varphi\und\chi$, $\varphi\oder \chi$, 
	and $\varphi\pf\chi$.
\item If $\varphi$ is a proposition, $\varphi?$ is a program. 
\item Every elementary program is a program.
\item If $\alpha$ and $\beta$ are programs, so are 
	$\alpha;\beta$, $\alpha\cup\beta$, and $\alpha^{\ast}$.
\item If $\alpha$ is a program and $\varphi$ a proposition, 
	$[\alpha]\varphi$ and $\auf\alpha\zu\varphi$ 
	are propositions.
\end{dingautolist}
%%%
A \textbf{Kripke--model} is a triple $\auf F, R, \beta\zu$, where 
$R : \Pi_0 \pf \wp(F^2)$, and $\beta : \mbox{\rm PV} \pf \wp(F)$. 
We extend the maps $R$ and $\beta$ as follows. 
%%%
\begin{equation}
\begin{split}
\oli{\beta}(\nicht\varphi) & := F - \oli{\beta}(\varphi) \\
\oli{\beta}(\varphi\und\chi) & := \oli{\beta}(\varphi) 
	\cap \oli{\beta}(\chi) \\
\oli{\beta}(\varphi\oder\chi) & := \oli{\beta}(\varphi) 
	\cup \oli{\beta}(\chi) \\
\oli{\beta}(\varphi\pf\chi) & := (- \oli{\beta}(\varphi)) 
	\cup \oli{\beta}(\chi) \\
\oli{R}(\varphi?) & := \{\auf x,x\zu : x \in \oli{\beta}(\varphi)\} \\
\oli{R}(\alpha\cup\beta) & := \oli{R}(\alpha) \cup \oli{R}(\beta) \\
\oli{R}(\alpha;\beta) & := \oli{R}(\alpha) \circ \oli{R}(\beta) \\
\oli{R}(\alpha^{\ast}) & := \oli{R}(\alpha)^{\ast} \\
\oli{\beta}([\alpha]\varphi) & := 
	\{x : \text{for all $y:$ if } x\; \oli{R}(\alpha)\; y 
	\text{ then } y \in \oli{\beta}(\varphi)\} \\
\oli{\beta}(\auf\alpha\zu\varphi) & := 
	\{x : \text{there is }y: x\; \oli{R}(\alpha)\; y 
	\text{ and } y \in \oli{\beta}(\varphi)\} 
\end{split}
\end{equation}
%%%
We write $\auf F, R, \beta\zu \vDash \varphi$ if $x \in 
\oli{\beta}(\varphi)$. \textbf{Elementary PDL} (\textbf{EPDL}) 
%%%
\index{propositional dynamic logic!elementary}%%%
%%%
is the fragment 
of PDL that has no star. The elements of $\Pi_0$ are constants; 
they are like the modalities of modal logic. Obviously, it is 
possible to add also propositional constants.

In addition to $\mathsf{PDL}$, 
%%%
\index{$\mathsf{PDL}$}%%%
it also has a program constructor 
$^{\smallsmile}$.
$\alpha^{\smallsmile}$ denotes the converse of $\alpha$.
Hence, in a Kripke--frame $\oli{R}(\alpha^{\smallsmile}) =
\oli{R}(\alpha)^{\smallsmile}$. The axiomatization consists
in the axioms for $\mathsf{PDL}$ together with the axioms
$p \pf [\alpha]\auf \alpha^{\smallsmile}\zu p$,
$p \pf [\alpha^{\smallsmile}]\auf \alpha\zu p$ for every
program $\alpha$. The term {\it dynamic logic\/} will henceforth
refer to an extension of $\mathsf{PDL}^{\smallsmile}$ by
some axioms. The fragment without $^{\ast}$ is called
\textbf{elementary} $\textbf{PDL}$ with converse, and is 
denoted by $\mathsf{EPDL}^{\smallsmile}$. 
%%%
\index{$\mathsf{PDL}^{\smallsmile}$, $\mathsf{EPDL}^{\smallsmile}$}%%
%%%
An analog of B\"uchi's Theorem
holds for the logic $\mathsf{PDL}^{\smallsmile}(\prec)$.
%%
\begin{thm}
Let $A$ be a finite alphabet. A class of MZ--struc\-tu\-res over $A$
is regular iff it is axiomatizable over the logic of
all MZ--structures by means of constant formulae in
$\mathsf{PDL}^{\smallsmile}(\prec)$ (with constants for
letters from $A$).
\end{thm}
%%
\proofbeg
By Kleene's Theorem, a regular language is the extension of a regular
term. The language of such a term can be written down in 
$\mathsf{PDL}^{\smallsmile}$ using a constant formula. Conversely, 
if $\gamma$ is a constant
$\mathsf{PDL}^{\smallsmile}(\prec)$--formula it can be rewritten
into an $\mathsf{MSO}$--formula.
\proofend

The last point perhaps needs reflection. There is a straightforward
translation of $\mathsf{PDL}^{\smallsmile}$ into $\mathsf{MSO}$. We only
have to observe that the transitive closure of an $\mathsf{MSO}$--definable
relation is again $\mathsf{MSO}$--definable (see 
Exercise~\ref{ex:transclose}).
%%
\begin{align}
x\; R^{\ast}\; y\quad \Dpf\quad & 
(\forall X)(X(x) \und (\forall z)(\forall z')(X(z) \und
    z\; R\; z'. \\\notag
	& \qquad\qquad \pf .X(z')) \pf X(y))
\end{align}
%%
Notice also that we can eliminate $^{\smallsmile}$ from complex
programs using the following identities.
%%
\begin{subequations}
\begin{align}
\oli{R}((\alpha\cup\beta)^{\smallsmile}) & = 
        \oli{R}(\alpha^{\smallsmile} \cup \beta^{\smallsmile}) \\
\oli{R}((\alpha;\beta)^{\smallsmile}) & = 
        \oli{R}(\beta^{\smallsmile};\alpha^{\smallsmile}) \\
\oli{R}((\alpha^{\ast})^{\smallsmile}) & = 
        \oli{R}((\alpha^{\smallsmile})^{\ast}) \\
\oli{R}((\varphi?)^{\smallsmile}) & = \oli{R}(\varphi?)
\end{align}
\end{subequations}
%%
Hence, $\mathsf{PDL}^{\smallsmile}(\prec)$ can also be seen as
an axiomatic extension of $\mathsf{PDL}(\prec;\succ)$ by the
axioms $p \pf [\prec]\auf \succ\zu p$, $p \pf [\succ]\auf \prec\zu p$.
Now let $\Theta$ be a dynamic logic.  Recall from 
Section~\ref{kap6}.\ref{kap6-3} the definition of $\Vdash_{\Theta}$, 
the global consequence associated with $\Theta$.

Now, we shall assume that we have a language
$\mathsf{PDL}^{\smallsmile}(\prec;D)$, where $D$ is a set of
constants. For simplicity, we shall assume that for each letter
$a \in A$, $D$ contains a constant $\uli{a}$. However, there may
be additional constants. It is those constants that we shall 
investigate here. We shall show (i) that these constants can
be eliminated in an explicit way, (ii) that one can always add
constants such that $A$ can be be described purely by contact
rules.
%%
\begin{defn}
%%%
\index{definition!global implicit}%%
%%%
Let $\Theta$ be a dynamic logic and $\varphi(q)$ a formula. 
$\varphi(q)$ \textbf{globally implicitly defines} $q$ \textbf{in} 
$\Theta$ if $\varphi(q); \varphi(q') \Vdash_{\Theta} q \dpf q'$.
\end{defn}
%%
Features (or constants, for that matter) that are implicitly defined
are called \textbf{inessential}. Here the leading idea is that an 
inessential feature does not constitute a distinctive phonemic feature, 
because removing the distinction that this feature induces on the alphabet
turns out to induce an injective map. Formally, this is spelled out
as follows. Let $A \times \{0,1\}$ be an alphabet, and assume that
the second component indicates the value of the feature $\mathsf{c}$. Let
$\pi \colon A \times \{0,1\} \pf A$ be the projection onto the first
factor. Suppose that the language $L$ can be axiomatized by the
constant formula $\varphi(\mathsf{c})$. $\varphi(\mathsf{c})$
defines $\mathsf{c}$ implicitly if $\oli{\pi} \colon  L' \pf L$ is injective.
This in turn means that the map $\pi$ is a pre--phonemicization. For
in principle we could do without the feature.  Yet, it is not clear
that we can simply eliminate it.  In $\mathsf{PDL}^{\smallsmile}
\oplus \varphi(\mathsf{c})$ we call $\mathsf{c}$ \textbf{eliminable} 
%%%
\index{constant!eliminable}%%
%%%
if there is a formula $\chi$ provably equivalent to $\varphi(\mathsf{c})$
that uses only the constants of $\varphi$ without $\mathsf{c}$. In the
present case, however, an inessential feature is also eliminable.
Notice first of all that a regular language over an alphabet $B$ is
definable by means a constant formula over the logic of all strings,
with constants $\uli{b}$ for every element $b$ of $B$.
By Lemma~\ref{lem:constantbeth}, it is therefore enough
to show the claim for the logic of all strings. Moreover, by a suitable
replacement of other variables by new constants we may reduce the
problem to the case where $p$ is the only variable occurring in the
formula. Now the language $L$ is regular over the alphabet
$A \times \{0,1\}$. Therefore, $\oli{\pi}[L]$ is regular as well.
This means that it can be axiomatized using a formula without the
constant $\mathsf{c}$. However, this only means that we can make the
representation of words more compact. Ideally, we also wish to describe
for given $a \in A$, in which context we find $\auf a, 0\zu$ (an
$a$ lacking $\mathsf{c}$) and in which context we find $\auf a, 1\zu$
(an $a$ having $\mathsf{c}$).  This can be done. Let $\GA =
\auf A, Q, q_0, F, \delta\zu$ be a finite state automaton.
Then $L_{\GA}(q) := \{\vec{x} : q_0 \stackrel{\vec{x}}{\pf} q\}$
is a regular language (for $L_{\GA}(q) = L(\auf A, Q, q_0,
\{q\}, \delta\zu)$, and the latter is a finite state automaton).
Furthermore, $A^{\ast} = \bigcup_{q \in Q} L_{\GA}(q)$.
If $\GA$ is deterministic, then $L_{\GA}(q) \cap L_{\GA}(q') 
= \varnothing$ whenever $q \neq q'$. Now, let $\GB$ be a 
deterministic finite state automaton over $A \times \{0,1\}$ such 
that $\vec{x} \in L(\GB)$ iff $\vec{x} \vDash \varphi(\mathsf{c})$.
Suppose we have a constraint $\chi$, where $\chi$ is a constant
formula.
%%%
\begin{defn}
The \textbf{Fisher--Ladner closure} 
%%%
\index{Fisher--Ladner closure}%%%
\index{$\FL(\chi)$}%%%
%%%
of $\chi$, $\FL(\chi)$, is defined as follows.
%%%
\begin{subequations}
\begin{align}
\FL(p_i) & := \{p_i\} \\
\FL(\gamma) & := \{\gamma\} \\
\FL(\chi\und\chi') & := \{\chi\und\chi'\}
        \cup \FL(\chi) \cup \FL(\chi') \\
\FL(\auf \alpha\cup \beta\zu\chi) & :=
        \{\auf\alpha\cup\beta\zu\chi\}
        \cup \FL(\auf\alpha\zu\chi)
        \cup \FL(\auf\beta\zu\chi) \\
\FL(\auf \alpha;\beta\zu\chi) & :=
        \{\auf\alpha;\beta\zu\chi\}
        \cup \FL(\auf\alpha\zu\auf\beta\zu\chi) \\
\FL(\auf \alpha^{\ast}\zu\chi) & :=
        \{\auf \alpha^{\ast}\zu\chi\} \cup
        \FL(\auf \alpha\zu\auf\alpha^{\ast}\zu\chi)
	\cup \FL(\chi) \\
\FL(\auf \varphi?\zu\chi) & :=
        \{\auf\varphi?\zu\chi\} \cup
        \FL(\varphi) \cup \FL(\chi) \\
\FL(\auf \alpha\zu\chi) & :=
        \{\auf\alpha\zu\chi\} \cup
        \FL(\chi) \qquad \alpha \mbox{ basic}
\end{align}
\end{subequations}
\end{defn}
%%%
The Fisher--Ladner closure covers only 
$\mathsf{PDL}(\prec;\succ)$--formulae, but this is actually enough 
for our purposes. For each formula $\sigma$ in the Fisher--Ladner 
closure of $\chi$ we introduce a constant $c(\sigma)$.
In addition, we add the following axioms.
%%
\begin{equation}
\begin{split}
c(\nicht\sigma) & \dpf \nicht c(\sigma) \\
c(\sigma\und\tau) & \dpf c(\sigma) \und c(\tau) \\
c(\auf\varphi?\zu\sigma) & \dpf c(\varphi) \und c(\sigma) \\
c(\auf\alpha\cup\beta\zu\sigma) & \dpf c(\auf\alpha\zu\sigma)
        \oder c(\auf\beta\zu\sigma) \\
c(\auf\alpha;\beta\zu\sigma) & \dpf c(\auf\alpha\zu\auf\beta\zu\sigma)
        \\
c(\auf\alpha^{\ast}\zu\sigma) & \dpf c(\sigma) \oder 
        c(\auf\alpha\zu\auf\alpha^{\ast}\zu\sigma) \\
c(\auf\prec\zu\sigma) & \dpf \auf\prec\zu c(\sigma) \\
c(\auf\succ\zu\sigma) & \dpf \auf\succ\zu c(\sigma)
\end{split}
\end{equation}
%%
\index{cooccurrence restrictions}%%
%%%
We call these formulae \textbf{cooccurrence restrictions}.
After the introduction of these formulae as axioms 
$\sigma \dpf c(\sigma)$ is provable for every $\sigma \in
\FL(\chi)$. In particular, $\chi \dpf c(\chi)$
is provable. This means that we can eliminate $\chi$ in favour
of $c(\chi)$. The formulae that we have just added do not contain
any of $?$, $\cup$, $^{\smallsmile}$, $^{\ast}$ or $;$. We only
have the most simple axioms, stating that some constant is true
before or after another. Now we construct the following automaton.
Let $\vartheta$ be a subset of $\FL(\chi)$. Then put
%%
\begin{equation}
q_{\vartheta} := \gund_{\gamma \in \vartheta} c(\gamma) \und
    \gund_{\gamma \not\in \vartheta} \nicht c(\gamma)
\end{equation}
%%
Now let $Q$ be the set of all consistent $q_{\vartheta}$.
Furthermore, put $q_{\vartheta} \stackrel{a}{\pf} q_{\eta}$
iff $q_{\vartheta} \und \auf \prec; \uli{a}?\zu q_{\eta}$
is consistent. Let $F := \{q_{\vartheta} : [\prec]\bot \in \vartheta\}$
and $B := \{q_{\vartheta} : [\succ]\bot \in \vartheta\}$. For every
$b \in B$, $\auf A, Q, b, F, \delta\zu$ is a finite state
automaton. Then
%%
\begin{equation}
L := \bigcup_{b \in B} L(\auf A, Q, b, F, \delta\zu)
\end{equation}
%%
is a regular language. It immediately follows that the automaton
above is well--defined and for every subformula $\alpha$ of $\chi$
the set of positions $i$ such that $\auf \vec{x}, i\zu \vDash \alpha$
is uniquely fixed. Hence, for every $\vec{x}$ there exists exactly
one accepting run of the automaton. $\auf \vec{x}, i\zu \vDash \psi$
iff $\psi$ holds at the $i$th position of the accepting run.

We shall apply this to our problem. Let $\varphi(\mathsf{c})$ be an
implicit definition of $\mathsf{c}$. Construct the automaton
$\GA(\varphi(\mathsf{c}))$ for $\varphi(\mathsf{c})$
as just shown, and lump together all states that do not contain
$c(\varphi(\mathsf{c}))$ into a single state $q'$ and put
$q' \stackrel{a}{\pf} q'$ for every $a$. All states different
from $q'$ are accepting. This defines the automaton $\GB$. Now
let $C := \{q_{\vartheta} : \mathsf{c} \in \vartheta\}$. The
language $\bigcup_{c \in C} L_{\GB}(q)$ is regular, and it
possesses a description in terms of the constants $\uli{a}$,
$a \in A$, alone.
%%
\begin{defn}
%%%
\index{definition!global explicit}%%
%%%
Let $\Theta$ be a logic and $\varphi(q)$ a formula. Further, let
$\delta$ be a formula not containing $q$. We say that $\delta$
\textbf{globally explicitly defines} $q$ \textbf{in} $\Theta$
\textbf{with respect to} $\varphi$ if $\varphi(q) \Vdash_{\Theta}
\delta \dpf q$.
\end{defn}
%%
Obviously, if $\delta$ globally explicitly defines $q$ with respect 
to $\varphi(q)$
then $\varphi(q)$ globally implicitly defines $q$. On the other hand,
if $\varphi(q)$ globally implicitly defines $q$ then it is not necessarily
the case that there is an explicit definition for it. It very much
depends on the logic in addition to the formula whether there is.
%%%
\index{Beth property!global}%%
%%%
A logic is said to have the \textbf{global Beth--property} if for
any global implicit definition there is a global explicit definition.
Now suppose that we have a formula $\varphi$ implicitly
defining $q$. Suppose further that $\delta$ is an explicit definition.
Then the following is valid.
%%
\begin{equation}
\Vdash_{\Theta} \varphi(q) \dpf \varphi(\delta)
\end{equation}
%%
The logic $\Theta \oplus \varphi$ defined by adding the formula $\varphi$
as an axiom to $\Theta$ can therefore equally well be axiomatized by
$\Theta \oplus \varphi(\delta)$. The following is relatively easy
to show.
%%%
\begin{lem}
\label{lem:constantbeth}
Let $\Theta$ be a modal logic, and $\gamma$ a constant formula.
Suppose that $\Theta$ has the global Beth--property. Then
$\Theta \oplus \gamma$ also has the global Beth--property.
\end{lem}
%%%
\begin{thm}
%%%
\label{thm:regbeth}
%%%
Every logic of a regular string language has the global
Beth--property.
\end{thm}
%%%
If the axiomatization is infinite, by the described procedure we
get an infinite array of formulae. This does not have a regular
solution in general, as the reader is asked to show in the exercises.

The procedure of phonemicization is inverse to the procedure of
adding features that we have looked at in the previous section.
We shall briefly look at this procedure from a phonological
point of view. Assume that we have an alphabet $A$ of phonemes,
containing also the syllable boundary marker {\tt +} and the
word boundary marker {\tt \#}. These are not brackets, they are separators.
Since a word boundary is also a syllable boundary, no extra marking
of the syllable is done at the word boundary. Let us now ask what
are the rules of syllable and word structure in a language. The
minimal assumption is that any combination of phonemes may form
a syllable. This turns out to be false. Syllables are in fact
constrained by a number of (partly language dependent) principles.
This can partly be explained by the fact that vocal tract has a 
certain physiognomy that discourages certain phoneme combinations 
while it enhances others. These properties also lead to a deformation 
of sounds in contact,
%%%
\index{sandhi}%%
%%%
which is called \textbf{sandhi}, a term borrowed from Sanskrit grammar.
A particular example of sandhi is assimilation ([np] $>$ [mp]). Sandhi
rules exist in nearly all languages, but the scope and character varies
greatly. Here, we shall call {\it sandhi\/} any constraint that
is posed on the occurrence of two phonemes (or sounds) next to
each other. Sandhi rules are 2--templates in the sense of the 
following definition.
%%%
\begin{defn}
%%%%
\index{template}%%%
\index{template language}%%
%%%%
Let $A$ be an alphabet. An $n$--\textbf{template over} $A$ (or
\textbf{template of length} $n$) is a cartesian product of length
$n$ of subsets of $A$. A language $L$ is an $n$--\textbf{template
language} if there is a finite set $\CP$ of length $n$ such that
$L$ is the set of words $\vec{x}$ such that every subword of length
$n$ belongs to at least one template from $\CP$. $L$ is a
\textbf{template language} if there is an $n$ such that $L$ is
an $n$--template language.
\end{defn}
%%%
Obviously, an $n$--template language is an $n+1$--template language.
Furthermore, 1--template languages have the form $B^{\ast}$ where
$B \subseteq A$. So the first really interesting class is that of
the 2--template languages.  It is clear that if the alphabet is finite,
we may actually define an $n$--template to be just a member of $A^n$.
Hence, a template language is defined by naming all those sequences
of bounded length that are allowed to occur.
%%%
\begin{prop}
A language is a template language iff its class of
$A$--strings is axiomatizable by finitely many positive 
EPDL--formulae.
\end{prop}
%%
To make this more realistic we shall allow also boundary templates.
Namely, we shall allow a set $\CP^-$ of left edge templates and a set
$\CP^+$ of right edge templates. $\CP^-$ lists the admissible
$n$--prefixes of a word and $\CP^+$ the admissible $n$--suffixes.
Call such languages \textbf{boundary template languages}. 
%%%
\index{template language!boundary}%%%
%%
Notice that phonological processes are conditioned by certain boundaries, 
but we have added the boundary markers to the alphabet. This
effectively eliminates the need for boundary templates in the
description here. We have not explored the question what would happen
if they were eliminated from the alphabet.
%%%
\begin{prop}
A language is a boundary template language iff its
class of $A$--strings is axiomatizable by finitely many 
EPDL--formulae.
\end{prop}
%%%
It follows from Theorem~\ref{thm:buechi} that template languages are
regular (which is easy to prove anyhow). However, the language
$\mbox{\tt c}\mbox{\tt a} ^+\mbox{\tt c} \cup \mbox{\tt d}%
\mbox{\tt a}^+\mbox{\tt d}$ is regular but not a template
language.

The set of templates effectively names the legal
transitions of an automaton that uses the alphabet $A$ itself as
the set of states to recognize the language. We shall define this
notion, using a slightly different concept here, namely that of a
%%%
\index{finite state automaton!partial}%%
%%%%
\textbf{partial finite state automaton}. This is a quintuple $\GA =
\auf A, Q, I, F, \delta\zu$, such that $A$ is the \textbf{input
alphabet}, $Q$ the set of internal states, $I$ the set of initial
states, $F$ the set of accepting states and $\delta : A \times Q
\stackrel{p}{\pf} Q$ a partial function. $\GA$ \textbf{accepts} $\vec{x}$
if there is a computation from some $q \in I$ to some $q' \in F$
%%%%
\index{language!2--template}%%%
%%%
with $\vec{x}$ as input. $\GA$ is a \textbf{2--template language} 
if $Q = A$ and $\delta(a,b)$ is either undefined or $\delta(a,b) = b$.

The reason for concentrating on 2--template languages is the philosophy
of naturalness. Basically, grammars are natural if the nonterminal
symbols can be identified with terminal symbols, that is, for 
every nonterminal $X$ there is a terminal $a$ such that for every 
$X$--string $\vec{x}$ we have $\Cont_L(\vec{x}) = \Cont_L(a)$. For 
a regular grammar this means in essence that a
string beginning with $a$ has the same distribution
as the letter $a$ itself. A moment's reflection reveals that this is
the same as the property of being 2--template. Notice that the 2--template
property of words and syllables was motivated from the nature
of the articulatory organs, and we have described a parser that
recognizes whether something is a syllable or a word. Although it
seems prima facie plausible that there are also auditory
constraints on phoneme sequences we know of no plausible constraint
that could illustrate it. We shall therefore concentrate on the
former. What we shall now show is that syllables are not 2--template.
This will motivate either adding structure or adding more features
to the description of syllables. These features are necessarily
nonphonemic.

We shall show that nonphonemic features exist by looking at syllable
structure. It is not possible to outline a general theory of syllable
structure.  However, the following sketch may be given (see
\cite{grewendorf:wissen}). The sounds are aligned into a 
so--called \textbf{sonoricity hierarchy}, which is shown in 
Table~\ref{tab:son} (vd.\ = voiced, vl.\ = voiceless).
%%
\begin{table}
\caption{The Sonoricity Hierarchy}
%%%
\index{sonoricity hierarchy}
%%%
\label{tab:son}
\begin{center}
\begin{tabular}{l@{\;}ll@{\;}ll@{\;}l}
 & dark vowels & $>$ & mid vowels & $>$ & high vowels \\ 
 & [a], [o] & & [\ae], [\oe] & & [i], [y] \\ 
\\
$>$ & r--sounds & $>$ & nasals; laterals & $>$ & vd.~fricatives \\
    & [r]       &     & [m], [n]; [l]    &     & [z], [\textyogh] \\
\\
$>$ & vd.~plosives & $>$ & vl.~fricatives & $>$ & vl.~plosives \\
    & [b], [d]     &     & [s], [\textesh] &    & [p], [t] 
\end{tabular}
\end{center}
\end{table}
%%
The syllable is organized as follows.
%%
\begin{quote}
{\sl Syllable Structure.} Within a syllable the sonoricity increases
monotonically and then decreases.
\end{quote}
%%%
This means that a syllable must contain at least one sound
which is at least as sonorous as all the others in the syllable.
It is called the \textbf{sonoricity peak}. We shall make the following
assumption that will simplify the discussion.
%%%
\begin{quote}
{\sl Sonoricity Peak.}
The sonoricity peak can be constituted by vowels only.
\end{quote}
%%%
This wrongly excludes the syllable [krk], or [dn]. The latter is
heard in the German {\tt verschwinden} (`to disappear') 
[\textsecstress f\textepsilon \textinvscr \textprimstress\textesh 
w\i ndn]. (The second {\tt e} that appears in writing is hardly 
ever pronounced.) However, even if the assumption is relaxed, the 
problem that we shall address will remain.

The question is: how can we implement these constraints?  There are 
basically two ways of doing that. (a) We state
them by means of $\mathsf{PDL}^{\smallsmile}$--formulae.
This is the descriptive approach. (b) We code them.
This means that we add some features in such a way that the
resulting restrictions become specifiable by 2--templates.
The second approach has some motivation as well. The added
features can be identified as states of a productive (or analytic)
device. Thus, while the solution under (a) tells us what the
constraint actually is, the approach under (b) gives us features
which we can identify as (sets of) states of a (finite state)
machine that actually parses or produces these structures.
That this can be done is expressed in the following corollary of 
the Coding Theorem.
%%
\begin{thm}
Any regular language is the homomorphic image of a boundary
2--template language.
\end{thm}
%%
So, we only need to add features. Phonological string languages
are regular, so this method can be applied. Let us see how we can
find a 2--template solution for the sonoricity hierarchy. We introduce a
feature $\alpha$ and its negation $- \alpha$. We start with the
alphabet $P$, and let $C \subseteq P$ be the set of consonants.
The new alphabet is
%%
\begin{equation}
\Xi := P \times \{-\alpha\} \cup C \times \{\alpha\}
\end{equation}
%%
Let $\son(a)$ be the sonoricity of $a$. (It is some number such 
that the facts of Table~\ref{tab:son} fall out.)
%%
\begin{equation}
\begin{split}
\nabla := & \phantom{\mbox{}\cup\mbox{}}
\{\auf\auf a, \alpha\zu, \auf a', \alpha\zu\zu :
                \son(a) \leq \son(a')\}  \\
        & \cup \{\auf\auf a, -\alpha\zu, \auf a', -\alpha\zu\zu :
                \son(a) \geq \son(a')\} \\
        & \cup \{\auf\auf a, \alpha\zu, \auf a', -\alpha\zu\zu :
                a' \not\in C, \son(a) \leq \son(a')\} \\
        & \cup \{\auf\auf a, -\alpha\zu, \auf a', \alpha'\zu\zu :
                a \in \{\mbox{\tt +},\mbox{\tt \#}\}\}
\end{split}
\end{equation}
%%
As things are defined, any subword of a word is in the language. We need
to mark the beginning and the end of a sequence in a special way, as
described above. This detail shall be ignored here.

$\alpha$ has a clear phonetic interpretation: it signals the rise
of the sonoricity. It has a natural correlate in what de
Saussure calls `explosive articulation'. A phoneme carrying
$\alpha$ is pronounced with explosive articulation, a phoneme carrying
$-\alpha$ is pronounced with `implosive articulation'. (See 
\cite{desaussure:grundfragen}.) So, $\alpha$ actually has
an articulatory (and an auditory) correlate. But it is a nonphonemic
feature; it has been introduced in addition to the phonemic features
in order to constrain the choice of the next phoneme. As de Saussure
remarks, it makes the explicit marking of the syllable boundary
unnecessary. The syllable boundary is exactly where the implosive
articulation changes to explosive articulation. However, some linguists
(for example van der Hulst in \shortcite{hulst:dutch}) 
%%%
\index{van der Hulst, Harry}%%%
%%%
have provided a
completely different answer. For them, a syllable is structured in
the following way.
%%
\index{onset}%%
\index{rhyme}%%
\index{nucleus}%%
\index{coda}%%
%%
\begin{equation}
\mbox{[onset \quad [nucleus \quad coda]]}
\end{equation}
%%
So, the grammar that generates the phonological strings is actually
not a regular grammar but context free (though it makes only very
limited use of phrase structure rules). $\alpha$ marks the onset,
while $- \alpha$ marks the nucleus together with the coda (which is
also called \textbf{rhyme}). So, we have three possible ways to arrive
at the constraint for the syllable structure: we postulate an
axiom, we introduce a new feature, or we assume more structure.

We shall finally return to the question of spelling out the
relation between deep and surface phonological representations.
We describe here the simplest kind of a machine that transforms
strings into strings, the {\it finite state transducer}.
%%
\begin{defn}
%%%
\index{transducer!finite state}%%
%%%
Let $A$ and $B$ be alphabets. A (\textbf{partial}) 
\textbf{finite state transducer from} $A$ \textbf{to} $B$ is a 
sextuple $\GT = \auf A, B, Q, i_0, F, \delta\zu$ such that $i_0 \in Q$,
$F \subseteq Q$ and $\delta \colon Q \times A_{\varepsilon}
\pf \wp(Q \times B^{\ast})$ where $\delta(q,\vec{x})$ is always
finite for every $\vec{x} \in A_{\varepsilon}$. $Q$ is called the
set of \textbf{states}, 
%%%
\index{state}%%
%%%
$i_0$ is called the \textbf{initial state},
%%%
\index{state!initial}%%
%%%
$F$ the set of \textbf{accepting states} 
%%%%
\index{state!accepting}%%
%%%
and $\delta$ the \textbf{transition function}. 
%%%
\index{transition function}%%
%%%%
$\GT$ is called \textbf{deterministic}
%%%
\index{transducer!deterministic finite state}%%
%%%
if $\delta(q,a)$ contains at most one element for every
$q \in Q$ and every $a \in A$.
\end{defn}
%%
We call $A$ the \textbf{input alphabet} and $B$ the
%%%
\index{alphabet!input}%%%
\index{alphabet!output}%%%
%%%
\textbf{output alphabet}. The transducer differs from a finite
automaton in the transition function. This function does
not only say into which state the automaton may change but
also what symbol(s) it will output on going into that state.
Notice that the transducer may also output an empty string
and that it allows for empty transitions. These are not
eliminable (as they would be in the finite state automaton)
since the machine may accompany the change in state by a
nontrivial output. We write
%%
\begin{equation}
q \stackrel{\vec{x} : \vec{y}}{\longrightarrow} q'
\end{equation}
%%
if the transducer changes from state $q$ with input
$\vec{x}$ ($\in A^{\ast}$) into the state
$q'$ and outputs the string $\vec{y}$ ($\in B^{\ast}$).
This is defined as follows.
%%
\begin{equation}
q \stackrel{\vec{x} : \vec{y}}{\longrightarrow} q', \quad
\mbox{ if }
\begin{cases}
 & (q',\vec{y}) \in \delta(q,\vec{x}) \\
 \\
\text{ or } &
 \text{ for some $q'', \vec{u}, \vec{u}_1, \vec{v}, \vec{v}_1:$} \\
 & q    \stackrel{\vec{u} : \vec{v}}{\longrightarrow} q''
    \stackrel{\vec{u}_1 : \vec{v}_1}{\longrightarrow} q'
\\
 & \text{ and
	$\vec{x} = \vec{u}\conc\vec{u}_1, \vec{y} = \vec{v} \conc \vec{v}_1$.}
\end{cases}
\end{equation}
%%
Finally one defines
%%
\begin{equation}
L(\GT) := \{\auf \vec{x}, \vec{y}\zu : \mbox{ there is }
    q \in F \mbox{ with }
    i_0 \stackrel{\vec{x}: \vec{y}}{\longrightarrow} q\}
\end{equation}
%%
Transducers can be used to describe the effect of rules.
One can write, for example, a transducer $\goth{Syl}$ that
syllabifies a given input according to the constraints on
syllable structure. Its input alphabet is $A \cup \{
\mbox{\tt +}, \mbox{\tt \#}\}$, where $A$ is the set of 
phonemes, {\tt +} the word boundary and {\tt \#} the syllable 
boundary. The output alphabet is $A \times \{\mbox{\tt o}, 
\mbox{\tt n}, \mbox{\tt c}\} \cup \{\mbox{\tt +}, \mbox{\tt \#}\}$. 
Here, {\tt o} stands for `onset', {\tt n} for `nucleus,' and 
{\tt c} for `coda'. The machine annotates each phoneme stating 
whether it belongs to the onset of a syllable, to its nucleus or 
its coda. Additionally, the machine inserts a syllable boundary 
wherever necessary. (So, one may leave the input partially or 
entirely unspecified for the 
syllable boundaries. The machine will look which syllable segmentation
can or must be introduced.) Now we write a machine $\goth{AVh}$
which simulates the actions of final devoicing. It has one
state, $i_0$, it is deterministic and the transition function
consists in
$\auf \mbox{[b]}, \mbox{\tt c}\zu : \auf \mbox{[p]}, \mbox{\tt c}\zu$,
$\auf \mbox{[d]}, \mbox{\tt c}\zu : \auf \mbox{[t]}, \mbox{\tt c}\zu$,
$\auf \mbox{[g]}, \mbox{\tt c}\zu : \auf \mbox{[k]}, \mbox{\tt c}\zu$ as
well as
$\auf \mbox{[z]}, \mbox{\tt c}\zu : \auf \mbox{[s]}, \mbox{\tt c}\zu$ and
$\auf \mbox{[v]}, \mbox{\tt c}\zu : \auf \mbox{[f]}, \mbox{\tt c}\zu$.
Everywhere else we have
$\auf P, \alpha\zu : \auf P, \alpha\zu$, $P$ a phoneme,
$\alpha \in \{\mbox{\tt a},\mbox{\tt c},\mbox{\tt n}\}$.

The success of the construction is guaranteed by a general theorem
known as the {\it Transducer Theorem}. It says that the image under
transduction of a regular language is again a regular language.
The proof is not hard. First, by adding some states, we can replace 
the function $\delta \colon Q \times A_{\varepsilon} \pf 
\wp(Q \times B^{\ast})$ by a function 
$\delta^{\diamond} \colon Q^{\diamond} \times A_{\varepsilon} \pf %
\wp(Q^{\diamond} \times B_{\varepsilon})$ for some set $Q^{\diamond}$. 
The details of this construction  are left to the reader.
Next we replace this function by the function 
$\delta^2 \colon Q \times A_{\varepsilon} 
\times B_{\varepsilon} \pf \wp(Q)$. What we now have is an automaton 
over the alphabet $A_{\varepsilon} \times B_{\varepsilon}$. We now 
take over the notation from the Section~\ref{kap4}.\ref{kap4-3} and write 
$\vec{x} \sotimes \vec{y}$ for the pair consisting of $\vec{x}$ 
and $\vec{y}$. We define
%%
\begin{equation}
(\vec{u}\sotimes\vec{v}) \conc (\vec{w}\sotimes\vec{x})
    := (\vec{u}\conc\vec{w}) \sotimes (\vec{v}\conc \vec{x})
\end{equation}
%%
\begin{defn}
Let $R$ be a regular term. We define $L^2(R)$ as follows.
%%
\begin{subequations}
\begin{align}
L^2(0) & := \varnothing \\
L^2(\vec{x}\sotimes \vec{y}) & := \{\vec{x}\sotimes\vec{y}\} 
   \qquad \qquad
(\vec{x}\sotimes\vec{y} \in A_{\varepsilon} \times B_{\varepsilon}) \\
L^2(R \cdot S) & := \{\Gx \conc \Gy : \Gx \in L^2(R), \Gy \in L^2(S)\} \\
L^2(R \cup S) & := L^2(R) \cup L^2(S) \\
L^2(R^{\ast}) & := L^2(R)^{\ast} 
\end{align}
\end{subequations}
%%
\index{regular relation}%%%
\index{relation!regular}%%%
%%%
A \textbf{regular relation on} $A$ is a relation of the form
$L^2(R)$ for some regular term $R$.
\end{defn}
%%
\begin{thm}
A relation $Z \subseteq A^{\ast} \times B^{\ast}$ is
regular iff there is a finite state transducer
$\GT$ such that $L(\GT) = Z$.
\end{thm}
%%
This is essentially a consequence of the Kleene's Theorem.
In place of the alphabets $A$ we have chosen the alphabet
$A_{\varepsilon} \times B_{\varepsilon}$. Now observe
that the transitions $\varepsilon : \varepsilon$ do not
%% Hier aufpassen ich glaube das nicht %%
add anything to the language. We can draw a lot of conclusions
from this.
%%
\begin{cor}[Transducer Theorem]
%%%
\index{Transducer Theorem}%%%
%%%
\label{cor:transducer}
The following holds.
%%
\begin{dingautolist}{192}
\item
Regular relations are closed unter intersection and converse.
\item
If $H \subseteq A^{\ast}$ is regular so is $H \times B^{\ast}$.
If $K \subseteq B^{\ast}$ is regular so is $A^{\ast} \times K$.
\item
If $Z  \subseteq A^{\ast}\times B^{\ast}$ is a regular relation,
so are the projections
\begin{itemize}
\item $\pi_1[Z] := \{\vec{x} :
    \mbox{\it there is }\vec{y}\; \mbox{\it with }
        \auf \vec{x}, \vec{y}\zu \in Z\}$,
\item $\pi_2[Z] := \{\vec{y} :
    \mbox{\it there is }\vec{x}\; \mbox{\it with }
        \auf \vec{x}, \vec{y}\zu \in Z\}$.
\end{itemize}
\item
If $Z$ is a regular relation and $H \subseteq A^{\ast}$ a
regular set then $Z[H]$ also is regular.
$$Z[H] := \{\vec{y} : \mbox{\it there is }
\vec{x} \in H \; \mbox{\it with }\auf \vec{x}, \vec{y} \zu \in Z\}$$
\end{dingautolist}
\end{cor}
%%
One can distinguish two ways of using a transducer. The first is
as a machine which checks for a pair of strings whether they
stand in a particular regular relation. The second, whether
for a given string over the input alphabet there is a string
over the output alphabet that stands in the given relation to it.
In the first use we can always transform the transducer into
a deterministic one that recognizes the same set. In the second
case this is impossible. The relation $\{\auf \mbox{\tt a},
\mbox{\tt a}^n\zu : n \in \omega\}$ is regular but there is no
deterministic translation algorithm. One easily finds a
language in which there is no deterministic algorithm in any
of the directions. From the previous results we derive the
following consequence.
%%%
\begin{cor}[Kaplan \& Kay]
%%%
\index{Kaplan, Ron}%%%
\index{Kay, Martin}%%%
%%%
Let $R \subseteq A^{\ast} \times B^{\ast}$ and $S \subseteq
B^{\ast}\times C^{\ast}$ be regular relations. Then
$R \circ S \subseteq A^{\ast} \times C^{\ast}$ is regular.
\end{cor}
%%
\proofbeg
By assumption and the previous theorems, both $R \times C^{\ast}$
and $A^{\ast} \times S$ are regular. Furthermore, $(R \times C^{\ast}) 
\cap (A^{\ast} \times S) =
\{\auf \vec{x}, \vec{y}, \vec{z}\zu : \auf \vec{x}, \vec{y}\zu
\in R, \auf \vec{y}, \vec{z}\zu \in S\}$ is regular, and so is
its projection onto $A^{\ast}\times B^{\ast}$, which is exactly
$R \circ S$.
\proofend

This theorem is important. It says that the composition of rules
which define regular relations defines a regular relation again. 
Effectively, what distinguishes regular relations from Type 1 grammars 
is that the latter allow arbitrary iterations of the same process, 
while the former do not.

{\it Notes on this section.} 
There is every reason to believe that the mapping from 
phonemes to phones is not constant but context dependent.  
In particular, final devoicing is believed by some not to 
be a phonological process, rather, it is the effect of a 
contextually conditioned change of realization of the 
voice--feature (see \cite{portodell:voicing}). In other words, 
on the phonological level nothing changes, but the realization 
of the phonemes is changed, sometimes so radically that they 
sound like the realization of a different phoneme (though in a 
different environment). This simplifies phonological processes 
at the cost of complicating the realization map.

The idea of eliminating features was
formulated in \cite{kracht:essential} and already brought
into correspondence with the notion of implicit definability.
Concerning long and short vowels, Hungarian is an interesting 
case. The vowels {\tt i}, {\tt o}, {\tt \"o}, {\tt u}, {\tt \"u}
show length contrast alone, while the long and short forms of 
{\tt a} and {\tt e} also differ in lip attitude and/or aperture. 
%%%
\index{Hungarian}%%
%%%
Sauvageot noted in \shortcite{sauvageot:edification} that 
Hungarian moved towards a system where length alone is not 
distinctive. Effectively, it moves to eliminate the feature 
\textsf{short}.
%%
\vplatz 
\exercise 
Show that for every given string in a language there is a 
separation into syllables that conforms to the {\sl Syllable 
Structure\/} constraint.
%%%
\vplatz
\exercise
Let $\Pi_0 := \{\zeta_i : i < n\}$ be a finite set of basic
programs. Define $M := \{\zeta_i : i < n\} \cup \{\zeta_i^ {\smallsmile}
: i < n\}$. Show that for every $\mathsf{EPDL}^{\smallsmile}$
formula $\varphi$ there is a modal formula $\delta$ over the
set $M$ of modalities such that $\mathsf{PDL}^{\smallsmile}
\vdash \delta \dpf \varphi$. {\it Remark.} A modal formula is
a formula that has no test, and no $\cup$ and $;$. Whence it
can be seen as a $\mathsf{PDL}^{\smallsmile}$--formula.
%%%
\vplatz
\exercise
The results of the previous section show that there is a translation
$^{\heartsuit}$ of $\mathsf{PDL}^{\smallsmile}(M)$ into
$\mathsf{QML}(M)$. Obviously, the problematic symbols are $^{\ast}$
and $^{\smallsmile}$. With respect to $^{\smallsmile}$ the technique
shown above works. Can you suggest a perspicuous translation of
$[\alpha^{\ast}]\varphi$? {\it Hint.} $[\alpha^{\ast}]\varphi$
holds if $\varphi$ holds in the smallest set of worlds closed under
$\alpha$--successors containing the current world. This can be
expressed in $\mathsf{QML}$ rather directly.
%%%
\vplatz
\exercise
Show that in Theorem~\ref{thm:regbeth} the assumption of regularity
is necessary. {\it Hint.} For example, show that the logic of
$L = \{\mbox{\tt a}^{2^n}\mbox{\tt c}\mbox{\tt a}^n : n \in
\omega\}$ fails to have the global Beth--property.
%%%
\vplatz
\exercise
\label{ex:beth}
Prove Lemma~\ref{lem:constantbeth}.
%%
\vplatz%% 
\exercise%%
%%%
\index{Indo--European}%%
%%%
One of the aims of historical linguistics is to reconstruct the 
affiliation of languages, preferrably by reconstructing a parent 
language for a certain group of languages and showing how the 
languages of that group developed from that parent language. The 
success of the reconstruction lies in the establishment of so--called 
sound correspondences. In the easiest case they take the shape of 
correspondences between sounds of the various languages. Let us 
take the Indo--European (I.--E.) languages. The ancestor of this 
language, called Indo--European, is not known directly to us, if it 
at all existed. The proof of its existence is --- among other --- 
the successful establishment of such correspondences. Their 
reliability and range of applicability have given credibility to 
the hypothesis of its existence. Its sound structure is reconstructed, 
and is added to the sound correspondences. (We base the correspondence 
on the written language, viz.\ transcriptions thereof.)
%%%
\begin{center}
\begin{tabular}{|l|l|l|l|l|}
\hline
I--E & Sanskrit & Greek & Latin & (meaning) \\
\hline\hline
{\tt g\textsuperscript{h}ermos} & {\tt gharma\textsubdot{h}} & 
	{\tt thermos} & {\tt formus} & `warm' \\
{\tt o\textsubarch{u}is} & {\tt avi\textsubdot{h}} & 
	{\tt ois} & {\tt ovis} & `sheep' \\
{\tt s\textsubarch{u}os} & {\tt sva\textsubdot{h}} & 
	{\tt hos} & {\tt suus} & `his' \\
{\tt sept\textsubring{m}} & {\tt sapta} & 
	{\tt hepta} & {\tt septem} & `seven' \\
{\tt dek\textsuperscript{u}\textsubring{m}} & {\tt da\'sa} & 
	{\tt deka} & {\tt decem} & `ten' \\
{\tt ne\textsubarch{u}os} & {\tt nava\textsubdot{h}} & 
	{\tt neos} & {\tt novus} & `new' \\
{\tt \textroundcap{g}enos} & {\tt jana\textsubdot{h}} & 
	{\tt genos} & {\tt genus} &  `gender' \\
{\tt s\textsubarch{u}epnos} & {\tt svapna\textsubdot{h}} & 
	{\tt hypnos} & {\tt somnus} & `sleep'
\\\hline
\end{tabular}
\end{center}
%%
Some sounds of one language have exact correspondences in another.
For example, I.--E.\ $^{\ast}${\tt p} corresponds to {\tt p} across 
all languages. (The added star indicates a reconstructed entity.)
With other sounds the correspondence is not so clear. I.--E.\ 
$^{\ast}${\tt e} and $^{\ast}${\tt a} become {\tt a} in Sanskrit. 
Sanskrit {\tt a} in fact has multiple correspondences in other 
languages. Finally, sounds develop differently in different 
environments. In the onset, I.--E.\ $^{\ast}${\tt s} becomes 
Sanskrit {\tt s}, but it becomes {\tt \textsubdot{h}} at the end 
of the word. The details need not interest us here. Write 
a transducer for all sound correspondences displayed here.
%%%
\vplatz
\exercise
(Continuing the previous exercise.)
Let $L_i$, $i < n$, be languages over alphabets $A_i$.
Show the following: {\it Suppose $R$ is a regular relation
between $L_i$, $i < n$. Then there is an alphabet $P$,
a proto--language $Q \subseteq P^{\ast}$, and regular
relations $R_i \subseteq P^{\ast} \times A_i^{\ast}$, $i < n$, 
such that (a) for every $\vec{x} \in P$ there is exactly one
$\vec{y}$ such that $\vec{x}\, R_i\, \vec{y}$ and (b) $L_i$
is the image of $P$ under $R_i$.}
%%
\vplatz
\exercise
\index{Finnish}%%
Finnish has a phenomenon called {\it vowel harmony}.
There are three kinds of vowels: back vowels ([a], [o], [u],
written {\tt a}, {\tt o} and {\tt u}, respectively), front vowels
([\ae], [\o], [y], written {\tt \"a}, {\tt \"o} and {\tt y},
respectively) and neutral vowels ([e], [i], written {\tt e}
and {\tt i}).  The principle is this.
%%
\begin{quote}
{\sl Vowel harmony (Finnish).} A word contains not both a
back and a front harmonic vowel.
\end{quote}
%%
The vowel harmony only goes up to the word boundary. So, it
is possible to combine two words with different harmony.
Examples are {\tt osakeyhti\"o} `share holder company'.
It consists of the back harmonic word {\tt osake} `share'
and the front harmonic word {\tt yhti\"o} `society'.
First, give an $\mathsf{PDL}^{\smallsmile}$--definition of
strings that satisfy Finnish vowel harmony. It follows that
there is a finite automaton that recognizes this language.
Construct such an automaton. {\it Hint.} You may need to
explicitly encode the word boundary.

 \section{Axiomatic Classes II: Exhaustively Ordered Trees}
\label{kap5-4}%%
\nocite{doner:tree}%%
\nocite{thatcherwright}%%
%
%
%
The theorem by B\"uchi on axiomatic classes of strings has a
very interesting analogon for exhaustively ordered trees.
We shall prove it here; however, we shall only show those
facts that are not proved in a completely similar way.
Subsequently, we shall outline the importance of this
theorem for syntactic theory. The reader should consult
Section~\ref{kap1}.\ref{kap1-4} for notation. Ordered
trees are structures over a language that has two binary
relation symbols, $\sqsubset$ and $<$. We also take
labels from $A$ and $N$ (!) in the form of constants
and get the language $\mathsf{MSO}^b$. 
%%%
\index{$\mathsf{MSO}^b$}%%%
%%%
In this language the
set of exhaustively ordered trees is a finitely
axiomatizable class of structures. We consider first
the postulates. $<$ is transitive and irreflexive,
$\uppx{x}$ is linear for every $x$, and there is a largest
element, and every subset has a largest and a smallest element
with respect to  $<$. From this it follows in particular that
below an arbitrary element there is a leaf. Here are now the
axioms listed in the order just described.
%%
\begin{subequations}
\begin{align}
& (\forall xyz)(x < y \und y < z .\pf. x < z) \\
& (\forall x)\nicht (x < x) \\
& (\forall xyz)(x < y \und x < z. \pf . y < z \oder y \doteq z \oder
    y > z) \\
& (\exists x)(\forall y)(y < x \oder y \doteq x) \\
& (\forall P)(\exists x)(\forall y)(P(x) \und y < x .\pf.
    \nicht P(y)) \\
& (\forall P)(\exists x)(\forall y)(P(x) \und y < x .\pf.
    \nicht P(y))
\end{align}
\end{subequations}
%%
In what is to follow we use the abbreviation $x \leq y := x < y \oder
x \doteq y$. Now we shall lay down the axioms for the ordering.
$\sqsubset$ is transitive and irreflexive, it is linear on the
leaves, and we have $x \sqsubset y$ iff for all leaves
$u$ below $x$ and all leaves $v \leq y$ we have $u \sqsubset v$.
Finally, there are only finitely many leaves, a fact which we can
express by requiring that every set of nodes has a smallest
and a largest member (with respect to $\sqsubset$).
We put $b(x) := \nicht (\exists y)(y < x)$.
%%
\begin{subequations}
\begin{align}
& (\forall xyz)(x \sqsubset y \und y \sqsubset z. \pf .x \sqsubset z) \\
& (\forall x)\nicht (x \sqsubset x) \\
& (\forall xy)(b(x) \und b(y) .\pf.
    x \sqsubset y \oder x \doteq y \oder y \sqsubset x) \\
& (\forall xy)(x \sqsubset y. \dpf .
    (\forall uv)(b(u) \und u \leq x \und b(v) \und v \leq y.
    \pf .u \sqsubset v)) 
\\
& (\forall P)\{(\forall x)(P(x) \pf b(x)).  \\\notag
& \quad \pf. \phantom{\mbox{}\und\mbox{}}
    (\exists y)(P(y) \und (\forall z)(P(z) \pf
    \nicht (y \sqsubset z))) \\\notag
& \quad \phantom{\mbox{}\pf. \mbox{}}
    \und (\exists y)(P(y) \und (\forall z)(P(z) \pf
    \nicht (z \sqsupset y)))\}
\end{align}
\end{subequations}
%%
Thirdly,  we must regulate the distribution of the labels.
%%
\begin{subequations}
\begin{align}
%& (\forall x)(\goder \auf \uli{a}(x) : a \in N\zu \dpf 
%	\nicht \goder \auf \uli{A}(x) : A \in N\zu) \\
& (\forall x)(b(x) \dpf \goder \auf \uli{a}(x) : a \in A\zu) \\
%& (\forall x)(b(x) \pf \gund \auf \uli{a}(x) \pf \nicht \uli{b}(x) :
%    a \neq b\zu) \\
& (\forall x)(\nicht b(x) \pf \goder \auf \uli{A}(x) : A \in N\zu) \\
& (\forall x)(\gund \auf \uli{\alpha}(x) \pf \nicht \uli{\beta}(x)
    : \alpha \neq \beta\zu)
\end{align}
\end{subequations}
%%
The fact that a tree is exhaustively ordered is described by the
following formula.
%%
\begin{equation}
(\forall xy)(\nicht (x \leq y \oder y \leq x) .\pf .
    x \sqsubset y \oder y \sqsubset x)
\end{equation}
%%
\begin{prop}
The following are finitely $\mathsf{MSO}^b$--axiomatizable
classes.
%%
\begin{dingautolist}{192}
\item The class of ordered trees.
\item The class of finite exhaustively ordered trees.
\end{dingautolist}
\end{prop}
%%
Likewise we can define a quantified modal language.
However, we shall change the base as follows, using the
results of Exercise~\ref{ueb:schwester}. We assume 8 operators,
$M_8 := \{\oben, \oben^+, \unten, \unten^+, \rechts, \rechts^+,
\links, \links^+\}$, 
%%%
\index{$\oben$, $\oben^+$, $\unten$, $\unten^+$, $\rechts$, $\rechts^+$, $\links$, $\links^+$}%%% 
%%%
which correspond to the 
relations $\prec$, $<$, $\succ$, $>$, {\it immediate left
sister of}, {\it left sister of}, {\it immediate right sister
of}, as well as {\it right sister of}. These relations are
MSO--definable from the original ones, and conversely the original 
relations can be MSO--defined from the present ones. Let 
$\GT = \auf T, <, \sqsubset\zu$ be an exhaustively ordered tree. 
Then we define $R \colon M_8 \pf \wp(T)$ as follows.
%%
\begin{equation}
\begin{split}
x\; R(\rechts^+)\; y & := x \sqsubset y \und (\exists z)(x \prec z
\und y \prec z) \\
x\; R(\rechts)\; y & := x\; R(\rechts^+)\; y \und
    \nicht (x\; R(\rechts^+) \circ R(\rechts^+)\; y) \\
x\; R(\links^+)\; y & := x \sqsupset y \und (\exists z)(x \prec z
	\und y \prec z) \\
x\; R(\links)\; y & := x\; R(\links^+)\; y \und
    \nicht (x\; R(\links^+) \circ R(\links^+)\; y) \\
x\; R(\oben^+)\; y & := x < y \\
x\; R(\oben)\; y & := x \prec y \\
x\; R(\unten^+)\; y & := x > y \\
x\; R(\unten)\; y & := x \succ y 
\end{split}
\end{equation}
%%
The resulting structure we call $M(\GT)$. Now if $T$ as well as
$R$ are given, then the relations $\prec$, $\succ$, $<$, $>$,
and $\sqsubset$, as well as $\sqsupset$ are definable.
First we define $\oben^{\ast} \varphi := \varphi \oder
\oben^+ \varphi$, and likewise for the other relations.
Then $R(\oben^{\ast}) = \Delta \cup R(\oben^+)$.
%%
\begin{equation}
\begin{split}
\prec & = R(\oben) & \succ & = R(\unten) \\
< & = R(\oben^+) & > & = R(\unten^+) \\
\sqsubset & = R(\oben^{\ast}) \circ R(\rechts^+) \circ
    R(\unten^{\ast}) \\
\sqsupset & = R(\oben^{\ast}) \circ R(\links^+) \circ
    R(\unten^{\ast}) 
\end{split}
\end{equation}
%%
Analogously, as with the strings we can show that the following
properties are axiomatizable: (a) that $R(\links^+)$
is transitive and irreflexive with converse relation
$R(\rechts^+)$; (b) that $R(\links^+)$ is the transitive
closure of $R(\links)$ and $R(\rechts^+)$ the transitive closure
of $R(\rechts)$. Likewise for $R(\oben^+)$ and $R(\oben)$,
$R(\unten^+)$ and $R(\unten)$. With the help of the axiom
below we axiomatically capture the condition that $\uppx{x}$
is linear:
%%
\begin{equation}
\oben^+ p \und \oben^+ q .\pf.
\oben^+ (p \und q) \oder \oben^+ (p \und \oben^+ q)
\oder \oben^+ (q \und \oben^+ p)
\end{equation}
%%
The other axioms are more complex. Notice first the
following.
%%
\begin{lem}
Let $\auf T, < , \sqsubset\zu$ be an exhaustively ordered tree
and $x, y \in T$. Then $x \neq y$ iff
(a) $x < y$ or (b) $x > y$ or (c) $x \sqsubset y$ or
(d) $x \sqsupset y$.
\end{lem}
%%
Hence the following definitions.
%%
\begin{align}
\auf\neq\zu \varphi & := \oben^+ \varphi \oder \unten^+ \varphi
         \oder
    \oben^{\ast} \rechts^+ \unten^{\ast} \varphi
         \oder
    \oben^{\ast} \links^+ \unten^{\ast} \varphi \\
\master \varphi & := \varphi \und [\neq] \varphi
\end{align}
%%
So we add the following set of axioms.
%%
\begin{multline}
\{\master \varphi \pf \qrechts^+ \varphi,
\master \varphi \pf \qlinks^+ \varphi,  
\master \varphi \pf \qoben^+ \varphi,  
\master \varphi \pf \qunten^+ \varphi \\
\master \varphi \pf \master\master \varphi,  
\master \varphi \pf \varphi, 
\varphi \pf \master \nicht\master\nicht \varphi\}
\end{multline}
%%
(Most of them are already derivable. The axiom system is therefore
not minimal.) These axioms see to it that in a connected structure 
every node is reachable from any other by means of the basic relations,
moreover, that it is reachable in one step using $R(\master)$.
Here we have
%%
\begin{equation}
R(\master) =
\{\auf x,y\zu : \mbox{\it there is \/} z:
x \leq z \geq y\}
\end{equation}
%%
Notice that this always holds in a tree and that conversely it
follows from the above axioms that $R(\oben^+)$ possesses a
largest element. Now we put
%%
\begin{equation}
b(\varphi) := \varphi \und \qunten \bot \und [\neq] \nicht \varphi
\end{equation}
%%
$b(\varphi)$ holds at a node $x$  iff
$x$ is a leaf and $\varphi$ is true exactly at $x$. Now we can
axiomatically capture the conditions that
$R(\rechts^+)$ must be linear on the set of leaves.
%%
\begin{equation}
\qunten \bot \und \auf\neq\zu b(q). \pf.
\rechts^+ p \oder \links^+ p
\end{equation}
%%
Finally, we have to add axioms which constrain the distribution of
the labels. The reader will be able to supply them. A 
forest is defined here as the disjoint union of trees.
%%
\begin{prop}
The class of exhaustively ordered forests is fi\-ni\-te\-ly
$\mathsf{QML}^b$--axi\-o\-ma\-ti\-sa\-ble.
\end{prop}
%%
We already know that $\mathsf{QML}^b$ can be embedded into
$\mathsf{MSO}^b$. The converse is as usual somewhat difficult.
To this end we proceed as in the case of strings. We introduce
an analogon of restricted quantifiers. We define
functions $\oben$, $\oben^+$, $\unten$, $\unten^+$,
$\rechts$, $\rechts^+$, $\links$, $\links^+$, as well as
$\auf\neq\zu$ on unary predicates, whose meaning should
be self explanatory. For example
%%
\begin{subequations}
\begin{align}
(\oben \varphi)(x) & := (\exists y \succ x)\varphi(y) \\
(\oben^+ \varphi)(x) & := (\exists y > x)\varphi(y)
\end{align}
\end{subequations}
%%
where $y \not\in \fr(\varphi)$.
Finally let $O$ be defined by
%%
\begin{equation}
O(\varphi) := (\forall x)\nicht \varphi(x)
\end{equation}
%%
$O(\varphi)$ says that $\varphi(x)$ is nowhere satisfiable. Let 
$P_x$ be a  predicate variable which
does not occur in $\varphi$. Define $\{P_x/x\}\varphi$ inductively
as described in Section~\ref{kap5}.\ref{kap5-2}. Let $\gamma(P_x) =
\{\beta(x)\}$. Then we have
%%
\begin{equation}
\auf \GM, \gamma, \beta\zu \vDash \varphi\qquad
\Dpf\qquad \auf \GM, \gamma, \beta\zu \vDash
\{P_x/x\}\varphi 
\end{equation}
%%
Therefore put
%%
\begin{equation}
(E x)\varphi(x) := (\exists P_x)(\nicht O(P_x) \und
    O(P_x \und \auf\neq\zu P_x).  \pf .\{P_x/x\}\varphi)
\end{equation}
%%
Because of this we have for all exhaustively
ordered trees $\GT$
%%
\begin{equation}
\auf \GT, \gamma, \beta\zu \vDash (\exists x)\varphi\qquad
\Dpf\qquad
\auf \GT, \gamma, \beta\zu \vDash
(E x)\varphi
\end{equation}
%%
Let again $h \colon P \pf \mbox{\it PV\/}$ be a bijection from
the set of predicate variables of $\mathsf{MSO}^b$ onto
the set of proposition variables or $\mathsf{QML}^b$.
%%
\begin{equation}
\begin{split}
(\uli{a}(x))^{\diamond} & := Q^a & (P(y))^{\diamond} & := h(P) \\
(\oben \varphi)^{\diamond} & := \oben \varphi^{\diamond} &
(\unten \varphi)^{\diamond} & := \unten \varphi^{\diamond} \\
(\links\varphi)^{\diamond} & := \links \varphi^{\diamond} &
(\rechts\varphi)^{\diamond} & := \rechts \varphi^{\diamond} \\
(\oben^+\varphi)^{\diamond} & := \oben^+ \varphi^{\diamond} &
(\unten^+\varphi)^{\diamond} & := \unten^+ \varphi^{\diamond} \\
(\rechts^+\varphi)^{\diamond} & := \rechts^+ \varphi^{\diamond} &
(\links^+ \varphi))^{\diamond} & := \links^+ \varphi^{\diamond} \\
(\nicht \varphi)^{\diamond} & := \nicht \varphi^{\diamond} &
(O(\varphi))^{\diamond} & := \master \nicht \varphi^{\diamond} \\
(\varphi_1 \und \varphi_2)^{\diamond} &
    := \varphi_1^{\diamond} \und \varphi_2^{\diamond} &
(\varphi_1 \oder \varphi_2)^{\diamond} &
    := \varphi_1^{\diamond} \oder \varphi_2^{\diamond} \\
((\exists P)\varphi)^{\diamond} & := (\exists h(P))\varphi^{\diamond} &
((\forall P)\varphi)^{\diamond} & := (\forall h(P))\varphi^{\diamond}
\end{split}
\end{equation}
%%
Then the desired embedding of $\mathsf{MSO}^b$
into $\mathsf{QML}^b$ is shown.
%%
\begin{thm}
Let $\varphi$ be an $\mathsf{MSO}^b$--formula with at most
one free variable, the object variable $x_0$. Then there exists
a $\mathsf{QML}^b$--formula $\varphi^M$ such that for all
exhaustively ordered trees $\GT$:
%%
\begin{equation}
\auf \GT, \beta\zu \vDash \varphi(x_0)
\text{ iff }\auf M(\GT), \beta(x_0)\zu \vDash
\varphi(x_0)^M
\end{equation}
%%
\end{thm}
%%
\begin{cor}
\label{cor:mqlb} Modulo the identification $\GT \mapsto M(\GT)$
$\mathsf{MSO}^b$ and $\mathsf{QML}^b$ define the same model
classes of exhaustively ordered trees. Further: $\CK$ is a
finitely axiomatizable class of $\mathsf{MSO}^b$--struc\-tu\-res 
iff $M(\CK)$ is a finitely axiomatizable class of
$\mathsf{QML}^b$--structures.
\end{cor}
%%
For the purpose of definition of a code we suspend the difference
between terminal and nonterminal symbols.
%%
\begin{defn}
\index{code}%%
\index{faithfulness}%%
%%
Let $G = \auf \Sigma, N,A,R\zu$ be a CFG$^{\ast}$
and $\varphi \in \mathsf{QML}^b$ a constant formula (with constants
over $A$). We say, $G$ is \textbf{faithful for} $\varphi$ if there is a
set $H_{\varphi} \subseteq N$ such that for every tree $\GT$ and
every node $w \in T$: $\auf \GT, w\zu \vDash \varphi$ iff
$\ell(w) \in H_{\varphi}$. We also say that $H_{\varphi}$
\textbf{codes} $\varphi$ \textbf{with respect to} $G$.
Let $\varphi$ be a $\mathsf{QML}^b$--formula and $n$ a
natural number. An $n$--\textbf{code for} $\varphi$ is a pair
$\auf G, H\zu$ such that $L_B(G)$ is the set of all at most
$n$--ary branching, finite, exhaustively ordered trees over
$A \cup N$ and $H$ codes $\varphi$ in $G$. $\varphi$ is called
$n$--\textbf{codable} 
%%%
\index{formula!codable}%%%
%%%
if there is an $n$--code for $\varphi$.  $\varphi$ is called 
\textbf{codable} if there is an $n$--code
for $\varphi$ for every $n$.
\end{defn}
%%
Notice that for technical reasons we must restrict ourselves
to at most $n$--branching trees since we can otherwise not write
down a CFG$^{\ast}$ as a code. Let
$G = \auf \Sigma, N, A, R\zu$ and $G' = \auf \Sigma', N', A, R'\zu$
be grammars$^{\ast}$ over $A$. 
%%%
\index{grammar$^{\ast}$!product of {\faul}s}%%
%%%
The \textbf{product} is defined by
%%
\begin{equation}
G \times G' = \auf \Sigma\times\Sigma', N \times N', A, R \times R'\zu
\end{equation}
%%
where
%%
\begin{multline}
R \times R' := 
     \{\auf X,X'\zu \pf \auf \alpha_0, \alpha'_0\zu \dotsb
    \auf \alpha_{n-1},\alpha'_{n-1}\zu : \\
    X \pf \alpha_0\dotsb \alpha_{n-1} \in R,
    X' \pf \alpha'_0\dotsb \alpha'_{n-1} \in R'\} 
\end{multline}
%%
To prove the analogon of the Coding Theorem (\ref{thm:code}) for 
strings we shall have to use a trick. As one can easily show
the direct extension on trees is false since we have
also taken the nonterminal symbols as symbols of the
language. So we proceed as follows. Let $h \colon N \pf N'$ be
a map and $\GT = \auf T, <, \sqsubset, \ell\zu$ a tree with
labels in $A \cup N$. Then let $h[\GT] :=
\auf T, <, \sqsubset, h_A \circ \ell\zu$ where
$h_A \restriction N := h$ and $h_A(a) := a$ for all $a \in A$.
%%%
\index{projection}%%
%%%
Then $h[\GT]$ is called a \textbf{projection of} $\GT$.
If $\CK$ is a class of trees, then let $h[\CK] :=
\{h[\GT] : \GT \in \CK\}$. 
%%
\begin{thm}[Thatcher \& Wright, Doner]
%%%
\index{Thatcher, J.~W.}%%
\index{Wright, J.~B.}%%
\index{Doner, J.~E.}%%
Let $A$ be a terminal alphabet, $N$ a nonterminal alphabet 
and $n \in \omega$. A class of exhaustively ordered, at most
$n$--branching finite trees over $A \cup N$ is finitely
axiomatizable in $\mathsf{MSO}^b$ iff it is the
projection onto $A \cup N$ of a context free$^{\ast}$ class of
ordered trees over some alphabet.
\end{thm}
%%
Here a class of trees is \textbf{context free}$^{\ast}$ if it
is the class of trees generated by some CFG$^{\ast}$.
Notice that the symbol $\varepsilon$ is not problematic as it
was for regular languages. We may look at it as an
independent symbol which can be the label of a leaf. However,
if this is to be admitted, we must assume that the terminal
alphabet may be $A_{\varepsilon}$ and not $A$. Notice that
the union of two context free sets of trees is not necessarily
itself context free. (This again is different for regular
languages, since the structures did not contain the nonterminal
symbols.)

From now on the proof is more or less the same. First one shows
the codability of $\mathsf{QML}^b$--formulae. Then one argues
as follows. Let $\auf G, H\zu$ be the code of a formula $\varphi$.
We restrict the set of symbols (that is, both $N$ as well as $A$)
to $H$. In this way we get a grammar$^{\ast}$ which only generates
trees that satisfy $\varphi$. Finally we define the projection
$h \colon H \pf A \cup N$ as follows. Put $h(a) := a$, $a \in A$,
and $h(Y) := X$ if $L_B(G) \vDash (\forall x)( \uli{Y}(x) \pf %
\uli{X}(x))$. In order for this to be well defined we must 
therefore have for all $Y \in H$ an $X \in N$ with this property. 
In this
%%%
\index{code!uniform}%%
%%%
case we call the code \textbf{uniform}. Uniform codability follows
easily from codability since we can always construct products
$G \times G'$ of grammars$^{\ast}$ so that
$G = \auf \Sigma, N, A, R\zu$ and $L_B(G \times G') \vDash
\uli{X}(\auf x,y\zu)$ iff $L_B(G) \vDash
\uli{X}(x)$. The map $h$ is nothing but the projection onto
the first component.
%%
\begin{thm}
\label{thm:codeb}
Every constant $\mathsf{QML}^b$--formula is uniformly codable.
\end{thm}
%%
\proofbeg We only deliver a sketch of the proof. We choose an $n$
and show the uniform $n$--codability. For ease of exposition we
illustrate the proof for $n = 2$. For the formulae $\uli{a}(x)$, $a
\in A$, and $\uli{Y}(x)$, $Y \in N$, nothing special has to be
done. Again, the booleans are easy. There remain the modal
operators and the quantifiers. Before we begin we shall introduce
a somewhat more convenient notation. As usual we assume that we
have a grammar$^{\ast}$ $G = \auf \Sigma, N, A, R\zu$ as well as
some sets $H_{\eta}$ for certain formulae. Now we take the product
with a new grammar$^{\ast}$ and define $H_{\varphi}$. In place of
explicit labels we now use the formulae themselves, where $\eta$
stands for the set of labels from $H_{\eta}$.

The basic modalities are as follows. Put
%%
\begin{equation}
\boldmath{2} := \auf \{0,1\}, \{0,1\}, A, R_2\zu
\end{equation}
%%
where $R_2$ consists of all possible $n$--branching rules of a 
grammar in standard form. To code $\unten \eta$, we form the 
product of $G$ with \textbf{2}. However, we only choose a subset 
of rules and of the start symbols. Namely, we put $\Sigma' := 
\Sigma \times \{0,1\}$ and $H'_{\eta} := H_{\eta} \times \{0,1\}$, 
$H'_{\unten \eta} := N \times \{1\}$. The rules are all rules 
of the form
%%
\begin{equation}
\begin{array}{l}
\binbaum{\unten \eta}{\top}{\eta}
\binbaum{\unten \eta}{\eta}{\top}
\binbaum{\nicht\unten \eta}{\nicht\eta}{\nicht\eta}
\end{array}
\end{equation}
%%
Now we proceed to $\oben\eta$. Here $\Sigma'_{\oben\eta}
:= N \times \{0\}$.
%%
\begin{equation}
\begin{array}{l}
\binbaum{\eta}{\oben \eta}{\oben \eta}
\binbaum{\nicht\eta}{\nicht\oben\eta}{\nicht\oben\eta}
\end{array}
\end{equation}
%%
With $\rechts\eta$ we choose $\Sigma'_{\rechts\eta} := \Sigma \times
\{0\}$.
%%
\begin{equation}
\begin{array}{l}
\binbaum{\top}{\rechts\eta}{\eta}
\binbaum{\top}{\nicht\rechts \eta}{\nicht \eta}
\end{array}
\end{equation}
%%
Likewise, $\Sigma'_{\links\eta}$ is the start symbol of
$G'$ in the case of $\links\eta$.
%%
\begin{equation}
\begin{array}{l}
\binbaum{\top}{\eta}{\links\eta}
\binbaum{\top}{\nicht\eta}{\nicht \links \eta}
\end{array}
\end{equation}
%%
We proceed to the transitive relations. Notice that on binary 
branching trees $\rechts^+ \eta \dpf \rechts \eta$ and 
$\links^+ \eta \dpf \links \eta$. Now let us look at the 
relation $\unten^+\eta$.
%%
\begin{equation}
\begin{array}{l}
\binbaum{\unten^+ \eta}{\eta \oder \unten^+ \eta}{\top}
\binbaum{\unten^+ \eta}{\top}{\eta \oder \unten^+ \eta} \\
\binbaum{\quad\nicht\unten^+ \eta}{\nicht (\eta \oder \unten^+ \eta)\quad}%
{\nicht (\eta \oder \unten^+ \eta)}
\end{array}
\end{equation}
%%
The set of start symbols is $\Sigma \times \{0,1\}$.
Next we look at $\oben^+ \eta$.
%%
\begin{equation}
\begin{array}{l}
\binbaum{\eta\oder\oben^+ \eta}{\oben^+ \eta}{\oben^+\eta}
\binbaum{\nicht (\eta \oder \oben^+ \eta)}{\nicht\oben^+ %
\eta}{\nicht \oben^+\eta}
\end{array}
\end{equation}
%%
The set of start symbols is $\Sigma' := \Sigma \times \{0\}$.

Finally we study the quantifier $(\exists p)\eta$.
Let $\eta' := \eta[\mathsf{c}/p]$, where $\mathsf{c}$ is a new constant.
Our terminal alphabet is now $A \times \{0,1\}$, the
nonterminal alphabet $N \times \{0,1\}$.  We assume that
$\auf G^1, H^1_{\theta}\zu$ is a uniform code for $\theta$, $\theta$
an arbitrary subformula of $\eta'$. For every subset $\Sigma$ of
the set $\Delta$ of all subformulae of $\eta'$ we put
%%
\begin{equation}
L_{\Sigma} := \gund_{\theta \in \Sigma} \theta \und
    \gund_{\theta \in \Delta - \Sigma} \nicht\theta
\end{equation}
%%
Then $\auf G^1, H^1_{\Sigma}\zu$ is a code for $L_{\Sigma}$
where
%%
\begin{equation}
H^1_{\Sigma} := \bigcap_{\theta \in \Sigma} H^1_{\theta} \cap
    \bigcap_{\theta \in \Delta - \Sigma} (N - H^1_{\theta})
\end{equation}
%%
Now we build a new CFG$^{\ast}$, $G^2$. Put $N^2 := N \times
\{0,1\} \times \wp(N^1)$. The rules of $G^2$ are
all rules of the form
%%
\begin{equation}
\begin{array}{l}
\binbaum{\auf X, i, \Sigma\zu}{\auf Y_0, j_0, \Theta_0\zu}%
{\auf Y_1, j_1, \Theta_1\zu}
\end{array}
\end{equation}
%%
where $\auf X,i\zu \in H^1_{\Sigma}$, $\auf Y_0, j_0\zu \in %
H^1_{\Theta_0}$,  $\auf Y_1, j_1\zu \in H^1_{\Theta_1}$ and
$\Sigma \pf \Theta_0\Theta_1$ is a rule of $G^1$. (This in turn
is the case if there are $X$, $Y_0$ and $Y_1$ as well as $i$, $j_0$
and $j_1$ such that $\auf X,i\zu \pf \auf Y_0, j_0\zu\;
\auf Y_1,j_1\zu \in R$.) Likewise for unary rules. Now
we go over to the grammar$^{\ast}$ $G^3$, with $N^3 := N \times
\wp(\wp(N^1))$. Here we take all rules of the form
%%
\begin{equation}
\begin{array}{l}
\binbaum{\auf X, \BA\zu}{\auf Y_0, \BB_0\zu}{\auf Y_1,\BB_1\zu}
\end{array}
\end{equation}
%%
where $\BA$ is the set of all $\Sigma$ for which there are
$\Theta_0$, $\Theta_1$ and $i$, $j_0$, $j_1$ such that
%%
\begin{equation}
\begin{array}{l}
\binbaum{\auf X, i, \Sigma\zu}{\auf Y_0, j_0, \Theta_0\zu}%
{\auf Y_1, j_1, \Theta_1\zu}
\end{array}
\end{equation}
%%
is a rule of $G^2$.
\proofend
%%
\\
{\it Notes on this section.} From complexity theory we know that
CFLs, being in \textbf{PTIME}, actually possess a description using 
first order logic plus inflationary fixed point operator. This means 
that we can describe the set of strings in $L(G)$ for a CFG by means 
of a formula that uses first order logic plus inflationary fixed points. 
Since we can assume $G$ to be binary branching and invertible, it
suffices to find a constituent analysis of the string. This is a
set of subsets of the string, and so of too high complexity. What
we need is a first order description of the constituency in terms
of the string alone. The exercises describe a way to do this.
%%
\vplatz
\exercise
Show the following: $<$ is definable from $\prec$, likewise $>$.
Also, trees can be axiomatized alternatively with $\prec$ (or
$\succ$). Show furthermore that in ordered trees $\prec$
is uniquely determined from $<$. Give an explicit definition.
%%
\vplatz
\exercise
Let $x\, L\; y$ if $x$ and $y$ are sisters and $x \sqsubset y$.
Show that in ordered trees $L$ can be defined with $\sqsubset$
and conversely.
%%%
\vplatz
\exercise
Let $\GT$ be a tree over $A$ and $N$ such that every node that
is not preterminal is at least 2--branching. Let $\vec{x} =
x_0\dotsb x_{n-1}$ be the associated string. Define a set
$C \subseteq n^3$ as follows. $\auf i,j,k\zu \in C$ iff
the least node above $x_i$ and $x_j$ is lower
than the least node above $x_i$ and $x_k$. Further, for $X \in N$,
define $L_X \subseteq n^2$ by $\auf i,j\zu \in L_X$ iff
the least node above $x_i$ and $x_j$ has label $X$. Show that
$C$ uniquely codes the tree structure $\GT$ and $L_X$, $X \in N$,
the labelling. Finally, for every $a \in A$ we have a unary
relation $T_a \subseteq n$ to code the nodes of category $a$.
Axiomatize the trees in terms of the relations $C$, $L_X$, $X \in N$,
and $T_a$, $a \in A$.
%%%
\vplatz
\exercise
Show that a string of length $n$ possesses at most $2^{c n^3}$
different constituent structures for some constant $c$.

 \section{Transformational Grammar}
\label{kap5-5}
%
%
%
In this section we shall discuss the so--called
\textbf{Transformational Grammar}, or \textbf{TG}.
%%%%
\index{Transformational Grammar}%%
\index{TG (see Transformational Grammar)}%%
%%%%%
Transformations have been introduced by Zellig Harris. They were
operations that change one syntactic structure into another
without changing the meaning. The idea to use transformations 
has been adopted by Noam Chomsky, who developed a very rich
theory of transformations. Let us look at a simple example, 
a phenomenon known as \textbf{topicalization}.
%%
\index{topicalisation}%%
%%
\index{English}%%%
%%
\begin{align}
\label{ex:651} & \mbox{\tt Harry likes trains.} \\
\label{ex:652} & \mbox{\tt Trains, Harry likes.} 
\end{align}
%%
We have two different English sentences, of which the first is in
normal serialization, namely SVO, and the second in OSV order.
In syntactic jargon we say that in the second sentence the
object has been topicalized. (The metaphor is by the way a
dynamic one. Speaking statically, one would prefer to express that
differently.) The two sentences have different uses and
probably also different meanings, but the meaning difference
is hard to establish. For the present discussion this is however
not really relevant.  A transformation is a rule that allows us
for example to transform \eqref{ex:651} into \eqref{ex:652}.
Transformations have the form $\mbox{\rm SD} \Longrightarrow
%%%
\index{structural description}%%
\index{structural change}%%
%%%
\mbox{\rm SC}$. Here {\rm SD} stands for \textbf{structural
description} and {\rm SC} for \textbf{structural change}. The rule
\textbf{TOP}, for \textbf{topicalization}, may be formulated as follows.
%%
\begin{equation}
\label{eq:trans}
\mbox{\rm NP$_1$ V NP$_2$ Y} \Longrightarrow
    \mbox{\rm NP$_2$ NP$_1$ V Y}
\end{equation}
%%
This means the following. If a structure can be decomposed into an
NP followed by a V and a second NP followed in turn by an arbitrary 
string, then the rule may be applied. In that case it moves the 
second NP to the position immediately to the left of the first NP. 
Notice that Y is a variable for arbitrary strings while NP and V are 
variables for constituents of category NP and V, respectively. Since 
a string can possess several NPs or Vs we must have for every
category a denumerable set of variables. Alternatively, we may
write $[W]_{\text{NP}}$. This denotes an arbitrary string
which is an NP--constituent. We agree that \eqref{eq:trans} can 
also be applied to a subtree of a tree (just as the string replacement 
rules of Thue--processes apply to substrings).  

Analogously, we may formulate also the reversal of this rule:
%%
\begin{equation}
\mbox{\rm NP$_2$ NP$_1$ V Y} \Longrightarrow
    \mbox{\rm NP$_1$ V NP$_2$ Y}
\end{equation}
%%
However, one should be extremely careful with such rules. They
often turn out to be too restrictive and often also too liberal.
Let us look again at \textbf{TOP}. As formulated, it cannot be
applied to \eqref{ex:653} and \eqref{ex:655}, even though
topicalization is admissible, as \eqref{ex:654} and
\eqref{ex:656} show.
%%%
\begin{align}
\label{ex:653} & \mbox{\tt Harry might like trains.} \\
\label{ex:654} & \mbox{\tt Trains, Harry might like.} \\
\label{ex:655} & \mbox{\tt Harry certainly likes trains.} \\
\label{ex:656} & \mbox{\tt Trains, Harry certainly likes.}
\end{align}
%
The problem is that in the SD {\rm V} only stands for the verb,
not for the complex consisting of the verb and the auxiliaries.
So, we have to change the SD in such a way that it allows the
examples above. Further, it must not be disturbed by eventually
intervening adverbials.

German exemplifies a construction which is one of  
the strongest arguments in favour of transformations, namely the 
so--called V2--phenomenon. In German, the verb is at the end of the
clause if that clause is subordinate. In a main clause, however,
the part of the verb cluster that carries the inflection is
moved to second position in the sentence. Compare the following
sentences.
%%
\begin{align}
\label{ex:657} & \dotsc,\mbox{\tt da{\ss} Hans sein Auto repariert.} \\\notag
                & \mbox{..., that Hans his car repairs.} \\
\label{ex:658} & \mbox{\tt Hans repariert sein Auto.} \\\notag
                & \mbox{Hans repairs his car.} \\
\label{ex:659} & \dotsc,\mbox{\tt da{\ss} Hans nicht in die Oper %
gehen will.} \\\notag
                & \mbox{..., that Hans not into the opera go wants.} \\
\label{ex:6510} & \mbox{\tt Hans will nicht in die Oper gehen.} \\\notag
                & \mbox{Hans wants not into the opera go.} \\
\label{ex:6511} & \dotsc,\mbox{\tt da{\ss} Hans im Unterricht %
selten aufpa{\ss}t.} \\\notag
                 & \mbox{..., that Hans in class rarely %
{\sc pref}--attention.pay.} \\
\label{ex:6512} & \mbox{\tt Hans pa{\ss}t im Unterricht selten auf.} \\\notag
                 & \mbox{Hans attention.pay in class rarely {\sc pref}.}
\end{align}
%%
As is readily seen, the auxiliaries and the verb are together in the
subordinate clause, in the main clause the last of the series
(which carries the inflection) moves into second place.  Furthermore,
as the last example illustrates, it can happen that certain
prefixes of the verb are left behind when the verb moves.
In transformational grammar one speaks of
%%%%%
\index{V2--movement}%%%
%%%%%
\textbf{V2--movement}. This is a transformation that takes the
inflection carrier and moves it to second place in a main clause.
A similar phenomenon is what might be called \textbf{damit}- or 
%%%
\index{damit--split}%%%
\textbf{davor--split}, which is found mainly in northern Germany.
%%%
\index{German}%%%
%%
\begin{align}
\label{ex:6513} & \mbox{\tt Da hat er mich immer vor gewarnt.} \\\notag
                 & \mbox{{\sc da} has he me always {\sc vor} warned.} \\\notag
                 & \mbox{\it He has always warned me of that.} \\
\label{ex:6514} & \mbox{\tt Da konnte ich einfach nicht mit rechnen.} \\\notag
                 & \mbox{{\sc da} could I simply not {\sc mit} reckon.} \\\notag
                 & \mbox{\it I simply could not reckon with that.} 
\end{align}
%%
We leave it to the reader to picture the complications that arise
when one wants to formulate the transformations when V2--move\-ment
and {\rm damit}- or {\rm davor}--split may operate. Notice also
that the order of application of these rules must be reckoned
with.

A big difference between V2--movement and {\rm damit}--split is
that the latter is optional and may apply in subordinate clauses,
while the former is obligatory and restricted to main clauses.
%%
\begin{align}
\label{ex:6515} & \mbox{\tt Er hat mich immer davor gewarnt.} \\
\label{ex:6516} & ^{\ast}\mbox{\tt Er mich immer davor gewarnt hat.} \\
\label{ex:6517} & \mbox{\tt Ich konnte einfach nicht damit rechnen.} \\
\label{ex:6518} & ^{\ast}\mbox{\tt Ich damit einfach nicht rechnen konnte.}
\end{align}
%%
In \eqref{ex:6516} we have reversed the effect of both transformations
of \eqref{ex:6513}. The sentence is ungrammatical. If we only apply
V2--movement, however, we get \eqref{ex:6515}, which is grammatical.
Likewise for \eqref{ex:6518} and \eqref{ex:6517}. In contrast to
Harris, Chomsky 
%%%
\index{Harris, Zellig S.}%%%
\index{Chomsky, Noam}%%
%%%
did not construe transformations as mediating between
grammatical sentences (although also Harris did allow to pass through
illegitimate structures). He insisted that there is a two layered
process of generation of structures. First, a simple grammar
(context free, preferrably) generates so--called \textbf{deep structures}.
%%%%
\index{deep structure}%%
%%%%
These deep structures may be seen as the canonical
representations, like Polish Notation or infix notation, where the
meaning can be read off immediately. However, these structures may
not be legitimate objects of the language. For example, at deep
structure, the verb of a German sentence appears in final position
(where it arguably belongs) but alas these sentences are not
grammatical as main clauses. Hence, transformations must apply.
Some of them apply optionally, for example {\rm damit}- and {\rm
davor}--split, some obligatorily, for example V2--movement. At the
end of the transformational cycle stands the \textbf{surface
structure}.
%%%%
\index{surface structure}%%
\index{derivation}%%
%%%%
The second process is also called (somewhat ambiguously) 
\textbf{derivation}. The split between these two processes has its
advantages, as can be seen in the case of German. For if we assume
that the main clause is not the deep structure, but derived from a
deep structure that looks like a surface subordinate clause, the
entire process for generating German sentences is greatly
simplified. Some have even proposed that all languages have
universally the same deep structure, namely SVO in 
\cite{kayne:antisymmetry}; or right branching, allowing 
both SVO and SOV deep structure. The latter has been defended 
in \cite{haider:branching} (dating from 1991) and 
\cite{haider:downright,haider:extraposition}. Since
the overwhelming majority of languages belongs to either of these
types, such claims are not without justification. The differences
that can be observed in languages are then caused not by the first
process, generating the deep structure, but entirely by the
second, the transformational component. However, as might be
immediately clear, this is on the one hand theoretically possible
but on the other hand difficult to verify empirically. Let us look
at a problem. In German, the order of nominal constituents is free
(within bounds).
%%%%%
\begin{align}
\label{ex:6519} & \mbox{\tt Der Vater schenkt dem Sohn einen Hund.} \\
\label{ex:6520} & \mbox{\tt Einen Hund schenkt der Vater dem Sohn.} \\
\label{ex:6521} & \mbox{\tt Dem Sohn schenkt der Vater einen Hund.} \\
\label{ex:6522} & \mbox{\tt Dem Sohn schenkt einen Hund der Vater.} \\\notag
                 & \mbox{\it The father gives a dog to the son.}
\end{align}
%%
How can we decide which of the serializations are generated at
deep structure and which ones are not? (It is of course conceivable
that all of them are deep structure serializations and even that
none of them is.) This question has not found a satisfactory
answer to date. The problem is what to choose as a diagnostic tool
to identify the deep structure. In the beginning of
transformational grammar it was thought that the meaning of a
sentence is assigned at deep structure. The transformations are
not meaning related, they only serve to make the structure
`speakable'. This is reminiscent of Harris' idea that
transformations leave the meaning invariant, the only difference
being that Harris' conceived of transformations as mediating
between sentences of the language. Now, if we assume this then
different meanings in the sentences suffice to establish that the
deep structures of the corresponding sentences are different,
though we are still at a loss to say which sentence has which deep
structure. Later, however, the original position was given up (on
evidence that surface structure did contribute to the meaning in
the way that deep structure did) and a new level was introduced,
the so--called \textbf{Logical Form} (\textbf{LF}),
%%%%%
\index{logical form}%%
%%%%%
which was derived from surface structure by means of further
transformations. We shall not go into this, however. Suffice it
to say that this increased even more the difficulty in establishing
with precision the deep structure(s) from which a given sentence
originates.

Let us return to the sentences \eqref{ex:6519} --
\eqref{ex:6522}. They are certainly not identical.
\eqref{ex:6519} sounds more neutral, \eqref{ex:6521} and
\eqref{ex:6520} are somewhat marked, and \eqref{ex:6522} finally
is somewhat unusual. The sentences also go together with different
stress patterns, which increases the problem here somewhat.
However, these differences are not exactly semantical, and indeed
it is hard to say what they consist in.

Transformational grammar is very powerful. Every recursively
enumerable language can be generated by a relatively simple
TG. This has been shown by Stanley Peters and R.~Ritchie 
\shortcite{petersritchie:base,petersritchie:power}.
In  the exercises the reader is asked to prove a variant of these
theorems. The transformations that we have given above are
problematic for a simple reason. The place from which material
has been moved is lost. The new structure is actually not
distinguishable from the old one. Of course, often we can know
what the previous structure was, but only when we know which
transformation has been applied. However, it has been observed
that the place from which an element has been moved influences
the behaviour of the structure. For example, Chomsky 
%%%
\index{Chomsky, Noam}%%%
%%%
has argued that {\tt want to} can be contracted to {\tt wanna} in 
American English; 
%%%
\index{English}%%%
%%%
however, this happens only if no element has been placed 
between {\tt want} and {\tt to} during the derivation. For 
example, contraction is permitted in \eqref{ex:6524},
in \eqref{ex:6526} however it is not, since {\tt the man} was
the subject of the lower infinitive (standing to the left of
the verb), and had been raised from there.
%%%
\begin{align}
\label{ex:6523} & \mbox{\tt We want to leave.} \\
\label{ex:6524} & \mbox{\tt We wanna leave.} \\
\label{ex:6525} & \mbox{\tt This is the man we want to leave us alone.} \\
\label{ex:6526} & ^{\ast}\mbox{\tt This is the man we wanna leave us alone.}
\end{align}
%%
The problem is that the surface structure does not know that the
element {\tt the man} has once been in between {\tt want} and {\tt
to}. Therefore, one has assumed that the moved element leaves
behind a so--called \textbf{trace}, written $t$.
%%%
\index{trace}%%
%%%%
For other reasons the trace also got an index, which is a natural
number, the same one that is given to the moved element (=
antecedent of the trace). So, \eqref{ex:652} `really' looks like
this.
%%
\begin{equation}
\label{ex:6527} \mbox{\tt Trains$_1$, Harry likes $t_1$.}
\end{equation}
%%
We have chosen the index $1$ but any other would have done equally
well. The indices as well as the $t$ are not audible, and they
are not written either (except in linguistic textbooks, of course).
Now the surface structure contains traces and is therefore markedly
different from what we actually hear or read. Whence one assumed
--- finally --- that there is a further process turning a surface
structure into a pronounceable structure, the so--called 
\textbf{Phonological Form} (\textbf{PF}).
%%%%%
\index{Phonological Form}%%
%%%%%
PF is nothing but the phonological representation of the sentence.
On PF there are no traces and no indices, no (or hardly any)
constituent brackets.

One of the most important arguments in favour of traces and the
instrument of coindexing was the distribution of pronouns.
In the theory one distinguishes
%%%%%
\index{referential expression}%%
\index{anaphor}%%
%%%%%
\textbf{referential expressions} (like {\tt Harry} or {\tt the
train}) from \textbf{anaphors}. To the latter belong pronouns ({\tt
I}, {\tt you}, {\tt we}) as well as reflexive pronouns ({\tt
oneself}). The distribution of these three is subject to certain
rules which are regulated in part by structural criteria.
%%%
\begin{align}
\label{ex:6528} & \mbox{\tt Harry likes himself.} \\
\label{ex:6529} & \mbox{\tt Harry believed that John was responsible
    for} \\\notag
	& \quad \mbox{\tt himself.} \\
\label{ex:6530} & \mbox{\tt Harry believed to be responsible for himself.}
\end{align}
%%
{\tt Himself} is a subject--oriented anaphor. Where it appears, it
refers to the subject of the same sentence. Semantically, it is
interpreted as a variable which is identical to the variable of the
subject. As \eqref{ex:6529} shows, the domain of a reflexive ends
with the finite sentence. The antecedent of {\tt himself} must be
taken to be John, not Harry. Otherwise, we would have to have {\tt
him} in place of {\tt himself}. \eqref{ex:6530} shows that
sometimes also phonetically empty pronouns can appear. In other
languages they are far more frequent (for example in 
%%%
\index{Latin}%%%
%%%
Latin or in
%%%
\index{Italian}%%
%%%
Italian). Subject pronouns may often be omitted. One says that
these languages have an empty pronoun, called {\rm pro} (`little
PRO').
%%%
\index{little pro}%%
%%%
Additionally to the question of the subject also structural factors
are involved.
%%
\begin{align}
\label{ex:6531} & \mbox{\tt Nat was driving his car and Peter, too.} \\
\label{ex:6532} & \mbox{\tt His car made Nat happy  and Peter, too.}
\end{align}
%%
We may understand \eqref{ex:6531} in two ways: either Peter
was driving his (= Peter's) car or Nat's car.
\eqref{ex:6532} allows only one reading (on condition that `his' refers 
to Nat, which it does not have to): Peter was happy
about Nat's car, not Peter's. This has arguably nothing to do
with semantical factors, but only with the fact that in the
first sentence, but not in the second, the pronoun is bound
by its antecedent. Binding is defined as follows.
%%%
\begin{defn}
%%%
\index{binding}%%
%%%
Let $\GT$ be a tree with labelling $\ell$. $x$ \textbf{binds} $y$ if
(a) the smallest branching node that properly dominates
$x$ dominates $y$, but $x$ does not dominate $y$, and
(b) $x$ and $y$ carry the same index.
\end{defn}
%%%
The structural condition (a) of the definition is called
\textbf{c--command}.
%%%
\index{c--command}%%
%%%
(A somewhat modified definition is found below.) The antedecent
c--commands the pronoun in case of binding. In \eqref{ex:6531}
the pronoun {\tt his} is c--commanded by {\tt Nat}. For the
smallest constituent properly containing {\tt Nat} is the
entire sentence. In \eqref{ex:6532} the pronoun {\tt his} is not
c--commanded by {\tt Nat}. (This is of course not entirely clear
and must be argued for independently.)

There is a rule of distribution for pronouns that is as follows:
the reflexive pronoun has to be bound by the subject of the
sentence. A nonreflexive pronoun however may not be bound
by the subject of the sentence. This applies to German as well
as to English. Let us look at \eqref{ex:6533}.
%%%%%
\begin{equation}
\label{ex:6533} \mbox{\tt Himself, Harry likes.}
\end{equation}
%%%%%
If this sentence is grammatical, then binding is computed not only
at surface structure but at some other level. For the pronoun {\tt
himself} is not c--commanded by the subject {\tt Harry}. The
structure that is being assumed is [{\tt Himself} [{\tt Harry
likes}]]. Such consideration have played a role in the
introduction of traces. Notice however that none of the conclusions
is inevitable. They are only inevitable moves within a certain 
theory (because it makes certain assumptions). It has to be 
said though that binding was {\it the\/} central diagnostic tool 
of transformational grammar. Always if it was diagnosed that there
was no c--command relation between an anaphor and some element one
has concluded that some movement must have taken place from a
position, where c--command still applied.

In the course of time the concept of transformation has undergone
revision. TG allowed deletions, but only if they were recoverable: this
means that if one has the output structure and the name of the
transformation that has been applied one can reconstruct the input
structure. (Effectively, this means that the transformations are
partial injective functions.) In the so--called \textbf{Theory of
Government and Binding} (\textbf{GB})
%%%%
\index{GB (see Theory of Government and Binding)}%%
\index{Theory of Government and Binding}%%
\index{Chomsky, Noam}%%%
%%%%
Chomsky has banned deletion altogether from the list of options.
The only admissible transformation was movement, which was
later understood as copy and delete (which in effect had the
same result but was theoretically a bit more elegant).
The movement transformation was called \textbf{Move}--$\alpha$
%%%
\index{Move--$\alpha$}%%%
%%%
and allowed to move any element anywhere (if only the landing site
had the correct syntactic label). Everything else was regulated by
conditions on the admissibility of structures.

Quite an interesting complication arose in the form of the
so--called \textbf{parasitic gaps}.
%%%
\index{parasitic gap}%%
%%%
\begin{equation}
\label{ex:6534} \mbox{\tt Which papers did you file without reading?}
\end{equation}
%%%
We are dealing here with two verbs, which share the same direct object
({\tt to file} and {\tt to read}). However, at deep structure only
one them could have had the overt object phrase {\tt which papers} as
its object and so at deep structure we either had something like
\eqref{ex:6535} or something like \eqref{ex:6536}.
%%%
\begin{align}
\label{ex:6535} & \mbox{\tt you did file which papers without reading?} \\
\label{ex:6536} & \mbox{\tt you did file without reading which papers?}
\end{align}
%%%
It was assumed that essentially \eqref{ex:6535} was the deep structure
while the verb {\tt to read} (in its form {\tt reading}, of course)
just got an empty coindexed object.
%%%
\begin{equation}
\label{ex:6537} \mbox{\tt you did file which papers$_1$ without reading $e_1$?}
\end{equation}
%%%
However, the empty element is not bound in this configuration.
English does not allow such structures. The transformation that
moves the wh--constituent at the beginning of the sentence however
sees to it that a surface structure the pronoun is bound. This
means that binding is not something that is decided at deep
structure alone but also at surface structure. However, it cannot
be one of the levels alone (see \cite{frey:bedingungen}). We
have just come to see that deep structure alone gives the wrong
result. If one replaces {\tt which papers} by {\tt which paper
about yourself} then we have an example in which the binding
conditions apply neither exclusively at deep structure nor
exclusively at surface structure. And the example shows that
traces form an integral part of the theory.

A plethora of problems have since appeared that challenged the view and
the theory had to be revised over and over again in order to cope
with them. One problem area were the quantifiers and their scope.
In German, 
%%%
\index{German}%%%
%%%
quantifiers have scope more or less as in the surface
structure, while in English 
%%%
\index{English}%%%
%%%
matters are different (not to mention
other languages here). Another problem is coordination. In a
coordinative construction we may intuitively speaking delete
elements. However, deletion is not an option any more.  So, one
has to assume that the second conjunct contains empty elements,
whose distribution must be explained. The deep structure of
\eqref{ex:6538} is for example \eqref{ex:6539}. For many
reasons, \eqref{ex:6540} or \eqref{ex:6541} would however
be more desirable.
%%%%
\begin{align}
\label{ex:6538} & \mbox{\tt Karl hat Maria ein Fahrrad gestohlen und
    Peter} \\\notag
	& \quad \mbox{\tt ein Radio.} \\\notag
    & \mbox{Karl has Maria a bicycle stolen and Peter a radio.} \\\notag
        & \mbox{\it Karl has stolen a bicycle from Maria and %
	a radio from Peter.} \\
\label{ex:6539} & \mbox{\tt Karl$_1$ Maria ein Fahrrad}\; [\mbox{\tt
    gestohlen hat}]_2 \\\notag
    & \quad \mbox{\tt und}\; e_1\; \mbox{\tt Peter ein Radio $e_2$.} \\
\label{ex:6540} & \mbox{\tt Karl}\; [[\mbox{\tt Maria ein Fahrrad}] 
	\;\mbox{\tt und}\; [\mbox{\tt Peter ein Auto}]] \\\notag
	& \quad \mbox{\tt gestohlen hat.} \\
\label{ex:6541} & \mbox{\tt Karl}\; 
	[[\mbox{\tt Maria ein Fahrrad gestohlen hat}]\\\notag
    & \quad 
	\mbox{\tt und}\; [\mbox{\tt Peter ein Radio gestohlen hat.}]]
\end{align}
%%
We shall conclude this section with a short description of GB. It is 
perhaps not an overstatement to say that GB has been the most popular 
variant of TG, so that it is perhaps most fruitful to look at this 
theory rather than previous ones (or even the subsequent 
\textbf{Minimalist Program}).  
%%%
\index{Minimalist Program}%%%
%%%
GB is divided into several subtheories, so--called
%%%
\index{module}%%
%%%
\textbf{modules}. Each of the modules is responsible for its
particular set of phenomena. There is
%%%%
\begin{dingautolist}{192}
\item Binding Theory,
\item the ECP (Empty Category Principle),
\item Control Theory,
\item Bounding Theory,
\item the Theory of Government,
\item Case Theory,
\item $\Theta$--Theory,
\item Projection Theory.
\end{dingautolist}
%%%%
The following four  levels of representation were distinguished.
%%%
\begin{dingautolist}{202}
\item D--Structure (formerly {\it deep structure\/}),
\item S--Structure (formerly {\it surface structure\/}),
\item Phonetic Form (PF) and
\item Logical Form (LF).
\end{dingautolist}
%%%%
There is only one transformation, called \textbf{Move}--$\alpha$. It
takes a constituent of category $\alpha$ ($\alpha$ arbitrary) and
moves it to another place either by putting it in place of an
empty constituent of category $\alpha$ (substitution) or by
adjoining it to a constituent. Binding Theory however requires
that trace always have to be bound, and so movement always is into
a position c--commanding the trace. Substitution is defined as
follows. Here $X$ and $Y$ are variables for strings $\alpha$ and
$\gamma$ category symbols. $i$ is a variable for a natural number.
It is part of the representation (more exactly, it is part of the
label, which we may construe as a pair of a category symbol and a
set of natural numbers). $i$ may occur in the left hand side (SC) 
namely, if it figures in the label $\alpha$.  So, if 
$\alpha = \auf C, I\zu$, $C$ a category label and $I \subseteq 
\omega$, $\alpha \oplus i := \auf C, I \cup \{i\}\zu$.
%%%
\begin{equation}
\mbox{\rm Substitution:} \quad
    [X\; [e]_{\alpha}\; Y\; [Z]_{\alpha}\; W] 
    \Longrightarrow
    [X\; [Z]_{\alpha\oplus i}\; Y\; [t_i]_{\alpha}\; W]
\end{equation}
%%%
Adjunction is the following transformation.
%%%
\begin{equation}
\mbox{\rm Adjunction:}\quad
    [X\; [Y]_{\alpha}\; Z]_{\gamma} \Longrightarrow
        [[Y]_{\alpha\oplus i}\; [X\; [t_i]_{\alpha}\; Z]_{\gamma}]_{\gamma}
\end{equation}
%%%%
Both rules make the constituent move leftward. Corresponding rightward 
rules can be formulated analogously. (In present day theory it is 
assumed that movement is always to the left. We shall not go into this, 
however.) In both cases the constituent on the right hand side,
%%%
\index{antecedent}%%
\index{trace}%%
%%%
$[X]_{\alpha \oplus i}$, is called the \textbf{antecedent} of the
trace, $t_i$. This terminology is not arbitrary: traces in GB are
considered as anaphoric elements. In what is to follow we shall
not consider adjunction since it leads to complications that
go beyond the scope of this exposition. For details we refer 
to \cite{kracht:adjunction}. For the understanding of the
basic techniques (in particular with respect to
Section~\ref{kap5}.\ref{kap5-7}) it is enough if we look at substitution.

As in TG, the D--structure is generated first. How this is done
is not exactly clear. Chomsky 
%%%
\index{Chomsky, Noam}%%%
%%%
assumes in \shortcite{chomsky:lgb} that
it is freely generated and then checked for conformity with the
principles. Subsequently, the movement transformation operates
until the conditions for an S--structure are satisfied. Then a
copy of the structure is passed on to the component which transforms
it into a PF. (PF is only a level of representation, therefore
there must be a process to arrive at PF.) For example,
symbols like $t_i$, $e$, which are empty, are deleted together
with all or part of the constituent brackets. The original structure
meanwhile is subjected to another transformational process
until it has reached the conditions of Logical Form and is
directly interpretable semantically. Quantifiers appear in their
correct scope at LF. This model is also known as the 
%%%
\index{T--model}
%%%
\textbf{T--model}.

We begin with the phrase structure, which is conditioned by the
theory of projection. The conditions of theory of projection must
in fact be obeyed at all levels (with the exception of PF).
This theory is also known as
%%%%
\index{$\oli{X}$--syntax}%%
%%%%
$\oli{X}$--\textbf{syntax}. It differentiates between simple
categorial labels (for example V, N, A, P, I and C, to name the
most important ones) and a level of projection. The categorial
labels are either
%%%
\index{category!lexical}%%
\index{category!functional}%%
%%%%
\textbf{lexical} or \textbf{functional}. Levels of projection are natural
numbers, starting with 0. The higher the number the higher the level.
In the most popular version one distinguishes exactly 3 levels
for all categories (while in \cite{jackendoff:xbar} it was 
originally possible to specify the numbers of levels for each 
category independently).
The levels are added to the categorial label as superscripts.
So $\mbox{\rm N}^2$ is synonymous with
%%
\begin{equation}
\left[\begin{array}{l@{\quad : \quad}l}
    \mbox{\sc cat} & \mbox{\rm N} \\
    \mbox{\sc proj} & 2
    \end{array}\right]
\end{equation}
%%
If X is a categorial symbol then XP is the highest projection. In
our case NP is synonymous with $\mbox{\rm N}^2$. The rules are at
most binary branching. The non--branching rules are
%%
\begin{equation}
X^{j+1} \pf X^j
\end{equation}
%%
\index{head}%%
%%%%
$\mbox{\rm X}^j$ is the \textbf{head of} $\mbox{\rm X}^{j+1}$. There
are, furthermore, the following rules:
%%
\begin{equation}
X^{j+1} \pf X^j\quad \mbox{\rm YP},
\qquad X^{j+1} \pf \mbox{\rm YP}\quad X^j
\end{equation}
%%
\index{complement}%%
\index{specifier}%%
%%%%%
Here, {\rm YP} is called the \textbf{complement} of $\mbox{\rm X}^j$
if $j = 0$, and the \textbf{specifier} if $j = 1$.  Finally, we have
these rules.
%%
\begin{equation}
X^j \pf X^j\quad \mbox{\rm YP},
\qquad X^j \pf \mbox{\rm YP}\quad X^j
\end{equation}
%%
\index{adjunct}%%
%%%%
Here {\rm YP} is called the \textbf{adjunct of} $X^j$. The last rules
create a certain difficulty. We have two occurrences of the symbol
$\mbox{\rm X}^j$. This motivated the distinction between a
%%%
\index{category}%%
\index{segment}%%
%%%%
\textbf{category} (= connected sets of nodes carrying the same label)
and \textbf{segments} thereof.
The complications that arise from this definition have been widely
used by Chomsky in \shortcite{chomsky:barriers}. The relation
\textbf{head of} is transitive. Hence $x$ with category $\mbox{\rm N}^i$
is the head of $y$ with $\mbox{\rm N}^j$, if all nodes $z$ with
$x < z < y$ have category $\mbox{\rm N}^k$ for some $k$.
By necessity, we must have $i \leq k \leq j$.

Heads possess in addition to their category label also a
%%%
\index{subcategorization frame}%%
%%%
\textbf{subcategorization frame}. This frame determines which 
arguments the head needs and to which arguments it assigns case 
and/or a $\theta$--role. $\theta$--roles are needed to recover 
an argument in the semantic representation. For example, there 
are roles for \textbf{agent}, \textbf{experiencer}, \textbf{theme}, 
\textbf{instrument} and so on. These are coded by suggestive names 
such as $\theta_a$, $\theta_e$, $\theta_{th}$, $\theta_{inst}$, and so on.  
{\tt see} gets for example the following subcategorization frame.
%%%
\begin{equation}
\mbox{\tt see}: \qquad \auf \mbox{\rm NP}[\theta_e], \mbox{\rm
NP}[\mbox{\sc acc}, \theta_{th}]\zu
\end{equation}
%%
It is on purpose that the verb does not assign case to its subject.
It only assigns a $\theta$--role. The case is assigned only by
virtue of the verb getting the finiteness marker. The subcategorization
frames dictate how the local structure surrounding a head looks
%%%
\index{licensing}%%
%%%
like. One says that the head \textbf{licenses} nodes in the deep
structure, namely those which correspond to entries of its
subcategorization frame. It will additionally determine that
certain elements get case and/or a $\theta$--role. Case- and
$\Theta$--Theory determine which elements need
case/$\theta$--roles and how they can get them from a head.
%%%
\index{argument!internal}%%
\index{argument!external}%%
%%%%
One distinguishes between \textbf{internal} and \textbf{external
arguments}. There is at most one external argument, and it is
signalled  in the frame by underlining it. It is found at deep
structure outside of the maximal projection of the head (some
theorists also think that it occupies the specifier of the
projection of the head, but the details do not really matter
here). Further, only one of the internal arguments is a
complement. This is already a consequence of $\oli{X}$--syntax;
the other arguments therefore have to be adjuncts at D--structure.

One of the great successes of the theory is the analysis of {\tt
seem}. The uninflected {\tt seem} has the following frame.
%%
\begin{equation}
\mbox{\tt seem}: \qquad \auf \mbox{\rm INFL}^2[\theta_t]\zu
\end{equation}
%%
(INFL is the symbol of inflection. This frame is valid only for
the variant which selects infinitives.) This verb has an internal
argument, which must be realized by the complement in the
syntactic tree. The verb assigns a $\theta$--role to this
argument. Once it is inflected, it has a subject position, which
is assigned case but no $\theta$--role. A caseless NP inside the
complement must be moved into the subject position of {\tt seem}
in syntax, since being an NP it needs case. It can only appear in
that position, however, if at deep structure it has been assigned
a $\theta$--role. The subject of the embedded infinitive however
is a canonical choice: it only gets a $\theta$--role, but still
needs case.
%%%%
\begin{align}
\label{ex:6542} & \mbox{\mtt Jan$_{\seins}$ seems $[t_{\seins}$
to sleep}] \\
\label{ex:6543} & ^{\ast}\mbox{\mtt Seems \mbox{\rm [}Jan to sleep%
\mbox{\rm ]}}
\end{align}
%%%
It is therefore possible to distinguish two types of intransitive
verbs, those which assign a $\theta$--role to their subject ({\tt
fall}) and those which do not ({\tt seem}). There were general
laws on subcategorization frames, such as
%%%
\begin{quote}
\index{Burzio, Luigi}%%
%%%
{\sl Burzio's Generalization.} A verb assigns case to its
    governed NP--argument if and only it assigns a
    $\theta$--role to its external argument.
\end{quote}
%%
The Theory of Government is responsible among other for case
assignment. It is assumed that nominative and accusative could
not be assigned by heads (as we --- wrongly, at least according to
this theory --- said above) but only in a specific configuration.
The simplest configuration is that between head and complement. A
verb having a direct complement licenses a direct object position.
This position is qua structural property (being sister to an
element licensing it) assigned accusative. The following is taken
from \cite{stechowsternefeld}, p. 293.
%%
\begin{defn}
%%%%
\index{government}%%
\index{government!proper}%%
%%%%
$x$ with label $\alpha$ \textbf{governs} $y$ with label $\beta$ 
iff (1) $x$ and $y$ are dominated by the same nodes with
label $X${\rm P}, $X$ arbitrary, and (2) either $\alpha = X^0$,
where $X$ is lexical or $\alpha = \mbox{\rm AGR}^0$ and (3) $x$
c--commands $y$. $x$ \textbf{governs} $y$ \textbf{properly} if $x$
governs $y$ and either $\alpha = X^0$, $X$ lexical, or $x$ and $y$
are coindexed.
\end{defn}
%%
(Since labels are currently construed as pairs $\auf X^i, P\zu$,
where $X^i$ is a category symbol with projection and $P$ a set of
natural numbers, we say that $x$ and $y$ are coindexed if the
second component of the label of $x$ and the second component of
the label of $y$ are not disjoint.) The ECP is responsible for
the distribution of empty categories. In GB there is a whole army
of different empty categories: $e$, a faceless constituent
into which one could move, $t$, the trace, {\rm PRO} and {\rm
pro}, which were pronouns. The ECP says among other that $t$ must
always be properly governed, while  {\rm PRO} may never be
governed. We remark that traces are not allowed to move. In
Section~\ref{kap5}.\ref{kap5-7} we consider this restriction more closely.
The Bounding Theory concerns itself with the distance that
syntactic processes may cover. It (or better: notions of distance)
is considered in detail in Section~\ref{kap5}.\ref{kap5-7}. Finally, we
remark that Transformational Grammar also works with conditions on
derivations. Transformations could not be applied in any order but
had to follow certain orderings. A very important one (which was
the only one to remain in GB)
%%%
\index{cyclicity}%%
%%%
was \textbf{cyclicity}. Let $y$ be the antecedent of $x$ after
movement and $z \succ y$. Then let the interval $[x,z]$ be
called the \textbf{domain} of this instance of movement.
%%%
\begin{defn}
%%%%
\index{cyclicity}%%
\index{bounding node}%%
%%%%
Let $\Gamma$ be a set of syntactic categories.  $x$ is called a
\textbf{bounding node} if the label of $x$ is in $\Gamma$. A
derivation is called \textbf{cyclic} if for any two instances of
movement $\beta_1$ and $\beta_2$ and their domains $B_1$ and $B_2$
the following holds: if $\beta_1$ was applied before $\beta_2$
then every bounding node from $B_1$ is dominated (not necessarily
properly) by some bounding node from $B_2$ and every bounding node
from $B_2$ dominates (not necessarily properly) a bounding node
from $B_1$.
\end{defn}
%%%
Principally, all finite sentences are bounding nodes. However, it
has been argued by Rizzi (and others following him) that the
choice of bounding categories is language dependent.

{\it Notes on this section.}
This exposition may suffice to indicate how complex the theory
was. We shall not go into the details of parametrization of
grammars and learnability. We have construed transformations 
as acting on labelled (ordered) trees. No attempt has been 
made to precisify the action of transformations on trees. 
Also, we have followed common practice to write $t_{\seins}$, 
even though strictly speaking $t$ is a symbol. So, it would 
have been more appropriate to write $\mbox{\tt\textgreek{t}}_{\seins}$, 
say, to make absolutely clear that there is a symbol that gets 
erased. (In TG, deletion really erased the symbol. Today transformations 
may not delete, but deletion must take place on the way to 
PF, since there are plenty of `empty' categories.)
%%
\vplatz
\exercise
Coordinators like {\tt and}, {\tt or} and {\tt
not} have quite a flexible syntax, as was already remarked at the
end of Section~\ref{kap3}.\ref{kap3-3}. We have {\tt cat and dog}, 
{\tt read and write}, {\tt green and blue} and so on. What difficulties
arise in connection with $\oli{X}$--syntax for these words? What
solutions can you propose?
%%
\vplatz
\exercise
A transformation is called {\it minimal} if it
replaces at most two adjacent symbols by at most two adjacent
symbols. Let $L$ be a recursively enumerable language. Construct a
regular grammar $G$ and a finite set of minimal transformations
such that the generated set of strings is $L$. Here the criterion
for a derivation to be finished is that no transformation can be
applied. {\it Hint.} If $L$  is recursively enumerable there is a
Turing machine which generates $L$ from a given regular set of
strings.
%%
\vplatz
\exercise
(Continuing the previous exercise.) We additionally require that
the deep structure generated by $G$ as well as all intermediate
structures conform to $\oli{X}$--syntax.
%%
\vplatz 
\exercise 
Write a 2--LMG that accommodates German V2 and
{\rm damit}-- and {\rm davor}--split.
%%
\vplatz
\exercise
It is believed that if traces are allowed to move, we can create
unbound traces by movement of traces. Show that this is not
a necessary conclusion. However, the ambiguities that arise from
allowing such movement on condition that it does not make itself
unbound are entirely harmless.

 \section{GPSG and HPSG}
\label{kap5-6}
%
%
%
In the 1980s, several alternatives to transformational grammar were
being developed. One alternative was categorial grammar, which we have
discussed in Chapter~\ref{kap3}. Others were the grammar
formalisms that used a declarative (or model theoretic) definition
of syntactic structures. These are 
%%%%
\index{Generalized Phrase Structure Grammar}
%%%%
Generalised Phrase Structure Grammar (mentioned already in
Section~\ref{kap5}.\ref{kap5-1}) and 
%%%%
\index{LFG (see Lexical Functional Grammar)}%%
\label{Lexical Functional Grammar}%%
%%%%
\textbf{Le\-xi\-cal--Func\-tio\-nal Grammar} (\textbf{LFG}).
GPSG later developed into 
%%%%
\index{Head Driven Phrase Structure Grammar}%%
%%%%
HPSG. In  this section we shall deal mainly with GPSG and HPSG. 
Our aim is twofold. We shall give an overview of the
expressive mechanism that is being used in these theories,
and we shall show how to translate these expressive devices
into a suitable polymodal logic.

In order to justify the introduction of transformational grammar, Chomsky 
%%%
\index{Chomsky, Noam}%%%
%%%
had given several arguments to show that traditional 
theories were completely inadequate. In particular, he targeted 
the theory of finite automata (which was very popular in the 1950s) 
and the structuralism. His criticism of finite automata is up to 
now unchallenged. His negative assessment of structuralism, however, 
was based on factual errors. First of all, Chomsky has made a 
caricature of Bloomfields 
%%%
\index{Bloomfield, Leonard}%%%
%%%
structuralism by equating it with the 
claim that natural languages are strongly context free (see the 
discussion by Manaster--Ramer 
%%%
\index{Manaster--Ramer, Alexis}%%
\index{Kac, Michael B.}%%
%%%
and Kac~\shortcite{manasterramerkac:concept}). Even if
this was not the case, his arguments of the insufficiency of CFGs
are questionable. Some linguists, notably Gerald Gazdar 
%%%
\index{Gazdar, Gerald}%%%
\index{Pullum, Geoffrey}%%%
%%%
and Geoffrey Pullum, after reviewing these and other proofs
eventually came to the conclusion that contrary to what has
hitherto been believed all natural languages were context free.
However, the work of Riny Huybregts 
%%%
\index{Huybregts, Riny}%%%
%%%
and Stuart Shieber, 
%%%
\index{Shieber, Stuart}%%
%%%
which we have discussed already in Section~\ref{kap2}.\ref{kap2-6} 
put a preliminary end to this story. On the other hand, as 
Rogers~\shortcite{rogers:mso} 
%%%
\index{Rogers, James}%%%
%%%
and Kracht~\shortcite{kracht:codes} 
%%%
\index{Kracht, Marcus}%%
%%%
have later shown, the theories of 
English proposed inside of GB actually postulated an essentially 
context free structure for it. Hence English is still (from a 
theoretical point of view) strongly context free.

An important argument against context free rules has been the fact
that simple regularities of language such as agreement cannot be
formulated in them. This was one of the main arguments by Paul
Postal~\shortcite{postal:constituent}  
%%%
\index{Postal, Paul}%%%
%%%
against the structuralists 
(and other people), even though strangely enough TG and GB did not 
have much to say about it either. Textbooks only offer vague 
remarks about agreement to the effect that heads agree with their 
specifiers in certain features. Von Stechow 
%%%
\index{von Stechow, Arnim}%%%
\index{Sternefeld, Wolfgang}%%%
%%%
and Sternefeld \shortcite{stechowsternefeld} are more
precise in this respect. In order to formulate this exactly, one
needs AVSs and variables for values (and structures). These tools
were introduced by GPSG into the apparatus of context free rules,
as we have shown in Section~\ref{kap5}.\ref{kap5-1}. Since we have discussed 
this already, let us go over to word order variation. Let us note
that GPSG takes over $\oli{X}$--syntax more or less without
change. It does, however, not insist on binary branching. (It 
allows even unbounded branching, which puts it just slightly 
outside of context freeness. However, the bound on branching 
may seem unnatural, see Section~\ref{kap5}.\ref{kap5-4}.) Second,
GPSG separates the context free rules into two components: one is
responsible for generating the dominance relation, the other for
the precedence relation between sisters. The following rule
determines that a node with label {\rm VP} can have daughters,
which may occur in any order.
%%
\begin{equation}
\mbox{\rm VP} \pf
\mbox{\rm NP}[\mbox{\it nom\/}]\;
\mbox{\rm NP}[\mbox{\it dat\/}]\;
\mbox{\rm NP}[\mbox{\it acc\/}]\;
\mbox{\rm V}
\end{equation}
%%
This rule stands for no less than 24 different context free rules.
In order to get for example the German 
%%%
\index{German}
%%%
word order of the subordinate clause we now add the following condition.
%%
\begin{equation}
\mbox{\rm N} \prec \mbox{\rm V}
\end{equation}
%%
This says that every daughter with label {\rm N} is to the left of
any daughter with label {\rm V}. Hence there only remain 6 context
free rules, namely those in which the verb is at the end of the
clause. (See in this connection the examples \eqref{ex:6519} --
\eqref{ex:6522}.) For German one would however not propose this
analysis since it does not allow to put any adverbials in
between the arguments of the verb. If one uses binary branching 
trees, the word order problems reappear again in the form of order 
of discharge (for which GPSG has no special mechanism). There are 
languages for which this is better suited.
For example, Staal~\shortcite{staal:sanskrit} 
%%%
\index{Staal, J.~F.}%%
%%%
has argued that Sanskrit
has the following word orders: SVO, SOV, VOS and OVS. If we allow
the following rules without specifying the linear order, these
facts are accounted for.
%%
\begin{equation}
\mbox{\rm VP} \pf \mbox{\rm NP}[\mbox{\it nom\/}]\quad
\mbox{\rm V}^1,
\qquad
\mbox{\rm V}^1 \pf \mbox{\rm NP}[\mbox{\it acc\/}]\quad 
\mbox{\rm V}^0
\end{equation}
%%
All four possibilities can be generated --- and no more.

Even if we ignore word order variation of the kind just described there
remain a lot of phenomena that we must account for. GPSG has
found a method of capturing the effect of a single movement
transformation by means of a special device. It first of all
defines \textbf{metarules},
%%%%%
\index{metarule}%%
%%%%%
which generate rules from rules. For example, to account for movement 
we propose that in addition to $[[\dotsb Y\dotsb]_W]_V$ also the 
tree $[Y_i\; [\dotsb t_i\dotsb]_W]_V$ will be a legitimate tree. 
To make this happen, there shall be an additional unary rule that 
allows to derive the latter tree whenever the former is derivable. 
The introduction 
of these rules can be captured by a general scheme, a metarule. 
However, in the particular case at hand one must be a bit more 
careful. It is actually necessary to do a certain amount of bookkeeping 
with the categories. GPSG borrows from categorial grammar the category 
$W/Y$, where $W$ and $Y$ are standard categories. In place of the rule 
$V \pf W$ one writes $V \pf Y\; W/Y$. The official notation is
%%
\begin{equation}
\left[\begin{array}{l}
W \\
\mbox{\sc slash} \; :\;  Y
\end{array}\right]
\end{equation}
%%
How do we see to it that the feature $[\mbox{\sc slash} \; : \; Y]$ is
%%%
\index{feature}%%
%%%
correctly distributed? Also here GPSG has tried to come up with a
principled answer. GPSG distinguishes \textbf{foot features} from
\textbf{head features}. Their behaviour is quite distinct. Every
feature is either a foot feature or a head feature. The attribute
{\sc slash} is
%%%
\index{feature!head}%%
\index{feature!foot}%%
%%%
classified as a foot feature. (It is perhaps unfortunate that it
is called a feature and not an attribute, but this is a minor
issue.) For a foot feature such as {\sc slash}, the {\sc
slash}--features of the mother are the \textbf{unification} of 
the {\sc slash}--features of the daughters, which corresponds to the 
logical meet. 
%%%
\index{unification}%%
%%%%
Let us look more closely into that. If $W$ is 
an AVS and $f$ a feature then we denote by $f(W)$ the value 
of $f$ in $W$.
%%%
\begin{defn}
Let $G$ be a set of rules over AVSs. $f$ is a
\textbf{foot feature in} $G$ if for every maximally instantiated rule
$A \pf B_0\dotsb B_{n-1}$ the following holds.
%%
\begin{equation}
f(A) = \gund_{i < n} f(B_i)
\end{equation}
\end{defn}
%%%
So, what this says is that the {\sc slash}--feature can be passed
on from mother to any number of its daughters. In this way 
\cite{gazdarpullumsag:gpsg} have seen to it that parasitic gaps 
can also be handled (see the previous section on this phenomenon). 
However, extreme care is needed. For the rules do not allow to count 
how many constituents of the same category have been extracted. 
\textbf{Head features} are being distributed roughly as follows.
%%%
\begin{quote}
{\sl Head Feature Convention.}
Let $A \pf B_0 \dotsb B_{n-1}$ be a rule with head $B_i$, and
$f$ a head feature. Then $f(A) = f(B_i)$.
\end{quote}
%%
The exact formulation of the distribution scheme for head features
however is much more complex than for foot features.  We shall not
go into the details here.

This finishes our short introduction to GPSG. It is immediately
clear that the languages generated by GPSG are context free
if there are only finitely many category symbols and bounded 
branching. In order for this to be the case, the syntax of paths 
in an AVS was severely restricted.
%%%%
\begin{defn}
%%%
\index{path}%%%
%%%
Let $A$ be an AVS. A \textbf{path in} $A$ is a sequence
$\auf f_i : i < n\zu$ such that $f_{n-1} \circ \dotsb \circ f_0(A)$
is defined. The value of this expression is the \textbf{value} of the
path.
\end{defn}
%%%%
In \cite{gazdarpullumsag:gpsg} it was required that only those paths
were legitimate in which no attribute occurs twice. In this way the
finiteness is a simple matter. The following is left to the reader
as an exercise.
%%%%
\begin{prop}
\label{prop:endkat}
Let $A$ be a finite set of attributes and $F$ a finite set of paths
over $A$. Then every set of pairwise non--equivalent AVSs is finite.
\end{prop}
%%%%%
Subsequently to the discovery on the word order of Dutch and Swiss
German this restriction finally had to fall. Further, some people
had anyway argued that the syntactic structure of the verbal complex
is quite different, and that this applies also to German. The verbs
in a sequence of infinitives were argued to form a constituent, the
so called
%%%
\index{verb cluster}%%
\index{German}%%%
%%%%
\textbf{verb cluster}. This has been claimed in the GB framework for
German and Dutch. 
%%%
\index{Dutch}%%%
%%%
Also, Joan Bresnan, Ron Kaplan, Stanley Peters and Annie Zaenen 
argue in \shortcite{bresnanetal:dutch}
%%%
\index{Bresnan, Joan}%%%
\index{Kaplan, Ron}%%%
\index{Peters, Stanley}%%
\index{Zaenen, Annie}%%%
%%%
for a different analysis, based on principles of LFG. Central to LFG 
is the assumption that there are three (or even more) distinct 
structures that are being built simultaneously:
%%%
\index{f--structure}\index{a--structure}\index{c--structure}%%%
%%
\begin{dinglist}{43}
\item
\textbf{c--structure} or constituent structure: this is the structure
where the linear precedence is encoded and also the syntactic
structure.
\item
\textbf{f--structure} or functional structure: this is the structure
where the grammatical relations (subject, object) but also
discourse relations (topic) are encoded.
\item
\textbf{a--structure} or argument structure: this is the structure
that encodes argument relations ($\theta$--roles).
\end{dinglist}
%%
A rule specifies a piece of c-, f- and a--structure together with
correspondences between the structures.
For simplicity we shall ignore a--structure from now on. An
example is provided in Table~\ref{tab:lfg}.
%%
\begin{table}
\caption{An LFG--Grammar}
\label{tab:lfg}
$$\begin{array}{lccc}
\mbox{\rm S} & \quad\pf\quad & \mbox{\rm NP} & \mbox{\rm VP} \\
             &               & (\uparrow \mbox{\sc subj} =
             \downarrow) & \uparrow = \downarrow \\
             \\
\mbox{\rm NP} & \quad \pf \quad & \mbox{\rm Det} & \mbox{\rm N} \\
              &   & \uparrow = \downarrow & \uparrow = \downarrow
              \\
              \\
\mbox{\rm VP} & \quad\pf\quad & \mbox{\rm V} & \mbox{\rm PP} \\
              &               & \uparrow = \downarrow &
                (\uparrow \mbox{\sc obj} = \downarrow)
\end{array}$$
\end{table}
%%
The rules have two lines: the upper line specifies a context free
phrase structure rule of the usual kind for the c--structure. The 
lower line tells us how the c--structure relates to the f--structure. 
These correspondences will allow to define a unique f--structure
(together with the universal rules of language). The rule if
applied creates a local tree in the c--structure, consisting of
three nodes, say 0, 00, 01, with label {\rm S}, {\rm NP} and {\rm
VP}, respectively. The corresponding f--structure is different.
This is indicated by the equations. To make the ideas precise, we
shall assume two sets of nodes in the universe, $C$ and $F$, which
are sets of c--structure and f--structure nodes, respectively. And we
assume a function {\sc func}, which maps $C$ to $F$. It is clear
now how to translate context free rules into first--order
formulae. We directly turn to the f--structure statements.
C--structure is a tree, F--structure is an AVS. Using the function
{\sc up} to map a node to its mother, the equation $(\uparrow
\mbox{\sc subj} = \downarrow)$ is translated as follows:
%%
\begin{equation}
(\uparrow \mbox{\sc subj} = \downarrow)(x) :=
\mbox{\sc subj} \circ \mbox{\sc func} \circ \mbox{\sc up}(x)
\doteq \mbox{\sc func}(x)
\end{equation}
%%
In simpler terms: I am my mother's subject. The somewhat simpler
statement $\uparrow = \downarrow$ is translated by
%%
\begin{equation}
(\uparrow = \downarrow)(x) :=
\mbox{\sc func} \circ \mbox{\sc up}(x) \doteq \mbox{\sc func}(x)
\end{equation}
%%
Here the f--structure does not add a node, since the predicate
installs itself into the root node (by the second condition),
while the subject NP is its {\sc subj}--value. Notice that the
statements are local path equations, which are required to hold of
the c--structure node under which they occur. LFG uses the fact
that f--structure is flatter than c--structure to derive the Dutch
and Swiss German sentences using rules of this kind, despite the
fact that the c--structures are not context free.

While GPSG and LFG still assume a phrase structure skeleton that
plays an independent role in the theory, HPSG actually offers a
completely homogeneous theory that makes no distinction between
the sources from which a structure is constrained. What made this
possible is the insight that the attribute value formalism can
also encode structure. A very simple possibility of taking care of
the structure is the following. Already in GPSG there was a
feature {\sc subcat} whose value was the subcategorization frame
of the head. Since the subcategorization frame must map into a
structure we require that in the rules
%%
\begin{equation}
X \pf Y[\mbox{\sc subcat}\; : \; A]\quad B
\end{equation}
%%
where $B \leq A$. (Notice that the order of the constituents does 
not play any role.) This means nothing but that $B$ is being
subsumed under $A$ that is to say that it is a special $A$. The
difference with GPSG is now that we allow to stack the feature
{\sc subcat} arbitrarily deep. For example, we can attribute to
the German word {\tt geben} (`to give') the following category.
%%%%
\begin{equation}
\left[\begin{array}{l@{\; :\;}l}
\mbox{\sc cat} & v \\
\mbox{\sc subcat} &
    \left[
    \begin{array}{l@{\; :\;}l}
        \mbox{\sc cat} & \mbox{\rm NP} \\
    \mbox{\sc case} & \mbox{\it nom\/} \\
    \mbox{\sc subcat} &
        \left[
        \begin{array}{l@{\; :\;}l}
            \mbox{\sc cat} & \mbox{\rm NP} \\
        \mbox{\sc case} & \mbox{\it dat\/} \\
        \mbox{\sc subcat} &
            \left[
            \begin{array}{l@{\; :\;}l}
                \mbox{\sc cat} & \mbox{\rm NP} \\
            \mbox{\sc case} & \mbox{\it acc\/}
            \end{array}\right]
        \end{array}\right]
    \end{array}\right]
\end{array}\right]
\end{equation}
%%%%
The rules of combination for category symbols have to be adapted
accordingly. This requires some effort but is possible without
problems. HPSG essentially follows this line, however pushing the
use of AVSs to the limit. Not only the categories, also the entire
geometrical structure is now coded using AVSs. HPSG also uses
structure variables. This is necessary in particular for the
semantics, which HPSG treats in the same way as syntax. (In this
it differs from GPSG. The latter uses a Montagovian approach,
pairing syntactic rules with semantical rules. In HPSG --- and 
LFG for that matter ---, the semantics is coded up like syntax.) 
Parallel to the development of GPSG and related frameworks, the 
so called {\it constraint based approaches\/} to natural language 
processing were introduced.  \cite{shieber:constraint} provides 
a good reference.
%%
\begin{defn}
\index{$s \doteq t$, $s \uparrow$}%%%
\index{constraint}%%%
\index{constraint!basic}%%%
%%%
A \textbf{basic constraint language} is a finite set $F$ of unary 
function symbols. A \textbf{constraint} is either an equation 
`$s(x) \boldsymbol{\doteq} t(x)$' or a statement `$s(x) \uparrow$'. 
A \textbf{constraint model} is a partial algebra 
$\GA = \auf A, \Pi\zu$ for the signature.
We write $\GA \vDash s \boldsymbol{\doteq} t$ iff for every $a$,
$s^{\GA}(a)$ and $t^{\GA}(a)$ are defined and equal. $\GA \vDash
s(x)\boldsymbol{\uparrow}$ iff for every $a$, $s^{\GA}(a)$ is defined.
\end{defn}
%%%
Often, one has a particular constant $\uli{o}$, which serves as
the root, and one considers equations of the form $s(\uli{o})
\boldsymbol{\doteq} t(\uli{o})$. Whereas the latter type of equation 
holds only at the root, the above type of equations are required to 
hold {\it globally}. We shall deal only with globally valid equations.
Notice that we can encode atomic values into this language by
interpreting an atom $p$ as a unary function $f_p$ with the idea
being that $f_p$ is defined at $b$ iff $p$ holds of
$b$. (Of course, we wish to have $f_p(f_p(b)) = f_p(b)$ if the
latter is defined, but we need not require that.) We give a
straightforward interpretation of this language into modal logic.
For each $f \in F$, take a modality, which we call by the same
name. Every $f$ satisfies $\auf f\zu p \und \auf f\zu  q . \pf.
\auf f\zu (p \und q)$. (This logic is known as $\mathsf{K.alt_1}$, 
see \cite{kracht:av}.) With each term $t$ we associate a modality in 
the obvious way: $f^{\mu} := f$, $(f(t))^{\mu} := f; t^{\mu}$. Now, 
the formula $s \boldsymbol{\doteq} t$ is translated by
%%
\begin{equation}
\begin{split}
(s^{\mu} \boldsymbol{\doteq} t^{\mu}) & 
	:= \auf s^{\mu}\zu p \dpf \auf t^{\mu}\zu p \\
(s \boldsymbol{\uparrow})^{\mu} & := \auf s^{\mu}\zu \top
\end{split}
\end{equation}
%%
Finally, given $\GA = \auf A, \Pi\zu$ we define a Kripke--frame
$\GA^{\mu} := \auf A, R\zu$ with $x\; R(f)\; y$ iff
$f(x)$ is defined and equals $y$. Then
%%
\begin{equation}
\GA \vDash s \boldsymbol{\doteq} t
\quad\Dpf\quad
\GA^{\mu} \vDash (s \boldsymbol{\doteq} t)^{\mu}
\end{equation}
%%
Further,
%%
\begin{equation}
\GA \vDash (s \boldsymbol{\uparrow})
\quad\Dpf\quad
\GA^{\mu} \vDash (s \boldsymbol{\uparrow})^{\mu}
\end{equation}
%%
Now, this language of constraints has been extended in various
ways. The attribute--value structures of Section~\ref{kap5}.\ref{kap5-1}
effectively extend this language by boolean connectives. $[C \; :
\; A]$ is a shorthand for $\auf C\zu A^{\mu}$, where $A^{\mu}$ is
the modal formula associated with $A$. Moreover, following the
discussion of Section~\ref{kap5}.\ref{kap5-4} we use $\links$, $\rechts$,
$\oben$ and $\unten$ to steer around in the phrase structure
skeleton. HPSG uses a different encoding. It assumes an attribute
called {\sc daughters}, whose value is a list. A list in turn is
an AVS which is built recursively using the predicates {\sc first}
and {\sc rest}. (The reader may write down the path language for
lists.) The notions of a Kripke--frame and a generalized
Kripke--frame are then defined as usual. The
Kripke--frames take the role of the actual syntactic objects,
while the AVSs are simply formulae to talk about them.

The logic $L_0$ is of course not very interesting. What we want to
have is a theory of the existing objects, not just all conceivable
ones. A particular concern in syntactic theory is therefore the
formulation of an adequate theory of the linguistic objects, be
it a universal theory of all linguistic objects, or be it a theory of
the linguistic objects of a particular language. We may cast this
in logical terms in the following way. We start with a set (or
class) $\CK$ of Kripke--frames. The theory of that class is
$\mathsf{Th}\, \CK$. It would be most preferrable if for any
given Kripke--frame  $\GF$ we had $\GF \vDash \mathsf{Th}\, \CK$
iff $\GF \in \CK$. Unfortunately, this is not always
the case. We shall see, however, that the situation is as good as
one can hope for. Notice the implications of the setup. Given,
say, the admissible structures of English, we get a modal logic
$L_M(\mbox{\rm Eng})$, which is an extension of $L_0$. Moreover,
if $L_M(\mbox{\rm Univ})$ is the modal logic of all existing
linguistic objects, then $L_M(\mbox{\rm Eng})$ furthermore is an
axiomatic extension of $L_M(\mbox{\rm Univ})$. There are sets
$\Gamma$ and $E$ of formulae such that
%%
\begin{equation}
L_0 \subseteq L_M(\mbox{\rm Univ}) = L_0 \oplus \Gamma
\subseteq L_M(\mbox{\rm Eng}) = L_0 \oplus \Gamma \oplus E
\end{equation}
%%
If we want to know, for example, whether a particular formula
$\varphi$ is satisfiable in a structure of English it is not
enough to test it against the postulates of the logic $L_0$, nor
those of $L_M(\mbox{\rm Univ})$. Rather, we must show that it
is consistent with $L_M(\mbox{\rm Eng})$. These problems can have
very different complexity. While $L_0$ is decidable, this need not
be the case for $L_M(\mbox{\rm Eng})$ nor for $L_M(\mbox{\rm
Univ})$. The reason is that in order to know whether there is a
structure for a logic that satisfies the axioms we must first
guess that structure before we can check the axioms on it. If we
have no indication of its size, this can turn out to be
impossible. The exercises shall provide some examples. Another way
to see that there is a problem is this. $\varphi$ is a theorem of
$L_M(\mbox{\rm Eng})$ if it can be derived from $L_0 \cup \Gamma
\cup E$ using modus ponens (MP), substitution and (MN). However,
$\Gamma \cup E \Vdash_{L_0} \varphi$ iff $\varphi$ can
be derived from $L_0 \cup\Gamma \cup E$ using (MP) and (MN) alone.
Substitution, however, is very powerful. Here we shall be
concerned with the difference in expressive power of the basic
constraint language and the modal logic. The basic constraint
language allows to express that two terms (called paths for
obvious reasons) are identical. There are two ways in which such
an identity can be enforced. (a) By an axiom: then this axiom must 
hold of all structures under consideration. An example is provided 
by the agreement rules of a language. (b) As a datum: then we are 
asked to satisfy the equation in a particular structure. In modal 
logic, only equations as axioms are expressible. Except for trivial 
cases there is no formula $\varphi(s,t)$ in polymodal 
$\mathsf{K.alt_1}$ such that
%%
\begin{equation}
\auf \GA^{\mu}, \beta, x\zu \vDash \varphi \Dpf
    s^{\GA}(x) = t^{\GA}(x)
\end{equation}
%%
Hence, modal logic is expressibly weaker than predicate logic,
in which such a condition is easily written down. Yet, it is not
clear that such conditions are at all needed in natural language.
All that is needed is to be able to state conditions of that
kind on all structures --- which we can in fact do. (See
\cite{kracht:av} for an extensive discussion.)

HPSG also uses {\it types}. Types are properties of nodes. As
such, they can be modelled by unary predicates in $\mathsf{MSO}$, or by
boolean constants in modal logic. For example, we have represented
the atomic values by proposition constants.  In GPSG, the atomic
values were assigned only to Type 0 features. HPSG goes further
than that by typing AVSs. Since the AVS is interpreted in a
Kripke--frame, this creates no additional difficulty. Reentrancy
is modelled by path equations in constraint languages, and can 
be naturally expressed using modal languages, as we have
seen.  As an example, we consider the agreement rule \eqref{eq:61ddd} 
again.
%%
\begin{equation}
\left[\begin{array}{l@{\; : \;}l}
\mbox{\sc cat} & \mbox{\it s}
\end{array}\right]
\pf
\left[\begin{array}{l@{\; : \;}l}
\mbox{\sc cat} & \mbox{\it np} \\
\mbox{\sc agr} & \framebox{1}
\end{array}\right]
\qquad
\left[\begin{array}{l@{\; : \;}l}
\mbox{\sc cat} & \mbox{\it vp} \\
\mbox{\sc agrs} & \framebox{1} \\
\end{array}\right]
\end{equation}
%%
In the earlier \cite{pollardsag:hpsg1} the idea of reentrancy was
motivated by {\it information sharing}. What the label
$\framebox{1}$ says is that any information available under that
node in one occurrence is available at any other occurrence. One
way to make this true is to simply say that the two occurrences of
$\framebox{1}$ are not distinct in the structure. (An analogy
might help here. In the notation $\{a, \{a,b\}\}$ the two
occurrences of $a$ do not stand for different things of the
universe: they both denote $a$, just that the linear notation
forces us to write it down twice.) There is a way to enforce this
in modal logic. Consider the following formula.
%%
\begin{equation}
\begin{array}{l}
\auf \mbox{\sc cat}\zu \mathsf{s} \und
    \unten(\links \bot \und \auf \mbox{\sc cat}\zu
    \mathsf{np} \und \rechts (\auf \mbox{\sc cat}\zu\mathsf{vp}
    \und \rechts \bot)) \\
\qquad \pf \unten(\links \bot \und \auf\mbox{\sc agr}\zu p \dpf
    \rechts \auf \mbox{\sc agrs}\zu p)
\end{array}
\end{equation}
%%
This formula says that if we have an S which consists of an NP and
a VP, then whatever is the value of {\sc agr} of the NP also is
the value of {\sc agrs} of the VP.

The constituency structure that the rules specify can be written
down using quantified modal logic. As an exercise further down
shows, $\mathsf{QML}$ is so powerful that first--order $\mathsf{ZFC}$ can 
be encoded. (See Section~\ref{kap1}.\ref{kap1-1} for the definition 
of $\mathsf{ZFC}$.) 
In $\mathsf{MSO}(\boldsymbol{\in}, \boldsymbol{=})$ one can 
write down an axiom that forces sets to be well--founded with 
respect to $\in$ and even write down the axioms of $\mathsf{NBG}$ 
(von Neumann--G\"odel--Bernays Set Theory), which differs from 
$\mathsf{ZFC}$ in having a simpler scheme for set comprehension. In 
its place we have this axiom.
%%
\begin{quote}
{\sl Class Comprehension.} $(\forall P)(\forall x)(\exists y)(%
    \forall z)(z \in y \dpf z \in x \und P(z))$.
\end{quote}
%%
It says that from a set $x$ and an arbitrary subset of the
universe $P$ (which does not have to be a set)  there is a {\it
set\/} of all things that belong to both $x$ and $P$. In presence
of the results by Thatcher, Doner and Wright all this may sound
paradoxical. However, the introduction of structure variables has
made the structures into acyclic graphs rather than trees.
However, our reformulation of HPSG is not expressed in $\mathsf{QML}$
but in the much weaker polymodal logic. Thus, theories of
linguistic objects are extensions of polymodal $\mathsf{K}$. However,
as \cite{kracht:kuznetsov} 
%%%
\index{Kracht, Marcus}%%%
%%%
shows, by introducing enough
modalities one can axiomatize a logic such that a Kripke--frame
$\auf F, R\zu$ is a frame for this logic iff $\auf F,
R(\boldsymbol{\in})\zu$ is a model of $\mathsf{NGB}$. This means that 
effectively any higher order logic can be encoded into HPSG notation, 
since it is reducible to set theory, and thereby to polymodal logic. 
Although this is not per se an argument against using the notation, it
shows that anything goes and that a claim to the effect that such
and such phenomenon can be accounted for in HPSG is empirically
vacuous.

{\it Notes on this section.} One of the seminal works in GPSG 
besides \cite{gazdarpullumsag:gpsg} is the study of word order 
in German 
%%%
\index{German}%%%
by Hans Uszkoreit~\shortcite{uszkoreit:german}. 
%%%
\index{Uszkoreit, Hans}%%%
%%%
The constituent structure of the continental Germanic languages has been 
a focus of considerable debate between the different grammatical frameworks.
The discovery of Swiss German 
%%%
\index{Swiss German}%%%
%%%
actually put an end to the debate
whether or not context free rules are appropriate. In GSPG it is
assumed that the dominance and the precedence relations are
specified separately. Rules contain a dominance skeleton and a
specification that says which of the orderings is admissible.
However, as Almerindo Ojeda~\shortcite{ojeda:precedence} has shown,
GPSG can also generate cross serial dependencies of the Swiss
German type. One only has to relax the requirement that the
daughters of a node must be linearly ordered to a requirement that
the yield of the tree must be so ordered.
%%%
\vplatz 
\exercise 
Show that all axioms of $\mathsf{ZFC}$ and also {\sl Class Comprehension} 
are expressible in $\mathsf{MSO}(\boldsymbol{\in}, \boldsymbol{=})$.
%%
\vplatz 
\exercise 
Show that the logic $L_0$ of any number of basic
modal operators satisfying $\wD p \und \wD q \pf \wD (p \und q)$
is decidable. This shows the decidability of $L_0$. {\it Hint.}
Show that any formula is equivalent to a disjunction of
conjunctions of statements of the form $\auf \delta\zu\pi$, where
$\delta$ is a sequence of modalities and $\pi$ is either nonmodal
or of the form $[m]\bot$.
%%%
\vplatz 
\exercise 
Write a grammar using LFG--rules of the kind described above to 
generate the crossing dependencies of Swiss German.
%%
\vplatz 
\exercise 
Let $A$ be an alphabet, $T$ a Turing machine over $A$. The computation 
of $T$ can be coded onto a grid of numbers $\BZ \times \BN$. Take this 
grid to be a Kripke--structure, with basic relations the immediate 
horizontal successor and predecessor, the transitive closure of these
relations, and the vertical successor. Take constants $c_a$ for
every $a \in A \cup Q \cup \{\nabla\}$. $c_{\nabla}$ codes the
position of the read write head. Now formulate an axiom
$\varphi_T$ such that a Kripke--structure satisfies $\varphi_T$ 
iff it represents a computation of $T$.

 \section{Formal Structures of GB}
\label{kap5-7}
%
%
%
We shall close this chapter with a survey of the basic mathematical
constructs of GB. The first complex concerns constraints on syntactic 
structures. GB has many types of such constraints. It has for example 
many principles that describe the geometrical configuration within 
which an element can operate. A central definition is that of 
idc--command, often referred to as {\it c--command}, although the 
latter was originally defined differently.
%%%%
\begin{defn}
%%%%
\index{idc--command}%%
\index{c--command}%%
%%%%
Let $\GT = \auf T, <\zu$ be a tree, $x, y \in T$. $x$ 
\textbf{idc--commands} $y$ if for every $z > x$ we have $z \geq y$. 
A constituent $\low{x}$ \textbf{idc--commands} a constituent $\low{y}$ 
if $x$ idc--commands $y$.
\end{defn}
%%%%
In \shortcite{koster:domains}, Jan Koster has proposed an attempt to
formulate GB without the use of movement transformations. The basic
idea was that the traces in the surface structure leave enough
indication of the deep structure that we can replace talk of deep
structure and derivations by talk about the surface structure
alone. The general principle that Koster proposed was as follows.
Let $x$ be a node with label $\delta$, and let $\delta$
be a so--called dependent element. (Dependency is 
defined with reference to the category.) Then there must
exist a uniquely defined node $y$ with label $\alpha$
which c--commands $x$, and is local to $x$. Koster required
in addition that $\alpha$ and $\delta$ shared a property.
However, in formulating this condition it turns out to be
easier to constrain the possible choices of $\delta$ and $\alpha$.
In addition to the parameters $\alpha$ and $\delta$ it
remains to say what locality is. Anticipating our definitions
somewhat we shall say that we have $x\; R\; y$ for a certain
relation $R$. \cite{barkerpullum:command} have surveyed the notions of
locality that enter in the definition of $R$ that were used in
the literature and given a definition of command relation.
Using this, \cite{kracht:aspects} developed a theory
of command relations that we shall outline here.
%%%%
\begin{defn}
Let $\auf T,<\zu$ be a tree and $R \subseteq T^2$ a
relation. $R$ is called a \textbf{command relation} 
%%%
\index{command relation}%%
%%%
if there is a function $f_R \colon T \pf T$ such that (1) -- (3) hold. 
$R$ is a \textbf{monotone command relation} 
%%%
\index{command relation!monotone}%%%
%%%
if in addition it satisfies (4), 
and \textbf{tight} if it satisfies (1) -- (5).
%%%
\index{command relation!tight}%%
%%%%
\begin{dingautolist}{192}
\item $R_x := \{y : x\; R\; y\} = \low{f_R(x)}$.
\item $x < f_R(x)$ for all $ x < r$.
\item $f_R(r) = r$.
\item If $x \leq y$ then $f_R(x) \leq f_R(y)$.
\item If $x < f_R(y)$ then $f_R(x) \leq f_R(y)$.
\end{dingautolist}
%%%%%
\end{defn}
%%%%
The first class that we shall study is the class of tight
command relations. Let $\GT$ be a tree and $P \subseteq T$.
We say,  $x$ $P$--\textbf{commands} $y$ if for every $z > x$ with
$z \in P$ we have $z \geq y$.  We denote the relation of
$P$--command by $K(P)$. If we choose $P = T$ we exactly get
idc--command. The following theorem is left as an exercise.
%%%%
\begin{prop}
\label{prop:dicht}
Let $R$ be a binary relation on the tree $\auf T, <\zu$.
$R$ is a tight command relation iff $R = K(P)$ for some 
$P \subseteq T$.
\end{prop}
%%%
Let $\GT$ be a tree. We denote by $\mbox{\rm MCr}(\GT)$ the set of
monotone command relations on $\GT$. This set is closed under
intersection, union and relation composition. We even have
%%%
\begin{equation}
\begin{split}
f_{R \cup S}(x) & = \max \{f_R(x), f_S(x)\} \\
f_{R \cap S}(x) & = \min \{f_R(x), f_S(x)\} \\
f_{R \circ S}(x) & = (f_S \circ f_R)(x)
\end{split}
\end{equation}
%%%%
For union and intersection this holds without assuming monotonicity. 
For relation composition, however, it is needed. For suppose 
$x\; R \circ S\; y$. Then we can conclude that $x\; R\; f_R(x)$
and $f_R(x)\; S\; y$. Hence $x\; R \circ S\; y$ iff
$y \leq f_S(f_R(x))$, from which the claim now follows.
Now we set
%%%
\begin{equation}
%%%
\index{$\goth{MCr}(\GT)$}%%%
%%%
\goth{MCr}(\GT) := \auf \mbox{\rm MCr}(\GT), \cap, \cup, \circ\zu
\end{equation}
%%%
$\goth{MCr}(\GT)$ is a  distributive lattice with respect to
$\cap$ and $\cup$. What is more, there are additional laws
of distribution concerning relation composition.
%%%%
\begin{prop}
\label{prop:distributoid}
Let $R, S, T \in \mbox{\rm MCr}(\GT)$. Then
%%%%
\begin{dingautolist}{192}
\item
$R \circ (S \cap T) = (R \circ S) \cap (R \circ T)$, \\
$(S \cap T) \circ R = (S \circ R) \cap (T \circ R)$.
\item
$R \circ (S \cup T) = (R \circ S) \cup (R \circ T)$, \\
$(S \cup T) \circ R = (S \circ R) \cup (T \circ R)$.
\end{dingautolist}
%%
\end{prop}
%%%%
\proofbeg
%%
Let $x$ be an element of the tree. Then
%%
\begin{equation}
\begin{split}
f_{R \circ (S \cap T)}(x) & = f_{S \cap T} \circ f_R(x) \\
 & = \min \{f_S(f_R(x)), f_T(f_R(x))\} \\
 & = \min \{f_{R \circ S}(x), f_{R \circ T}(x)\} \\
 & = f_{(R \circ S) \cap (R \circ T)}(x)
\end{split}
\end{equation}
%%
The other claims can be shown analogously.
\proofend
%%%
\begin{defn}
%%%%
\index{command relation!generated}
\index{command relation!chain like}%%
%%%%
Let $\GT$ be a tree, $R \in \mbox{\rm MCr}(\GT)$.
$R$ is called \textbf{generated} if it can be produced from tight
command relations by means of $\cap$, $\cup$ and $\circ$. $R$ is
called \textbf{chain like} if it can be generated from tight
relations with $\circ$ alone.
\end{defn}
%%%%
\begin{thm}
$R$ is generated iff $R$ is an intersection of chain
line command relations.
\end{thm}
%%%%
\proofbeg
Because of Proposition~\ref{prop:distributoid} we can move
$\circ$ to the inside of $\cap$ and $\cup$. Furthermore, we can
move $\cap$ outside of the scope of $\cup$. It remains to be
shown that the union of two chain like command
relations is an intersection of chain like command relations.
This follows from Lemma~\ref{lem:spleiss}.
\proofend
%%%
\begin{lem}
Let $R = K(P)$ and $S = K(Q)$ be tight. Then
%%
\begin{equation}
R \cup S = (R \circ S) \cap (S \circ R) \cap K(P \cap Q)
\end{equation}
%%
\end{lem}
%%%%
\proofbeg
Let $x$ be given. We look at $f_R(x) $ and $f_S(x)$.
Case 1. $f_R(x) < f_S(x)$. Then $f_{R \cup S}(x) = f_S(x)$.
On the right hand side we have $f_S \circ f_R(x) = f_S(x)$,
since $S$ is tight. $f_R \circ f_S(x) \geq f_S(x)$, as well as
$f_{K(P \cap Q)}(x) \geq f_S(x)$. Case 2.  $f_S(x) < f_R(x)$.
Analogously. Case 3. $f_S(x) = f_R(x)$. Then $f_{S \cup R}(x) %
= f_R(x) = f_S(x)$, whence $f_R \circ f_S(x), f_S \circ f_R(x) %
\geq f_{S \cup R}(x)$. The smallest node above $x$ which is
both in $P$ and in $Q$ is clearly in $f_S(x)$. Hence we have
$f_{K(P\cap Q)}(x) = f_S(x)$. Hence equality holds in all
cases.
\proofend

We put
%%%
\index{$K(P) \bullet K(Q)$}%%
%%%
\begin{equation}
K(P) \bullet K(Q) := K(P \cap P)
\end{equation}
%%
The operation $\bullet$ is defined only on tight command relations.
If $\auf R_i : i < m\zu$ is a sequence of command relations,
then $R_0 \circ R_1 \circ \dotsb \circ R_{n-1}$ is called its
%%%
\index{command relation!product of {\faul}s}%%
%%%
\textbf{product}. In what is to follow we shall characterize a
union of chain like relations as the intersection of products.
To this end we need some definitions. The first is that of
a \textbf{shuffling}. This operation mixes two sequences in such a
way that the liner order inside the sequences is respected.
%%%
\begin{defn}
%%%%
\index{shuffling}%%
\index{embedding}%%
%%%%
Let $\rho = \auf a_i : i < m\zu$ and $\sigma =
\auf b_j : j < n\zu$ be sequences of objects.
A \textbf{shuffling} of $\rho$ and $\sigma$ is a sequence
$\auf c_k : k < m+n\zu$ such that there are injective
monotone functions $f \colon n \pf m+n$ and $g \colon m \pf m+n$
such that $\im(f) \cap \im(g) = \varnothing$ and $\im(f) \cup \im(g) 
= m+n$, as well as $c_{f(i)} = a_i$ for all $i < m$ and $c_{g(j)} = b_j$
for all $j < n$. $f$ and $g$ are called the \textbf{embeddings}
of the shuffling.
\end{defn}
%%%%
\begin{defn}
%%%%
\index{command relation!weakly associated}%%
%%%%
Let $\rho = \auf R_i : i < m\zu$ and $\sigma =
\auf S_j : j < n\zu$ be sequences of tight command relations.
Then $T$ is called \textbf{weakly associated with} $\rho$ and
$\sigma$ if there is a shuffling $\tau = \auf T_i : i < m+n\zu$
of $\rho$ and $\sigma$ together with embeddings $f$ and $g$
such that
%%
\begin{equation}
T = T_0 \circ^0 T_1 \circ^1 T_2 \dotsb \circ^{n-2} T_{n-1}
\end{equation}
%%
where $\circ^i \in \{\circ, \bullet\}$ for $i < n-1$
and $\circ^i = \circ$ always if $\{i, i+1\} \subseteq
\im(f)$ or $\{i,i+1\} \subseteq \im(g)$.
\end{defn}
%%%
If $m = n = 2$, we have the following shufflings.
%%%
\begin{equation}
\begin{array}{l}
\auf R_0, R_1, S_0, S_1\zu, \quad \auf R_0, S_0, R_1, S_1\zu, \quad
    \auf R_0, S_0, S_1, R_1\zu,  \\
\auf S_0, R_0, R_1, S_1\zu, \quad \auf S_0, R_0, S_1, R_1\zu, \quad
    \auf S_0, S_1, R_0, R_1\zu 
\end{array}
\end{equation}
%%%
The sequence $\auf R_1, S_0, S_1, R_0\zu$ is not a shuffling because
the order of the  $R_i$ is not respected. In general there exist
up to ${m+n \choose n}$ different shufflings. For every shuffling
there are up to $2^{n-1}$ weakly associated command relations
(if $n \leq m$).  For example the following command relations
are weakly associated to the third shuffling.
%%
\begin{equation}
R_0 \bullet S_0 \circ S_1 \bullet S_1, \quad
R_0 \circ S_0 \circ S_1 \circ R_1
\end{equation}
%%%
The relation $R_0 \circ S_0 \bullet S_1 \circ R_1$ is however
not weakly associated to it since $\bullet$ may not occur in
between two $S$.
%%%
\begin{lem}
\label{lem:spleiss}
Let $\rho = \auf R_i : i < m\zu$ and $\sigma =
\auf S_i : i < n\zu$  be sequences of tight command relations
with product $T$ and $U$, respectively. Then $T \cup U$ is the
intersection of all chain like command relations
which are products of sequences weakly associated with
a shuffling of $\rho$ and $\sigma$.
\end{lem}

In practice one has restricted attention to command relations
which are characterized by certain sets of nodes, such as the set
of all maximal projections, the set of all finite sentences, the
set of all sentences in the indicative mood and so on. If we
choose $P$ to be the set of nodes carrying a label subsuming the
category of finite sentences, then we get the following: if $x$ is
a reflexive anaphor, it has to be c--commanded by a subject, which
it in turn $P$--commands. (The last condition makes sure that the
subject is a subject of the same sentence.) There is a plethora of
similar examples where command relations play a role in defining
the range of phenomena. Here, one took not just any old set of
nodes but those that where {\it definable}. To precisify this, let
$\auf \GT, \ell\zu$ with $\ell \colon T \pf N$ be a labelled tree and
$Q \subseteq N$. Then $K(Q) := K(\ell^{-1}(Q))$ is called a
%%%%
\index{command relation!definable}%%
%%%%%
\textbf{definable tight command relation}.
%%%%
\begin{defn}
Let $\GT$ be a tree and $R \subseteq T\times T$. $P$ is
called a (\textbf{definable}) \textbf{command relation} if it
can be obtained from definable tight command relations
by means of composition, union and intersection.
\end{defn}
%%%
In follows from the previous considerations that the union
of definable relations is an intersection of chains of
tight relations. A particular role is played by subjacency.
%%%
\index{subjacency}%%
%%%
The antecedent of a trace must be 1--subjacent to a trace.
As  is argued in \cite{kracht:adjunction} on the basis of
\cite{chomsky:barriers} this relation is exactly
%%%
\begin{equation}
K(\mbox{\sf ip}) \circ K(\mbox{\sf cp}) 
\end{equation}
%%

The movement and copy--transformations create so--called {\it chains}.
Chains connect elements in different positions with each other. The
mechanism inside the grammar is coindexation. For as we have said
in Section~\ref{kap5}.\ref{kap5-5} traces must be properly governed, and
this means that an antecendent must c--com\-mand its trace in addition to
being coindexed with it. This is a restriction on the structures
as well as on the movement transformations. Using coindexation
one also has the option of associating antecedent and trace
without assuming that anything has ever moved. The transformational
history can anyway be projected form the S--structure up to minor
(in fact inessential) variations. This means that we need not
care whether the S--structure has been obtained by transformations
or by some other process introducing the indexation (this is what
Koster has argued for). The association between antecedent and
trace can also be done in a different way, namely by collecting
sets of constituents. We call a chain a certain set of constituents.
In a chain the members may be thought to be coindexed, but this is 
not necessary. Chomsky 
%%%
\index{Chomsky, Noam}%%%
%%%
has once again
introduced the idea in the 1990s that movement is the sequence of
copying and deletion and made this one of the main innovations of
the reform in the Minimalist Program (see \cite{chomsky:minimalist}).
Deletion here is simply marking as phonetically empty (so the
copy remains but is marked). However, the same idea can be
introduced into GB without substantial change. Let us do this
here and introduce in place of Move--$\alpha$ the transformation
%%%%
\index{Copy--$\alpha$}%%%
%%%%
\textbf{Copy}--$\alpha$. It will turn out that it is actually not
necessary to say which of the members of the chain has been obtained
by copying from which other member. The reason is simple:
the copy (= antecedent) c--commands the original (= trace) but
the latter does not c--command the former. Knowing who is in a 
chain with whom is therefore enough. This is the central insight
that is used in the theory of chains in \cite{kracht:chains}
which we shall now outline.  We shall see below that copying
gives more information on the derivation than movement, so that
we must be careful in saying that nothing has changed by introducing
copy--movement.

Recall that constituents are subtrees. In what is to follow we
shall not distinguish between a set of nodes and the constituent
%%%
\index{ac--command}%%
%%%
that is based on that set. Say that $x$ \textbf{ac--commands} 
$y$ if $x$  and $y$ are incomparable, $x$ idc--commands $y$ but $y$ 
does not idc--command $x$.
%%%%
\begin{defn}
%%%%
\index{chain}%%
\index{chain!trace \faul}%%
\index{chain!copy \faul}%%
\index{chain!head \faul}%%
\index{chain!foot \faul}%%
%%%
Let $\GT$ be a tree. A set $\Delta$ of constituents of $\GT$ which 
is linearly ordered with respect to ac--command is called a 
\textbf{chain in} $\GT$. The element which is highest with respect 
to ac--command is called the \textbf{head of} $\Delta$, the lowest 
the \textbf{foot}. $\Delta$ is a \textbf{copy chain} if any two 
members are isomorphic. $\Delta$ is a \textbf{trace chain} if all 
non heads are traces.
\end{defn}
%%%
The definition of chains can be supplemented with more detail
in the case of copy chains. This will be needed in the sequel.
%%%
\begin{defn}
%%%%
\index{copy chain$^{\ast}$}%%
\index{chain!associated}
%%%%
Let $\GT$ be a tree. A \textbf{copy chain}$^{\ast}$ \textbf{in} $\GT$
is a pair $\auf \Delta, \Phi\zu$ for which the following holds.
%%
\begin{dingautolist}{192}
\item
$\Delta$ is a chain.
\item
$\Phi = \{\phi_{\GC,\GD} : \GC, \GD \in \Delta\}$ is a family of
isomorphisms such that for all $\GC, \GD, \GA \in \Delta$ we have
\begin{enumerate}
\item $\phi_{\GC,\GC} = 1_{\GC}$
\item $\phi_{\GC,\GA} = \phi_{\GC,\GD} \circ \phi_{\GD,\GA}$
\end{enumerate}
\end{dingautolist}
%%%
The \textbf{chain associated with} $\auf \Delta, \Phi\zu$ is $\Delta$.
\end{defn}
%%
Often we shall identify a chain$^{\ast}$ with its associated chain.
The isomorphisms give explicit information which elements
of the various constituents are counterparts of which others.
%%%%
\begin{defn}
Let $\GT$ be a tree and $\CD = \auf \Delta, \Phi\zu$ a copy
%%%%
\index{$x \approx_{\CD} y$, $[x]_{\CD}$}%%%
%%%%
chain$^{\ast}$. Then we put $x \approx_{\CD} y$ if there is a map
$\varphi \in \Phi$ such that $\phi(x) = y$.  We put
$[x]_{\CD} := \{y : x \approx_{\CD} y\}$. If $C$ is a set
of copy chains$^{\ast}$ then let $\approx_C$ be the smallest equivalence
relation generated by all $\approx_{\CD}$, $\CD \in C$.
Further, let $[x]_C := \{y : x \approx_C y\}$.
\end{defn}
%%
%%%%
\begin{defn}
%%%%
\index{link}%%
\index{link map}%%
\index{map!ascending}%%
%%%%
Let $\auf \Delta, \Phi\zu$ be a copy chain$^{\ast}$, $\GC, \GD \in \Delta$.
$\GC$ is said to be \textbf{immediately above} $\GD$ if there is no
$\GE \in \Delta$ distinct from $\GC$ and $\GD$ which ac--commands
$\GD$ and is ac--commanded by $\GC$. A \textbf{link of} $\Delta$ is
a triple $\auf \GC, \phi_{\GD,\GC}, \GD\zu$ where $\GC$ is immediately
above $\GD$. $\phi$ is called a \textbf{link map} if it occurs in a
link. An \textbf{ascending map} is a composition of link maps.
\end{defn}
%%%%
\begin{lem}
Let $\phi$ be a link map. Then $t(\phi(x)) < t(x)$.
\end{lem}
%%%%
\proofbeg
Let $\phi = \phi_{\GC,\GD}$, $\GC = \low{v}$, $\GD = \low{w}$.
Further, let $t_{\GC}(x)$ be the depth of $x$ in $\GC$,
$t_{\GD}(\phi(x))$ the depth of $\phi(x)$ in $\GD$. Then
$t_{\GC}(x) = t_{\GD}(\phi(x))$, since $\phi$ is an isomorphism.
On the other hand $t(x) = t(v) + t_{\GC}(x)$ and $t(\phi(x)) = t(w) +
t_{\GD}(\phi(x))  = t(w) + t_{\GC}(x)$. The claim now follows
from the next lemma given the remark that $v$ c--commands $w$,
but $w$ does not c--command $v$.
\proofend
%%%%
\begin{lem}
Let $\GT = \auf T, <\zu$ be a tree, $x, y \in T$.
If $x$ ac--commands $y$, $t(x) \leq t(y)$.
\end{lem}
%%%%
\proofbeg
There exists a uniquely defined $z$ with $z \succ x$. By
definition of c--command we have $z \geq y$. But $y \neq z$,
since $y$ is not comparable with $x$. Hence $y < z$. Now we
have $t(x) = t(z) + 1$ and $t(y) > t(z) = t(x) - 1$. Whence
the claim.
\proofend

%%%%
We call a pair $\auf \GT, C\zu$ a \textbf{copy chain tree}
%%%
\index{copy chain tree}%%
\index{CCT (see copy chain tree)}%%%
%%%%
(\textbf{CCT}) if $C$ is a set of copy chains$^{\ast}$ on $\GT$,
$\GT$ a finite tree. We consider among other the following
constraints.
%%%%
\begin{quote}
%%%%
\index{uniqueness}%%
%%%%
{\sl Uniqueness}.
Every constituent of $\GT$ is contained in exactly one chain.
\\
%%%%
\index{Liberation}%%
%%%%
{\sl Liberation}. Let $\Gamma, \Delta$ be chain, $\GC \in \Gamma$
    and $\GD_0, \GD_1 \in \Delta$ with $\GD_0 \neq \GD_1$ such
    that $\GD_0, \GD_1 \subseteq \GC$. Then $\GC$ is the foot
    of $\Gamma$.
\end{quote}
%%%
\begin{lem}
Let $K$ be a CCT which satisfies {\sl Uniqueness} and
{\sl Liberation}. Further, let $\phi$ and $\phi'$ be link maps with
$\im(\phi) \cap \im(\phi') \neq \varnothing$. Then already 
$\phi = \phi'$.
\end{lem}
%%%%
\proofbeg
Let $\phi \colon \GC \pf \GD$, $\phi' \colon \GC' \pf \GD'$ be link 
maps.  If $\im(\phi) \cap \im(\phi') \neq \varnothing$ then 
$\GD \subseteq \GD'$ or $\GD' \subseteq \GD$. Without loss of 
generality we may assume the first. If $\GD \subsetneq \GD'$ then 
also $\GC \subseteq \GD'$, since $\GD$ c--commands $\GC$. By 
{\sl Liberation\/} $\GD'$ is the foot of its chain, in contradiction 
to our assumption. Hence we have $\GD = \GD'$. By {\sl Uniqueness}, 
$\GC$, $\GC'$ and $\GD$ are therefore in the same chain. Since 
$\phi$ and $\phi'$ are link maps, we must have $\GC = \GC'$. 
Hence $\phi = \phi'$.
\proofend
%%%%
\begin{defn}
%%%
\index{root}%%
%%%
Let $K$ be a CCT. $x$ is called a \textbf{root} if $x$ is not in the
image of a link map.
\end{defn}
%%%%
Then proof of the following theorem is now easy to provide.
It is left for the reader.
%%%%
\begin{prop}
\label{prop:kandek}
Let $K$ be a CCT which satisfies {\sl Uniqueness} and {\sl Liberation}.
Let $x$ be an element and $\tau_i$, $i < m$, $\phi_j$, $j < n$,
link maps, and $y$, $z$ roots such that
%%
\begin{equation}
x = \tau_{m-1} \circ \tau_{m-2} \circ \dotsb \circ \tau_0(y) =
    \phi_{n-1} \circ \phi_{n-2} \circ \dotsb \circ \phi_0(z) 
\end{equation}
%%
Then we have $y = z$, $m = n$ and $\tau_i = \phi_i$ for all
$i < n$.
\end{prop}
%%%
Hence, for given $x$ there is a uniquely defined root $x_r$ with
$x \approx_C x_r$. Further, there exists a unique sequence 
$\auf \phi_i : i < n\zu$ of link maps such that $x$ is the image 
of $\phi_{n-1} \circ \dotsb \circ \phi_0$. This sequence we call 
the \textbf{canonical decomposition of} $x$.
%%%
\index{canonical decomposition}%%
%%%%
\begin{prop}
Let $K$ be a CCT satisfying {\sl Uniqueness} and {\sl Liberation}.
Then the following are equivalent.
%%%
\begin{dingautolist}{192}
\item
$x \approx_C y$.
\item
$x_r = y_r$.
\item
There exist two ascending maps $\chi$ and $\tau$
with $y = \tau \circ \chi^{-1}(x)$.
\end{dingautolist}
\end{prop}
%%%%
\proofbeg
\ding{192} $\Pf$ \ding{194}. Let $x \approx_C y$.
Then there exists a sequence $\auf \sigma_i : i < p\zu$ of link 
maps or inverses thereof such that 
$y = \sigma_{p-1} \circ \dotsb \circ \sigma_0(x)$.
Now if $\sigma_{i}$ is a link map and $\sigma_{i+1}$ an inverse
link map, then $\sigma_{i+1} = \sigma_{i}^{-1}$. Hence we may
assume that for some $q \leq p$ all $\sigma_i$, $i < q$,
are inverse link maps and all $\sigma_i$, $p > i \geq q$,
are link maps. Now put $\tau := \sigma_{p} \circ \sigma_{p-1}
\dotsb \circ \sigma_q$ and $\chi := \sigma_0 \circ \sigma_{1}
\circ \dotsb \circ \sigma_{q-1}$. $\chi$ and $\tau$ are ascending
maps. So, \ding{194} obtains. \ding{194} $\Pf$ \ding{193}. Let 
ascending maps $\chi$ and $\tau$ be given with 
$y = \tau \circ \chi^{-1}(x)$.
Put $u := \chi^{-1}(x)$. Then $u = \rho(u_r)$ for some
ascending map $\rho$. Further, $x = \chi(u) = \chi \circ %
\rho(u_r)$ and $y = \tau(u) = \tau \circ \rho(u_r)$.
Now, $u_r$ is a root and $x$ as well as $y$ are images of
$u_r$ under ascending maps. Hence $u_r$ is a root of $x$
and $y$. This however means that $u_r = x_r = y_r$. Hence,
\ding{193} obtains. \ding{193} $\Pf$ \ding{192} is straightforward.
\proofend

The proof also establishes the following fact.
%%%%
\begin{lem}
Every ascending map is a canonical decomposition.
Every composition of maps equals a product $\tau \circ \chi^{-1}$
where $\tau$ and $\chi$ are ascending maps.
A minimal composition of link maps and their inverses is
unique.
\end{lem}
%%%%

Let $x$ be an element and $\auf \phi_i : i < n\zu$ its canonical
decomposition. Then we call
%%
\begin{equation}
T_K(x) := \{\phi_{j-1} \circ \phi_{j-2}\circ\dotsb\circ\phi_0(x) :
    j \leq n\}
\end{equation}
%%
%%%%
\index{trajectory}%%
%%%%
the \textbf{trajectory of} $x$. The trajectory mirrors the history
of $x$ in the process of derivation. We call \textbf{root line}
of $x$ the set
%%%
\begin{equation}
W_K(x) := \{y : y \in T_K(x), y \text{ idc--commands }
    x_r\}
\end{equation}
%%%
Notice that $x_r$ idc--commands itself. The \textbf{peak of} $x$
is the element of $W_K(x)$ of smallest depth. We write $x_{\pi}$
%%%
\index{$\pi_x$, $x_{\pi}$, $\zeta_x$, $x_{\zeta}$}%%%
%%%%
for the peak of $x$ and $\pi_x$ for the ascending map which
sends $x$ to $x_{\pi}$.
%%%%
\begin{defn}
Let $K$ be a CCT satisfying {\sl Uniqueness} and {\sl Liberation}.
If $r$ is the root of the tree then $r$ is the \textbf{zenith of} $r$,
the \textbf{zenith map} is $\zeta_r := 1_T$. If $x \neq r$ then
the \textbf{zenith map} is the composition $\zeta_y \circ \pi_x$,
where $y \succ x_{\pi}$. The \textbf{zenith of} $x$ equals $\zeta_y \circ %
\pi_x(x)$. We write $x_{\zeta}$ for the zenith of $x$.
\end{defn}
%%%%
\begin{defn}
%%%
\index{link map!orbital}%%
%%%
A link map is called \textbf{orbital} if it occurs in a minimal
decomposition of the zenith map.
\end{defn}
%%%%%
At last we can formulate the following restriction on CCTs.
%%%%
\begin{quote}
%%%%
\index{No Recycling}%%
%%%%
{\sl No Recycling}. All link maps are orbital.
\end{quote}

The effect of a copy transformation is that (1) it adds a new
constituent and (2) this constituent is added to an already
existing chain as a head. Hence the whole derivation can be
thought of as a process which generates a tree together with
its chains. These can be explicitly described and this eliminates
the necessity of talking about transformations.
%%%
\begin{defn}
A \textbf{copy chain structure} (\textbf{CCS}) is a CCT
%%%%%
\index{copy chain structure}
\index{CCS (see copy chain structure)}%%%
%%%%%
$K = \auf \GT, C\zu$ which satisfies {\sl Uniqueness},
{\sl Liberation} and {\sl No Recycling}.
\end{defn}
%%%
Everything that one wants to say about transformations and
derivations can be said also about copy chain structures.
The reason for this is the following fact. We call a CCT
%%%
\index{tree}%%
%%%
simply a \textbf{tree} if every chain consists of a single
constituent. Then also this tree is a CCS.
A transformation can naturally be defined as an operation
between CCSs. It turns out that Copy--$\alpha$
turns a CCS into a CCS. The reason for this is that traces
have to be bound and may not be moved.  (Only in order to
reflect this in the definition of the CCSs the condition
{\sl No Recycling\/} has been introduced. Otherwise it was
unnecessary.) The following now holds.
%%%
\begin{thm}
A CCT is a CCS iff it is obtained from a tree by
successive application of Copy--$\alpha$.
\end{thm}
%%%
%%%
Transformational grammar and HPSG are not as different as one
might think. The appearance to the contrary is created by the fact
that TG is written up using trees, while HPSG has acyclic
structures, which need not be trees. In this section we shall show
that GB actually defines structures that are more similar to
acyclic graphs than to trees. The basis for the alternative
formulation is the idea that instead of movement transformations
we define an operation that changes the dominance relation. If the
daughter constituent $z$ of $x$ moves and becomes a daughter
constituent of $y$ then we can simply add to the dominance
relation the pair $\auf z,y\zu$. This rather simple idea has to be
worked out carefully. For first we have to change from using the
usual transitive dominance relation the immediate dominance
relation. Second one has to take care of the linear order of the
elements at the surface since it is now not any more represented.
%%
\begin{defn}
A \textbf{multidominance structure} (\textbf{MDS}) is a triple 
%%%%
\index{multidominance structure}%%
\index{MDS (see multidominance structure)}%%%
%%%%%
$\auf M, \prec, r\zu$ such that
$\auf M, \prec\zu$ is a directed acyclic graph with root
$r$ and for every $x < r$ the set $M(x) := \{y : x \prec y\}$
is linearly ordered by $<$.
\end{defn}
%%%%
With an MDS we only have coded the dominance relation between the
constituents. In order to include order we cannot simply add
another relation as we did with trees. Depending on the branching
number, a fair number of new relations will have to be added,
which represent the relations {\it the $i$th daughter of} (where
$i < n$, the maximum branching number). Since we are dealing with
binary branching trees we need only two of these relations.
%%%%%
\index{multidominance structure!ordered}%%%
\index{OMDS (see ordered multidominance structure)}%%
%%%%%
\begin{defn}
An \textbf{ordered} (\textbf{binary branching}) 
\textbf{multidominance structure} (\textbf{OMDS}) is a quadruple 
$\auf M, \prec_0, \prec_1, r\zu$
such that the following holds:
%%%
\begin{dingautolist}{192}
\item $\auf M, \prec_0 \cup \prec_1, r\zu$ is an MDS.
\item From $x \succ_0 y$ and $x \succ_0 z$ follows $y = z$.
\item From $x \succ_1 y$ and $x \succ_1 z$ follows $y = z$.
\item If $x \succ_1 z$ for some $z$ then there exists a
    $y \neq z$ with $x \succ_0 y$.
\end{dingautolist}
\end{defn}
%%%
(The reader may verify that \ding{193} and \ding{195} together imply 
that $\succ_0 \cap \succ_1 = \varnothing$.)
Let $\auf \GT, <, \sqsubset\zu$ be a binary branching ordered tree.
Then we put $x \prec_0 y$ if $x$ is a daughter of $y$ and there is
no daughter $z$ of $y$ with $z \sqsubset x$. Further, we write
$x \prec_1 y$ if $x$ is a daughter of $y$ but not $x \prec_0 y$.
%%%%
\begin{thm}
Let $K = \auf \GT, C\zu$ be a CCS over an ordered binary branching
tree with root $r$. Put $M := [x]_C$, $x \in T$, as well as for 
$i = 0,1$, $[x]_C \prec_i [y]_C$ iff there is an 
$x' \approx_C x$ and an $y' \approx_C y$ with $x' \prec_i y'$. 
Finally let 
%%%
\begin{equation}
M(K) := \auf M, \prec_0, \prec_1, [r]_K\zu
\end{equation}
%%%
Then $M(K)$ is an OMDS.
\end{thm}
%%%%
Now we want to deal with the problem of finding the CCS from
the OMDS.
%%%%
\begin{defn}
Let $\auf M, \prec_0, \prec_1, r\zu$ be an OMDS. An
\textbf{identifier} is a sequence $I = \auf x_i : i < n\zu$ such that
$r \succ x_0$ and $x_i \succ x_{i+1}$ for all $i \in n$.
$\CI(\GM)$ denotes the set of all identifiers of $\GM$.
The \textbf{address of} 
%%%
\index{address}%%%
%%%
$I$ is that sequence $\auf \gamma_i : i < n\zu$ such that for
all $i < n$ one has $x_{i} \prec_{\gamma_i} x_{i-1}$.
\end{defn}
%%%%
The following is easy to see.
%%%%
\begin{prop}
The set of addresses of an OMDS is a tree domain.
\end{prop}
%%%%
This means that we have already identified the tree structure.
What remains to do is to find the chains. The order is irrelevant, so
we ignore it. At first we want to establish which elements are
%%%
\index{element!overt}%%
%%%
overt. In a CCS an element $x$ is called \textbf{overt} if for
every $y \geq x$ the constituent $\low{y}$ is the head of its
chain. This we can also describe in the associated MDS. We say
%%%
\index{link}%%
\index{link!maximal}%%
\index{identifier!S--\faul}%%
%%%
a pair $\auf x,y\zu$ is a \textbf{link in} $\auf M, \prec, r\zu$
if $x \prec y$. The link is \textbf{maximal} if $y$ is maximal with
respect to $<$ in $M(x)$. An \textbf{S--identifier} is an identifier
$I = \auf x_i : i < n\zu$ where $\auf x_{i-1}, x_{i}\zu$
is a maximal link for all $i < n$. (For the purpose of this 
definition, $x_{-1}$ is the root.) The overt elements are exactly 
the S--identifiers.
%%%
%%%%
\begin{defn}
\index{link extension}%%
%%%
Let $\GM = \auf M, \prec, r\zu$ and $\GM' = \auf M', \prec', r'\zu$
be MDSs. Then $\GM'$ is called a  \textbf{link extension of} $\GM$
if $M' = M$, $r' = r$ and $\prec'\; = \;\prec \cup \;\{\auf x,y\zu\}$,
where $\auf x,y\zu$ is maximal in $\GM'$.
\end{defn}
%%%%
One finds out easily that if $K'$ is derived from $K$ by simple
copying then $M(K')$ is isomorphic to a link extension of $M(K)$.
Let conversely $\GM'$ be a link extension of $\GM$ and $K$ a CCS
such that $M(K) \cong \GM$. Then we claim that there is a CCS $K'$
for which $M(K') \cong \GM'$ and which results by copying from $K$.
This is unique up to isomorphism. The tree is given by $\CI(\GM')$.
Further, let the tree of $K$ be exactly $\GI(\GM)$. First we have
$\CI(\GM) \subset \CI(\GM')$, and the identity is an embedding
whose image contains all identifiers which do not contain the
subsequence $x;y$. Let now $y'$ be maximal with respect to $<$
in $\GM$. Further, let $I$ be the S--identifier of $y$ and $I'$
the S--identifier of $y'$ in $\GM$. Then $I' = I;J$ for some $J$
since $y' < y$. Define $\phi : I;J;x;K \mapsto I;x;K$. This is an
isomorphism of the constituent $\low{I;J;x}$ onto the constituent
$\low{I;x}$. Now we define the chains$^{\ast}$ on $\CI(\GM')$. Let
$\CD = \auf \Delta, \Phi\zu$ be the chain of $K$ which contains
the constituent $\low{I;J;xK}$. Then let
$\CD' := \auf \Delta \cup \{\im(\phi)\},
\Phi'\zu$, where $\Phi' := \Phi \cup \{\phi \circ \chi :
    \chi \in \Phi\} \cup \{\chi \circ \phi^{-1} :
    \chi \in \Phi\zu$.
For every other chain $\CC$ let $\CC' := \CC$. Finally
for an identifier $L < I;J;x;K$ we put $\CK_L := \auf \{\low{L}\}, %
\{1_{\low{L}}\}\zu$. Then we put
%%%
\begin{equation}
K' := \auf \CI(\GM'), <, \varepsilon, \{\CC' : \CC \in C\}
    \cup \{\CK_L : L < I;J;x;K\}\zu 
\end{equation}
%%%
This is a CCS. Evidently it satisfies {\sl Uniqueness}. Further,
{\sl Liberation\/} is satisfied as one easily checks. For {\sl No
Recycling\/} it suffices that the new link map is orbital. 
This is easy to see. 

Now, how does one define the kinds of structures that are common in
GB? One approximation is the following. We say a \textbf{trace chain
structure} 
%%%
\index{trace chain structure}%%%
%%%
is a pair $\auf \GT, C\zu$ where $C$ is a set of trace
chains. If we have a CCS we get the trace chain structure
relatively easily. To this end we replace all maximal nonovert
constituents in a tree by a trace (which is a one node tree). This
however deletes some chain members! Additionally it may happen
that some traces are not any more bound. Hence we say that a trace
chain structure is a pair $\auf \GT, C\zu$ which results from a
CCS by deleting overt constituents. Now one can define trace chain
structures also from MDSs, and it turns out that if two CCSs $K$
and $K'$ have isomorphic MDSs then their trace chain structures
are isomorphic. This has the following reason. An MDS is
determined from $\GT$ and $\approx_C$ alone. We can determine the
root of every element from $\GT$ and $\approx_C$, and further also
the root line. From this we can define the peak of every element
and therefore also the zenith. The overt elements are exactly the
elements in zenith position. Except for the overt element, the trace 
chain structure contains also the traces. These are exactly
the overt daughters of the overt elements.

Let us summarize. There exists a biunique correspondence
between derivations of trace chain structures, derivations of
CCSs and derivations of MDSs.  Further, there is a biunique
correspondence between MDSs and trace chain structures.
Insofar the latter two structures are exactly equivalent.
CCSs contain more information over the derivation (see the
exercises).
%%%%
\vplatz
\exercise
This example shows why we cannot use the ordering
$<$ in the MDSs. Let $\GM = \auf \{0,1,2\}, \prec, 0\zu$ and
$\GM' = \auf \{0,1,2\}, \prec', 0\zu$ with
$\prec = \{\auf 2,1\zu, \auf 1,0\zu\}$ and
$\prec' = \prec \cup \{\auf 2,0\zu\}$.
Evidently $\prec^+ = \prec'^+$.  Construct $\CI(\GM)$ and
$\CI(\GM')$ as well as the connected CCS.
%%%%
\vplatz
\exercise
Prove Proposition~\ref{prop:dicht}.
%%
\vplatz
\exercise
Show Lemma~\ref{lem:spleiss}.
%%
\vplatz
\exercise
Show that ac--command is transitive.
%%
\vplatz
\exercise
Show Proposition~\ref{prop:kandek}.
%%%
\vplatz
\exercise
Let the CCS in Figure~\ref{fig:zenith} be given. The members of
a chain are annotated by the same upper case Greek letter.
Trivial chains are not shown. Let the link maps be
$\phi_{\Gamma} \colon 2 \mapsto 4$, $\phi_{\Delta} \colon i \mapsto i + 6
\quad (i < 6)$, and also $\phi_{\Theta} \colon i \mapsto i + 13 
\quad (i < 13)$. Compute $[i]_{C}$ for every $i$. If instead of 
$\phi_{\Delta}$ we take the map $\phi'_{\Delta}$  how do the 
equivalence classes change?
%%
\begin{equation}
\phi'_{\Delta} \colon 1 \mapsto 8, 2 \mapsto 7, 3 \mapsto 9, 4
\mapsto 10, 5 \mapsto 11
\end{equation}
%%%
Determine the peak and the zenith of every element and the
maps.
%%%%
\begin{figure}
\begin{center}
\begin{picture}(22,35)
%%
\put(2,2){\line(0,1){9}}
    \put(2,2){\makebox(0,0){$\bullet$}}
        \put(2,1){\makebox(0,0){$1$}}
\put(5,2){\line(-1,1){3}}
    \put(5,2){\makebox(0,0){$\bullet$}}
        \put(5,1){\makebox(0,0){$2$}}
        \put(5,2.4){\vector(0,1){2.2}}
        \put(6,3.5){\makebox(0,0)[l]{$\Gamma$}}
\put(5,5){\line(-1,1){3}}
    \put(5,5){\makebox(0,0){$\bullet$}}
        \put(5,6){\makebox(0,0)[l]{$4$}}
    \put(2,5){\makebox(0,0){$\bullet$}}
        \put(1,5){\makebox(0,0)[r]{$3$}}
    \put(2,8){\makebox(0,0){$\bullet$}}
        \put(1,8){\makebox(0,0)[5]{$5$}}
        \put(2.4,8.4){\vector(1,1){5.2}}
        \put(5,12){\makebox(0,0){$\Delta$}}
\put(2,11){\makebox(0,0){$\bullet$}}
    \put(1,11){\makebox(0,0)[r]{$6$}}
%%
\put(2,11){\line(1,1){6}}
\put(8,17){\line(0,1){3}}
\put(8,20){\line(1,1){12}}
% \put(2,11){\makebox(0,0){$\bullet$}}
    \put(8,17){\makebox(0,0){$\bullet$}}
        \put(7,17){\makebox(0,0){$12$}}
\put(8,20){\makebox(0,0){$\bullet$}}
    \put(7,20){\makebox(0,0){$13$}}
%%
\put(8,8){\line(0,1){9}}
    \put(8,8){\makebox(0,0){$\bullet$}}
        \put(8,7){\makebox(0,0){$7$}}
\put(11,8){\line(-1,1){3}}
    \put(11,8){\makebox(0,0){$\bullet$}}
        \put(11,7){\makebox(0,0){$8$}}
\put(11,11){\line(-1,1){3}}
    \put(11,11){\makebox(0,0){$\bullet$}}
        \put(11,10){\makebox(0,0){$10$}}
    \put(8,11){\makebox(0,0){$\bullet$}}
        \put(7,11){\makebox(0,0)[r]{$9$}}
    \put(8,14){\makebox(0,0){$\bullet$}}
        \put(9,14){\makebox(0,0)[l]{$11$}}
        \put(8.4,17.4){\vector(1,1){5.2}}
        \put(11,21){\makebox(0,0){$\Theta$}}
%%
\put(14,26){\line(0,-1){15}}
    \put(14,26){\makebox(0,0){$\bullet$}}
        \put(13,27){\makebox(0,0){$26$}}
    \put(14,23){\makebox(0,0){$\bullet$}}
        \put(16,24){\makebox(0,0)[r]{$25$}}
    \put(14,20){\makebox(0,0){$\bullet$}}
        \put(13,20){\makebox(0,0)[r]{$19$}}
%%
\put(14,23){\line(1,-1){9}}
\put(14,17){\makebox(0,0){$\bullet$}}
    \put(13,17){\makebox(0,0)[r]{$18$}}
\put(14,17){\line(1,-1){3}}
    \put(14,14){\makebox(0,0){$\bullet$}}
        \put(13,14){\makebox(0,0)[r]{$16$}}
\put(14,14){\line(1,-1){3}}
    \put(14,11){\makebox(0,0){$\bullet$}}
        \put(14,10){\makebox(0,0){$14$}}
    \put(17,11){\makebox(0,0){$\bullet$}}
        \put(17,10){\makebox(0,0){$15$}}
    \put(17,14){\makebox(0,0){$\bullet$}}
        \put(17,13){\makebox(0,0){$17$}}
%%
\put(20,17){\line(0,-1){6}}
    \put(20,17){\makebox(0,0){$\bullet$}}
        \put(21,18){\makebox(0,0)[r]{$24$}}
    \put(20,14){\makebox(0,0){$\bullet$}}
        \put(19,14){\makebox(0,0)[r]{$22$}}
    \put(23,14){\makebox(0,0){$\bullet$}}
        \put(23,13){\makebox(0,0){$23$}}
\put(20,14){\line(1,-1){3}}
    \put(23,11){\makebox(0,0){$\bullet$}}
        \put(23,10){\makebox(0,0){$21$}}
    \put(20,11){\makebox(0,0){$\bullet$}}
        \put(20,10){\makebox(0,0){$20$}}
%%
\put(20,32){\line(1,-1){3}}
    \put(17,29){\makebox(0,0){$\bullet$}}
        \put(16,30){\makebox(0,0){$27$}}
    \put(20,32){\makebox(0,0){$\bullet$}}
        \put(20,33){\makebox(0,0){$29$}}
    \put(23,29){\makebox(0,0){$\bullet$}}
        \put(23,28){\makebox(0,0){$28$}}
\end{picture}
\end{center}
\caption{A Copy--Chain Structure}
\label{fig:zenith}
\end{figure}
%%
\vplatz
\exercise
Let $d(n)$ be the largest number of nonisomorphic CCSs which have
(up to isomorphism) the same MDS.  Show that $d(n) \in O(2^n)$.

 \newpage
%%
%% Seitenlayout
%%
{\small
 \newenvironment{eintrag}{\begin{tabular}{@{\quad}p{1.6cm}@{}p{9.8cm}}}%
{\end{tabular}}
\chapter*{Bibliography}
\markboth{Bibliography}{}
%%%
Ajdukiewicz, Kazimierz 
\\
\begin{eintrag}
1936 & Die syntaktische {K}onnexit\"at [The syntactic connectivity]. 
        {\em Studia Philosophica} 1:1 -- 27.
\end{eintrag}
\\[2.5mm]
Barendregt, Henk
\\
\begin{eintrag}
1985 & {\em The Lambda Calculus. Its Syntax and Semantics}.
	Studies in Logic and the Foundations of Mathematics 103. 
	2nd ed. Amsterdam: Elsevier.
\end{eintrag}
\\[2.5mm]
Barker, Chris, and Pullum, Geoffrey.
\\
\begin{eintrag}
1990 & A theory of command relations. {\em Linguistics and Philosophy} 
13:1--34.
\end{eintrag}
\\[2.5mm]
Bauer, Brigitte L.~M. 
\\
\begin{eintrag}
1995 & {\em The {E}mergence and {D}evelopment of {SVO} {P}atterning in
  {L}atin and {F}rench}. Oxford: Oxford University Press.
\end{eintrag}
\\[2.5mm]
Bird, Steven, and Ellison, Mark
\\
\begin{eintrag}
1994 & One--level phonology: {A}utosegmental representations and rules as
  finite automata. {\em Computational Linguistics} 20:55 -- 90.
\end{eintrag}
\\[2.5mm]
Blackburn, Patrick
\\
\begin{eintrag}
1993 & Modal logic and attribute value structures.  In {\em Diamonds and 
	Defaults}, Maarten de Rijke (ed.),  19 -- 65.  (Synthese
	Library 229.) Dordrecht: Kluwer.
\end{eintrag}
\\[2.5mm]
Blok, Wim J., and Pigozzi, Don~J. 
\\
\begin{eintrag}
1990 & Algebraizable logics. 
{\em Memoirs of the Americal Mathematical Society}, 77(396).
\end{eintrag}
\\[2.5mm]
Bochvar, D.~A.
\\
\begin{eintrag}
1938 & On a three--valued logical calculus and its application to the
  analysis of contradictions.
  {\em Mathematicheskii Sbornik} 4:287 -- 308.
\end{eintrag}
\\[2.5mm]
B\"ottner, Michael, and Th\"ummel, Wolf (eds.)
\\
\begin{eintrag}
2000 & {\em Variable--free {S}emantics}. Artikulation und
  Sprache 3. Osnabr\"uck: secolo Verlag.
\end{eintrag}
\\[2.5mm]
Bresnan, Joan, Kaplan, Ronald~M., Peters, Stanley, and Zaenen, Annie
\\
\begin{eintrag}
1987 & Cross--{S}erial {D}ependencies in {D}utch. In {\em The {F}ormal 
  {C}omplexity of {N}atural {L}anguage}, Walter Savitch, Emmon Bach, William 
  Marsch, and Gila Safran--Naveh (eds.), 286 -- 319. Dordrecht: Reidel.
\end{eintrag}
\\[2.5mm]
B\"uchi, J.
\\
\begin{eintrag}
1960 & Weak second--order arithmetic and finite automata.
  {\em Zeitschrift f\"ur Mathematische Logik und Grundlagen der
  Mathematik} 6:66 -- 92.
\end{eintrag}
\newpage
\noindent
Burmeister, Peter
\\
\begin{eintrag}
1986 &  {\em A {M}odel {T}heoretic {O}riented {A}pproach to {P}artial
  {A}lgebras}. Berlin: Akademie Verlag. \\
2002 & {L}ecture {N}otes on {U}niversal {A}lgebra. {M}any {S}orted {P}artial
  {A}lgebras. Manuscript available via internet.
\end{eintrag}
\\[3mm]
Burris, Stanley, and Sankappanavar, H.~P.
\\
\begin{eintrag}
1981 & {\em A Course in Universal Algebra}.  Graduate Texts in 
	Mathematics 78. Berlin/New York: Springer.
\end{eintrag}
\\[3mm]
Buszkowski, Wojciech
\\
\begin{eintrag}
1997 & Mathematical linguistics and proof theory.
   In {\em Handbook of Logic and Language}, Johan van Benthem and 
   Alice ter Meulen (eds.), 683 -- 736. Amsterdam: Elsevier.
\end{eintrag}
\\[3mm]
Carpenter, Bob
\\
\begin{eintrag}
1992 & {\em The Logic of Typed Feature Structures}.
	Cambridge Tracts in Theoretical Computer Science 32. Cambridge: 
	Cambridge University Press.
\end{eintrag}
\\[3mm]
Chandra, A.~K., Kozen, D.~C., and Stockmeyer, L.~J.
\\
\begin{eintrag}
1981 & Alternation. {\em Journal of the Association for Computing Machinery} 
	28:114 -- 133.
\end{eintrag}
\\[3mm]
Chomsky, Noam and Halle, Morris
\\
\begin{eintrag}
1968 & {\em The sound pattern of {E}nglish}.
       New York: Harper and Row.
\end{eintrag}
\\[3mm]
Chomsky, Noam
\\
\begin{eintrag}
1959 & On certain formal properties of grammars.
       {\em Information and Control} 2:137 -- 167.
\\
1962 & Context--free grammars and pushdown storage.  {\em MIT Research 
  Laboratory of Electronics Quarterly Progress Report} 65.
\\
1981 & {\em Lecture Notes on Government and Binding}. Dordrecht: Foris.
\\
1986 & {\em Barriers}. Cambridge (Mass.): MIT Press.
\\
1993 & A minimalist program for linguistic theory. In {\em The View from 
	Building 20: Essays in Honour of Sylvain Bromberger}, Ken Hale and 
	Samuel J. Keyser (eds.), 1 -- 52. Cambridge (Mass.): MIT Press.
\end{eintrag}
\\[3mm]
Church,  Alonzo
\\
\begin{eintrag}
1933 & A set of postulates for the foundation of logic.
       {\em Annals of Mathematics} 2:346 -- 366.
\\
1940 & A formulation of the simple theory of types.
       {\em Journal of Symbolic Logic} 5:56 -- 68.
\end{eintrag}
\\[3mm]
Coulmas, Florian
\\
\begin{eintrag}
2003 & {\em Writing Systems. An introduction to their linguistic 
	analysis}. Cambridge: Cambridge University Press.
\end{eintrag}
\newpage
\noindent
Culy, Christopher 
\\
\begin{eintrag}
1987 & The {C}omplexity of the {V}ocabulary of {B}ambara.  In 
	{\em The {F}ormal {C}omplexity of {N}atural {L}anguage}, 
	Walter Savitch, Emmon Bach, William Marsch, and Gila Safran--Naveh
        (eds.), 345 -- 351. Dordrecht: Reidel.
\end{eintrag}
\\[2.8mm]
Curry, Haskell~B. 
\\
\begin{eintrag}
1930 & Grundlagen der kombinatorischen {L}ogik [{F}oundations of
  combinatory logic]. {\em American Journal of Mathematics} 
  52:509 -- 536, 789 -- 834.
\\
1977 & {\em Foundations of Mathematical Logic}. 2nd ed. New York: 
	Dover Publications.
\end{eintrag}
\\[2.8mm]
Davey, B.~A., and Priestley, H.~A.
\\
\begin{eintrag}
1990 & {\em Lattices and Order}. Cambridge: Cambridge University Press.
\end{eintrag}
\\[2.8mm]
Deutsch, David, Ekert, Artur, and Luppacchini, Rossella
\\
\begin{eintrag}
2000 & Machines, logic and quantum physics. {\em Bulletin of Symbolic 
	Logic} 6:265 -- 283.
\end{eintrag}
\\[2.8mm]
Doner, J.~E.
\\
\begin{eintrag}
1970 & Tree acceptors and some of their applications.
       {\em Journal of Computer and Systems Sciences} 4:406 -- 451.
\end{eintrag}
\\[2.8mm]
Dowty, David~R., Wall, Robert~E., and Peters, Stanley 
\\
\begin{eintrag}
1981 & {\em Introduction to Montague Semantics}. Synthese Library 11. 
	Dordrecht: Reidel.
\end{eintrag}
\\[2.8mm]
Dresner, Eli
\\
\begin{eintrag}
2001 & Tarski's {R}estricted {F}orm and {N}eale's {Q}uantificational
  {T}reatment of {P}roper {N}ames. {\em Linguistics and Philosophy} 
  24:405 -- 415.
\\
2002 & Holism, {L}anguage {A}cquisition, and {A}lgebraic {L}ogic.
     {\em Linguistics and Philosophy} 25:419 -- 452.
\end{eintrag}
\\[2.8mm]
Dymetman, Marc
\\
\begin{eintrag}
1991 & Inherently reversible grammars, logic programming and computability.
   In {\em Proceedings of the ACL Workshop: Reversible Grammars in
   Natural Language Processing}.
\\
1992 & {\em Transformations de Grammaires Logiques et R\'eversibilit\'e}
  [Transformations of Logical Grammars and Reversibility].
   Ph.\ D.\  diss., Universit\'e Joseph Fourier, Grenoble.
\end{eintrag}
\\[2.8mm]
Ebbinghaus, Hans-Dieter, and Flum, J\"org 
\\
\begin{eintrag}
1995 & {\em Finite {M}odel {T}heory}. Perspectives in Mathematical 
	Logic. Berlin/New York: Springer.
\end{eintrag}
\\[2.8mm]
Ebert, Christian, and Kracht, Marcus
\\\begin{eintrag}
2000 & Formal syntax and semantics of case stacking languages.
        In {\em Proceedings of the EACL 2000}.
\end{eintrag}
\newpage
\noindent
van Eijck, Jan
\\\begin{eintrag}
1994 & Presupposition failure: a comedy of errors. {\em Formal Aspects 
	of Computing} 3.
\end{eintrag}
\\[1.95mm]
Eisenberg, Peter
\\\begin{eintrag}
1973 & A {N}ote on `{I}dentity of {C}onstituents'. {\em Linguistic 
	Inquiry} 4:417 -- 420.
\end{eintrag}
\\[1.95mm]
Ewen, Colin~J., and van~der Hulst, Harry
\\\begin{eintrag}
2001 & {\em The Phonological Structure of Words}.
	Cambridge: Cambridge University Press.
\end{eintrag}
\\[1.95mm]
Ferreir\'os, Jos\'e
\\\begin{eintrag}
2001 & The road to modern logic --- an interpretation.
    {\em The Bulletin of Symbolic Logic} 7:441 -- 484.
\end{eintrag}
\\[1.95mm]
Fiengo, Robert, and May, Robert 
\\
\begin{eintrag}
1994 & {\em Indices and Identity}. Linguistic Inquiry Monographs 24. 
	Cambridge (Mass.): MIT Press.
\end{eintrag}
\\[1.95mm]
Fine, Kit
\\\begin{eintrag}
1992 & Transparency, {P}art {I}: {R}eduction. Unpublished manuscript, UCLA.
\end{eintrag}
\\[1.95mm]
Frege, Gottlob 
\\\begin{eintrag}
1962 & Funktion und {B}egriff [{F}unction and {C}oncept].
  In {\em Funktion, Begriff, Bedeutung.  F\"unf logische Studien} 
	[{F}unction, {C}oncept, {M}eaning. {F}ive logical {S}tudies], 
	G\"unther Patzig (ed.), 17 -- 39. G\"ottingen: Vandenhoeck \& Ruprecht.
\end{eintrag}
\\[1.95mm]
Frey, Werner 
\\\begin{eintrag}
1993 & {\em {S}yntaktische {B}edingungen f\"{u}r die semantische
  {I}nterpretation} [Syntactic {C}onditions for the {S}emantic 
  {I}nterpretation]. Number~35 in Studia Grammatica. Berlin: 
  Akademie Verlag.
\end{eintrag}
\\[1.95mm]
Fromkin, V. (ed.)
\\\begin{eintrag}
2000 & {\em Linguistics: An Introduction to linguistic theory}.
         London: Blackwell.
\end{eintrag}
\\[1.95mm]
Gamut, L.~T.~F. 
\\\begin{eintrag}
1991a & {\em Logic, Language and Meaning. Vol. 1: Introduction to 
	Logic}. Chicago: The University of Chicago Press.
\\
1991b & {\em Logic, Language and Meaning. Vol. 2: Intensional Logic and
  Logical Grammar}. Chicago: The University of Chicago Press.
\end{eintrag}
\\[1.95mm]
G\"ardenfors, Peter
\\\begin{eintrag}
1988 & {\em Knowledge in Flux}. Cambridge (Mass.): MIT Press.
\end{eintrag}
\\[1.95mm]
Gazdar, Gerald, Klein, Ewan, Pullum, Geoffrey, and Sag, Ivan.
\\\begin{eintrag}
1985 & {\em Generalized Phrase Structure Grammar}. London: Blackwell.
\end{eintrag}
\\[2mm]
Gazdar, Gerald, Pullum, Geoffrey, Carpenter, Bob, Hukari, T., and Levine, R.
\\\begin{eintrag}
1988 & Category structures. {\em Computational Linguistics} 14:1 -- 19.
\end{eintrag}
\newpage
\noindent
Geach, Peter 
\\\begin{eintrag}
1972 & A {P}rogram for {S}yntax. In {\em Semantics for Natural Language}, 
	Donald Davidson and Gilbert Harman (eds.). (Synthese Library 40.) 
	Dordrecht: Reidel.
\end{eintrag}
\\[2.8mm]
Geller, M.~M., and  Harrison, M.~A.
\\\begin{eintrag}
1977 & On {$LR(k)$} grammars and languages. {\em Theoretical Computer 
	Science} 4:245 -- 276.
\end{eintrag}
\\[2.8mm]
Geurts, Bart
\\\begin{eintrag}
1998 & Presupposition and {A}naphors in {A}ttitude {C}ontexts.
    {\em Linguistics and Philosophy} 21:545 -- 601.
\end{eintrag}
\\[2.8mm]
Ginsburg, Seymour and Spanier, Edwin~H. 
\\\begin{eintrag}
1964 & Bounded {ALGOL}--{L}ike {L}anguages. {\em Transactions of the 
	American Mathematical Society} 113:333 -- 368.
\\
1966 & Semigroups, {P}resburger {F}ormulas, and {L}anguages.
	{\em Pacific Journal of Mathematics} 16:285 -- 296.
\end{eintrag}
\\[2.8mm]
Ginsburg, Seymour
\\\begin{eintrag}
1975 & {\em Algebraic and Automata--Theoretic Properties of Formal
  Languages}. Amsterdam: North--Holland.
\end{eintrag}
\\[2.8mm]
Goldstern, Martin, and Judah, Haim 
\\\begin{eintrag}
1995 & {\em The Incompleteness Phenomenon}. Wellesley (Mass.): 
	AK Peters.
\end{eintrag}
\\[2.8mm]
Gr\"atzer, George 
\\\begin{eintrag}
1968 & {\em Universal Algebra}. New York: van Nostrand.
\\
1971 & {\em Lattice {T}heory: {F}irst {C}oncepts and {D}istributive
  {L}attices}. Freeman.
\end{eintrag}
\\[2.8mm]
Greibach, Sheila~A. 
\\\begin{eintrag}
1967 & A new normal--form theorem for context--free phrase structure
  grammars. {\em Journal of the Association for Computing Machinery} 
  13:42 -- 52.
\end{eintrag}
\\[2.8mm]
Grewendorf, G\"unter, Hamm, Friedrich, and Sternefeld, Wolfgang
\\\begin{eintrag}
1987 & {\em Sprachliches Wissen. Eine Einf\"uhrung in moderne Theorien der
  grammatischen Beschreibung\/} [{K}nowledge of {L}anguage. 
	An {I}ntroduction to {M}odern {T}heories of {G}rammatical 
	{D}escription].
  Number 695 in suhrkamp taschenbuch wissenschaft. Frankfurt a.M.: 
	Suhrkamp Verlag.
\end{eintrag}
\\[2.8mm]
Groenink, Annius~V. 
\\\begin{eintrag}
1997a & Mild context--sensitivity and tuple--based generalizations of
  context--grammar. {\em Linguistics and Philosophy} 20:607--636.
\\
1997b & {\em Surface without Structure. Word Order and Tractability Issues in
  Natural Language Ana\-ly\-sis}. Ph.\ D.\ diss., University of Utrecht.
\end{eintrag}
\newpage\noindent
de~Groote, Philippe
\\
\begin{eintrag}
2001 & Towards {A}bstract {C}ategorial {G}rammars.
  In {\em Association for Computational Linguistics, 39th Annual
  Meeting and 10th Conference of the European Chapter}, 148 -- 155,
  Toulouse.
\end{eintrag}
\\[2.25mm]
Haider, Hubert 
\\\begin{eintrag}
1991 & Die menschliche Sprachf\"ahigkeit --- exaptiv und kognitiv opak
	[The human language faculty --- exaptive and cognitively opaque].
       {\em Kognitionswissenschaft} 2:11 -- 26.
\\
1993 & {\em Deutsche Syntax --- generativ. Vorstudien zur Theorie einer
       projektiven Grammatik} [German Syntax --- generative. Preliminary 
       studies towards a theory of projective grammar]. T\"ubingen: 
       Gunter Narr Verlag.
\\
1995 & Downright down to the right. In {\em On {E}xtraction and 
	{E}xtraposition}, U.~Lutz and J.~Pafel (eds.), 145 -- 271. 
	Amsterdam: John Benjamins.
\\
1997 & Extraposition. In {\em Rightward Movement}, 
	D.~Beerman, D.~LeBlanc, and H.~van Riemsdijk (eds.), 
	115 -- 151. Amsterdam: John Benjamins.
\\
2000 & {B}ranching and {D}ischarge. In {\em Lexical Specification and 
	Insertion}, Peter Coopmans, Martin Everaert, and Jane Grimshaw 
	(eds.), 135 -- 164. (Current Issues in Linguistic Theory 197.) 
        Amsterdam: John Benjamins.
\end{eintrag}
\\[2.25mm]
Halmos, Paul 
\\\begin{eintrag}
1956 & Homogeneous locally finite polyadic boolean algebras of infinite
  degree. {\em Fundamenta Mathematicae} 43:255 -- 325.
\end{eintrag}
\\[2.25mm]
Hamm, Friedrich, and van Lambalgen, Michiel 
\\\begin{eintrag}
2003 & Event Calculus, Nominalization and the Progressive. {\em Linguistics 
and Philosophy} 26:381 -- 458.
\end{eintrag}
\\[2.25mm]
Harkema, Henk
\\\begin{eintrag}
2001 & A {C}haracterization of {M}inimalist {L}anguages.  In 
        {\em Logical Aspects of Computational Linguistics (LACL '01)}, 
	Philippe de~Groote, Glyn Morrill, and Christian Retor\'e (eds.),
	193 -- 211. (Lecture Notes in Artificial Intelligence 2099.)
	Berlin/New York: Springer.
\end{eintrag}
\\[2.25mm]
Harris, Zellig~S. 
\\\begin{eintrag}
1963 & {\em Structural Linguistics}. The University of Chicago Press.
\\
1979 & {\em Mathematical Structures of Language}. Huntington (NY): 
	Robert E. Krieger Publishing Company.
\end{eintrag}
\\[2.25mm]
Harrison, Michael~A. 
\\\begin{eintrag}
1978 & {\em Introduction to Formal Language Theory}. 
	Reading (Mass.): Addison Wesley.
\end{eintrag}
\\[2.25mm]
Hausser, Roland~R. 
\\\begin{eintrag}
1984 & {\em Surface Compositional Grammar}. M\"unchen: Wilhelm Finck 
	Verlag.
\end{eintrag}
\newpage\noindent
Heim, Irene 
\\\begin{eintrag}
1983 & On the projection problem for presuppositions. In 
	{\em Proceedings of the 2nd West Coast Conference on Formal 
	Linguistics}, M.~Barlow and D.~Flickinger, D.~Westcoat (eds.), 
        114 -- 126, Stanford University.
\end{eintrag}
\\[1.9mm]
Hendriks, Herman 
\\\begin{eintrag}
2001 & Compositionality and {M}odel--{T}heoretic {I}nterpretation.
{\em Journal of Logic, Language and Information} 10:29 -- 48.
\end{eintrag}
\\[1.9mm]
Henkin, Leon, Monk, Donald, and Tarski, Alfred 
\\\begin{eintrag}
1971 & {\em Cylindric Algebras. Part 1}. Studies in Logic 
	and the Foundation of Mathematics 64.  Amsterdam: North--Holland.
\end{eintrag}
\\[1.9mm]
Hindley, J.~R., Lercher, B., and Seldin, J.~P.
\\\begin{eintrag}
1972 & {\em Introduction to Combinatory Logic}. London Mathematical 
	Society Lecture Notes 7. Oxford: Oxford University Press.
\end{eintrag}
\\[1.9mm]
Hindley, J.~R. and Longo, L.
\\\begin{eintrag}
1980 & Lambda calculus models and extensionality. {\em Zeitschrift 
	f\"ur mathematische Logik und Grundlagen der Mathematik} 
	26:289 -- 310.
\end{eintrag}
\\[1.9mm]
Hintikka, Jaakko
\\\begin{eintrag}
1962 & {\em Knowledge and Belief. An Introduction into the logic of 
	the two notions}. Ithaca: Cornell University Press.
\end{eintrag}
\\[1.9mm]
Hodges, Wilfrid 
\\\begin{eintrag}
2001 & Formal features of compositionality. {\em Journal of Logic, 
	Language and Information} 10:7 -- 28.
\end{eintrag}
\\[1.9mm]
Hopcroft, John~E., and Ullman, Jeffrey~D. 
\\\begin{eintrag}
1969 & {\em Formal Languages and their Relation to Automata}.
	Reading (Mass.): Addison Wesley.
\end{eintrag}
\\[1.9mm]
van der Hulst, Harry
\\\begin{eintrag}
1984 & {\em Syllable Structure and Stress in Dutch}. Dordrecht: Foris.
\end{eintrag}
\\[1.9mm]
Huybregts, Riny
\\\begin{eintrag}
1984 & {O}verlapping {D}ependencies in {D}utch. {\em Utrecht Working 
	Papers in Linguistics} 1:3 -- 40.
\end{eintrag}
\\[1.9mm]
IPA
\\\begin{eintrag}
1999 & {\em Handbook of the International Phonetic Association}.
	Cambridge: Cambridge University Press.
\end{eintrag} 
\\[1.9mm]
Jackendoff, Ray
\\\begin{eintrag}
1977 & {\em $\oli{X}$--{S}yntax: {A} {S}tudy of {P}hrase {S}tructure}.
	Linguistic Inquiry Monographs 2. Cambridge (Mass.): 
	MIT Press.
\end{eintrag}
\\[1.9mm]
Jacobson, Pauline 
\\\begin{eintrag}
1999 & Toward a {V}ariable--{F}ree semantics. {\em Linguistics and 
	Philosophy} 22:117 -- 184.
\end{eintrag}
\newpage\noindent
\begin{eintrag}
2002 & The ({D}is){O}rganisation of the {G}rammar: 25 {Y}ears.
	{\em Linguistics and Philosophy} 25:601 -- 626.
\end{eintrag}
\\[3.4mm]
Johnson, J.~S.
\\\begin{eintrag}
1969 & Nonfinitizability of classes of representable polyadic algebras.
	{\em Journal of Symbolic Logic} 34:344 -- 352.
\end{eintrag}
\\[3.4mm]
Johnson, Mark
\\\begin{eintrag}
1988 & {\em Attribute--Value Logic and the Theory of Grammar}. 
	CSLI Lecture Notes 16. Stanford: CSLI.
\end{eintrag}
\\[3.4mm]
Jones, Burton~W. 
\\\begin{eintrag}
1955 & {\em The Theory of Numbers}. New York: Holt, Rinehart and Winston. 
\end{eintrag}
\\[3.4mm]
Joshi, Aravind, Levy, Leon~S., and Takahashi, Masako 
\\\begin{eintrag}
1975 & Tree {A}djunct {G}rammars. {\em Journal of Computer and System 
	Sciences} 10:136 -- 163.
\end{eintrag}
\\[3.4mm]
Joshi, Aravind~K. 
\\\begin{eintrag}
1985 & Tree adjoining grammars: {H}ow much context--sensitivity is 
	required to provide reasonable structural descriptions?
	In {\em Natural {L}anguage {P}arsing. {P}sychological, 
	{C}omputational, and {T}heoretical {P}erspectives}, 
        David Dowty, Lauri Karttunen, and Arnold Zwicky (eds.), 
	206--250. Cambridge: Cambridge University Press.
\end{eintrag}
\\[3.4mm]
Just, Winfried and Weese, Martin 
\\\begin{eintrag}
1996 & {\em Discovering Modern Set Theory. Vol.\ I: The Basics}. 
	Graduate Studies in Mathematics 8. AMS.
\\
1997 & {\em Discovering Modern Set Theory. Vol.\ II: Set--Theoretic 
	Tools for Every Mathematician}. Graduate Studies in Mathematics 
	18.  AMS.
\end{eintrag}
\\[3.4mm]
Kac, Michael~B., Manaster--Ramer, Alexis, and Rounds, William~C. 
\\\begin{eintrag}
1987 & Simultaneous--{D}istributive {C}oordination and {C}ontext--{F}reeness.
	{\em Computational Linguistics} 13:25 -- 30.
\end{eintrag}
\\[3.4mm]
Kaplan, Ron~M., and Kay, Martin 
\\\begin{eintrag}
1994 & Regular {M}odels of {P}honological {R}ule {S}ystems.
	{\em Computational Linguistics} 20:331 -- 378.
\end{eintrag}
\\[3.4mm]
Karttunen, Lauri
\\\begin{eintrag}
1974 & Presuppositions and linguistic context. {\em Theoretical Linguistics} 
	1:181 -- 194.
\end{eintrag}
\\[3.4mm]
Kasami, Tadao, Seki, Hiroyuki, and Fujii, Mamoru 
\\\begin{eintrag}
1987 & Generalized context--free grammars, multiple context--free 
	grammars and head grammars. Technical report, Osaka University.
\end{eintrag}
%\\[2mm]
\newpage\noindent
Kasami, Tadao 
\\\begin{eintrag}
1965 & An efficient recognition and syntax--analysis algorithm for
	context--free languages. Technical report, Air Force Cambridge 
	Research Laboratory, Bedford (Mass.). Science Report AFCRL--65--758.
\end{eintrag}
\\[2.1mm]
Kayne, Richard~S. 
\\\begin{eintrag}
1994 & {\em The {A}ntisymmetry of {S}yntax}. Linguistic 
	Inquiry Monographs 25. Cambridge (Mass.): MIT Press.
\end{eintrag}
\\[2.1mm]
Keenan, Edward~L., and Faltz, Leonard~L. 
\\\begin{eintrag}
1985 & {\em Boolean Semantics for Natural Language}. Dordrecht: Reidel.
\end{eintrag}
\\[2.1mm]
Keenan, Edward L., and Westerst{\aa}hl, Dag
\\\begin{eintrag}
1997 & Generalized quantifiers. In {\em Handbook of Logic and Language}, 
	Johan van Benthem and Alice ter Meulen (eds.), 835 -- 893. 
	Amsterdam: Elsevier.
\end{eintrag}
\\[2.1mm]
Kempson, Ruth
\\\begin{eintrag}
1975 & {\em Presupposition and the delimitation of semantics}. 
	Cambridge: Cambridge University Press.
\end{eintrag}
\\[2.1mm]
Kleene, Stephen~C. 
\\\begin{eintrag}
1956 & Representation of events in nerve nets. In {\em Automata Studies}, 
	C.~E. Shannon and J.~McCarthy (eds.), 3 -- 40. 
	Princeton: Princeton University Press.
\end{eintrag}
\\[2.1mm]
Knuth, Donald 
\\\begin{eintrag}
1956 & On the translation of languages from left to right.
       {\em Information and Control} 8:607 -- 639.
\end{eintrag}
\\[2.1mm]
Koskenniemi, Kimmo
\\\begin{eintrag}
1983 & Two--level morphology. {A} general computational model for 
	word--form recognition. Technical Report~11, Department of 
	General Linguistics, University of Helsinki.
\end{eintrag}
\\[2.1mm]
Koster, Jan
\\\begin{eintrag}
1986 & {\em Domains and Dynasties: {T}he Radical Autonomy of Syntax}.
	Dordrecht: Foris.
\end{eintrag}
\\[2.1mm]
Koymans, J.~P.~C. 
\\\begin{eintrag}
1982 & Models of the {L}ambda {C}alculus. {\em Information and 
	Control} 52:306 -- 332.
\end{eintrag}
\\[2.1mm]
Kracht, Marcus
\\\begin{eintrag}
1993 & Mathematical aspects of command relations. In {\em Proceedings 
	of the EACL 93}, 241 -- 250.
\\
1994 & Logic and {C}ontrol: {H}ow {T}hey {D}etermine the {B}ehaviour of
	{P}resuppositions. In {\em Logic and Information Flow}, 
	Jan van Eijck and Albert Visser (eds.), 88 -- 111. 
	Cambridge (Mass.): MIT Press.
\end{eintrag}
\newpage\noindent
\begin{eintrag}
1995a & Is there a genuine modal perspective on feature structures?
	{\em Linguistics and Philosophy} 18:401 -- 458. \\
1995b & Syntactic {C}odes and {G}rammar {R}efinement. {\em Journal 
	of Logic, Language and Information} 4:41 -- 60.
\\
1997 & Inessential {F}eatures. In {\em Logical Aspects of Computational 
	Linguistics (LACL '96)}, Christian Retor\'{e} (ed.), 43 -- 62. 
	(Lecture Notes in Artificial Intelligence 1328.)
	Berlin/New York: Springer.
\\
1998 & Adjunction {S}tructures and {S}yntactic {D}omains. In {\em The 
	Mathematics of Sentence Structure. Trees and Their Logics}, 
	Uwe M\"onnich and Hans-Peter Kolb (eds.), 259 -- 299. (Studies 
	in Generative Grammar 44.) Berlin: Mouton de Gruyter.
\\
1999 & {\em Tools and {T}echniques in {M}odal {L}ogic}. 
	Studies in Logic and the Foundations of Mathematics 142. 
	Amsterdam: Elsevier.
\\
2001a & Modal {L}ogics {T}hat {N}eed {V}ery {L}arge {F}rames.
	{\em Notre Dame Journal of Formal Logic} 42:141 -- 173.
\\
2001b & Syntax in {C}hains. {\em Linguistics and Philosophy}
	24:467 -- 529.
\\
2003 & Against the feature bundle theory of case. In 
	{\em New Perspectives on Case Theory}, Eleonore Brandner 
	and Heike Zinsmeister (eds.), 165 -- 190. Stanford: CSLI.
\end{eintrag}
\\[2.2mm]
Kuroda, S.~Y.
\\\begin{eintrag}
1964 & Classes of languages and linear bounded automata. {\em Information 
	and Control} 7:207 -- 223.
\end{eintrag}
\\[2.2mm]
Lamb, Sydney~M. 
\\\begin{eintrag}
1966 & {\em Outline of {S}tratificational {G}rammar}. Washington: 
	Georgetown University Press.
\end{eintrag}
\\[2.2mm]
Lambek, Joachim
\\\begin{eintrag}
1958 & The {M}athematics of {S}entence {S}tructure. 
	{\em The American Mathematical Monthly} 65:154 -- 169.
\end{eintrag}
\\[2.2mm]
Landweber, Peter~S. 
\\\begin{eintrag}
1963 & Three theorems on phrase structure grammars of type 1.
	{\em Information and Control} 6:131 -- 137.
\end{eintrag}
\\[2.2mm]
Langholm, Tore 
\\\begin{eintrag}
2001 & A {D}escriptive {C}haracterisation of {I}ndexed {G}rammars.
	{\em Grammars} 4:205 -- 262.
\end{eintrag}
\\[2.2mm]
Lehmann, Winfred~P. 
\\\begin{eintrag}
1993 & {\em Theoretical {B}ases of {I}ndo--{E}uropean {L}inguistics}.
	London: Routledge.
\end{eintrag}
\\[2.2mm]
Leibniz, Gottfried~Wilhelm 
\\\begin{eintrag}
2000 & {\em Die Grund\-la\-gen des lo\-gi\-schen Kal\-k\"uls
        (Lateinisch--Deutsch)} [The {F}oundations of the Logical Calculus
	(Latin--German)]. Phi\-lo\-so\-phi\-sche Bibliothek 525. 
	Hamburg: Meiner Verlag.
\end{eintrag}
\\[2mm]
% Seitenumbruch
Levelt, Willem~P. 
\\\begin{eintrag}
1991 & {\em Speaking. From Intention to Articulation}. 2nd ed.
	Cambridge (Mass.): MIT Press.
\end{eintrag}
\\[3.2mm]
Lyons, John 
\\\begin{eintrag}
1968 & {\em Introduction to Theoretical Linguistics}. Cambridge: 
	Cambridge University Press.
\\
1978 & {\em Semantics. Vol. 1}. Cambridge: Cambridge University Press. 
\end{eintrag}
\\[3.2mm]
Manaster--Ramer, Alexis, and Kac, Michael~B. 
\\\begin{eintrag}
1990 & The {C}oncept of {P}hrase {S}tructure. {\em Linguistics and 
	Philosophy} 13:325 -- 362.
\end{eintrag}
\\[3.2mm]
Manaster--Ramer, Alexis, Moshier, M.~Andrew, and Zeitman, R.~Suzanne 
\\\begin{eintrag}
1992 & An {E}xtension of {O}gden's {L}emma. Manuscript. Wayne State 
	University, 1992.
\end{eintrag}
\\[3.2mm]
Manaster--Ramer, Alexis
\\\begin{eintrag}
1986 & 
Copying in natural languages, context--freeness and queue grammars.
In {\em Proceedings of the 24th Annual Meeting of the Association for
  Computational Linguistics}, 85 -- 89.
\end{eintrag}
\\[3.2mm]
Markov, A.~A.
\\\begin{eintrag}
1947 & On the impossibility of certain algorithms in the theory of
  associative systems ({R}ussian). {\em Doklady Akad\'emii Nauk SSSR} 
	55:587 -- 590.
\end{eintrag}
\\[3.2mm]
Marsh, William, and Partee, Barbara~H. 
\\\begin{eintrag}
1987 & How {N}on--{C}ontext {F}ree is {V}ariable {B}inding?
   	In {\em The {F}ormal {C}omplexity of {N}atural {L}anguage}, 
	Walter Savitch, Emmon Bach, William Marsch, and Gila 
	Safran--Naveh (eds.), 369 -- 386. Dordrecht: Reidel.
\end{eintrag}
\\[3.2mm]
Mel'\v{c}uk, Igor
\\\begin{eintrag}
1988 & {\em Dependency Syntax: Theory and Practice}. SUNY Linguistics 
	Series. Albany: State University of New York Press.
\\
2000 & {\em Cours de Morphologie G\'e\-n\'e\-rale\/} [General Morphology. A
 	 Coursebook]. Volume 1 -- 5. Montr\'eal: Les Presses de 
	l'Universit\'e de Montr\'eal.
\end{eintrag}
\\[3.2mm]
Meyer, A.~R.
\\\begin{eintrag}
1982 & What is a model of the lambda calculus? {\em Information and 
	Control} 52:87 -- 122.
\end{eintrag}
\\[3.2mm]
Michaelis, Jens, and Kracht, Marcus 
\\\begin{eintrag}
1997 & Semilinearity as a syntactic invariant. In {\em Logical Aspects 
	of Computational Linguistics (LACL '96)}, Christian Retor\'e (ed.), 
 	329 -- 345. (Lecture Notes in Artificial Intelligence 1328.)
	Heidelberg: Springer.
\end{eintrag}
\newpage\noindent
Michaelis, Jens, and Wartena, Christian 
\\\begin{eintrag}
1997 & How linguistic constraints on movement conspire to yield 
	languages analyzable with a restricted form of {LIGs}.
	In {\em Proceedings of the Conference on Formal Grammar 
	(FG '97), \emph{Aix en Provence}}, 158--168.
\\
1999 & {LIG}s with reduced derivation sets. In {\em Constraints and 
	Resources in Natural Language Syntax and Semantics}, Gosse Bouma, 
	Geert{--}Jan~M. Kruijff, Erhard Hinrichs, and Richard~T. 
	Oehrle (eds.), 263--279. Stanford: CSLI.
\end{eintrag}
\\[4.6mm]
Michaelis, Jens
\\\begin{eintrag}
2001a & Derivational minimalism is mildly context--sensitive.
	In {\em Logical Aspects of Computational Linguistics (LACL '98)},
	Michael Moortgat (ed.), 179 -- 198. (Lecture Notes in Artificial 
	Intelligence 2014.) Heidelberg: Springer.
\\
2001b & {\em On {F}ormal {P}roperties of {M}inimalist {G}rammars}.
	Ph.\ D.\ diss., Universit\"at Potsdam.
\\
2001c & Transforming linear context--free rewriting systems into 
	minimalist grammars. In {\em Logical Aspects of Computational 
	Linguistics (LACL '01)}, Philippe de~Groote, Glyn Morrill, and 
	Christian Retor{\'e} (eds.), 228 -- 244. (Lecture Notes 
	in Artificial Intelligence 2099.) Heidelberg: Springer.
\end{eintrag}
\\[4.6mm]
Miller, Philip~H. 
\\\begin{eintrag}
1991 & {S}candinavian {E}xtraction {P}henomena {R}evisited: {W}eak and
  {S}trong {G}enerative {C}apacity. {\em Linguistics and Philosophy} 
	14:101 -- 113.
\\
1999 & {\em Strong Generative Capacity. The Semantics of Linguistic
  Formalisms}. Stanford: CSLI.
\end{eintrag}
\\[4.6mm]
Mitchell, J.~C. 
\\\begin{eintrag}
1990 & Type systems for programming languages. In {\em Handbook of 
	Theoretical Computer Science, Vol B. Formal Models and Semantics}, 
	Jan van Leeuwen (ed.), 365 -- 458. Amsterdam: Elsevier.
\end{eintrag}
\\[4.6mm]
Monk, Donald~J. 
\\\begin{eintrag}
1969 & Nonfinitizability of classes of representable cylindric 
	algebras. {\em Journal of Symbolic Logic} 34:331 -- 343.
\\
1976 & {\em Mathematical Logic}. Berlin, Heidelberg: Springer.
\end{eintrag}
\\[4.6mm]
M\"onnich, Uwe
\\\begin{eintrag}
1999 & On cloning context--freeness. In {\em The Mathematics of Sentence 
	Structure. Trees and their Logics}, Hans-Peter Kolb and Uwe 
	M\"onnich (eds.), 195 -- 229. (Studies in Generative Grammar 44.) 
	Berlin: Mouton de Gruyter.
\end{eintrag}
\newpage\noindent
Moschovakis, Yannis
\\\begin{eintrag}
1994 & Sense and denotation as algorithm and value. In 
	{\em Proceedings of the ASL Meeting 1990, Helsinki}, 
	Juha Oikkonen and Jouko V\"a\"an\"anen (eds.), 210 -- 249.
	(Lecture Notes in Logic 2.) Berlin and Heidelberg: Springer.
\end{eintrag}
\\[2.7mm]
Myhill, John
\\\begin{eintrag}
1960 & Linear bounded automata. Technical report, Wright--Patterson 
	Air Force Base.
\end{eintrag}
\\[2.7mm]
Ogden, R.~W., Ross, R,~J., and Winkelmann, K.
\\\begin{eintrag}
1985 & An ``{I}nterchange {L}emma'' for {C}ontext {F}ree {L}anguages.
	{\em SIAM Journal of Computing} 14:410 -- 415.
\end{eintrag}
\\[2.7mm]
Ogden, R.~W.
\\\begin{eintrag}
1968 & A helpful result for proving inherent ambiguity. {\em Mathematical 
	Systems Sciences} 2:191 -- 194.
\end{eintrag}
\\[2.7mm]
Ojeda, Almerindo~E. 
\\\begin{eintrag}
1988 & A {L}inear {P}recedence {A}ccount of {C}ross--{S}erial
  {D}ependencies. {\em Linguistics and Philosophy} 11:457 -- 492.
\end{eintrag}
\\[2.7mm]
Pentus, Mati 
\\\begin{eintrag}
1995 & Models for the {L}ambek calculus. {\em Annals of Pure and 
	Applied Logic} 75:179 -- 213.
\\
1997 & {P}roduct--{F}ree {L}ambek--{C}alculus and {C}ontext--{F}ree
  {G}rammars. {\em Journal of Symbolic Logic} 62:648 -- 660.
\end{eintrag}
\\[2.7mm]
Peters, Stanley~P., and Ritchie, R.~W.
\\\begin{eintrag}
1971 & On restricting the base component of transformational grammars.
	{\em Information and Control} 18:483 -- 501.
\\
1973 & On the generative power of transformational grammars.
	{\em Information Sciences} 6:49 -- 83.
\end{eintrag}
\\[2.7mm]
Pigozzi, Don J., and Salibra, Antonino
\\\begin{eintrag}
1995 & The {A}bstract {V}ariable--{B}inding {C}alculus.
	{\em Studia Logica} 55:129 -- 179.
\end{eintrag}
\\[2.7mm]
Pigozzi, Don~J. 
\\\begin{eintrag}
1991 & Fregean {A}lgebraic {L}ogic. In {\em Algebraic Logic}, 
	Hajnal Andr\'eka, Donald Monk, and Istv\'an N\'emeti (eds.), 
	475 -- 504. (Colloquia Mathematica Societatis J\'anos Bolyai 54.), 
	Budapest and Amsterdam: J\'anos Bolyai Matematikai 
	T\'arsulat and North--Holland.
\end{eintrag}
\\[2.7mm]
Pogodalla, Sylvain
\\\begin{eintrag}
2001 & {\em R\'eseaux de preuve et g\'en\'eration pour les grammaires de
  type logiques} [Proof nets and generation for type logical grammars].
	Ph.\ D.\ diss., Institut National Polytechnique de Lorraine.
\end{eintrag}
\newpage\noindent
Pollard, Carl J., and Sag, Ivan 
\\\begin{eintrag}
1987 & {\em Information--{B}ased {S}yntax and {S}emantics. Vol.~1}.
	CSLI Lecture Notes 13. Stanford: CSLI.
\\
1994 & {\em Head--Driven Phrase Structure Grammar}. Chicago: 
	The University of Chicago Press.
\end{eintrag}
\\[2.2mm]
Pollard, Carl~J. 
\\\begin{eintrag}
1984 & {\em Generalized Phrase Structure Grammar, Head Grammars and 
	Natural Language}. Ph.\ D.\ diss., Stanford University.
\end{eintrag}
\\[2.2mm]
Port, R.~F., and O'Dell, M.~L. 
\\\begin{eintrag}
1985 & Neutralization of syllable--final voicing in {G}erman.
	Technical Report~4, Indiana University, Bloomington.
\end{eintrag}
\\[2.2mm]
Post, Emil~L. 
\\\begin{eintrag}
1936 & Finite combinatory processes -- formulation. {\em Journal of 
	Symbolic Logic} 1:103 -- 105.
\\
1943 & Formal reductions of the combinatorial decision problem.
	{\em Americal Journal of Mathematics} 65:197 -- 215.
\\
1947 & Recursive unsolvability of a problem of {T}hue.
	{\em Journal of Symbolic Logic} 11:1 -- 11.
\end{eintrag}
\\[2.2mm]
Postal, Paul 
\\\begin{eintrag}
1964 & {\em {C}onstituent {S}tructure: {A} {S}tudy of {C}ontemporary
  {M}odels of {S}yntax}. The Hague: Mouton.
\end{eintrag}
\\[2.2mm]
Prucnal, T., and Wro\'{n}ski, A.
\\\begin{eintrag}
1974 & An algebraic characterization of the notion of structural
  completeness. {\em Bulletin of Section Logic of the Polish Academy 
	of Sciences} 3:20 -- 33.
\end{eintrag}
\\[2.2mm]
Quine, Willard van Orman 
\\\begin{eintrag}
1960 & Variables explained away. {\em Proceedings of American 
	Philosophical Society} 104:343 -- 347.
\end{eintrag}
\\[2.2mm]
Radzinski, Daniel
\\\begin{eintrag}
1990 & {U}nbounded {S}yntactic {C}opying in {M}andarin {C}hinese.
	{\em Linguistics and Philosophy} 13:113 -- 127, 1990.
\end{eintrag}
\\[2.2mm]
Rambow, Owen 
\\\begin{eintrag}
1994 & {\em Formal and Computational Aspects of Natural Language Syntax}.
	Ph.\ D.\ diss., University of Pennsylvania.
\end{eintrag}
\\[2.2mm]
Recanati, Fran\c{c}ois 
\\\begin{eintrag}
2000 & {\em Oratio Obliqua, Oratio Recta. An Essay on Metarepresentation}.
	Cambridge (Mass.): MIT Press.
\end{eintrag}
\\[2.2mm]
Roach, Kelly
\\\begin{eintrag}
1987 & Formal properties of head grammars. In {\em Mathematics of Language}, 
	Alexis Manaster{--}Ramer (ed.), 293--347. 
	Amsterdam: John Benjamins.
\end{eintrag}
\newpage\noindent
Rogers, James 
\\\begin{eintrag}
1994 & {\em Studies in the {L}ogic of {T}rees with {A}pplications to
  {G}rammar {F}ormalisms}. Ph.\ D.\ diss., University of Delaware, 
  Department of Computer \& Information Sciences.
\end{eintrag}
\\[2.5mm]
Rounds, William~C. 
\\\begin{eintrag}
1988 & {LFP}: {A} {L}ogic for {L}inguistic {D}escription and an {A}nalysis
  of its {C}omplexity. {\em Computational Linguistics} 14:1 -- 9.
\end{eintrag}
\\[2.5mm]
Russell, Bertrand 
\\\begin{eintrag}
1905 & On denoting. {\em Mind} 14:479 -- 493.
\end{eintrag}
\\[2.5mm]
Sain, Ildik\'{o}, and Thompson, Richard~S. 
\\\begin{eintrag}
1991 & Finite {S}chema {A}xiomatization of {Q}uasi--{P}olyadic {A}lgebras.
	In {\em Algebraic Logic}, Hajnal Andr\'eka, Donald Monk, and 
	Istv\'an N\'emeti (eds.), 539 -- 571. (Colloquia Mathematica 
	Societatis J\'anos Bolyai 54.) Budapest and Amsterdam: J\'anos 
	Bolyai Matematikai T\'arsulat and North--Holland.
\end{eintrag}
\\[2.5mm]
Salomaa, Arto~K. 
\\\begin{eintrag}
1973 & {\em Formal Languages}. New York: Academic Press.
\end{eintrag}
\\[2.5mm]
van~der Sandt, Rob A.
\\\begin{eintrag}
1988 & {\em Context and Presupposition}. London: Croom Helm.
\end{eintrag}
\\[2.5mm]
de Saussure, Ferdinand
\\\begin{eintrag}
1965 & {\em Course in General Linguistics}. Columbus: McGraw--Hill.
\end{eintrag}
\\[2.5mm]
Sauvageot, Aur\'{e}lien 
\\\begin{eintrag}
1971 & {\em L'{E}dification de la {L}angue {H}ongroise\/} [The building of
  the Hungarian language]. Paris: Klincksieck.
\end{eintrag}
\\[2.5mm]
Sch\"onfinkel, Moses
\\\begin{eintrag}
1924 & {\"U}ber die {B}austeine der ma\-the\-ma\-ti\-schen {L}ogik [{O}n
  the building blocks of mathematical logic]. {\em Mathematische Annalen} 
	92:305 -- 316.
\end{eintrag}
\\[2.5mm]
Seki, Hiroyuki, Matsumura, Takashi, Fujii, Mamoru, and Kasami, Tadao.
\\\begin{eintrag}
1991 & On multiple context--free grammars. {\em Theoretical Computer 
	Science} 88:191 -- 229.
\end{eintrag}
\\[2.5mm]
Sestier, A.
\\\begin{eintrag}
1960 & Contributions \`{a} une th\'{e}orie ensembliste des classifications
  linguistiques [{C}ontributions to a set--theoretical theory of
  classifications]. In {\em Actes du Ier Congr\`{e}s de l'AFCAL}, 
  293 -- 305, Grenoble.
\end{eintrag}
\\[2.5mm]
Shieber, Stuart 
\\\begin{eintrag}
1985 & Evidence against the {C}ontext--{F}reeness of {N}atural 
	{L}anguages. {\em Linguistics and Philosophy} 8:333 -- 343.
\end{eintrag}
\\
\begin{eintrag}
1992 & {\em Constraint--{B}ased {G}rammar {F}ormalisms}. 
	Cambridge (Mass.): MIT Press.
\end{eintrag}
\\[3.3mm]
Smullyan, Raymond~M. 
\\\begin{eintrag}
1961 & {\em Theory of Formal Systems}. Annals of Mathematics 
	Studies 47. Princeton: Princeton University Press.
\end{eintrag}
\\[3.3mm]
Staal, J.~F. 
\\\begin{eintrag}
1967 & {\em Word Order in Sanskrit and Universal Grammar}.
	Foundations of Language, Supplementary Series No.~5. 
	Dordrecht: Reidel.
\end{eintrag}
\\[3.3mm]
Stabler, Edward~P. 
\\\begin{eintrag}
1997 & Derivational {M}inimalism. In {\em Logical Aspects of Computational 
	Linguistics (LACL '96)}, Christian Retor\'{e} (ed.),  68 -- 95.
	(Lecture Notes in Artificial Intelligence 1328.)
	Heidelberg: Springer.
\end{eintrag}
\\[3.3mm]
von Stechow, Arnim, and Sternefeld, Wolfgang
\\\begin{eintrag}
1987 & {\em Bausteine syntaktischen Wissens. Ein Lehrbuch der generativen
  Grammatik\/} [{B}uilding blocks of syntactic knowledge. A textbook of
  generative grammar]. Opladen: Westdeutscher Verlag.
\end{eintrag}
\\[3.3mm]
Steedman, Mark 
\\\begin{eintrag}
1990 & Gapping as constituent coordination. {\em Linguistics and 
	Philosophy} 13:207 -- 263.
\\
1996 & {\em Surface Structure and Interpretation}. 
	Linguistic Inquiry Monographs 30. Cambridge (Mass.): 
	MIT Press.
\end{eintrag}
\\[3.3mm]
Tarski, Alfred 
\\\begin{eintrag}
1983 & The concept of truth in formalized languages. In 
	{\em Logic, Semantics, Metamathematics}, J.~Corcoran (ed.), 
	152 -- 178. Indianapolis: Hackett Publishing.
\end{eintrag}
\\[3.3mm]
Tesni\`ere, Lucien
\\\begin{eintrag}
1982 & {\em El\'ements de syntaxe structurale\/} [Elements of structural
  syntax]. 4th ed. Paris: Klincksieck.
\end{eintrag}
\\[3.3mm]
Thatcher, J.~W., and Wright, J.~B.
\\\begin{eintrag}
1968 & Generalized finite automata theory with an application to a 
	decision problem of second--order logic. {\em Mathematical 
	Systems Theory} 2:57 -- 81.
\end{eintrag}
\\[3.3mm]
Thue, Axel 
\\\begin{eintrag}
1914 & Probleme \"uber {V}er\"anderungen von {Z}ei\-chen\-rei\-hen nach
  gegebenen {R}egeln [{P}roblems concerning changing strings according to
  given rules]. {\em Skrifter utgit av Videnskapsselkapet i Kristiania, {I}.
  Mathematisk--naturvidenskabelig klasse} 10.
\end{eintrag}
\newpage
\noindent
Trakht\'enbrodt, B.~A.
\\\begin{eintrag}
1950 & N\'evozmo\v{z}nost' algorifma dl\'a probl\'emy razr\'e\v{s}imosti na
  kon\'e\v{c}nyh klassah [{I}mpossibility of an algorithm for the decision
  problem of finite classes]. {\em Doklady Akad\'emii Nauk SSSR}, 
  569 -- 572.
\end{eintrag}
\\[2.3mm]
Turing, Alan~M. 
\\\begin{eintrag}
1936 & On computable numbers, with an application to the 
	{E}ntscheidungsproblem. {\em Proceedings of the London 
	Mathematical Society} 42:230 -- 265.
\end{eintrag}
\\[2.3mm]
Uszkoreit, Hans
\\\begin{eintrag}
1987 & {\em Word {O}rder and {C}onstituent {S}tructure in {G}erman}.
	CSLI Lecture Notes 8. Stanford: CSLI.
\end{eintrag}
\\[2.3mm]
Vaught, Robert~L. 
\\\begin{eintrag}
1995 & {\em Set Theory. An Introduction}. 2nd ed. Basel: 
	Birkh\"auser.
\end{eintrag}
\\[2.3mm]
Frank Veltman.
\\\begin{eintrag}
1985 & {\em Logics for Conditionals}. Ph.\ D.\ diss., Department of 
	Philosophy, University of Amsterdam.
\end{eintrag}
\\[2.3mm]
Vijay{--}Shanker, K., Weir, David~J., and Joshi, Aravind~K. 
\\\begin{eintrag}
1986 & Tree adjoining and head wrapping. In {\em Proceedings of the 11th 
	International Conference on {C}omputational {L}inguistics 
	({COLING} '86)\emph{, Bonn}}, 202--207.
\\
1987 & Characterizing structural descriptions produced by various grammar
	  formalisms. In {\em Proceedings of the 25th Meeting of the 
	Association for Computational Linguistics ({ACL} '87)\emph{, 
	Stanford, CA}}, 104 -- 111.
\end{eintrag}
\\[2.3mm]
Villemonte de~la Clergerie, Eric
\\\begin{eintrag}
2002a & Parsing {MCS} {L}anguages with {T}hread {A}utomata.
	In {\em Proceedings of the Sixth International Workshop on Tree
  Adjoining Grammars and Related Formalisms (TAG+6), \emph{Venezia}}.
\\
2002b & Parsing mildly context--sensitive languages with thread automata.
	In {\em Proceedings of the 19th International Conference on
        Computational Linguistics (COLING 02)\emph{, Taipei}}.
\end{eintrag}
\\[2.3mm]
Weir, David~J. 
\\\begin{eintrag}
1988 & {\em Characterizing {M}ildly {C}ontext--{S}ensitive {G}rammar
  {F}ormalisms}. Ph.\ D.\ diss., University of Pennsylvania, Philadelphia.
\end{eintrag}
\\[2.3mm]
Younger, D.~H. 
\\\begin{eintrag}
1967 & Recognition and parsing of context--free languages in time $n^3$.
	{\em Information and Control} 10:189 -- 208.
\end{eintrag}
\\[2.3mm]
Zadrozny, Wlodek
\\\begin{eintrag}
1994 & From {C}ompositional {S}emantics to {S}ystematic {S}emantics.
	{\em Linguistics and Philosophy} 17:329 -- 342.
\end{eintrag}
\\[2mm]
Zimmermann, Thomas~Ede 
\\\begin{eintrag}
1999 & Meaning {P}ostulates and the {M}odel--{T}heoretic {A}pproach to
  {N}atural {L}anguage {S}emantics. {\em Linguistics and Philosophy} 
	22:529 -- 561.
\end{eintrag}

 \printindex}
\end{document}
