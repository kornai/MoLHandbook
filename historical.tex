
\section{Mathematical tools for historical linguistics}

Tools for improving the understanding of how and why languages change on historical timescales, up to a few thousand years

Participants:
\begin{itemize}
\item Dan Dediu dan.dediu@univ-lyon2.fr
\item Johannes Dellert johannes.dellert@uni-tuebingen.de
\item Gerhard Jäger gerhard.jaeger@uni-tuebingen.de
\item W. Garrett Mitchener garrett.mitchener@gmail.com, mitchenerg@cofc.edu
\item Igor Yanovich igor.yanovich@gmail.com
\end{itemize}

\begin{itemize}

\item Statistical tools for corpus analysis:
  PMI scores for extracting sound correspondences, resampling from lexical databases to estimate what random data would look like, automated cognate clustering using a combination of some phonetic similarity measure with some standard clustering method; (plus Johannes mentioned a number of less-well-known approaches)

\item Phylogenetic trees:
 Given sets of homologous features in a collection of languages, construct trees indicating which languages are likely to have close common ancestors.

 Researchers: Donald Ringe, Tandy Warnow, Luay Nakhleh,
 Steven N. Evans

 Web sites:
 \begin{itemize}
 \item \url{https://www.cs.utexas.edu/~tandy/histling.html}
 \end{itemize}


\item Dynamical systems and language change:
 \begin{itemize}
 \item Iterated maps and differential equations can represent how
  language changes arise and spread.  These models are often
  derived from the models in the mathematical biology literature:
  population game dynamics, such as the replicator equation; and
  spread of contagious diseases, such as SIR models.


 \item Given data on the use of a novel construction, estimate parameters of a dynamical system model.

 \item  Qualitatively model the social forces that drive languages to extinction.
 \end{itemize}

 Researchers: Gerhard Jäger, Anthony Kroch, W. Garrett Mitchener, Lisa Pearl, Martin Nowak, Charles Yang, Steven Strogatz, Daniel Abrams, Dan Dediu, D. Robert Ladd


\item Unsorted:
 \begin{itemize}
 \item game-theoretic approaches to language change/language evolution, crucially explaining the breadth of the field beyond e.g. the replicator-mutator dynamics (I suspect you’d have more specific ideas); perhaps some population-genetic methodology

 \item tree-generating processes

 \item character change processes (in the phylogenetic context)

 \item induction of covariance matrices by trees/slightly more general graphs, and probabilities for observed character patterns induced by such
 \end{itemize}

\end{itemize}

%% Emacs stuff:

%%% Local Variables:
%%% mode: latex
%%% TeX-master: t
%%% End:
